\documentclass[
  jou,
  floatsintext,
  longtable,
  a4paper,
  nolmodern,
  notxfonts,
  notimes,
  colorlinks=true,linkcolor=blue,citecolor=blue,urlcolor=blue]{apa7}

\usepackage{amsmath}
\usepackage{amssymb}



\usepackage[bidi=default]{babel}
\babelprovide[main,import]{spanish}
\StartBabelCommands{spanish}{captions} [unicode, fontenc=TU EU1 EU2, charset=utf8] \SetString{\keywordname}{Palabras
Claves}
\EndBabelCommands


% get rid of language-specific shorthands (see #6817):
\let\LanguageShortHands\languageshorthands
\def\languageshorthands#1{}

\RequirePackage{longtable}
\RequirePackage{threeparttablex}

\makeatletter
\renewcommand{\paragraph}{\@startsection{paragraph}{4}{\parindent}%
	{0\baselineskip \@plus 0.2ex \@minus 0.2ex}%
	{-.5em}%
	{\normalfont\normalsize\bfseries\typesectitle}}

\renewcommand{\subparagraph}[1]{\@startsection{subparagraph}{5}{0.5em}%
	{0\baselineskip \@plus 0.2ex \@minus 0.2ex}%
	{-\z@\relax}%
	{\normalfont\normalsize\bfseries\itshape\hspace{\parindent}{#1}\textit{\addperi}}{\relax}}
\makeatother




\usepackage{longtable, booktabs, multirow, multicol, colortbl, hhline, caption, array, float, xpatch}
\usepackage{subcaption}
\renewcommand\thesubfigure{\Alph{subfigure}}
\setcounter{topnumber}{2}
\setcounter{bottomnumber}{2}
\setcounter{totalnumber}{4}
\renewcommand{\topfraction}{0.85}
\renewcommand{\bottomfraction}{0.85}
\renewcommand{\textfraction}{0.15}
\renewcommand{\floatpagefraction}{0.7}

\usepackage{tcolorbox}
\tcbuselibrary{listings,theorems, breakable, skins}
\usepackage{fontawesome5}

\definecolor{quarto-callout-color}{HTML}{909090}
\definecolor{quarto-callout-note-color}{HTML}{0758E5}
\definecolor{quarto-callout-important-color}{HTML}{CC1914}
\definecolor{quarto-callout-warning-color}{HTML}{EB9113}
\definecolor{quarto-callout-tip-color}{HTML}{00A047}
\definecolor{quarto-callout-caution-color}{HTML}{FC5300}
\definecolor{quarto-callout-color-frame}{HTML}{ACACAC}
\definecolor{quarto-callout-note-color-frame}{HTML}{4582EC}
\definecolor{quarto-callout-important-color-frame}{HTML}{D9534F}
\definecolor{quarto-callout-warning-color-frame}{HTML}{F0AD4E}
\definecolor{quarto-callout-tip-color-frame}{HTML}{02B875}
\definecolor{quarto-callout-caution-color-frame}{HTML}{FD7E14}

%\newlength\Oldarrayrulewidth
%\newlength\Oldtabcolsep


\usepackage{hyperref}




\providecommand{\tightlist}{%
  \setlength{\itemsep}{0pt}\setlength{\parskip}{0pt}}
\usepackage{longtable,booktabs,array}
\usepackage{calc} % for calculating minipage widths
% Correct order of tables after \paragraph or \subparagraph
\usepackage{etoolbox}
\makeatletter
\patchcmd\longtable{\par}{\if@noskipsec\mbox{}\fi\par}{}{}
\makeatother
% Allow footnotes in longtable head/foot
\IfFileExists{footnotehyper.sty}{\usepackage{footnotehyper}}{\usepackage{footnote}}
\makesavenoteenv{longtable}

\usepackage{graphicx}
\makeatletter
\newsavebox\pandoc@box
\newcommand*\pandocbounded[1]{% scales image to fit in text height/width
  \sbox\pandoc@box{#1}%
  \Gscale@div\@tempa{\textheight}{\dimexpr\ht\pandoc@box+\dp\pandoc@box\relax}%
  \Gscale@div\@tempb{\linewidth}{\wd\pandoc@box}%
  \ifdim\@tempb\p@<\@tempa\p@\let\@tempa\@tempb\fi% select the smaller of both
  \ifdim\@tempa\p@<\p@\scalebox{\@tempa}{\usebox\pandoc@box}%
  \else\usebox{\pandoc@box}%
  \fi%
}
% Set default figure placement to htbp
\def\fps@figure{htbp}
\makeatother







\usepackage{newtx}

\defaultfontfeatures{Scale=MatchLowercase}
\defaultfontfeatures[\rmfamily]{Ligatures=TeX,Scale=1}





\title{El aborto: Consecuencias y Consideraciones Éticas en la Sociedad
Contemporánea}


\shorttitle{El Aborto}


\usepackage{etoolbox}



\ccoppy{\textcopyright~2025}





\authorsnames{Edison Achalma,Yeno Areste,Cristían Galindo}





\affiliation{
{Economía, Universidad Nacional de San Cristóbal de Huamanga}}




\leftheader{Achalma, Areste and Galindo}

\date{2015-05-14}


\abstract{Este trabajo aborda el tema del aborto desde una perspectiva
crítica y educativa, explorando sus causas, tipos y las graves
consecuencias tanto físicas como psicológicas asociadas. Se analizan los
miedos y presiones sociales que influyen en la decisión de abortar, así
como las implicaciones éticas y legales de este acto. El documento busca
informar y reflexionar sobre el valor de la vida y las responsabilidades
inherentes a la toma de decisiones en situaciones de embarazo no
deseado, ofreciendo una visión integral que combina datos científicos,
testimonios y consideraciones morales. }

\keywords{Aborto, Adolescentes, Consecuencias}

\authornote{\par{\addORCIDlink{Edison Achalma}{0000-0001-6996-3364}} 
\par{ }
\par{   Los autores no tienen conflictos de intereses que
revelar.  Quiero agradecer a todos mis maestros ya que ellos me enseñan
valorar los estudios y a superarme cada día, también agradezco a mis
padres porque ellos están en los días más difíciles de mi vida como
estudiante. Y agradezco a Dios por darme la salud que tengo, estoy
seguro que mis metas planteadas darán fruto en el futuro y por ende me
debo esforzar cada día para ser mejor en la universidad y en todo lugar
sin olvidar el respeto que engrandece a la persona.  Los roles de autor
se clasificaron utilizando la taxonomía de roles de colaborador (CRediT;
https://credit.niso.org/) de la siguiente manera:  Edison
Achalma:   conceptualización, redacción; Yeno
Areste:   redacción; Cristían Galindo:   redacción}
\par{La correspondencia relativa a este artículo debe dirigirse a Edison
Achalma, Email: \href{mailto:elmer.achalma.09@unsch.edu.pe}{elmer.achalma.09@unsch.edu.pe}}
}

\usepackage{pbalance} 
\usepackage{float}
\makeatletter
\let\oldtpt\ThreePartTable
\let\endoldtpt\endThreePartTable
\def\ThreePartTable{\@ifnextchar[\ThreePartTable@i \ThreePartTable@ii}
\def\ThreePartTable@i[#1]{\begin{figure}[!htbp]
\onecolumn
\begin{minipage}{0.5\textwidth}
\oldtpt[#1]
}
\def\ThreePartTable@ii{\begin{figure}[!htbp]
\onecolumn
\begin{minipage}{0.5\textwidth}
\oldtpt
}
\def\endThreePartTable{
\endoldtpt
\end{minipage}
\twocolumn
\end{figure}}
\makeatother


\makeatletter
\let\endoldlt\endlongtable		
\def\endlongtable{
\hline
\endoldlt}
\makeatother

\newenvironment{twocolumntable}% environment name
{% begin code
\begin{table*}[!htbp]%
\onecolumn%
}%
{%
\twocolumn%
\end{table*}%
}% end code

\urlstyle{same}



\makeatletter
\@ifpackageloaded{caption}{}{\usepackage{caption}}
\AtBeginDocument{%
\ifdefined\contentsname
  \renewcommand*\contentsname{Tabla de contenidos}
\else
  \newcommand\contentsname{Tabla de contenidos}
\fi
\ifdefined\listfigurename
  \renewcommand*\listfigurename{Listado de Figuras}
\else
  \newcommand\listfigurename{Listado de Figuras}
\fi
\ifdefined\listtablename
  \renewcommand*\listtablename{Listado de Tablas}
\else
  \newcommand\listtablename{Listado de Tablas}
\fi
\ifdefined\figurename
  \renewcommand*\figurename{Figura}
\else
  \newcommand\figurename{Figura}
\fi
\ifdefined\tablename
  \renewcommand*\tablename{Tabla}
\else
  \newcommand\tablename{Tabla}
\fi
}
\@ifpackageloaded{float}{}{\usepackage{float}}
\floatstyle{ruled}
\@ifundefined{c@chapter}{\newfloat{codelisting}{h}{lop}}{\newfloat{codelisting}{h}{lop}[chapter]}
\floatname{codelisting}{Listado}
\newcommand*\listoflistings{\listof{codelisting}{Listado de Listados}}
\makeatother
\makeatletter
\makeatother
\makeatletter
\@ifpackageloaded{caption}{}{\usepackage{caption}}
\@ifpackageloaded{subcaption}{}{\usepackage{subcaption}}
\makeatother
\makeatletter
\@ifpackageloaded{fontawesome5}{}{\usepackage{fontawesome5}}
\makeatother

% From https://tex.stackexchange.com/a/645996/211326
%%% apa7 doesn't want to add appendix section titles in the toc
%%% let's make it do it
\makeatletter
\xpatchcmd{\appendix}
  {\par}
  {\addcontentsline{toc}{section}{\@currentlabelname}\par}
  {}{}
\makeatother

%% Disable longtable counter
%% https://tex.stackexchange.com/a/248395/211326

\usepackage{etoolbox}

\makeatletter
\patchcmd{\LT@caption}
  {\bgroup}
  {\bgroup\global\LTpatch@captiontrue}
  {}{}
\patchcmd{\longtable}
  {\par}
  {\par\global\LTpatch@captionfalse}
  {}{}
\apptocmd{\endlongtable}
  {\ifLTpatch@caption\else\addtocounter{table}{-1}\fi}
  {}{}
\newif\ifLTpatch@caption
\makeatother

\begin{document}

\maketitle

\hypertarget{toc}{}
\tableofcontents
\newpage
\section[Introduction]{El aborto}

\setcounter{secnumdepth}{-\maxdimen} % remove section numbering

\setlength\LTleft{0pt}


En este~trabajo~he llevado a cabo el tema~El Aborto, ya que en estos
últimos años se ha visto el incremento de~adolescentes~que abortan no
sólo en nuestro país ya que en~España~por ejemplo ha aumentado un 19\%
más últimamente por eso es muy importante para nosotros los estudiantes,
informarnos para así valorar la importancia que tiene la vida, y que por
más que a veces hagamos las cosas a la ligera también debemos pensar que
tiene sus consecuencias y que por eso debemos ser responsables en tomar
buenas decisiones ante un problema.

La finalidad que quiero trasmitir es que antes de tomar una decisión
como el de abortar debemos conocer los~riesgos~que pueden ocurrir y
saber cómo afrontar una situación así.

\section{El Aborto}\label{el-aborto}

La palabra ``aborto'' proviene del latín \emph{Abortus}, donde \emph{Ab}
significa ``mal'' y \emph{Ortus}, ``nacimiento''. Así, se refiere a un
parto anticipado, una privación del nacimiento o un nacimiento antes de
tiempo.

\subsection{Definición}\label{definiciuxf3n}

El aborto es la interrupción deliberada del proceso fisiológico del
embarazo, resultando en la muerte del producto de la concepción, ya sea
dentro o fuera del claustro materno.

\subsection{Causas}\label{causas}

El aborto tiene raíces predominantemente psicológicas, motivadas por
diversos miedos:

\begin{itemize}
\item
  \textbf{Miedo por las capacidades económicas:} La preocupación por no
  poder mantener a un hijo puede llevar a una mujer a abortar. Este
  temor se relaciona con la falta de confianza en Dios, olvidando que lo
  más valioso para una madre es su hijo. Nuestra sociedad de consumo y
  valores superficiales ha desvalorizado la vida prenatal, creando
  miedos infundados. Un ejemplo notable es el de Kay James, quien, a
  pesar de nacer en circunstancias económicas adversas, logró una vida
  exitosa.
\item
  \textbf{Miedo al juicio social:} Especialmente en adolescentes
  embarazadas durante el noviazgo, el temor a lo que piensen padres o la
  comunidad puede influir en la decisión. La vida, dada por Dios,
  debería estar por encima de cualquier juicio humano.
\item
  \textbf{Miedo al embarazo y al parto:} Ser madre es la misión más
  noble de una mujer, un proceso natural que no debería ser temido,
  comparado incluso con la aceptación de esta función en el reino
  animal.
\item
  \textbf{Problemas de salud:} Ejemplos históricos como el de Beethoven,
  nacido de padres con graves problemas de salud, demuestran que no
  todos los embarazos con riesgos deben terminar en aborto.
\item
  \textbf{Violación:} Aunque es un trauma profundo, incluso en estos
  casos, hablamos de un ser humano en desarrollo.
\end{itemize}

\subsection{Tipos de Aborto}\label{tipos-de-aborto}

\begin{itemize}
\item
  \textbf{Aborto Espontáneo:} Ocurre de manera natural.
\item
  \textbf{Aborto Inducido:} Es intencional, con o sin apoyo médico, y
  puede tener múltiples motivaciones sociales o legales.
\item
  \textbf{Aborto Terapéutico:} Justificado médicamente para:

  \begin{itemize}
  \tightlist
  \item
    Salvar la vida de la madre.
  \item
    Proteger su salud física o mental.
  \item
    Evitar el nacimiento de un niño con enfermedades graves.
  \item
    Reducir el número de fetos en embarazos múltiples.
  \end{itemize}
\end{itemize}

\subsection{Consecuencias}\label{consecuencias}

El aborto puede llevar a una serie de complicaciones:

\begin{enumerate}
\def\labelenumi{\arabic{enumi}.}
\tightlist
\item
  \textbf{Succión, Legrado o Aspiración:} Riesgo de infecciones,
  traumas, hemorragias, entre otros.
\item
  \textbf{Dilatación y Curetaje (D y C):} Además de los riesgos
  anteriores, puede causar perforaciones uterinas.
\item
  \textbf{Dilatación y Evacuación (D y E):} Añade riesgos de infecciones
  específicas y complicaciones en futuros embarazos.
\item
  \textbf{Inyección Salina:} Riesgo de ruptura uterina y embolismo.
\item
  \textbf{Prostaglandinas:} Posibles rupturas uterinas, sepsis, y fallos
  cardíacos o renales.
\item
  \textbf{Extracción Menstrual:} Puede resultar en infecciones si no se
  confirma el embarazo.
\item
  \textbf{Mifeprex o Mifepristona (RU 486):} Riesgos de infecciones
  severas y sangrado excesivo.
\item
  \textbf{Aborto por Nacimiento Parcial:} Alta posibilidad de hemorragia
  y necesidad de histerectomía.
\item
  \textbf{Píldora del Día Después:} Cambios en la vasculatura que pueden
  llevar a alteraciones hemorrágicas.
\end{enumerate}

Otras complicaciones incluyen un aumento en cesáreas, nacimientos
prematuros, y problemas de fertilidad.

\section{Conclusiones}\label{conclusiones}

El aborto es el asesinato de una persona, ya que desde el momento de la
concepción el feto es considerado como tal, con~derechos~que lo protegen
ante la~ley.

aborto viola el quinto mandamiento no matarás que Dios nos encomendó
para realizarnos como personas.

Aprendimos que los diferentes tipos de abortos que existen son
terriblemente inhumanos y crueles y que tienen consecuencias al
practicarse.

\section{Publicaciones Similares}\label{publicaciones-similares}

Si te interesó este artículo, te recomendamos que explores otros blogs y
recursos relacionados que pueden ampliar tus conocimientos. Aquí te dejo
algunas sugerencias:

\begin{enumerate}
\def\labelenumi{\arabic{enumi}.}
\tightlist
\item
  \href{https://achalmaedison.netlify.app/blog/posts/2015-05-14-el-aborto/index.pdf}{\faIcon{file-pdf}}
  \href{https://achalmaedison.netlify.app/blog/posts/2015-05-14-el-aborto}{El
  Aborto}
\item
  \href{https://achalmaedison.netlify.app/blog/posts/2017-04-23-sitios-web-asombrosos/index.pdf}{\faIcon{file-pdf}}
  \href{https://achalmaedison.netlify.app/blog/posts/2017-04-23-sitios-web-asombrosos}{Sitios
  Web Asombrosos}
\item
  \href{https://achalmaedison.netlify.app/blog/posts/2017-05-23-el-mercantilismo/index.pdf}{\faIcon{file-pdf}}
  \href{https://achalmaedison.netlify.app/blog/posts/2017-05-23-el-mercantilismo}{El
  Mercantilismo}
\item
  \href{https://achalmaedison.netlify.app/blog/posts/2020-05-23-comandos-de-google-assistant/index.pdf}{\faIcon{file-pdf}}
  \href{https://achalmaedison.netlify.app/blog/posts/2020-05-23-comandos-de-google-assistant}{Comandos
  De Google Assistant}
\item
  \href{https://achalmaedison.netlify.app/blog/posts/2020-09-15-plan-de-negocio-exportacion-de-trucha-arcoires/index.pdf}{\faIcon{file-pdf}}
  \href{https://achalmaedison.netlify.app/blog/posts/2020-09-15-plan-de-negocio-exportacion-de-trucha-arcoires}{Plan
  De Negocio Exportacion De Trucha Arcoires}
\item
  \href{https://achalmaedison.netlify.app/blog/posts/2021-07-13-plan-de-negocio-exportacion-de-tuna/index.pdf}{\faIcon{file-pdf}}
  \href{https://achalmaedison.netlify.app/blog/posts/2021-07-13-plan-de-negocio-exportacion-de-tuna}{Plan
  De Negocio Exportacion De Tuna}
\item
  \href{https://achalmaedison.netlify.app/blog/posts/2021-07-14-comandos-de-blogdown/index.pdf}{\faIcon{file-pdf}}
  \href{https://achalmaedison.netlify.app/blog/posts/2021-07-14-comandos-de-blogdown}{Comandos
  De Blogdown}
\item
  \href{https://achalmaedison.netlify.app/blog/posts/2021-10-01-gestion-publica-y-administracion-publica/index.pdf}{\faIcon{file-pdf}}
  \href{https://achalmaedison.netlify.app/blog/posts/2021-10-01-gestion-publica-y-administracion-publica}{Gestion
  Publica Y Administracion Publica}
\item
  \href{https://achalmaedison.netlify.app/blog/posts/2021-10-01-reformas-y-modernizacion-de-la-gestion-publica/index.pdf}{\faIcon{file-pdf}}
  \href{https://achalmaedison.netlify.app/blog/posts/2021-10-01-reformas-y-modernizacion-de-la-gestion-publica}{Reformas
  Y Modernizacion De La Gestion Publica}
\item
  \href{https://achalmaedison.netlify.app/blog/posts/2022-01-23-cadena\%20de\%20suministros/index.pdf}{\faIcon{file-pdf}}
  \href{https://achalmaedison.netlify.app/blog/posts/2022-01-23-cadena\%20de\%20suministros}{Cadena
  De Suministros}
\item
  \href{https://achalmaedison.netlify.app/blog/posts/2022-04-22-economia-agraria/index.pdf}{\faIcon{file-pdf}}
  \href{https://achalmaedison.netlify.app/blog/posts/2022-04-22-economia-agraria}{Economia
  Agraria}
\item
  \href{https://achalmaedison.netlify.app/blog/posts/2022-06-02-impacto-del-cambio-climatico/index.pdf}{\faIcon{file-pdf}}
  \href{https://achalmaedison.netlify.app/blog/posts/2022-06-02-impacto-del-cambio-climatico}{Impacto
  Del Cambio Climatico}
\item
  \href{https://achalmaedison.netlify.app/blog/posts/2023-05-11-cualidades-de-los-servidores-publicos/index.pdf}{\faIcon{file-pdf}}
  \href{https://achalmaedison.netlify.app/blog/posts/2023-05-11-cualidades-de-los-servidores-publicos}{Cualidades
  De Los Servidores Publicos}
\item
  \href{https://achalmaedison.netlify.app/blog/posts/2023-05-12-la-economia-peruana-entre-1970-1990/index.pdf}{\faIcon{file-pdf}}
  \href{https://achalmaedison.netlify.app/blog/posts/2023-05-12-la-economia-peruana-entre-1970-1990}{La
  Economia Peruana Entre 1970 1990}
\item
  \href{https://achalmaedison.netlify.app/blog/posts/2023-05-16-economia-regional/index.pdf}{\faIcon{file-pdf}}
  \href{https://achalmaedison.netlify.app/blog/posts/2023-05-16-economia-regional}{Economia
  Regional}
\end{enumerate}

Esperamos que encuentres estas publicaciones igualmente interesantes y
útiles. ¡Disfruta de la lectura!

Buendía, A. (2005) ~Archivos de México,75(4), 387-388.

Flores, A. (1999).~La reorganización de la biblioteca del Hospital
Mocel.~México: UNAM

Higashida, B. (1995).~Educación para la salud.~México: Interamericana
Mac Graw Hill.

Juárez, B. y Martínez, P. (2000). El derecho a la vida






\end{document}
