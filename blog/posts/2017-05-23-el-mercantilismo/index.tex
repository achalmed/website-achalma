\documentclass[
  jou,
  floatsintext,
  longtable,
  a4paper,
  nolmodern,
  notxfonts,
  notimes,
  colorlinks=true,linkcolor=blue,citecolor=blue,urlcolor=blue]{apa7}

\usepackage{amsmath}
\usepackage{amssymb}



\usepackage[bidi=default]{babel}
\babelprovide[main,import]{spanish}
\StartBabelCommands{spanish}{captions} [unicode, fontenc=TU EU1 EU2, charset=utf8] \SetString{\keywordname}{Palabras
Claves}
\EndBabelCommands


% get rid of language-specific shorthands (see #6817):
\let\LanguageShortHands\languageshorthands
\def\languageshorthands#1{}

\RequirePackage{longtable}
\RequirePackage{threeparttablex}

\makeatletter
\renewcommand{\paragraph}{\@startsection{paragraph}{4}{\parindent}%
	{0\baselineskip \@plus 0.2ex \@minus 0.2ex}%
	{-.5em}%
	{\normalfont\normalsize\bfseries\typesectitle}}

\renewcommand{\subparagraph}[1]{\@startsection{subparagraph}{5}{0.5em}%
	{0\baselineskip \@plus 0.2ex \@minus 0.2ex}%
	{-\z@\relax}%
	{\normalfont\normalsize\bfseries\itshape\hspace{\parindent}{#1}\textit{\addperi}}{\relax}}
\makeatother




\usepackage{longtable, booktabs, multirow, multicol, colortbl, hhline, caption, array, float, xpatch}
\usepackage{subcaption}
\renewcommand\thesubfigure{\Alph{subfigure}}
\setcounter{topnumber}{2}
\setcounter{bottomnumber}{2}
\setcounter{totalnumber}{4}
\renewcommand{\topfraction}{0.85}
\renewcommand{\bottomfraction}{0.85}
\renewcommand{\textfraction}{0.15}
\renewcommand{\floatpagefraction}{0.7}

\usepackage{tcolorbox}
\tcbuselibrary{listings,theorems, breakable, skins}
\usepackage{fontawesome5}

\definecolor{quarto-callout-color}{HTML}{909090}
\definecolor{quarto-callout-note-color}{HTML}{0758E5}
\definecolor{quarto-callout-important-color}{HTML}{CC1914}
\definecolor{quarto-callout-warning-color}{HTML}{EB9113}
\definecolor{quarto-callout-tip-color}{HTML}{00A047}
\definecolor{quarto-callout-caution-color}{HTML}{FC5300}
\definecolor{quarto-callout-color-frame}{HTML}{ACACAC}
\definecolor{quarto-callout-note-color-frame}{HTML}{4582EC}
\definecolor{quarto-callout-important-color-frame}{HTML}{D9534F}
\definecolor{quarto-callout-warning-color-frame}{HTML}{F0AD4E}
\definecolor{quarto-callout-tip-color-frame}{HTML}{02B875}
\definecolor{quarto-callout-caution-color-frame}{HTML}{FD7E14}

%\newlength\Oldarrayrulewidth
%\newlength\Oldtabcolsep


\usepackage{hyperref}




\providecommand{\tightlist}{%
  \setlength{\itemsep}{0pt}\setlength{\parskip}{0pt}}
\usepackage{longtable,booktabs,array}
\usepackage{calc} % for calculating minipage widths
% Correct order of tables after \paragraph or \subparagraph
\usepackage{etoolbox}
\makeatletter
\patchcmd\longtable{\par}{\if@noskipsec\mbox{}\fi\par}{}{}
\makeatother
% Allow footnotes in longtable head/foot
\IfFileExists{footnotehyper.sty}{\usepackage{footnotehyper}}{\usepackage{footnote}}
\makesavenoteenv{longtable}

\usepackage{graphicx}
\makeatletter
\newsavebox\pandoc@box
\newcommand*\pandocbounded[1]{% scales image to fit in text height/width
  \sbox\pandoc@box{#1}%
  \Gscale@div\@tempa{\textheight}{\dimexpr\ht\pandoc@box+\dp\pandoc@box\relax}%
  \Gscale@div\@tempb{\linewidth}{\wd\pandoc@box}%
  \ifdim\@tempb\p@<\@tempa\p@\let\@tempa\@tempb\fi% select the smaller of both
  \ifdim\@tempa\p@<\p@\scalebox{\@tempa}{\usebox\pandoc@box}%
  \else\usebox{\pandoc@box}%
  \fi%
}
% Set default figure placement to htbp
\def\fps@figure{htbp}
\makeatother


% definitions for citeproc citations
\NewDocumentCommand\citeproctext{}{}
\NewDocumentCommand\citeproc{mm}{%
  \begingroup\def\citeproctext{#2}\cite{#1}\endgroup}
\makeatletter
 % allow citations to break across lines
 \let\@cite@ofmt\@firstofone
 % avoid brackets around text for \cite:
 \def\@biblabel#1{}
 \def\@cite#1#2{{#1\if@tempswa , #2\fi}}
\makeatother
\newlength{\cslhangindent}
\setlength{\cslhangindent}{1.5em}
\newlength{\csllabelwidth}
\setlength{\csllabelwidth}{3em}
\newenvironment{CSLReferences}[2] % #1 hanging-indent, #2 entry-spacing
 {\begin{list}{}{%
  \setlength{\itemindent}{0pt}
  \setlength{\leftmargin}{0pt}
  \setlength{\parsep}{0pt}
  % turn on hanging indent if param 1 is 1
  \ifodd #1
   \setlength{\leftmargin}{\cslhangindent}
   \setlength{\itemindent}{-1\cslhangindent}
  \fi
  % set entry spacing
  \setlength{\itemsep}{#2\baselineskip}}}
 {\end{list}}
\usepackage{calc}
\newcommand{\CSLBlock}[1]{\hfill\break\parbox[t]{\linewidth}{\strut\ignorespaces#1\strut}}
\newcommand{\CSLLeftMargin}[1]{\parbox[t]{\csllabelwidth}{\strut#1\strut}}
\newcommand{\CSLRightInline}[1]{\parbox[t]{\linewidth - \csllabelwidth}{\strut#1\strut}}
\newcommand{\CSLIndent}[1]{\hspace{\cslhangindent}#1}





\usepackage{newtx}

\defaultfontfeatures{Scale=MatchLowercase}
\defaultfontfeatures[\rmfamily]{Ligatures=TeX,Scale=1}





\title{El mercantilismo: Desarrollo económico en Europa}


\shorttitle{Mercantilismo}


\usepackage{etoolbox}


\course{Metodología (EDUC 5101)}
\professor{Blanca Rivera Guillén}
\duedate{05/23/2017}

\ccoppy{\textcopyright~2025}






\authorsnames[{1,2},{1},{1},{1},{1},{1}]{Edison Achalma,Jeancarlos
Alcarráz,Félix Bermudo,Jhony Conga,Juan Curi,Yuri Fernández}







\authorsaffiliations{
{},{Economía, Universidad Nacional de San Cristóbal de Huamanga}}




\leftheader{Achalma, Alcarráz, Bermudo, Conga, Curi and Fernández}

\date{2017-05-23}


\abstract{This paper explores the mercantilist economic thought that
emerged in Europe between the 16th and 18th centuries, a period marked
by the transition from feudalism to national markets. Mercantilism,
characterized by policies promoting trade surpluses and protectionism,
aimed to strengthen nation-states through the accumulation of precious
metals and the regulation of international trade. The study examines its
historical context, key principles such as the emphasis on exports over
imports, and its lasting contributions, including the concept of balance
of payments. It also analyzes how mercantilist ideas influenced the
development of internal markets, colonial exploitation, and early
capitalism. Drawing from historical texts, the paper highlights the role
of merchants and governments in shaping economic policies and their
relevance in modern trade debates. The findings underscore
mercantilism's impact on economic nationalism and its enduring legacy in
global commerce. }

\keywords{mercantilism, trade surplus, protectionism, economic
nationalism, balance of payments}

\authornote{\par{\addORCIDlink{Edison Achalma}{0000-0001-6996-3364}} 
\par{ }
\par{   Los autores no tienen conflictos de intereses que revelar.  Si
la voluntad es la fuerza motriz que impulsa al hombre, pues aquella es
quién hace al hombre. El trabajo va dirigido a los profesores que nos
acompañan a diario en nuestra formación profesional y a aquellas mujeres
que son nuestra inspiración diaria, nuestras madres.  Los roles de autor
se clasificaron utilizando la taxonomía de roles de colaborador (CRediT;
https://credit.niso.org/) de la siguiente manera:  Edison
Achalma:   redacción, conceptualización; Jeancarlos
Alcarráz:   redacción; Félix Bermudo:   redacción; Jhony
Conga:   redacción; Juan Curi:   redacción; Yuri Fernández:   redacción}
\par{La correspondencia relativa a este artículo debe dirigirse a Edison
Achalma, Email: \href{mailto:elmer.achalma.09@unsch.edu.pe}{elmer.achalma.09@unsch.edu.pe}}
}

\usepackage{pbalance} 
\usepackage{float}
\makeatletter
\let\oldtpt\ThreePartTable
\let\endoldtpt\endThreePartTable
\def\ThreePartTable{\@ifnextchar[\ThreePartTable@i \ThreePartTable@ii}
\def\ThreePartTable@i[#1]{\begin{figure}[!htbp]
\onecolumn
\begin{minipage}{0.5\textwidth}
\oldtpt[#1]
}
\def\ThreePartTable@ii{\begin{figure}[!htbp]
\onecolumn
\begin{minipage}{0.5\textwidth}
\oldtpt
}
\def\endThreePartTable{
\endoldtpt
\end{minipage}
\twocolumn
\end{figure}}
\makeatother


\makeatletter
\let\endoldlt\endlongtable		
\def\endlongtable{
\hline
\endoldlt}
\makeatother

\newenvironment{twocolumntable}% environment name
{% begin code
\begin{table*}[!htbp]%
\onecolumn%
}%
{%
\twocolumn%
\end{table*}%
}% end code

\urlstyle{same}



\makeatletter
\@ifpackageloaded{caption}{}{\usepackage{caption}}
\AtBeginDocument{%
\ifdefined\contentsname
  \renewcommand*\contentsname{Tabla de contenidos}
\else
  \newcommand\contentsname{Tabla de contenidos}
\fi
\ifdefined\listfigurename
  \renewcommand*\listfigurename{Listado de Figuras}
\else
  \newcommand\listfigurename{Listado de Figuras}
\fi
\ifdefined\listtablename
  \renewcommand*\listtablename{Listado de Tablas}
\else
  \newcommand\listtablename{Listado de Tablas}
\fi
\ifdefined\figurename
  \renewcommand*\figurename{Figura}
\else
  \newcommand\figurename{Figura}
\fi
\ifdefined\tablename
  \renewcommand*\tablename{Tabla}
\else
  \newcommand\tablename{Tabla}
\fi
}
\@ifpackageloaded{float}{}{\usepackage{float}}
\floatstyle{ruled}
\@ifundefined{c@chapter}{\newfloat{codelisting}{h}{lop}}{\newfloat{codelisting}{h}{lop}[chapter]}
\floatname{codelisting}{Listado}
\newcommand*\listoflistings{\listof{codelisting}{Listado de Listados}}
\makeatother
\makeatletter
\makeatother
\makeatletter
\@ifpackageloaded{caption}{}{\usepackage{caption}}
\@ifpackageloaded{subcaption}{}{\usepackage{subcaption}}
\makeatother
\makeatletter
\@ifpackageloaded{fontawesome5}{}{\usepackage{fontawesome5}}
\makeatother

% From https://tex.stackexchange.com/a/645996/211326
%%% apa7 doesn't want to add appendix section titles in the toc
%%% let's make it do it
\makeatletter
\xpatchcmd{\appendix}
  {\par}
  {\addcontentsline{toc}{section}{\@currentlabelname}\par}
  {}{}
\makeatother

%% Disable longtable counter
%% https://tex.stackexchange.com/a/248395/211326

\usepackage{etoolbox}

\makeatletter
\patchcmd{\LT@caption}
  {\bgroup}
  {\bgroup\global\LTpatch@captiontrue}
  {}{}
\patchcmd{\longtable}
  {\par}
  {\par\global\LTpatch@captionfalse}
  {}{}
\apptocmd{\endlongtable}
  {\ifLTpatch@caption\else\addtocounter{table}{-1}\fi}
  {}{}
\newif\ifLTpatch@caption
\makeatother

\begin{document}

\maketitle

\hypertarget{toc}{}
\tableofcontents
\newpage
\section[Introduction]{El mercantilismo}

\setcounter{secnumdepth}{-\maxdimen} % remove section numbering

\setlength\LTleft{0pt}


La edad media, dominada por formas de organización política feudal y un
sistema económico señorial, y de escaso avance científico, empieza a
romperse con la apertura de las rutas comerciales con el extremo oriente
y con la intensa acumulación de metales preciosos de las naciones
europeas, ocasionada por la conquista de América. La concepción
mercantilista empieza en el siglo XVI, pero toma fuerza en el siglo XVII
y abarca parte del siglo XVIII.

La característica principal del mercantilismo es la de ser el
pensamiento económico fundamental que surgió con la construcción de los
mercados nacionales europeos debido al auge comercial que se tenía por
el descubrimiento de nuevas rutas comerciales. El mercantilismo sienta
las bases teóricas que soportan la construcción y el fortalecimiento del
mercado interno, hecho que supone la regulación del comercio
internacional para ponerlo en función del desarrollo nacional.

En esta monografía encontraremos lo esencial para entender los
pensamientos económicos del mercantilismo, los grandes aportes a la
economía de ese entonces y la perpetuidad de algunas de sus ideas en el
tiempo, los autores de gran importancia que con sus aportes reforzaron
las ideas del mercantilismo.

\section{Mercantilismo}\label{mercantilismo}

\subsection{Concepto}\label{concepto}

Según Pérez J. (2011) ``el pensamiento mercantilista surgió a inicios
del XVI Y XVII Y abarco parte del siglo XVIII. El mercantilismo se basa
en dos teorías importantes: el superávit comercial y la política
proteccionista''.

\begin{itemize}
\tightlist
\item
  \hspace{0pt} \textbf{Teoría del superávit comercial:} es cuando el
  valor total de las exportaciones de un país es superior al valor total
  de sus importaciones.
\item
  \hspace{0pt} \textbf{Teoría de la política proteccionista:} es cuando
  un país implanta políticas de restricción comercial a la importación,
  para proteger a la Producción nacional.
\end{itemize}

La suma de estas dos políticas dio origen al mercantilismo pues el
superávit fiscal permitía en mayor medida el ingreso de divisas al país
y con las políticas proteccionistas, los productos internos tenían un
valor de venta mayor al de los productos importados, cumpliendo así la
primera teoría (mayor exportación que importación). Es así que en esta
época nace el capitalismo acompañado de un excesivo control de comercio
exterior, debido al elevado flujo de mercaderías que circulaba entre los
países.

No está de más decir que para realizar un estudio sobre la teoría del
comercio internacional se ara mención del mercantilismo, fuente del
proteccionismo que aun ejerce un atractivo, basada en argumentos
simplistas y erróneos, pero que fascinan por su sencillez y enfoque
nacionalista.

\subsection{Definición del
mercantilismo}\label{definiciuxf3n-del-mercantilismo}

Según Landreth y Colander
(\citeproc{ref-landrethHistoryEconomicThought2002}{2002}) el
mercantilismo es

el nombre que se le ha dado a 250 años de literatura económica y a la
práctica económica implantada entre 1500 y 1750. En tanto que la
literatura económica del escolasticismo fue escrita por los monjes
medievales, la teoría económica del mercantilismo fue trabajo de
mercaderes\ldots\ldots.

El mercantilismo se ha caracterizado como el tiempo en el que cada
persona era su propio economista (p.~36).

El mercantilismo se ha caracterizado por darle una mayor importancia a
la producción en el lugar del consumo, este tipo de economía surge en
contra del escolasticismo medieval al darle prioridad al poder y la
riqueza de la nación.

Los escritores mercantilistas estaban estrechamente conectado a la
política económica y al interés particular de los mercaderes y el
comercio. Regresando al punto principal para sintetizar la definición
del mercantilismo como una doctrina que surgió en los siglos XVI Y XVII
abarcando parte del siglo XVIII, comprendiéndose como el enriquecimiento
de las naciones mediante la acuñación o acumulación de metales
preciosos.

\subsection{El mercantilismo como pensamiento
económico.}\label{el-mercantilismo-como-pensamiento-econuxf3mico.}

Para Márquez y Silva
(\citeproc{ref-marquezPensamientoEconomicoCon2008}{2008}) el pensamiento
económico del mercantilismo se da a conocer en la siguiente cita.

El mercantilismo -como pensamiento económico- se entiende como el
conjunto de ideas que dominaron durante la época en que se construyeron
los mercados europeos, en su fase previa a la revolución industrial. El
mercantilismo transformó no sólo la forma de producir y comerciar, sino
que cambió la sociedad, las instituciones y el Estado, así como la forma
en que éstas se insertan en un proceso de globalización comercial. Esta
inserción obligó a gobernantes y pensadores a tener una mirada menos
interesada en los feudos y más en el conjunto de un emergente Estado
nación (p.46).

Los cambios que se dieron en ese entonces obligaron a las personas a
cambiar sus modos de producción para adecuarse a la globalización
comercial que se dio por la apertura de nuevas rutas comerciales y en
parte por las conquistas de las colonias. El flujo del comercio estuvo
en constante expansión permitiendo que las ideas del mercantilismo se
expandan y se conviertan con el tiempo en una las más importantes
escuelas económicas cuyas ideas perduran.

\section{Cinco preguntas hacia el
mercantilismo}\label{cinco-preguntas-hacia-el-mercantilismo}

\subsection{¿Cuáles fueron los antecedentes históricos de la escuela
mercantilista?}\label{cuuxe1les-fueron-los-antecedentes-histuxf3ricos-de-la-escuela-mercantilista}

Los antecedentes históricos de la escuela mercantilista sucedieron en el
periodo de la edad media con el feudalismo, la cual tenían una economía
basada en la tierra Y se caracterizó por ser autosuficiente. El periodo
del feudalismo tuvo a la escuela escolástica que estableció un puente
entre los antiguos griegos y romanos y los europeos del momento en que
se dinamiza la construcción de los mercados en la Europa medieval y abre
el camino al mercantilismo.

Para Brue y Grant
(\citeproc{ref-brueHistoriaPensamientoEconomico2009}{2009}) los
antecedentes históricos generados antes del mercantilismo son.

La autosuficiencia de la comunidad feudal lentamente le cedió el paso al
nuevo sistema del capitalismo mercantil. Las ciudades, que tenían un
crecimiento gradual durante la Edad Media, aumentaron en importancia. El
comercio floreció tanto al interior de cada país como entre los países y
se expandió la utilización del dinero. El descubrimiento del oro en el
hemisferio occidental facilitó el creciente volumen del comercio y
estimuló las teorías acerca de los metales preciosos. Los grandes
descubrimientos geográficos, basados en parte en el desarrollo de la
navegación, ampliaron la esfera del comercio (p.13).

A inicios del siglo XVI el comerciante mediaba cada vez más entre el
productor y el consumidor. Aun cuando a los ojos de la aristocracia de
terratenientes seguían siendo comerciantes despreciables, los
comerciantes capitalistas se convertían en figuras clave en el mundo de
los negocios. Las inversiones dejaron en ellos una gran cantidad de
ingresos a raíz de las nuevas rutas comerciales y la conquista de sus
respectivas colonias.

Entre el sigo XVII y XVIII el mercantilismo tomo mayor fuerza ya que el
modelo económico proponía que la riqueza de una nación sería calculada
dependiendo del tesoro de metales preciosos que ésta acumulara. En esa
época, Europa estaba sumergida en su tarea de expandir los imperios que
la formaban, y el tremendo saqueo de riquezas de sus nuevas colonias
trajo consigo la propuesta de este nuevo modelo económico.

La escuela mercantilista suponía la implementación de un sistema que
protegiera la balanza comercial positiva. Sin embargo, la justificación
real era que para las grandes potencias era imposible fijar su potencial
económico en el globo en base al comercio, porque en realidad el período
estuvo marcado por el ingreso de los tesoros saqueados de las provincias
conquistadas. De esta forma, las grandes potencias representaban su
capacidad económica en base a la cantidad de moneda que sus tesoros
podías soportar. Por otro lado, el modelo también hacía énfasis en la
comercialización de los excedentes de producción, en los mercados
internacionales. Esto hizo necesaria la existencia de una relación
cambiaria que valorara los bienes producto de las transacciones. Por lo
que, la divisa de esa época era el oro puro.

\subsection{Los principios más importantes de la escuela
mercantilista.}\label{los-principios-muxe1s-importantes-de-la-escuela-mercantilista.}

\begin{enumerate}
\def\labelenumi{\arabic{enumi}.}
\tightlist
\item
  \hspace{0pt} \textbf{El oro y la plata son la forma más deseable de
  riqueza}.
\end{enumerate}

Los mercantilistas tendían a igualar la riqueza de una nación con la
cantidad de lingotes de oro y plata que poseía. Algunos de los primeros
mercantilistas creían que esos metales preciosos eran el único tipo de
riqueza al que se podía aspirar. Todos valoraban los lingotes como la
única forma de alcanzar el poder y la riqueza. Por consiguiente, era
necesario un excedente de exportaciones de un país para generar pagos en
moneda dura. Incluso cuando estaban en guerra, las naciones exportaban
bienes para el enemigo, siempre y cuando los productos se pagaran en
oro.

El ensayista francés Michel de Montaigne escribió en 1580: ``La utilidad
de un hombre es el daño de otro\ldots{} No es posible obtener cualquier
utilidad si no es a costa de otro (Citado por Brue y Grant
(\citeproc{ref-brueHistoriaPensamientoEconomico2009}{2009}), p.~16)''.
Todos los países no exportaban simultáneamente más de lo que importaban.
Por consiguiente, el propio país debía promover las exportaciones y
acumular riquezas a costa de sus vecinos. Sólo una nación poderosa podía
conquistar y conservar colonias, dominar las rutas del comercio, ganar
guerras en contra de sus rivales y competir con éxito en el comercio
internacional. Conforme a este concepto estático de la vida económica,
había una cantidad fija de recursos económicos en el mundo; un país
podía incrementar sus recursos sólo a costa de otro.

El mercantilismo nacionalista condujo de una manera muy natural al
militarismo. Los poderosos navíos y las flotas mercantes eran un
requerimiento. Debido a que las pesquerías eran ``cunas de marinos'', es
decir, ya que eran terrenos de capacitación para el personal naval, los
mercantilistas le impusieron una ``cuaresma política'' a Inglaterra en
1549. Se prohibía por ley que las personas comieran carne ciertos días
de la semana, con el fin de asegurar un mercado doméstico para el
pescado y por tanto una demanda de marineros. Ese decreto se mantuvo
enérgicamente durante alrededor de un siglo y no desapareció de los
libros de estatutos sino hasta el siglo XIX.

Importación libre de impuestos de materia prima que no se produce
domésticamente, protección de bienes fabricados y materia prima que se
podían producir domésticamente y restricción a las exportaciones de
materia prima. Este énfasis en las exportaciones y la renuencia a
importar se ha llamado ``el temor de los bienes''. Los intereses del
comerciante tenían preeminencia sobre los del consumidor doméstico. Los
comerciantes recibían flujos de oro a cambio de sus exportaciones,
mientras que las restricciones sobre las importaciones reducían la
disponibilidad de bienes para el consumo local. En consecuencia, el oro
y la plata se acumulaban, supuestamente mejorando la riqueza y el poder
del país.

Las prohibiciones contra el movimiento exterior de materia prima
ayudaron a mantener bajos los precios de las exportaciones de bienes
terminados. Por ejemplo, una ley aprobada en 1565-1566 durante el
reinado de la reina Isabel prohibía la exportación de ovejas vivas. Las
penalidades por violar esa ley eran la confiscación de la propiedad, un
año en prisión y la amputación de la mano izquierda. La pena de muerte
se prescribía por una segunda ofensa. La exportación de lana cruda
estaba prohibida y se aplicaban las mismas penalidades en una ley
promulgada durante el reinado de Carlos II (1660-1685).

\begin{enumerate}
\def\labelenumi{\arabic{enumi}.}
\tightlist
\item
  \hspace{0pt} \textbf{Colonialismo y monopolio del comercio.}
\end{enumerate}

Los comerciantes capitalistas favorecían la colonización y querían
mantener a las colonias eternamente dependientes de la madre patria y
subordinadas a ella. Cualquiera de los beneficios que se extendían hacia
las colonias debido al crecimiento y el poder militar de la madre patria
era un subproducto accidental de la política de explotación.

Las Actas de Navegación Inglesas de 1651 y 1660 son buenos ejemplos de
esta política. Los bienes importados hacia Gran Bretaña y las colonias
se debían transportar en barcos ingleses o coloniales, o en barcos del
país en donde se originaban los bienes. Ciertos productos coloniales
sólo se debían vender a Inglaterra y otros se debían desembarcar en
Inglaterra antes de enviarlos por barco a países extranjeros. Las
importaciones extranjeras hacia las colonias estaban restringidas o
prohibidas. Las manufacturas coloniales fueron frenadas o en algunos
casos prohibidas, de manera que los territorios dependientes seguían
siendo proveedores de materia prima de bajo costo e importadores de
bienes fabricados en Inglaterra.

Oposición a peajes, impuestos internos y otras restricciones sobre el
movimiento de bienes. Los teóricos y practicantes mercantilistas
reconocían que los derechos de transporte y los impuestos podían
estrangular a las empresas de negocios e incrementar el precio de las
exportaciones de un país. Un ejemplo extremo de esto es la situación en
el río Elba en 1685. ¡Un envío de sesenta tablones de Sajonia a Hamburgo
requirió el pago de cincuenta y cuatro tablones en las estaciones de
peaje a lo largo del camino! En consecuencia, sólo seis tablones
llegaron al punto de destino.

Sin embargo, es importante observar que los mercantilistas no favorecían
el libre comercio interno en el sentido de permitir que las personas se
dedicaran a cualquier comercio que desearan. Por el contrario, los
mercantilistas preferían el otorgamiento de monopolios y privilegios
comerciales exclusivos, siempre que pudieran adquirirlos.

Un poderoso gobierno central, era necesario un poderoso gobierno central
para promover las metas del mercantilismo. El gobierno les otorgaba
privilegios de monopolio a las compañías dedicadas al comercio exterior
y restringía el libre ingreso a los negocios en el propio país, con el
fin de limitar la competencia. La agricultura, la minería y la industria
se promovían con subsidios del gobierno y se protegían de las
importaciones por medio de aranceles. Además, el gobierno regulaba
estrechamente los métodos de producción y la calidad de los bienes, de
manera que un país no se ganara una mala reputación para sus productos
en los mercados extranjeros, lo que obstaculizaba las importaciones. En
otras palabras, los mercantilistas confiaban muy poco en su propio
criterio y honestidad, y creían que el interés común de los comerciantes
requería que el gobierno prohibiera un trabajo deficiente y materiales
de mala calidad. El resultado fue un desconcertante laberinto de
regulaciones que gobernaba la producción de bienes.

Por consiguiente, se requería un poderoso gobierno nacional para
asegurar una regulación nacional uniforme. Los gobiernos centrales
también eran necesarios para lograr las metas expuestas anteriormente:
un nacionalismo, proteccionismo, colonialismo y comercio internacional
no obstaculizados por peajes y excesivos impuestos.

\subsection{¿Qué principios de la escuela mercantilista se convirtieron
en contribuciones
perdurables?}\label{quuxe9-principios-de-la-escuela-mercantilista-se-convirtieron-en-contribuciones-perdurables}

Para Brue y Grant
(\citeproc{ref-brueHistoriaPensamientoEconomico2009}{2009}) los
mercantilistas hicieron una contribución a la economía al hacer hincapié
en la importancia del comercio internacional. En ese contexto, también
desarrollaron la noción económica y contable de lo que hoy día se conoce
como la balanza de pagos entre una nación y el resto del mundo. Pero
fuera de esas contribuciones, los mercantilistas contribuyeron con muy
poco a la teoría económica como se conoce hoy en día. La mayoría de
ellos no logró captar que un país se volvía más rico no sólo al
empobrecer a sus vecinos, sino también al descubrir una mayor cantidad
de recursos naturales, producir más bienes de capital y utilizar la mano
de obra en una forma más eficiente (p.16).

Es un desarrollo mercantil con su propia justificación mercantil los
principales que pusieron en práctica fueron los países europeos y
tomaron como base el desarrollo económico y se pensaba que en el
desarrollo era lo que enriquecía a un país, tanto como las relaciones
mercantiles, la gran tarea de los países es poder garantizar un saldo
positivo de las exportaciones sobre las importaciones realizadas en un
país en vía de desarrollo, lograr una balanza comercial positiva, para
crecer necesitamos un apoyo a la producción , control de los
intercambios, apoyo de nuestros representantes legales, apoyo por la
marina, obtención y aumentos a las reservas monetarias.

Los mercantilistas hicieron una última contribución a la economía al
hacer hincapié en la importancia del comercio internacional. En ese
contexto, también desarrollaron la noción económica y contable de lo que
hoy día se conoce como la balanza de pagos entre una nación y el resto
del mundo. Pero fuera de esas contribuciones, los mercantilistas
(excepto Petty y tal vez Mun) contribuyeron con muy poco a la teoría
económica como se conoce hoy en día. La mayoría de ellos no logró captar
que un país se volvía más rico no sólo al empobrecer a sus vecinos, sino
también al descubrir una mayor cantidad de recursos naturales, producir
más bienes de capital y utilizar la mano de obra en una forma más
eficiente. Tampoco comprendieron que todas las naciones se enriquecen
simultáneamente a partir de la especialización y el comercio y que los
salarios más elevados para los trabajadores no conducen al ocio y a una
reducción de la participación de la fuerza laboral. Pero aun cuando los
mercantilistas hicieron muy pocas contribuciones directas a la teoría
económica sí contribuyeron indirectamente a la economía y al desarrollo
económico. En primer lugar, influyeron permanentemente en las actitudes
hacia el comerciante. La aristocracia medieval había clasificado a las
personas dedicadas a los negocios como ciudadanos despreciables de
segunda clase, sumergidos en el estiércol del comercio y el intercambio
de dinero. Los mercantilistas les dieron respetabilidad e importancia a
los comerciantes, con el argumento de que, cuando sus actividades están
canalizadas en la forma apropiada por el gobierno, no sólo se enriquecen
ellos mismos, sino también el rey y el reino. La aristocracia de
terratenientes con el tiempo empezó a participar en empresas de negocios
sin perder su posición ni su dignidad. Por último, entregaron a sus
hijos en matrimonio a los descendientes de familias de negocios, uniendo
así a los linajes aristocráticos con las fortunas comerciales. En
segundo término, el mercantilismo tuvo un impacto indirecto sobre la
economía al promover el nacionalismo, una fuerza que hoy día aún existe.
Las regulaciones del gobierno central se requieren cuando se necesitan
pesos, medidas y acuñación uniformes; cuando la producción y el comercio
todavía no se han desarrollado lo suficiente para permitir la confianza
en que la competencia les proporcione a los consumidores una amplia
elección de bienes; y cuando los riesgos financieros del comercio son
tan elevados que son necesarios privilegios de monopolio para inducir
una disposición a correr más riesgos de la que ocurriría de otra manera.

En tercer lugar, las privilegiadas compañías comerciales constituidas,
ancestros de la corporación moderna, ayudaron a transformar la
organización económica de Europa al introducir nuevos productos, abrir
mercados para los bienes fabricados y proporcionar incentivos para el
crecimiento de la inversión de capital. Por último, el mercantilismo
hizo una contribución permanente al desarrollo económico al expandir el
mercado interno, promover el libre movimiento de bienes sin las trabas
de los peajes, establecer leyes e impuestos uniformes y proteger a las
personas y los bienes en tránsito dentro y entre los países.

Implementación de la teoría Mercantilista en los primeros tiempos se
organizaban expediciones sueltas que enviaba cada armador o comerciante;
pero el contrabando y los piratas obligaron a las autoridades a formar
flotas compuestas por varias naves artilladas que navegaban juntas. A
partir de 1573 este sistema de ``flotas y galeones'' se volvió
obligatorio y oficial y todo navío debía ir o regresar de México
formando parte de la flota bajo pena de severas sanciones.

\subsection{El mercantilismo y su utilidad en las diferentes
épocas.}\label{el-mercantilismo-y-su-utilidad-en-las-diferentes-uxe9pocas.}

Algunas de las doctrinas del mercantilismo no han desaparecido por
completo; ciertas ideas y políticas presentes en los siglos XX y XXI se
asemejan a las ideas de hace 200 o 300 años. Por ejemplo, durante la
Gran Depresión mundial de la década de 1930, las naciones aprobaron
aranceles elevados y devaluaron sus monedas para restringir las
importaciones y promover las exportaciones. Esos aranceles estaban
diseñados para reducir las importaciones, de manera que la mano de obra
doméstica ociosa y los recursos de capital se pudieran emplear para
satisfacer la demanda de los bienes que previamente se importaban. Desde
un punto de vista ideal, expandirían la producción doméstica y el
ingreso. También se pensaba que la devaluación de la moneda reduciría
las importaciones de una nación al hacer que éstas resultaran más
costosas en términos de la moneda nacional. Además, la devaluación de la
moneda de un país supuestamente incrementaría sus exportaciones, debido
a que los extranjeros necesitarían menos unidades de su propia moneda
para comprar bienes producidos en el extranjero. Por desgracia, esas
políticas mercantilistas no funcionan como están diseñadas si los socios
comerciales ejercen represalias con incrementos en los aranceles y con
la devaluación de su propia moneda. Esas represalias fueron las que se
presentaron en la Gran Depresión. Una nación tras otra aprobó aranceles
más elevados y devaluó su moneda. El resultado total fue la pérdida de
las ganancias de la especialización y del comercio internacional y el
colapso del sistema monetario internacional. A finales de los años 1980
y principios de los años 1990, muchos estadounidenses expresaron una
gran preocupación por los considerables déficits en la balanza comercial
de Estados Unidos. Este ``temor de los bienes'' era legítimo hasta el
grado de que esos considerables déficits reflejaban condiciones
domésticas e internacionales que tarde o temprano necesitarían
corregirse. Sin embargo, este temor produjo propuestas para aprobar los
aranceles, imponer cuotas de importación, otorgar subsidios a los
exportadores, requerir un ``contenido doméstico'' en algunos productos
importados y permitir las exenciones de monopolio para las empresas
estadounidenses dedicadas a las exportaciones.

Los economistas señalaron que esa serie de políticas, si se aprobaban,
constituirían un regreso a los preceptos pasados de moda del
mercantilismo. También se ha acusado a Japón de adherirse a una política
de promoción de las exportaciones y restricción de las importaciones.
Sus continuos y considerables superávit comerciales a todo lo largo de
las décadas de los años 1980 y 1990 reflejaban en parte un ``temor de
los bienes'' en el extranjero. También reflejaban un deseo de
``capturar'' mercados internacionales rentables.

Al tener superávit comercial tan grande, se negaban a los consumidores
japoneses algunos de los beneficios potenciales del consumo, derivados
de la especialización y el comercio internacionales. Algunas naciones en
vías de desarrollo todavía promueven el nacionalismo como una forma de
superar el tribalismo y las lealtades locales que obstaculizan el
desarrollo económico. Con frecuencia también ofrecen concesiones de
monopolio para fomentar las nuevas inversiones y erigir barreras
comerciales con el fin de proteger a las industrias domésticas en su
inicio. El mercantilismo persiste hasta muy avanzada la primera década
del siglo XXI. En Estados Unidos el offshoring, la práctica de trasladar
las operaciones de empresas domésticas a naciones con una mano de obra
más barata ha atraído una considerable atención. Ahora, al ``temor de
los bienes'' se suma un ``temor de los servicios''. Puesto que los
trabajos fabriles se trasladan al extranjero, los trabajadores de las
industrias de servicio en Estados Unidos gozan de una razonable
seguridad.

Sin embargo, debido a los adelantos tecnológicos que reducen
significativamente el costo de las comunicaciones globales, las
operaciones como los centros de llamadas a clientes para servicios
financieros y apoyo técnico de computadoras se han reubicado de Estados
Unidos a la India. La pérdida de trabajo real y potencial debido al
offshoring ha incitado llamados en busca de protección. Los estándares
ambientales y laborales como un aspecto del comercio también han llegado
a ocupar el primer plano, y las economías avanzadas requieren
regulaciones más rigurosas para las naciones en vías de desarrollo.
Afirman que los estándares más flexibles en las naciones en vías de
desarrollo proporcionan una ventaja comercial injusta al mantener
precios más bajos a costa del ambiente y de la explotación de los
trabajadores. En las negociaciones más recientes de la Organización
Mundial de Comercio (OMC), las naciones en vías de desarrollo se unieron
para resistir a los intentos de las economías avanzadas de imponer
restricciones más rigurosas. La estrategia de China para el crecimiento
económico en la primera década del siglo XXI incluye el mantenimiento de
grandes excedentes comerciales, al hacer que las exportaciones sean
económicas y las importaciones costosas, y al mantener en un nivel bajo
el valor del yuan chino en los mercados de bolsa internacionales.

El enfoque mercantilista de China ha impulsado un poderoso crecimiento
económico, pero también ha atraído las críticas internacionales y los
llamados para erigir barreras comerciales con el fin de compensar lo
que, en 2004 John Kerry, el candidato demócrata a la presidencia de
Estados Unidos llamó ``una manipulación predatoria de la moneda''. En
resumen, las ideas mercantilistas aún sobreviven. Sin embargo, es
importante comprender que las ideas y las políticas sólo reflejan
aspectos de la doctrina total del mercantilismo. Además, hoy en día las
naciones aplican esas ideas en circunstancias diferentes, por razones
distintas y en el contexto de políticas sociales diversas a las de la
época mercantilista.

\subsection{¿A quiénes beneficiaba o trataba de beneficiar la escuela
mercantilista?}\label{a-quiuxe9nes-beneficiaba-o-trataba-de-beneficiar-la-escuela-mercantilista}

Para Brue y Grant
(\citeproc{ref-brueHistoriaPensamientoEconomico2009}{2009}) ``esta
doctrina beneficiaba a los comerciantes capitalistas, a los reyes y
funcionarios del gobierno. En específico, favorecía a quienes eran más
poderosos y tenían los monopolios y privilegios mejores'' (p.16).

Las decisiones autónomas por de derecho divino decididas por el rey,
implicaba el respaldado de la iglesia católica, aquella que cedió sus
tierras, sus feudos a posición de los terratenientes, productores,
quienes años más tarde tomarían el poder político, social y económico de
dichas naciones, concediéndose como burgueses hacia esos entonces, los
monarcas que permanecieron en su trono aumentaban su poder, tomaban al
máximo el manejo de su nación de mano con sus funcionarios, familiares
que sumaban a la decisión de un futura situación. El monopolio
concentrado en su comercio exterior, cual generaba ingresos desmedidos
obteniéndose tas las invasiones a América, agregando su proteccionismo y
nacionalismo contribuían en superar sus propios límites de riquezas,
maximizándolas y propiciando su nivel político, social y económico.

Para Landreth y Colander
(\citeproc{ref-landrethHistoryEconomicThought2002}{2002}) el propósito
de la actividad económica, de acuerdo con la mayoría de los
mercantilistas, era la producción\ldots.

Para los mercantilistas, la riqueza de una nación no estaba definida en
términos de la suma de la riqueza individual. Se proponía aumentar la
riqueza de una nación al estimular de manera simultánea la producción,
aumentar las exportaciones y restringir el consumo doméstico (p.37).

La finalidad del mercantilismo constaba en aumentar la riqueza de una
nación, mas no una riqueza individual; considerando estimular de manera
continua la producción de bienes de todo orden, con índices mínimos en
pérdida de ineficiencia social por parte de los productores, obteniendo
mayores ganancias, recaudaciones fiscales sustentadas por los
consumidores; superando el índice de riqueza y poder, tras monopolizar
el mercado comercial (externo). Un comercio externo, de exportaciones
restringiendo el consumo doméstico, para una mayor acumulación de
riqueza exhaustiva.

Una característica innata del mercantilismo es la política nacionalista
y el proteccionismo, se promulgaba que el estado debe ejercer un férreo
control sobre la industria y el comercio para aumentar el poder de la
nación al lograr que las exportaciones superen el valor de las
importaciones. El mercantilismo tuvo gran éxito al estimular el
crecimiento de la industria.

Según Landreth y Colander
(\citeproc{ref-landrethHistoryEconomicThought2002}{2002}) un país debe
impulsar las exportaciones y desalentar las importaciones mediante los
aranceles, las cuotas, los subsidios, los impuestos y medidas similares,
a fin de lograr una balanza favorable de comercio. Debe estimularse la
producción mediante la intervención gubernamental en la economía
doméstica y a través de la regulación del comercio exterior (p.37).

Las exportaciones incrementan los fondos de capital, vender sus
productos a un mercado externo no en uno nacional. Adquiriendo en
minoría bienes externos, por la mínima cantidad de ingreso de estos,
determinados por los aranceles (impuesto establecido por un país
importador hacia uno exportador), impuestos (dinero que se ha de pagar
al estado) y subsidios (pago del gobierno a un productor o varios) este
último aumenta la oferta nacional, haciendo más beneficiosa al consumo
de productos nacionales. Una intervención gubernamental, estatal o del
estado estimulando la producción, generando un mayor progreso en la
economía local, siendo estos bienes producidos y consumidos en su
círculo comercial, controlando la cantidad de entrada y salida de
productos y el consumo de estos.

\subsection{¿De qué forma la escuela mercantilista era válida, útil o
correcta para su
época?}\label{de-quuxe9-forma-la-escuela-mercantilista-era-vuxe1lida-uxfatil-o-correcta-para-su-uxe9poca}

Para Brue y Grant
(\citeproc{ref-brueHistoriaPensamientoEconomico2009}{2009}) los
argumentos a favor de la acumulación de lingotes de oro y plata, aun
cuando exagerados, tenían cierto sentido en un periodo de transición
entre la economía autosuficiente de la Edad Media y la economía de
dinero y crédito de los tiempos modernos (p.18).

El mercantilismo siendo un régimen económico que imperó durante los
siglos XVI y XVII en Europa y fue aplicado, por consiguiente, en
América.

Y promulgaba que el estado debe ejercer un férreo control sobre la
industria y el comercio para aumentar el poder de la nación al lograr
que las exportaciones superen el valor de las importaciones. El
mercantilismo tuvo gran éxito al estimular el crecimiento de la
industria. La esencia de la actividad económica se centra en la
adquisición de monedas y metales de oro y plata como única forma de
enriquecerse el estado.

El mercantilismo es centralista al considerar que es el propio estado el
que debe organizar y programar la adquisición de metales preciosos. Con
el mercantilismo aparece por primera vez el concepto de balanza
comercial, ya que los países se ven forzados a desarrollar al máximo las
exportaciones de productos pagaderos en oro y plata y reducir en lo
posible las importaciones que supongan pagos en este tipo de moneda. El
mercantilismo propicia una balanza comercial constantemente favorable.

Esta doctrina implica una gran dedicación al marco legal que regula la
producción y el comercio, como vías de conseguir una óptima organización
que lo facilite: desarrollo de la infraestructura del país,
comunicaciones, puertos, desarrollos de mercados exteriores que absorban
exportaciones, etc.

En conclusión, nos remontamos a aquellas épocas, siguientes del
medievalismo, épocas propicias para enriquecerse de la manera más
simple, sometiendo a ciudades a un estricto control, y de una creación
de ciudades estado, implicando que el poder político lo tenían los
reyes, monarcas, aquellos que destinaban el futuro de sus ciudades que
estaba impuesta por la máxima cantidad de oro y de riquezas poseídas.

\section{Principales Exponentes Del Pensamiento
Mercantilista}\label{principales-exponentes-del-pensamiento-mercantilista}

Roll (\citeproc{ref-rollHistoriaDoctrinasEconomicas2014}{2014}) afirma
que:

Así, contemplamos los actos de un labrador en la siembra, cuando arroja
el grano abundante y bueno en la tierra lo tomamos más bien por un loco
que por un labrador; pero cuando pensamos en su tarea en la época de la
cosecha, que es el final de sus esfuerzos, descubrimos el mérito pingue
producto de sus actos (p.26).

Vemos aquí que el legado especial del director de la Compañía de las
Indias Orientales ha tomado un carácter general; que nos hace saber y
entender la importancia del comercio en la economía, puesto que los
mercantilistas creían que la riqueza de una nación estaba en el comercio
de metales precioso, ahora en pleno siglo XXI ponemos afirmar
rotundamente que el comercio de metales no genera ganancia
significativa.

Brue y Grant (\citeproc{ref-brueHistoriaPensamientoEconomico2009}{2009})
afirma que:

Colbert era un defensor de la acumulación de lingotes, creía que la
fuerza de un Estado depende de sus finanzas, las cuales se bajan en su
cobranza de impuestos, y que, a su vez, la recolección de impuestos es
mayor si abunda el dinero. Favorecía la expansión de las exportaciones,
la reducción de las importaciones y las leyes que impedía la salida de
lingotes del país (p.38).

Jean Baptiste Colbert como la mayoría de mercantilistas estaba a favor
de la acumulación de metales preciosos, por otro lado, daba indicios se
lo que hoy llamamos balanza comercial favorable, el cual nos menciona
para que una a economía de un país sea beneficiosa las exportaciones
deben ser mayor que las importaciones, caso contrario el país estaría en
déficit.

\section{Conclusión}\label{conclusiuxf3n}

\begin{enumerate}
\def\labelenumi{\arabic{enumi}.}
\item
  \hspace{0pt} El mercantilismo es un conjunto de ideas que considera
  que la prosperidad de una nación o estado depende del capital que
  pueda tener es toda búsqueda de riqueza y de predominio territorial,
  como la demostración de ser una gran potencia y en mayor escala la
  única.
\item
  \hspace{0pt} También se puede decir que fue un conjunto de ideas
  económicas que tendía al fortalecimiento de los nuevos estados
  europeos, mediante la creciente intervención del gobierno en la
  economía y el exagerado nacionalismo en las relaciones entre los
  distintos países.
\item
  \hspace{0pt} Fue la corriente que impulso a las grandes potencias a
  conquistar los mares, nuevos horizontes con el fin de aumentar sus
  riquezas y territorios.
\item
  \hspace{0pt} Esta corriente demuestra según sus antecedentes que toda
  nación que se encierra termina explotando, como también que la grande
  potencia siempre busca su conveniencia en toda negociación, es por
  esto que América se convirtió en granero de Europa.
\item
  Importancia preponderante dada a los metales preciosos que el Estado
  debía procurar acrecentar al máximo. Para alcanzar tal objetivo había
  que tratar de obtener una Balanza Comercial Favorable, fomentando
  especialmente la exportación de artículos industrializados y
  restringiendo su importación. Esto significaba, a su vez, el estímulo
  de las industrias de elaboración, para el consumo interno y para las
  ventas al exterior. Así se formaron las primeras Manufacturas,
  establecimientos de magnitud mucho mayor que los talleres de artesanía
  corrientes hasta entonces.
\item
  \hspace{0pt} Conveniencia de la adquisición de colonias como fuentes
  de materias primas (eventualmente de metales preciosos) y mercados de
  los bienes elaborados en la metrópoli. La aplicación de estas ideas
  contribuyó, especialmente, al progreso de Francia e Inglaterra.
\end{enumerate}

\section{Publicaciones Similares}\label{publicaciones-similares}

Si te interesó este artículo, te recomendamos que explores otros blogs y
recursos relacionados que pueden ampliar tus conocimientos. Aquí te dejo
algunas sugerencias:

\begin{enumerate}
\def\labelenumi{\arabic{enumi}.}
\tightlist
\item
  \href{https://achalmaedison.netlify.app/blog/posts/2015-05-14-el-aborto/index.pdf}{\faIcon{file-pdf}}
  \href{https://achalmaedison.netlify.app/blog/posts/2015-05-14-el-aborto}{El
  Aborto}
\item
  \href{https://achalmaedison.netlify.app/blog/posts/2017-04-23-sitios-web-asombrosos/index.pdf}{\faIcon{file-pdf}}
  \href{https://achalmaedison.netlify.app/blog/posts/2017-04-23-sitios-web-asombrosos}{Sitios
  Web Asombrosos}
\item
  \href{https://achalmaedison.netlify.app/blog/posts/2017-05-23-el-mercantilismo/index.pdf}{\faIcon{file-pdf}}
  \href{https://achalmaedison.netlify.app/blog/posts/2017-05-23-el-mercantilismo}{El
  Mercantilismo}
\item
  \href{https://achalmaedison.netlify.app/blog/posts/2020-05-23-comandos-de-google-assistant/index.pdf}{\faIcon{file-pdf}}
  \href{https://achalmaedison.netlify.app/blog/posts/2020-05-23-comandos-de-google-assistant}{Comandos
  De Google Assistant}
\item
  \href{https://achalmaedison.netlify.app/blog/posts/2020-09-15-plan-de-negocio-exportacion-de-trucha-arcoires/index.pdf}{\faIcon{file-pdf}}
  \href{https://achalmaedison.netlify.app/blog/posts/2020-09-15-plan-de-negocio-exportacion-de-trucha-arcoires}{Plan
  De Negocio Exportacion De Trucha Arcoires}
\item
  \href{https://achalmaedison.netlify.app/blog/posts/2021-07-13-plan-de-negocio-exportacion-de-tuna/index.pdf}{\faIcon{file-pdf}}
  \href{https://achalmaedison.netlify.app/blog/posts/2021-07-13-plan-de-negocio-exportacion-de-tuna}{Plan
  De Negocio Exportacion De Tuna}
\item
  \href{https://achalmaedison.netlify.app/blog/posts/2021-07-14-comandos-de-blogdown/index.pdf}{\faIcon{file-pdf}}
  \href{https://achalmaedison.netlify.app/blog/posts/2021-07-14-comandos-de-blogdown}{Comandos
  De Blogdown}
\item
  \href{https://achalmaedison.netlify.app/blog/posts/2021-10-01-gestion-publica-y-administracion-publica/index.pdf}{\faIcon{file-pdf}}
  \href{https://achalmaedison.netlify.app/blog/posts/2021-10-01-gestion-publica-y-administracion-publica}{Gestion
  Publica Y Administracion Publica}
\item
  \href{https://achalmaedison.netlify.app/blog/posts/2021-10-01-reformas-y-modernizacion-de-la-gestion-publica/index.pdf}{\faIcon{file-pdf}}
  \href{https://achalmaedison.netlify.app/blog/posts/2021-10-01-reformas-y-modernizacion-de-la-gestion-publica}{Reformas
  Y Modernizacion De La Gestion Publica}
\item
  \href{https://achalmaedison.netlify.app/blog/posts/2022-01-23-cadena\%20de\%20suministros/index.pdf}{\faIcon{file-pdf}}
  \href{https://achalmaedison.netlify.app/blog/posts/2022-01-23-cadena\%20de\%20suministros}{Cadena
  De Suministros}
\item
  \href{https://achalmaedison.netlify.app/blog/posts/2022-04-22-economia-agraria/index.pdf}{\faIcon{file-pdf}}
  \href{https://achalmaedison.netlify.app/blog/posts/2022-04-22-economia-agraria}{Economia
  Agraria}
\item
  \href{https://achalmaedison.netlify.app/blog/posts/2022-06-02-impacto-del-cambio-climatico/index.pdf}{\faIcon{file-pdf}}
  \href{https://achalmaedison.netlify.app/blog/posts/2022-06-02-impacto-del-cambio-climatico}{Impacto
  Del Cambio Climatico}
\item
  \href{https://achalmaedison.netlify.app/blog/posts/2023-05-11-cualidades-de-los-servidores-publicos/index.pdf}{\faIcon{file-pdf}}
  \href{https://achalmaedison.netlify.app/blog/posts/2023-05-11-cualidades-de-los-servidores-publicos}{Cualidades
  De Los Servidores Publicos}
\item
  \href{https://achalmaedison.netlify.app/blog/posts/2023-05-12-la-economia-peruana-entre-1970-1990/index.pdf}{\faIcon{file-pdf}}
  \href{https://achalmaedison.netlify.app/blog/posts/2023-05-12-la-economia-peruana-entre-1970-1990}{La
  Economia Peruana Entre 1970 1990}
\item
  \href{https://achalmaedison.netlify.app/blog/posts/2023-05-16-economia-regional/index.pdf}{\faIcon{file-pdf}}
  \href{https://achalmaedison.netlify.app/blog/posts/2023-05-16-economia-regional}{Economia
  Regional}
\end{enumerate}

Esperamos que encuentres estas publicaciones igualmente interesantes y
útiles. ¡Disfruta de la lectura!

\section{Referencias}\label{referencias}

\phantomsection\label{refs}
\begin{CSLReferences}{1}{0}
\bibitem[\citeproctext]{ref-brueHistoriaPensamientoEconomico2009}
Brue, S. L., \& Grant, R. R. (2009). \emph{Historia {Del Pensamiento
Económico}}. Cengage Learning. \url{http://latinoamerica.cengage.com}

\bibitem[\citeproctext]{ref-landrethHistoryEconomicThought2002}
Landreth, H., \& Colander, D. C. (2002). \emph{History of Economic
Thought} (4th ed). Houghton Mifflin.

\bibitem[\citeproctext]{ref-marquezPensamientoEconomicoCon2008}
Márquez, Y., \& Silva, J. (2008). \emph{‪Pensamiento económico con
énfasis en pensamiento económico público‬}.

\bibitem[\citeproctext]{ref-rollHistoriaDoctrinasEconomicas2014}
Roll, E. (2014). \emph{Historia De Las Doctrinas Económicas}. Fondo de
Cultura Económica.

\end{CSLReferences}

ASOCIACION FONDO DE INVESTIGACION y EDITORES (2009). Lima - Perù
\hspace{0pt}Introduccion a la economia, enfoque social.

PORTO, J. P. (2017). Definición del mercantilismo. Definición.de, 1-2.






\end{document}
