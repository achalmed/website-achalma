\documentclass[
  doc,
  floatsintext,
  longtable,
  a4paper,
  nolmodern,
  notxfonts,
  notimes,
  colorlinks=true,linkcolor=blue,citecolor=blue,urlcolor=blue]{apa7}

\usepackage{amsmath}
\usepackage{amssymb}

\geometry{inner=1in, outer=1in}
\fancyhfoffset[LE,RO]{0cm}


\usepackage[bidi=default]{babel}
\babelprovide[main,import]{spanish}
\StartBabelCommands{spanish}{captions} [unicode, fontenc=TU EU1 EU2, charset=utf8] \SetString{\keywordname}{Palabras
Claves}
\EndBabelCommands


% get rid of language-specific shorthands (see #6817):
\let\LanguageShortHands\languageshorthands
\def\languageshorthands#1{}

\RequirePackage{longtable}
\RequirePackage{threeparttablex}

\makeatletter
\renewcommand{\paragraph}{\@startsection{paragraph}{4}{\parindent}%
	{0\baselineskip \@plus 0.2ex \@minus 0.2ex}%
	{-.5em}%
	{\normalfont\normalsize\bfseries\typesectitle}}

\renewcommand{\subparagraph}[1]{\@startsection{subparagraph}{5}{0.5em}%
	{0\baselineskip \@plus 0.2ex \@minus 0.2ex}%
	{-\z@\relax}%
	{\normalfont\normalsize\bfseries\itshape\hspace{\parindent}{#1}\textit{\addperi}}{\relax}}
\makeatother




\usepackage{longtable, booktabs, multirow, multicol, colortbl, hhline, caption, array, float, xpatch}
\setcounter{topnumber}{2}
\setcounter{bottomnumber}{2}
\setcounter{totalnumber}{4}
\renewcommand{\topfraction}{0.85}
\renewcommand{\bottomfraction}{0.85}
\renewcommand{\textfraction}{0.15}
\renewcommand{\floatpagefraction}{0.7}

\usepackage{tcolorbox}
\tcbuselibrary{listings,theorems, breakable, skins}
\usepackage{fontawesome5}

\definecolor{quarto-callout-color}{HTML}{909090}
\definecolor{quarto-callout-note-color}{HTML}{0758E5}
\definecolor{quarto-callout-important-color}{HTML}{CC1914}
\definecolor{quarto-callout-warning-color}{HTML}{EB9113}
\definecolor{quarto-callout-tip-color}{HTML}{00A047}
\definecolor{quarto-callout-caution-color}{HTML}{FC5300}
\definecolor{quarto-callout-color-frame}{HTML}{ACACAC}
\definecolor{quarto-callout-note-color-frame}{HTML}{4582EC}
\definecolor{quarto-callout-important-color-frame}{HTML}{D9534F}
\definecolor{quarto-callout-warning-color-frame}{HTML}{F0AD4E}
\definecolor{quarto-callout-tip-color-frame}{HTML}{02B875}
\definecolor{quarto-callout-caution-color-frame}{HTML}{FD7E14}

%\newlength\Oldarrayrulewidth
%\newlength\Oldtabcolsep


\usepackage{hyperref}



\usepackage{color}
\usepackage{fancyvrb}
\newcommand{\VerbBar}{|}
\newcommand{\VERB}{\Verb[commandchars=\\\{\}]}
\DefineVerbatimEnvironment{Highlighting}{Verbatim}{commandchars=\\\{\}}
% Add ',fontsize=\small' for more characters per line
\usepackage{framed}
\definecolor{shadecolor}{RGB}{241,243,245}
\newenvironment{Shaded}{\begin{snugshade}}{\end{snugshade}}
\newcommand{\AlertTok}[1]{\textcolor[rgb]{0.68,0.00,0.00}{#1}}
\newcommand{\AnnotationTok}[1]{\textcolor[rgb]{0.37,0.37,0.37}{#1}}
\newcommand{\AttributeTok}[1]{\textcolor[rgb]{0.40,0.45,0.13}{#1}}
\newcommand{\BaseNTok}[1]{\textcolor[rgb]{0.68,0.00,0.00}{#1}}
\newcommand{\BuiltInTok}[1]{\textcolor[rgb]{0.00,0.23,0.31}{#1}}
\newcommand{\CharTok}[1]{\textcolor[rgb]{0.13,0.47,0.30}{#1}}
\newcommand{\CommentTok}[1]{\textcolor[rgb]{0.37,0.37,0.37}{#1}}
\newcommand{\CommentVarTok}[1]{\textcolor[rgb]{0.37,0.37,0.37}{\textit{#1}}}
\newcommand{\ConstantTok}[1]{\textcolor[rgb]{0.56,0.35,0.01}{#1}}
\newcommand{\ControlFlowTok}[1]{\textcolor[rgb]{0.00,0.23,0.31}{\textbf{#1}}}
\newcommand{\DataTypeTok}[1]{\textcolor[rgb]{0.68,0.00,0.00}{#1}}
\newcommand{\DecValTok}[1]{\textcolor[rgb]{0.68,0.00,0.00}{#1}}
\newcommand{\DocumentationTok}[1]{\textcolor[rgb]{0.37,0.37,0.37}{\textit{#1}}}
\newcommand{\ErrorTok}[1]{\textcolor[rgb]{0.68,0.00,0.00}{#1}}
\newcommand{\ExtensionTok}[1]{\textcolor[rgb]{0.00,0.23,0.31}{#1}}
\newcommand{\FloatTok}[1]{\textcolor[rgb]{0.68,0.00,0.00}{#1}}
\newcommand{\FunctionTok}[1]{\textcolor[rgb]{0.28,0.35,0.67}{#1}}
\newcommand{\ImportTok}[1]{\textcolor[rgb]{0.00,0.46,0.62}{#1}}
\newcommand{\InformationTok}[1]{\textcolor[rgb]{0.37,0.37,0.37}{#1}}
\newcommand{\KeywordTok}[1]{\textcolor[rgb]{0.00,0.23,0.31}{\textbf{#1}}}
\newcommand{\NormalTok}[1]{\textcolor[rgb]{0.00,0.23,0.31}{#1}}
\newcommand{\OperatorTok}[1]{\textcolor[rgb]{0.37,0.37,0.37}{#1}}
\newcommand{\OtherTok}[1]{\textcolor[rgb]{0.00,0.23,0.31}{#1}}
\newcommand{\PreprocessorTok}[1]{\textcolor[rgb]{0.68,0.00,0.00}{#1}}
\newcommand{\RegionMarkerTok}[1]{\textcolor[rgb]{0.00,0.23,0.31}{#1}}
\newcommand{\SpecialCharTok}[1]{\textcolor[rgb]{0.37,0.37,0.37}{#1}}
\newcommand{\SpecialStringTok}[1]{\textcolor[rgb]{0.13,0.47,0.30}{#1}}
\newcommand{\StringTok}[1]{\textcolor[rgb]{0.13,0.47,0.30}{#1}}
\newcommand{\VariableTok}[1]{\textcolor[rgb]{0.07,0.07,0.07}{#1}}
\newcommand{\VerbatimStringTok}[1]{\textcolor[rgb]{0.13,0.47,0.30}{#1}}
\newcommand{\WarningTok}[1]{\textcolor[rgb]{0.37,0.37,0.37}{\textit{#1}}}

\providecommand{\tightlist}{%
  \setlength{\itemsep}{0pt}\setlength{\parskip}{0pt}}
\usepackage{longtable,booktabs,array}
\usepackage{calc} % for calculating minipage widths
% Correct order of tables after \paragraph or \subparagraph
\usepackage{etoolbox}
\makeatletter
\patchcmd\longtable{\par}{\if@noskipsec\mbox{}\fi\par}{}{}
\makeatother
% Allow footnotes in longtable head/foot
\IfFileExists{footnotehyper.sty}{\usepackage{footnotehyper}}{\usepackage{footnote}}
\makesavenoteenv{longtable}

\usepackage{graphicx}
\makeatletter
\newsavebox\pandoc@box
\newcommand*\pandocbounded[1]{% scales image to fit in text height/width
  \sbox\pandoc@box{#1}%
  \Gscale@div\@tempa{\textheight}{\dimexpr\ht\pandoc@box+\dp\pandoc@box\relax}%
  \Gscale@div\@tempb{\linewidth}{\wd\pandoc@box}%
  \ifdim\@tempb\p@<\@tempa\p@\let\@tempa\@tempb\fi% select the smaller of both
  \ifdim\@tempa\p@<\p@\scalebox{\@tempa}{\usebox\pandoc@box}%
  \else\usebox{\pandoc@box}%
  \fi%
}
% Set default figure placement to htbp
\def\fps@figure{htbp}
\makeatother







\usepackage{newtx}

\defaultfontfeatures{Scale=MatchLowercase}
\defaultfontfeatures[\rmfamily]{Ligatures=TeX,Scale=1}





\title{Guía para Comenzar a Crear Sitios Web con Blogdown: Utilizando R
y Hugo}


\shorttitle{Crear Sitios con Blogdown}


\usepackage{etoolbox}



\ccoppy{\textcopyright~2021}



\author{Edison Achalma}



\affiliation{
{Escuela Profesional de Economía, Universidad Nacional de San Cristóbal
de Huamanga}}




\leftheader{Achalma}

\date{2021-07-14}


\abstract{This article introduces Blogdown, an R package for creating
static websites using R Markdown and Hugo. It provides a step-by-step
guide on how to install Blogdown, set up a new site, preview it locally,
create blog posts, build the site, check for errors, update Hugo and
dependencies, customize the site's theme, and deploy the site using
services like Netlify. The aim is to empower users, from beginners to
advanced, to build and manage their own web presence with ease and
flexibility. }

\keywords{Blogdown, R Markdown, Hugo, static websites, web development}

\authornote{\par{\addORCIDlink{Edison Achalma}{0000-0001-6996-3364}} 
\par{ }
\par{   El autor no tiene conflictos de interés que revelar.    Los
roles de autor se clasificaron utilizando la taxonomía de roles de
colaborador (CRediT; https://credit.niso.org/) de la siguiente
manera:  Edison Achalma:   conceptualización, redacción}
\par{La correspondencia relativa a este artículo debe dirigirse a Edison
Achalma, Email: \href{mailto:elmer.achalma.09@unsch.edu.pe}{elmer.achalma.09@unsch.edu.pe}}
}

\makeatletter
\let\endoldlt\endlongtable
\def\endlongtable{
\hline
\endoldlt
}
\makeatother

\urlstyle{same}



\makeatletter
\@ifpackageloaded{caption}{}{\usepackage{caption}}
\AtBeginDocument{%
\ifdefined\contentsname
  \renewcommand*\contentsname{Tabla de contenidos}
\else
  \newcommand\contentsname{Tabla de contenidos}
\fi
\ifdefined\listfigurename
  \renewcommand*\listfigurename{Listado de Figuras}
\else
  \newcommand\listfigurename{Listado de Figuras}
\fi
\ifdefined\listtablename
  \renewcommand*\listtablename{Listado de Tablas}
\else
  \newcommand\listtablename{Listado de Tablas}
\fi
\ifdefined\figurename
  \renewcommand*\figurename{Figura}
\else
  \newcommand\figurename{Figura}
\fi
\ifdefined\tablename
  \renewcommand*\tablename{Tabla}
\else
  \newcommand\tablename{Tabla}
\fi
}
\@ifpackageloaded{float}{}{\usepackage{float}}
\floatstyle{ruled}
\@ifundefined{c@chapter}{\newfloat{codelisting}{h}{lop}}{\newfloat{codelisting}{h}{lop}[chapter]}
\floatname{codelisting}{Listado}
\newcommand*\listoflistings{\listof{codelisting}{Listado de Listados}}
\makeatother
\makeatletter
\makeatother
\makeatletter
\@ifpackageloaded{caption}{}{\usepackage{caption}}
\@ifpackageloaded{subcaption}{}{\usepackage{subcaption}}
\makeatother
\makeatletter
\@ifpackageloaded{fontawesome5}{}{\usepackage{fontawesome5}}
\makeatother

% From https://tex.stackexchange.com/a/645996/211326
%%% apa7 doesn't want to add appendix section titles in the toc
%%% let's make it do it
\makeatletter
\xpatchcmd{\appendix}
  {\par}
  {\addcontentsline{toc}{section}{\@currentlabelname}\par}
  {}{}
\makeatother

%% Disable longtable counter
%% https://tex.stackexchange.com/a/248395/211326

\usepackage{etoolbox}

\makeatletter
\patchcmd{\LT@caption}
  {\bgroup}
  {\bgroup\global\LTpatch@captiontrue}
  {}{}
\patchcmd{\longtable}
  {\par}
  {\par\global\LTpatch@captionfalse}
  {}{}
\apptocmd{\endlongtable}
  {\ifLTpatch@caption\else\addtocounter{table}{-1}\fi}
  {}{}
\newif\ifLTpatch@caption
\makeatother

\begin{document}

\maketitle

\hypertarget{toc}{}
\tableofcontents
\newpage
\section[Introduction]{Guía para Comenzar a Crear Sitios Web con
Blogdown}

\setcounter{secnumdepth}{-\maxdimen} % remove section numbering

\setlength\LTleft{0pt}


En la era digital actual, la creación de sitios web personales o
profesionales no tiene por qué ser un proceso complejo. Si eres un
apasionado de la programación y quieres adentrarte en el mundo de la
creación de sitios web estáticos, ¡estás en el lugar indicado! Hoy te
hablaré sobre \textbf{Blogdown}, una herramienta poderosa para crear
sitios web utilizando R Markdown y Hugo. A través de este artículo,
descubrirás cómo aprovechar el potencial de estos paquetes de R para
crear un sitio web desde cero, de manera sencilla y con un control total
sobre el contenido.

Blogdown es ideal para quienes buscan una forma rápida, flexible y
eficaz de generar sitios web estáticos, y en este artículo, te guiaré
paso a paso para que puedas crear tu propio sitio sin dificultades.
¿Estás listo para transformar tus ideas en páginas web de calidad?
¡Sigue leyendo!

\section{Comenzando con Blogdown}\label{comenzando-con-blogdown}

\subsection{¿Qué es Blogdown y cómo se
instala?}\label{quuxe9-es-blogdown-y-cuxf3mo-se-instala}

Blogdown es un paquete de R que facilita la creación de sitios web
estáticos, utilizando R Markdown para la gestión de contenido y Hugo
para la generación rápida de páginas. Si ya tienes R y RStudio
instalados en tu sistema, ¡es muy fácil empezar!

Para instalar Blogdown, solo necesitas ejecutar el siguiente comando en
tu consola de R:

\begin{Shaded}
\begin{Highlighting}[]
\FunctionTok{install.packages}\NormalTok{(}\StringTok{"blogdown"}\NormalTok{)}
\end{Highlighting}
\end{Shaded}

Una vez que hayas instalado Blogdown, estás listo para empezar a
construir tu sitio web.

\subsection{Creación de un Nuevo
Sitio}\label{creaciuxf3n-de-un-nuevo-sitio}

El primer paso para empezar a trabajar con Blogdown es crear un nuevo
sitio. Esto lo puedes hacer fácilmente con el comando:

\begin{Shaded}
\begin{Highlighting}[]
\NormalTok{blogdown}\SpecialCharTok{::}\FunctionTok{new\_site}\NormalTok{()}
\end{Highlighting}
\end{Shaded}

Este comando creará un nuevo sitio web en el directorio que selecciones,
generando la estructura básica del sitio, incluyendo un archivo de
configuración y las carpetas necesarias.

\subsection{Previsualización en el
Navegador}\label{previsualizaciuxf3n-en-el-navegador}

Una vez que hayas creado tu sitio, querrás verlo en acción. Para ello,
puedes iniciar un servidor local con el comando:

\begin{Shaded}
\begin{Highlighting}[]
\NormalTok{blogdown}\SpecialCharTok{::}\FunctionTok{serve\_site}\NormalTok{()}
\end{Highlighting}
\end{Shaded}

Este comando abrirá tu sitio en el navegador y te permitirá ver cómo
luce en tiempo real.

\section{Creando Publicaciones y Construyendo tu
Sitio}\label{creando-publicaciones-y-construyendo-tu-sitio}

\subsection{Añadiendo una Nueva
Publicación}\label{auxf1adiendo-una-nueva-publicaciuxf3n}

Una de las características más útiles de Blogdown es la posibilidad de
crear publicaciones de manera sencilla. Para agregar una nueva entrada a
tu blog, usa el siguiente comando:

\begin{Shaded}
\begin{Highlighting}[]
\NormalTok{blogdown}\SpecialCharTok{::}\FunctionTok{new\_post}\NormalTok{(}\AttributeTok{title =} \StringTok{"Mi Primera Publicación"}\NormalTok{)}
\end{Highlighting}
\end{Shaded}

Este comando creará un archivo de publicación en el directorio
\texttt{content/post} y lo abrirá para que puedas empezar a escribir.

\subsection{Generando el Sitio Web
Estático}\label{generando-el-sitio-web-estuxe1tico}

Cuando ya hayas agregado contenido y quieras generar el sitio web
estático, utiliza:

\begin{Shaded}
\begin{Highlighting}[]
\NormalTok{blogdown}\SpecialCharTok{::}\FunctionTok{build\_site}\NormalTok{()}
\end{Highlighting}
\end{Shaded}

Este comando compila todos los archivos necesarios y genera el sitio web
en el directorio \texttt{public}, listo para ser implementado.

\section{Verificando y Actualizando tu Sitio
Web}\label{verificando-y-actualizando-tu-sitio-web}

\subsection{Comprobando Errores en el
Sitio}\label{comprobando-errores-en-el-sitio}

Antes de proceder con la construcción de tu sitio, siempre es
recomendable verificar si hay errores. Utiliza:

\begin{Shaded}
\begin{Highlighting}[]
\NormalTok{blogdown}\SpecialCharTok{::}\FunctionTok{check\_site}\NormalTok{()}
\end{Highlighting}
\end{Shaded}

Este comando realiza una revisión completa del sitio y te avisa de
cualquier posible error que deba corregirse.

\subsection{Manteniendo Hugo
Actualizado}\label{manteniendo-hugo-actualizado}

Hugo es el motor detrás de la generación del sitio. Para asegurarte de
que siempre estás utilizando la versión más reciente, puedes actualizar
Hugo con:

\begin{Shaded}
\begin{Highlighting}[]
\NormalTok{blogdown}\SpecialCharTok{::}\FunctionTok{update\_hugo}\NormalTok{()}
\end{Highlighting}
\end{Shaded}

\subsection{Actualización de
Dependencias}\label{actualizaciuxf3n-de-dependencias}

Blogdown depende de varios paquetes y bibliotecas que ayudan a la
creación y gestión de tu sitio. Para mantenerlas al día, ejecuta:

\begin{Shaded}
\begin{Highlighting}[]
\NormalTok{blogdown}\SpecialCharTok{::}\FunctionTok{update\_dependencies}\NormalTok{()}
\end{Highlighting}
\end{Shaded}

\section{Personalización y Mejora de tu
Sitio}\label{personalizaciuxf3n-y-mejora-de-tu-sitio}

\subsection{Modificando el Tema del
Sitio}\label{modificando-el-tema-del-sitio}

Un aspecto clave de un sitio web es su apariencia, y con Blogdown,
puedes modificar el tema de manera sencilla. Si deseas editar el tema
actual de tu sitio, usa:

\begin{Shaded}
\begin{Highlighting}[]
\NormalTok{blogdown}\SpecialCharTok{::}\FunctionTok{edit\_theme}\NormalTok{()}
\end{Highlighting}
\end{Shaded}

Si no estás satisfecho con el tema, puedes instalar uno nuevo. Para
ello, usa:

\begin{Shaded}
\begin{Highlighting}[]
\NormalTok{blogdown}\SpecialCharTok{::}\FunctionTok{install\_theme}\NormalTok{(}\StringTok{"nombre\_del\_tema"}\NormalTok{)}
\end{Highlighting}
\end{Shaded}

\subsection{Comandos Personalizados con
Hugo}\label{comandos-personalizados-con-hugo}

Si necesitas ejecutar algún comando específico de Hugo, puedes hacerlo
fácilmente con:

\begin{Shaded}
\begin{Highlighting}[]
\NormalTok{blogdown}\SpecialCharTok{::}\FunctionTok{hugo\_cmd}\NormalTok{(}\StringTok{"comando\_personalizado"}\NormalTok{)}
\end{Highlighting}
\end{Shaded}

Este comando te permite ejecutar cualquier comando que Hugo soporte para
mejorar y personalizar tu sitio aún más.

\section{Desplegando tu Sitio en la
Web}\label{desplegando-tu-sitio-en-la-web}

\subsection{Implementando en Netlify}\label{implementando-en-netlify}

Una de las ventajas de Blogdown es su integración con servicios de
hosting como Netlify. Si utilizas Netlify para alojar tu sitio, este
comando actualizará la configuración:

\begin{Shaded}
\begin{Highlighting}[]
\NormalTok{blogdown}\SpecialCharTok{::}\FunctionTok{update\_netlify}\NormalTok{()}
\end{Highlighting}
\end{Shaded}

\section{Conclusión: Tu Camino Hacia el Éxito con
Blogdown}\label{conclusiuxf3n-tu-camino-hacia-el-uxe9xito-con-blogdown}

Blogdown es una herramienta versátil y poderosa para crear sitios web
estáticos usando R. Con unos pocos comandos y algo de creatividad,
puedes diseñar y personalizar tu sitio de manera eficiente. Ya sea que
estés creando un blog personal o un sitio profesional, Blogdown ofrece
todo lo necesario para ayudarte a construir tu presencia en línea de
forma rápida y sin complicaciones.

Recuerda que la clave del éxito en la creación de un sitio web está en
la simplicidad y la personalización. ¡No dudes en experimentar con los
diferentes comandos y herramientas que Blogdown te ofrece para crear un
sitio web que refleje tu estilo único!

\section{Publicaciones Similares}\label{publicaciones-similares}

Si te interesó este artículo, te recomendamos que explores otros blogs y
recursos relacionados que pueden ampliar tus conocimientos. Aquí te dejo
algunas sugerencias:

\begin{enumerate}
\def\labelenumi{\arabic{enumi}.}
\tightlist
\item
  \href{https://achalmaedison.netlify.app/blog/posts/2015-05-14-el-aborto/index.pdf}{\faIcon{file-pdf}}
  \href{https://achalmaedison.netlify.app/blog/posts/2015-05-14-el-aborto}{El
  Aborto}
\item
  \href{https://achalmaedison.netlify.app/blog/posts/2017-04-23-sitios-web-asombrosos/index.pdf}{\faIcon{file-pdf}}
  \href{https://achalmaedison.netlify.app/blog/posts/2017-04-23-sitios-web-asombrosos}{Sitios
  Web Asombrosos}
\item
  \href{https://achalmaedison.netlify.app/blog/posts/2017-05-23-el-mercantilismo/index.pdf}{\faIcon{file-pdf}}
  \href{https://achalmaedison.netlify.app/blog/posts/2017-05-23-el-mercantilismo}{El
  Mercantilismo}
\item
  \href{https://achalmaedison.netlify.app/blog/posts/2020-05-23-comandos-de-google-assistant/index.pdf}{\faIcon{file-pdf}}
  \href{https://achalmaedison.netlify.app/blog/posts/2020-05-23-comandos-de-google-assistant}{Comandos
  De Google Assistant}
\item
  \href{https://achalmaedison.netlify.app/blog/posts/2020-09-15-plan-de-negocio-exportacion-de-trucha-arcoires/index.pdf}{\faIcon{file-pdf}}
  \href{https://achalmaedison.netlify.app/blog/posts/2020-09-15-plan-de-negocio-exportacion-de-trucha-arcoires}{Plan
  De Negocio Exportacion De Trucha Arcoires}
\item
  \href{https://achalmaedison.netlify.app/blog/posts/2021-07-13-plan-de-negocio-exportacion-de-tuna/index.pdf}{\faIcon{file-pdf}}
  \href{https://achalmaedison.netlify.app/blog/posts/2021-07-13-plan-de-negocio-exportacion-de-tuna}{Plan
  De Negocio Exportacion De Tuna}
\item
  \href{https://achalmaedison.netlify.app/blog/posts/2021-07-14-comandos-de-blogdown/index.pdf}{\faIcon{file-pdf}}
  \href{https://achalmaedison.netlify.app/blog/posts/2021-07-14-comandos-de-blogdown}{Comandos
  De Blogdown}
\item
  \href{https://achalmaedison.netlify.app/blog/posts/2021-10-01-gestion-publica-y-administracion-publica/index.pdf}{\faIcon{file-pdf}}
  \href{https://achalmaedison.netlify.app/blog/posts/2021-10-01-gestion-publica-y-administracion-publica}{Gestion
  Publica Y Administracion Publica}
\item
  \href{https://achalmaedison.netlify.app/blog/posts/2021-10-01-reformas-y-modernizacion-de-la-gestion-publica/index.pdf}{\faIcon{file-pdf}}
  \href{https://achalmaedison.netlify.app/blog/posts/2021-10-01-reformas-y-modernizacion-de-la-gestion-publica}{Reformas
  Y Modernizacion De La Gestion Publica}
\item
  \href{https://achalmaedison.netlify.app/blog/posts/2022-01-23-cadena\%20de\%20suministros/index.pdf}{\faIcon{file-pdf}}
  \href{https://achalmaedison.netlify.app/blog/posts/2022-01-23-cadena\%20de\%20suministros}{Cadena
  De Suministros}
\item
  \href{https://achalmaedison.netlify.app/blog/posts/2022-04-22-economia-agraria/index.pdf}{\faIcon{file-pdf}}
  \href{https://achalmaedison.netlify.app/blog/posts/2022-04-22-economia-agraria}{Economia
  Agraria}
\item
  \href{https://achalmaedison.netlify.app/blog/posts/2022-06-02-impacto-del-cambio-climatico/index.pdf}{\faIcon{file-pdf}}
  \href{https://achalmaedison.netlify.app/blog/posts/2022-06-02-impacto-del-cambio-climatico}{Impacto
  Del Cambio Climatico}
\item
  \href{https://achalmaedison.netlify.app/blog/posts/2023-05-11-cualidades-de-los-servidores-publicos/index.pdf}{\faIcon{file-pdf}}
  \href{https://achalmaedison.netlify.app/blog/posts/2023-05-11-cualidades-de-los-servidores-publicos}{Cualidades
  De Los Servidores Publicos}
\item
  \href{https://achalmaedison.netlify.app/blog/posts/2023-05-12-la-economia-peruana-entre-1970-1990/index.pdf}{\faIcon{file-pdf}}
  \href{https://achalmaedison.netlify.app/blog/posts/2023-05-12-la-economia-peruana-entre-1970-1990}{La
  Economia Peruana Entre 1970 1990}
\item
  \href{https://achalmaedison.netlify.app/blog/posts/2023-05-16-economia-regional/index.pdf}{\faIcon{file-pdf}}
  \href{https://achalmaedison.netlify.app/blog/posts/2023-05-16-economia-regional}{Economia
  Regional}
\end{enumerate}

Esperamos que encuentres estas publicaciones igualmente interesantes y
útiles. ¡Disfruta de la lectura!






\end{document}
