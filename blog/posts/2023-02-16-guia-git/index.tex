% Options for packages loaded elsewhere
\PassOptionsToPackage{unicode}{hyperref}
\PassOptionsToPackage{hyphens}{url}
\PassOptionsToPackage{dvipsnames,svgnames,x11names}{xcolor}
%
\documentclass[
  a4paper,
]{article}

\usepackage{amsmath,amssymb}
\usepackage{iftex}
\ifPDFTeX
  \usepackage[T1]{fontenc}
  \usepackage[utf8]{inputenc}
  \usepackage{textcomp} % provide euro and other symbols
\else % if luatex or xetex
  \usepackage{unicode-math}
  \defaultfontfeatures{Scale=MatchLowercase}
  \defaultfontfeatures[\rmfamily]{Ligatures=TeX,Scale=1}
\fi
\usepackage{lmodern}
\ifPDFTeX\else  
    % xetex/luatex font selection
\fi
% Use upquote if available, for straight quotes in verbatim environments
\IfFileExists{upquote.sty}{\usepackage{upquote}}{}
\IfFileExists{microtype.sty}{% use microtype if available
  \usepackage[]{microtype}
  \UseMicrotypeSet[protrusion]{basicmath} % disable protrusion for tt fonts
}{}
\makeatletter
\@ifundefined{KOMAClassName}{% if non-KOMA class
  \IfFileExists{parskip.sty}{%
    \usepackage{parskip}
  }{% else
    \setlength{\parindent}{0pt}
    \setlength{\parskip}{6pt plus 2pt minus 1pt}}
}{% if KOMA class
  \KOMAoptions{parskip=half}}
\makeatother
\usepackage{xcolor}
\usepackage[top=2.54cm,right=2.54cm,bottom=2.54cm,left=2.54cm]{geometry}
\setlength{\emergencystretch}{3em} % prevent overfull lines
\setcounter{secnumdepth}{-\maxdimen} % remove section numbering
% Make \paragraph and \subparagraph free-standing
\makeatletter
\ifx\paragraph\undefined\else
  \let\oldparagraph\paragraph
  \renewcommand{\paragraph}{
    \@ifstar
      \xxxParagraphStar
      \xxxParagraphNoStar
  }
  \newcommand{\xxxParagraphStar}[1]{\oldparagraph*{#1}\mbox{}}
  \newcommand{\xxxParagraphNoStar}[1]{\oldparagraph{#1}\mbox{}}
\fi
\ifx\subparagraph\undefined\else
  \let\oldsubparagraph\subparagraph
  \renewcommand{\subparagraph}{
    \@ifstar
      \xxxSubParagraphStar
      \xxxSubParagraphNoStar
  }
  \newcommand{\xxxSubParagraphStar}[1]{\oldsubparagraph*{#1}\mbox{}}
  \newcommand{\xxxSubParagraphNoStar}[1]{\oldsubparagraph{#1}\mbox{}}
\fi
\makeatother

\usepackage{color}
\usepackage{fancyvrb}
\newcommand{\VerbBar}{|}
\newcommand{\VERB}{\Verb[commandchars=\\\{\}]}
\DefineVerbatimEnvironment{Highlighting}{Verbatim}{commandchars=\\\{\}}
% Add ',fontsize=\small' for more characters per line
\newenvironment{Shaded}{}{}
\newcommand{\AlertTok}[1]{\textcolor[rgb]{1.00,0.33,0.33}{\textbf{#1}}}
\newcommand{\AnnotationTok}[1]{\textcolor[rgb]{0.42,0.45,0.49}{#1}}
\newcommand{\AttributeTok}[1]{\textcolor[rgb]{0.84,0.23,0.29}{#1}}
\newcommand{\BaseNTok}[1]{\textcolor[rgb]{0.00,0.36,0.77}{#1}}
\newcommand{\BuiltInTok}[1]{\textcolor[rgb]{0.84,0.23,0.29}{#1}}
\newcommand{\CharTok}[1]{\textcolor[rgb]{0.01,0.18,0.38}{#1}}
\newcommand{\CommentTok}[1]{\textcolor[rgb]{0.42,0.45,0.49}{#1}}
\newcommand{\CommentVarTok}[1]{\textcolor[rgb]{0.42,0.45,0.49}{#1}}
\newcommand{\ConstantTok}[1]{\textcolor[rgb]{0.00,0.36,0.77}{#1}}
\newcommand{\ControlFlowTok}[1]{\textcolor[rgb]{0.84,0.23,0.29}{#1}}
\newcommand{\DataTypeTok}[1]{\textcolor[rgb]{0.84,0.23,0.29}{#1}}
\newcommand{\DecValTok}[1]{\textcolor[rgb]{0.00,0.36,0.77}{#1}}
\newcommand{\DocumentationTok}[1]{\textcolor[rgb]{0.42,0.45,0.49}{#1}}
\newcommand{\ErrorTok}[1]{\textcolor[rgb]{1.00,0.33,0.33}{\underline{#1}}}
\newcommand{\ExtensionTok}[1]{\textcolor[rgb]{0.84,0.23,0.29}{\textbf{#1}}}
\newcommand{\FloatTok}[1]{\textcolor[rgb]{0.00,0.36,0.77}{#1}}
\newcommand{\FunctionTok}[1]{\textcolor[rgb]{0.44,0.26,0.76}{#1}}
\newcommand{\ImportTok}[1]{\textcolor[rgb]{0.01,0.18,0.38}{#1}}
\newcommand{\InformationTok}[1]{\textcolor[rgb]{0.42,0.45,0.49}{#1}}
\newcommand{\KeywordTok}[1]{\textcolor[rgb]{0.84,0.23,0.29}{#1}}
\newcommand{\NormalTok}[1]{\textcolor[rgb]{0.14,0.16,0.18}{#1}}
\newcommand{\OperatorTok}[1]{\textcolor[rgb]{0.14,0.16,0.18}{#1}}
\newcommand{\OtherTok}[1]{\textcolor[rgb]{0.44,0.26,0.76}{#1}}
\newcommand{\PreprocessorTok}[1]{\textcolor[rgb]{0.84,0.23,0.29}{#1}}
\newcommand{\RegionMarkerTok}[1]{\textcolor[rgb]{0.42,0.45,0.49}{#1}}
\newcommand{\SpecialCharTok}[1]{\textcolor[rgb]{0.00,0.36,0.77}{#1}}
\newcommand{\SpecialStringTok}[1]{\textcolor[rgb]{0.01,0.18,0.38}{#1}}
\newcommand{\StringTok}[1]{\textcolor[rgb]{0.01,0.18,0.38}{#1}}
\newcommand{\VariableTok}[1]{\textcolor[rgb]{0.89,0.38,0.04}{#1}}
\newcommand{\VerbatimStringTok}[1]{\textcolor[rgb]{0.01,0.18,0.38}{#1}}
\newcommand{\WarningTok}[1]{\textcolor[rgb]{1.00,0.33,0.33}{#1}}

\providecommand{\tightlist}{%
  \setlength{\itemsep}{0pt}\setlength{\parskip}{0pt}}\usepackage{longtable,booktabs,array}
\usepackage{calc} % for calculating minipage widths
% Correct order of tables after \paragraph or \subparagraph
\usepackage{etoolbox}
\makeatletter
\patchcmd\longtable{\par}{\if@noskipsec\mbox{}\fi\par}{}{}
\makeatother
% Allow footnotes in longtable head/foot
\IfFileExists{footnotehyper.sty}{\usepackage{footnotehyper}}{\usepackage{footnote}}
\makesavenoteenv{longtable}
\usepackage{graphicx}
\makeatletter
\def\maxwidth{\ifdim\Gin@nat@width>\linewidth\linewidth\else\Gin@nat@width\fi}
\def\maxheight{\ifdim\Gin@nat@height>\textheight\textheight\else\Gin@nat@height\fi}
\makeatother
% Scale images if necessary, so that they will not overflow the page
% margins by default, and it is still possible to overwrite the defaults
% using explicit options in \includegraphics[width, height, ...]{}
\setkeys{Gin}{width=\maxwidth,height=\maxheight,keepaspectratio}
% Set default figure placement to htbp
\makeatletter
\def\fps@figure{htbp}
\makeatother

% Preámbulo
\usepackage{comment} % Permite comentar secciones del código
\usepackage{marvosym} % Agrega símbolos adicionales
\usepackage{graphicx} % Permite insertar imágenes
\usepackage{mathptmx} % Fuente de texto matemática
\usepackage{amssymb} % Símbolos adicionales de matemáticas
\usepackage{lipsum} % Crea texto aleatorio
\usepackage{amsthm} % Teoremas y entornos de demostración
\usepackage{float} % Control de posiciones de figuras y tablas
\usepackage{rotating} % Rotación de elementos
\usepackage{multirow} % Celdas combinadas en tablas
\usepackage{tabularx} % Tablas con ancho de columna ajustable
\usepackage{mdframed} % Marcos alrededor de elementos flotantes

% Series de tiempo
\usepackage{booktabs}


% Configuración adicional

\makeatletter
\@ifpackageloaded{caption}{}{\usepackage{caption}}
\AtBeginDocument{%
\ifdefined\contentsname
  \renewcommand*\contentsname{Tabla de contenidos}
\else
  \newcommand\contentsname{Tabla de contenidos}
\fi
\ifdefined\listfigurename
  \renewcommand*\listfigurename{Listado de Figuras}
\else
  \newcommand\listfigurename{Listado de Figuras}
\fi
\ifdefined\listtablename
  \renewcommand*\listtablename{Listado de Tablas}
\else
  \newcommand\listtablename{Listado de Tablas}
\fi
\ifdefined\figurename
  \renewcommand*\figurename{Figura}
\else
  \newcommand\figurename{Figura}
\fi
\ifdefined\tablename
  \renewcommand*\tablename{Tabla}
\else
  \newcommand\tablename{Tabla}
\fi
}
\@ifpackageloaded{float}{}{\usepackage{float}}
\floatstyle{ruled}
\@ifundefined{c@chapter}{\newfloat{codelisting}{h}{lop}}{\newfloat{codelisting}{h}{lop}[chapter]}
\floatname{codelisting}{Listado}
\newcommand*\listoflistings{\listof{codelisting}{Listado de Listados}}
\makeatother
\makeatletter
\makeatother
\makeatletter
\@ifpackageloaded{caption}{}{\usepackage{caption}}
\@ifpackageloaded{subcaption}{}{\usepackage{subcaption}}
\makeatother
\ifLuaTeX
\usepackage[bidi=basic]{babel}
\else
\usepackage[bidi=default]{babel}
\fi
\babelprovide[main,import]{spanish}
% get rid of language-specific shorthands (see #6817):
\let\LanguageShortHands\languageshorthands
\def\languageshorthands#1{}
\ifLuaTeX
  \usepackage{selnolig}  % disable illegal ligatures
\fi
\usepackage[]{biblatex}
\addbibresource{../../../references.bib}
\usepackage{bookmark}

\IfFileExists{xurl.sty}{\usepackage{xurl}}{} % add URL line breaks if available
\urlstyle{same} % disable monospaced font for URLs
\hypersetup{
  pdftitle={Guía de Git Cómo trabajar en equipo en proyectos},
  pdfauthor={Edison Achalma},
  pdflang={es},
  colorlinks=true,
  linkcolor={blue},
  filecolor={Maroon},
  citecolor={Blue},
  urlcolor={Blue},
  pdfcreator={LaTeX via pandoc}}

\title{Guía de Git Cómo trabajar en equipo en proyectos}
\usepackage{etoolbox}
\makeatletter
\providecommand{\subtitle}[1]{% add subtitle to \maketitle
  \apptocmd{\@title}{\par {\large #1 \par}}{}{}
}
\makeatother
\subtitle{Aprende a usar Git para controlar versiones, colaborar con
otros desarrolladores y mantener tu código organizado.}
\author{Edison Achalma}
\date{2023-02-16}

\begin{document}
\maketitle

\section{Introducción}\label{introducciuxf3n}

\textbf{¿Estás interesado en aprender los fundamentos de Git y GitHub?}
¡Has llegado al lugar perfecto! En este blog, te presentaremos una guía
completa que podrás utilizar como referencia diaria.

Tanto si estás comenzando tu viaje en el control de versiones como si
deseas mejorar tus habilidades en Git y aprovechar al máximo GitHub,
encontrarás aquí una guía clara y concisa para dar tus primeros pasos.

Los sistemas de control de versiones, como Git, son imprescindibles en
las prácticas. Estos sistemas te permiten realizar un seguimiento de los
cambios en tu código fuente, revertir a versiones anteriores y crear
ramas para experimentar con nuevas ideas o funcionalidades.

Hoy en día, los repositorios de Git albergan muchos proyectos de
software, y plataformas como GitHub, GitLab y Bitbucket facilitan la
colaboración y el intercambio de código entre desarrolladores.

En esta guía, te mostraremos cómo instalar y configurar Git en GNU Linux
(Ubuntu). Exploraremos dos métodos diferentes de instalación, cada uno
con sus propios beneficios, para que puedas elegir el que mejor se
adapte a tus necesidades específicas.

\begin{quote}
Recuerda que es esencial tener un buen dominio de Git y GitHub para
colaborar eficientemente en proyectos y aprovechar todas sus
funcionalidades. ¡Vamos a sumergirnos en el fascinante mundo de Git y
GitHub!
\end{quote}

\section{Entendiendo cómo funciona
Git}\label{entendiendo-cuxf3mo-funciona-git}

Git es el sistema de control de versiones (SCV) de código abierto más
utilizado, diseñado para rastrear los cambios realizados en los
archivos. Tanto empresas como programadores confían en Git para
colaborar en el desarrollo de software y aplicaciones.

Un proyecto en Git se compone de tres elementos principales: \textbf{el
directorio de trabajo, el área de preparación y el directorio Git.}

El directorio de trabajo es donde agregas, borras y editas tus archivos.
Luego, los cambios se preparan (indexan) en el área de preparación. Una
vez que confirmas tus cambios, se guarda una instantánea de los mismos
en el directorio Git.

Git está disponible para múltiples plataformas, incluyendo Linux,
Windows y Mac. Esto hace que Git sea accesible para todos. Aunque el
software puede tener una curva de aprendizaje pronunciada, existen
numerosos tutoriales disponibles para ayudarte a dominarlo.

Así que, independientemente de tu sistema operativo, puedes aprovechar
los beneficios de Git y aprovechar su poderoso conjunto de
características. Con el tiempo y la práctica, te convertirás en un
experto en el uso de Git para gestionar tus proyectos de manera
eficiente y colaborar de forma efectiva con otros desarrolladores.

\section{Comandos básicos de Git}\label{comandos-buxe1sicos-de-git}

Aquí te presentamos algunos comandos básicos de Git que es importante
que conozcas:

\begin{enumerate}
\def\labelenumi{\arabic{enumi}.}
\tightlist
\item
  \texttt{git\ init}: Inicializa un nuevo repositorio de Git en la
  carpeta actual.
\item
  \texttt{git\ clone\ {[}url{]}}: Clona un repositorio existente en la
  carpeta actual.
\item
  \texttt{git\ add\ {[}file{]}}: Agrega un archivo al área de
  preparación (stage) para ser incluido en el próximo commit.
\item
  \texttt{git\ commit\ -m\ "{[}message{]}"}: Realiza un commit (guarda
  un punto de referencia) con un mensaje que describe los cambios
  realizados.
\item
  \texttt{git\ status}: Muestra el estado actual del repositorio,
  incluyendo los archivos modificados y los que están pendientes de
  commit.
\item
  \texttt{git\ log}: Muestra un historial de todos los commits
  realizados en el repositorio.
\item
  \texttt{git\ diff}: Muestra las diferencias entre los cambios
  realizados y el último commit.
\item
  \texttt{git\ branch}: Muestra una lista de todas las ramas existentes
  en el repositorio.
\item
  \texttt{git\ checkout\ {[}branch{]}}: Cambia a una rama específica.
\item
  \texttt{git\ merge\ {[}branch{]}}: Combina los cambios de una rama
  específica con la rama actual.
\item
  \texttt{git\ config\ -\/-global\ user.email\ "tu-email@example.com"}
\item
  \texttt{git\ config\ -\/-global\ user.name\ "tu-usuario-GitHub"}
\end{enumerate}

Estos son solo algunos de los comandos básicos de Git. También existen
muchos otros comandos avanzados disponibles para realizar tareas más
complejas, como trabajar con ramas remotas y manejar conflictos.

\section{Instalación y Configuración de
Git}\label{instalaciuxf3n-y-configuraciuxf3n-de-git}

\textbf{Git} es una herramienta fundamental para el control de versiones
y colaboración en proyectos de desarrollo de software. Aquí te
explicaremos cómo instalar y configurar Git en Ubuntu.

\subsection{Instalación de Git con paquetes
predeterminados}\label{instalaciuxf3n-de-git-con-paquetes-predeterminados}

Si deseas una instalación rápida y estable de Git, puedes utilizar los
paquetes predeterminados. Si buscas la versión más reciente o necesitas
funciones específicas, puedes optar por la instalación desde la fuente.

En primer lugar, verifica si Git ya está instalado en tu Ubuntu
ejecutando el siguiente comando en la terminal:

\begin{Shaded}
\begin{Highlighting}[]
\FunctionTok{git} \AttributeTok{{-}{-}version}
\end{Highlighting}
\end{Shaded}

Si obtienes una salida similar a la siguiente, significa que Git ya está
instalado:

\begin{Shaded}
\begin{Highlighting}[]
\FunctionTok{git}\NormalTok{ version 2.34.1}
\end{Highlighting}
\end{Shaded}

Si Git no está instalado, puedes utilizar el administrador de paquetes
APT de Ubuntu para instalarlo:

\begin{enumerate}
\def\labelenumi{\arabic{enumi}.}
\item
  Abre una terminal.
\item
  Actualiza la lista de paquetes disponibles con el siguiente comando:

\begin{Shaded}
\begin{Highlighting}[]
\FunctionTok{sudo}\NormalTok{ apt update}
\end{Highlighting}
\end{Shaded}
\item
  Instala Git utilizando el siguiente comando:

\begin{Shaded}
\begin{Highlighting}[]
\FunctionTok{sudo}\NormalTok{ apt install git}
\end{Highlighting}
\end{Shaded}
\item
  Verifica la instalación ejecutando el comando

\begin{Shaded}
\begin{Highlighting}[]
\FunctionTok{git} \AttributeTok{{-}{-}version}
\end{Highlighting}
\end{Shaded}
\end{enumerate}

Una vez que Git esté instalado, es recomendable configurarlo según tus
necesidades. Sigue estos pasos para configurar Git en Ubuntu:

\begin{enumerate}
\def\labelenumi{\arabic{enumi}.}
\item
  Establece tu nombre de usuario en Git con el comando:

\begin{Shaded}
\begin{Highlighting}[]
\FunctionTok{git}\NormalTok{ config }\AttributeTok{{-}{-}global}\NormalTok{ user.name }\StringTok{"Tu Nombre"}
\end{Highlighting}
\end{Shaded}
\item
  Establece tu dirección de correo electrónico en Git con el comando:

\begin{Shaded}
\begin{Highlighting}[]
\FunctionTok{git}\NormalTok{ config }\AttributeTok{{-}{-}global}\NormalTok{ user.email }\StringTok{"tu.correo@example.com"}
\end{Highlighting}
\end{Shaded}
\item
  Para verificar la configuración actual de Git, utiliza el comando:

\begin{Shaded}
\begin{Highlighting}[]
\FunctionTok{git}\NormalTok{ config }\AttributeTok{{-}{-}list}
\end{Highlighting}
\end{Shaded}
\end{enumerate}

También es recomendable configurar un editor de texto para escribir los
mensajes de commit. Puedes hacerlo con el siguiente comando,
reemplazando \textbf{``nano''} o \textbf{``vim''} por el editor de texto
de tu preferencia:

\begin{Shaded}
\begin{Highlighting}[]
\FunctionTok{git}\NormalTok{ config }\AttributeTok{{-}{-}global}\NormalTok{ core.editor }\StringTok{"nano"}
\end{Highlighting}
\end{Shaded}

Con estos pasos, habrás instalado y configurado Git en tu sistema
Ubuntu. Ahora estás listo para aprovechar todas las ventajas que ofrece
esta poderosa herramienta de control de versiones en tus proyectos de
desarrollo.

\subsection{Instalación de Git desde la
fuente}\label{instalaciuxf3n-de-git-desde-la-fuente}

Si estás buscando una forma más flexible de instalar Git y quieres tener
la versión más reciente, puedes compilar el software desde la fuente.
Aunque este método requiere más tiempo y no se integrará con el
administrador de paquetes, te permitirá personalizar las opciones de
instalación.

Antes de comenzar, verifica la versión actualmente instalada de Git
ejecutando el siguiente comando: \texttt{git\ -\/-version}. Si Git ya
está instalado, obtendrás un resultado similar a este:
\texttt{git\ version\ 2.34.1}.

Asegúrate de tener instalado el software necesario para compilar Git.
Puedes hacerlo actualizando el índice de paquetes locales y luego
instalando las dependencias relevantes. Ejecuta los siguientes comandos:

\begin{Shaded}
\begin{Highlighting}[]
\FunctionTok{sudo}\NormalTok{ apt update}
\FunctionTok{sudo}\NormalTok{ apt install libz{-}dev libssl{-}dev libcurl4{-}gnutls{-}dev libexpat1{-}dev gettext cmake gcc}
\end{Highlighting}
\end{Shaded}

Una vez instaladas las dependencias, crea un directorio temporal y
accede a él. Aquí es donde descargarás el archivo tarball de Git.
Ejecuta los siguientes comandos:

\begin{Shaded}
\begin{Highlighting}[]
\FunctionTok{mkdir}\NormalTok{ tmp}
\BuiltInTok{cd}\NormalTok{ /tmp}
\end{Highlighting}
\end{Shaded}

Desde el sitio web oficial de Git, navega hasta la lista de tarballs
disponibles en
\texttt{https://mirrors.edge.kernel.org/pub/software/scm/git/} y
descarga la versión que desees utilizar. Por ejemplo, si quieres
descargar la versión 2.34.1, puedes ejecutar el siguiente comando:

\begin{Shaded}
\begin{Highlighting}[]
\ExtensionTok{curl} \AttributeTok{{-}o}\NormalTok{ git.tar.gz https://mirrors.edge.kernel.org/pub/software/scm/git/git{-}2.34.1.tar.gz}
\end{Highlighting}
\end{Shaded}

Descomprime el archivo tarball ejecutando el siguiente comando:

\begin{Shaded}
\begin{Highlighting}[]
\FunctionTok{tar} \AttributeTok{{-}zxf}\NormalTok{ git.tar.gz}
\end{Highlighting}
\end{Shaded}

A continuación, accede al nuevo directorio de Git con el siguiente
comando:

\begin{Shaded}
\begin{Highlighting}[]
\BuiltInTok{cd}\NormalTok{ git{-}}\PreprocessorTok{*}
\end{Highlighting}
\end{Shaded}

Ahora, puedes crear el paquete e instalarlo ejecutando los siguientes
comandos:

\begin{Shaded}
\begin{Highlighting}[]
\FunctionTok{make}\NormalTok{ prefix=/usr/local all}
\FunctionTok{sudo}\NormalTok{ make prefix=/usr/local install}
\end{Highlighting}
\end{Shaded}

Una vez completado el proceso, sustituye la shell actual para utilizar
la versión de Git recién instalada ejecutando el siguiente comando:

\begin{Shaded}
\begin{Highlighting}[]
\BuiltInTok{exec}\NormalTok{ bash}
\end{Highlighting}
\end{Shaded}

Para verificar que la instalación se haya realizado correctamente,
comprueba la versión de Git nuevamente ejecutando el comando
\texttt{git\ -\/-version}. Deberías obtener un resultado similar a este:
\texttt{git\ version\ 2.34.1}.

\begin{quote}
¡Con Git instalado correctamente, ahora puedes continuar con la
configuración y aprovechar todas las funcionalidades que ofrece esta
poderosa herramienta de control de versiones en tus proyectos!
\end{quote}

\subsection{Configuración de Git}\label{configuraciuxf3n-de-git}

Una vez que hayas elegido la versión de Git con la que estás satisfecho,
es importante configurar Git para que los mensajes de confirmación que
generes contengan la información correcta y te respalden a medida que
desarrollas tu proyecto de software.

La configuración de Git se realiza a través del comando
\texttt{git\ config}. Específicamente, debemos proporcionar nuestro
nombre y dirección de correo electrónico, ya que Git inserta esta
información en cada confirmación que realizamos. Podemos agregar esta
información ejecutando los siguientes comandos:

\begin{Shaded}
\begin{Highlighting}[]
\FunctionTok{git}\NormalTok{ config }\AttributeTok{{-}{-}global}\NormalTok{ user.name }\StringTok{"Your Name"}
\FunctionTok{git}\NormalTok{ config }\AttributeTok{{-}{-}global}\NormalTok{ user.email }\StringTok{"your.email@example.com"}
\end{Highlighting}
\end{Shaded}

Para verificar los elementos de configuración que hemos creado, podemos
ejecutar el siguiente comando:

\begin{Shaded}
\begin{Highlighting}[]
\FunctionTok{git}\NormalTok{ config }\AttributeTok{{-}{-}list}
\end{Highlighting}
\end{Shaded}

La información que ingreses se almacenará en el archivo de configuración
de Git. Si deseas editarlo manualmente con el editor de texto de tu
elección (en este tutorial utilizaremos nano), puedes ejecutar el
siguiente comando:

\begin{Shaded}
\begin{Highlighting}[]
\FunctionTok{nano}\NormalTok{ \textasciitilde{}/.gitconfig}
\end{Highlighting}
\end{Shaded}

En el archivo \texttt{\textasciitilde{}/.gitconfig}, encontrarás los
siguientes contenidos:

\begin{Shaded}
\begin{Highlighting}[]
\ExtensionTok{[user]}
    \ExtensionTok{name}\NormalTok{ = Your Name}
    \ExtensionTok{email}\NormalTok{ = your.email@example.com}
\end{Highlighting}
\end{Shaded}

Para salir del editor de texto, presiona CTRL + X, luego Y y finalmente
ENTER.

Existen muchas otras opciones de configuración que puedes ajustar, pero
estas dos son esenciales. Si omites este paso, es probable que veas
mensajes de advertencia al realizar una confirmación con Git. Esto
implica más trabajo para ti, ya que tendrás que revisar las
confirmaciones anteriores y corregir la información.

Al configurar Git correctamente, aseguras que tus confirmaciones tengan
la información adecuada y facilitas el seguimiento y control de los
cambios en tu proyecto de software.

\section{Cómo Obtener y Configurar tus Claves SSH para Git y
GitHub}\label{cuxf3mo-obtener-y-configurar-tus-claves-ssh-para-git-y-github}

Si estás utilizando GitHub sin configurar una clave SSH, te estás
perdiendo de una gran comodidad. Piensa en todo el tiempo que has
gastado ingresando tu correo electrónico y contraseña en la consola cada
vez que haces un commit, podrías haberlo utilizado para programar.

Pero ya no más. Aquí tienes una guía rápida para generar y configurar
una clave SSH con GitHub, para que nunca más tengas que autenticarte de
forma convencional.

\subsection{Verificar la existencia de una clave
SSH}\label{verificar-la-existencia-de-una-clave-ssh}

En primer lugar, verifica si ya has generado claves SSH para tu máquina.
Abre una terminal y ejecuta el siguiente comando:

\begin{Shaded}
\begin{Highlighting}[]
\FunctionTok{ls} \AttributeTok{{-}al}\NormalTok{ \textasciitilde{}/.ssh}
\end{Highlighting}
\end{Shaded}

Si ya has generado las claves SSH, deberías ver una salida similar a
esta:

\begin{Shaded}
\begin{Highlighting}[]
\ExtensionTok{{-}rw{-}{-}{-}{-}{-}{-}{-}}\NormalTok{  1 usuario usuario  1766 Jul  7  2018 id\_rsa}
\ExtensionTok{{-}rw{-}r{-}{-}r{-}{-}}\NormalTok{  1 usuario usuario   414 Jul  7  2018 id\_rsa.pub}
\ExtensionTok{{-}rw{-}{-}{-}{-}{-}{-}{-}}\NormalTok{  1 usuario usuario 12892 Feb  5 18:39 known\_hosts}
\end{Highlighting}
\end{Shaded}

Si tus claves ya existen, continúa con la sección \textbf{Copia tu clave
pública de SSH} a continuación.

Si no ves ninguna salida o si el directorio no existe (recibes un
mensaje de ``No such file or directory''), entonces ejecuta el siguiente
comando:

\begin{Shaded}
\begin{Highlighting}[]
\FunctionTok{mkdir} \VariableTok{$HOME}\NormalTok{/.ssh}
\end{Highlighting}
\end{Shaded}

A continuación, genera un nuevo par de claves con el siguiente comando:

\begin{Shaded}
\begin{Highlighting}[]
\FunctionTok{ssh{-}keygen} \AttributeTok{{-}t}\NormalTok{ rsa }\AttributeTok{{-}b}\NormalTok{ 4096 }\AttributeTok{{-}C}\NormalTok{ achalma\_pinguino@gmail.com}
\end{Highlighting}
\end{Shaded}

Presiona Enter para crear el archivo con el nombre predeterminado y
luego ingresa una clave para proteger el archivo (por ejemplo,
``pepito69'').

Ahora verifica que tus claves existan con el comando
\texttt{ls\ -al\ \textasciitilde{}/.ssh} y asegúrate de que la salida
sea similar a la mencionada anteriormente.

\textbf{Nota:} Las claves SSH siempre se generan como un par de claves
públicas (id\_rsa.pub) y privadas (id\_rsa). Es extremadamente
importante que \textbf{nunca reveles tu clave privada} y que
\textbf{solo uses tu clave pública} para autenticarte en servicios como
GitHub.

\subsection{Agrega tu clave SSH a
ssh-agent}\label{agrega-tu-clave-ssh-a-ssh-agent}

\textbf{ssh-agent} es un programa que se inicia cuando te conectas y
almacena tus claves privadas. Para que funcione correctamente, debe
estar en ejecución y tener una copia de tu clave privada.

Primero, asegúrate de que \textbf{ssh-agent} se está ejecutando
ejecutando el siguiente comando:

\begin{Shaded}
\begin{Highlighting}[]
\BuiltInTok{eval} \StringTok{"}\VariableTok{$(}\FunctionTok{ssh{-}agent} \AttributeTok{{-}s}\VariableTok{)}\StringTok{"} \CommentTok{\# para Mac y Linux}
\end{Highlighting}
\end{Shaded}

o:

\begin{Shaded}
\begin{Highlighting}[]
\BuiltInTok{eval}\NormalTok{ ssh{-}agent }\AttributeTok{{-}s}
\end{Highlighting}
\end{Shaded}

\begin{Shaded}
\begin{Highlighting}[]
\FunctionTok{ssh{-}agent} \AttributeTok{{-}s} \CommentTok{\# para Windows}
\end{Highlighting}
\end{Shaded}

A continuación, agrega tu clave privada a \textbf{ssh-agent} con el
siguiente comando:

\begin{Shaded}
\begin{Highlighting}[]
\FunctionTok{ssh{-}add}\NormalTok{ \textasciitilde{}/.ssh/id\_rsa}
\end{Highlighting}
\end{Shaded}

\subsection{Copia tu clave pública de
SSH}\label{copia-tu-clave-puxfablica-de-ssh}

Después, necesitarás copiar tu clave pública de SSH al portapapeles.

Para Linux o Mac, puedes imprimir el contenido de tu clave pública en la
consola con el siguiente comando:

\begin{Shaded}
\begin{Highlighting}[]
\FunctionTok{cat}\NormalTok{ \textasciitilde{}/.ssh/id\_rsa.pub }\CommentTok{\# Linux}
\end{Highlighting}
\end{Shaded}

Aparecerá una cadena de números y letras. Si al final está el correo que
registraste antes, debes borrarlo antes de pegarlo en GitHub.

Luego, resalta y copia el resultado.

Para Windows, simplemente ejecuta el siguiente comando:

\begin{Shaded}
\begin{Highlighting}[]
\ExtensionTok{clip} \OperatorTok{\textless{}}\NormalTok{ \textasciitilde{}/.ssh/id\_rsa.pub }\CommentTok{\# Windows}
\end{Highlighting}
\end{Shaded}

\subsection{Agrega tu clave SSH pública a
GitHub}\label{agrega-tu-clave-ssh-puxfablica-a-github}

Accede a la página de \textbf{configuración} de tu cuenta de GitHub
\href{https://github.com/settings/keys}{aquí} y haz clic en el botón
``New SSH key''.

A continuación, asigna un título descriptivo a tu clave y pégala en el
campo correspondiente a tu clave pública (id\_rsa.pub).

Por último, prueba la autenticación con el siguiente comando:

\begin{Shaded}
\begin{Highlighting}[]
\FunctionTok{ssh} \AttributeTok{{-}T}\NormalTok{ git@github.com}
\end{Highlighting}
\end{Shaded}

Si has seguido correctamente todos estos pasos, deberías ver el
siguiente mensaje:

\begin{Shaded}
\begin{Highlighting}[]
\ExtensionTok{Hi}\NormalTok{ tu\_usuario! You}\StringTok{\textquotesingle{}ve successfully authenticated, but GitHub does not provide shell access.}
\end{Highlighting}
\end{Shaded}

o

\begin{Shaded}
\begin{Highlighting}[]
\ExtensionTok{Warning:}\NormalTok{ Permanently added the ECDSA host key for IP address }\StringTok{\textquotesingle{}140.82.114.3\textquotesingle{}}\NormalTok{ to the list of known hosts.}
\ExtensionTok{Hi}\NormalTok{ achalmed! You}\StringTok{\textquotesingle{}ve successfully authenticated, but GitHub does not provide shell access.}
\end{Highlighting}
\end{Shaded}

Para obtener más información sobre SSH, puedes consultar la
\href{https://cli.github.com/}{documentación}.

\subsection{CLI de GitHub}\label{cli-de-github}

En este tutorial, te mostraré cómo instalar y configurar la CLI de
GitHub en Linux para facilitar tus operaciones en GitHub directamente
desde la terminal.

\textbf{Paso 1: Instalación de la CLI de GitHub en Linux}

Para comenzar, debemos instalar la CLI de GitHub en Linux. Ejecuta los
siguientes comandos en tu terminal:

\begin{Shaded}
\begin{Highlighting}[]
\FunctionTok{sudo}\NormalTok{ apt install gh }\CommentTok{\# versión 2.12.1+dfsg1{-}1}
\end{Highlighting}
\end{Shaded}

\textbf{Paso 2: Autenticación con GitHub}

Una vez que hayas instalado la CLI de GitHub, puedes autenticarte con tu
cuenta de GitHub. Ejecuta el siguiente comando:

\begin{Shaded}
\begin{Highlighting}[]
\ExtensionTok{gh}\NormalTok{ auth login}
\end{Highlighting}
\end{Shaded}

Aparecerán varias opciones. A continuación, elige las siguientes
opciones:

\begin{itemize}
\tightlist
\item
  ¿En qué cuenta quieres iniciar sesión? GitHub.com
\item
  ¿Cuál es tu protocolo preferido para las operaciones de Git? SSH
\item
  ¿Deseas generar una nueva clave SSH para agregar a tu cuenta de
  GitHub? Sí
\item
  Ingresa una frase de contraseña para tu nueva clave SSH (opcional)
  ****** (usa tu propia frase de contraseña)
\item
  Título para tu clave SSH: achalmagit
\item
  ¿Cómo te gustaría autenticar la CLI de GitHub? Iniciar sesión con un
  navegador web
\end{itemize}

Luego, copia el código proporcionado y pégalo en tu navegador para
completar la conexión con GitHub.

\begin{quote}
¡Listo! Ahora estás autenticado con éxito y puedes usar la CLI de GitHub
para trabajar con tus repositorios en la terminal.
\end{quote}

\textbf{Paso 3: Utilizar la CLI de GitHub}

Una vez que hayas completado la autenticación, puedes comenzar a
utilizar la CLI de GitHub en la terminal. Simplemente ingresa el código
de contraseña (******) cuando se te solicite para realizar operaciones
en tus repositorios de GitHub.

Para obtener más información sobre la CLI de GitHub y todas sus
capacidades, puedes consultar la
\href{https://cli.github.com/}{documentación oficial}.

\begin{quote}
¡Disfruta utilizando la CLI de GitHub para simplificar tus interacciones
con GitHub desde la terminal!
\end{quote}

\section{Crear un Repositorio}\label{crear-un-repositorio}

En este tutorial, te mostraré cómo crear un nuevo repositorio GIT local
utilizando el comando \texttt{git\ init}. Este proceso te permitirá
iniciar un repositorio para tu proyecto y realizar un seguimiento de los
cambios a lo largo del tiempo.

\textbf{Paso 1: Inicializar un nuevo repositorio}

Para iniciar un nuevo repositorio GIT, simplemente ejecuta el siguiente
comando en tu terminal:

\begin{Shaded}
\begin{Highlighting}[]
\FunctionTok{git}\NormalTok{ init}
\end{Highlighting}
\end{Shaded}

Esto creará un nuevo repositorio GIT vacío en tu directorio actual.

Si deseas especificar un nombre para tu proyecto al crear el
repositorio, puedes utilizar el siguiente comando:

\begin{Shaded}
\begin{Highlighting}[]
\FunctionTok{git}\NormalTok{ init [nombre del proyecto]}
\end{Highlighting}
\end{Shaded}

\begin{quote}
Recuerda que este paso solo se realiza una vez al inicio del proyecto.
\end{quote}

\textbf{Comandos en Visual Studio Code}

En Visual Studio Code, puedes interactuar con GIT directamente desde la
interfaz de usuario. Aquí hay algunos comandos clave que puedes
utilizar:

\begin{enumerate}
\def\labelenumi{\arabic{enumi}.}
\tightlist
\item
  Abre el terminal integrado en Visual Studio Code desde la carpeta de
  tu proyecto.
\item
  Ejecuta \texttt{git\ init} para inicializar el repositorio.
\item
  Utiliza \texttt{git\ add} seguido del nombre de archivo para agregar
  archivos individuales al repositorio \texttt{git\ add\ index.html}.
\item
  Si deseas agregar todos los archivos modificados y no rastreados,
  puedes ejecutar \texttt{git\ add\ .}.
\item
  Haz un commit de los cambios utilizando el comando
  \texttt{git\ commit\ -m\ "mensaje\ del\ commit"}.
\item
  Si aún no tienes una rama principal (main), puedes crearla con el
  comando \texttt{git\ branch\ -M\ main}.
\item
  Para vincular tu repositorio local a un repositorio remoto en GitHub,
  utiliza el comando
  \texttt{git\ remote\ add\ origin\ {[}URL\ del\ repositorio{]}}.
\item
  Finalmente, puedes enviar tus cambios al repositorio remoto utilizando
  \texttt{git\ push\ -u\ origin\ main}. Asegúrate de ingresar tu nombre
  de usuario y contraseña correctamente.
\end{enumerate}

También puedes importar código desde otro repositorio o iniciar un
repositorio con código de proyectos Subversion, Mercurial o TFS.

Recuerda que es importante configurar GIT previamente y configurar la
autenticación utilizando claves SSH o GitHub CLI, según tus
preferencias.

\begin{quote}
¡Ahora estás listo para crear y gestionar repositorios con GIT en Visual
Studio Code!
\end{quote}

\subsection{Clonar un Repositorio con
Git}\label{clonar-un-repositorio-con-git}

\begin{quote}
El comando \texttt{git\ clone} se utiliza para copiar un repositorio, ya
sea desde un servidor remoto o desde una ubicación local. Si el
repositorio está en un servidor remoto, se utiliza la siguiente
sintaxis:
\texttt{git\ clone\ nombredeusuario@host:/ruta/al/repositorio}. En
cambio, si el repositorio se encuentra en una ubicación local, se
utiliza: \texttt{git\ clone\ /ruta/al/repositorio}.
\end{quote}

\subsubsection{¿Qué es git clone?}\label{quuxe9-es-git-clone}

El comando \texttt{git\ clone} se utiliza para apuntar a un repositorio
existente y hacer una copia del mismo en otra ubicación. Este comando
creará un nuevo directorio, lo configurará para utilizar Git y copiará
los archivos del repositorio en él. Sin clonar un repositorio Git, no
podrás realizar cambios en él ni contribuir con tu trabajo.

\subsubsection{Cómo Clonar un Repositorio
Git}\label{cuxf3mo-clonar-un-repositorio-git}

Git es un sistema de control de versiones ampliamente utilizado en el
mundo empresarial. Antes de poder comenzar a trabajar en un repositorio,
es necesario clonarlo en tu computadora local. En este artículo, te
mostraré cómo clonar un repositorio Git en Ubuntu. Estos pasos son
aplicables para clonar repositorios desde plataformas populares como
GitHub, Bitbucket, GitLab y otras basadas en Git.

\subsubsection{Clonar un Repositorio
Remoto}\label{clonar-un-repositorio-remoto}

Supongamos que deseas clonar un repositorio remoto desde GitHub,
Bitbucket u otra plataforma en la nube hacia tu máquina local.

\begin{enumerate}
\def\labelenumi{\arabic{enumi}.}
\item
  Abre la terminal y navega hasta la ubicación donde deseas que se copie
  el repositorio, por ejemplo:

\begin{Shaded}
\begin{Highlighting}[]
\BuiltInTok{cd}\NormalTok{ /home/ubuntu/}
\end{Highlighting}
\end{Shaded}
\item
  Cada repositorio remoto de Git tiene una URL única. Inicia sesión en
  tu plataforma de desarrollo preferida, como GitHub, y copia la URL de
  tu repositorio.
\item
  Utiliza el siguiente comando \texttt{git\ clone} seguido de la URL del
  repositorio para clonarlo en tu máquina local. Por ejemplo:

\begin{Shaded}
\begin{Highlighting}[]
\FunctionTok{sudo}\NormalTok{ git clone https://github.com/usuario/repositorio}
\end{Highlighting}
\end{Shaded}
\end{enumerate}

Asegúrate de reemplazar ``usuario'' con tu nombre de usuario de GitHub y
``repositorio'' con el nombre de tu repositorio.

\begin{enumerate}
\def\labelenumi{\arabic{enumi}.}
\setcounter{enumi}{3}
\tightlist
\item
  Se te pedirá la contraseña para la autenticación, después de lo cual
  Git descargará automáticamente una copia de tu repositorio en el
  directorio de trabajo actual.
\end{enumerate}

\subsubsection{Clonar en una Carpeta
Específica}\label{clonar-en-una-carpeta-especuxedfica}

Si deseas clonar el repositorio en una carpeta específica, puedes
utilizar el siguiente comando:

\begin{Shaded}
\begin{Highlighting}[]
\FunctionTok{sudo}\NormalTok{ git clone }\OperatorTok{\textless{}}\NormalTok{repositorio}\OperatorTok{\textgreater{}} \OperatorTok{\textless{}}\NormalTok{directorio}\OperatorTok{\textgreater{}}
\end{Highlighting}
\end{Shaded}

Por ejemplo, supongamos que deseas clonar tu repositorio en la carpeta
\texttt{/home/desarrollador}:

\begin{Shaded}
\begin{Highlighting}[]
\FunctionTok{sudo}\NormalTok{ git clone https://github.com/usuario/repositorio /home/desarrollador}
\end{Highlighting}
\end{Shaded}

\begin{quote}
Ahora has aprendido cómo clonar un repositorio Git en tu máquina local
utilizando el comando \texttt{git\ clone}. ¡Comienza a trabajar en tu
repositorio clonado y realiza cambios en él!
\end{quote}

\subsubsection{Clonar un Repositorio
Superficialmente}\label{clonar-un-repositorio-superficialmente}

Si necesitas clonar un repositorio grande con un historial extenso de
confirmaciones, el proceso puede llevar mucho tiempo. Sin embargo, en
esos casos, existe la opción de realizar un clon superficial en el cual
puedes especificar las últimas ``n'' confirmaciones que deseas clonar.
Esto resultará en un proceso mucho más rápido y ocupará menos espacio en
tu sistema.

Aquí tienes la sintaxis para realizar un clon superficial, donde ``n''
representa el número de confirmaciones más recientes que deseas clonar:

\begin{Shaded}
\begin{Highlighting}[]
\FunctionTok{sudo}\NormalTok{ git clone }\AttributeTok{{-}{-}depth}\OperatorTok{=}\NormalTok{n }\OperatorTok{\textless{}}\NormalTok{repo}\OperatorTok{\textgreater{}}
\end{Highlighting}
\end{Shaded}

Por ejemplo, si deseas clonar solamente la última confirmación de tu
repositorio, puedes utilizar el siguiente comando:

\begin{Shaded}
\begin{Highlighting}[]
\FunctionTok{sudo}\NormalTok{ git clone }\AttributeTok{{-}{-}depth}\OperatorTok{=}\NormalTok{1 https://github.com/test\_user/demo.git}
\end{Highlighting}
\end{Shaded}

De manera similar, si deseas clonar las últimas 10 confirmaciones de tu
repositorio, puedes ejecutar este comando:

\begin{Shaded}
\begin{Highlighting}[]
\FunctionTok{sudo}\NormalTok{ git clone }\AttributeTok{{-}{-}depth}\OperatorTok{=}\NormalTok{10 https://github.com/test\_user/demo.git}
\end{Highlighting}
\end{Shaded}

Clonar una Rama Específica de Git

Si únicamente deseas clonar una rama específica (por ejemplo,
``working'') en lugar de todo el repositorio, puedes utilizar la opción
\texttt{-branch} en el comando \texttt{git\ clone}:

\begin{Shaded}
\begin{Highlighting}[]
\FunctionTok{git}\NormalTok{ clone }\AttributeTok{{-}{-}branch}\OperatorTok{=}\NormalTok{working https://github.com/test\_user/demo.git}
\end{Highlighting}
\end{Shaded}

\begin{quote}
¡Y eso es todo! Como puedes ver, clonar un repositorio Git en Ubuntu es
bastante sencillo. Ahora puedes utilizar estas técnicas para realizar
clones superficiales o clonar ramas específicas según tus necesidades.
\end{quote}

\subsubsection{Utilizar Git y GitHub para subir
proyectos}\label{utilizar-git-y-github-para-subir-proyectos}

Para comenzar a gestionar tus proyectos utilizando Git y GitHub, es
necesario realizar algunos pasos. En este artículo, te mostraremos cómo
subir un proyecto desde tu computadora a GitHub y también te enseñaremos
a utilizar los principales comandos a través de la terminal de Linux.

\textbf{Crear un nuevo repositorio}

En primer lugar, debes crear un nuevo repositorio en la página web de
GitHub. Dirígete a la página ``New Project'' y proporciona un nombre
para tu repositorio, una descripción y selecciona si deseas que el
repositorio sea público o privado (recuerda que los repositorios
privados requieren una suscripción). Una vez que hayas completado todos
los campos, haz clic en ``Create Repository'' y tu repositorio estará
listo.

\textbf{Subir un proyecto}

Para subir un proyecto, primero debes crear un proyecto Git localmente
en tu computadora. Para ello, ve al directorio raíz de tu proyecto a
través de la terminal utilizando el siguiente comando:

\begin{Shaded}
\begin{Highlighting}[]
\BuiltInTok{cd}\NormalTok{ ruta/al/archivo}
\end{Highlighting}
\end{Shaded}

Una vez que estés en el directorio del proyecto, inicializa Git
ejecutando el siguiente comando:

\begin{Shaded}
\begin{Highlighting}[]
\FunctionTok{git}\NormalTok{ init}
\end{Highlighting}
\end{Shaded}

Verás una confirmación de que el proyecto Git ha sido creado.

Ahora debes agregar los archivos al repositorio Git. Puedes hacerlo
utilizando una de las siguientes opciones, según si deseas agregar un
archivo específico o todos los archivos existentes en el proyecto:

\begin{Shaded}
\begin{Highlighting}[]
\FunctionTok{git}\NormalTok{ add .  }\CommentTok{\# Añade todos los archivos existentes en la carpeta al proyecto Git}
\FunctionTok{git}\NormalTok{ add nombredelarchivo.extension  }\CommentTok{\# Añade únicamente el archivo especificado al proyecto}
\end{Highlighting}
\end{Shaded}

Es importante que identifiques todos los cambios que realices en tu
proyecto con un comentario. Puedes hacerlo utilizando el siguiente
comando:

\begin{Shaded}
\begin{Highlighting}[]
\FunctionTok{git}\NormalTok{ commit }\AttributeTok{{-}m} \StringTok{\textquotesingle{}comentario\textquotesingle{}}
\end{Highlighting}
\end{Shaded}

Ahora estás listo para subir el proyecto a GitHub. Para hacerlo, debes
agregar el repositorio remoto ejecutando el siguiente comando en la
terminal:

\begin{Shaded}
\begin{Highlighting}[]
\FunctionTok{git}\NormalTok{ remote add origin git@github.com:achalmed/achalmaedison.web.git}
\end{Highlighting}
\end{Shaded}

Asegúrate de reemplazar ``achalmed'' con tu nombre de usuario en GitHub
y ``achalmaedison.web.git'' con el nombre de tu repositorio previamente
creado. Una vez que hayas ingresado los datos correctos, presiona Enter
y podrás subir el proyecto ejecutando el siguiente comando:

\begin{Shaded}
\begin{Highlighting}[]
\FunctionTok{git}\NormalTok{ push }\AttributeTok{{-}u}\NormalTok{ origin master}
\end{Highlighting}
\end{Shaded}

Ahora tu proyecto estará cargado en tu cuenta de GitHub. Puedes
verificarlo accediendo a la página del proyecto.

En caso de que encuentres un error llamado ``fatal: remote origin
already exists'' durante el paso anterior, debes ejecutar el siguiente
comando:

\begin{Shaded}
\begin{Highlighting}[]
\FunctionTok{git}\NormalTok{ remote rm origin}
\end{Highlighting}
\end{Shaded}

Luego, repite el proceso desde el principio para agregar el repositorio
remoto correctamente.

\begin{quote}
Con estos pasos, has aprendido cómo utilizar Git y GitHub para subir tus
proyectos y comenzar a gestionarlos desde allí. ¡Aprovecha al máximo
estas herramientas para colaborar y controlar las versiones de tus
proyectos de manera efectiva!
\end{quote}

\section{Observa tu Repositorio}\label{observa-tu-repositorio}

\subsection{git status}\label{git-status}

El comando \texttt{git\ status} muestra una lista de los archivos que
han sido modificados junto con los archivos que están pendientes de ser
preparados o confirmados.

Para listar los archivos no sincronizados, puedes utilizar
\texttt{git\ status\ -s}.

\begin{quote}
Verifica el estado actual del repositorio. Lista los archivos nuevos o
modificados que aún no han sido confirmados: \texttt{git\ status}
\end{quote}

\subsection{git diff}\label{git-diff}

\begin{itemize}
\item
  \texttt{git\ diff}: Muestra los cambios en archivos que aún no han
  sido preparados. Se utiliza para hacer una lista de todos los
  conflictos.
\item
  \texttt{git\ diff\ -\/-base\ \textless{}nombre-archivo\textgreater{}}:
  Permite ver los conflictos en relación al archivo base.
\item
  \texttt{git\ diff\ \textless{}rama-origen\textgreater{}\ \textless{}rama-destino\textgreater{}}:
  Se utiliza para ver los conflictos entre ramas antes de fusionarlas.
\item
  \texttt{git\ diff\ -\/-cached}: Muestra los cambios en los archivos
  preparados (staged).
\item
  \texttt{git\ diff\ HEAD}: Muestra todos los cambios en archivos, tanto
  preparados como no preparados.
\item
  \texttt{git\ diff\ commit1\ commit2}: Muestra los cambios entre dos
  identificadores de confirmación.
\end{itemize}

\begin{quote}
Muestra las diferencias de los cambios realizados y que no han sido
agregados a un commit.
\end{quote}

\subsection{git blame {[}archivo{]}}\label{git-blame-archivo}

\texttt{git\ blame\ {[}archivo{]}} lista las fechas y autores de los
cambios realizados en un archivo.

\subsection{git show}\label{git-show}

\begin{itemize}
\item
  \texttt{git\ show}: Se utiliza para mostrar información sobre
  cualquier objeto de Git.
\item
  \texttt{git\ show\ {[}confirmación{]}:{[}archivo{]}}: Muestra los
  cambios de un archivo para un identificador de confirmación y/o
  archivo específico.
\end{itemize}

\subsection{git log}\label{git-log}

\texttt{git\ log}: Muestra una lista de los cambios realizados
(commits), cuántas copias hay en el repositorio y el número de commits.
Muestra el historial completo de cambios.

\texttt{git\ log\ -p\ {[}archivo/directorio{]}}: Muestra el historial de
cambios para un archivo o directorio, incluyendo las diferencias.

\texttt{git\ log\ -\/-oneline}

\texttt{git\ log\ -\/-oneline\ -\/-decorate\ -\/-all\ -\/-graph\ -\/-since=2018-12-04}

\begin{quote}
\texttt{git\ log} se utiliza para ver el historial del repositorio,
mostrando detalles específicos de cada confirmación. Al ejecutar el
comando, obtendrás una salida similar a esta:
\end{quote}

\begin{Shaded}
\begin{Highlighting}[]
\ExtensionTok{achalmaubuntu\textbackslash{}@hp{-}pavilion:\textbackslash{}\textasciitilde{}/Documents/GitHub/achalmed\textbackslash{}$}\NormalTok{ git log}

\ExtensionTok{commit}\NormalTok{ 7e320e8bc6939626195a83def24f91308683a87e }\ErrorTok{(}\ExtensionTok{HEAD} \AttributeTok{{-}}\OperatorTok{\textgreater{}}\NormalTok{ main, origin/main, origin/HEAD}\KeywordTok{)}
\ExtensionTok{Author:}\NormalTok{ achalmed }\OperatorTok{\textless{}}\NormalTok{achalmaedison@outlook.com}\OperatorTok{\textgreater{}}
\ExtensionTok{Date:}\NormalTok{   Sun Jun 4 14:02:57 2023 }\AttributeTok{{-}0500}

    \ExtensionTok{update}
\end{Highlighting}
\end{Shaded}

\section{Trabajando con Ramas}\label{trabajando-con-ramas}

En este artículo, exploraremos las diferentes operaciones relacionadas
con las ramas en Git. Resaltaré los puntos clave de cada comando:

\subsection{Listar y crear ramas}\label{listar-y-crear-ramas}

\begin{itemize}
\item
  \textbf{Listar ramas locales}: El comando \texttt{git\ branch} muestra
  todas las ramas locales presentes en el repositorio. Puedes utilizarlo
  para obtener una lista de las ramas existentes.
\item
  \textbf{Listar todas las ramas}: Para ver tanto las ramas locales como
  las remotas, puedes usar \texttt{git\ branch\ -av}. Esto proporcionará
  una lista completa de todas las ramas del repositorio.
\item
  \textbf{Cambiar a una rama}: Si deseas cambiar a una rama específica y
  actualizar tu directorio de trabajo, utiliza
  \texttt{git\ checkout\ {[}mi\_rama{]}}.
\item
  \textbf{Crear una nueva rama}: Utiliza
  \texttt{git\ branch\ {[}nuevo\_nombre\_rama{]}} para crear una nueva
  rama con el nombre especificado.
\item
  \textbf{Eliminar una rama}: Si deseas eliminar una rama, utiliza
  \texttt{git\ branch\ -d\ {[}nombre\_rama{]}}. Esto eliminará la rama
  especificada.
\item
  \textbf{Ver todas las ramas}: Para obtener una lista de todas las
  ramas creadas en el proyecto, puedes utilizar
  \texttt{git\ branch\ -a}.
\end{itemize}

\subsection{Crear ramas para pruebas de
versión}\label{crear-ramas-para-pruebas-de-versiuxf3n}

Si necesitas crear una rama para realizar pruebas de versión u otras
modificaciones, sigue estos pasos:

\begin{enumerate}
\def\labelenumi{\arabic{enumi}.}
\item
  \textbf{Crear una nueva rama}: Utiliza
  \texttt{git\ branch\ {[}Rama2{]}} para crear una nueva rama basada en
  la rama actual.
\item
  \textbf{Ver los commits}: Utiliza \texttt{git\ log\ -\/-oneline} para
  mostrar todos los commits realizados en la rama.
\item
  \textbf{Indicar las ramas}: Puedes utilizar \texttt{git\ branch} para
  ver las ramas del proyecto y determinar en qué rama te encuentras
  actualmente.
\item
  \textbf{Cambiar a otra rama}: Utiliza \texttt{git\ checkout\ Rama2}
  para cambiar al flujo de trabajo de otra rama, por ejemplo, una rama
  de prueba.
\end{enumerate}

Una vez que hayas realizado las modificaciones de prueba, sigue estos
pasos:

\begin{enumerate}
\def\labelenumi{\arabic{enumi}.}
\setcounter{enumi}{4}
\item
  \textbf{Agregar los cambios}: Utiliza \texttt{git\ add\ .} para
  agregar los archivos modificados, incluyendo los cambios realizados en
  la rama de prueba.
\item
  \textbf{Realizar el commit}: Utiliza
  \texttt{git\ commit\ -m\ "Saludos,\ agregado\ a\ la\ rama\ de\ prueba"}
  para realizar el commit de los cambios.
\item
  \textbf{Ver los commits}: Utiliza \texttt{git\ log\ -\/-oneline} para
  mostrar todos los commits realizados en la nueva rama.
\end{enumerate}

Para regresar a la rama principal (Master o main), sigue estos pasos:

\begin{enumerate}
\def\labelenumi{\arabic{enumi}.}
\tightlist
\item
  \textbf{Cambiar a la rama principal}: Utiliza
  \texttt{git\ checkout\ {[}master{]}} para cambiar nuevamente al flujo
  de trabajo de la rama principal. Ten en cuenta que los cambios
  realizados en la rama de prueba no estarán presentes aquí.
\end{enumerate}

Si deseas ver los cambios en los archivos, minimiza la consola de Git.

Para enviar finalmente los archivos al repositorio en línea, escribe el
siguiente comando:

\begin{enumerate}
\def\labelenumi{\arabic{enumi}.}
\tightlist
\item
  \textbf{Enviar los cambios}: Utiliza
  \texttt{git\ push\ origin\ {[}Rama2{]}} para enviar los cambios al
  repositorio en línea. También puedes reemplazar \texttt{{[}Rama2{]}}
  con \texttt{main} si esa es la rama principal en tu repositorio en
  línea, como GitHub.
\end{enumerate}

\begin{quote}
¡Recuerda que trabajar con ramas te permite experimentar y realizar
pruebas sin afectar la rama principal del proyecto!
\end{quote}

\subsection{\texorpdfstring{Fusionando ramas con
\texttt{git\ checkout\ {[}archivo{]}}}{Fusionando ramas con git checkout {[}archivo{]}}}\label{fusionando-ramas-con-git-checkout-archivo}

El comando \texttt{git\ checkout\ {[}archivo{]}} es muy útil para crear
y navegar entre ramas en Git. Aquí están los puntos clave que debes
tener en cuenta:

\begin{itemize}
\item
  \textbf{Crear una nueva rama y cambiar automáticamente a ella}:
  Utiliza \texttt{git\ checkout\ -b\ {[}nombre-rama{]}} para crear una
  nueva rama y cambiar inmediatamente a ella.
\item
  \textbf{Cambiar entre ramas}: Para cambiar de una rama a otra,
  simplemente utiliza \texttt{git\ checkout\ {[}nombre-rama{]}}.
\end{itemize}

Cuando se trata de fusionar ramas, sigue estos pasos:

\begin{enumerate}
\def\labelenumi{\arabic{enumi}.}
\item
  \textbf{Cambiar a la rama de destino}: Antes de fusionar las ramas,
  asegúrate de estar en la rama de destino (por ejemplo, la rama
  \texttt{master}). Utiliza \texttt{git\ checkout\ {[}branch\_b{]}} para
  cambiar a esa rama.
\item
  \textbf{Fusionar la rama}: Utiliza
  \texttt{git\ merge\ {[}branch\_a{]}} para fusionar la rama
  \texttt{branch\_a} en la rama actual (\texttt{branch\_b}). Esto
  combinará los cambios realizados en \texttt{branch\_a} con la rama
  actual.
\end{enumerate}

Es importante destacar que al fusionar ramas, podrían surgir conflictos.
Sin embargo, si estás utilizando Visual Code, resolver los conflictos
será más sencillo gracias a sus herramientas de resolución de conflictos
integradas.

Para realizar la fusión de ramas correctamente, sigue estos pasos:

\begin{enumerate}
\def\labelenumi{\arabic{enumi}.}
\item
  \textbf{Cambiar a la rama principal}: Asegúrate de estar en la rama
  principal (por ejemplo, \texttt{master}) antes de realizar la fusión.
  Utiliza \texttt{git\ checkout\ {[}master{]}} para cambiar a la rama
  principal.
\item
  \textbf{Realizar la fusión}: Utiliza \texttt{git\ merge\ {[}Rama2{]}}
  para fusionar la rama \texttt{Rama2} en la rama principal.
\end{enumerate}

Recuerda que fusionar ramas te permite combinar los cambios realizados
en diferentes ramas, lo que facilita el trabajo colaborativo y la
incorporación de nuevas características a tu proyecto.

\begin{quote}
¡Experimenta con diferentes ramas y disfruta de la flexibilidad y el
poder de Git!
\end{quote}

\subsection{\texorpdfstring{Incorporando cambios con
\texttt{git\ rebase\ {[}nombre\_rama{]}}}{Incorporando cambios con git rebase {[}nombre\_rama{]}}}\label{incorporando-cambios-con-git-rebase-nombre_rama}

El comando \texttt{git\ rebase\ {[}nombre\_rama{]}} es una herramienta
poderosa que nos permite incorporar los cambios de una rama en otra.
Aquí están los puntos clave que debes conocer:

\begin{itemize}
\tightlist
\item
  \textbf{Incorporar cambios de una rama}: Utiliza
  \texttt{git\ rebase\ {[}nombre\_rama{]}} para traer los cambios de la
  rama especificada y aplicarlos en la rama actual. Esto puede ser útil
  para mantener un historial de cambios más limpio y lineal.
\end{itemize}

\subsection{\texorpdfstring{Etiquetando commits con
\texttt{git\ tag\ {[}versiones{]}}}{Etiquetando commits con git tag {[}versiones{]}}}\label{etiquetando-commits-con-git-tag-versiones}

Las etiquetas en Git nos permiten marcar puntos específicos en la
historia de nuestro proyecto. Aquí están los puntos clave que debes
tener en cuenta:

\begin{itemize}
\item
  \textbf{Etiquetar el commit actual}: Utiliza
  \texttt{git\ tag\ {[}nombre\_tag{]}} para crear una etiqueta en el
  commit actual. Por ejemplo, puedes usar \texttt{git\ tag\ v0.0.1} para
  marcar la versión v0.0.1.
\item
  \textbf{Etiquetar un commit específico}: Si deseas etiquetar un commit
  específico, puedes usar
  \texttt{git\ tag\ {[}nombre\_tag{]}/commit\_SHA1}, donde
  \texttt{commit\_SHA1} es el identificador único del commit.
\end{itemize}

Para acceder al commit donde se encuentra una etiqueta, utiliza el
siguiente comando:

\begin{Shaded}
\begin{Highlighting}[]
\FunctionTok{git}\NormalTok{ checkout }\PreprocessorTok{[}\SpecialStringTok{nombre\_tag}\PreprocessorTok{]}
\end{Highlighting}
\end{Shaded}

Las etiquetas son útiles para marcar puntos de lanzamiento o versiones
importantes en tu proyecto. Los desarrolladores las utilizan para
indicar hitos significativos, como v1.0 o v2.0. Por ejemplo, puedes
crear una etiqueta v1.1.0 en el commit específico utilizando el
siguiente comando:

\begin{Shaded}
\begin{Highlighting}[]
\FunctionTok{git}\NormalTok{ tag v1.1.0 }\PreprocessorTok{[}\SpecialStringTok{insertar}\PreprocessorTok{{-}}\SpecialStringTok{ID}\PreprocessorTok{{-}}\SpecialStringTok{commit}\PreprocessorTok{{-}}\SpecialStringTok{aquí}\PreprocessorTok{]}
\end{Highlighting}
\end{Shaded}

\subsubsection{Ejemplos}\label{ejemplos}

\begin{enumerate}
\def\labelenumi{\arabic{enumi}.}
\item
  \textbf{Creación de una etiqueta de versión:} Utiliza el comando
  \texttt{git\ tag\ 20-06-2023v1.0.0\ -m\ "Versión\ 1\ del\ proyecto"}
  para crear una etiqueta que represente una versión específica de tu
  proyecto.
\item
  \textbf{Subir etiquetas en línea:} Utiliza el comando
  \texttt{git\ push\ origin\ master\ -\/-tags} para enviar las etiquetas
  al repositorio remoto y compartirlas con otros colaboradores.
\end{enumerate}

\begin{quote}
Las etiquetas son una forma conveniente de marcar y acceder a puntos
importantes en tu historial de cambios. ¡Aprovecha su potencial y mantén
un seguimiento claro de las versiones de tu proyecto!
\end{quote}

\subsection{\texorpdfstring{Personalizando la configuración con
\texttt{git\ config}}{Personalizando la configuración con git config}}\label{personalizando-la-configuraciuxf3n-con-git-config}

El comando \texttt{git\ config} te permite personalizar la configuración
de Git según tus necesidades. Aquí tienes algunos puntos clave:

\begin{itemize}
\item
  \textbf{Agregar un alias}: Puedes utilizar el comando
  \texttt{git\ config\ -\/-global\ alias.lodag\ \textquotesingle{}log\ -\/-oneline\ -\/-decorate\ -\/-all\ -\/-graph\textquotesingle{}}
  para crear un alias. Los alias son atajos que te permiten ejecutar
  comandos largos y complejos de una manera más sencilla.
\item
  \textbf{Ver la lista de alias}: Si deseas ver la lista de alias
  creados, utiliza el comando
  \texttt{git\ config\ -\/-global\ -\/-get-regexp\ alias}. Esto te
  mostrará todos los alias definidos en la configuración global de Git.
\item
  \textbf{Eliminar un alias}: Si ya no necesitas un alias, puedes
  eliminarlo utilizando el comando
  \texttt{git\ config\ -\/-global\ -\/-unset\ alias.trololo}, donde
  ``trololo'' es el nombre del alias que deseas eliminar.
\end{itemize}

Además de los alias, \texttt{git\ config} también te permite establecer
otras configuraciones específicas del usuario, como el correo
electrónico y el nombre de usuario. Aquí tienes algunos ejemplos:

\begin{itemize}
\item
  \textbf{Establecer el correo electrónico}: Utiliza el comando
  \texttt{git\ config\ -\/-global\ user.email\ tuemail@ejemplo.com} para
  configurar tu dirección de correo electrónico. El uso de la opción
  \texttt{-\/-global} indica que esta configuración se aplicará a todos
  los repositorios locales.
\item
  \textbf{Configuración local}: Si deseas utilizar diferentes
  direcciones de correo electrónico para diferentes repositorios, puedes
  usar el comando
  \texttt{git\ config\ -\/-local\ user.email\ tuemail@ejemplo.com}. Esto
  establecerá la configuración solo para el repositorio actual.
\end{itemize}

Personalizar la configuración de Git según tus preferencias te permite
trabajar de manera más eficiente y adaptar Git a tus necesidades
específicas. ¡Aprovecha las opciones de \texttt{git\ config} para
optimizar tu flujo de trabajo!

\section{Realiza cambios en Git}\label{realiza-cambios-en-git}

Cuando trabajas con Git, es importante saber cómo hacer cambios y
preparar tus archivos correctamente. Aquí tienes algunos puntos clave:

\subsection{\texorpdfstring{\texttt{git\ add\ {[}archivo{]}}}{git add {[}archivo{]}}}\label{git-add-archivo}

\begin{itemize}
\item
  \textbf{Preparar un archivo específico}: Utiliza el comando
  \texttt{git\ add\ {[}archivo{]}} para preparar un archivo en
  particular y agregarlo al área de preparación. Por ejemplo, si deseas
  indexar el archivo \texttt{temp.txt}, ejecuta el siguiente comando:

\begin{Shaded}
\begin{Highlighting}[]
\FunctionTok{git}\NormalTok{ add temp.txt}
\end{Highlighting}
\end{Shaded}
\item
  \textbf{Preparar todos los archivos modificados}: Si has realizado
  cambios en varios archivos y deseas prepararlos todos, utiliza el
  comando \texttt{git\ add\ .}. Esto sincronizará y preparará todos los
  archivos modificados para el siguiente commit.
\item
  \textbf{Actualizar los cambios}: Si ya has preparado algunos archivos
  y deseas actualizar los cambios realizados, puedes utilizar el comando
  \texttt{git\ add\ -u}. Esto agregará al área de preparación todos los
  archivos que hayan sido modificados o eliminados.
\end{itemize}

\subsubsection{Ejemplos}\label{ejemplos-1}

\begin{itemize}
\item
  \textbf{Agregar un archivo específico}: Si deseas agregar el archivo
  \texttt{index.html} al área de preparación, ejecuta el siguiente
  comando:

\begin{Shaded}
\begin{Highlighting}[]
\FunctionTok{git}\NormalTok{ add index.html}
\end{Highlighting}
\end{Shaded}
\item
  \textbf{Actualizar todos los cambios}: Si has realizado modificaciones
  en varios archivos y deseas prepararlos todos, utiliza el siguiente
  comando:

\begin{Shaded}
\begin{Highlighting}[]
\FunctionTok{git}\NormalTok{ add .}
\end{Highlighting}
\end{Shaded}
\end{itemize}

Cuando hayas realizado tus cambios y preparado los archivos, sigue los
siguientes pasos para confirmarlos y enviarlos al repositorio en línea:

\begin{enumerate}
\def\labelenumi{\arabic{enumi}.}
\tightlist
\item
  Verifica la rama actual utilizando el comando \texttt{git\ branch}.
\item
  Utiliza \texttt{git\ add\ .} para agregar todos los archivos
  modificados al repositorio local. También puedes usar
  \texttt{git\ add\ {[}nombreDelArchivo{]}} para agregar un archivo
  específico.
\item
  Genera un commit para confirmar los cambios utilizando el comando
  \texttt{git\ commit\ -m\ "Aquí\ va\ el\ mensaje"}. Asegúrate de
  proporcionar un mensaje descriptivo que explique los cambios
  realizados.
\item
  Finalmente, envía los archivos al repositorio en línea utilizando el
  comando \texttt{git\ push\ origin\ {[}rama{]}}. La rama puede ser
  \texttt{master} o \texttt{main}, dependiendo de tu configuración.
\end{enumerate}

Si deseas verificar el estado de la transacción después de cada
instrucción, puedes utilizar el comando \texttt{git\ status}. Esto te
mostrará información sobre los archivos preparados, los cambios
pendientes y el estado actual del repositorio.

\begin{quote}
¡Recuerda que hacer cambios y preparar correctamente tus archivos es
fundamental para mantener un flujo de trabajo eficiente con Git!
\end{quote}

\section{Realizar commits en Git y fusionar
ramas}\label{realizar-commits-en-git-y-fusionar-ramas}

Cuando trabajas con Git, es esencial entender cómo realizar commits y
fusionar ramas. Aquí tienes algunos puntos clave:

\subsection{\texorpdfstring{\texttt{git\ commit}}{git commit}}\label{git-commit}

\begin{itemize}
\item
  El comando \texttt{git\ commit} crea una instantánea de los cambios
  realizados y los guarda en el directorio Git.
\item
  Utiliza
  \texttt{git\ commit\ -m\ "El\ mensaje\ que\ acompaña\ al\ commit\ va\ aquí"}
  para agregar un mensaje descriptivo que explique los cambios
  realizados en el commit.
\end{itemize}

Recuerda que los cambios confirmados no se envían automáticamente al
repositorio remoto.

\begin{itemize}
\tightlist
\item
  Si deseas agregar y hacer commit de todos los archivos rastreados en
  el historial versionado al mismo tiempo, utiliza
  \texttt{git\ commit\ -am\ "mensaje\ del\ commit"}.
\end{itemize}

\subsubsection{Ejemplos}\label{ejemplos-2}

\begin{itemize}
\item
  Ejecuta \texttt{git\ commit\ -m\ "Comienzo\ del\ proyecto"} para crear
  un commit con el mensaje ``Comienzo del proyecto''.
\item
  Si deseas agregar y hacer commit de todos los archivos modificados,
  utiliza \texttt{git\ commit\ -am\ "Párrafo\ y\ tamaño\ de\ fuente"}.
\end{itemize}

\subsection{\texorpdfstring{\texttt{git\ add\ .\ \&\&\ git\ commit\ -m\ "Cambionumeros"}}{git add . \&\& git commit -m "Cambionumeros"}}\label{git-add-.-git-commit--m-cambionumeros}

\begin{itemize}
\tightlist
\item
  Para agregar y hacer commit de los cambios en una sola línea, utiliza
  el comando
  \texttt{git\ add\ .\ \&\&\ git\ commit\ -m\ "Cambionumeros"}.
\end{itemize}

\subsection{\texorpdfstring{\texttt{git\ merge\ {[}nombre\_rama{]}}}{git merge {[}nombre\_rama{]}}}\label{git-merge-nombre_rama}

\begin{itemize}
\tightlist
\item
  El comando \texttt{git\ merge\ {[}nombre\_rama{]}} fusiona una rama
  con otra rama activa.
\end{itemize}

\subsubsection{Descripción}\label{descripciuxf3n}

Supongamos que tenemos las ramas ``develop'' y ``feature'', y queremos
integrar la rama ``feature'' en ``develop''. Sigue estos pasos:

\begin{enumerate}
\def\labelenumi{\arabic{enumi}.}
\tightlist
\item
  Cambia a la rama ``develop'' utilizando
  \texttt{git\ checkout\ develop}.
\item
  Ejecuta \texttt{git\ merge\ feature} para fusionar la rama ``feature''
  en la rama ``develop''.
\end{enumerate}

\subsection{\texorpdfstring{\texttt{git\ revert\ -\/-m\ 1\ SHA1\_merge}}{git revert -\/-m 1 SHA1\_merge}}\label{git-revert---m-1-sha1_merge}

\begin{itemize}
\tightlist
\item
  \texttt{git\ revert\ -m\ 1\ SHA1\_merge} revierte un merge específico
  especificando el SHA1 del merge que deseas revertir.
\end{itemize}

\subsection{\texorpdfstring{\texttt{git\ revert\ HEAD}}{git revert HEAD}}\label{git-revert-head}

\begin{itemize}
\tightlist
\item
  \texttt{git\ revert\ HEAD}
\item
  \texttt{git\ revert\ -\/-no-commit\ HEAD}
\item
  \texttt{git\ revert\ -\/-no-commit\ HEAD\textasciitilde{}1}
\item
  \texttt{git\ revert\ -\/-continue}
\end{itemize}

\subsubsection{Descripción}\label{descripciuxf3n-1}

El comando \texttt{git\ revert} revierte un commit en la rama actual.
Puedes revertir dos o más commits juntos utilizando el mismo comando de
revert para que se tomen como uno solo.

Recuerda que realizar commits y fusionar ramas correctamente es
fundamental para mantener un historial versionado limpio y una
colaboración efectiva en Git.

\section{\texorpdfstring{Retrocediendo en el tiempo con Git:
\texttt{git\ reset}}{Retrocediendo en el tiempo con Git: git reset}}\label{retrocediendo-en-el-tiempo-con-git-git-reset}

Cuando trabajamos en un proyecto de Git, es posible que necesitemos
retroceder en el tiempo y regresar a un commit anterior. Para ello, el
comando \texttt{git\ reset} es muy útil. Veamos cómo se utiliza:

\begin{itemize}
\item
  \texttt{git\ reset\ SHA1\_commit/{[}nombre\_rama{]}}
\item
  \texttt{git\ reset\ -\/-soft\ SHA1\_commit/{[}nombre\_rama{]}}
\item
  \texttt{git\ reset\ -\/-hard\ SHA1\_commit/{[}nombre\_rama{]}}
\end{itemize}

\subsubsection{Descripción}\label{descripciuxf3n-2}

El comando \texttt{git\ reset} nos permite regresar a un commit anterior
y deshacer cambios en nuestro proyecto. Dependiendo de las opciones que
utilicemos, los cambios se manejan de diferentes formas:

\begin{itemize}
\item
  \texttt{git\ reset\ SHA1\_commit/{[}nombre\_rama{]}}: Regresa a un
  commit o rama anterior y deja los cambios fuera del área de
  preparación.
\item
  \texttt{git\ reset\ -\/-soft\ SHA1\_commit/{[}nombre\_rama{]}}:
  Regresa a un commit o rama anterior y deja los cambios hechos en el
  commit que se va a quitar listos en el área de preparación.
\item
  \texttt{git\ reset\ -\/-hard\ SHA1\_commit/{[}nombre\_rama{]}}:
  Regresa a un commit o rama anterior y elimina por completo los
  cambios, dejando en blanco el área de preparación y el directorio de
  trabajo.
\end{itemize}

Es importante tener en cuenta que esta operación no afectará a otras
ramas, por lo que puedes realizar nuevos commits a partir de ese punto
sin modificar otras ramas del proyecto.

Adicionalmente, el comando \texttt{git\ reset\ -\/-hard\ HEAD} nos
permite resetear el índice y el directorio de trabajo al último estado
de confirmación.

Conocer y utilizar \texttt{git\ log} con sus diferentes opciones nos
permitirá manejar correctamente la creación de ramas, los movimientos
entre ellas y los avances y retrocesos entre commits.

\subsubsection{Regresando a un commit anterior en
Git}\label{regresando-a-un-commit-anterior-en-git}

A veces, necesitamos retroceder en el tiempo a un punto anterior en
nuestro proyecto. Para ello, podemos utilizar el comando
\texttt{git\ checkout} de la siguiente manera:

\begin{enumerate}
\def\labelenumi{\arabic{enumi}.}
\tightlist
\item
  Utiliza el comando \texttt{git\ log\ -\/-oneline} para ver la
  estructura de los últimos commits.
\item
  Ejecuta \texttt{git\ checkout\ {[}766abcd{]}} para regresar a un
  commit previo.
\end{enumerate}

Existen diferentes formas de retroceder en el tiempo a commits
anteriores. Además de \texttt{checkout}, también podemos utilizar
\texttt{reset} con los atributos \texttt{soft} o \texttt{hard}.

\begin{itemize}
\tightlist
\item
  \texttt{git\ reset\ -\/-soft\ {[}568abcj{]}}: Retrocede a un commit
  previo manteniendo los cambios.
\item
  \texttt{git\ reset\ -\/-soft\ HEAD\textasciitilde{}}: Deshace solo el
  último commit.
\item
  \texttt{git\ reset\ -\/-hard\ {[}789abcd{]}}: Elimina permanentemente
  los cambios realizados después de un commit específico.
\end{itemize}

Si deseamos eliminar los cambios después del último commit, podemos
utilizar \texttt{git\ reset\ -\/-hard\ HEAD\textasciitilde{}}. También
es posible utilizar el comando \texttt{stash} para descartar los cambios
antes de retornar a un commit.

\begin{quote}
Recuerda tener precaución al utilizar estos comandos, ya que pueden
modificar el historial del proyecto y afectar a otros colaboradores.
Siempre es recomendable crear una copia de seguridad o consultar con tu
equipo antes de realizar cambios significativos en el repositorio.
\end{quote}

\section{Synchronize}\label{synchronize}

El proceso de sincronización en Git es fundamental para mantener
actualizado el repositorio local con los cambios realizados en el
repositorio remoto. A continuación, se presentan algunos comandos clave
para sincronizar el repositorio local con el remoto.

\subsection{\texorpdfstring{\texttt{git\ fetch}}{git fetch}}\label{git-fetch}

El comando \texttt{git\ fetch} permite al usuario obtener todos los
objetos de un repositorio remoto que no se encuentran actualmente en el
directorio de trabajo local. Esto no fusiona automáticamente los cambios
con el repositorio local, solo los descarga y los hace accesibles para
su revisión.

\subsubsection{Ejemplo}\label{ejemplo}

Para obtener los cambios desde el repositorio en GitHub, puedes
ejecutar:

\begin{Shaded}
\begin{Highlighting}[]
\FunctionTok{git}\NormalTok{ fetch origin}
\end{Highlighting}
\end{Shaded}

\subsection{\texorpdfstring{\texttt{git\ pull}}{git pull}}\label{git-pull}

El comando \texttt{git\ pull} combina los cambios realizados en el
repositorio remoto con el directorio de trabajo local. Básicamente,
actualiza las ediciones de línea en el repositorio local. Es una
combinación de los comandos \texttt{git\ fetch} y \texttt{git\ merge}.

\subsubsection{Ejemplo}\label{ejemplo-1}

Para obtener los cambios y fusionarlos en el repositorio local, puedes
utilizar:

\begin{Shaded}
\begin{Highlighting}[]
\FunctionTok{git}\NormalTok{ pull origin }\PreprocessorTok{[}\SpecialStringTok{rama}\PreprocessorTok{]}
\end{Highlighting}
\end{Shaded}

o simplemente:

\begin{Shaded}
\begin{Highlighting}[]
\FunctionTok{git}\NormalTok{ pull }\AttributeTok{{-}{-}rebase}
\end{Highlighting}
\end{Shaded}

\subsection{\texorpdfstring{\texttt{git\ remote}}{git remote}}\label{git-remote}

El comando \texttt{git\ remote} se utiliza para administrar las
conexiones a repositorios remotos.

\begin{itemize}
\tightlist
\item
  \texttt{git\ remote}: Muestra todos los repositorios remotos
  conectados.
\item
  \texttt{git\ remote\ -v}: Lista las conexiones junto con sus URLs.
\end{itemize}

\subsubsection{Ejemplo}\label{ejemplo-2}

\begin{itemize}
\tightlist
\item
  Para agregar un repositorio remoto, utiliza el siguiente comando:
\end{itemize}

\begin{Shaded}
\begin{Highlighting}[]
\FunctionTok{git}\NormalTok{ remote add }\PreprocessorTok{[}\SpecialStringTok{nombre\_repo}\PreprocessorTok{]} \PreprocessorTok{[}\SpecialStringTok{URL}\PreprocessorTok{]}
\end{Highlighting}
\end{Shaded}

\begin{itemize}
\tightlist
\item
  Para eliminar una conexión a un repositorio remoto específico,
  utiliza:
\end{itemize}

\begin{Shaded}
\begin{Highlighting}[]
\FunctionTok{git}\NormalTok{ remote remove }\PreprocessorTok{[}\SpecialStringTok{nombre\_repo}\PreprocessorTok{]}
\end{Highlighting}
\end{Shaded}

\subsection{\texorpdfstring{\texttt{git\ push}}{git push}}\label{git-push}

El comando \texttt{git\ push} se utiliza para enviar los cambios locales
a la rama principal del repositorio remoto. Permite sincronizar los
commits locales con el repositorio remoto.

\subsubsection{Ejemplo}\label{ejemplo-3}

Para enviar los cambios al repositorio en línea, puedes utilizar el
siguiente comando:

\begin{Shaded}
\begin{Highlighting}[]
\FunctionTok{git}\NormalTok{ push origin }\PreprocessorTok{[}\SpecialStringTok{rama}\PreprocessorTok{]}
\end{Highlighting}
\end{Shaded}

En caso de que no desees enviar los cambios a la rama principal,
reemplaza \texttt{{[}rama{]}} por el nombre de la rama deseada.

\begin{quote}
Recuerda que al redactar tu propio contenido, es fundamental utilizar
tus propias palabras y expresiones únicas para evitar cualquier problema
de plagio. Las sugerencias proporcionadas son solo ejemplos de cómo
puedes mejorar la redacción del texto original.
\end{quote}

\section{Finally}\label{finally}

En esta sección, exploraremos algunos comandos adicionales de Git que
pueden resultar útiles en tu flujo de trabajo. A continuación, destacaré
los puntos clave de cada comando:

\subsection{\texorpdfstring{\texttt{git\ command\ -\/-help}}{git command -\/-help}}\label{git-command---help}

El comando \texttt{git\ command\ -\/-help} te proporciona información
detallada sobre el uso y las opciones disponibles para un comando
específico de Git. Si tienes dudas, siempre puedes recurrir a
\texttt{git\ help} para obtener ayuda.

\subsection{\texorpdfstring{\texttt{git\ stash}}{git stash}}\label{git-stash}

El comando \texttt{git\ stash} te permite guardar temporalmente los
cambios que aún no están listos para ser confirmados. Esto te permite
cambiar de tarea o rama sin perder tus modificaciones actuales.

\subsection{\texorpdfstring{\texttt{git\ ls-tree}}{git ls-tree}}\label{git-ls-tree}

El comando \texttt{git\ ls-tree} muestra información sobre un objeto de
árbol en el repositorio. Muestra los nombres y modos de cada ítem, así
como el valor blob de SHA-1. Puedes utilizar \texttt{git\ ls-tree\ HEAD}
para ver el árbol actual (HEAD) del repositorio.

\subsection{\texorpdfstring{\texttt{git\ cat-file}}{git cat-file}}\label{git-cat-file}

El comando \texttt{git\ cat-file} se utiliza para ver información sobre
un objeto específico en el repositorio. Puedes usar la opción
\texttt{-p} junto con el valor SHA-1 del objeto para ver su información
detallada.

\subsection{\texorpdfstring{\texttt{git\ grep}}{git grep}}\label{git-grep}

El comando \texttt{git\ grep} te permite buscar frases o palabras
específicas en los árboles de confirmación, el directorio de trabajo y
el área de preparación. Por ejemplo, para buscar la frase
``www.hostinger.com'' en todos los archivos, puedes ejecutar
\texttt{git\ grep\ "www.hostinger.com"}.

\subsection{\texorpdfstring{\texttt{gitk}}{gitk}}\label{gitk}

El comando \texttt{gitk} muestra una interfaz gráfica para el
repositorio local. Puedes utilizarlo para visualizar la historia del
repositorio, las ramas y los commits de una manera más visual e
interactiva.

\subsection{\texorpdfstring{\texttt{git\ instaweb}}{git instaweb}}\label{git-instaweb}

El comando \texttt{git\ instaweb} te permite explorar tu repositorio
local utilizando la interfaz GitWeb. Puedes configurar diferentes
opciones, como el servidor HTTP, para acceder a la interfaz.

\subsection{\texorpdfstring{\texttt{git\ gc}}{git gc}}\label{git-gc}

El comando \texttt{git\ gc} realiza la limpieza y optimización del
repositorio local. Elimina archivos innecesarios y optimiza el
almacenamiento para mejorar el rendimiento.

\subsection{\texorpdfstring{\texttt{git\ archive}}{git archive}}\label{git-archive}

El comando \texttt{git\ archive} te permite crear archivos zip o tar que
contienen los archivos de un árbol de repositorio específico. Puedes
especificar el formato y el árbol que deseas archivar.

\subsection{\texorpdfstring{\texttt{git\ prune}}{git prune}}\label{git-prune}

El comando \texttt{git\ prune} elimina los objetos del repositorio que
no tienen apuntadores entrantes. Esto ayuda a limpiar el repositorio y
eliminar objetos no utilizados.

\subsection{\texorpdfstring{\texttt{git\ fsck}}{git fsck}}\label{git-fsck}

El comando \texttt{git\ fsck} realiza una comprobación de integridad del
sistema de archivos git y busca objetos corruptos en el repositorio. Es
útil para detectar posibles problemas de integridad.

\subsection{\texorpdfstring{\texttt{git\ rebase}}{git rebase}}\label{git-rebase}

El comando \texttt{git\ rebase} se utiliza para aplicar cambios de una
rama a otra. Permite mover, modificar o combinar commits para mantener
una historia de confirmaciones más limpia y estructurada.

\subsection{\texorpdfstring{\texttt{git\ rm}}{git rm}}\label{git-rm}

El comando \texttt{git\ rm} se utiliza para eliminar archivos del
repositorio de Git. Puedes especificar los archivos que deseas eliminar
y luego confirmar los cambios.

\subsubsection{Borrar archivos/carpetas del
repositorio}\label{borrar-archivoscarpetas-del-repositorio}

En este apartado, exploraremos cómo eliminar archivos o carpetas en un
repositorio utilizando los comandos de Git. A continuación, resaltaré
los puntos clave de cada paso:

\begin{enumerate}
\def\labelenumi{\arabic{enumi}.}
\item
  \textbf{Actualizar los cambios:} Antes de borrar archivos o carpetas,
  es importante asegurarse de que todos los cambios estén actualizados
  en el repositorio. Puedes hacerlo utilizando el siguiente comando:

\begin{Shaded}
\begin{Highlighting}[]
\FunctionTok{git}\NormalTok{ add }\AttributeTok{{-}u}
\end{Highlighting}
\end{Shaded}

  Este comando actualizará todos los cambios realizados en el
  repositorio.
\item
  \textbf{Realizar un commit:} Después de asegurarte de que los cambios
  están actualizados, debes crear un commit para registrar la
  eliminación de los archivos o carpetas innecesarios. Puedes utilizar
  el siguiente comando:

\begin{Shaded}
\begin{Highlighting}[]
\FunctionTok{git}\NormalTok{ commit }\AttributeTok{{-}m} \StringTok{"Elimino archivos innecesarios"}
\end{Highlighting}
\end{Shaded}

  Recuerda proporcionar un mensaje descriptivo para el commit.
\item
  \textbf{Subir los cambios:} Finalmente, debes enviar los cambios al
  repositorio en línea. Utiliza el siguiente comando para realizarlo:

\begin{Shaded}
\begin{Highlighting}[]
\FunctionTok{git}\NormalTok{ push origin master}
\end{Highlighting}
\end{Shaded}

  Ten en cuenta que ``master'' puede ser reemplazado por ``main'' si esa
  es la rama principal en tu repositorio.
\end{enumerate}

Si deseas eliminar un archivo específico, utiliza el siguiente comando:

\begin{Shaded}
\begin{Highlighting}[]
\FunctionTok{git}\NormalTok{ rm miarchivo.php}
\end{Highlighting}
\end{Shaded}

En caso de querer eliminar una carpeta completa, puedes utilizar el
siguiente comando:

\begin{Shaded}
\begin{Highlighting}[]
\FunctionTok{git}\NormalTok{ rm }\AttributeTok{{-}r}\NormalTok{ micarpeta}
\end{Highlighting}
\end{Shaded}

Recuerda que después de ejecutar el comando \texttt{git\ rm}, debes
realizar un commit y luego enviar los cambios al repositorio utilizando
\texttt{git\ push}.

\section{Conclusión}\label{conclusiuxf3n}

Aprender los comandos básicos de Git es fundamental para los
desarrolladores, ya que les permite gestionar fácilmente el código
fuente de sus proyectos. Aunque pueda llevar algo de tiempo recordarlos
todos, nuestra hoja de trucos de Git puede ser de gran utilidad.

¡Practica estos comandos de Git y aprovecha al máximo tus habilidades de
desarrollo! ¡Te deseamos mucho éxito!

\emph{Edison Achalma}


\printbibliography


\end{document}
