\documentclass[
  jou,
  floatsintext,
  longtable,
  a4paper,
  nolmodern,
  notxfonts,
  notimes,
  colorlinks=true,linkcolor=blue,citecolor=blue,urlcolor=blue]{apa7}

\usepackage{amsmath}
\usepackage{amssymb}



\usepackage[bidi=default]{babel}
\babelprovide[main,import]{spanish}
\StartBabelCommands{spanish}{captions} [unicode, fontenc=TU EU1 EU2, charset=utf8] \SetString{\keywordname}{Palabras
Claves}
\EndBabelCommands


% get rid of language-specific shorthands (see #6817):
\let\LanguageShortHands\languageshorthands
\def\languageshorthands#1{}

\RequirePackage{longtable}
\RequirePackage{threeparttablex}

\makeatletter
\renewcommand{\paragraph}{\@startsection{paragraph}{4}{\parindent}%
	{0\baselineskip \@plus 0.2ex \@minus 0.2ex}%
	{-.5em}%
	{\normalfont\normalsize\bfseries\typesectitle}}

\renewcommand{\subparagraph}[1]{\@startsection{subparagraph}{5}{0.5em}%
	{0\baselineskip \@plus 0.2ex \@minus 0.2ex}%
	{-\z@\relax}%
	{\normalfont\normalsize\bfseries\itshape\hspace{\parindent}{#1}\textit{\addperi}}{\relax}}
\makeatother




\usepackage{longtable, booktabs, multirow, multicol, colortbl, hhline, caption, array, float, xpatch}
\usepackage{subcaption}
\renewcommand\thesubfigure{\Alph{subfigure}}
\setcounter{topnumber}{2}
\setcounter{bottomnumber}{2}
\setcounter{totalnumber}{4}
\renewcommand{\topfraction}{0.85}
\renewcommand{\bottomfraction}{0.85}
\renewcommand{\textfraction}{0.15}
\renewcommand{\floatpagefraction}{0.7}

\usepackage{tcolorbox}
\tcbuselibrary{listings,theorems, breakable, skins}
\usepackage{fontawesome5}

\definecolor{quarto-callout-color}{HTML}{909090}
\definecolor{quarto-callout-note-color}{HTML}{0758E5}
\definecolor{quarto-callout-important-color}{HTML}{CC1914}
\definecolor{quarto-callout-warning-color}{HTML}{EB9113}
\definecolor{quarto-callout-tip-color}{HTML}{00A047}
\definecolor{quarto-callout-caution-color}{HTML}{FC5300}
\definecolor{quarto-callout-color-frame}{HTML}{ACACAC}
\definecolor{quarto-callout-note-color-frame}{HTML}{4582EC}
\definecolor{quarto-callout-important-color-frame}{HTML}{D9534F}
\definecolor{quarto-callout-warning-color-frame}{HTML}{F0AD4E}
\definecolor{quarto-callout-tip-color-frame}{HTML}{02B875}
\definecolor{quarto-callout-caution-color-frame}{HTML}{FD7E14}

%\newlength\Oldarrayrulewidth
%\newlength\Oldtabcolsep


\usepackage{hyperref}




\providecommand{\tightlist}{%
  \setlength{\itemsep}{0pt}\setlength{\parskip}{0pt}}
\usepackage{longtable,booktabs,array}
\usepackage{calc} % for calculating minipage widths
% Correct order of tables after \paragraph or \subparagraph
\usepackage{etoolbox}
\makeatletter
\patchcmd\longtable{\par}{\if@noskipsec\mbox{}\fi\par}{}{}
\makeatother
% Allow footnotes in longtable head/foot
\IfFileExists{footnotehyper.sty}{\usepackage{footnotehyper}}{\usepackage{footnote}}
\makesavenoteenv{longtable}

\usepackage{graphicx}
\makeatletter
\newsavebox\pandoc@box
\newcommand*\pandocbounded[1]{% scales image to fit in text height/width
  \sbox\pandoc@box{#1}%
  \Gscale@div\@tempa{\textheight}{\dimexpr\ht\pandoc@box+\dp\pandoc@box\relax}%
  \Gscale@div\@tempb{\linewidth}{\wd\pandoc@box}%
  \ifdim\@tempb\p@<\@tempa\p@\let\@tempa\@tempb\fi% select the smaller of both
  \ifdim\@tempa\p@<\p@\scalebox{\@tempa}{\usebox\pandoc@box}%
  \else\usebox{\pandoc@box}%
  \fi%
}
% Set default figure placement to htbp
\def\fps@figure{htbp}
\makeatother







\usepackage{newtx}

\defaultfontfeatures{Scale=MatchLowercase}
\defaultfontfeatures[\rmfamily]{Ligatures=TeX,Scale=1}





\title{Economía Peruana 1970-1990: Analiza las políticas económicas y
desafíos en Perú durante dos décadas críticas}


\shorttitle{Economía Peruana 1970-1990}


\usepackage{etoolbox}



\ccoppy{\textcopyright~2023}



\author{Edison Achalma}



\affiliation{
{Escuela Profesional de Economía, Universidad Nacional de San Cristóbal
de Huamanga}}




\leftheader{Achalma}

\date{2023-05-12}


\abstract{This article examines the Peruvian economy from 1970 to 1990,
analyzing economic policies, reforms, and the resulting challenges
during this period. It covers the administrations of Juan Velasco
Alvarado, Francisco Morales Bermúdez, Fernando Belaúnde Terry, and Alan
García Pérez, detailing measures like agrarian reform, industrial
promotion, control over foreign investment, and economic adjustments.
The document highlights the transition from state interventionism to
economic liberalization in the 1990s, discussing impacts like inflation,
fiscal deficits, and social inequalities. }

\keywords{Peruvian Economy, economic crisis Peru, Economic reforms, Lost
decade, hyperinflation Peru}

\authornote{\par{\addORCIDlink{Edison Achalma}{0000-0001-6996-3364}} 
\par{ }
\par{   El autor no tiene conflictos de interés que revelar.    Los
roles de autor se clasificaron utilizando la taxonomía de roles de
colaborador (CRediT; https://credit.niso.org/) de la siguiente
manera:  Edison Achalma:   conceptualización, redacción}
\par{La correspondencia relativa a este artículo debe dirigirse a Edison
Achalma, Email: \href{mailto:elmer.achalma.09@unsch.edu.pe}{elmer.achalma.09@unsch.edu.pe}}
}

\usepackage{pbalance} 
\usepackage{float}
\makeatletter
\let\oldtpt\ThreePartTable
\let\endoldtpt\endThreePartTable
\def\ThreePartTable{\@ifnextchar[\ThreePartTable@i \ThreePartTable@ii}
\def\ThreePartTable@i[#1]{\begin{figure}[!htbp]
\onecolumn
\begin{minipage}{0.5\textwidth}
\oldtpt[#1]
}
\def\ThreePartTable@ii{\begin{figure}[!htbp]
\onecolumn
\begin{minipage}{0.5\textwidth}
\oldtpt
}
\def\endThreePartTable{
\endoldtpt
\end{minipage}
\twocolumn
\end{figure}}
\makeatother


\makeatletter
\let\endoldlt\endlongtable		
\def\endlongtable{
\hline
\endoldlt}
\makeatother

\newenvironment{twocolumntable}% environment name
{% begin code
\begin{table*}[!htbp]%
\onecolumn%
}%
{%
\twocolumn%
\end{table*}%
}% end code

\urlstyle{same}



\makeatletter
\@ifpackageloaded{caption}{}{\usepackage{caption}}
\AtBeginDocument{%
\ifdefined\contentsname
  \renewcommand*\contentsname{Tabla de contenidos}
\else
  \newcommand\contentsname{Tabla de contenidos}
\fi
\ifdefined\listfigurename
  \renewcommand*\listfigurename{Listado de Figuras}
\else
  \newcommand\listfigurename{Listado de Figuras}
\fi
\ifdefined\listtablename
  \renewcommand*\listtablename{Listado de Tablas}
\else
  \newcommand\listtablename{Listado de Tablas}
\fi
\ifdefined\figurename
  \renewcommand*\figurename{Figura}
\else
  \newcommand\figurename{Figura}
\fi
\ifdefined\tablename
  \renewcommand*\tablename{Tabla}
\else
  \newcommand\tablename{Tabla}
\fi
}
\@ifpackageloaded{float}{}{\usepackage{float}}
\floatstyle{ruled}
\@ifundefined{c@chapter}{\newfloat{codelisting}{h}{lop}}{\newfloat{codelisting}{h}{lop}[chapter]}
\floatname{codelisting}{Listado}
\newcommand*\listoflistings{\listof{codelisting}{Listado de Listados}}
\makeatother
\makeatletter
\makeatother
\makeatletter
\@ifpackageloaded{caption}{}{\usepackage{caption}}
\@ifpackageloaded{subcaption}{}{\usepackage{subcaption}}
\makeatother
\makeatletter
\@ifpackageloaded{fontawesome5}{}{\usepackage{fontawesome5}}
\makeatother

% From https://tex.stackexchange.com/a/645996/211326
%%% apa7 doesn't want to add appendix section titles in the toc
%%% let's make it do it
\makeatletter
\xpatchcmd{\appendix}
  {\par}
  {\addcontentsline{toc}{section}{\@currentlabelname}\par}
  {}{}
\makeatother

%% Disable longtable counter
%% https://tex.stackexchange.com/a/248395/211326

\usepackage{etoolbox}

\makeatletter
\patchcmd{\LT@caption}
  {\bgroup}
  {\bgroup\global\LTpatch@captiontrue}
  {}{}
\patchcmd{\longtable}
  {\par}
  {\par\global\LTpatch@captionfalse}
  {}{}
\apptocmd{\endlongtable}
  {\ifLTpatch@caption\else\addtocounter{table}{-1}\fi}
  {}{}
\newif\ifLTpatch@caption
\makeatother

\begin{document}

\maketitle

\hypertarget{toc}{}
\tableofcontents
\newpage
\section[Introduction]{Economía Peruana 1970-1990}

\setcounter{secnumdepth}{-\maxdimen} % remove section numbering

\setlength\LTleft{0pt}


¿Sabías que en apenas dos décadas, Perú pasó de una ambiciosa reforma
agraria a una hiperinflación devastadora que marcó a toda una
generación? Durante los años 1970 a 1990, la economía peruana vivió
transformaciones radicales, desde políticas de nacionalización hasta
ajustes neoliberales, dejando lecciones imborrables. Este periodo, lleno
de contrastes, refleja cómo decisiones económicas moldearon el destino
de millones. En este artículo, exploraremos las principales medidas
económicas de los gobiernos de Velasco, Morales, Belaunde, García y
Fujimori, analizando su impacto en la sociedad y la estabilidad del
país. Desde la reforma agraria hasta el ``Fujishock'', descubrirás cómo
estas políticas definieron el rumbo de Perú, sus aciertos y sus costos
sociales.

\subsection{Reforma Agraria y Nacionalización: El Gobierno de General
Juan Velasco Alvarado
(1970-1975)}\label{reforma-agraria-y-nacionalizaciuxf3n-el-gobierno-de-general-juan-velasco-alvarado-1970-1975}

\textbf{Las reformas económicas de Juan Velasco Alvarado}, implementadas
a partir de 1968, transformaron profundamente la estructura productiva y
social del Perú, buscando equidad distributiva y soberanía nacional.
\textbf{La Reforma Agraria de 1969}, pilar del gobierno, expropió
latifundios para crear cooperativas, promoviendo el principio de ``la
tierra es de quien la trabaja'' (Decreto Ley 17716, 1969). En 1970, la
Ley General de Industrias priorizó sectores estratégicos con incentivos
tributarios, reservando al Estado el monopolio de industrias básicas,
mientras la Ley de Minería reguló la inversión extranjera y otorgó al
Estado el control de la comercialización minera (Decreto Ley 18225,
1970). Se instauraron comunidades laborales en industria, minería y
pesca, otorgando a los trabajadores participación en la gestión y
beneficios, junto con la Ley de Estabilidad Laboral (1970), que
garantizaba seguridad tras tres meses de empleo. Sin embargo, tmbien se
observa la contradicción de estas medidas: aunque promovían justicia
social, su implementación autoritaria limitó libertades ciudadanas,
evidenciando un enfoque vertical. Estas reformas, aunque ambiciosas,
enfrentaron críticas por su falta de cohesión y sostenibilidad
económica.

\subsection{Ajuste Económico y Crisis: El Gobierno de Francisco Morales
Bermúdez Cerruti
(1975-1980)}\label{ajuste-econuxf3mico-y-crisis-el-gobierno-de-francisco-morales-bermuxfadez-cerruti-1975-1980}

\textbf{El ajuste económico implementado durante el gobierno de
Francisco Morales Bermúdez (1975-1980) marcó un período de profundas
transformaciones en el Perú, caracterizado por medidas de austeridad y
liberalización para enfrentar una crisis económica.} En un contexto de
reservas internacionales negativas y un déficit fiscal que alcanzó el
6.3\% del PBI en 1976, el gobierno promovió la inversión privada,
eliminó subsidios, buscando reducir el gasto público y mejorar la
balanza comercial. Estas políticas, sin embargo, desencadenaron una
inflación que escaló al 73.7\% en 1978 y una caída del PBI del 3.8\% en
el mismo año, exacerbada por minidevaluaciones y una reducción de la
demanda interna. Según Thorp y Bertram (1978), las estrategias de
ajuste, influenciadas por las exigencias del FMI para liberar el tipo de
cambio, priorizaron la estabilidad macroeconómica sobre el bienestar
social, lo que provocó huelgas generales debido a la caída de los
salarios reales. No obstante, el aumento de las exportaciones de
petróleo (254\% anual entre 1977-1979) y la mejora de los términos de
intercambio en 1979 mitigaron parcialmente la brecha externa, que pasó
de -7.4\% a -1.8\% del PBI en 1978.

\subsection{Década Perdida y Fenómeno del Niño
(1980-1985)}\label{duxe9cada-perdida-y-fenuxf3meno-del-niuxf1o-1980-1985}

\textbf{La Década Perdida en el Perú}, marcada por la confluencia de
crisis económicas, sociales y ambientales, representa un período crítico
en la historia del país, especialmente entre 1980 y 1985 bajo el segundo
gobierno de Fernando Belaúnde Terry. Este período, caracterizado por la
guerra interna con Sendero Luminoso, el devastador \textbf{Fenómeno del
Niño de 1983} y una hiperinflación galopante, dejó al Perú en un estado
de aislamiento político y financiero (Thorp y Bertram, 2013). La
economía social de mercado, regida por la Constitución de 1979, enfrentó
desafíos estructurales agravados por una política económica ortodoxa
supervisada por el FMI, que priorizó la reducción del déficit fiscal.
Sin embargo, estas medidas no lograron contrarrestar la caída del PBI,
que registró un crecimiento negativo de -12.2\% en 1983, ni la
disminución del ingreso per cápita de 1,232 dólares en 1980 a 1,050
dólares en 1985. Comparativamente, autores como ECLAC (1985) destacan
que la región latinoamericana también sufrió una contracción económica,
pero el caso peruano se agudizó por la dependencia de exportaciones
primarias, cuyos precios colapsaron. \textbf{La Década Perdida} no solo
refleja un retroceso económico, sino un quiebre social que marcó a
generaciones.

\textbf{El Fenómeno del Niño de 1983} exacerbó la crisis al devastar
sectores clave como la agricultura (-12\%), la pesca (-40\%) y la
minería (-8\%), según datos del Banco Central de Reserva del Perú (BCRP,
1984). Las políticas ortodoxas, enfocadas en reducir el gasto fiscal y
promover la inversión extranjera, no lograron mitigar el impacto de este
desastre climático ni la salida de capitales privados, impulsada por
altas tasas de interés internacionales y la inseguridad interna. La
inversión cayó drásticamente de 21.2\% del PBI en 1982 a 12.2\% en 1985,
evidenciando un colapso económico estructural. Desde un enfoque
deductivo, la combinación de factores exógenos (clima, precios
internacionales) y endógenos (terrorismo, políticas inadecuadas) creó un
círculo vicioso de estancamiento. Este análisis subraya la necesidad de
políticas económicas resilientes frente a shocks externos.

\subsection{Hiperinflación y Políticas Heterodoxas en el Perú: El
Gobierno de Alan García
(1985-1990)}\label{hiperinflaciuxf3n-y-poluxedticas-heterodoxas-en-el-peruxfa-el-gobierno-de-alan-garcuxeda-1985-1990}

\textbf{La hiperinflación en el Perú} durante el primer gobierno de Alan
García (1985-1990) marcó un período de colapso económico y social,
caracterizado por políticas heterodoxas que desafiaron las directrices
del Fondo Monetario Internacional (FMI). Implementando medidas como la
congelación de precios, salarios, tasas de interés y el tipo de cambio,
junto con una devaluación inicial del dólar del 12\%, el gobierno buscó
estimular la producción y limitar el pago de la deuda externa al 10\% de
las exportaciones (Webb y Fernández Baca, 1990). Sin embargo, estas
políticas generaron desequilibrios macroeconómicos, con una inflación
acumulada de 2,178,482\% y una caída del PBI real, exacerbada por el
desabastecimiento y la especulación. La declaración de inelegibilidad
para créditos internacionales en 1986 aisló al país financieramente,
mientras que el intento de estatizar la banca privada en 1987 desató
protestas y el surgimiento del movimiento Libertad. Comparativamente,
autores como Sachs (1989) señalan que las políticas heterodoxas en
América Latina a menudo subestimaron los riesgos de inflación
descontrolada. \textbf{La hiperinflación} no solo devastó la economía,
sino que erosionó la confianza social.

\textbf{Las políticas heterodoxas} de García, inicialmente exitosas en
1985 al impulsar la producción, revelaron su insostenibilidad para 1987,
cuando la crisis de balanza de pagos y la caída de reservas
internacionales desencadenaron los ``paquetazos'' económicos de 1988,
profundizando la recesión (BCRP, 1989). El aumento de precios de
servicios básicos (gasolina +30\%, agua +10\%) y el déficit fiscal
descontrolado agudizaron el desabastecimiento, las huelgas y la
violencia social.

\subsection{Reformas Neoliberales de los
90}\label{reformas-neoliberales-de-los-90}

\textbf{Las reformas neoliberales en el Perú} durante el gobierno de
Alberto Fujimori (1990-2000) transformaron radicalmente la economía y la
estructura estatal, buscando estabilización macroeconómica y reinserción
en el sistema financiero internacional. Iniciadas en 1990 con el
``Fujishock'', estas reformas incluyeron la liberalización de mercados
financieros, la privatización de monopolios estatales en sectores como
salud, educación y minería, y la flexibilización del mercado laboral,
respaldadas por el Banco Mundial (Guerra-García, 1999). La Comisión de
Promoción de la Inversión Privada (COPRI), creada en 1992, aceleró la
privatización, mientras que la restricción de la oferta monetaria y la
liberación de precios redujeron la inflación, aunque dispararon los
costos de productos básicos. Comparativamente, autores como Williamson
(1990) destacan que el Consenso de Washington, base de estas políticas,
priorizó la disciplina fiscal sobre el bienestar social. \textbf{Las
reformas neoliberales} lograron estabilización económica, pero a un alto
costo social.

\textbf{El costo social de las reformas neoliberales} fue significativo,
afectando principalmente a trabajadores, campesinos y jubilados, con un
aumento de la pobreza del 44\% en 1990 al 60\% tras el ``Fujishock''
(DESCO, 1990). La liberalización del comercio exterior y la eliminación
de controles de precios generaron alzas inmediatas en productos
esenciales, mientras que la reducción de salarios reales y derechos
laborales profundizó la desigualdad. Estas medidas, aunque efectivas
para reducir el déficit fiscal y atraer inversión extranjera,
subestimaron el impacto en los sectores más vulnerables.

\section{Conclusión}\label{conclusiuxf3n}

La economía peruana ha sufrido de una deficiente asignación de recursos
y bajos niveles de inversión privada. Durante las décadas de los
sesenta, setenta y ochenta el sistema económico presentaba un cuadro con
precios regulados y excesivo intervencionismo estatal. Asimismo, la
actividad económica se desarrollaba en un ambiente de fuerte
inestabilidad macroeconómica y los altos endeudamientos públicos
terminaban con episodios de insolvencia fiscal. De otro lado, eran
recurrentes las crisis de balanza de pagos, que terminaban en grandes
devaluaciones de la moneda; y, por si fuera poco, se vivieron periodos
inflacionarios e hiperinflacionarios, mientras en la década de los 90 se
inició con los ajustes para estabilizar la economía peruana.

El Perú de hoy es macro económicamente más estable y próspero, pero es
también más desigual y excluyente. La riqueza se ha concentrado y el
Estado se ha debilitado y retrocedido, mientras
los~sempiternos~problemas de la pobreza y la discriminación se han
profundizado.

\section{Bibliografía}\label{bibliografuxeda}

\begin{itemize}
\item
  Carlos PARODI TRECE, ``Perú, 1960-2000: políticas económicas y
  sociales en entornos cambiantes.'' 2001
\item
  SEGUNDO GOBIERNO DE FERNANDO BELAUNDE TERRY: TRABAJO REALIZADO POR
  ESTUDIANTES DE LA UNIVERSIDAD DE LIMA
\item
  Waldo, M. B. (2013). Contexto internacional y desempeño macroeconómico
  en américa latina y el Perú: 1980-2012. Lima, Perú: Departamento de
  Economía -- Pontificia Universidad Católica del Perú
\item
  Alfredo, D., \& Raúl, G.C. (2013).La economía mundial ¿Hacia dónde
  vamos? Lima, Perú: Pontificia Universidad Católica del Perú.
\item
  Efraín, G. O. (1991) Una economía bajo violencia: Perú, 1980-1990.
  Lima, Perú: IEP Instituto de Estudios Peruanos
\item
  Eduardo, M.P. (2013). Los desafíos del Perú. Lima, Perú: Universidad
  del Pacífico.
\item
  Abusada, R., F. Dubois, E. Morón (2000) La Reforma Incompleta:
  Rescatando los noventa. Lima, Perú. Instituto Peruano de
  Economía/Universidad del Pacífico
\item
  Galeón. (n.d). Historia. Obtenida el 11 de julio del 2015
\item
  La década de los noventa y los dos gobiernos de Alberto Fujimori.
  Obtenida el 11 de julio del 2015, de
  http://accion-popular-juventud-arequipa.blogspot.com/2010/03/segundo-gobierno-de-fernando-belaunde.html
\item
  Álvarez Palenzuela, V.A. (coord.) (2008). Historia Universal de la
  Edad Media, Barcelona: Ariel.
\item
  Bloch, M. (1989). Feudal Society: The Growth of Ties of Dependence,
  Londres: Routledge.
\item
  Boutruche, R. (1995). Señorío y feudalismo, 1.Los vínculos de
  dependencia, Madrid: Siglo XXI.
\item
  Calva, J. L. (1988). Los campesinos y su devenir en las economías de
  mercado, México, Siglo XXI.
\item
  Ganshof, F.L. (1975). El feudalismo, Barcelona: Ariel, Barcelona.
\item
  Huizinga, J. (2001). El Otoño de la Edad Media, Madrid: Alianza.
\item
  Little, L. y B. Rosenwein (ed.) (2003). La Edad Media a debate,
  Madrid: Akal.
\item
  Mitre, E. (2004). Introducción a la Historia de la Edad Media europea,
  Madrid: Istmo.
\item
  Thorp, R., \& Bertram, G. (1978). Peru 1890-1977: Growth and Policy in
  an Open Economy. Columbia University Press.
\item
  Banco Central de Reserva del Perú. (1984). \emph{Memoria Anual 1983}.
  Lima: BCRP.
\item
  ECLAC. (1985). \emph{Economic Survey of Latin America and the
  Caribbean 1984}. Santiago: United Nations.
\item
  Banco Central de Reserva del Perú. (1989). \emph{Memoria Anual 1988}.
  Lima: BCRP.
\item
  Sachs, J. (1989). \emph{Social Conflict and Populist Policies in Latin
  America}. NBER Working Paper No.~2897. Cambridge: National Bureau of
  Economic Research.
\item
  Webb, R., \& Fernández Baca, G. (1990). \emph{Perú en números 1990}.
  Lima: Instituto Cuánto.
\item
  DESCO. (1990). \emph{Resumen Semanal, n.° 577, 6-12 julio de 1990}.
  Lima: DESCO.
\item
  Guerra-García, G. (1999). \emph{Perú: Reformas estructurales y
  políticas de ajuste}. Lima: Instituto de Estudios Peruanos.
\item
  Williamson, J. (1990). \emph{What Washington Means by Policy Reform}.
  En J. Williamson (Ed.), \emph{Latin American Adjustment: How Much Has
  Happened?} (pp.~7-20). Washington, DC: Institute for International
  Economics.
\end{itemize}

\section{Publicaciones Similares}\label{publicaciones-similares}

Si te interesó este artículo, te recomendamos que explores otros blogs y
recursos relacionados que pueden ampliar tus conocimientos. Aquí te dejo
algunas sugerencias:

\begin{enumerate}
\def\labelenumi{\arabic{enumi}.}
\tightlist
\item
  \href{https://achalmaedison.netlify.app/blog/posts/2015-05-14-el-aborto/index.pdf}{\faIcon{file-pdf}}
  \href{https://achalmaedison.netlify.app/blog/posts/2015-05-14-el-aborto}{El
  Aborto}
\item
  \href{https://achalmaedison.netlify.app/blog/posts/2017-04-23-sitios-web-asombrosos/index.pdf}{\faIcon{file-pdf}}
  \href{https://achalmaedison.netlify.app/blog/posts/2017-04-23-sitios-web-asombrosos}{Sitios
  Web Asombrosos}
\item
  \href{https://achalmaedison.netlify.app/blog/posts/2017-05-23-el-mercantilismo/index.pdf}{\faIcon{file-pdf}}
  \href{https://achalmaedison.netlify.app/blog/posts/2017-05-23-el-mercantilismo}{El
  Mercantilismo}
\item
  \href{https://achalmaedison.netlify.app/blog/posts/2020-05-23-comandos-de-google-assistant/index.pdf}{\faIcon{file-pdf}}
  \href{https://achalmaedison.netlify.app/blog/posts/2020-05-23-comandos-de-google-assistant}{Comandos
  De Google Assistant}
\item
  \href{https://achalmaedison.netlify.app/blog/posts/2020-09-15-plan-de-negocio-exportacion-de-trucha-arcoires/index.pdf}{\faIcon{file-pdf}}
  \href{https://achalmaedison.netlify.app/blog/posts/2020-09-15-plan-de-negocio-exportacion-de-trucha-arcoires}{Plan
  De Negocio Exportacion De Trucha Arcoires}
\item
  \href{https://achalmaedison.netlify.app/blog/posts/2021-07-13-plan-de-negocio-exportacion-de-tuna/index.pdf}{\faIcon{file-pdf}}
  \href{https://achalmaedison.netlify.app/blog/posts/2021-07-13-plan-de-negocio-exportacion-de-tuna}{Plan
  De Negocio Exportacion De Tuna}
\item
  \href{https://achalmaedison.netlify.app/blog/posts/2021-07-14-comandos-de-blogdown/index.pdf}{\faIcon{file-pdf}}
  \href{https://achalmaedison.netlify.app/blog/posts/2021-07-14-comandos-de-blogdown}{Comandos
  De Blogdown}
\item
  \href{https://achalmaedison.netlify.app/blog/posts/2021-10-01-gestion-publica-y-administracion-publica/index.pdf}{\faIcon{file-pdf}}
  \href{https://achalmaedison.netlify.app/blog/posts/2021-10-01-gestion-publica-y-administracion-publica}{Gestion
  Publica Y Administracion Publica}
\item
  \href{https://achalmaedison.netlify.app/blog/posts/2021-10-01-reformas-y-modernizacion-de-la-gestion-publica/index.pdf}{\faIcon{file-pdf}}
  \href{https://achalmaedison.netlify.app/blog/posts/2021-10-01-reformas-y-modernizacion-de-la-gestion-publica}{Reformas
  Y Modernizacion De La Gestion Publica}
\item
  \href{https://achalmaedison.netlify.app/blog/posts/2022-01-23-cadena\%20de\%20suministros/index.pdf}{\faIcon{file-pdf}}
  \href{https://achalmaedison.netlify.app/blog/posts/2022-01-23-cadena\%20de\%20suministros}{Cadena
  De Suministros}
\item
  \href{https://achalmaedison.netlify.app/blog/posts/2022-04-22-economia-agraria/index.pdf}{\faIcon{file-pdf}}
  \href{https://achalmaedison.netlify.app/blog/posts/2022-04-22-economia-agraria}{Economia
  Agraria}
\item
  \href{https://achalmaedison.netlify.app/blog/posts/2022-06-02-impacto-del-cambio-climatico/index.pdf}{\faIcon{file-pdf}}
  \href{https://achalmaedison.netlify.app/blog/posts/2022-06-02-impacto-del-cambio-climatico}{Impacto
  Del Cambio Climatico}
\item
  \href{https://achalmaedison.netlify.app/blog/posts/2023-05-11-cualidades-de-los-servidores-publicos/index.pdf}{\faIcon{file-pdf}}
  \href{https://achalmaedison.netlify.app/blog/posts/2023-05-11-cualidades-de-los-servidores-publicos}{Cualidades
  De Los Servidores Publicos}
\item
  \href{https://achalmaedison.netlify.app/blog/posts/2023-05-12-la-economia-peruana-entre-1970-1990/index.pdf}{\faIcon{file-pdf}}
  \href{https://achalmaedison.netlify.app/blog/posts/2023-05-12-la-economia-peruana-entre-1970-1990}{La
  Economia Peruana Entre 1970 1990}
\item
  \href{https://achalmaedison.netlify.app/blog/posts/2023-05-16-economia-regional/index.pdf}{\faIcon{file-pdf}}
  \href{https://achalmaedison.netlify.app/blog/posts/2023-05-16-economia-regional}{Economia
  Regional}
\end{enumerate}

Esperamos que encuentres estas publicaciones igualmente interesantes y
útiles. ¡Disfruta de la lectura!






\end{document}
