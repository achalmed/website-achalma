\documentclass[
  stu,
  floatsintext,
  longtable,
  a4paper,
  nolmodern,
  notxfonts,
  notimes,
  colorlinks=true,linkcolor=blue,citecolor=blue,urlcolor=blue]{apa7}

\usepackage{amsmath}
\usepackage{amssymb}



\usepackage[bidi=default]{babel}
\babelprovide[main,import]{spanish}


\babelfont{rm}[,RawFeature={fallback=mainfontfallback}]{Latin Modern
Roman}
% get rid of language-specific shorthands (see #6817):
\let\LanguageShortHands\languageshorthands
\def\languageshorthands#1{}

\RequirePackage{longtable}
\RequirePackage{threeparttablex}

\makeatletter
\renewcommand{\paragraph}{\@startsection{paragraph}{4}{\parindent}%
	{0\baselineskip \@plus 0.2ex \@minus 0.2ex}%
	{-.5em}%
	{\normalfont\normalsize\bfseries\typesectitle}}

\renewcommand{\subparagraph}[1]{\@startsection{subparagraph}{5}{0.5em}%
	{0\baselineskip \@plus 0.2ex \@minus 0.2ex}%
	{-\z@\relax}%
	{\normalfont\normalsize\bfseries\itshape\hspace{\parindent}{#1}\textit{\addperi}}{\relax}}
\makeatother




\usepackage{longtable, booktabs, multirow, multicol, colortbl, hhline, caption, array, float, xpatch}
\usepackage{subcaption}


\renewcommand\thesubfigure{\Alph{subfigure}}
\setcounter{topnumber}{2}
\setcounter{bottomnumber}{2}
\setcounter{totalnumber}{4}
\renewcommand{\topfraction}{0.85}
\renewcommand{\bottomfraction}{0.85}
\renewcommand{\textfraction}{0.15}
\renewcommand{\floatpagefraction}{0.7}

\usepackage{tcolorbox}
\tcbuselibrary{listings,theorems, breakable, skins}
\usepackage{fontawesome5}

\definecolor{quarto-callout-color}{HTML}{909090}
\definecolor{quarto-callout-note-color}{HTML}{0758E5}
\definecolor{quarto-callout-important-color}{HTML}{CC1914}
\definecolor{quarto-callout-warning-color}{HTML}{EB9113}
\definecolor{quarto-callout-tip-color}{HTML}{00A047}
\definecolor{quarto-callout-caution-color}{HTML}{FC5300}
\definecolor{quarto-callout-color-frame}{HTML}{ACACAC}
\definecolor{quarto-callout-note-color-frame}{HTML}{4582EC}
\definecolor{quarto-callout-important-color-frame}{HTML}{D9534F}
\definecolor{quarto-callout-warning-color-frame}{HTML}{F0AD4E}
\definecolor{quarto-callout-tip-color-frame}{HTML}{02B875}
\definecolor{quarto-callout-caution-color-frame}{HTML}{FD7E14}

%\newlength\Oldarrayrulewidth
%\newlength\Oldtabcolsep


\usepackage{hyperref}




\providecommand{\tightlist}{%
  \setlength{\itemsep}{0pt}\setlength{\parskip}{0pt}}
\usepackage{longtable,booktabs,array}
\usepackage{calc} % for calculating minipage widths
% Correct order of tables after \paragraph or \subparagraph
\usepackage{etoolbox}
\makeatletter
\patchcmd\longtable{\par}{\if@noskipsec\mbox{}\fi\par}{}{}
\makeatother
% Allow footnotes in longtable head/foot
\IfFileExists{footnotehyper.sty}{\usepackage{footnotehyper}}{\usepackage{footnote}}
\makesavenoteenv{longtable}

\usepackage{graphicx}
\makeatletter
\newsavebox\pandoc@box
\newcommand*\pandocbounded[1]{% scales image to fit in text height/width
  \sbox\pandoc@box{#1}%
  \Gscale@div\@tempa{\textheight}{\dimexpr\ht\pandoc@box+\dp\pandoc@box\relax}%
  \Gscale@div\@tempb{\linewidth}{\wd\pandoc@box}%
  \ifdim\@tempb\p@<\@tempa\p@\let\@tempa\@tempb\fi% select the smaller of both
  \ifdim\@tempa\p@<\p@\scalebox{\@tempa}{\usebox\pandoc@box}%
  \else\usebox{\pandoc@box}%
  \fi%
}
% Set default figure placement to htbp
\def\fps@figure{htbp}
\makeatother







\usepackage{fontspec} 

\defaultfontfeatures{Scale=MatchLowercase}
\defaultfontfeatures[\rmfamily]{Ligatures=TeX,Scale=1}

  \setmainfont[,RawFeature={fallback=mainfontfallback}]{Latin Modern
Roman}




\title{Análisis del Conflicto Minero en Las Bambas: Impactos
Ambientales, Sociales y Estrategias de Resolución}


\shorttitle{Conflicto Las Bambas 2025}


\usepackage{etoolbox}


\course{Manejo de Conflictos}
\professor{Dr.~\ldots{}}
\duedate{06/18/2025}

\ccoppy{\textcopyright~2025}



\author{Edison Achalma}



\affiliation{
{Escuela Profesional de Economía, Universidad Nacional de San Cristóbal
de Huamanga}}




\leftheader{Achalma}

\date{2025-06-18}


\abstract{This study analyzes the Las Bambas mining conflict in
Apurímac, Peru, from 2015 to 2025, focusing on its socio-environmental
causes, negotiation strategies, and alternative resolution mechanisms.
The research identifies the lack of prior consultation, environmental
degradation, and social impacts, such as forced relocation and
criminalization of protests, as key conflict triggers. It evaluates the
effectiveness of tripartite dialogue tables, highlighting their limited
success due to mistrust and unfulfilled agreements. The study proposes
mediation, equitable arbitration, and inclusive consultation processes
to achieve sustainable solutions. Drawing on reports from Ojo Público
(2023) and CooperAcción (2025), the analysis underscores the need for
transparent governance and community participation to balance economic
development with environmental and social justice in Peru's mining
sector. }

\keywords{Mining conflict, socio-environmental impacts, prior
consultation, conflict resolution, Peru}

\authornote{\par{\addORCIDlink{Edison Achalma}{0000-0001-6996-3364}} 
\par{ }
\par{   El autor no tiene conflictos de interés que revelar.  A mi
familia y docentes, por su apoyo incondicional en mi proceso de
aprendizaje, y a las comunidades de Apurímac, cuya lucha inspira este
trabajo.  Los roles de autor se clasificaron utilizando la taxonomía de
roles de colaborador (CRediT; https://credit.niso.org/) de la siguiente
manera:  Edison Achalma:   conceptualización, redacción}
\par{La correspondencia relativa a este artículo debe dirigirse a Edison
Achalma, Email: \href{mailto:elmer.achalma.09@unsch.edu.pe}{elmer.achalma.09@unsch.edu.pe}}
}

\makeatletter
\let\endoldlt\endlongtable
\def\endlongtable{
\hline
\endoldlt
}
\makeatother

\urlstyle{same}



\makeatletter
\@ifpackageloaded{caption}{}{\usepackage{caption}}
\AtBeginDocument{%
\ifdefined\contentsname
  \renewcommand*\contentsname{Tabla de contenidos}
\else
  \newcommand\contentsname{Tabla de contenidos}
\fi
\ifdefined\listfigurename
  \renewcommand*\listfigurename{List of Figures}
\else
  \newcommand\listfigurename{List of Figures}
\fi
\ifdefined\listtablename
  \renewcommand*\listtablename{List of Tables}
\else
  \newcommand\listtablename{List of Tables}
\fi
\ifdefined\figurename
  \renewcommand*\figurename{Figura}
\else
  \newcommand\figurename{Figura}
\fi
\ifdefined\tablename
  \renewcommand*\tablename{Tabla}
\else
  \newcommand\tablename{Tabla}
\fi
}
\@ifpackageloaded{float}{}{\usepackage{float}}
\floatstyle{ruled}
\@ifundefined{c@chapter}{\newfloat{codelisting}{h}{lop}}{\newfloat{codelisting}{h}{lop}[chapter]}
\floatname{codelisting}{Listing}
\newcommand*\listoflistings{\listof{codelisting}{List of Listings}}
\makeatother
\makeatletter
\makeatother
\makeatletter
\@ifpackageloaded{caption}{}{\usepackage{caption}}
\@ifpackageloaded{subcaption}{}{\usepackage{subcaption}}
\makeatother

% From https://tex.stackexchange.com/a/645996/211326
%%% apa7 doesn't want to add appendix section titles in the toc
%%% let's make it do it
\makeatletter
\xpatchcmd{\appendix}
  {\par}
  {\addcontentsline{toc}{section}{\@currentlabelname}\par}
  {}{}
\makeatother

%% Disable longtable counter
%% https://tex.stackexchange.com/a/248395/211326

\usepackage{etoolbox}

\makeatletter
\patchcmd{\LT@caption}
  {\bgroup}
  {\bgroup\global\LTpatch@captiontrue}
  {}{}
\patchcmd{\longtable}
  {\par}
  {\par\global\LTpatch@captionfalse}
  {}{}
\apptocmd{\endlongtable}
  {\ifLTpatch@caption\else\addtocounter{table}{-1}\fi}
  {}{}
\newif\ifLTpatch@caption
\makeatother

\begin{document}

\maketitle

\hypertarget{toc}{}
\tableofcontents
\newpage
\section[Introduction]{Análisis del Conflicto Minero en Las Bambas}

\setcounter{secnumdepth}{5}

\setlength\LTleft{0pt}


El conflicto minero en Las Bambas, ubicado en la región de Apurímac,
Perú, constituye un caso emblemático de las tensiones socioambientales
derivadas de proyectos extractivos en el país. Desde el inicio de las
operaciones de la mina en 2015, gestionada por la empresa MMG Limited,
las comunidades locales, principalmente campesinas e indígenas, han
denunciado impactos ambientales significativos, como la contaminación de
fuentes hídricas y la degradación de tierras agrícolas, así como
impactos sociales, incluyendo la reubicación forzada y la
criminalización de líderes comunales. Estos factores han desencadenado
protestas, bloqueos del Corredor Minero Sur y enfrentamientos violentos,
con al menos cinco muertes reportadas entre 2015 y 2025. La falta de
consulta previa, conforme al Convenio 169 de la OIT, y la percepción de
incumplimientos por parte de la empresa y el Estado han exacerbado el
conflicto, evidenciando desafíos estructurales en la gobernanza minera
peruana.

La pregunta de investigación central que guía este estudio es: ¿Cómo han
influido la falta de consulta previa, la violencia en las protestas y el
rol del Estado en las estrategias de negociación y resolución del
conflicto en Las Bambas? Para responder a esta interrogante, el trabajo
se plantea tres objetivos principales: 1) analizar las causas del
conflicto, identificando los impactos ambientales y sociales como
desencadenantes clave; 2) evaluar la efectividad de las estrategias de
negociación empleadas por MMG, las comunidades y el Estado; y 3)
proponer mecanismos alternativos de resolución de conflictos, como
mediación y arbitraje, para promover soluciones sostenibles.

El estudio del conflicto de Las Bambas es importante desde la
perspectiva del manejo de conflictos por varias razones. En primer
lugar, el caso refleja los retos de compatibilizar el desarrollo
económico, basado en la explotación minera, con la protección de los
derechos de las comunidades y el medioambiente. En segundo lugar, ofrece
lecciones valiosas sobre la importancia de procesos participativos y
transparentes para prevenir la escalada de tensiones socioambientales.
Finalmente, el análisis de Las Bambas contribuye a la discusión sobre la
sostenibilidad de la minería en Perú, un país donde el sector representa
cerca del 60\% de las exportaciones, pero también es la principal fuente
de conflictos sociales, según la Defensoría del Pueblo (2025).

El trabajo de investigación se estructura en tres capítulos principales.
El primero describe el contexto del conflicto, incluyendo los
antecedentes históricos de la mina, los actores involucrados y los
impactos ambientales y sociales reportados. El segundo analiza las
dinámicas del conflicto, abordando la violencia en las protestas, la
falta de consulta previa y el rol del Estado como mediador. El tercero
evalúa las estrategias de negociación implementadas y propone mecanismos
alternativos de resolución, basándose en experiencias y recomendaciones
de expertos. Este análisis busca no solo comprender las causas y
dinámicas del conflicto, sino también contribuir con propuestas para una
gestión más equitativa y sostenible de los proyectos mineros en Perú.

\section{Contexto del Conflicto Minero en Las
Bambas}\label{contexto-del-conflicto-minero-en-las-bambas}

\subsection{Antecedentes históricos y descripción de la mina Las
Bambas}\label{antecedentes-histuxf3ricos-y-descripciuxf3n-de-la-mina-las-bambas}

El proyecto minero Las Bambas, ubicado en la región de Apurímac, Perú,
representa uno de los yacimientos de cobre más significativos del país,
contribuyendo aproximadamente al 20\% de la producción nacional de este
mineral. Según CooperAcción (2015), la exploración de la zona
mineralizada de Las Bambas comenzó en 1911, pero no fue hasta 2003
cuando el Estado peruano inició formalmente el proceso de licitación. En
este sentido, CooperAcción (2015) señala que, en agosto de 2004, la
empresa suiza Xstrata AG obtuvo la adjudicación del proyecto por 121
millones de dólares, abarcando los yacimientos de Chalcobamba,
Ferrobamba, Sulfobamba y Charcas, junto con concesiones mineras que
cubren aproximadamente 33,200 hectáreas.

Posteriormente, el proyecto experimentó cambios significativos en su
propiedad. CooperAcción (2015) y Wikipedia (2023) indican que, tras la
fusión de Xstrata con Glencore en 2013, el control pasó a
Glencore-Xstrata. Sin embargo, en abril de 2014, Glencore-Xstrata vendió
Las Bambas a un consorcio chino liderado por MMG Limited, junto con
Guoxin International Investment Corporation y CITIC Metal Co., por un
monto de 5,850 millones de dólares, según lo reportado por CooperAcción
(2015) y China y América Latina (2023). Este cambio marcó un hito en la
internacionalización del proyecto, consolidando la presencia de capital
chino en la minería peruana.

En cuanto a su desarrollo operativo, MMG Limited (2023) destaca que las
operaciones mineras iniciaron en noviembre de 2015, alcanzando la
producción comercial en julio de 2016. Para 2021, la mina producía
aproximadamente 400,000 toneladas de concentrado de cobre al año, según
Wikipedia (2023), lo que la posicionó como una de las minas de cobre más
grandes del mundo. Además, MMG Limited (2023) y CooperAcción (2023)
subrayan que la inversión total superó los 10,000 millones de dólares,
generando un impacto económico significativo al contribuir con cerca del
1\% del PBI nacional y regalías para Apurímac.

No obstante, el proyecto ha estado acompañado de conflictos sociales
desde sus inicios. Ojo Público (2023) señala que la reubicación de
comunidades, como Fuerabamba, fue un proceso clave para la construcción
de la mina, pero generó disputas por el incumplimiento de acuerdos y
problemas relacionados con la titularidad de tierras. En este contexto,
China y América Latina (2023) explica que la decisión de reemplazar el
mineroducto original por una carretera de transporte terrestre, tras la
adquisición por MMG, intensificó las tensiones, ya que requirió
modificaciones al Estudio de Impacto Ambiental (EIA) sin una adecuada
participación de las comunidades.

Finalmente, Ojo Público (2023) reporta que, entre 2020 y 2025, el
corredor minero de Las Bambas se convirtió en un foco de alta
conflictividad, con bloqueos recurrentes, ocupaciones de terrenos y
demandas de renegociación de acuerdos. Estos conflictos, que han
resultado en al menos cinco muertes y paros intermitentes, reflejan la
complejidad de las relaciones entre la empresa, las comunidades y el
Estado. MMG Limited (2023) indica que, en respuesta, la empresa ha
intensificado sus esfuerzos en sostenibilidad y diálogo social, aunque
las tensiones persisten hasta 2025, según Ojo Público (2023).

\subsection{Principales actores
involucrados}\label{principales-actores-involucrados}

El conflicto minero en Las Bambas, ubicado en la región de Apurímac,
Perú, involucra a múltiples actores con intereses y posiciones diversas
que han moldeado la dinámica de las tensiones entre 2015 y 2025. A
continuación, se describen los principales actores, sus intereses y
posturas, basándose en reportes confiables.

En primer lugar, las comunidades campesinas, como Fuerabamba, Chila,
Choaquere, Carmen Alto y las comunidades de Chumbivilcas, constituyen un
actor central en el conflicto. Según Ojo Público (2023), estas
comunidades exigen el cumplimiento de acuerdos previos, compensaciones
económicas justas, el saneamiento legal de sus tierras y el respeto a
sus territorios ancestrales. CooperAcción (2023a) añade que las
comunidades denuncian reiteradamente incumplimientos por parte de MMG
Limited en temas de reubicación, compensaciones y compromisos de
desarrollo social, lo que ha motivado protestas, bloqueos de vías,
ocupaciones de terrenos y paros. Por ejemplo, Ojo Público (2023) destaca
que las comunidades perciben abusos y falta de diálogo efectivo, lo que
ha intensificado sus acciones de resistencia frente a los impactos
ambientales y sociales del proyecto.

Por su parte, la empresa minera MMG Limited, liderada por un consorcio
chino-australiano, busca garantizar la continuidad operativa y la
seguridad jurídica de sus operaciones. Business \& Human Rights Resource
Centre (2023) señala que MMG defiende la legalidad de sus adquisiciones
y modificaciones al proyecto, como el cambio del mineroducto por
transporte terrestre, que generó nuevas tensiones al requerir ajustes al
Estudio de Impacto Ambiental (EIA). Asimismo, Ojo Público (2023) indica
que la empresa sostiene que cumple con los acuerdos establecidos,
argumentando que las paralizaciones generan pérdidas económicas
significativas para la región y el país. Sin embargo, CooperAcción
(2023a) y CBC (2023) critican a MMG por supuestos incumplimientos y por
recurrir a la fuerza pública para resolver disputas, lo que ha
deteriorado su relación con las comunidades.

El Estado peruano, representado por el gobierno central, el Ministerio
de Energía y Minas y gobiernos regionales, desempeña un rol estratégico
en el conflicto. Business \& Human Rights Resource Centre (2023) destaca
que el Estado prioriza la estabilidad económica y la continuidad del
proyecto, considerado clave para el desarrollo nacional debido a su
aporte al PBI y las regalías. No obstante, CooperAcción (2018) señala
que los intentos de mediación, como las mesas de diálogo y la
declaratoria de zonas de atención especial, han tenido resultados
limitados y han sido criticados por su falta de eficacia. Además, CBC
(2023) y CooperAcción (2018) coinciden en que el Estado ha sido
percibido como parcial hacia los intereses empresariales, especialmente
por el uso de la fuerza pública para desalojar a comuneros, lo que ha
generado desconfianza entre las comunidades.

Los trabajadores y sindicatos de la mina también son actores relevantes,
aunque su rol es menos prominente en comparación con las comunidades y
la empresa. Según CooperAcción (2023b), este grupo demanda mejores
condiciones laborales y el cumplimiento de acuerdos colectivos, lo que
ha llevado a paros y protestas que añaden una dimensión laboral al
conflicto. Estas acciones, aunque puntuales, reflejan tensiones internas
en la operación minera que complican la resolución integral del
conflicto.

Finalmente, las organizaciones de la sociedad civil y ONG, como
CooperAcción y otras agrupaciones, apoyan la defensa de los derechos de
las comunidades y promueven una mayor transparencia en el proyecto.
CooperAcción (2018) subraya que estas organizaciones denuncian los
impactos ambientales y sociales de Las Bambas, exigiendo responsabilidad
social empresarial y reformas en el marco legal para mejorar la
gobernanza minera. De manera similar, Ojo Público (2023) destaca su rol
en el acompañamiento a las comunidades y en la incidencia para prevenir
conflictos, lo que las posiciona como mediadoras críticas en el
escenario.

\subsection{Impactos ambientales y sociales del proyecto
minero}\label{impactos-ambientales-y-sociales-del-proyecto-minero}

En el ámbito ambiental, uno de los principales impactos ha sido la
contaminación del aire y el ruido causados por el transporte terrestre
de minerales. Convoca (2023) destaca que la modificación del Estudio de
Impacto Ambiental (EIA) en 2014, que reemplazó el mineroducto por una
carretera, resultó en el tránsito de hasta 370 camiones diarios a lo
largo del Corredor Minero Sur, afectando a 169 comunidades en Apurímac,
Cusco y Arequipa. Este tráfico ha generado polvo, emisiones y ruido,
deteriorando la calidad de vida y la salud de las poblaciones locales.
Además, Red Muqui (2023) señala que comunidades como Huancuire han
denunciado la contaminación de fuentes hídricas, como manantiales y
puquiales, debido a las actividades mineras, afectando a unas 500
familias que dependen de estos recursos para la agricultura y ganadería.
Wayka (2024) corrobora estas denuncias, reportando contaminación de
cuerpos de agua en Pichaqani, cerca del tajo Chalcobamba, lo que ha
motivado protestas pacíficas en la comunidad de Pumamarca.

Otro impacto ambiental significativo es la degradación del suelo y la
pérdida de tierras agrícolas. Ojo Público (2023) indica que el uso
intensivo de terrenos para la expansión minera y el transporte ha
reducido la disponibilidad de tierras productivas, especialmente en
áreas reubicadas como Nueva Fuerabamba y el predio Sallawi. Asimismo,
Pata Amarilla (2021) subraya que los deslizamientos de tierras y los
pasivos ambientales han afectado la ganadería y el acceso al agua
limpia, exacerbando la vulnerabilidad de las comunidades. El Ministerio
de Energía y Minas (2023) reconoce estos impactos en los EIA aprobados,
pero las comunidades cuestionan la eficacia de las medidas de mitigación
propuestas.

En el ámbito social, la reubicación de comunidades ha sido un factor
clave de conflicto. Ojo Público (2023) reporta que la construcción de
Las Bambas implicó el traslado de comunidades como Fuerabamba a Nueva
Fuerabamba, pero los incumplimientos en los acuerdos de reubicación,
compensaciones y acceso a servicios básicos han generado tensiones. Por
ejemplo, Red Muqui (2023) señala que las comunidades han ocupado
terrenos en respuesta a promesas no cumplidas, como el saneamiento legal
de tierras y el desarrollo de infraestructura. Estas disputas han
derivado en protestas y bloqueos, especialmente en el Corredor Minero,
donde las comunidades exigen compensaciones por los impactos del
transporte.

La criminalización de la protesta social es otro impacto social
relevante. Infobae (2025) documenta procesos judiciales contra líderes
ambientales y comunales, acusados de disturbios y daños agravados, lo
que organizaciones de derechos humanos han calificado como una
estrategia para desmovilizar la resistencia. Wayka (2024) añade que, en
2024, la represión policial contra manifestantes en Coyllurqui,
Cotabambas, intensificó las tensiones, con denuncias de violencia contra
comuneros que exigían soluciones a la contaminación hídrica.

Aunque el proyecto ha traído beneficios económicos, como un aumento del
53,1\% en la producción de cobre en Apurímac en 2025, según Rumbo Minero
(2025), persiste la percepción de una distribución desigual de estos
beneficios. Desde Adentro (2021) señala que, a pesar del crecimiento
económico y la generación de empleo, muchas comunidades continúan
demandando mayor inversión en infraestructura, saneamiento y proyectos
productivos. Esta desigualdad ha alimentado la desconfianza hacia MMG
Limited y el Estado, consolidando el conflicto como un desafío
estructural.

\section{Dinámicas del Conflicto}\label{dinuxe1micas-del-conflicto}

\subsection{Manifestaciones de violencia en las
protestas}\label{manifestaciones-de-violencia-en-las-protestas}

El conflicto minero de Las Bambas ha estado marcado por episodios
significativos de violencia entre 2015 y 2025, caracterizados por
bloqueos de carreteras y enfrentamientos con la Policía Nacional del
Perú (PNP). Estos eventos reflejan tensiones derivadas de disputas
socioambientales y la percepción de incumplimientos por parte de la
empresa MMG Limited y el Estado peruano. A continuación, se detallan los
principales episodios de violencia y los factores que desencadenaron su
escalada, basados en reportes confiables.

\subsubsection{Principales episodios de violencia
(2015-2025)}\label{principales-episodios-de-violencia-2015-2025}

\textbf{1. Enfrentamientos de septiembre de 2015 en Cotabambas y Grau}

En septiembre de 2015, cientos de pobladores de las provincias de
Cotabambas y Grau iniciaron protestas masivas contra MMG Las Bambas.
Según Actualidad Ambiental (2015), estas acciones fueron motivadas por
la aprobación de un Estudio de Impacto Ambiental (EIA) complementario
para una planta de molibdeno y un almacén de concentrados, percibidos
como contaminantes, sin consulta adecuada a las comunidades. El 28 de
septiembre, El País (2015) reporta que los enfrentamientos en Fuerabamba
entre manifestantes y la PNP resultaron en cuatro muertes y 15 heridos,
con el uso de gas lacrimógeno y disparos por parte de las fuerzas
policiales. En consecuencia, Infobae (2025) señala que se declaró un
estado de emergencia por 30 días en Apurímac y Cusco, con el despliegue
de 1,500 efectivos policiales. Además, Mongabay (2025) indica que 11
líderes comunales fueron procesados por delitos contra la seguridad
pública y el patrimonio, aunque fueron absueltos en 2025, evidenciando
el uso indebido del sistema judicial para reprimir la protesta social.

\textbf{2. Bloqueos y enfrentamientos en Chumbivilcas, 2021}

En 2021, las comunidades de Chumbivilcas, Cusco, intensificaron los
bloqueos del Corredor Minero Sur. Según Ojo Público (2021), estas
protestas exigían compensaciones por los impactos del transporte de
minerales, que generaba polvo y vibraciones en más de 150 comunidades.
El 20 de octubre, un enfrentamiento en el sector de Sayhua dejó al menos
10 heridos, entre comuneros y policías, tras una intervención policial
para despejar la vía, como reporta la Defensoría del Pueblo (2021). Este
episodio fue desencadenado por la falta de acuerdos claros sobre
compensaciones económicas y la percepción de incumplimientos en
compromisos de desarrollo local por parte de MMG, según CooperAcción
(2021).

\textbf{3. Protestas violentas en Coyllurqui, 2024}

En julio de 2024, la comunidad de Coyllurqui, Cotabambas, inició un paro
indefinido contra MMG Las Bambas. Wayka (2024) explica que los comuneros
denunciaban la contaminación de fuentes hídricas por actividades
mineras, exigiendo soluciones inmediatas. La intervención policial, que
incluyó bombas lacrimógenas, resultó en al menos cinco heridos, según
Epicentro TV (2024). Videos difundidos en redes sociales, reportados por
Wayka (2024), mostraron la detención violenta de una lideresa, Fermina
Pandia Laura, lo que intensificó las denuncias de represión. Swissinfo
(2024) añade que la falta de consulta previa sobre proyectos mineros
específicos fue un factor clave en la escalada de este conflicto.

\textbf{4. Absolución de defensores en 2025}

En abril de 2025, la Corte Superior de Apurímac absolvió a 11 defensores
comunales procesados por las protestas de 2015, quienes habían sido
sentenciados a ocho y nueve años de prisión. Infobae (2025) destaca que
este fallo estableció un precedente contra la criminalización de la
protesta, aunque las comunidades denunciaron que persisten riesgos
legales por otros procesos judiciales, según CooperAcción (2025). Este
evento, aunque no violento, refleja la violencia estructural ejercida a
través de la judicialización de líderes sociales.

\subsubsection{Factores desencadenantes de la escalada
violenta}\label{factores-desencadenantes-de-la-escalada-violenta}

\begin{enumerate}
\def\labelenumi{\arabic{enumi}.}
\tightlist
\item
  \textbf{Modificaciones al EIA sin consulta previa}: Según OCMAL
  (2023), la aprobación de cambios en el EIA, como la construcción de
  nuevas instalaciones sin participación comunitaria, generó
  desconfianza y rechazo, al percibirse como una amenaza al
  medioambiente y la salud.
\item
  \textbf{Falta de diálogo efectivo}: La Defensoría del Pueblo (2021)
  señala que la ausencia de espacios de diálogo inclusivos y
  transparentes entre MMG, el Estado y las comunidades exacerbó las
  tensiones, con demandas comunitarias desatendidas.
\item
  \textbf{Uso desproporcionado de la fuerza policial}: Amnistía
  Internacional (2024) denuncia el empleo de tácticas agresivas, como
  gas lacrimógeno y disparos, en las intervenciones policiales, lo que
  intensificó los enfrentamientos y generó acusaciones de represión.
\item
  \textbf{Percepción de incumplimientos}: CooperAcción (2021) indica que
  las comunidades denunciaron reiteradamente el incumplimiento de
  acuerdos sobre compensaciones y desarrollo local, alimentando la
  frustración que motivó las protestas.
\item
  \textbf{Criminalización de la protesta}: Mongabay (2025) destaca que
  la judicialización de líderes comunales, como en el caso de 2015, fue
  percibida como una estrategia para desmovilizar la resistencia,
  incrementando la polarización.
\end{enumerate}

\subsubsection{Resumen cronológico de episodios
violentos}\label{resumen-cronoluxf3gico-de-episodios-violentos}

\begin{longtable}[]{@{}
  >{\raggedright\arraybackslash}p{(\linewidth - 6\tabcolsep) * \real{0.0315}}
  >{\raggedright\arraybackslash}p{(\linewidth - 6\tabcolsep) * \real{0.2598}}
  >{\raggedright\arraybackslash}p{(\linewidth - 6\tabcolsep) * \real{0.3386}}
  >{\raggedright\arraybackslash}p{(\linewidth - 6\tabcolsep) * \real{0.3701}}@{}}
\toprule\noalign{}
\begin{minipage}[b]{\linewidth}\raggedright
Año
\end{minipage} & \begin{minipage}[b]{\linewidth}\raggedright
Episodio
\end{minipage} & \begin{minipage}[b]{\linewidth}\raggedright
Consecuencias
\end{minipage} & \begin{minipage}[b]{\linewidth}\raggedright
Factores desencadenantes
\end{minipage} \\
\midrule\noalign{}
\endhead
\bottomrule\noalign{}
\endlastfoot
2015 & Enfrentamientos en Fuerabamba & 4 muertos, 15 heridos, estado de
emergencia & EIA sin consulta, percepción de contaminación \\
2021 & Bloqueos en Chumbivilcas (Sayhua) & 10 heridos, tensión en
Corredor Minero & Falta de compensaciones, impacto del transporte \\
2024 & Protestas en Coyllurqui & 5 heridos, represión policial &
Contaminación hídrica, falta de consulta previa \\
2025 & Absolución de 11 defensores & Precedente contra criminalización &
Judicialización como represión estructural \\
\end{longtable}

\subsection{Falta de consulta previa y sus
implicaciones}\label{falta-de-consulta-previa-y-sus-implicaciones}

La ausencia o insuficiencia de procesos de consulta previa, libre e
informada, conforme al Convenio 169 de la Organización Internacional del
Trabajo (OIT), ha sido un factor determinante en la escalada del
conflicto minero de Las Bambas en Apurímac, Perú, entre 2015 y 2025.
Esta omisión ha generado desconfianza, protestas y una percepción de
vulneración de derechos entre las comunidades afectadas, intensificando
las tensiones con la empresa MMG Limited y el Estado peruano.

En primer lugar, la falta de consulta previa constituye una violación
directa de los derechos de los pueblos indígenas y campesinos. Según el
Instituto de Defensa Legal (2019), las comunidades de Chumbivilcas y
Apurímac han denunciado que no se han llevado a cabo procesos de
consulta adecuados, como lo exige el artículo 6 del Convenio 169 de la
OIT, que obliga al Estado a consultar a los pueblos antes de aprobar
proyectos que afecten sus territorios o modos de vida. Por ejemplo, el
Instituto de Defensa Legal (2023) señala que decisiones como la
reclasificación de vías para el transporte de minerales o las
modificaciones al Estudio de Impacto Ambiental (EIA) se implementaron
sin diálogo ni consentimiento, lo que ha sido percibido como una
imposición unilateral.

Además, la consulta previa requiere un diálogo intercultural de buena fe
orientado a obtener el consentimiento de las comunidades, especialmente
ante impactos significativos. Sin embargo, Red Muqui (2019) destaca que
en Las Bambas este proceso ha sido inexistente o deficiente, lo que ha
profundizado la desconfianza hacia MMG y el Estado. En este sentido, el
Ministerio de Energía y Minas (2020) documenta que la Comunidad
Campesina de Huancuire solicitó formalmente suspender actividades
mineras hasta garantizar una consulta adecuada, evidenciando la falta de
participación efectiva y el incumplimiento de estándares
internacionales.

La ausencia de consulta previa ha tenido un impacto directo en la
conflictividad social. Según Dialnet (2022), la percepción de que sus
derechos territoriales y culturales son vulnerados sin su participación
ha motivado protestas, bloqueos de carreteras y enfrentamientos en la
región, ya que las comunidades consideran que el proyecto avanza sin
respetar sus prioridades. Asimismo, DPLF (2016) subraya que la omisión
de la consulta previa no solo agrava los conflictos, sino que también
dificulta la resolución pacífica al generar un sentimiento de exclusión
entre las comunidades afectadas.

\subsubsection{Casos específicos documentados
(2015-2025)}\label{casos-especuxedficos-documentados-2015-2025}

\textbf{1. Reclamo por consulta de vías de transporte en Chumbivilcas
(2019)} El Instituto de Defensa Legal (2019) reporta que las comunidades
de Chumbivilcas exigieron la consulta previa para la reclasificación de
vías utilizadas por Las Bambas, argumentando que estas decisiones
afectaban sus territorios y derechos colectivos. La falta de este
proceso fue denunciada como una violación del Convenio 169,
incrementando las tensiones y motivando bloqueos del Corredor Minero
Sur.

\textbf{2. Solicitud de suspensión de actividades por Huancuire
(2019-2020)} El Ministerio de Energía y Minas (2020) indica que la
Comunidad Campesina de Huancuire pidió al Estado detener las actividades
relacionadas con el tajo Chalcobamba hasta que se realizara una consulta
previa que garantizara un diálogo intercultural efectivo. Esta solicitud
reflejó la frustración de la comunidad ante la falta de participación en
decisiones clave del proyecto.

\textbf{3. Modificaciones al EIA sin consulta adecuada} Dialnet (2022)
documenta que las modificaciones al EIA, como la sustitución del
mineroducto por transporte terrestre y la autorización de nuevas
infraestructuras, se aprobaron sin procesos de consulta previa o con
procedimientos insuficientes. Estas decisiones, percibidas como
inconsultas, han sido un factor recurrente en las protestas y la
escalada del conflicto.

\textbf{4. Informe alternativo sobre consulta previa en Perú (2019)} Red
Muqui (2019) señala en un informe alternativo que, a pesar de la
ratificación del Convenio 169 por Perú en 1991, proyectos extractivos
como Las Bambas han omitido consultas previas adecuadas. Además, el
informe destaca que, cuando las comunidades solicitan este derecho, el
Estado, particularmente el Ministerio de Energía y Minas y el
Viceministerio de Interculturalidad, suele rechazar o postergar los
procesos, lo que agrava la conflictividad.

\subsection{Rol del Estado en la mediación del
conflicto}\label{rol-del-estado-en-la-mediaciuxf3n-del-conflicto}

El Estado peruano, a través de instituciones como el Ministerio de
Energía y Minas (MINEM) y la Defensoría del Pueblo, ha desempeñado un
rol complejo en la mediación del conflicto minero de Las Bambas entre
2015 y 2025. Si bien ha promovido espacios de diálogo y monitoreo de
derechos, su intervención ha enfrentado críticas por su parcialidad,
falta de prevención y uso de medidas represivas, lo que ha limitado su
efectividad como mediador.

En primer lugar, el Ministerio de Energía y Minas ha actuado como
facilitador de mesas de diálogo para abordar las demandas de las
comunidades afectadas por el proyecto Las Bambas. Según Deutsche Welle
(2022), en 2022, durante el gobierno de Pedro Castillo, el MINEM convocó
una mesa de diálogo con las comunidades de Fuerabamba y Huancuire y MMG
Limited para resolver reclamos relacionados con incumplimientos de
acuerdos sociales. Asimismo, Ojo Público (2023) reporta que en 2019,
bajo el gobierno de Martín Vizcarra, se estableció una mesa de diálogo
que incluyó compromisos de inversión social y saneamiento de tierras,
aunque muchos de estos acuerdos no se cumplieron, generando mayor
desconfianza. Estas iniciativas reflejan un esfuerzo formal del MINEM
por mediar, pero su impacto ha sido cuestionado debido a la falta de
seguimiento y resultados concretos.

Por su parte, la Defensoría del Pueblo ha desempeñado un rol clave como
observadora y defensora de los derechos humanos. La Defensoría del
Pueblo (2025) ha documentado en sus reportes mensuales las
manifestaciones, bloqueos y demandas de las comunidades, participando
activamente en mesas de diálogo para facilitar acuerdos. Además,
CooperAcción (2025) destaca que la Defensoría ha denunciado la
criminalización de líderes sociales y defensores ambientales, instando
al Estado a garantizar un diálogo intercultural y el respeto a los
derechos colectivos. Este rol de monitoreo y acompañamiento ha sido una
fortaleza, aunque su capacidad para influir en decisiones estatales ha
sido limitada.

Sin embargo, la respuesta estatal también ha incluido medidas represivas
que han agravado el conflicto. RFI (2015) señala que, en septiembre de
2015, el gobierno de Ollanta Humala declaró un estado de emergencia en
seis provincias del sureste, movilizando a la policía y las fuerzas
armadas para controlar las protestas, lo que resultó en enfrentamientos
con cuatro muertos y 15 heridos. Esta intervención, según APRODEH
(2022), fue criticada por organizaciones de derechos humanos y
comunidades, que denunciaron abuso de poder y criminalización de la
protesta social. La percepción de que el Estado prioriza los intereses
de MMG sobre los de las comunidades ha sido un obstáculo recurrente en
su rol mediador.

Las limitaciones del Estado en la mediación del conflicto son evidentes
en su enfoque reactivo y falta de políticas preventivas. Conflictos
Mineros (2021) reporta que alcaldes de Apurímac han criticado la
ausencia de estrategias claras para prevenir conflictos en zonas
mineras, señalando que el Estado actúa solo cuando las tensiones
escalan. Además, Infobae (2022) indica que la confidencialidad de
algunos acuerdos entre MMG y las comunidades, sin un acompañamiento
estatal transparente, ha generado desconfianza y dificultado la
resolución del conflicto. Estas críticas reflejan un rol estatal
percibido como parcial y poco efectivo en garantizar una mediación
equitativa.

\section{Estrategias de Negociación y Resolución
Alternativa}\label{estrategias-de-negociaciuxf3n-y-resoluciuxf3n-alternativa}

\subsection{Mecanismos de diálogo
implementados}\label{mecanismos-de-diuxe1logo-implementados}

Entre 2015 y 2025, el conflicto minero de Las Bambas en Apurímac, Perú,
ha motivado la implementación de diversos mecanismos de diálogo para
abordar las tensiones entre las comunidades, la empresa MMG Limited y el
Estado peruano. Estos esfuerzos, liderados principalmente por mesas de
diálogo y negociaciones directas, han tenido resultados mixtos, con
avances parciales y limitaciones significativas, según reportes
oficiales y análisis de organizaciones especializadas.

En primer lugar, las mesas de diálogo tripartitas han sido el mecanismo
más recurrente. Actualidad Ambiental (2019) señala que estas mesas han
reunido a representantes de comunidades como Fuerabamba y Cotabambas,
MMG Limited y el Estado, a través de entidades como el Ministerio de
Energía y Minas (MINEM) y la Presidencia del Consejo de Ministros. Por
ejemplo, APRODEH (2022) reporta que, en 2019, el entonces primer
ministro Salvador del Solar participó en reuniones para negociar
soluciones a los bloqueos del Corredor Minero Sur, mostrando un
compromiso formal del Ejecutivo. Estas mesas se han centrado en temas
como inversión social, saneamiento de tierras y derechos humanos, con la
mediación de actores como la Defensoría del Pueblo y la Conferencia
Episcopal.

Además, la Defensoría del Pueblo ha desempeñado un rol activo como
facilitadora y observadora. La Defensoría del Pueblo (2024) destaca que
ha participado en múltiples espacios de diálogo, promoviendo un enfoque
intercultural y documentando acuerdos para garantizar el respeto a los
derechos de las comunidades. De manera similar, CooperAcción (2016)
subraya que la Conferencia Episcopal ha actuado como mediadora en
algunos encuentros, buscando generar confianza entre las partes. Estos
actores externos han aportado legitimidad al proceso, aunque no siempre
han logrado superar las posiciones rígidas de los involucrados.

Por otro lado, se han establecido diálogos sectoriales y técnicos para
abordar aspectos específicos del conflicto. CooperAcción (2016) indica
que las mesas han incluido ejes temáticos como responsabilidad social,
desarrollo sostenible y derechos humanos, pero han enfrentado desafíos
operativos. Entre estos, Red Muqui (2020) menciona la falta de recursos
para la participación de representantes comunitarios, la exclusión de
algunos actores relevantes y la inflexibilidad de MMG y el Estado en la
negociación de demandas clave, lo que ha limitado los avances.

El Ejecutivo ha desplegado acciones directas para desbloquear el
conflicto. Actualidad Ambiental (2019) reporta que, en 2019, tres
ministros fueron enviados a Apurímac y Cusco para negociar con
comunidades que mantenían bloqueos prolongados, mientras que
CooperAcción (2023) señala que se intentó declarar el Corredor Minero
como vía nacional para facilitar su mantenimiento y saneamiento legal.
Sin embargo, El Aporte de Las Bambas (2023) destaca que estas medidas
enfrentaron resistencia comunitaria y restricciones presupuestales,
reduciendo su efectividad.

A pesar de estos esfuerzos, los resultados han sido limitados. La
Defensoría del Pueblo (2024) documenta que, aunque las mesas han
generado acuerdos parciales, como compromisos de inversión social, la
persistencia de bloqueos y protestas refleja una falta de confianza en
los procesos. Además, Red Muqui (2020) señala que la criminalización de
líderes comunales, como los procesados en 2015, ha fracturado el
diálogo, dificultando acuerdos sostenibles. CooperAcción (2016)
recomienda establecer una presidencia colegiada para las mesas, mayor
transparencia en los acuerdos y una evaluación constante del
cumplimiento de compromisos para mejorar la efectividad del diálogo.

\subsection{Evaluación de estrategias de
negociación}\label{evaluaciuxf3n-de-estrategias-de-negociaciuxf3n}

Las estrategias de negociación empleadas por MMG Limited y las
comunidades afectadas por el proyecto minero Las Bambas en Apurímac,
Perú, entre 2015 y 2025, han sido variadas, pero han enfrentado desafíos
significativos que han limitado su efectividad. A continuación, se
evalúan estas estrategias, destacando sus resultados y las limitaciones
identificadas por expertos, con énfasis en el estilo de negociación de
MMG que ha sido señalado como un factor que fomenta protestas.

En primer lugar, MMG ha participado activamente en mesas de diálogo
tripartitas y subgrupos de trabajo para abordar demandas comunitarias.
Rumbo Minero (2022) reporta que, en 2022, la empresa se involucró en
subgrupos con comunidades como Chila y Choaquere para revisar acuerdos
pendientes relacionados con saneamiento de tierras y desarrollo social.
Sin embargo, Ojo Público (2023) señala que las comunidades han
denunciado la falta de avances concretos en estos compromisos, como la
ejecución de obras de infraestructura y compensaciones económicas, lo
que ha generado percepción de incumplimiento y estancamiento en las
negociaciones.

Por su parte, las comunidades han centrado sus estrategias de
negociación en demandas específicas, como la participación económica en
la operación minera. Ojo Público (2022) destaca que un punto clave ha
sido la solicitud de las comunidades para proveer servicios de
transporte de minerales, exigiendo que MMG contrate un mayor número de
vehículos locales, como camionetas y camiones encapsulados. No obstante,
Infobae (2022) indica que MMG ha ofrecido montos significativamente
menores a los esperados por las comunidades, lo que ha resultado en un
punto muerto en las negociaciones y ha motivado bloqueos del Corredor
Minero Sur como medida de presión.

Otro aspecto relevante de la negociación ha sido la gestión de acuerdos
relacionados con la compra de tierras y la reubicación de comunidades.
Según SENACE (2023), MMG adquirió terrenos para reubicar a comunidades
como Fuerabamba, pero Ojo Público (2023) subraya que las comunidades han
reclamado pagos pendientes, saneamiento legal de tierras y el
reconocimiento de zonas de influencia directa, lo que ha generado
tensiones constantes. Esta falta de cumplimiento ha sido un obstáculo
para avanzar hacia acuerdos sostenibles, alimentando la desconfianza
hacia la empresa.

Las limitaciones de estas estrategias han sido ampliamente analizadas
por expertos. CooperAcción (2025) señala que la percepción de
incumplimiento por parte de MMG, especialmente en acuerdos firmados hace
más de una década, ha erosionado la confianza de las comunidades,
dificultando un diálogo constructivo. Además, Ojo Público (2022) critica
el estilo de negociación de MMG, que ha sido acusado de ofrecer
propuestas económicas insuficientes y de intentar dividir a las
comunidades mediante incentivos selectivos, lo que ha exacerbado las
protestas en lugar de mitigarlas. Esta práctica, según Otra Mirada
(2023), ha sido vista como una estrategia para debilitar la cohesión
comunitaria, generando mayor resistencia.

Asimismo, la falta de transparencia y comunicación efectiva ha sido un
obstáculo significativo. CooperAcción (2025) destaca que la negociación
se ha visto afectada por la poca claridad en las cifras, compromisos y
plazos, así como por la exclusión de algunos actores comunitarios en las
mesas de diálogo. Esta falta de inclusión ha generado divisiones
internas entre las comunidades, complicando el proceso. Por otro lado,
Infobae (2022) menciona que la criminalización de líderes sociales, como
los procesados por protestas, ha creado un clima de confrontación que ha
dificultado el establecimiento de un diálogo de buena fe, incluso tras
la absolución de algunos líderes en 2025.

MMG ha argumentado que las demandas comunitarias son, en ocasiones,
desproporcionadas y que la intervención de agentes externos complica las
negociaciones. Sin embargo, Actualidad Ambiental (2019) señala que
expertos consideran estas afirmaciones como parte de una narrativa
empresarial para justificar la falta de avances, en lugar de abordar las
demandas legítimas de las comunidades. Este enfoque ha contribuido a la
percepción de que el estilo de negociación de MMG prioriza sus intereses
económicos sobre un compromiso genuino con las comunidades, fomentando
un ciclo de protestas y tensiones.

\subsection{Mecanismos alternativos de resolución de
conflictos}\label{mecanismos-alternativos-de-resoluciuxf3n-de-conflictos}

En el contexto del conflicto minero de Las Bambas y otros conflictos
mineros similares en Perú, los mecanismos alternativos de resolución de
conflictos (MARC), como la mediación, el arbitraje de conciencia y las
mesas de concertación, han sido propuestos y, en algunos casos,
aplicados para superar las limitaciones de los enfoques judiciales y la
confrontación directa. A continuación, se describen estos mecanismos y
sus ventajas, basándose en la literatura especializada y experiencias
relevantes entre 2015 y 2025.

En primer lugar, la mediación ha sido ampliamente recomendada como un
mecanismo para facilitar el diálogo en conflictos mineros. Según el
Centro de Análisis y Resolución de Conflictos de la PUCP (2023), la
mediación implica la intervención de un tercero neutral que promueve la
comunicación entre las partes, como comunidades, empresas y el Estado,
para alcanzar acuerdos consensuados. Un ejemplo práctico es el ``Proceso
de Mediación Corredor Minero'', documentado por Agnitio (2023), donde se
aplicaron técnicas de mediación para abordar disputas relacionadas con
el transporte de minerales en el Corredor Minero Sur. Este enfoque
permite un entendimiento intercultural, reduce costos y tiempos en
comparación con procesos judiciales y fomenta acuerdos voluntarios que
pueden ser más duraderos.

Otro mecanismo innovador es el arbitraje de conciencia, propuesto como
una alternativa equitativa para resolver conflictos mineros. La
Universidad de Lima (2023) explica que este proceso se basa en
principios de justicia y equidad, en lugar de normas legales estrictas,
permitiendo que un tribunal arbitral emita una decisión equilibrada en
un plazo máximo de cinco días. Este mecanismo busca garantizar que
ninguna parte pierda, promoviendo la continuidad de las operaciones
mineras mientras se respetan las demandas comunitarias. La Cámara Minera
del Perú (2023a) destaca su rapidez, transparencia y aceptación por
parte de comunidades y empresas, aunque requiere capacitación previa y
voluntad política para su implementación efectiva.

Además, la conciliación y las mesas de concertación han sido utilizadas
para promover acuerdos inclusivos. La Cámara Minera del Perú (2023b)
señala que estas herramientas fomentan la participación activa de todos
los actores, incluyendo comunidades, empresas, el Estado y la sociedad
civil, con el objetivo de construir soluciones conjuntas. Por ejemplo,
el Consorcio de Investigación Económica y Social (2016) menciona que las
mesas de concertación han permitido abordar temas como el desarrollo
sostenible y la responsabilidad social en conflictos mineros, mejorando
las relaciones sociales y el capital social. Estas iniciativas requieren
un seguimiento continuo para asegurar el cumplimiento de los acuerdos
alcanzados.

\subsubsection{Ventajas de los mecanismos
alternativos}\label{ventajas-de-los-mecanismos-alternativos}

\begin{enumerate}
\def\labelenumi{\arabic{enumi}.}
\item
  \textbf{Mediación}:

  El Centro de Análisis y Resolución de Conflictos de la PUCP (2023)
  subraya que la mediación facilita la comunicación intercultural,
  promoviendo acuerdos que respetan las cosmovisiones de las comunidades
  indígenas y campesinas. Además, reduce significativamente los costos y
  tiempos en comparación con los procesos judiciales, aumentando la
  probabilidad de soluciones sostenibles.
\item
  \textbf{Arbitraje de conciencia}:

  La Universidad de Lima (2023) destaca que este mecanismo ofrece
  resoluciones rápidas y transparentes, basadas en criterios de equidad
  que son más accesibles para las comunidades. Su capacidad para
  minimizar la politización y evitar paralizaciones prolongadas lo
  convierte en una herramienta atractiva para conflictos como el de Las
  Bambas.
\item
  \textbf{Conciliación y mesas de concertación}:

  La Cámara Minera del Perú (2023b) indica que estas estrategias
  promueven la inclusión de todos los actores, fortaleciendo el capital
  social y permitiendo la evaluación continua de los acuerdos. Esto
  contribuye a transformar la cultura organizacional y a prevenir
  futuros conflictos.
\end{enumerate}

\section{Conclusiones}\label{conclusiones}

\begin{enumerate}
\def\labelenumi{\arabic{enumi}.}
\item
  El conflicto minero en Las Bambas, iniciado en 2015, tiene como causas
  principales la falta de consulta previa, los impactos ambientales y
  sociales del proyecto, y la percepción de incumplimientos por parte de
  MMG Limited y el Estado. Según Convoca (2023) y Red Muqui (2023), la
  contaminación de fuentes hídricas, la degradación de tierras agrícolas
  y el impacto del transporte terrestre han afectado gravemente a
  comunidades como Fuerabamba y Pumamarca. Socialmente, la reubicación
  forzada, la criminalización de líderes comunales y la distribución
  desigual de beneficios económicos han generado protestas y bloqueos
  del Corredor Minero Sur, con al menos cinco muertes reportadas entre
  2015 y 2025, según Ojo Público (2023) y Swissinfo (2024). La ausencia
  de procesos participativos conformes al Convenio 169 de la OIT ha
  exacerbado la desconfianza, consolidando el conflicto como un desafío
  estructural en la gobernanza minera peruana.
\item
  Las estrategias de negociación, principalmente mesas de diálogo
  tripartitas, han logrado avances parciales, como compromisos de
  inversión social y saneamiento de tierras, pero han sido insuficientes
  para resolver el conflicto de manera sostenible. CooperAcción (2016) y
  Ojo Público (2022) destacan que la falta de transparencia, el
  incumplimiento de acuerdos y el estilo de negociación de MMG,
  percibido como rígido y divisorio, han limitado el progreso. La
  criminalización de líderes comunales, documentada por Infobae (2025),
  y la exclusión de actores clave en las negociaciones han fracturado el
  diálogo, mientras que la percepción de parcialidad del Estado hacia
  los intereses empresariales ha erosionado la confianza de las
  comunidades, según APRODEH (2022).
\item
  Los mecanismos alternativos, como la mediación, el arbitraje de
  conciencia y las mesas de concertación, presentan un potencial
  significativo para abordar el conflicto de Las Bambas. El Centro de
  Análisis y Resolución de Conflictos de la PUCP (2023) y la Universidad
  de Lima (2023) destacan que la mediación promueve un diálogo
  intercultural que respeta las cosmovisiones comunitarias, mientras que
  el arbitraje de conciencia ofrece resoluciones rápidas y equitativas.
  Sin embargo, su implementación enfrenta desafíos, como la falta de
  capacitación en negociación intercultural y la necesidad de mayor
  voluntad política, según Agnitio (2023). Estas herramientas, aunque
  aplicadas en casos como el ``Proceso de Mediación Corredor Minero'',
  requieren mayor institucionalización para garantizar resultados
  sostenibles.
\end{enumerate}

\section{Recomendaciones}\label{recomendaciones}

\begin{enumerate}
\def\labelenumi{\arabic{enumi}.}
\item
  El Estado, a través del Ministerio de Energía y Minas y el
  Viceministerio de Interculturalidad, debe garantizar procesos de
  consulta previa, libre e informada antes de aprobar modificaciones al
  Estudio de Impacto Ambiental o proyectos mineros. Estos procesos deben
  ser inclusivos, transparentes y respetar las cosmovisiones de las
  comunidades, como lo recomienda Red Muqui (2019), para prevenir
  conflictos y legitimar las decisiones.
\item
  Se propone institucionalizar la mediación y el arbitraje de conciencia
  en conflictos mineros, con la participación de terceros neutrales,
  como la Defensoría del Pueblo o la Conferencia Episcopal. Según la
  Cámara Minera del Perú (2023a), estos mecanismos deben incluir
  capacitación en comunicación intercultural para mediadores y actores
  involucrados, asegurando acuerdos equitativos y sostenibles.
\item
  MMG Limited debe adoptar un enfoque de negociación más inclusivo y
  transparente, publicando actas detalladas de los acuerdos y
  estableciendo plazos claros para su cumplimiento, como sugiere
  CooperAcción (2016). El Estado debe supervisar estos compromisos a
  través de una entidad independiente para garantizar su ejecución y
  evitar percepciones de incumplimiento.
\item
  El Estado debe abstenerse de usar la fuerza pública de manera
  desproporcionada y revisar los procesos judiciales contra líderes
  comunales, siguiendo el precedente de la absolución de 2025
  documentado por Infobae (2025). La Defensoría del Pueblo (2025) debe
  fortalecer su rol en la protección de los derechos de los
  manifestantes para evitar la escalada violenta.
\item
  MMG y el Estado deben priorizar inversiones en infraestructura,
  saneamiento y proyectos productivos que beneficien directamente a las
  comunidades afectadas, como lo señala Desde Adentro (2021). Estas
  iniciativas deben diseñarse con la participación activa de las
  comunidades para garantizar una distribución equitativa de los
  beneficios económicos de la mina.
\end{enumerate}

\section{Bibliografía}\label{bibliografuxeda}

Actualidad Ambiental. (2015). \emph{Apurímac: Enfrentamientos por
proyecto minero ``Las Bambas'' dejan 4 muertos y 15 heridos}.
https://www.actualidadambiental.pe/apurimac-enfrentamientos-por-proyecto-minero-las-bambas-deja-4-muertos-y-15-heridos/

Actualidad Ambiental. (2019). \emph{Las Bambas: Cinco puntos claves para
entender el conflicto}.
https://www.actualidadambiental.pe/las-bambas-cinco-puntos-claves-para-entender-el-conflicto/

Agnitio. (2023). \emph{Diseccionando los conflictos mineros en el Perú:
Breve análisis, deficiencias, problemas de gestión, propuestas de mejora
y su efecto inmediato en la captación de inversión minera}.
https://agnitio.pe/articulo/diseccionando-los-conflictos-mineros-en-el-peru-breve-analisis-deficiencias-problemas-de-gestion-propuestas-de-mejora-y-su-efecto-inmediato-en-la-captacion-de-inversion-minera/

Amnistía Internacional. (2024). \emph{Perú: Muertes y lesiones en
protestas podrían implicar a presidenta y cadena de mando como
responsables penales}.
https://www.amnesty.org/es/latest/news/2024/07/peru-killings-injuries-protests-could-implicate-president-chain-command-criminally-responsible/

APRODEH. (2022). \emph{Se está logrando resolver el conflicto Las
Bambas}.
https://www.aprodeh.org.pe/se-esta-logrando-resolver-el-conflicto-las-bambas/

Business \& Human Rights Resource Centre. (2023). \emph{Perú: Comentario
sobre intereses en el proyecto Las Bambas de MMG de capital chino, sede
de conflictos violentos antiminero}.
https://www.business-humanrights.org/es/últimas-noticias/perú-comentario-sobre-intereses-en-del-proyecto-las-bambas-de-mmg-de-capital-chino-sede-de-conflictos-violentos-antimineros/

Cámara Minera del Perú. (2023a). \emph{La Cámara Minera del Perú crea
herramientas para la solución de conflictos en minería}.
https://camaraminera.com.pe/la-camara-minera-del-peru-crea-herramientas-para-la-solucion-de-conflictos-en-mineria

Cámara Minera del Perú. (2023b). \emph{La Cámara Minera del Perú lanza
nuevo diplomado presencial: Herramientas para la solución de conflictos
en minería}.
https://camaraminera.com.pe/la-camara-minera-del-peru-lanza-nuevo-diplomado-presencial-herramientas-para-la-solucion-de-conflictos-en-mineria

CBC. (2023). \emph{Desalojo policial en Pumamarca: Caso sector Sallawi}.
https://cbc.org.pe/desalojo-policial-en-pumamarca-caso-sector-sallawi/

Centro de Análisis y Resolución de Conflictos de la PUCP. (2023).
\emph{Mesa III: Mediación en conflictos sociales y socioambientales}.
https://carc.pucp.edu.pe/comunicados/mesa-iii-mediacion-en-conflictos-sociales-y-socioambientales/

China y América Latina. (2023). \emph{Las Bambas en Perú: Inversión
china, efecto económico y conflictos sociales}.
https://chinayamericalatina.com/las-bambas-en-peru-inversion-china-efecto-economico-y-conflictos-sociales/

Conflictos Mineros. (2021). \emph{Las Bambas: Alcaldes de Apurímac piden
al gobierno solucionar conflicto en Corredor Minero del Sur}.
https://conflictosmineros.org.pe/2021/12/09/las-bambas-alcaldes-de-apurimac-piden-al-gobierno-solucionar-conflicto-en-corredor-minero-del-sur/

Consorcio de Investigación Económica y Social. (2016). \emph{Medios,
oportunidades y gestión: La duración de los conflictos mineros en el
Perú}.
https://cies.org.pe/wp-content/uploads/2016/07/medios-oportunidades-y-gestion-la-duracion-de-los-conflictos-mineros-en-el-peru.pdf

Convoca. (2023). \emph{Las Bambas: Secretismo e incumplimientos
ambientales de la mina más grande del Perú}.
https://convoca.pe/agenda-propia/las-bambas-secretismo-e-incumplimientos-ambientales-de-la-mina-mas-grande-del-peru

CooperAcción. (2015). \emph{Las Bambas: Informe OCM}.
https://cooperaccion.org.pe/wp-content/uploads/2015/10/2015-10-Las-Bambas-informe-OCM.pdf

CooperAcción. (2016). \emph{Diálogo fracturado: Análisis de los procesos
de diálogo en Las Bambas}.
https://cooperaccion.org.pe/wp-content/uploads/2016/05/dialogofracturado.pdf

CooperAcción. (2018). \emph{Gobernanza y gobernabilidad en Las Bambas}.
https://cooperaccion.org.pe/wp-content/uploads/2018/09/Gobernanza-y-gobernabilidad-en-Las-Bambas\_WEB1.pdf

CooperAcción. (2021). \emph{Las Bambas: El gobierno anuncia que todo
está en calma, pero los conflictos retornan}.
https://cooperaccion.org.pe/opinion/las-bambas-el-gobierno-anuncia-que-todo-esta-en-calma-pero-los-conflictos-retornan/

CooperAcción. (2023a). \emph{Las Bambas, Apurímac y la lista de
pendientes}.
https://cooperaccion.org.pe/opinion/las-bambas-apurimac-y-la-lista-de-pendientes/

CooperAcción. (2023b). \emph{Las Bambas: Crónica de un conflicto
anunciado que parece no tener fin}.
https://cooperaccion.org.pe/opinion/las-bambas-cronica-de-un-conflicto-anunciado-que-parece-no-tener-fin/

CooperAcción. (2025). \emph{Archivan 2 procesos a líderes sociales por
el caso Las Bambas, pero aún enfrentan riesgo por 3 procesos más}.
https://cooperaccion.org.pe/archivan-2-procesos-a-lideres-sociales-por-el-caso-las-bambas-pero-aun-enfrentan-riesgo-por-3-procesos-mas/

Defensoría del Pueblo. (2021). \emph{Reporte de Conflictos Sociales N°
203}.
https://www.defensoria.gob.pe/wp-content/uploads/2025/05/Reporte-Mensual-de-Conflictos-Sociales-N°-254-Abr\_2025.pdf

Defensoría del Pueblo. (2024). \emph{Reporte Mensual de Conflictos
Sociales N° 253}.
https://www.defensoria.gob.pe/wp-content/uploads/2024/04/ReporteConflictos\_marzo2024.pdf

Defensoría del Pueblo. (2025). \emph{Reporte Mensual de Conflictos
Sociales N° 253}.
https://www.defensoria.gob.pe/wp-content/uploads/2025/04/3.-Reporte-Mensual-de-Conflictos-Sociales-N°-253-Mar\_2025.pdf

Desde Adentro. (2021). \emph{El impacto de Las Bambas}.
https://www.desdeadentro.pe/2021/12/el-impacto-de-las-bambas/

Deutsche Welle. (2022). \emph{Gobierno peruano convoca a diálogo en
conflicto por mina Las Bambas}.
https://www.dw.com/es/gobierno-peruano-convoca-a-diálogo-en-conflicto-por-mina-las-bambas/a-61643287

Dialnet. (2022). \emph{El conflicto socioambiental en Las Bambas:
Análisis desde el Convenio 169 de la OIT}.
https://dialnet.unirioja.es/descarga/articulo/8603149.pdf

DPLF. (2016). \emph{¿Por qué la consulta previa no funciona en proyectos
extractivos en el Perú?}
https://dplfblog.com/2016/05/03/por-que-la-consulta-previa-no-funciona-en-proyectos-extractivos-en-el-peru/

El Aporte de Las Bambas. (2023). \emph{Procesos de diálogo}.
https://www.elaportedelasbambas.pe/home-relacionamiento-con-la-comunidad-procesos-de-dialogo.html

El País. (2015). \emph{Cuatro muertos en las protestas contra un
proyecto minero en Perú}.
https://elpais.com/internacional/2015/09/29/actualidad/1443490267\_288762.html

Epicentro TV. (2024). \emph{Efectivos de la policía reprimen a
manifestantes que realizan un paro indefinido contra la minera Las
Bambas}. https://t.co/SczEnZQHS9

Infobae. (2022). \emph{Las Bambas: Cinco datos para entender el
conflicto entre comuneros y la minera MMG}.
https://www.infobae.com/america/peru/2022/04/29/las-bambas-cinco-datos-para-entender-el-conflicto-entre-comuneros-y-la-minera-mmg/

Infobae. (2025a). \emph{Perú: Minera Las Bambas podría llevar esta
semana a la condena a 11 defensores ambientales tras una década de
juicios}.
https://www.infobae.com/peru/2025/04/21/peru-minera-las-bambas-podria-llevar-esta-semana-a-la-condena-a-11-defensores-ambientales-tras-una-decada-de-juicios/

Infobae. (2025b). \emph{Las Bambas y una década de resistencia: La
histórica absolución de 11 defensores que sella un precedente para la
minería en el Perú}.
https://www.infobae.com/peru/2025/04/23/las-bambas-y-una-decada-de-resistencia-la-historica-absolucion-de-11-defensores-que-sella-un-precedente-para-la-mineria-en-el-peru/

Instituto de Defensa Legal. (2019). \emph{Comunidades de Chumbivilcas
piden consulta previa de vías utilizadas por Las Bambas}.
https://www.idl.org.pe/comunidades-de-chumbivilcas-piden-consulta-previa-de-vias-utilizadas-por-las-bambas/

Instituto de Defensa Legal. (2023). \emph{Las Bambas: Seis preguntas y
respuestas de un conflicto que no termina}.
https://www.idl.org.pe/las-bambas-seis-preguntas-y-respuestas-de-un-conflicto-que-no-termina-2/

Ministerio de Energía y Minas. (2020). \emph{Minera Las Bambas: Tajo
Chalcobamba}. https://cdn.www.gob.pe/uploads/document/file/3804742/580
MINERA LAS BAMBAS-TAJO CHALCOBAMBA.pdf.pdf

Ministerio de Energía y Minas. (2023). \emph{Estudio de Impacto
Ambiental del Proyecto Minero Las Bambas}.
https://www.gob.pe/institucion/minem/informes-publicaciones/6149792-estudio-de-impacto-ambiental-del-proyecto-minero-las-bambas

MMG Limited. (2023). \emph{Acerca de Las Bambas: Historia}.
https://www.lasbambas.com/seccion-acerca-de-las-bambas-historia

Mongabay. (2025). \emph{Perú: Criminalizados por protestar contra minera
logran justicia}.
https://es.mongabay.com/2025/05/peru-criminalizados-protestar-contra-minera-justicia/

OCMAL. (2023). \emph{Las Bambas: Toda la cronología del conflicto que
mantiene bloqueado el Corredor Minero}.
https://www.ocmal.org/las-bambas-toda-la-cronologia-del-conflicto-que-mantiene-bloqueado-corredor-minero/

Ojo Público. (2021). \emph{El conflicto se acentúa: El corredor minero
más importante del Perú}.
https://ojo-publico.com/edicion-regional/el-conflicto-se-acentua-el-corredor-minero-mas-importante-peru

Ojo Público. (2022). \emph{Las Bambas: Sin acuerdos sobre montos para
contratar empresas comunales}.
https://ojo-publico.com/edicion-regional/las-bambas-sin-acuerdos-sobre-montos-contratar-empresas-comunales

Ojo Público. (2023). \emph{El conflicto se acentúa: El corredor minero
más importante del Perú}.
https://ojo-publico.com/edicion-regional/el-conflicto-se-acentua-el-corredor-minero-mas-importante-peru

Otra Mirada. (2023). \emph{Las Bambas: Para comprender un poco más el
conflicto}.
https://otramirada.pe/las-bambas-para-comprender-un-poco-más-el-conflicto

Pata Amarilla. (2021). \emph{Entre el cese de operaciones y el conflicto
minero: El caso Las Bambas}.
https://www.patamarilla.com/2021/12/entre-el-cese-de-operaciones-y-el-conflicto-minero-el-caso-las-bambas/

Red Muqui. (2019). \emph{Informe alternativo 2019: Perú y el Convenio
169 de la OIT}.
https://muqui.org/wp-content/uploads/2024/02/Informe\_Alternativo\_2019\_Peru\_Convenio\_169.pdf

Red Muqui. (2020). \emph{Las Bambas: El diálogo y la acusación judicial
contra 19 comuneros}.
https://muqui.org/las-bambas-el-dialogo-y-la-acusacion-judicial-contra-19-comuneros/

Red Muqui. (2023). \emph{Apurímac: Continúa protesta contra empresa
minera Las Bambas}.
https://muqui.org/apurimac-continua-protesta-contra-empresa-minera-las-bambas/

RFI. (2015). \emph{Perú declara estado de emergencia por conflicto en
Las Bambas}.
https://www.rfi.fr/es/americas/20150930-peru-declara-estado-de-emergencia-por-conflicto-en-las-bambas

Rumbo Minero. (2022). \emph{Las Bambas: Diálogo con comunidades
campesinas de Cotabambas y minera}.
https://www.rumbominero.com/peru/las-bambas-dialogo-comunidades-campesinas-cotabambas-y-minera/

Rumbo Minero. (2025). \emph{Apurímac registró un aumento de 53,1\%,
impulsado por la mayor extracción de cobre en Ferrobamba y Chalcobamba,
operados por Las Bambas}. https://t.co/eKhaKMlc3R

SENACE. (2023). \emph{Resumen Ejecutivo de la Tercera MEIA Las Bambas}.
https://www.senace.gob.pe/wp-content/uploads/filebase/comunicaciones/eia-meia/unidad-minera-las-bambas-3era-mod/Resumen-Ejecutivo-de-la-Tercera-MEIA-Las-Bambas.pdf

Swissinfo. (2024). \emph{Dos muertos en violentas protestas contra el
mayor proyecto minero de Perú}.
https://www.swissinfo.ch/spa/dos-muertos-en-violentas-protestas-contra-el-mayor-proyecto-minero-de-perú/41688794

Universidad de Lima. (2023). \emph{Arbitraje de conciencia para resolver
conflictos sociales en minería}.
https://www.ulima.edu.pe/pregrado/derecho/noticias/arbitraje-de-conciencia-para-resolver-conflictos-sociales-en-mineria

Wayka. (2024a). \emph{Pumamarca en protesta contra MMG Las Bambas:
Responsabilizamos al gobierno por cualquier muerto}.
https://wayka.pe/pumamarca-en-protesta-contra-mmg-las-bambas-responsabilizamos-al-gobierno-por-cualquier-muerto/

Wayka. (2024b). \emph{Miles de campesinos de la comunidad de Pumamarca
en Apurímac continúan con su protesta pacífica}. https://t.co/CUWZIzL5kg

Wayka. (2024c). \emph{11 dirigentes comuneros de Apurímac fueron
sentenciados injustamente a 8 y 9 años de prisión por participar en las
protestas contra el proyecto minero Las Bambas durante setiembre del
2015}. https://t.co/U9piq7jlW1

Wikipedia. (2023). \emph{Mina Las Bambas}.
https://es.wikipedia.org/wiki/Mina\_Las\_Bambas






\end{document}
