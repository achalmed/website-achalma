% Options for packages loaded elsewhere
\PassOptionsToPackage{unicode}{hyperref}
\PassOptionsToPackage{hyphens}{url}
\PassOptionsToPackage{dvipsnames,svgnames,x11names}{xcolor}
%
\documentclass[
  a4paper,
]{article}

\usepackage{amsmath,amssymb}
\usepackage{iftex}
\ifPDFTeX
  \usepackage[T1]{fontenc}
  \usepackage[utf8]{inputenc}
  \usepackage{textcomp} % provide euro and other symbols
\else % if luatex or xetex
  \usepackage{unicode-math}
  \defaultfontfeatures{Scale=MatchLowercase}
  \defaultfontfeatures[\rmfamily]{Ligatures=TeX,Scale=1}
\fi
\usepackage{lmodern}
\ifPDFTeX\else  
    % xetex/luatex font selection
\fi
% Use upquote if available, for straight quotes in verbatim environments
\IfFileExists{upquote.sty}{\usepackage{upquote}}{}
\IfFileExists{microtype.sty}{% use microtype if available
  \usepackage[]{microtype}
  \UseMicrotypeSet[protrusion]{basicmath} % disable protrusion for tt fonts
}{}
\makeatletter
\@ifundefined{KOMAClassName}{% if non-KOMA class
  \IfFileExists{parskip.sty}{%
    \usepackage{parskip}
  }{% else
    \setlength{\parindent}{0pt}
    \setlength{\parskip}{6pt plus 2pt minus 1pt}}
}{% if KOMA class
  \KOMAoptions{parskip=half}}
\makeatother
\usepackage{xcolor}
\usepackage[top=2.54cm,right=2.54cm,bottom=2.54cm,left=2.54cm]{geometry}
\setlength{\emergencystretch}{3em} % prevent overfull lines
\setcounter{secnumdepth}{-\maxdimen} % remove section numbering
% Make \paragraph and \subparagraph free-standing
\ifx\paragraph\undefined\else
  \let\oldparagraph\paragraph
  \renewcommand{\paragraph}[1]{\oldparagraph{#1}\mbox{}}
\fi
\ifx\subparagraph\undefined\else
  \let\oldsubparagraph\subparagraph
  \renewcommand{\subparagraph}[1]{\oldsubparagraph{#1}\mbox{}}
\fi


\providecommand{\tightlist}{%
  \setlength{\itemsep}{0pt}\setlength{\parskip}{0pt}}\usepackage{longtable,booktabs,array}
\usepackage{calc} % for calculating minipage widths
% Correct order of tables after \paragraph or \subparagraph
\usepackage{etoolbox}
\makeatletter
\patchcmd\longtable{\par}{\if@noskipsec\mbox{}\fi\par}{}{}
\makeatother
% Allow footnotes in longtable head/foot
\IfFileExists{footnotehyper.sty}{\usepackage{footnotehyper}}{\usepackage{footnote}}
\makesavenoteenv{longtable}
\usepackage{graphicx}
\makeatletter
\def\maxwidth{\ifdim\Gin@nat@width>\linewidth\linewidth\else\Gin@nat@width\fi}
\def\maxheight{\ifdim\Gin@nat@height>\textheight\textheight\else\Gin@nat@height\fi}
\makeatother
% Scale images if necessary, so that they will not overflow the page
% margins by default, and it is still possible to overwrite the defaults
% using explicit options in \includegraphics[width, height, ...]{}
\setkeys{Gin}{width=\maxwidth,height=\maxheight,keepaspectratio}
% Set default figure placement to htbp
\makeatletter
\def\fps@figure{htbp}
\makeatother

% Preámbulo
\usepackage{comment} % Permite comentar secciones del código
\usepackage{marvosym} % Agrega símbolos adicionales
\usepackage{graphicx} % Permite insertar imágenes
\usepackage{mathptmx} % Fuente de texto matemática
\usepackage{amssymb} % Símbolos adicionales de matemáticas
\usepackage{lipsum} % Crea texto aleatorio
\usepackage{amsthm} % Teoremas y entornos de demostración
\usepackage{float} % Control de posiciones de figuras y tablas
\usepackage{rotating} % Rotación de elementos
\usepackage{multirow} % Celdas combinadas en tablas
\usepackage{tabularx} % Tablas con ancho de columna ajustable
\usepackage{mdframed} % Marcos alrededor de elementos flotantes

% Series de tiempo
\usepackage{booktabs}


% Configuración adicional

\makeatletter
\makeatother
\makeatletter
\makeatother
\makeatletter
\@ifpackageloaded{caption}{}{\usepackage{caption}}
\AtBeginDocument{%
\ifdefined\contentsname
  \renewcommand*\contentsname{Tabla de contenidos}
\else
  \newcommand\contentsname{Tabla de contenidos}
\fi
\ifdefined\listfigurename
  \renewcommand*\listfigurename{Listado de Figuras}
\else
  \newcommand\listfigurename{Listado de Figuras}
\fi
\ifdefined\listtablename
  \renewcommand*\listtablename{Listado de Tablas}
\else
  \newcommand\listtablename{Listado de Tablas}
\fi
\ifdefined\figurename
  \renewcommand*\figurename{Figura}
\else
  \newcommand\figurename{Figura}
\fi
\ifdefined\tablename
  \renewcommand*\tablename{Tabla}
\else
  \newcommand\tablename{Tabla}
\fi
}
\@ifpackageloaded{float}{}{\usepackage{float}}
\floatstyle{ruled}
\@ifundefined{c@chapter}{\newfloat{codelisting}{h}{lop}}{\newfloat{codelisting}{h}{lop}[chapter]}
\floatname{codelisting}{Listado}
\newcommand*\listoflistings{\listof{codelisting}{Listado de Listados}}
\makeatother
\makeatletter
\@ifpackageloaded{caption}{}{\usepackage{caption}}
\@ifpackageloaded{subcaption}{}{\usepackage{subcaption}}
\makeatother
\makeatletter
\@ifpackageloaded{tcolorbox}{}{\usepackage[skins,breakable]{tcolorbox}}
\makeatother
\makeatletter
\@ifundefined{shadecolor}{\definecolor{shadecolor}{rgb}{.97, .97, .97}}
\makeatother
\makeatletter
\makeatother
\makeatletter
\makeatother
\ifLuaTeX
\usepackage[bidi=basic]{babel}
\else
\usepackage[bidi=default]{babel}
\fi
\babelprovide[main,import]{spanish}
% get rid of language-specific shorthands (see #6817):
\let\LanguageShortHands\languageshorthands
\def\languageshorthands#1{}
\ifLuaTeX
  \usepackage{selnolig}  % disable illegal ligatures
\fi
\usepackage[]{biblatex}
\addbibresource{../../../../references.bib}
\IfFileExists{bookmark.sty}{\usepackage{bookmark}}{\usepackage{hyperref}}
\IfFileExists{xurl.sty}{\usepackage{xurl}}{} % add URL line breaks if available
\urlstyle{same} % disable monospaced font for URLs
\hypersetup{
  pdftitle={Aparición del Pensamiento Socialista},
  pdfauthor={Edison Achalma},
  pdflang={es},
  colorlinks=true,
  linkcolor={blue},
  filecolor={Maroon},
  citecolor={Blue},
  urlcolor={Blue},
  pdfcreator={LaTeX via pandoc}}

\title{Aparición del Pensamiento Socialista}
\usepackage{etoolbox}
\makeatletter
\providecommand{\subtitle}[1]{% add subtitle to \maketitle
  \apptocmd{\@title}{\par {\large #1 \par}}{}{}
}
\makeatother
\subtitle{Aparición del pensamiento socialista}
\author{Edison Achalma}
\date{2018-04-23}

\begin{document}
\maketitle
\ifdefined\Shaded\renewenvironment{Shaded}{\begin{tcolorbox}[breakable, frame hidden, enhanced, boxrule=0pt, interior hidden, borderline west={3pt}{0pt}{shadecolor}, sharp corners]}{\end{tcolorbox}}\fi

\hypertarget{presentaciuxf3n}{%
\section{Presentación}\label{presentaciuxf3n}}

El actual movimiento de resistencia global contra el capitalismo, bajo
la consigna de ``otro mundo es posible'', ha resaltado el pensamiento
socialista como una alternativa. En este contexto, resulta imperativo
estudiar, comprender y debatir la teoría del surgimiento del pensamiento
socialista.

La historia del pensamiento socialista a nivel mundial enriquece la
constante batalla de ideas. Entre los pensadores socialistas destacados
se encuentran Carlos Marx y Saint-Simon. Marx aborda las bases
científicas e industriales para alcanzar una sociedad sin clases,
mientras que Saint-Simon destaca la importancia de la ciencia y la
industria en este proceso.

Además, Charles Fourier propone la formación de grupos autosuficientes
conocidos como falanges, como una vía para la construcción de una
sociedad igualitaria.

El estudio de estos pensadores socialistas y su legado resulta esencial
para comprender los fundamentos y los desafíos del movimiento de
resistencia global actual y reafirmar la consigna de ``otro mundo es
posible''.

\hypertarget{apariciuxf3n-del-pensamiento-socialista}{%
\section{Aparición del pensamiento
socialista}\label{apariciuxf3n-del-pensamiento-socialista}}

\hypertarget{antecedentes-histuxf3ricos-del-socialismo}{%
\subsection{Antecedentes históricos del
socialismo}\label{antecedentes-histuxf3ricos-del-socialismo}}

``La Revolución Industrial trajo consigo el fin de la seguridad
económica de la antigua estructura basada en la agricultura, las aldeas
y las artesanías. En su lugar, surgieron barrios pobres abarrotados,
donde el vicio, el crimen, la enfermedad, el hambre y la miseria se
convirtieron en una forma de vida.

En aquel entonces, las familias de los trabajadores que resultaban
heridos o fallecían en accidentes industriales recibían escasa o ninguna
compensación. Los trabajadores asalariados carecían de derechos,
mientras que los sindicatos eran considerados ilegales.

Cada periodo de declive en la producción y el empleo agravaba la miseria
de los trabajadores, y cada nuevo avance en la industrialización, si
bien a largo plazo generaba más empleos de los que destruía, implicaba
que cientos de miles de artesanos se veían arrojados al mercado laboral.

La creciente pobreza de las masas resultaba cada vez más opresiva a
medida que las grandes fortunas se multiplicaban.''

\hypertarget{tipos-de-socialismo}{%
\section{Tipos de Socialismo}\label{tipos-de-socialismo}}

\hypertarget{socialismo-utuxf3pico-alrededor-de-1800}{%
\subsection{Socialismo Utópico (alrededor de
1800)}\label{socialismo-utuxf3pico-alrededor-de-1800}}

El socialismo utópico, la primera corriente del pensamiento socialista
moderno, se desarrolló en los siglos XVIII y XIX en Europa. Surgió como
una crítica a las terribles condiciones inhumanas a las que eran
sometidos los obreros urbanos y los campesinos proletarizados debido al
capitalismo industrial de la época. Según Carlos Marx, los sistemas
propuestos por figuras como Saint-Simón (1760-1825), Fourier
(1772-1873), Owen (1771-1858), entre otros, surgieron en las etapas
iniciales y rudimentarias de la lucha entre el proletariado y la
burguesía.

Estos pensadores desarrollaron sus ideas en una época en la que los
trabajadores industriales aún eran débiles y estaban desorganizados.
Estaban desmoralizados por los rápidos cambios de la Revolución
Industrial, carecían de protecciones laborales y aún no eran conscientes
de su poder latente. Los socialistas utópicos consideraban que la
economía de mercado competitiva era injusta e irracional. Abogaban por
la solidaridad universal en lugar de la lucha de clases, y buscaban la
cooperación de los capitalistas en sus proyectos, incluso buscando su
financiamiento.

\hypertarget{socialismo-de-estado}{%
\subsection{Socialismo de Estado}\label{socialismo-de-estado}}

El socialismo de Estado se refiere a un sistema en el cual el gobierno
posee y opera todos o algunos sectores específicos de la economía, con
el objetivo de lograr objetivos sociales más allá de la mera obtención
de utilidades. Un ejemplo de esto fue la antigua Unión Soviética, donde
hasta hace poco tiempo, los sectores más importantes de la economía eran
propiedad y estaban bajo el control del Estado. Sin embargo, el
socialismo de Estado también puede existir dentro de un sistema
capitalista. En Estados Unidos, podemos encontrar ejemplos como el
sistema federal de seguridad social, la Tennessee Valley Authority y el
servicio postal.

Históricamente, el Estado socialista se consideraba como un poder
imparcial que podía ser influenciado para beneficiar a la clase
trabajadora si se extendía el derecho al voto y se educaba y organizaba
a los trabajadores. En este contexto, el Estado asumía el rol de
empleador al hacerse cargo de las empresas, o podía fomentar y subsidiar
cooperativas, donde los trabajadores o los consumidores se convertían en
propietarios. Louis Blanc fue uno de los principales defensores del
Estado socialista.

\hypertarget{socialismo-cristiano}{%
\subsection{Socialismo Cristiano}\label{socialismo-cristiano}}

El socialismo cristiano se desarrolló en Inglaterra y Alemania después
de 1848, y tuvo a Charles Kingsley como uno de sus principales
defensores en Inglaterra. Surgió tras la derrota de los movimientos
radicales en ambos países. Esta corriente del socialismo ofrecía a los
trabajadores el consuelo de la religión como forma de mitigar su dolor y
brindarles esperanza. Se consideraba que la Biblia sería el manual para
los líderes gubernamentales, los empleadores y los trabajadores, y que
el amor y la camaradería mutua eran el orden divino. En este enfoque, se
proponía que la propiedad en manos de los ricos se mantuviera en
fideicomiso para beneficio de todos.

El socialismo cristiano rechazaba la violencia y la lucha de clases,
abogando en su lugar por reformas sanitarias, educación, legislación
laboral en las fábricas y la promoción de cooperativas. Es importante
destacar que existen diferentes versiones e interpretaciones de esta
corriente, dependiendo de la afiliación religiosa y las creencias
cristianas de cada persona. Desde el siglo XIX, el socialismo cristiano
ha estado presente en la Comunión Anglicana, promoviendo la
identificación entre cristianismo y socialismo. Algunos sostienen que el
socialismo cristiano se remonta a la época de Jesús, argumentando que
Jesús predicaba y practicaba la igualdad entre las personas.

\hypertarget{anarquismo}{%
\subsection{Anarquismo}\label{anarquismo}}

El anarquismo, con Pierre-Joseph Proudhon (1809-1865) como uno de sus
primeros defensores, sostiene la abolición de todas las formas de
gobierno por considerarlas coercitivas. Los anarquistas rechazan la
noción de un orden impuesto en la sociedad y en su lugar promueven la
formación de grupos autónomos basados en el esfuerzo voluntario y la
asociación. Argumentan que la naturaleza humana es intrínsecamente buena
y que es el Estado y sus instituciones quienes la corrompen. Proponen
reemplazar la propiedad privada por la propiedad colectiva del capital,
administrada por grupos cooperativos.

Los anarquistas imaginan comunidades dedicadas a la producción y al
intercambio con otras comunidades, donde las asociaciones de productores
controlan la producción en sectores agrícolas, industriales e incluso
intelectuales y artísticos. Se espera que las asociaciones de
consumidores coordinen aspectos como la vivienda, la iluminación, la
salud, los alimentos y la higiene pública. La sociedad anarquista se
caracterizaría por la comprensión mutua, la cooperación y la absoluta
libertad. Se fomentaría la iniciativa individual y se evitaría cualquier
tendencia hacia la uniformidad y la autoridad centralizada. Aunque los
métodos para lograr sus objetivos pueden variar, la comunidad ideal
propuesta por los anarquistas guarda similitudes con la de los
socialistas utópicos.

\hypertarget{socialismo-marxista}{%
\subsection{Socialismo Marxista}\label{socialismo-marxista}}

Principales defensores: Marx y Engels.

El socialismo marxista, también conocido como socialismo científico, fue
desarrollado por Karl Marx (1818-1883) y Friedrich Engels (1821-1895).
Esta corriente busca distinguirse de otras corrientes socialistas
existentes en el siglo XIX. Las premisas teóricas del socialismo
científico se basan en el análisis científico de la sociedad, utilizando
el materialismo histórico para extraer las leyes de su evolución.
Además, el marxismo desempeñó un papel fundamental en la construcción
del campo socialista y en los movimientos anticoloniales y de liberación
nacional del siglo XX.

\hypertarget{el-socialismo-marxista-o-cientuxedfico}{%
\subsubsection{El socialismo marxista o
científico:}\label{el-socialismo-marxista-o-cientuxedfico}}

El socialismo marxista se fundamenta en la teoría del valor trabajo y en
la teoría de la explotación por parte de los capitalistas hacia aquellos
que reciben un salario. Aunque Marx y Engels despreciaban el sistema
capitalista, reconocían su capacidad para aumentar la productividad y la
producción. Sin embargo, también señalaban las luchas de clases y las
contradicciones inherentes al capitalismo, lo que eventualmente
conduciría a su derrocamiento y a su reemplazo por el socialismo. El
Estado capitalista oprime a los trabajadores, por lo que la clase
trabajadora, al derrocar al Estado burgués, establecerá su propia
dictadura del proletariado para eliminar la clase burguesa. En el
socialismo resultante, se permite la propiedad privada de bienes de
consumo, pero los medios de producción y la tierra son de propiedad
pública y son gestionados por el gobierno central. La producción y la
inversión se planifican, y la utilidad y el libre mercado son
reemplazados por una distribución planificada.

\hypertarget{comunismo-seguxfan-marx}{%
\subsubsection{Comunismo según Marx:}\label{comunismo-seguxfan-marx}}

El comunismo es la etapa de la sociedad que eventualmente reemplaza al
socialismo. En el comunismo, el lema es ``De cada uno según su
capacidad, a cada uno según su necesidad''. Esto presupone una
abundancia de bienes en relación con los deseos, la eliminación del pago
monetario por el trabajo realizado y una dedicación a la sociedad
similar a la lealtad de una persona hacia su familia. En esta etapa,
desaparecen las clases sociales antagónicas y el gobierno sobre las
personas es reemplazado por la administración de las cosas, como los
grandes sistemas de transporte y los complejos industriales.

Actualmente, los países comunistas han establecido en realidad un
socialismo de Estado o están en proceso de hacerlo. En la época actual,
el comunismo no existe en ningún lugar, excepto en pequeñas comunidades
cooperativas motivadas principalmente por razones religiosas u otras
causas comunes. En estas comunidades, las personas trabajan juntas,
comparten sus ganancias y retiran de un fondo común las cosas que
necesitan.

\hypertarget{revisionismo}{%
\subsection{Revisionismo}\label{revisionismo}}

En Alemania, Eduard Bernstein (1850-1932) fue un defensor del
revisionismo, mientras que en Inglaterra, los socialistas fabianos,
liderados por Sydney y Beatrice Webb (1859-1947; 1858-1943), también
adoptaron posturas revisionistas. Estos revisionistas rechazaban la
lucha de clases y negaban que el Estado necesariamente fuera un
instrumento de la clase adinerada. En cambio, depositaban sus esperanzas
en la educación, las campañas electorales y en obtener el control del
gobierno a través del voto. Proponían que el gobierno regulara los
monopolios, controlara las condiciones laborales en las fábricas,
asumiera la responsabilidad de algunos servicios públicos y gradualmente
ampliara su propiedad de los medios de producción.

El revisionismo comprende una variedad de filosofías políticas
antiautoritarias que forman parte del socialismo y que buscan crear una
sociedad sin jerarquías políticas, económicas o sociales. En esta
sociedad, las instituciones violentas o coercitivas se disolverían, y
todas las personas tendrían acceso libre e igualitario a las
herramientas de información y producción. También se plantea la
reducción drástica del alcance de las instituciones coercitivas y
jerárquicas. El socialismo democrático, por su parte, es una ideología
crítica que diversos movimientos, corrientes y organizaciones utilizan
para expresar que su posición y propósito abarcan tanto el socialismo
como la democracia. Aunque a menudo se utiliza como sinónimo de
``socialdemocracia'', su alcance es más amplio, incluyendo diversas
corrientes.

\hypertarget{sindicalismo}{%
\subsection{Sindicalismo}\label{sindicalismo}}

Georges Sorel (1847-1922) fue un promotor y popularizador del
sindicalismo en los círculos laborales de los países latinos de Europa.
Los sindicalistas se oponían al parlamento y al militarismo, y sostenían
que el socialismo se corrompía al involucrarse en la actividad política
y parlamentaria. Consideraban que la participación en el parlamento
conducía al oportunismo y a la búsqueda de influencia política. En
cambio, sostenían que los trabajadores necesitaban un sindicato fuerte
que no se involucrara en el juego burgués de buscar reformas sociales y
mejoras en las condiciones laborales. Los sindicatos no deberían
ocuparse de huelgas, fondos de seguro, contratos sindicales, tesorerías
sindicales o reformas graduales. Las huelgas debían ser fomentadas para
despertar la conciencia revolucionaria y la militancia de los
trabajadores, y el sabotaje se debía utilizar como arma en la lucha de
clases. A largo plazo, se esperaba que una huelga general liderada por
un sindicato fuerte derrocara al capitalismo, y cada industria estaría
organizada como una unidad autónoma gestionada por los trabajadores.
Estas unidades estarían federadas y formarían un centro administrativo,
con la expectativa de que el gobierno coercitivo desapareciera en última
instancia.

\hypertarget{socialismo-de-gremios}{%
\subsection{Socialismo de gremios}\label{socialismo-de-gremios}}

G. D. H. Cole (1889-1959), profesor de economía en la Universidad de
Oxford, fue el principal defensor del socialismo de gremios. Este
movimiento británico, basado en el gradualismo y la reforma, alcanzó su
apogeo alrededor de la Primera Guerra Mundial. Los socialistas de
gremios reconocían al Estado como una institución necesaria para
expresar los intereses generales de los ciudadanos como consumidores.
Sin embargo, proponían que la administración real de las industrias
estuviera en manos de los empleados (los productores), organizados en
sus gremios industriales, y no en manos del gobierno. El gobierno, por
su parte, debería desarrollar una política económica general para toda
la comunidad, no solo para los trabajadores. Cada trabajador sería un
socio en la empresa para la cual trabajaba, y esta era la esencia de la
``democracia industrial'' promovida por los socialistas de gremios. En
lugar de una división entre capital y trabajo, la nación se dividiría en
productores y consumidores, cada grupo con su asociación nacional, el
gremio y el gobierno. De esta manera, los productores y los consumidores
formarían una sociedad de iguales.

El socialismo de gremios es un movimiento político que abogaba por el
control de la industria por parte de los trabajadores a través de
gremios organizados por ramas empresariales. Su enfoque se caracterizaba
por el socialismo individualista, cooperativo y antie statal, con un
liderazgo distribuido y no centralizado. En ocasiones se lo consideraba
la contraparte anglosajona del anarcosindicalismo latino, aunque no
había una relación formal entre ambos. Este movimiento se originó en el
Reino Unido, y su periodo de mayor influencia fue el primer cuarto del
siglo XX.

El socialismo de gremios se inspiró en parte en los gremios de artesanos
y otros trabajadores especializados que existieron en la Inglaterra
medieval. Cuestionaba la tendencia hacia el aburguesamiento del
fabianismo y el extremismo colectivista del marxismo, y compartía con el
anarquismo la importancia de la voluntad individual como impulsora de
los cambios sociales, así como el espíritu autogestionario y
federalista.

\hypertarget{caracteruxedsticas-comunes-del-socialismo}{%
\section{Características comunes del
Socialismo}\label{caracteruxedsticas-comunes-del-socialismo}}

Las diversas corrientes del socialismo comparten características
comunes:

\begin{enumerate}
\def\labelenumi{\arabic{enumi}.}
\tightlist
\item
  Rechazo de la armonía de intereses: Todas estas corrientes rechazan la
  noción clásica de que los intereses de diferentes clases sociales
  están en armonía. Reconocen los conflictos inherentes al sistema
  capitalista y la desigualdad de poder entre capital y trabajo.
\item
  Oposición al laissez-faire: Con excepción de los anarquistas, todas
  las corrientes socialistas se oponen al concepto de laissez-faire, que
  defiende una mínima intervención del Estado en la economía. Consideran
  necesario un mayor control estatal para regular y proteger los
  derechos de los trabajadores.
\item
  Crítica a la ley de los mercados de Say: Los socialistas rechazan la
  idea de Jean-Baptiste Say de que la oferta crea su propia demanda y
  sostienen que el capitalismo es propenso a crisis periódicas o
  estancamiento general. Argumentan que el sistema no garantiza un
  equilibrio económico y puede llevar a desequilibrios y problemas.
\item
  Crítica al egoísmo del capitalismo: El capitalismo, según los
  socialistas, fomenta una mentalidad egoísta basada en la obtención de
  ganancias y la acumulación de riqueza. Sin embargo, consideran que en
  un entorno adecuado pueden surgir virtudes humanas más nobles, como la
  solidaridad y el espíritu de compartir.
\item
  Acción pública y propiedad colectiva: Todas las corrientes socialistas
  abogan por la intervención estatal y la propiedad colectiva de los
  medios de producción con el objetivo de mejorar las condiciones de las
  masas. Esto puede ser a través del gobierno central, gobiernos locales
  o empresas cooperativas.
\end{enumerate}

\hypertarget{a-quiuxe9nes-beneficiaba-o-trataba-de-beneficiar-el-socialismo}{%
\subsection{¿A quiénes beneficiaba o trataba de beneficiar el
socialismo?}\label{a-quiuxe9nes-beneficiaba-o-trataba-de-beneficiar-el-socialismo}}

El socialismo buscaba beneficiar a los trabajadores, despertar la
conciencia de la sociedad e inspirar a los reformadores de la clase
media a promover la legislación reformista. Sin embargo, algunos
críticos argumentan que al apartar a los trabajadores de la organización
de sindicatos y partidos políticos, el socialismo también servía a los
patronos y terratenientes.

\hypertarget{en-quuxe9-forma-el-socialismo-era-vuxe1lido-uxfatil-o-correcto-en-su-uxe9poca}{%
\subsection{¿En qué forma el socialismo era válido, útil o correcto en
su
época?}\label{en-quuxe9-forma-el-socialismo-era-vuxe1lido-uxfatil-o-correcto-en-su-uxe9poca}}

En sus inicios en el siglo XIX, el socialismo utópico expresaba las
preocupaciones sociales de la humanidad. Si bien no se enfrentaron
directamente a los problemas de la pobreza y las crisis económicas, los
socialistas contribuyeron al centrar su atención en estos problemas sin
resolver.

El socialismo desempeñó un papel histórico útil al promover reformas
como las leyes laborales, la sanidad pública, las asociaciones
cooperativas, las leyes de compensación laboral, los sindicatos y las
pensiones. Estas ideas y políticas se han institucionalizado en las
naciones capitalistas actuales.

\hypertarget{principios-perdurables-del-socialismo}{%
\subsection{Principios perdurables del
socialismo}\label{principios-perdurables-del-socialismo}}

El socialismo sentó las bases del pensamiento económico socialista
contemporáneo, que enfatiza la propiedad estatal de los medios de
producción junto con la planificación y coordinación a nivel nacional.

Las recomendaciones de políticas hechas por los socialistas se han
institucionalizado en la actualidad dentro de naciones capitalistas.

Su énfasis en el análisis del crecimiento del poder del monopolio, del
problema de la distribución del ingreso y de la realidad de los ciclos
de negocios. Este énfasis y análisis obligaron a una reevaluación de las
suposiciones básicas y de las teorías aceptadas dentro de la profesión
de la economía.

\hypertarget{principales-representantes}{%
\section{Principales representantes}\label{principales-representantes}}

\hypertarget{henry-comte-de-saint-simon-1760-1825}{%
\subsection{Henry Comte de Saint-Simon
(1760-1825)}\label{henry-comte-de-saint-simon-1760-1825}}

Proveniente de una familia noble francesa empobrecida, Henry Comte de
Saint-Simon destacó en la Revolución Estadounidense, donde luchó del
lado colonial como oficial. Durante las primeras etapas de la Revolución
Francesa, renunció a su título y se convirtió en un destacado
especulador al adquirir tierras nacionalizadas de la Iglesia y los
emigrados, pagando con assignats, cuyo valor se depreciaba rápidamente.
Saint-Simon fue encarcelado por esta actividad, pero más tarde fue
liberado tras la caída de Robespierre.

Sus principales ideas y propuestas fueron las siguientes:

\begin{itemize}
\tightlist
\item
  Estableció una distinción entre los productores y los no productores.
\item
  Propuso un parlamento industrial compuesto por tres cámaras: la cámara
  de invención, integrada por artistas e ingenieros encargados de
  diseñar obras públicas; la cámara de revisión, dirigida por
  científicos, encargada de evaluar proyectos y supervisar la educación;
  y la cámara de ejecución, compuesta por líderes de la industria,
  encargada de llevar a cabo los proyectos y controlar el presupuesto.
\item
  Rechazó la suposición fundamental de los economistas clásicos de que
  los intereses de la industria coinciden con el interés general. En sus
  últimos escritos, mostró una preocupación humanitaria por la clase
  trabajadora.
\item
  Su entusiasmo por la industria a gran escala inspiró el desarrollo de
  grandes bancos, ferrocarriles, carreteras, el Canal de Suez y grandes
  empresas industriales.
\item
  Sostenía que la prosperidad de Francia depende del progreso de las
  ciencias, las bellas artes y las profesiones.
\item
  Consideraba que los científicos, artesanos y líderes de las empresas
  industriales eran los más capacitados y útiles para dirigir la
  sociedad en la época actual, por lo que deberían tener el poder
  administrativo.
\item
  Comparaba la comunidad con una pirámide, donde la nación debía ser
  estructurada de manera similar, con capas cada vez más valiosas desde
  la base hasta la cima, siendo la monarquía la corona de esa pirámide.
\item
  Aunque algunos de sus discípulos defendían la apropiación de la
  propiedad privada, él mismo no lo respaldaba.
\end{itemize}

\hypertarget{charles-fourier-1772-1837}{%
\subsection{Charles Fourier
(1772-1837)}\label{charles-fourier-1772-1837}}

Charles Fourier fue un excéntrico socialista utópico que logró reunir un
numeroso grupo de seguidores durante los últimos años de su vida e
incluso después de su fallecimiento. A diferencia de un revolucionario,
dirigía sus llamados principalmente a los ricos y al rey. Proveniente de
una familia de comerciantes de clase media que perdió gran parte de sus
bienes durante la Revolución Francesa, Fourier trabajó en varias tiendas
de telas y otros negocios. A lo largo de su vida, siendo un trabajador
pobre, tuvo que adquirir conocimientos en su tiempo libre en las salas
de lectura de las bibliotecas. Los títulos de sus libros reflejan la
naturaleza inusual de su pensamiento: ``Theory of the Four Movements and
the General Destinies'' (1808), ``The Theory of Universal Unity'' (1829)
y ``The New Industrial and Social World'' (1829).

Sus principales ideas y propuestas fueron las siguientes:

\begin{itemize}
\tightlist
\item
  Fourier consideraba que la competencia multiplicaba el desperdicio en
  la venta, y los empresarios se veían obligados a retener o destruir
  bienes para aumentar los precios.
\item
  Su solución para los problemas sociales era eliminar las barreras
  artificiales que impedían la interacción armónica de las doce pasiones
  humanas (cinco sentidos, cuatro grupos de pasiones: amistad, amor,
  sentimiento familiar y ambición, y tres pasiones distributivas:
  planeación, cambio y unidad).
\item
  Propuso la organización de comunidades cooperativas llamadas
  falansterios o falanges.
\item
  La vida cooperativa era fundamental en su pensamiento, ya que creía
  que era la forma de transformar el entorno y generar un nuevo tipo de
  persona noble. Estas falanges proporcionarían seguridad social desde
  el nacimiento hasta la muerte.
\item
  Fourier defendía la idea de brindar garantías a cada individuo,
  asegurando un mínimo de subsistencia, seguridad y comodidad.
\item
  Criticaba la excesiva especialización y advertía que el trabajo
  rutinario en la línea de ensamblaje desviaba y frustraba al individuo,
  a pesar de aumentar considerablemente la producción.
\end{itemize}

\hypertarget{simonde-de-sismondi-1773-1842}{%
\subsection{Simonde de Sismondi
(1773-1842)}\label{simonde-de-sismondi-1773-1842}}

Simonde de Sismondi fue un economista e historiador suizo de ascendencia
francesa. Durante los disturbios revolucionarios de 1793-1794, él y su
familia se refugiaron en Inglaterra. A su regreso a Suiza, vendieron la
mayor parte de sus propiedades y adquirieron una pequeña granja en
Italia, donde trabajaban personalmente. Más tarde, Sismondi regresó a
Ginebra, donde escribió numerosas obras eruditas, incluyendo ``History
of the Italian Republic of the Middle Ages'' en dieciséis tomos, y
``History of the French'' en veintinueve tomos.

Las principales ideas y propuestas de Sismondi fueron las siguientes:

\begin{itemize}
\tightlist
\item
  Manifestó que la empresa capitalista sin restricciones no producía los
  resultados esperados por Adam Smith y Jean-Baptiste Say, sino que
  estaba destinada a generar amplia miseria y desempleo.
\item
  Contribuyó a la teoría del ciclo de negocios al plantear la
  posibilidad de la sobreproducción y las crisis. Creía que cuando los
  salarios se mantienen en el nivel de subsistencia, se liberan fondos
  de capital para invertir en maquinaria.
\item
  Sostenía que solo la intervención del Estado podía garantizar un
  salario suficiente para vivir y una seguridad social mínima para los
  trabajadores. Sismondi negaba que la mayor producción posible
  coincidiera necesariamente con la mayor felicidad de las personas.
  Consideraba que una producción más pequeña pero bien distribuida era
  preferible para el interés general.
\item
  Abogaba por promover la agricultura en lugar de la urbanización.
  Apoyaba impuestos sobre herencias, la descontinuación de los derechos
  de patente para frenar nuevos inventos y enfriar el fervor por los
  descubrimientos, instar a los empleadores a proporcionar seguridad a
  los trabajadores en su vejez, enfermedad y desempleo, fomentar la
  cooperación y solidaridad entre trabajadores y empleadores, y
  compartir las ganancias.
\item
  Fue el primero en aplicar el término ``proletario'' para referirse al
  trabajador asalariado, refiriéndose a aquellos que no tenían nada, no
  pagaban impuestos y solo podían contribuir al país mediante su
  progenie, es decir, su prole.
\end{itemize}

\hypertarget{robert-owen-1771-1858}{%
\subsection{Robert Owen (1771-1858):}\label{robert-owen-1771-1858}}

Robert Owen fue el socialista utópico más espectacular y famoso. Hijo de
un quincallero y talabartero galés, asistió a la escuela solo durante
unos años. A los nueve años, comenzó a trabajar como ayudante en una
tienda del vecindario y más adelante en tiendas de mercancías generales
en Londres. A los dieciocho años, solicitó un préstamo de 100 libras y
se asoció con un mecánico capaz de construir la recién inventada
maquinaria textil.

Las principales ideas y propuestas de Robert Owen fueron las siguientes:

\begin{itemize}
\tightlist
\item
  Su tesis fundamental sostenía que el entorno modela la naturaleza
  humana, ya sea para bien o para mal. Según Owen, los seres humanos no
  tienen la capacidad de moldear sus propios caracteres, sino que son
  moldeados por las circunstancias que los rodean. Debido a que el
  carácter está determinado por estas circunstancias, las personas no
  son verdaderamente responsables de sus acciones. Por lo tanto, Owen
  abogaba por moldear a las personas hacia el bien en lugar de
  castigarlas por sus acciones negativas. Tanto Owen como Fourier
  basaban todas sus teorías, sueños y programas en la creencia de que
  proporcionar mejores condiciones de trabajo resultaría en el
  desarrollo de individuos mejores. Owen creía firmemente en que todos
  deberían esforzarse por servir a la comunidad como medio para alcanzar
  la máxima felicidad personal.
\item
  Owen fue pionero en la implementación de salarios de eficiencia, que
  consistían en ofrecer un salario por encima del nivel de mercado, con
  el objetivo de mejorar la productividad y reducir la rotación del
  personal.
\item
  Abogaba por la adopción de una tasa de interés fija, con la esperanza
  de que los propietarios de capital la abandonaran voluntariamente en
  el futuro.
\item
  Su influencia en el socialismo, las cooperativas y el sindicalismo fue
  significativa. La palabra ``socialismo'', en su sentido moderno, se
  utilizó por primera vez en la revista oweniana ``Co-operative
  Magazine'' en 1827 para referirse a los seguidores de las doctrinas
  cooperativistas de Owen.
\item
\end{itemize}

\hypertarget{louis-blanc-1811-1882}{%
\subsection{Louis Blanc (1811-1882)}\label{louis-blanc-1811-1882}}

Considerado uno de los fundadores del socialismo de Estado. Fue un
reformador, periodista e historiador francés proveniente de una familia
monárquica. Durante la Revolución Francesa, su abuelo, un próspero
comerciante, fue guillotinado, lo que llevó a la empobrecimiento de su
familia tras la caída de Napoleón. La publicación de su obra
``Organisation du Travail'' en 1839 lo catapultó a la fama y le otorgó
una posición de liderazgo en el movimiento socialista. En la revolución
de 1848, fue elegido para formar parte del gobierno provisional que
derrocó a la monarquía, convirtiéndose en el primer socialista
reconocido en ocupar un cargo público en cualquier parte. Bajo la
presión de Blanc y sus seguidores, el gobierno estableció talleres
nacionales con el fin de brindar empleo a los desempleados.

\begin{itemize}
\tightlist
\item
  Blanc sostenía que el sufragio universal transformaría al Estado en un
  instrumento de progreso y bienestar.
\item
  Propugnaba que el Estado se convirtiera en el ``banquero de los
  pobres'' y abogaba por la creación de un banco de propiedad pública
  encargado de distribuir crédito entre las cooperativas.
\item
  Planteaba que los capitalistas podrían unirse a las asociaciones y
  recibir una tasa de interés fija sobre su capital, garantizada por el
  Estado.
\item
  Según Blanc, bajo el reinado de la libre competencia, los salarios
  tenderían a disminuir continuamente, lo que consideraba una ley
  general. Ante el crecimiento demográfico, abogaba por instar a las
  madres de familias pobres a ser estériles, lo cual consideraba una
  lucha necesaria debido a la limitación de recursos.
\item
  Proponía que el gobierno se considerara como el regulador supremo de
  la producción y se le otorgara un gran poder para cumplir con esta
  función. Esto implicaba combatir y superar la competencia. Blanc
  abogaba por que el gobierno emitiese un préstamo para establecer
  talleres sociales en las ramas más importantes de la industria
  nacional.
\end{itemize}

\hypertarget{charles-kingsley-1819-1875}{%
\subsection{Charles Kingsley
(1819-1875)}\label{charles-kingsley-1819-1875}}

Fue un clérigo, poeta, novelista y reformador. Además de ser profesor de
historia moderna en Cambridge y canónigo de Westminster, Kingsley y
otros socialistas se esforzaron por socializar a los cristianos y
cristianizar a los socialistas.

En 1848, los cristianos publicaron un diario semanal llamado ``Politics
for the People''. Kingsley escribió una serie de cartas dirigidas a los
cartistas bajo el seudónimo del párroco Lot. En su segunda carta,
realizó una apasionada defensa de los pobres.

En su tercera carta, Kingsley adoptó una postura más moderada como un
socialista cristiano típico. Escribió lo siguiente: ``Amigos míos, debo
decirles que en la Biblia encontrarán lo que anhelan, prometido con una
justicia aún mayor de la que cualquier hombre les haya prometido en
estos tiempos. Encontrarán en el libro lo que buscan. ¿Cuáles son las
cosas que ustedes desean con más fervor? ¿No es una de ellas que nadie
reciba un salario sin trabajar? La Biblia declara inmediatamente que
aquel que no trabaje, no comerá. Y como la Biblia se dirige tanto a los
ricos como a los pobres, estas palabras se aplican tanto a los ricos
ociosos como a los pobres ociosos.''

En otras partes de su obra, Kingsley escribió: ``Dios solo reformará a
la sociedad si nosotros nos reformamos a nosotros mismos. Mientras
tanto, el demonio está ansioso por ayudarnos a cambiar las leyes, el
parlamento, la tierra y el cielo, sin iniciar nunca esa petición tan
impertinente y personal''.

La postura de Kingsley combinaba su fe cristiana con su compromiso con
la justicia social y la preocupación por los menos privilegiados. Su
trabajo y sus escritos reflejan su convicción de que el cambio social
debe basarse en una transformación personal y una responsabilidad
colectiva.


\printbibliography


\end{document}
