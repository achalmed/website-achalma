% Options for packages loaded elsewhere
\PassOptionsToPackage{unicode}{hyperref}
\PassOptionsToPackage{hyphens}{url}
\PassOptionsToPackage{dvipsnames,svgnames,x11names}{xcolor}
%
\documentclass[
  a4paper,
]{article}

\usepackage{amsmath,amssymb}
\usepackage{iftex}
\ifPDFTeX
  \usepackage[T1]{fontenc}
  \usepackage[utf8]{inputenc}
  \usepackage{textcomp} % provide euro and other symbols
\else % if luatex or xetex
  \usepackage{unicode-math}
  \defaultfontfeatures{Scale=MatchLowercase}
  \defaultfontfeatures[\rmfamily]{Ligatures=TeX,Scale=1}
\fi
\usepackage{lmodern}
\ifPDFTeX\else  
    % xetex/luatex font selection
\fi
% Use upquote if available, for straight quotes in verbatim environments
\IfFileExists{upquote.sty}{\usepackage{upquote}}{}
\IfFileExists{microtype.sty}{% use microtype if available
  \usepackage[]{microtype}
  \UseMicrotypeSet[protrusion]{basicmath} % disable protrusion for tt fonts
}{}
\makeatletter
\@ifundefined{KOMAClassName}{% if non-KOMA class
  \IfFileExists{parskip.sty}{%
    \usepackage{parskip}
  }{% else
    \setlength{\parindent}{0pt}
    \setlength{\parskip}{6pt plus 2pt minus 1pt}}
}{% if KOMA class
  \KOMAoptions{parskip=half}}
\makeatother
\usepackage{xcolor}
\usepackage[top=2.54cm,right=2.54cm,bottom=2.54cm,left=2.54cm]{geometry}
\setlength{\emergencystretch}{3em} % prevent overfull lines
\setcounter{secnumdepth}{-\maxdimen} % remove section numbering
% Make \paragraph and \subparagraph free-standing
\ifx\paragraph\undefined\else
  \let\oldparagraph\paragraph
  \renewcommand{\paragraph}[1]{\oldparagraph{#1}\mbox{}}
\fi
\ifx\subparagraph\undefined\else
  \let\oldsubparagraph\subparagraph
  \renewcommand{\subparagraph}[1]{\oldsubparagraph{#1}\mbox{}}
\fi


\providecommand{\tightlist}{%
  \setlength{\itemsep}{0pt}\setlength{\parskip}{0pt}}\usepackage{longtable,booktabs,array}
\usepackage{calc} % for calculating minipage widths
% Correct order of tables after \paragraph or \subparagraph
\usepackage{etoolbox}
\makeatletter
\patchcmd\longtable{\par}{\if@noskipsec\mbox{}\fi\par}{}{}
\makeatother
% Allow footnotes in longtable head/foot
\IfFileExists{footnotehyper.sty}{\usepackage{footnotehyper}}{\usepackage{footnote}}
\makesavenoteenv{longtable}
\usepackage{graphicx}
\makeatletter
\def\maxwidth{\ifdim\Gin@nat@width>\linewidth\linewidth\else\Gin@nat@width\fi}
\def\maxheight{\ifdim\Gin@nat@height>\textheight\textheight\else\Gin@nat@height\fi}
\makeatother
% Scale images if necessary, so that they will not overflow the page
% margins by default, and it is still possible to overwrite the defaults
% using explicit options in \includegraphics[width, height, ...]{}
\setkeys{Gin}{width=\maxwidth,height=\maxheight,keepaspectratio}
% Set default figure placement to htbp
\makeatletter
\def\fps@figure{htbp}
\makeatother

\makeatletter
\makeatother
\makeatletter
\makeatother
\makeatletter
\@ifpackageloaded{caption}{}{\usepackage{caption}}
\AtBeginDocument{%
\ifdefined\contentsname
  \renewcommand*\contentsname{Tabla de contenidos}
\else
  \newcommand\contentsname{Tabla de contenidos}
\fi
\ifdefined\listfigurename
  \renewcommand*\listfigurename{Listado de Figuras}
\else
  \newcommand\listfigurename{Listado de Figuras}
\fi
\ifdefined\listtablename
  \renewcommand*\listtablename{Listado de Tablas}
\else
  \newcommand\listtablename{Listado de Tablas}
\fi
\ifdefined\figurename
  \renewcommand*\figurename{Figura}
\else
  \newcommand\figurename{Figura}
\fi
\ifdefined\tablename
  \renewcommand*\tablename{Tabla}
\else
  \newcommand\tablename{Tabla}
\fi
}
\@ifpackageloaded{float}{}{\usepackage{float}}
\floatstyle{ruled}
\@ifundefined{c@chapter}{\newfloat{codelisting}{h}{lop}}{\newfloat{codelisting}{h}{lop}[chapter]}
\floatname{codelisting}{Listado}
\newcommand*\listoflistings{\listof{codelisting}{Listado de Listados}}
\makeatother
\makeatletter
\@ifpackageloaded{caption}{}{\usepackage{caption}}
\@ifpackageloaded{subcaption}{}{\usepackage{subcaption}}
\makeatother
\makeatletter
\@ifpackageloaded{tcolorbox}{}{\usepackage[skins,breakable]{tcolorbox}}
\makeatother
\makeatletter
\@ifundefined{shadecolor}{\definecolor{shadecolor}{rgb}{.97, .97, .97}}
\makeatother
\makeatletter
\makeatother
\makeatletter
\makeatother
\ifLuaTeX
\usepackage[bidi=basic]{babel}
\else
\usepackage[bidi=default]{babel}
\fi
\babelprovide[main,import]{spanish}
% get rid of language-specific shorthands (see #6817):
\let\LanguageShortHands\languageshorthands
\def\languageshorthands#1{}
\ifLuaTeX
  \usepackage{selnolig}  % disable illegal ligatures
\fi
\usepackage[]{biblatex}
\addbibresource{../../../../references.bib}
\IfFileExists{bookmark.sty}{\usepackage{bookmark}}{\usepackage{hyperref}}
\IfFileExists{xurl.sty}{\usepackage{xurl}}{} % add URL line breaks if available
\urlstyle{same} % disable monospaced font for URLs
\hypersetup{
  pdftitle={Análisis estadístico de los alumnos de Estadística de la serie 200 de Economía 2018-I},
  pdfauthor={Edison Achalma},
  pdflang={es},
  colorlinks=true,
  linkcolor={blue},
  filecolor={Maroon},
  citecolor={Blue},
  urlcolor={Blue},
  pdfcreator={LaTeX via pandoc}}

\title{Análisis estadístico de los alumnos de Estadística de la serie
200 de Economía 2018-I}
\usepackage{etoolbox}
\makeatletter
\providecommand{\subtitle}[1]{% add subtitle to \maketitle
  \apptocmd{\@title}{\par {\large #1 \par}}{}{}
}
\makeatother
\subtitle{Explorando los estadígrafos}
\author{Edison Achalma}
\date{2018-04-16}

\begin{document}
\maketitle
\ifdefined\Shaded\renewenvironment{Shaded}{\begin{tcolorbox}[boxrule=0pt, interior hidden, sharp corners, enhanced, borderline west={3pt}{0pt}{shadecolor}, breakable, frame hidden]}{\end{tcolorbox}}\fi

\hypertarget{tbl-1}{}
\begin{longtable}[]{@{}
  >{\centering\arraybackslash}p{(\columnwidth - 8\tabcolsep) * \real{0.1094}}
  >{\centering\arraybackslash}p{(\columnwidth - 8\tabcolsep) * \real{0.1562}}
  >{\centering\arraybackslash}p{(\columnwidth - 8\tabcolsep) * \real{0.1562}}
  >{\centering\arraybackslash}p{(\columnwidth - 8\tabcolsep) * \real{0.2656}}
  >{\centering\arraybackslash}p{(\columnwidth - 8\tabcolsep) * \real{0.3125}}@{}}
\caption{\label{tbl-1}Distribución de la edad de los estudiantes de la
serie 200 de Economía durante el curso de Estadística en el período
2018-I}\tabularnewline
\toprule\noalign{}
\begin{minipage}[b]{\linewidth}\centering
Valores
\end{minipage} & \begin{minipage}[b]{\linewidth}\centering
Frecuencia
\end{minipage} & \begin{minipage}[b]{\linewidth}\centering
Porcentaje
\end{minipage} & \begin{minipage}[b]{\linewidth}\centering
Porcentaje válido
\end{minipage} & \begin{minipage}[b]{\linewidth}\centering
Porcentaje acumulado
\end{minipage} \\
\midrule\noalign{}
\endfirsthead
\toprule\noalign{}
\begin{minipage}[b]{\linewidth}\centering
Valores
\end{minipage} & \begin{minipage}[b]{\linewidth}\centering
Frecuencia
\end{minipage} & \begin{minipage}[b]{\linewidth}\centering
Porcentaje
\end{minipage} & \begin{minipage}[b]{\linewidth}\centering
Porcentaje válido
\end{minipage} & \begin{minipage}[b]{\linewidth}\centering
Porcentaje acumulado
\end{minipage} \\
\midrule\noalign{}
\endhead
\bottomrule\noalign{}
\endlastfoot
10 & 1 & 0,9 & 0,9 & 0,9 \\
17 & 1 & 0,9 & 0,9 & 1,8 \\
18 & 6 & 5,4 & 5,4 & 7,2 \\
19 & 25 & 22,5 & 22,5 & 29,7 \\
20 & 20 & 18,0 & 18,0 & 47,7 \\
21 & 22 & 19,8 & 19,8 & 67,6 \\
22 & 13 & 11,7 & 11,7 & 79,3 \\
23 & 5 & 4,5 & 4,5 & 83,8 \\
24 & 3 & 2,7 & 2,7 & 86,5 \\
25 & 7 & 6,3 & 6,3 & 92,8 \\
26 & 2 & 1,8 & 1,8 & 94,6 \\
27 & 2 & 1,8 & 1,8 & 96,4 \\
28 & 1 & 0,9 & 0,9 & 97,3 \\
29 & 2 & 1,8 & 1,8 & 99,1 \\
42 & 1 & 0,9 & 0,9 & 100,0 \\
Total & 111 & 100,0 & 100,0 & \\
\end{longtable}

El cuadro muestra la distribución de la edad de los estudiantes de la
serie 200 de Economía en el periodo 2018-I. A continuación, se
proporciona una interpretación de las columnas del cuadro:

\begin{itemize}
\tightlist
\item
  Frecuencia: Indica la cantidad de estudiantes que se encuentran en
  cada grupo de edad.
\item
  Porcentaje: Representa el porcentaje de estudiantes que corresponde a
  cada grupo de edad con respecto al total de estudiantes.
\item
  Porcentaje válido: Muestra el porcentaje de estudiantes excluyendo
  aquellos valores nulos o no aplicables.
\item
  Porcentaje acumulado: Indica el porcentaje acumulado de estudiantes
  desde el inicio hasta cada grupo de edad.
\end{itemize}

Interpretación:

\begin{itemize}
\tightlist
\item
  Se observa que la mayoría de los estudiantes se encuentran en un rango
  de edad entre los 19 y los 25 años, con una mayor concentración en la
  edad de 19 años (22.5\% de los estudiantes) y 21 años (19.8\% de los
  estudiantes).
\item
  A medida que aumenta la edad, la cantidad de estudiantes tiende a
  disminuir.
\item
  En total, se registraron 111 estudiantes en la muestra.
\end{itemize}

\hypertarget{tbl-2}{}
\begin{longtable}[]{@{}
  >{\centering\arraybackslash}p{(\columnwidth - 8\tabcolsep) * \real{0.1094}}
  >{\centering\arraybackslash}p{(\columnwidth - 8\tabcolsep) * \real{0.1562}}
  >{\centering\arraybackslash}p{(\columnwidth - 8\tabcolsep) * \real{0.1562}}
  >{\centering\arraybackslash}p{(\columnwidth - 8\tabcolsep) * \real{0.2656}}
  >{\centering\arraybackslash}p{(\columnwidth - 8\tabcolsep) * \real{0.3125}}@{}}
\caption{\label{tbl-2}Distribución de los estudiantes de la serie 200 de
Economía que cursan Estadística durante el período
2018-I}\tabularnewline
\toprule\noalign{}
\begin{minipage}[b]{\linewidth}\centering
Valores
\end{minipage} & \begin{minipage}[b]{\linewidth}\centering
Frecuencia
\end{minipage} & \begin{minipage}[b]{\linewidth}\centering
Porcentaje
\end{minipage} & \begin{minipage}[b]{\linewidth}\centering
Porcentaje válido
\end{minipage} & \begin{minipage}[b]{\linewidth}\centering
Porcentaje acumulado
\end{minipage} \\
\midrule\noalign{}
\endfirsthead
\toprule\noalign{}
\begin{minipage}[b]{\linewidth}\centering
Valores
\end{minipage} & \begin{minipage}[b]{\linewidth}\centering
Frecuencia
\end{minipage} & \begin{minipage}[b]{\linewidth}\centering
Porcentaje
\end{minipage} & \begin{minipage}[b]{\linewidth}\centering
Porcentaje válido
\end{minipage} & \begin{minipage}[b]{\linewidth}\centering
Porcentaje acumulado
\end{minipage} \\
\midrule\noalign{}
\endhead
\bottomrule\noalign{}
\endlastfoot
200 & 105 & 94,6 & 94,6 & 94,6 \\
300 & 6 & 5,4 & 5,4 & 100,0 \\
Total & 111 & 100,0 & 100,0 & \\
\end{longtable}

El cuadro muestra información sobre el número de estudiantes de Economía
que están llevando el curso de Estadística de la serie 200 en el periodo
2018-I. A continuación, se proporciona una interpretación de las
columnas del cuadro:

\begin{itemize}
\tightlist
\item
  Frecuencia: Indica la cantidad de estudiantes que están tomando el
  curso de Estadística.
\item
  Porcentaje: Representa el porcentaje de estudiantes que corresponde a
  cada categoría con respecto al total de estudiantes.
\item
  Porcentaje válido: Muestra el porcentaje de estudiantes excluyendo
  aquellos valores nulos o no aplicables.
\item
  Porcentaje acumulado: Indica el porcentaje acumulado de estudiantes
  desde el inicio hasta cada categoría.
\end{itemize}

Interpretación:

\begin{itemize}
\tightlist
\item
  Del total de estudiantes de la serie 200 de Economía en el periodo
  2018-I, se encontró que 105 estudiantes (94.6\%) están llevando el
  curso de Estadística.
\item
  Además, hay 6 estudiantes (5.4\%) que no están llevando el curso de
  Estadística.
\item
  En resumen, se registraron 111 estudiantes en total en la muestra.
\end{itemize}

\hypertarget{tbl-3}{}
\begin{longtable}[]{@{}
  >{\centering\arraybackslash}p{(\columnwidth - 8\tabcolsep) * \real{0.2400}}
  >{\centering\arraybackslash}p{(\columnwidth - 8\tabcolsep) * \real{0.1333}}
  >{\centering\arraybackslash}p{(\columnwidth - 8\tabcolsep) * \real{0.1333}}
  >{\centering\arraybackslash}p{(\columnwidth - 8\tabcolsep) * \real{0.2267}}
  >{\centering\arraybackslash}p{(\columnwidth - 8\tabcolsep) * \real{0.2667}}@{}}
\caption{\label{tbl-3}Distribución del tiempo de llegada de los
estudiantes de la serie 200 de Economía que cursan Estadística durante
el período 2018-I}\tabularnewline
\toprule\noalign{}
\begin{minipage}[b]{\linewidth}\centering
Valores
\end{minipage} & \begin{minipage}[b]{\linewidth}\centering
Frecuencia
\end{minipage} & \begin{minipage}[b]{\linewidth}\centering
Porcentaje
\end{minipage} & \begin{minipage}[b]{\linewidth}\centering
Porcentaje válido
\end{minipage} & \begin{minipage}[b]{\linewidth}\centering
Porcentaje acumulado
\end{minipage} \\
\midrule\noalign{}
\endfirsthead
\toprule\noalign{}
\begin{minipage}[b]{\linewidth}\centering
Valores
\end{minipage} & \begin{minipage}[b]{\linewidth}\centering
Frecuencia
\end{minipage} & \begin{minipage}[b]{\linewidth}\centering
Porcentaje
\end{minipage} & \begin{minipage}[b]{\linewidth}\centering
Porcentaje válido
\end{minipage} & \begin{minipage}[b]{\linewidth}\centering
Porcentaje acumulado
\end{minipage} \\
\midrule\noalign{}
\endhead
\bottomrule\noalign{}
\endlastfoot
2 & 1 & 0,9 & 0,9 & 0,9 \\
5 & 1 & 0,9 & 0,9 & 1,8 \\
10 & 6 & 5,4 & 5,5 & 7,3 \\
12 & 1 & 0,9 & 0,9 & 8,2 \\
15 & 15 & 13,5 & 13,6 & 21,8 \\
20 & 21 & 18,9 & 19,1 & 40,9 \\
25 & 15 & 13,5 & 13,6 & 54,5 \\
27 & 1 & 0,9 & 0,9 & 55,5 \\
30 & 28 & 25,2 & 25,5 & 80,9 \\
35 & 4 & 3,6 & 3,6 & 84,5 \\
40 & 7 & 6,3 & 6,4 & 90,9 \\
45 & 6 & 5,4 & 5,5 & 96,4 \\
60 & 3 & 2,7 & 2,7 & 99,1 \\
70 & 1 & 0,9 & 0,9 & 100,0 \\
Total & 110 & 99,1 & 100,0 & \\
Perdidos - Sistema & 1 & 0,9 & & \\
Total & 111 & 100,0 & & \\
\end{longtable}

El cuadro presenta datos sobre el tiempo de llegada de los alumnos que
están tomando el curso de Estadística de la serie 200 en Economía
durante el periodo 2018-I. A continuación, se realiza una interpretación
de las columnas del cuadro:

Interpretación:

\begin{itemize}
\tightlist
\item
  Según los datos, se observa que el tiempo de llegada más común para
  los estudiantes es de 20 minutos, con 21 estudiantes (18.9\%) llegando
  en ese intervalo.
\item
  El segundo tiempo de llegada más frecuente es de 30 minutos, con 28
  estudiantes (25.2\%) llegando en ese lapso.
\item
  En general, la mayoría de los estudiantes (99.1\%) llega dentro de un
  rango de tiempo de 2 a 70 minutos.
\item
  También se indica que hay un estudiante que no especificó su tiempo de
  llegada.
\end{itemize}

\hypertarget{tbl-4}{}
\begin{longtable}[]{@{}
  >{\centering\arraybackslash}p{(\columnwidth - 8\tabcolsep) * \real{0.1094}}
  >{\centering\arraybackslash}p{(\columnwidth - 8\tabcolsep) * \real{0.1562}}
  >{\centering\arraybackslash}p{(\columnwidth - 8\tabcolsep) * \real{0.1562}}
  >{\centering\arraybackslash}p{(\columnwidth - 8\tabcolsep) * \real{0.2656}}
  >{\centering\arraybackslash}p{(\columnwidth - 8\tabcolsep) * \real{0.3125}}@{}}
\caption{\label{tbl-4}Distribución del número de cursos matriculados por
los estudiantes de la serie 200 de Economía que cursan Estadística
durante el período 2018-I}\tabularnewline
\toprule\noalign{}
\begin{minipage}[b]{\linewidth}\centering
Valores
\end{minipage} & \begin{minipage}[b]{\linewidth}\centering
Frecuencia
\end{minipage} & \begin{minipage}[b]{\linewidth}\centering
Porcentaje
\end{minipage} & \begin{minipage}[b]{\linewidth}\centering
Porcentaje válido
\end{minipage} & \begin{minipage}[b]{\linewidth}\centering
Porcentaje acumulado
\end{minipage} \\
\midrule\noalign{}
\endfirsthead
\toprule\noalign{}
\begin{minipage}[b]{\linewidth}\centering
Valores
\end{minipage} & \begin{minipage}[b]{\linewidth}\centering
Frecuencia
\end{minipage} & \begin{minipage}[b]{\linewidth}\centering
Porcentaje
\end{minipage} & \begin{minipage}[b]{\linewidth}\centering
Porcentaje válido
\end{minipage} & \begin{minipage}[b]{\linewidth}\centering
Porcentaje acumulado
\end{minipage} \\
\midrule\noalign{}
\endhead
\bottomrule\noalign{}
\endlastfoot
2 & 1 & 0,9 & 0,9 & 0,9 \\
3 & 1 & 0,9 & 0,9 & 1,8 \\
4 & 12 & 10,8 & 10,8 & 12,6 \\
5 & 32 & 28,8 & 28,8 & 41,4 \\
6 & 57 & 51,4 & 51,4 & 92,8 \\
7 & 6 & 5,4 & 5,4 & 98,2 \\
8 & 2 & 1,8 & 1,8 & 100,0 \\
Total & 111 & 100,0 & 100,0 & \\
\end{longtable}

El cuadro presenta datos sobre el número de cursos matriculados por los
alumnos que están tomando el curso de Estadística de la serie 200 en
Economía durante el periodo 2018-I. A continuación, se realiza una
interpretación de las columnas del cuadro:

Interpretación:

\begin{itemize}
\tightlist
\item
  Según los datos, se observa que la mayoría de los estudiantes (51.4\%)
  se matricularon en 6 cursos.
\item
  El segundo número de cursos más común es 5, con 32 estudiantes
  (28.8\%) inscribiéndose en esa cantidad.
\item
  En general, la mayoría de los estudiantes (98.2\%) se matriculó en 7
  cursos o menos.
\item
  También se indica que hay 2 estudiantes (1.8\%) que se matricularon en
  8 cursos.
\end{itemize}

\hypertarget{tbl-5}{}
\begin{longtable}[]{@{}
  >{\centering\arraybackslash}p{(\columnwidth - 8\tabcolsep) * \real{0.1094}}
  >{\centering\arraybackslash}p{(\columnwidth - 8\tabcolsep) * \real{0.1562}}
  >{\centering\arraybackslash}p{(\columnwidth - 8\tabcolsep) * \real{0.1562}}
  >{\centering\arraybackslash}p{(\columnwidth - 8\tabcolsep) * \real{0.2656}}
  >{\centering\arraybackslash}p{(\columnwidth - 8\tabcolsep) * \real{0.3125}}@{}}
\caption{\label{tbl-5}Distribución del número de veces que los
estudiantes de la serie 200 de Economía llevan el curso de Estadística
durante el período 2018-I}\tabularnewline
\toprule\noalign{}
\begin{minipage}[b]{\linewidth}\centering
Valores
\end{minipage} & \begin{minipage}[b]{\linewidth}\centering
Frecuencia
\end{minipage} & \begin{minipage}[b]{\linewidth}\centering
Porcentaje
\end{minipage} & \begin{minipage}[b]{\linewidth}\centering
Porcentaje válido
\end{minipage} & \begin{minipage}[b]{\linewidth}\centering
Porcentaje acumulado
\end{minipage} \\
\midrule\noalign{}
\endfirsthead
\toprule\noalign{}
\begin{minipage}[b]{\linewidth}\centering
Valores
\end{minipage} & \begin{minipage}[b]{\linewidth}\centering
Frecuencia
\end{minipage} & \begin{minipage}[b]{\linewidth}\centering
Porcentaje
\end{minipage} & \begin{minipage}[b]{\linewidth}\centering
Porcentaje válido
\end{minipage} & \begin{minipage}[b]{\linewidth}\centering
Porcentaje acumulado
\end{minipage} \\
\midrule\noalign{}
\endhead
\bottomrule\noalign{}
\endlastfoot
0 & 1 & 0,9 & 0,9 & 0,9 \\
1 & 86 & 77,5 & 77,5 & 78,4 \\
2 & 20 & 18,0 & 18,0 & 96,4 \\
3 & 4 & 3,6 & 3,6 & 100,0 \\
Total & 111 & 100,0 & 100,0 & \\
\end{longtable}

El cuadro presenta datos sobre el número de veces que los alumnos de la
serie 200 en Economía durante el periodo 2018-I han llevado el curso de
Estadística. A continuación, se realiza una interpretación de las
columnas del cuadro:

Interpretación:

\begin{itemize}
\tightlist
\item
  Según los datos, se observa que la mayoría de los estudiantes (77.5\%)
  han llevado el curso de Estadística una vez.
\item
  Un número considerable de estudiantes (18.0\%) ha llevado el curso dos
  veces.
\item
  Solo un pequeño porcentaje de estudiantes (3.6\%) ha llevado el curso
  tres veces.
\item
  También se indica que hay 1 estudiante (0.9\%) que no ha llevado el
  curso de Estadística.
\end{itemize}

\hypertarget{tbl-6}{}
\begin{longtable}[]{@{}
  >{\centering\arraybackslash}p{(\columnwidth - 8\tabcolsep) * \real{0.1364}}
  >{\centering\arraybackslash}p{(\columnwidth - 8\tabcolsep) * \real{0.1515}}
  >{\centering\arraybackslash}p{(\columnwidth - 8\tabcolsep) * \real{0.1515}}
  >{\centering\arraybackslash}p{(\columnwidth - 8\tabcolsep) * \real{0.2576}}
  >{\centering\arraybackslash}p{(\columnwidth - 8\tabcolsep) * \real{0.3030}}@{}}
\caption{\label{tbl-6}Distribución del género de los estudiantes de la
serie 200 de Economía que cursan Estadística durante el período
2018-I}\tabularnewline
\toprule\noalign{}
\begin{minipage}[b]{\linewidth}\centering
Valores
\end{minipage} & \begin{minipage}[b]{\linewidth}\centering
Frecuencia
\end{minipage} & \begin{minipage}[b]{\linewidth}\centering
Porcentaje
\end{minipage} & \begin{minipage}[b]{\linewidth}\centering
Porcentaje válido
\end{minipage} & \begin{minipage}[b]{\linewidth}\centering
Porcentaje acumulado
\end{minipage} \\
\midrule\noalign{}
\endfirsthead
\toprule\noalign{}
\begin{minipage}[b]{\linewidth}\centering
Valores
\end{minipage} & \begin{minipage}[b]{\linewidth}\centering
Frecuencia
\end{minipage} & \begin{minipage}[b]{\linewidth}\centering
Porcentaje
\end{minipage} & \begin{minipage}[b]{\linewidth}\centering
Porcentaje válido
\end{minipage} & \begin{minipage}[b]{\linewidth}\centering
Porcentaje acumulado
\end{minipage} \\
\midrule\noalign{}
\endhead
\bottomrule\noalign{}
\endlastfoot
Femenino & 40 & 36,0 & 36,0 & 36,0 \\
Masculino & 71 & 64,0 & 64,0 & 100,0 \\
Total & 111 & 100,0 & 100,0 & \\
\end{longtable}

\hypertarget{tbl-7}{}
\begin{longtable}[]{@{}
  >{\centering\arraybackslash}p{(\columnwidth - 8\tabcolsep) * \real{0.2083}}
  >{\centering\arraybackslash}p{(\columnwidth - 8\tabcolsep) * \real{0.1389}}
  >{\centering\arraybackslash}p{(\columnwidth - 8\tabcolsep) * \real{0.1389}}
  >{\centering\arraybackslash}p{(\columnwidth - 8\tabcolsep) * \real{0.2361}}
  >{\centering\arraybackslash}p{(\columnwidth - 8\tabcolsep) * \real{0.2778}}@{}}
\caption{\label{tbl-7}Distribución de la estatura de los estudiantes de
la serie 200 de Economía que cursan Estadística durante el período
2018-I}\tabularnewline
\toprule\noalign{}
\begin{minipage}[b]{\linewidth}\centering
Valores
\end{minipage} & \begin{minipage}[b]{\linewidth}\centering
Frecuencia
\end{minipage} & \begin{minipage}[b]{\linewidth}\centering
Porcentaje
\end{minipage} & \begin{minipage}[b]{\linewidth}\centering
Porcentaje válido
\end{minipage} & \begin{minipage}[b]{\linewidth}\centering
Porcentaje acumulado
\end{minipage} \\
\midrule\noalign{}
\endfirsthead
\toprule\noalign{}
\begin{minipage}[b]{\linewidth}\centering
Valores
\end{minipage} & \begin{minipage}[b]{\linewidth}\centering
Frecuencia
\end{minipage} & \begin{minipage}[b]{\linewidth}\centering
Porcentaje
\end{minipage} & \begin{minipage}[b]{\linewidth}\centering
Porcentaje válido
\end{minipage} & \begin{minipage}[b]{\linewidth}\centering
Porcentaje acumulado
\end{minipage} \\
\midrule\noalign{}
\endhead
\bottomrule\noalign{}
\endlastfoot
{[}146,98;150,60{]} & 12 & 10,8 & 10,8 & 10,8 \\
{[}150,61;154,23{]} & 10 & 9,0 & 9,0 & 19,8 \\
{[}154,24;157,86{]} & 9 & 8,1 & 8,1 & 27,9 \\
{[}157,87;161,49{]} & 17 & 15,3 & 15,3 & 43,2 \\
{[}161,50;165,12{]} & 28 & 25,2 & 25,2 & 68,5 \\
{[}165,13;168,75{]} & 15 & 13,5 & 13,5 & 82,0 \\
{[}168,76;172,38{]} & 12 & 10,8 & 10,8 & 92,8 \\
{[}172,39;176,02{]} & 8 & 7,2 & 7,2 & 100,0 \\
Total & 111 & 100,0 & 100,0 & \\
\end{longtable}

\hypertarget{tbl-8}{}
\begin{longtable}[]{@{}
  >{\centering\arraybackslash}p{(\columnwidth - 8\tabcolsep) * \real{0.1094}}
  >{\centering\arraybackslash}p{(\columnwidth - 8\tabcolsep) * \real{0.1562}}
  >{\centering\arraybackslash}p{(\columnwidth - 8\tabcolsep) * \real{0.1562}}
  >{\centering\arraybackslash}p{(\columnwidth - 8\tabcolsep) * \real{0.2656}}
  >{\centering\arraybackslash}p{(\columnwidth - 8\tabcolsep) * \real{0.3125}}@{}}
\caption{\label{tbl-8}Distribución del peso de los estudiantes de la
serie 200 de Economía que cursan Estadística durante el período
2018-I}\tabularnewline
\toprule\noalign{}
\begin{minipage}[b]{\linewidth}\centering
Valores
\end{minipage} & \begin{minipage}[b]{\linewidth}\centering
Frecuencia
\end{minipage} & \begin{minipage}[b]{\linewidth}\centering
Porcentaje
\end{minipage} & \begin{minipage}[b]{\linewidth}\centering
Porcentaje válido
\end{minipage} & \begin{minipage}[b]{\linewidth}\centering
Porcentaje acumulado
\end{minipage} \\
\midrule\noalign{}
\endfirsthead
\toprule\noalign{}
\begin{minipage}[b]{\linewidth}\centering
Valores
\end{minipage} & \begin{minipage}[b]{\linewidth}\centering
Frecuencia
\end{minipage} & \begin{minipage}[b]{\linewidth}\centering
Porcentaje
\end{minipage} & \begin{minipage}[b]{\linewidth}\centering
Porcentaje válido
\end{minipage} & \begin{minipage}[b]{\linewidth}\centering
Porcentaje acumulado
\end{minipage} \\
\midrule\noalign{}
\endhead
\bottomrule\noalign{}
\endlastfoot
{[}40;44{]} & 1 & 0,9 & 0,9 & 0,9 \\
{[}45;49{]} & 18 & 16,2 & 16,2 & 17,1 \\
{[}50;54{]} & 19 & 17,1 & 17,1 & 34,2 \\
{[}55;59{]} & 25 & 22,5 & 22,5 & 56,8 \\
{[}60;64{]} & 25 & 22,5 & 22,5 & 79,3 \\
{[}65;69{]} & 15 & 13,5 & 13,5 & 92,8 \\
{[}70;74{]} & 2 & 1,8 & 1,8 & 94,6 \\
{[}75;80{]} & 6 & 5,4 & 5,4 & 100,0 \\
Total & 111 & 100,0 & 100,0 & \\
\end{longtable}

\hypertarget{tbl-9}{}
\begin{longtable}[]{@{}
  >{\raggedright\arraybackslash}p{(\columnwidth - 8\tabcolsep) * \real{0.2785}}
  >{\centering\arraybackslash}p{(\columnwidth - 8\tabcolsep) * \real{0.1266}}
  >{\centering\arraybackslash}p{(\columnwidth - 8\tabcolsep) * \real{0.1266}}
  >{\centering\arraybackslash}p{(\columnwidth - 8\tabcolsep) * \real{0.2152}}
  >{\centering\arraybackslash}p{(\columnwidth - 8\tabcolsep) * \real{0.2532}}@{}}
\caption{\label{tbl-9}Distribución del distrito de residencia de los
estudiantes de la serie 200 de Economía que cursan Estadística durante
el período 2018-I}\tabularnewline
\toprule\noalign{}
\begin{minipage}[b]{\linewidth}\raggedright
Distrito
\end{minipage} & \begin{minipage}[b]{\linewidth}\centering
Frecuencia
\end{minipage} & \begin{minipage}[b]{\linewidth}\centering
Porcentaje
\end{minipage} & \begin{minipage}[b]{\linewidth}\centering
Porcentaje válido
\end{minipage} & \begin{minipage}[b]{\linewidth}\centering
Porcentaje acumulado
\end{minipage} \\
\midrule\noalign{}
\endfirsthead
\toprule\noalign{}
\begin{minipage}[b]{\linewidth}\raggedright
Distrito
\end{minipage} & \begin{minipage}[b]{\linewidth}\centering
Frecuencia
\end{minipage} & \begin{minipage}[b]{\linewidth}\centering
Porcentaje
\end{minipage} & \begin{minipage}[b]{\linewidth}\centering
Porcentaje válido
\end{minipage} & \begin{minipage}[b]{\linewidth}\centering
Porcentaje acumulado
\end{minipage} \\
\midrule\noalign{}
\endhead
\bottomrule\noalign{}
\endlastfoot
AYACUCHO & 42 & 37,8 & 38,5 & 38,5 \\
SAN JUAN BAUTISTA & 28 & 25,2 & 25,7 & 64,2 \\
CARMEN ALTO & 8 & 7,2 & 7,3 & 71,6 \\
JESÚS NAZARENO & 25 & 22,5 & 22,9 & 94,5 \\
ANDRÉS AVELINO CÁCERES & 5 & 4,5 & 4,6 & 99,1 \\
HUANTA & 1 & 0,9 & 0,9 & 100,0 \\
Total & 109 & 98,2 & 100,0 & \\
Perdidos - Sistema & 2 & 1,8 & & \\
Total & 111 & 100,0 & & \\
\end{longtable}

\hypertarget{tbl-10}{}
\begin{longtable}[]{@{}
  >{\centering\arraybackslash}p{(\columnwidth - 8\tabcolsep) * \real{0.2400}}
  >{\centering\arraybackslash}p{(\columnwidth - 8\tabcolsep) * \real{0.1333}}
  >{\centering\arraybackslash}p{(\columnwidth - 8\tabcolsep) * \real{0.1333}}
  >{\centering\arraybackslash}p{(\columnwidth - 8\tabcolsep) * \real{0.2267}}
  >{\centering\arraybackslash}p{(\columnwidth - 8\tabcolsep) * \real{0.2667}}@{}}
\caption{\label{tbl-10}Distribución del índice académico de los
estudiantes de la serie 200 de Economía que cursan Estadística durante
el período 2018-I}\tabularnewline
\toprule\noalign{}
\begin{minipage}[b]{\linewidth}\centering
Índice académico
\end{minipage} & \begin{minipage}[b]{\linewidth}\centering
Frecuencia
\end{minipage} & \begin{minipage}[b]{\linewidth}\centering
Porcentaje
\end{minipage} & \begin{minipage}[b]{\linewidth}\centering
Porcentaje válido
\end{minipage} & \begin{minipage}[b]{\linewidth}\centering
Porcentaje acumulado
\end{minipage} \\
\midrule\noalign{}
\endfirsthead
\toprule\noalign{}
\begin{minipage}[b]{\linewidth}\centering
Índice académico
\end{minipage} & \begin{minipage}[b]{\linewidth}\centering
Frecuencia
\end{minipage} & \begin{minipage}[b]{\linewidth}\centering
Porcentaje
\end{minipage} & \begin{minipage}[b]{\linewidth}\centering
Porcentaje válido
\end{minipage} & \begin{minipage}[b]{\linewidth}\centering
Porcentaje acumulado
\end{minipage} \\
\midrule\noalign{}
\endhead
\bottomrule\noalign{}
\endlastfoot
{[}4,98;6,18{]} & 4 & 3,6 & 3,7 & 3,7 \\
{[}6,18;7,38{]} & 1 & 0,9 & 0,9 & 4,6 \\
{[}7,38;8,58{]} & 2 & 1,8 & 1,9 & 6,5 \\
{[}8,58;9,78{]} & 6 & 5,4 & 5,6 & 12,0 \\
{[}9,78;10,98{]} & 28 & 25,2 & 25,9 & 38,0 \\
{[}10,98;12,18{]} & 42 & 37,8 & 38,9 & 76,9 \\
{[}12,18;13,38{]} & 18 & 16,2 & 16,7 & 93,5 \\
{[}13,38;14,58{]} & 7 & 6,3 & 6,5 & 100,0 \\
Total & 108 & 97,3 & 100,0 & \\
Perdidos - Sistema & 3 & 2,7 & & \\
Total & 111 & 100,0 & & \\
\end{longtable}

\hypertarget{tbl-11}{}
\begin{longtable}[]{@{}
  >{\raggedright\arraybackslash}p{(\columnwidth - 8\tabcolsep) * \real{0.2400}}
  >{\centering\arraybackslash}p{(\columnwidth - 8\tabcolsep) * \real{0.1333}}
  >{\centering\arraybackslash}p{(\columnwidth - 8\tabcolsep) * \real{0.1333}}
  >{\centering\arraybackslash}p{(\columnwidth - 8\tabcolsep) * \real{0.2267}}
  >{\centering\arraybackslash}p{(\columnwidth - 8\tabcolsep) * \real{0.2667}}@{}}
\caption{\label{tbl-11}Distribución del intervalo de notas de los
estudiantes de la serie 200 de Economía que cursan Estadística durante
el período 2018-I}\tabularnewline
\toprule\noalign{}
\begin{minipage}[b]{\linewidth}\raggedright
Intervalo de notas
\end{minipage} & \begin{minipage}[b]{\linewidth}\centering
Frecuencia
\end{minipage} & \begin{minipage}[b]{\linewidth}\centering
Porcentaje
\end{minipage} & \begin{minipage}[b]{\linewidth}\centering
Porcentaje válido
\end{minipage} & \begin{minipage}[b]{\linewidth}\centering
Porcentaje acumulado
\end{minipage} \\
\midrule\noalign{}
\endfirsthead
\toprule\noalign{}
\begin{minipage}[b]{\linewidth}\raggedright
Intervalo de notas
\end{minipage} & \begin{minipage}[b]{\linewidth}\centering
Frecuencia
\end{minipage} & \begin{minipage}[b]{\linewidth}\centering
Porcentaje
\end{minipage} & \begin{minipage}[b]{\linewidth}\centering
Porcentaje válido
\end{minipage} & \begin{minipage}[b]{\linewidth}\centering
Porcentaje acumulado
\end{minipage} \\
\midrule\noalign{}
\endhead
\bottomrule\noalign{}
\endlastfoot
DEFICIENTE & 3 & 2,7 & 2,8 & 2,8 \\
BAJO & 28 & 25,2 & 25,9 & 28,7 \\
REGULAR & 76 & 68,5 & 70,4 & 99,1 \\
BUENO & 1 & 0,9 & 0,9 & 100,0 \\
Total & 108 & 97,3 & 100,0 & \\
Perdidos - Sistema & 3 & 2,7 & & \\
Total & 111 & 100,0 & & \\
\end{longtable}

\hypertarget{tbl-12}{}
\begin{longtable}[]{@{}
  >{\raggedright\arraybackslash}p{(\columnwidth - 8\tabcolsep) * \real{0.2400}}
  >{\centering\arraybackslash}p{(\columnwidth - 8\tabcolsep) * \real{0.1333}}
  >{\centering\arraybackslash}p{(\columnwidth - 8\tabcolsep) * \real{0.1333}}
  >{\centering\arraybackslash}p{(\columnwidth - 8\tabcolsep) * \real{0.2267}}
  >{\centering\arraybackslash}p{(\columnwidth - 8\tabcolsep) * \real{0.2667}}@{}}
\caption{\label{tbl-12}Distribución del departamento del colegio de los
estudiantes de la serie 200 de Economía que cursan Estadística durante
el período 2018-I}\tabularnewline
\toprule\noalign{}
\begin{minipage}[b]{\linewidth}\raggedright
Departamento
\end{minipage} & \begin{minipage}[b]{\linewidth}\centering
Frecuencia
\end{minipage} & \begin{minipage}[b]{\linewidth}\centering
Porcentaje
\end{minipage} & \begin{minipage}[b]{\linewidth}\centering
Porcentaje válido
\end{minipage} & \begin{minipage}[b]{\linewidth}\centering
Porcentaje acumulado
\end{minipage} \\
\midrule\noalign{}
\endfirsthead
\toprule\noalign{}
\begin{minipage}[b]{\linewidth}\raggedright
Departamento
\end{minipage} & \begin{minipage}[b]{\linewidth}\centering
Frecuencia
\end{minipage} & \begin{minipage}[b]{\linewidth}\centering
Porcentaje
\end{minipage} & \begin{minipage}[b]{\linewidth}\centering
Porcentaje válido
\end{minipage} & \begin{minipage}[b]{\linewidth}\centering
Porcentaje acumulado
\end{minipage} \\
\midrule\noalign{}
\endhead
\bottomrule\noalign{}
\endlastfoot
AYACUCHO & 102 & 91,9 & 93,6 & 93,6 \\
APURÍMAC & 2 & 1,8 & 1,8 & 95,4 \\
CUSCO & 3 & 2,7 & 2,8 & 98,2 \\
ICA & 1 & 0,9 & 0,9 & 99,1 \\
LIMA & 1 & 0,9 & 0,9 & 100,0 \\
Total & 109 & 98,2 & 100,0 & \\
Perdidos - Sistema & 2 & 1,8 & & \\
Total & 111 & 100,0 & & \\
\end{longtable}

\hypertarget{tbl-13}{}
\begin{longtable}[]{@{}
  >{\raggedright\arraybackslash}p{(\columnwidth - 8\tabcolsep) * \real{0.2400}}
  >{\centering\arraybackslash}p{(\columnwidth - 8\tabcolsep) * \real{0.1333}}
  >{\centering\arraybackslash}p{(\columnwidth - 8\tabcolsep) * \real{0.1333}}
  >{\centering\arraybackslash}p{(\columnwidth - 8\tabcolsep) * \real{0.2267}}
  >{\centering\arraybackslash}p{(\columnwidth - 8\tabcolsep) * \real{0.2667}}@{}}
\caption{\label{tbl-13}Distribución de la provincia del colegio de los
estudiantes de la serie 200 de Economía que cursan Estadística durante
el período 2018-I}\tabularnewline
\toprule\noalign{}
\begin{minipage}[b]{\linewidth}\raggedright
Provincia
\end{minipage} & \begin{minipage}[b]{\linewidth}\centering
Frecuencia
\end{minipage} & \begin{minipage}[b]{\linewidth}\centering
Porcentaje
\end{minipage} & \begin{minipage}[b]{\linewidth}\centering
Porcentaje válido
\end{minipage} & \begin{minipage}[b]{\linewidth}\centering
Porcentaje acumulado
\end{minipage} \\
\midrule\noalign{}
\endfirsthead
\toprule\noalign{}
\begin{minipage}[b]{\linewidth}\raggedright
Provincia
\end{minipage} & \begin{minipage}[b]{\linewidth}\centering
Frecuencia
\end{minipage} & \begin{minipage}[b]{\linewidth}\centering
Porcentaje
\end{minipage} & \begin{minipage}[b]{\linewidth}\centering
Porcentaje válido
\end{minipage} & \begin{minipage}[b]{\linewidth}\centering
Porcentaje acumulado
\end{minipage} \\
\midrule\noalign{}
\endhead
\bottomrule\noalign{}
\endlastfoot
ANDAHUAYLAS & 2 & 1,8 & 1,8 & 1,8 \\
CANGALLO & 4 & 3,6 & 3,6 & 5,5 \\
CHINCHEROS & 2 & 1,8 & 1,8 & 7,3 \\
HUAMANGA & 61 & 55,0 & 55,5 & 62,7 \\
HUANCA SANCOS & 2 & 1,8 & 1,8 & 64,5 \\
HUANTA & 17 & 15,3 & 15,5 & 80,0 \\
ICA & 1 & 0,9 & 0,9 & 80,9 \\
LA CONVENCIÓN & 2 & 1,8 & 1,8 & 82,7 \\
LA MAR & 11 & 9,9 & 10,0 & 92,7 \\
LIMA & 1 & 0,9 & 0,9 & 93,6 \\
VÍCTOR FAJARDO & 5 & 4,5 & 4,5 & 98,2 \\
VILCAS HUAMÁN & 2 & 1,8 & 1,8 & 100,0 \\
Total & 110 & 99,1 & 100,0 & \\
Perdidos - Sistema & 1 & 0,9 & & \\
Total & 111 & 100,0 & & \\
\end{longtable}

\hypertarget{tbl-14}{}
\begin{longtable}[]{@{}
  >{\raggedright\arraybackslash}p{(\columnwidth - 8\tabcolsep) * \real{0.2785}}
  >{\centering\arraybackslash}p{(\columnwidth - 8\tabcolsep) * \real{0.1266}}
  >{\centering\arraybackslash}p{(\columnwidth - 8\tabcolsep) * \real{0.1266}}
  >{\centering\arraybackslash}p{(\columnwidth - 8\tabcolsep) * \real{0.2152}}
  >{\centering\arraybackslash}p{(\columnwidth - 8\tabcolsep) * \real{0.2532}}@{}}
\caption{\label{tbl-14}Distribución del distrito del colegio de los
estudiantes de la serie 200 de Economía que cursan Estadística durante
el período 2018-I}\tabularnewline
\toprule\noalign{}
\begin{minipage}[b]{\linewidth}\raggedright
Distrito
\end{minipage} & \begin{minipage}[b]{\linewidth}\centering
Frecuencia
\end{minipage} & \begin{minipage}[b]{\linewidth}\centering
Porcentaje
\end{minipage} & \begin{minipage}[b]{\linewidth}\centering
Porcentaje válido
\end{minipage} & \begin{minipage}[b]{\linewidth}\centering
Porcentaje acumulado
\end{minipage} \\
\midrule\noalign{}
\endfirsthead
\toprule\noalign{}
\begin{minipage}[b]{\linewidth}\raggedright
Distrito
\end{minipage} & \begin{minipage}[b]{\linewidth}\centering
Frecuencia
\end{minipage} & \begin{minipage}[b]{\linewidth}\centering
Porcentaje
\end{minipage} & \begin{minipage}[b]{\linewidth}\centering
Porcentaje válido
\end{minipage} & \begin{minipage}[b]{\linewidth}\centering
Porcentaje acumulado
\end{minipage} \\
\midrule\noalign{}
\endhead
\bottomrule\noalign{}
\endlastfoot
ANDAHUAYLAS & 1 & 0,9 & 0,9 & 0,9 \\
ANDRÉS AVELINO CÁCERES & 4 & 3,6 & 3,6 & 4,5 \\
AYACUCHO & 35 & 31,5 & 31,5 & 36,0 \\
BELÉN & 1 & 0,9 & 0,9 & 36,9 \\
CARAPO & 1 & 0,9 & 0,9 & 37,8 \\
CARMEN ALTO & 5 & 4,5 & 4,5 & 42,3 \\
CHIARA & 1 & 0,9 & 0,9 & 43,2 \\
CHOSICA & 1 & 0,9 & 0,9 & 44,1 \\
CHUSCHI & 1 & 0,9 & 0,9 & 45,0 \\
CUSCO & 1 & 0,9 & 0,9 & 45,9 \\
HUACCANA & 1 & 0,9 & 0,9 & 46,8 \\
HUALLA & 2 & 1,8 & 1,8 & 48,6 \\
HUAMANGUILLA & 1 & 0,9 & 0,9 & 49,5 \\
HUANCAPI & 1 & 0,9 & 0,9 & 50,5 \\
HUANCARAYLLA & 1 & 0,9 & 0,9 & 51,4 \\
HUANTA & 15 & 13,5 & 13,5 & 64,9 \\
JESÚS NAZARENO & 6 & 5,4 & 5,4 & 70,3 \\
KIMBIRI & 1 & 0,9 & 0,9 & 71,2 \\
LOS MOROCHUCOS & 2 & 1,8 & 1,8 & 73,0 \\
OCOBAMBA & 1 & 0,9 & 0,9 & 73,9 \\
SACSAMARCA & 1 & 0,9 & 0,9 & 74,8 \\
SALAS & 1 & 0,9 & 0,9 & 75,7 \\
SAN FRANCISCO & 1 & 0,9 & 0,9 & 76,6 \\
SAN JUAN BAUTISTA & 8 & 7,2 & 7,2 & 83,8 \\
SANTA ROSA & 4 & 3,6 & 3,6 & 87,4 \\
SARHUA & 1 & 0,9 & 0,9 & 88,3 \\
SUARAMA & 1 & 0,9 & 0,9 & 89,2 \\
SECCLLA & 1 & 0,9 & 0,9 & 90,1 \\
SIVIA & 1 & 0,9 & 0,9 & 91,0 \\
TALAVERA & 1 & 0,9 & 0,9 & 91,9 \\
TAMBO & 5 & 4,5 & 4,5 & 96,4 \\
TOTOS & 1 & 0,9 & 0,9 & 97,3 \\
VILCAS HUAMÁN & 1 & 0,9 & 0,9 & 98,2 \\
VINCHOS & 1 & 0,9 & 0,9 & 99,1 \\
CHUNGUI & 1 & 0,9 & 0,9 & 100,0 \\
Total & 111 & 100,0 & 100,0 & \\
\end{longtable}

\hypertarget{tbl-15}{}
\begin{longtable}[]{@{}
  >{\centering\arraybackslash}p{(\columnwidth - 8\tabcolsep) * \real{0.2400}}
  >{\centering\arraybackslash}p{(\columnwidth - 8\tabcolsep) * \real{0.1333}}
  >{\centering\arraybackslash}p{(\columnwidth - 8\tabcolsep) * \real{0.1333}}
  >{\centering\arraybackslash}p{(\columnwidth - 8\tabcolsep) * \real{0.2267}}
  >{\centering\arraybackslash}p{(\columnwidth - 8\tabcolsep) * \real{0.2667}}@{}}
\caption{\label{tbl-15}Distribución del número de hermanos de los
alumnos de la serie 200 de Economía que cursan Estadística durante el
período 2018-I}\tabularnewline
\toprule\noalign{}
\begin{minipage}[b]{\linewidth}\centering
Número de hermanos
\end{minipage} & \begin{minipage}[b]{\linewidth}\centering
Frecuencia
\end{minipage} & \begin{minipage}[b]{\linewidth}\centering
Porcentaje
\end{minipage} & \begin{minipage}[b]{\linewidth}\centering
Porcentaje válido
\end{minipage} & \begin{minipage}[b]{\linewidth}\centering
Porcentaje acumulado
\end{minipage} \\
\midrule\noalign{}
\endfirsthead
\toprule\noalign{}
\begin{minipage}[b]{\linewidth}\centering
Número de hermanos
\end{minipage} & \begin{minipage}[b]{\linewidth}\centering
Frecuencia
\end{minipage} & \begin{minipage}[b]{\linewidth}\centering
Porcentaje
\end{minipage} & \begin{minipage}[b]{\linewidth}\centering
Porcentaje válido
\end{minipage} & \begin{minipage}[b]{\linewidth}\centering
Porcentaje acumulado
\end{minipage} \\
\midrule\noalign{}
\endhead
\bottomrule\noalign{}
\endlastfoot
0 & 3 & 2,7 & 2,7 & 2,7 \\
1 & 7 & 6,3 & 6,3 & 9,0 \\
2 & 17 & 15,3 & 15,3 & 24,3 \\
3 & 28 & 25,2 & 25,2 & 49,5 \\
4 & 20 & 18,0 & 18,0 & 67,6 \\
5 & 13 & 11,7 & 11,7 & 79,3 \\
6 & 9 & 8,1 & 8,1 & 87,4 \\
7 & 9 & 8,1 & 8,1 & 95,5 \\
8 & 4 & 3,6 & 3,6 & 99,1 \\
9 & 1 & 0,9 & 0,9 & 100,0 \\
Total & 111 & 100,0 & 100,0 & \\
\end{longtable}

\hypertarget{tbl-16}{}
\begin{longtable}[]{@{}
  >{\raggedright\arraybackslash}p{(\columnwidth - 8\tabcolsep) * \real{0.2400}}
  >{\centering\arraybackslash}p{(\columnwidth - 8\tabcolsep) * \real{0.1333}}
  >{\centering\arraybackslash}p{(\columnwidth - 8\tabcolsep) * \real{0.1333}}
  >{\centering\arraybackslash}p{(\columnwidth - 8\tabcolsep) * \real{0.2267}}
  >{\centering\arraybackslash}p{(\columnwidth - 8\tabcolsep) * \real{0.2667}}@{}}
\caption{\label{tbl-16}Distribución de la actividad laboral de los
alumnos de la serie 200 de Economía que cursan Estadística durante el
período 2018-I}\tabularnewline
\toprule\noalign{}
\begin{minipage}[b]{\linewidth}\raggedright
Actividad laboral
\end{minipage} & \begin{minipage}[b]{\linewidth}\centering
Frecuencia
\end{minipage} & \begin{minipage}[b]{\linewidth}\centering
Porcentaje
\end{minipage} & \begin{minipage}[b]{\linewidth}\centering
Porcentaje válido
\end{minipage} & \begin{minipage}[b]{\linewidth}\centering
Porcentaje acumulado
\end{minipage} \\
\midrule\noalign{}
\endfirsthead
\toprule\noalign{}
\begin{minipage}[b]{\linewidth}\raggedright
Actividad laboral
\end{minipage} & \begin{minipage}[b]{\linewidth}\centering
Frecuencia
\end{minipage} & \begin{minipage}[b]{\linewidth}\centering
Porcentaje
\end{minipage} & \begin{minipage}[b]{\linewidth}\centering
Porcentaje válido
\end{minipage} & \begin{minipage}[b]{\linewidth}\centering
Porcentaje acumulado
\end{minipage} \\
\midrule\noalign{}
\endhead
\bottomrule\noalign{}
\endlastfoot
Sí & 42 & 37,8 & 38,5 & 38,5 \\
No & 67 & 60,4 & 61,5 & 100,0 \\
Total & 109 & 98,2 & 100,0 & \\
Perdidos - Sistema & 2 & 1,8 & & \\
Total & 111 & 100,0 & & \\
\end{longtable}

\hypertarget{tbl-17}{}
\begin{longtable}[]{@{}
  >{\raggedright\arraybackslash}p{(\columnwidth - 8\tabcolsep) * \real{0.2692}}
  >{\centering\arraybackslash}p{(\columnwidth - 8\tabcolsep) * \real{0.1282}}
  >{\centering\arraybackslash}p{(\columnwidth - 8\tabcolsep) * \real{0.1282}}
  >{\centering\arraybackslash}p{(\columnwidth - 8\tabcolsep) * \real{0.2179}}
  >{\centering\arraybackslash}p{(\columnwidth - 8\tabcolsep) * \real{0.2564}}@{}}
\caption{\label{tbl-17}Distribución del uso del comedor de la UNSCH
entre los alumnos de la serie 200 de Economía que cursan Estadística
durante el período 2018-I}\tabularnewline
\toprule\noalign{}
\begin{minipage}[b]{\linewidth}\raggedright
Uso del comedor UNSCH
\end{minipage} & \begin{minipage}[b]{\linewidth}\centering
Frecuencia
\end{minipage} & \begin{minipage}[b]{\linewidth}\centering
Porcentaje
\end{minipage} & \begin{minipage}[b]{\linewidth}\centering
Porcentaje válido
\end{minipage} & \begin{minipage}[b]{\linewidth}\centering
Porcentaje acumulado
\end{minipage} \\
\midrule\noalign{}
\endfirsthead
\toprule\noalign{}
\begin{minipage}[b]{\linewidth}\raggedright
Uso del comedor UNSCH
\end{minipage} & \begin{minipage}[b]{\linewidth}\centering
Frecuencia
\end{minipage} & \begin{minipage}[b]{\linewidth}\centering
Porcentaje
\end{minipage} & \begin{minipage}[b]{\linewidth}\centering
Porcentaje válido
\end{minipage} & \begin{minipage}[b]{\linewidth}\centering
Porcentaje acumulado
\end{minipage} \\
\midrule\noalign{}
\endhead
\bottomrule\noalign{}
\endlastfoot
Sí & 60 & 54,1 & 54,5 & 54,5 \\
No & 50 & 45,0 & 45,5 & 100,0 \\
Total & 110 & 99,1 & 100,0 & \\
Perdidos - Sistema & 1 & 0,9 & & \\
Total & 111 & 100,0 & & \\
\end{longtable}

\hypertarget{tbl-18}{}
\begin{longtable}[]{@{}
  >{\raggedright\arraybackslash}p{(\columnwidth - 8\tabcolsep) * \real{0.2400}}
  >{\centering\arraybackslash}p{(\columnwidth - 8\tabcolsep) * \real{0.1333}}
  >{\centering\arraybackslash}p{(\columnwidth - 8\tabcolsep) * \real{0.1333}}
  >{\centering\arraybackslash}p{(\columnwidth - 8\tabcolsep) * \real{0.2267}}
  >{\centering\arraybackslash}p{(\columnwidth - 8\tabcolsep) * \real{0.2667}}@{}}
\caption{\label{tbl-18}Distribución de los alumnos de Estadística de la
serie 200 de Economía durante el período 2018-I según si viven solos o
no.}\tabularnewline
\toprule\noalign{}
\begin{minipage}[b]{\linewidth}\raggedright
¿Viven solo?
\end{minipage} & \begin{minipage}[b]{\linewidth}\centering
Frecuencia
\end{minipage} & \begin{minipage}[b]{\linewidth}\centering
Porcentaje
\end{minipage} & \begin{minipage}[b]{\linewidth}\centering
Porcentaje válido
\end{minipage} & \begin{minipage}[b]{\linewidth}\centering
Porcentaje acumulado
\end{minipage} \\
\midrule\noalign{}
\endfirsthead
\toprule\noalign{}
\begin{minipage}[b]{\linewidth}\raggedright
¿Viven solo?
\end{minipage} & \begin{minipage}[b]{\linewidth}\centering
Frecuencia
\end{minipage} & \begin{minipage}[b]{\linewidth}\centering
Porcentaje
\end{minipage} & \begin{minipage}[b]{\linewidth}\centering
Porcentaje válido
\end{minipage} & \begin{minipage}[b]{\linewidth}\centering
Porcentaje acumulado
\end{minipage} \\
\midrule\noalign{}
\endhead
\bottomrule\noalign{}
\endlastfoot
Sí & 35 & 31,5 & 32,1 & 32,1 \\
No & 74 & 66,7 & 67,9 & 100,0 \\
Total & 109 & 98,2 & 100,0 & \\
Perdidos - Sistema & 2 & 1,8 & & \\
Total & 111 & 100,0 & & \\
\end{longtable}

\hypertarget{tbl-19}{}
\begin{longtable}[]{@{}
  >{\raggedright\arraybackslash}p{(\columnwidth - 8\tabcolsep) * \real{0.1739}}
  >{\centering\arraybackslash}p{(\columnwidth - 8\tabcolsep) * \real{0.1449}}
  >{\centering\arraybackslash}p{(\columnwidth - 8\tabcolsep) * \real{0.1449}}
  >{\centering\arraybackslash}p{(\columnwidth - 8\tabcolsep) * \real{0.2464}}
  >{\centering\arraybackslash}p{(\columnwidth - 8\tabcolsep) * \real{0.2899}}@{}}
\caption{\label{tbl-19}Distribución del estado civil de los alumnos de
Estadística de la serie 200 de Economía durante el período
2018-I.}\tabularnewline
\toprule\noalign{}
\begin{minipage}[b]{\linewidth}\raggedright
Estado Civil
\end{minipage} & \begin{minipage}[b]{\linewidth}\centering
Frecuencia
\end{minipage} & \begin{minipage}[b]{\linewidth}\centering
Porcentaje
\end{minipage} & \begin{minipage}[b]{\linewidth}\centering
Porcentaje válido
\end{minipage} & \begin{minipage}[b]{\linewidth}\centering
Porcentaje acumulado
\end{minipage} \\
\midrule\noalign{}
\endfirsthead
\toprule\noalign{}
\begin{minipage}[b]{\linewidth}\raggedright
Estado Civil
\end{minipage} & \begin{minipage}[b]{\linewidth}\centering
Frecuencia
\end{minipage} & \begin{minipage}[b]{\linewidth}\centering
Porcentaje
\end{minipage} & \begin{minipage}[b]{\linewidth}\centering
Porcentaje válido
\end{minipage} & \begin{minipage}[b]{\linewidth}\centering
Porcentaje acumulado
\end{minipage} \\
\midrule\noalign{}
\endhead
\bottomrule\noalign{}
\endlastfoot
Casado & 1 & 0,9 & 0,9 & 0,9 \\
Comprometido & 1 & 0,9 & 0,9 & 1,8 \\
Soltero & 70 & 63,1 & 63,1 & 64,9 \\
Soltera & 39 & 35,1 & 35,1 & 100,0 \\
Total & 111 & 100,0 & 100,0 & \\
\end{longtable}

\hypertarget{tbl-20}{}
\begin{longtable}[]{@{}
  >{\raggedright\arraybackslash}p{(\columnwidth - 8\tabcolsep) * \real{0.2692}}
  >{\centering\arraybackslash}p{(\columnwidth - 8\tabcolsep) * \real{0.1282}}
  >{\centering\arraybackslash}p{(\columnwidth - 8\tabcolsep) * \real{0.1282}}
  >{\centering\arraybackslash}p{(\columnwidth - 8\tabcolsep) * \real{0.2179}}
  >{\centering\arraybackslash}p{(\columnwidth - 8\tabcolsep) * \real{0.2564}}@{}}
\caption{\label{tbl-20}Distribución de la dependencia económica de los
alumnos de Estadística de la serie 200 de Economía durante el período
2018-I.}\tabularnewline
\toprule\noalign{}
\begin{minipage}[b]{\linewidth}\raggedright
Dependencia Económica
\end{minipage} & \begin{minipage}[b]{\linewidth}\centering
Frecuencia
\end{minipage} & \begin{minipage}[b]{\linewidth}\centering
Porcentaje
\end{minipage} & \begin{minipage}[b]{\linewidth}\centering
Porcentaje válido
\end{minipage} & \begin{minipage}[b]{\linewidth}\centering
Porcentaje acumulado
\end{minipage} \\
\midrule\noalign{}
\endfirsthead
\toprule\noalign{}
\begin{minipage}[b]{\linewidth}\raggedright
Dependencia Económica
\end{minipage} & \begin{minipage}[b]{\linewidth}\centering
Frecuencia
\end{minipage} & \begin{minipage}[b]{\linewidth}\centering
Porcentaje
\end{minipage} & \begin{minipage}[b]{\linewidth}\centering
Porcentaje válido
\end{minipage} & \begin{minipage}[b]{\linewidth}\centering
Porcentaje acumulado
\end{minipage} \\
\midrule\noalign{}
\endhead
\bottomrule\noalign{}
\endlastfoot
Sí & 81 & 73,0 & 74,3 & 74,3 \\
No & 28 & 25,2 & 25,7 & 100,0 \\
Total & 109 & 98,2 & 100,0 & \\
Perdidos - Sistema & 2 & 1,8 & & \\
Total & 111 & 100,0 & & \\
\end{longtable}

\hypertarget{tbl-21}{}
\begin{longtable}[]{@{}
  >{\raggedright\arraybackslash}p{(\columnwidth - 8\tabcolsep) * \real{0.2297}}
  >{\centering\arraybackslash}p{(\columnwidth - 8\tabcolsep) * \real{0.1351}}
  >{\centering\arraybackslash}p{(\columnwidth - 8\tabcolsep) * \real{0.1351}}
  >{\centering\arraybackslash}p{(\columnwidth - 8\tabcolsep) * \real{0.2297}}
  >{\centering\arraybackslash}p{(\columnwidth - 8\tabcolsep) * \real{0.2703}}@{}}
\caption{\label{tbl-21}Distribución de la primera profesión de los
alumnos de Estadística de la serie 200 de Economía durante el período
2018-I.}\tabularnewline
\toprule\noalign{}
\begin{minipage}[b]{\linewidth}\raggedright
Primera Profesión
\end{minipage} & \begin{minipage}[b]{\linewidth}\centering
Frecuencia
\end{minipage} & \begin{minipage}[b]{\linewidth}\centering
Porcentaje
\end{minipage} & \begin{minipage}[b]{\linewidth}\centering
Porcentaje válido
\end{minipage} & \begin{minipage}[b]{\linewidth}\centering
Porcentaje acumulado
\end{minipage} \\
\midrule\noalign{}
\endfirsthead
\toprule\noalign{}
\begin{minipage}[b]{\linewidth}\raggedright
Primera Profesión
\end{minipage} & \begin{minipage}[b]{\linewidth}\centering
Frecuencia
\end{minipage} & \begin{minipage}[b]{\linewidth}\centering
Porcentaje
\end{minipage} & \begin{minipage}[b]{\linewidth}\centering
Porcentaje válido
\end{minipage} & \begin{minipage}[b]{\linewidth}\centering
Porcentaje acumulado
\end{minipage} \\
\midrule\noalign{}
\endhead
\bottomrule\noalign{}
\endlastfoot
Sí & 109 & 98,2 & 98,2 & 98,2 \\
No & 2 & 1,8 & 1,8 & 100,0 \\
Total & 111 & 100,0 & 100,0 & \\
\end{longtable}

\hypertarget{tbl-22}{}
\begin{longtable}[]{@{}
  >{\raggedright\arraybackslash}p{(\columnwidth - 8\tabcolsep) * \real{0.2192}}
  >{\centering\arraybackslash}p{(\columnwidth - 8\tabcolsep) * \real{0.1370}}
  >{\centering\arraybackslash}p{(\columnwidth - 8\tabcolsep) * \real{0.1370}}
  >{\centering\arraybackslash}p{(\columnwidth - 8\tabcolsep) * \real{0.2329}}
  >{\centering\arraybackslash}p{(\columnwidth - 8\tabcolsep) * \real{0.2740}}@{}}
\caption{\label{tbl-22}Distribución de la movilidad propia de los
alumnos de Estadística de la serie 200 de Economía durante el período
2018-I.}\tabularnewline
\toprule\noalign{}
\begin{minipage}[b]{\linewidth}\raggedright
Movilidad Propia
\end{minipage} & \begin{minipage}[b]{\linewidth}\centering
Frecuencia
\end{minipage} & \begin{minipage}[b]{\linewidth}\centering
Porcentaje
\end{minipage} & \begin{minipage}[b]{\linewidth}\centering
Porcentaje válido
\end{minipage} & \begin{minipage}[b]{\linewidth}\centering
Porcentaje acumulado
\end{minipage} \\
\midrule\noalign{}
\endfirsthead
\toprule\noalign{}
\begin{minipage}[b]{\linewidth}\raggedright
Movilidad Propia
\end{minipage} & \begin{minipage}[b]{\linewidth}\centering
Frecuencia
\end{minipage} & \begin{minipage}[b]{\linewidth}\centering
Porcentaje
\end{minipage} & \begin{minipage}[b]{\linewidth}\centering
Porcentaje válido
\end{minipage} & \begin{minipage}[b]{\linewidth}\centering
Porcentaje acumulado
\end{minipage} \\
\midrule\noalign{}
\endhead
\bottomrule\noalign{}
\endlastfoot
Sí & 3 & 2,7 & 2,7 & 2,7 \\
No & 108 & 97,3 & 97,3 & 100,0 \\
Total & 111 & 100,0 & 100,0 & \\
\end{longtable}

\hypertarget{tbl-23}{}
\begin{longtable}[]{@{}
  >{\raggedright\arraybackslash}p{(\columnwidth - 10\tabcolsep) * \real{0.0909}}
  >{\centering\arraybackslash}p{(\columnwidth - 10\tabcolsep) * \real{0.1136}}
  >{\centering\arraybackslash}p{(\columnwidth - 10\tabcolsep) * \real{0.1818}}
  >{\centering\arraybackslash}p{(\columnwidth - 10\tabcolsep) * \real{0.1250}}
  >{\centering\arraybackslash}p{(\columnwidth - 10\tabcolsep) * \real{0.2841}}
  >{\centering\arraybackslash}p{(\columnwidth - 10\tabcolsep) * \real{0.2045}}@{}}
\caption{\label{tbl-23}Características de los alumnos de Estadística de
la serie 200 de Economía durante el período 2018-I.}\tabularnewline
\toprule\noalign{}
\begin{minipage}[b]{\linewidth}\raggedright
\end{minipage} & \begin{minipage}[b]{\linewidth}\centering
¿Trabajas?
\end{minipage} & \begin{minipage}[b]{\linewidth}\centering
¿Usa el comedor?
\end{minipage} & \begin{minipage}[b]{\linewidth}\centering
¿Vive solo?
\end{minipage} & \begin{minipage}[b]{\linewidth}\centering
¿Es tu primera profesión?
\end{minipage} & \begin{minipage}[b]{\linewidth}\centering
¿Tienes movilidad?
\end{minipage} \\
\midrule\noalign{}
\endfirsthead
\toprule\noalign{}
\begin{minipage}[b]{\linewidth}\raggedright
\end{minipage} & \begin{minipage}[b]{\linewidth}\centering
¿Trabajas?
\end{minipage} & \begin{minipage}[b]{\linewidth}\centering
¿Usa el comedor?
\end{minipage} & \begin{minipage}[b]{\linewidth}\centering
¿Vive solo?
\end{minipage} & \begin{minipage}[b]{\linewidth}\centering
¿Es tu primera profesión?
\end{minipage} & \begin{minipage}[b]{\linewidth}\centering
¿Tienes movilidad?
\end{minipage} \\
\midrule\noalign{}
\endhead
\bottomrule\noalign{}
\endlastfoot
Perdidos & 2 & 1 & 2 & 2 & 0 \\
Válido & 109 & 110 & 109 & 109 & 111 \\
\end{longtable}

\hypertarget{tbl-24}{}
\begin{longtable}[]{@{}
  >{\raggedright\arraybackslash}p{(\columnwidth - 8\tabcolsep) * \real{0.1493}}
  >{\centering\arraybackslash}p{(\columnwidth - 8\tabcolsep) * \real{0.1493}}
  >{\centering\arraybackslash}p{(\columnwidth - 8\tabcolsep) * \real{0.1493}}
  >{\centering\arraybackslash}p{(\columnwidth - 8\tabcolsep) * \real{0.2537}}
  >{\centering\arraybackslash}p{(\columnwidth - 8\tabcolsep) * \real{0.2985}}@{}}
\caption{\label{tbl-24}Nivel de instrucción de los padres de los alumnos
de Estadística de la serie 200 de Economía durante el período
2018-I.}\tabularnewline
\toprule\noalign{}
\begin{minipage}[b]{\linewidth}\raggedright
\end{minipage} & \begin{minipage}[b]{\linewidth}\centering
Frecuencia
\end{minipage} & \begin{minipage}[b]{\linewidth}\centering
Porcentaje
\end{minipage} & \begin{minipage}[b]{\linewidth}\centering
Porcentaje válido
\end{minipage} & \begin{minipage}[b]{\linewidth}\centering
Porcentaje acumulado
\end{minipage} \\
\midrule\noalign{}
\endfirsthead
\toprule\noalign{}
\begin{minipage}[b]{\linewidth}\raggedright
\end{minipage} & \begin{minipage}[b]{\linewidth}\centering
Frecuencia
\end{minipage} & \begin{minipage}[b]{\linewidth}\centering
Porcentaje
\end{minipage} & \begin{minipage}[b]{\linewidth}\centering
Porcentaje válido
\end{minipage} & \begin{minipage}[b]{\linewidth}\centering
Porcentaje acumulado
\end{minipage} \\
\midrule\noalign{}
\endhead
\bottomrule\noalign{}
\endlastfoot
PRIMARIA & 31 & 27.9 & 27.9 & 27.9 \\
SECUNDARIA & 48 & 43.2 & 43.2 & 71.2 \\
SUPERIOR & 32 & 28.8 & 28.8 & 100.0 \\
Total & 111 & 100.0 & 100.0 & \\
\end{longtable}

\hypertarget{tbl-25}{}
\begin{longtable}[]{@{}
  >{\raggedright\arraybackslash}p{(\columnwidth - 8\tabcolsep) * \real{0.2400}}
  >{\centering\arraybackslash}p{(\columnwidth - 8\tabcolsep) * \real{0.1333}}
  >{\centering\arraybackslash}p{(\columnwidth - 8\tabcolsep) * \real{0.1333}}
  >{\centering\arraybackslash}p{(\columnwidth - 8\tabcolsep) * \real{0.2267}}
  >{\centering\arraybackslash}p{(\columnwidth - 8\tabcolsep) * \real{0.2667}}@{}}
\caption{\label{tbl-25}Nivel de instrucción de las madres de los alumnos
de Estadística de la serie 200 de Economía durante el período
2018-I.}\tabularnewline
\toprule\noalign{}
\begin{minipage}[b]{\linewidth}\raggedright
\end{minipage} & \begin{minipage}[b]{\linewidth}\centering
Frecuencia
\end{minipage} & \begin{minipage}[b]{\linewidth}\centering
Porcentaje
\end{minipage} & \begin{minipage}[b]{\linewidth}\centering
Porcentaje válido
\end{minipage} & \begin{minipage}[b]{\linewidth}\centering
Porcentaje acumulado
\end{minipage} \\
\midrule\noalign{}
\endfirsthead
\toprule\noalign{}
\begin{minipage}[b]{\linewidth}\raggedright
\end{minipage} & \begin{minipage}[b]{\linewidth}\centering
Frecuencia
\end{minipage} & \begin{minipage}[b]{\linewidth}\centering
Porcentaje
\end{minipage} & \begin{minipage}[b]{\linewidth}\centering
Porcentaje válido
\end{minipage} & \begin{minipage}[b]{\linewidth}\centering
Porcentaje acumulado
\end{minipage} \\
\midrule\noalign{}
\endhead
\bottomrule\noalign{}
\endlastfoot
SIN INSTRUCCIÓN & 4 & 3.6 & 3.6 & 3.6 \\
PRIMARIA & 58 & 52.3 & 52.7 & 56.4 \\
SECUNDARIA & 38 & 34.2 & 34.5 & 90.9 \\
SUPERIOR & 10 & 9.0 & 9.1 & 100.0 \\
Total & 110 & 99.1 & 100.0 & \\
Perdidos - Sistema & 1 & 0.9 & & \\
Total & 111 & 100.0 & & \\
\end{longtable}

\hypertarget{tbl-26}{}
\begin{longtable}[]{@{}
  >{\centering\arraybackslash}p{(\columnwidth - 8\tabcolsep) * \real{0.2400}}
  >{\centering\arraybackslash}p{(\columnwidth - 8\tabcolsep) * \real{0.1333}}
  >{\centering\arraybackslash}p{(\columnwidth - 8\tabcolsep) * \real{0.1333}}
  >{\centering\arraybackslash}p{(\columnwidth - 8\tabcolsep) * \real{0.2267}}
  >{\centering\arraybackslash}p{(\columnwidth - 8\tabcolsep) * \real{0.2667}}@{}}
\caption{\label{tbl-26}Número de hijos de los alumnos de Estadística de
la serie 200 de Economía durante el período 2018-I.}\tabularnewline
\toprule\noalign{}
\begin{minipage}[b]{\linewidth}\centering
\end{minipage} & \begin{minipage}[b]{\linewidth}\centering
Frecuencia
\end{minipage} & \begin{minipage}[b]{\linewidth}\centering
Porcentaje
\end{minipage} & \begin{minipage}[b]{\linewidth}\centering
Porcentaje válido
\end{minipage} & \begin{minipage}[b]{\linewidth}\centering
Porcentaje acumulado
\end{minipage} \\
\midrule\noalign{}
\endfirsthead
\toprule\noalign{}
\begin{minipage}[b]{\linewidth}\centering
\end{minipage} & \begin{minipage}[b]{\linewidth}\centering
Frecuencia
\end{minipage} & \begin{minipage}[b]{\linewidth}\centering
Porcentaje
\end{minipage} & \begin{minipage}[b]{\linewidth}\centering
Porcentaje válido
\end{minipage} & \begin{minipage}[b]{\linewidth}\centering
Porcentaje acumulado
\end{minipage} \\
\midrule\noalign{}
\endhead
\bottomrule\noalign{}
\endlastfoot
0 & 80 & 72.1 & 74.8 & 74.8 \\
1 & 7 & 6.3 & 6.5 & 81.3 \\
2 & 2 & 1.8 & 1.9 & 83.2 \\
3 & 5 & 4.5 & 4.7 & 87.9 \\
4 & 2 & 1.8 & 1.9 & 89.7 \\
5 & 4 & 3.6 & 3.7 & 93.5 \\
6 & 3 & 2.7 & 2.8 & 96.3 \\
7 & 3 & 2.7 & 2.8 & 99.1 \\
9 & 1 & 0.9 & 0.9 & 100.0 \\
Total & 107 & 96.4 & 100.0 & \\
Perdidos - Sistema & 4 & 3.6 & & \\
Total & 111 & 100.0 & & \\
\end{longtable}


\printbibliography


\end{document}
