% Options for packages loaded elsewhere
\PassOptionsToPackage{unicode}{hyperref}
\PassOptionsToPackage{hyphens}{url}
\PassOptionsToPackage{dvipsnames,svgnames,x11names}{xcolor}
%
\documentclass[
  a4paper,
]{article}

\usepackage{amsmath,amssymb}
\usepackage{iftex}
\ifPDFTeX
  \usepackage[T1]{fontenc}
  \usepackage[utf8]{inputenc}
  \usepackage{textcomp} % provide euro and other symbols
\else % if luatex or xetex
  \usepackage{unicode-math}
  \defaultfontfeatures{Scale=MatchLowercase}
  \defaultfontfeatures[\rmfamily]{Ligatures=TeX,Scale=1}
\fi
\usepackage{lmodern}
\ifPDFTeX\else  
    % xetex/luatex font selection
\fi
% Use upquote if available, for straight quotes in verbatim environments
\IfFileExists{upquote.sty}{\usepackage{upquote}}{}
\IfFileExists{microtype.sty}{% use microtype if available
  \usepackage[]{microtype}
  \UseMicrotypeSet[protrusion]{basicmath} % disable protrusion for tt fonts
}{}
\makeatletter
\@ifundefined{KOMAClassName}{% if non-KOMA class
  \IfFileExists{parskip.sty}{%
    \usepackage{parskip}
  }{% else
    \setlength{\parindent}{0pt}
    \setlength{\parskip}{6pt plus 2pt minus 1pt}}
}{% if KOMA class
  \KOMAoptions{parskip=half}}
\makeatother
\usepackage{xcolor}
\usepackage[top=2.54cm,right=2.54cm,bottom=2.54cm,left=2.54cm]{geometry}
\setlength{\emergencystretch}{3em} % prevent overfull lines
\setcounter{secnumdepth}{-\maxdimen} % remove section numbering
% Make \paragraph and \subparagraph free-standing
\ifx\paragraph\undefined\else
  \let\oldparagraph\paragraph
  \renewcommand{\paragraph}[1]{\oldparagraph{#1}\mbox{}}
\fi
\ifx\subparagraph\undefined\else
  \let\oldsubparagraph\subparagraph
  \renewcommand{\subparagraph}[1]{\oldsubparagraph{#1}\mbox{}}
\fi


\providecommand{\tightlist}{%
  \setlength{\itemsep}{0pt}\setlength{\parskip}{0pt}}\usepackage{longtable,booktabs,array}
\usepackage{calc} % for calculating minipage widths
% Correct order of tables after \paragraph or \subparagraph
\usepackage{etoolbox}
\makeatletter
\patchcmd\longtable{\par}{\if@noskipsec\mbox{}\fi\par}{}{}
\makeatother
% Allow footnotes in longtable head/foot
\IfFileExists{footnotehyper.sty}{\usepackage{footnotehyper}}{\usepackage{footnote}}
\makesavenoteenv{longtable}
\usepackage{graphicx}
\makeatletter
\def\maxwidth{\ifdim\Gin@nat@width>\linewidth\linewidth\else\Gin@nat@width\fi}
\def\maxheight{\ifdim\Gin@nat@height>\textheight\textheight\else\Gin@nat@height\fi}
\makeatother
% Scale images if necessary, so that they will not overflow the page
% margins by default, and it is still possible to overwrite the defaults
% using explicit options in \includegraphics[width, height, ...]{}
\setkeys{Gin}{width=\maxwidth,height=\maxheight,keepaspectratio}
% Set default figure placement to htbp
\makeatletter
\def\fps@figure{htbp}
\makeatother

\makeatletter
\makeatother
\makeatletter
\makeatother
\makeatletter
\@ifpackageloaded{caption}{}{\usepackage{caption}}
\AtBeginDocument{%
\ifdefined\contentsname
  \renewcommand*\contentsname{Tabla de contenidos}
\else
  \newcommand\contentsname{Tabla de contenidos}
\fi
\ifdefined\listfigurename
  \renewcommand*\listfigurename{Listado de Figuras}
\else
  \newcommand\listfigurename{Listado de Figuras}
\fi
\ifdefined\listtablename
  \renewcommand*\listtablename{Listado de Tablas}
\else
  \newcommand\listtablename{Listado de Tablas}
\fi
\ifdefined\figurename
  \renewcommand*\figurename{Figura}
\else
  \newcommand\figurename{Figura}
\fi
\ifdefined\tablename
  \renewcommand*\tablename{Tabla}
\else
  \newcommand\tablename{Tabla}
\fi
}
\@ifpackageloaded{float}{}{\usepackage{float}}
\floatstyle{ruled}
\@ifundefined{c@chapter}{\newfloat{codelisting}{h}{lop}}{\newfloat{codelisting}{h}{lop}[chapter]}
\floatname{codelisting}{Listado}
\newcommand*\listoflistings{\listof{codelisting}{Listado de Listados}}
\makeatother
\makeatletter
\@ifpackageloaded{caption}{}{\usepackage{caption}}
\@ifpackageloaded{subcaption}{}{\usepackage{subcaption}}
\makeatother
\makeatletter
\@ifpackageloaded{tcolorbox}{}{\usepackage[skins,breakable]{tcolorbox}}
\makeatother
\makeatletter
\@ifundefined{shadecolor}{\definecolor{shadecolor}{rgb}{.97, .97, .97}}
\makeatother
\makeatletter
\makeatother
\makeatletter
\makeatother
\ifLuaTeX
\usepackage[bidi=basic]{babel}
\else
\usepackage[bidi=default]{babel}
\fi
\babelprovide[main,import]{spanish}
% get rid of language-specific shorthands (see #6817):
\let\LanguageShortHands\languageshorthands
\def\languageshorthands#1{}
\ifLuaTeX
  \usepackage{selnolig}  % disable illegal ligatures
\fi
\usepackage[]{biblatex}
\addbibresource{../../../../references.bib}
\IfFileExists{bookmark.sty}{\usepackage{bookmark}}{\usepackage{hyperref}}
\IfFileExists{xurl.sty}{\usepackage{xurl}}{} % add URL line breaks if available
\urlstyle{same} % disable monospaced font for URLs
\hypersetup{
  pdftitle={Factores que influyen en el rendimiento académico de la serie 100 y 200},
  pdfauthor={Edison Achalma},
  pdflang={es},
  colorlinks=true,
  linkcolor={blue},
  filecolor={Maroon},
  citecolor={Blue},
  urlcolor={Blue},
  pdfcreator={LaTeX via pandoc}}

\title{Factores que influyen en el rendimiento académico de la serie 100
y 200\thanks{Dedicamos este trabajo de investigación a nuestros padres,
quienes han sido nuestros incondicionales compañeros en esta travesía
universitaria.}}
\usepackage{etoolbox}
\makeatletter
\providecommand{\subtitle}[1]{% add subtitle to \maketitle
  \apptocmd{\@title}{\par {\large #1 \par}}{}{}
}
\makeatother
\subtitle{Procesamiento de datos}
\author{Edison Achalma}
\date{2018-12-26}

\begin{document}
\maketitle
\ifdefined\Shaded\renewenvironment{Shaded}{\begin{tcolorbox}[enhanced, interior hidden, sharp corners, boxrule=0pt, frame hidden, borderline west={3pt}{0pt}{shadecolor}, breakable]}{\end{tcolorbox}}\fi

\hypertarget{introducciuxf3n}{%
\section{Introducción}\label{introducciuxf3n}}

En el presente estudio, se busca evaluar las variables internas y
externas que impactan en el rendimiento académico de los alumnos de las
series 100 y 200 de la escuela de Economía. Con el fin de llevar a cabo
esta evaluación, se aplicó una encuesta escrita a 174 estudiantes
universitarios pertenecientes a dichas series.

Dentro de las variables evaluadas se encuentran tanto aspectos
cuantitativos como cualitativos, tales como el índice académico, sexo,
edad, horas de estudio, frecuencia de visita a la biblioteca, entre
otros. Estas variables nos proporcionaron datos suficientes para
analizar el desempeño y rendimiento académico de los estudiantes,
permitiéndonos así llegar a conclusiones certeras respecto a las
hipótesis planteadas.

Los resultados de esta investigación son de gran importancia para
comprender los factores que influyen en el rendimiento académico de los
estudiantes de las series 100 y 200, y contribuirán a la generación de
estrategias y recomendaciones para mejorar dicho rendimiento.

Estudiantes

\hypertarget{el-rendimiento-acaduxe9mico-de-las-series-100-y-200-2018-ii}{%
\section{El rendimiento académico de las series 100 y 200
(2018-II)}\label{el-rendimiento-acaduxe9mico-de-las-series-100-y-200-2018-ii}}

\hypertarget{observaciuxf3n}{%
\subsection{Observación}\label{observaciuxf3n}}

\hypertarget{objetivos}{%
\subsubsection{Objetivos:}\label{objetivos}}

El objetivo de este trabajo es evaluar el desempeño de los alumnos en
relación a diversas variables, tales como sexo, serie, índice académico,
número de cursos matriculados, horas de estudio, frecuencia de visita a
la biblioteca y uso de redes sociales.

\hypertarget{planteamiento-del-problema}{%
\subsection{Planteamiento del
problema}\label{planteamiento-del-problema}}

Este trabajo se propone responder a las siguientes interrogantes:

\begin{itemize}
\tightlist
\item
  ¿Cómo afecta el número de cursos matriculados al rendimiento
  académico?
\item
  ¿Cómo influyen las horas de estudio en el índice académico?
\item
  ¿Cuál es la proporción de rendimiento académico entre hombres y
  mujeres?
\item
  ¿Cómo se ve afectado el rendimiento académico por la frecuencia de uso
  de la biblioteca?
\item
  ¿Qué impacto tiene la frecuencia de uso de las redes sociales en el
  rendimiento académico?
\end{itemize}

\hypertarget{hipuxf3tesis}{%
\subsection{Hipótesis}\label{hipuxf3tesis}}

Con el fin de analizar, comparar y evaluar las variables que influyen en
el rendimiento académico, planteamos las siguientes posibles respuestas:

\begin{itemize}
\tightlist
\item
  Se espera que la serie 100 presente una mayor proporción de alumnos
  aprobados.
\item
  A medida que se disminuye el número de cursos matriculados, se espera
  que el índice académico aprobatorio supere el 50\%.
\item
  Se espera que a mayor cantidad de horas de estudio, exista una mayor
  probabilidad de obtener un mejor rendimiento académico.
\item
  Se presume que un menor uso de la biblioteca se asociará con una mayor
  probabilidad de desaprobación.
\item
  Se espera que las mujeres representen una proporción más alta en el
  rendimiento académico.
\end{itemize}

\hypertarget{trabajo-de-campo-y-resultados}{%
\subsection{Trabajo de campo y
resultados}\label{trabajo-de-campo-y-resultados}}

A continuación se presentan los datos recopilados durante el trabajo de
campo y los resultados obtenidos:

\begin{itemize}
\tightlist
\item
  Número total de alumnos: \(N = 174\)
\item
  Media poblacional: \(\mu = 11.5458\)
\item
  Desviación estándar poblacional: \(\sigma = 4.10\)
\end{itemize}

Se realizó una muestra piloto con los primeros 20 estudiantes, de la
cual se obtuvieron los siguientes datos:

\begin{itemize}
\tightlist
\item
  Tamaño de la muestra: \(n = 20\)
\item
  Desviación estándar de la muestra: \(s = 1.842081989\)
\item
  Media de la muestra: \(x = 10.8185\)
\end{itemize}

Utilizando los datos y considerando un nivel de confianza del 95\% y un
margen de error de 0.6, se determinó el tamaño necesario para una
muestra representativa mediante la fórmula:

\[
n = \frac{{N \cdot Z^2 \cdot \sigma^2}}{{(N-1) \cdot e^2 + Z^2 \cdot \sigma^2}}
\]

Sustituyendo los valores en la fórmula, se obtuvo:

\[
n = \frac{{174 \cdot 1.96^2 \cdot 1.842081989^2}}{{173 \cdot 0.6^2 + 1.96^2 \cdot 1.842081989^2}}
\]

El resultado obtenido fue \(n \approx 30.1158088\), lo cual indica que
se requieren 30 datos aleatorios para tener una muestra representativa.

A partir de la muestra de 30 datos, se realizaron las siguientes
estimaciones:

\begin{itemize}
\tightlist
\item
  Estimación de la media poblacional: \(\hat{u} = 10.720333\)
\item
  Estimación de la desviación estándar poblacional:
  \(\hat{\sigma} = 2.325168861\)
\end{itemize}

Estas estimaciones proporcionan información sobre los valores promedio y
la variabilidad de la población a partir de la muestra seleccionada.


\printbibliography


\end{document}
