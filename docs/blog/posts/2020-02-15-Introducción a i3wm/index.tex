% Options for packages loaded elsewhere
\PassOptionsToPackage{unicode}{hyperref}
\PassOptionsToPackage{hyphens}{url}
\PassOptionsToPackage{dvipsnames,svgnames,x11names}{xcolor}
%
\documentclass[
  a4paper,
]{article}

\usepackage{amsmath,amssymb}
\usepackage{iftex}
\ifPDFTeX
  \usepackage[T1]{fontenc}
  \usepackage[utf8]{inputenc}
  \usepackage{textcomp} % provide euro and other symbols
\else % if luatex or xetex
  \usepackage{unicode-math}
  \defaultfontfeatures{Scale=MatchLowercase}
  \defaultfontfeatures[\rmfamily]{Ligatures=TeX,Scale=1}
\fi
\usepackage{lmodern}
\ifPDFTeX\else  
    % xetex/luatex font selection
\fi
% Use upquote if available, for straight quotes in verbatim environments
\IfFileExists{upquote.sty}{\usepackage{upquote}}{}
\IfFileExists{microtype.sty}{% use microtype if available
  \usepackage[]{microtype}
  \UseMicrotypeSet[protrusion]{basicmath} % disable protrusion for tt fonts
}{}
\makeatletter
\@ifundefined{KOMAClassName}{% if non-KOMA class
  \IfFileExists{parskip.sty}{%
    \usepackage{parskip}
  }{% else
    \setlength{\parindent}{0pt}
    \setlength{\parskip}{6pt plus 2pt minus 1pt}}
}{% if KOMA class
  \KOMAoptions{parskip=half}}
\makeatother
\usepackage{xcolor}
\usepackage[top=2.54cm,right=2.54cm,bottom=2.54cm,left=2.54cm]{geometry}
\setlength{\emergencystretch}{3em} % prevent overfull lines
\setcounter{secnumdepth}{-\maxdimen} % remove section numbering
% Make \paragraph and \subparagraph free-standing
\ifx\paragraph\undefined\else
  \let\oldparagraph\paragraph
  \renewcommand{\paragraph}[1]{\oldparagraph{#1}\mbox{}}
\fi
\ifx\subparagraph\undefined\else
  \let\oldsubparagraph\subparagraph
  \renewcommand{\subparagraph}[1]{\oldsubparagraph{#1}\mbox{}}
\fi


\providecommand{\tightlist}{%
  \setlength{\itemsep}{0pt}\setlength{\parskip}{0pt}}\usepackage{longtable,booktabs,array}
\usepackage{calc} % for calculating minipage widths
% Correct order of tables after \paragraph or \subparagraph
\usepackage{etoolbox}
\makeatletter
\patchcmd\longtable{\par}{\if@noskipsec\mbox{}\fi\par}{}{}
\makeatother
% Allow footnotes in longtable head/foot
\IfFileExists{footnotehyper.sty}{\usepackage{footnotehyper}}{\usepackage{footnote}}
\makesavenoteenv{longtable}
\usepackage{graphicx}
\makeatletter
\def\maxwidth{\ifdim\Gin@nat@width>\linewidth\linewidth\else\Gin@nat@width\fi}
\def\maxheight{\ifdim\Gin@nat@height>\textheight\textheight\else\Gin@nat@height\fi}
\makeatother
% Scale images if necessary, so that they will not overflow the page
% margins by default, and it is still possible to overwrite the defaults
% using explicit options in \includegraphics[width, height, ...]{}
\setkeys{Gin}{width=\maxwidth,height=\maxheight,keepaspectratio}
% Set default figure placement to htbp
\makeatletter
\def\fps@figure{htbp}
\makeatother

\makeatletter
\makeatother
\makeatletter
\makeatother
\makeatletter
\@ifpackageloaded{caption}{}{\usepackage{caption}}
\AtBeginDocument{%
\ifdefined\contentsname
  \renewcommand*\contentsname{Tabla de contenidos}
\else
  \newcommand\contentsname{Tabla de contenidos}
\fi
\ifdefined\listfigurename
  \renewcommand*\listfigurename{Listado de Figuras}
\else
  \newcommand\listfigurename{Listado de Figuras}
\fi
\ifdefined\listtablename
  \renewcommand*\listtablename{Listado de Tablas}
\else
  \newcommand\listtablename{Listado de Tablas}
\fi
\ifdefined\figurename
  \renewcommand*\figurename{Figura}
\else
  \newcommand\figurename{Figura}
\fi
\ifdefined\tablename
  \renewcommand*\tablename{Tabla}
\else
  \newcommand\tablename{Tabla}
\fi
}
\@ifpackageloaded{float}{}{\usepackage{float}}
\floatstyle{ruled}
\@ifundefined{c@chapter}{\newfloat{codelisting}{h}{lop}}{\newfloat{codelisting}{h}{lop}[chapter]}
\floatname{codelisting}{Listado}
\newcommand*\listoflistings{\listof{codelisting}{Listado de Listados}}
\makeatother
\makeatletter
\@ifpackageloaded{caption}{}{\usepackage{caption}}
\@ifpackageloaded{subcaption}{}{\usepackage{subcaption}}
\makeatother
\makeatletter
\@ifpackageloaded{tcolorbox}{}{\usepackage[skins,breakable]{tcolorbox}}
\makeatother
\makeatletter
\@ifundefined{shadecolor}{\definecolor{shadecolor}{rgb}{.97, .97, .97}}
\makeatother
\makeatletter
\makeatother
\makeatletter
\makeatother
\ifLuaTeX
\usepackage[bidi=basic]{babel}
\else
\usepackage[bidi=default]{babel}
\fi
\babelprovide[main,import]{spanish}
% get rid of language-specific shorthands (see #6817):
\let\LanguageShortHands\languageshorthands
\def\languageshorthands#1{}
\ifLuaTeX
  \usepackage{selnolig}  % disable illegal ligatures
\fi
\usepackage[]{biblatex}
\addbibresource{../../../../references.bib}
\IfFileExists{bookmark.sty}{\usepackage{bookmark}}{\usepackage{hyperref}}
\IfFileExists{xurl.sty}{\usepackage{xurl}}{} % add URL line breaks if available
\urlstyle{same} % disable monospaced font for URLs
\hypersetup{
  pdfauthor={Edison Achalma},
  pdflang={es},
  colorlinks=true,
  linkcolor={blue},
  filecolor={Maroon},
  citecolor={Blue},
  urlcolor={Blue},
  pdfcreator={LaTeX via pandoc}}

\author{Edison Achalma}
\date{}

\begin{document}
\ifdefined\Shaded\renewenvironment{Shaded}{\begin{tcolorbox}[boxrule=0pt, breakable, interior hidden, borderline west={3pt}{0pt}{shadecolor}, frame hidden, enhanced, sharp corners]}{\end{tcolorbox}}\fi

\hypertarget{quuxe9-es-i3wm}{%
\section{¿Qué es i3wm?}\label{quuxe9-es-i3wm}}

\hypertarget{definiciuxf3n-de-i3wm}{%
\subsection{Definición de i3wm}\label{definiciuxf3n-de-i3wm}}

\hypertarget{caracteruxedsticas-principales-de-i3wm}{%
\subsection{Características principales de
i3wm}\label{caracteruxedsticas-principales-de-i3wm}}

\hypertarget{ventajas-de-utilizar-i3wm}{%
\section{Ventajas de utilizar i3wm}\label{ventajas-de-utilizar-i3wm}}

\hypertarget{eficiencia-y-rendimiento}{%
\subsection{Eficiencia y rendimiento}\label{eficiencia-y-rendimiento}}

\hypertarget{enfoque-minimalista}{%
\subsection{Enfoque minimalista}\label{enfoque-minimalista}}

\hypertarget{personalizaciuxf3n-y-flexibilidad}{%
\subsection{Personalización y
flexibilidad}\label{personalizaciuxf3n-y-flexibilidad}}

\hypertarget{uso-eficiente-del-espacio-en-el-escritorio}{%
\subsection{Uso eficiente del espacio en el
escritorio}\label{uso-eficiente-del-espacio-en-el-escritorio}}

\hypertarget{conceptos-fundamentales-de-i3wm}{%
\section{Conceptos fundamentales de
i3wm}\label{conceptos-fundamentales-de-i3wm}}

\hypertarget{gestiuxf3n-de-ventanas-en-mosaico-tiling-window-management}{%
\subsection{Gestión de ventanas en mosaico (tiling window
management)}\label{gestiuxf3n-de-ventanas-en-mosaico-tiling-window-management}}

\hypertarget{espacios-de-trabajo-workspaces-3.3.-modos-y-barras-de-estado-modes-and-status-bars}{%
\subsection{Espacios de trabajo (workspaces) 3.3. Modos y barras de
estado (modes and status
bars)}\label{espacios-de-trabajo-workspaces-3.3.-modos-y-barras-de-estado-modes-and-status-bars}}

\hypertarget{atajos-de-teclado-y-comandos-buxe1sicos}{%
\subsection{Atajos de teclado y comandos
básicos}\label{atajos-de-teclado-y-comandos-buxe1sicos}}

\hypertarget{instalaciuxf3n-de-i3wm}{%
\section{Instalación de i3wm}\label{instalaciuxf3n-de-i3wm}}

\hypertarget{requisitos-del-sistema}{%
\subsection{Requisitos del sistema}\label{requisitos-del-sistema}}

\hypertarget{instalaciuxf3n-en-distribuciones-populares-ejemplo-ubuntu-arch-linux}{%
\subsection{Instalación en distribuciones populares (ejemplo: Ubuntu,
Arch
Linux)}\label{instalaciuxf3n-en-distribuciones-populares-ejemplo-ubuntu-arch-linux}}

\hypertarget{primeros-pasos-con-i3wm}{%
\section{Primeros pasos con i3wm}\label{primeros-pasos-con-i3wm}}

\hypertarget{iniciar-sesiuxf3n-en-i3wm}{%
\subsection{Iniciar sesión en i3wm}\label{iniciar-sesiuxf3n-en-i3wm}}

\hypertarget{orientaciuxf3n-buxe1sica-en-el-entorno-de-i3wm}{%
\subsection{Orientación básica en el entorno de
i3wm}\label{orientaciuxf3n-buxe1sica-en-el-entorno-de-i3wm}}

\hypertarget{navegaciuxf3n-entre-ventanas-y-workspaces}{%
\subsection{Navegación entre ventanas y
workspaces}\label{navegaciuxf3n-entre-ventanas-y-workspaces}}

\hypertarget{recursos-adicionales-y-comunidad}{%
\section{Recursos adicionales y
comunidad}\label{recursos-adicionales-y-comunidad}}

\hypertarget{documentaciuxf3n-oficial-y-guuxedas-de-referencia}{%
\subsection{Documentación oficial y guías de
referencia}\label{documentaciuxf3n-oficial-y-guuxedas-de-referencia}}

\hypertarget{comunidades-en-luxednea-y-foros-de-discusiuxf3n}{%
\subsection{Comunidades en línea y foros de
discusión}\label{comunidades-en-luxednea-y-foros-de-discusiuxf3n}}

\hypertarget{sitios-web-y-blogs-recomendados-para-aprender-muxe1s-sobre-i3wm}{%
\subsection{Sitios web y blogs recomendados para aprender más sobre
i3wm}\label{sitios-web-y-blogs-recomendados-para-aprender-muxe1s-sobre-i3wm}}


\printbibliography


\end{document}
