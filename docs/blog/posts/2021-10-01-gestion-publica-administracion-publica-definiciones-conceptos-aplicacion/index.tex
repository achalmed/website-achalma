% Options for packages loaded elsewhere
\PassOptionsToPackage{unicode}{hyperref}
\PassOptionsToPackage{hyphens}{url}
\PassOptionsToPackage{dvipsnames,svgnames,x11names}{xcolor}
%
\documentclass[
  a4paper,
]{article}

\usepackage{amsmath,amssymb}
\usepackage{iftex}
\ifPDFTeX
  \usepackage[T1]{fontenc}
  \usepackage[utf8]{inputenc}
  \usepackage{textcomp} % provide euro and other symbols
\else % if luatex or xetex
  \usepackage{unicode-math}
  \defaultfontfeatures{Scale=MatchLowercase}
  \defaultfontfeatures[\rmfamily]{Ligatures=TeX,Scale=1}
\fi
\usepackage{lmodern}
\ifPDFTeX\else  
    % xetex/luatex font selection
\fi
% Use upquote if available, for straight quotes in verbatim environments
\IfFileExists{upquote.sty}{\usepackage{upquote}}{}
\IfFileExists{microtype.sty}{% use microtype if available
  \usepackage[]{microtype}
  \UseMicrotypeSet[protrusion]{basicmath} % disable protrusion for tt fonts
}{}
\makeatletter
\@ifundefined{KOMAClassName}{% if non-KOMA class
  \IfFileExists{parskip.sty}{%
    \usepackage{parskip}
  }{% else
    \setlength{\parindent}{0pt}
    \setlength{\parskip}{6pt plus 2pt minus 1pt}}
}{% if KOMA class
  \KOMAoptions{parskip=half}}
\makeatother
\usepackage{xcolor}
\usepackage[top=2.54cm,right=2.54cm,bottom=2.54cm,left=2.54cm]{geometry}
\setlength{\emergencystretch}{3em} % prevent overfull lines
\setcounter{secnumdepth}{-\maxdimen} % remove section numbering
% Make \paragraph and \subparagraph free-standing
\ifx\paragraph\undefined\else
  \let\oldparagraph\paragraph
  \renewcommand{\paragraph}[1]{\oldparagraph{#1}\mbox{}}
\fi
\ifx\subparagraph\undefined\else
  \let\oldsubparagraph\subparagraph
  \renewcommand{\subparagraph}[1]{\oldsubparagraph{#1}\mbox{}}
\fi


\providecommand{\tightlist}{%
  \setlength{\itemsep}{0pt}\setlength{\parskip}{0pt}}\usepackage{longtable,booktabs,array}
\usepackage{calc} % for calculating minipage widths
% Correct order of tables after \paragraph or \subparagraph
\usepackage{etoolbox}
\makeatletter
\patchcmd\longtable{\par}{\if@noskipsec\mbox{}\fi\par}{}{}
\makeatother
% Allow footnotes in longtable head/foot
\IfFileExists{footnotehyper.sty}{\usepackage{footnotehyper}}{\usepackage{footnote}}
\makesavenoteenv{longtable}
\usepackage{graphicx}
\makeatletter
\def\maxwidth{\ifdim\Gin@nat@width>\linewidth\linewidth\else\Gin@nat@width\fi}
\def\maxheight{\ifdim\Gin@nat@height>\textheight\textheight\else\Gin@nat@height\fi}
\makeatother
% Scale images if necessary, so that they will not overflow the page
% margins by default, and it is still possible to overwrite the defaults
% using explicit options in \includegraphics[width, height, ...]{}
\setkeys{Gin}{width=\maxwidth,height=\maxheight,keepaspectratio}
% Set default figure placement to htbp
\makeatletter
\def\fps@figure{htbp}
\makeatother

\makeatletter
\makeatother
\makeatletter
\makeatother
\makeatletter
\@ifpackageloaded{caption}{}{\usepackage{caption}}
\AtBeginDocument{%
\ifdefined\contentsname
  \renewcommand*\contentsname{Tabla de contenidos}
\else
  \newcommand\contentsname{Tabla de contenidos}
\fi
\ifdefined\listfigurename
  \renewcommand*\listfigurename{Listado de Figuras}
\else
  \newcommand\listfigurename{Listado de Figuras}
\fi
\ifdefined\listtablename
  \renewcommand*\listtablename{Listado de Tablas}
\else
  \newcommand\listtablename{Listado de Tablas}
\fi
\ifdefined\figurename
  \renewcommand*\figurename{Figura}
\else
  \newcommand\figurename{Figura}
\fi
\ifdefined\tablename
  \renewcommand*\tablename{Tabla}
\else
  \newcommand\tablename{Tabla}
\fi
}
\@ifpackageloaded{float}{}{\usepackage{float}}
\floatstyle{ruled}
\@ifundefined{c@chapter}{\newfloat{codelisting}{h}{lop}}{\newfloat{codelisting}{h}{lop}[chapter]}
\floatname{codelisting}{Listado}
\newcommand*\listoflistings{\listof{codelisting}{Listado de Listados}}
\makeatother
\makeatletter
\@ifpackageloaded{caption}{}{\usepackage{caption}}
\@ifpackageloaded{subcaption}{}{\usepackage{subcaption}}
\makeatother
\makeatletter
\@ifpackageloaded{tcolorbox}{}{\usepackage[skins,breakable]{tcolorbox}}
\makeatother
\makeatletter
\@ifundefined{shadecolor}{\definecolor{shadecolor}{rgb}{.97, .97, .97}}
\makeatother
\makeatletter
\makeatother
\makeatletter
\makeatother
\ifLuaTeX
\usepackage[bidi=basic]{babel}
\else
\usepackage[bidi=default]{babel}
\fi
\babelprovide[main,import]{spanish}
% get rid of language-specific shorthands (see #6817):
\let\LanguageShortHands\languageshorthands
\def\languageshorthands#1{}
\ifLuaTeX
  \usepackage{selnolig}  % disable illegal ligatures
\fi
\usepackage[]{biblatex}
\addbibresource{../../../../references.bib}
\IfFileExists{bookmark.sty}{\usepackage{bookmark}}{\usepackage{hyperref}}
\IfFileExists{xurl.sty}{\usepackage{xurl}}{} % add URL line breaks if available
\urlstyle{same} % disable monospaced font for URLs
\hypersetup{
  pdftitle={Gestión Pública y Administración Pública Definiciones, Conceptos y Aplicación},
  pdfauthor={Edison Achalma},
  pdflang={es},
  colorlinks=true,
  linkcolor={blue},
  filecolor={Maroon},
  citecolor={Blue},
  urlcolor={Blue},
  pdfcreator={LaTeX via pandoc}}

\title{Gestión Pública y Administración Pública Definiciones, Conceptos
y Aplicación}
\usepackage{etoolbox}
\makeatletter
\providecommand{\subtitle}[1]{% add subtitle to \maketitle
  \apptocmd{\@title}{\par {\large #1 \par}}{}{}
}
\makeatother
\subtitle{Comprendiendo la diferencia y la importancia de estas
categorías en el ámbito gubernamental.}
\author{Edison Achalma}
\date{2021-10-01}

\begin{document}
\maketitle
\ifdefined\Shaded\renewenvironment{Shaded}{\begin{tcolorbox}[sharp corners, frame hidden, interior hidden, breakable, boxrule=0pt, borderline west={3pt}{0pt}{shadecolor}, enhanced]}{\end{tcolorbox}}\fi

\hypertarget{indagar-definiciones-conceptos-y-aplicaciuxf3n-de-gestiuxf3n-puxfablica-y-administraciuxf3n-puxfablica.}{%
\section{Indagar definiciones, conceptos y aplicación de gestión pública
y administración
pública.}\label{indagar-definiciones-conceptos-y-aplicaciuxf3n-de-gestiuxf3n-puxfablica-y-administraciuxf3n-puxfablica.}}

\hypertarget{administraciuxf3n-puxfablica}{%
\subsection{Administración pública}\label{administraciuxf3n-puxfablica}}

Son actividades que se realiza para el buen manejo de recursos públicos.
Estas actividades la pueden realizar las instituciones, organismos ya
sean públicos o privados. Además, dichas instituciones pertenecer a
cualquiera de los 3 niveles de gobiernos: nacional, regional o local.

La administración pública está muy vinculada a 2 conceptos: Burocracia y
al resguardo de los recursos del estado. Además, se dice que los
objetivos de la administración pública no están muy detallados.

La administración pública perdura en el tiempo, porque lo constituyen
las instituciones del estado.

\hypertarget{gestiuxf3n-puxfablica}{%
\subsection{Gestión pública}\label{gestiuxf3n-puxfablica}}

La gestión pública se ocupa del uso de los medios adecuados para
alcanzar un fin colectivo, es decir trata de las herramientas de
decisión para la asignación y distribución de los recursos públicos.

En pocas palabras es un conjunto de acciones y actos mediante las cuales
las entidades de la administración pública tienden al logro de sus
fines, objetivos y/o metas.

Los objetivos que se logran mediante la gestión pública están muy
definidos tanto en el corto como en el largo plazo.

En conclusión, la diferencia principal entre ambos conceptos es que la
gestión pública es la parte más dinámica, puesto que es el conjunto de
actividades o acciones que se realizan para alcanzar los fines del
interés público, por otra parte, la administración pública es la parte
menos dinámica que recoge todas las acciones realizadas para alcanzar
los fines del estado.

\hypertarget{quuxe9-diferencia-existe-entre-la-gestiuxf3n-puxfablica-y-la-administraciuxf3n-puxfablica}{%
\section{¿Qué diferencia existe entre la gestión pública y la
administración
pública?}\label{quuxe9-diferencia-existe-entre-la-gestiuxf3n-puxfablica-y-la-administraciuxf3n-puxfablica}}

GESTIÓN PÚBLICA

ADMINISTRACIÓN PÚBLICA

\begin{itemize}
\tightlist
\item
  Es un conjunto de procesos y acciones mediante los cuales las
  entidades tienden al logro de sus fines, objetivos y metas.
\item
  Sus resultados se alcanzan a través de la gestión de políticas,
  recursos y programas.
\item
  Es más práctica para alcanzar los objetivos de un fin colectivo de la
  ciudadanía.
\item
  Cambia con los gobiernos.
\item
  Cambia su carácter político e ideológico.
\item
  Es persona jurídica y un conjunto de organizaciones que tienen una
  estructura piramidal y jerárquica.
\item
  Está integradas y marcadas por principios de legalidad, eficiencia y
  eficacia.
\item
  Está subordinadas al gobierno de turno para canalizar y atender las
  demandas sociales.
\item
  Perdura en el tiempo.
\item
  Se conforma por ``cuoteo político'', donde puede primar los intereses
  políticos sobre el bien común.
\end{itemize}

\hypertarget{son-sinuxf3nimos-la-gestiuxf3n-puxfablica-con-la-administraciuxf3n-puxfablica}{%
\subsection{¿Son sinónimos la gestión pública con la administración
pública?}\label{son-sinuxf3nimos-la-gestiuxf3n-puxfablica-con-la-administraciuxf3n-puxfablica}}

Para responder esta pregunta remontémonos a clases de lenguaje y
razonamiento verbal, recordemos que para que dos palabras o términos
sean sinónimos deben cumplir tres requisitos tales como pertenecer al
mismo campo semántico, la misma clase gramatical y el tercero y más
importante poseer significado parecido. Ahora analicemos los
significados de ambas frases, por un lado, la administración pública son
actividades que se realiza para el buen manejo de recursos públicos y la
gestión pública se ocupa del uso de los medios adecuados para alcanzar
un fin colectivo; pero, ¿este fin colectivo a través de qué? A través
del adecuado uso racional, eficiente, eficaz alcanzando la efectividad
de los recursos públicos o del estado.

Por este significado ya mostrado podemos decir y concluir que la
administración pública y la gestión pública son sinónimos, pero no
poseen el mismo significado, eso sí tienen significados parecidos, pero
no iguales valga la redundancia, incluso ambas palabras pertenecen al
mismo campo semántico y clase gramatical. Bueno llegamos a esta
conclusión desde el punto de vista con enfoque gramatical de su
significado y uso en el área de economía, administración y claramente en
su significado de uso en el sector público.

\hypertarget{cuuxe1l-es-la-importancia-de-cada-una-de-estas-categoruxedas-en-la-gestiuxf3n-gubernamental}{%
\subsection{¿Cuál es la importancia de cada una de estas categorías en
la gestión
gubernamental?}\label{cuuxe1l-es-la-importancia-de-cada-una-de-estas-categoruxedas-en-la-gestiuxf3n-gubernamental}}

\textbf{Gestión pública:}

La gestión pública es importante porque determina el correcto uso de los
recursos públicos y la consecución de las metas, planes y políticas
nacionales, sectoriales, regionales y locales.

En este contexto, la gestión pública se transforma en un repertorio
importante de instrumentos que se aplican para desarrollar las fuerzas
productivas, obteniendo resultados que tienen efecto directo en la vida
social y, en ese sentido, sí alude a la calidad de los métodos
relacionados con el cómo gobernar y cómo administrar la sociedad.

Además, la gestión pública es fundamental para que la capacidad de los
gobiernos demuestre sus atributos institucionales, los cuales son
indispensables para que cualquier déficit de Gobierno, sea contextuado
en los valores de la institucionalidad, la cual permite la coexistencia
de las instituciones y dependencias públicas de las distintas esferas.

\textbf{Administración pública:}

La importancia de la administración pública radica en que ésta se
encuentra ligada a la vida cotidiana de la ciudadanía, porque es la que
nos provee de los servicios públicos indispensables para la vida
comunitaria básica, que permite la convivencia de quienes se reúnen para
vivir en sociedad, y tales servicios están relacionados con el
transporte, seguridad pública y medioambiente.

\hypertarget{cuuxe1l-es-el-campo-de-aplicaciuxf3n-de-la-gestiuxf3n-puxfablica-y-la-administraciuxf3n-puxfablica-seuxf1ale-casos-y-ejemplos.}{%
\subsection{¿Cuál es el campo de aplicación de la gestión pública y la
administración pública?, señale casos y
ejemplos.}\label{cuuxe1l-es-el-campo-de-aplicaciuxf3n-de-la-gestiuxf3n-puxfablica-y-la-administraciuxf3n-puxfablica-seuxf1ale-casos-y-ejemplos.}}

\textbf{Campo de aplicación de:}

\textbf{Gestión pública:} El funcionamiento de las entidades públicas se
debe a la presencia de autoridades políticas y servidores públicos que
asumen un conjunto de atribuciones de acuerdo al cargo que ocupa en el
marco del mandato que le asigna su ley de creación. Las autoridades
políticas, en su rol decisor y el servidor público como ejecutante,
tienen que actuar de manera articulada y complementaria, dentro del
escalón que les asigna la organización.

Los decisores políticos tienen la responsabilidad de adoptar políticas
que conlleven a enfrentar y resolver problemas que afectan a la
comunidad en los diferentes ámbitos de su responsabilidad. Estas
políticas son adoptadas individualmente o reunidos en el colectivo al
que pertenecen (Consejo de ministros, Consejos Regionales, Concejos
Municipales, un directorio, etc.). Por su parte los servidores públicos
ponen en práctica las políticas haciendo uso de un conjunto de
tecnologías de gestión e instrumentos de orden normativo y gerencial

\textbf{Administración pública:} La~administración pública~es
un~campo~donde los líderes sirven a las comunidades para promover el
bien común y lograr un cambio positivo en el sector~público. Las
aptitudes que se requieren en el~campo de la administración pública,
tales como la gestión de proyectos y programas, se suelen transferir al
sector privado. Los administradores públicos a menudo, proporcionan
información a los clientes y supervisan e implementan varios programas
para las organizaciones.~Suelen ser responsables de dirigir y asesorar a
los empleados de la organización, como investigadores, oficiales de
programas y consultores. También, pueden evaluar programas y servicios
dentro de una organización, así como implementar cambios en iniciativas
de políticas públicas, el seguimiento y la aprobación de presupuestos,

Ejemplos:

\begin{itemize}
\tightlist
\item
  el programa Pensión 65 (también premiada por buenas prácticas) con su
  plan saberes productivos, estrategia para la protección de las
  personas adultas mayores en Vulnerabilidad.
\item
  el Proyecto de Ampliación Lima Airport Partners (LAP) del aeropuerto
  Jorge Chávez, el cual garantiza para el 2024 un aeropuerto al nivel de
  las instalaciones europeas más modernas, facilitando y promoviendo el
  turismo, el comercio y las comunicaciones.
\end{itemize}


\printbibliography


\end{document}
