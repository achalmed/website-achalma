% Options for packages loaded elsewhere
\PassOptionsToPackage{unicode}{hyperref}
\PassOptionsToPackage{hyphens}{url}
\PassOptionsToPackage{dvipsnames,svgnames,x11names}{xcolor}
%
\documentclass[
  letterpaper,
  DIV=11,
  numbers=noendperiod]{scrartcl}

\usepackage{amsmath,amssymb}
\usepackage{iftex}
\ifPDFTeX
  \usepackage[T1]{fontenc}
  \usepackage[utf8]{inputenc}
  \usepackage{textcomp} % provide euro and other symbols
\else % if luatex or xetex
  \usepackage{unicode-math}
  \defaultfontfeatures{Scale=MatchLowercase}
  \defaultfontfeatures[\rmfamily]{Ligatures=TeX,Scale=1}
\fi
\usepackage{lmodern}
\ifPDFTeX\else  
    % xetex/luatex font selection
\fi
% Use upquote if available, for straight quotes in verbatim environments
\IfFileExists{upquote.sty}{\usepackage{upquote}}{}
\IfFileExists{microtype.sty}{% use microtype if available
  \usepackage[]{microtype}
  \UseMicrotypeSet[protrusion]{basicmath} % disable protrusion for tt fonts
}{}
\makeatletter
\@ifundefined{KOMAClassName}{% if non-KOMA class
  \IfFileExists{parskip.sty}{%
    \usepackage{parskip}
  }{% else
    \setlength{\parindent}{0pt}
    \setlength{\parskip}{6pt plus 2pt minus 1pt}}
}{% if KOMA class
  \KOMAoptions{parskip=half}}
\makeatother
\usepackage{xcolor}
\setlength{\emergencystretch}{3em} % prevent overfull lines
\setcounter{secnumdepth}{-\maxdimen} % remove section numbering
% Make \paragraph and \subparagraph free-standing
\ifx\paragraph\undefined\else
  \let\oldparagraph\paragraph
  \renewcommand{\paragraph}[1]{\oldparagraph{#1}\mbox{}}
\fi
\ifx\subparagraph\undefined\else
  \let\oldsubparagraph\subparagraph
  \renewcommand{\subparagraph}[1]{\oldsubparagraph{#1}\mbox{}}
\fi


\providecommand{\tightlist}{%
  \setlength{\itemsep}{0pt}\setlength{\parskip}{0pt}}\usepackage{longtable,booktabs,array}
\usepackage{calc} % for calculating minipage widths
% Correct order of tables after \paragraph or \subparagraph
\usepackage{etoolbox}
\makeatletter
\patchcmd\longtable{\par}{\if@noskipsec\mbox{}\fi\par}{}{}
\makeatother
% Allow footnotes in longtable head/foot
\IfFileExists{footnotehyper.sty}{\usepackage{footnotehyper}}{\usepackage{footnote}}
\makesavenoteenv{longtable}
\usepackage{graphicx}
\makeatletter
\def\maxwidth{\ifdim\Gin@nat@width>\linewidth\linewidth\else\Gin@nat@width\fi}
\def\maxheight{\ifdim\Gin@nat@height>\textheight\textheight\else\Gin@nat@height\fi}
\makeatother
% Scale images if necessary, so that they will not overflow the page
% margins by default, and it is still possible to overwrite the defaults
% using explicit options in \includegraphics[width, height, ...]{}
\setkeys{Gin}{width=\maxwidth,height=\maxheight,keepaspectratio}
% Set default figure placement to htbp
\makeatletter
\def\fps@figure{htbp}
\makeatother

\KOMAoption{captions}{tableheading,figureheading}
\makeatletter
\makeatother
\makeatletter
\makeatother
\makeatletter
\@ifpackageloaded{caption}{}{\usepackage{caption}}
\AtBeginDocument{%
\ifdefined\contentsname
  \renewcommand*\contentsname{Tabla de contenidos}
\else
  \newcommand\contentsname{Tabla de contenidos}
\fi
\ifdefined\listfigurename
  \renewcommand*\listfigurename{Listado de Figuras}
\else
  \newcommand\listfigurename{Listado de Figuras}
\fi
\ifdefined\listtablename
  \renewcommand*\listtablename{Listado de Tablas}
\else
  \newcommand\listtablename{Listado de Tablas}
\fi
\ifdefined\figurename
  \renewcommand*\figurename{Figura}
\else
  \newcommand\figurename{Figura}
\fi
\ifdefined\tablename
  \renewcommand*\tablename{Tabla}
\else
  \newcommand\tablename{Tabla}
\fi
}
\@ifpackageloaded{float}{}{\usepackage{float}}
\floatstyle{ruled}
\@ifundefined{c@chapter}{\newfloat{codelisting}{h}{lop}}{\newfloat{codelisting}{h}{lop}[chapter]}
\floatname{codelisting}{Listado}
\newcommand*\listoflistings{\listof{codelisting}{Listado de Listados}}
\makeatother
\makeatletter
\@ifpackageloaded{caption}{}{\usepackage{caption}}
\@ifpackageloaded{subcaption}{}{\usepackage{subcaption}}
\makeatother
\makeatletter
\@ifpackageloaded{tcolorbox}{}{\usepackage[skins,breakable]{tcolorbox}}
\makeatother
\makeatletter
\@ifundefined{shadecolor}{\definecolor{shadecolor}{rgb}{.97, .97, .97}}
\makeatother
\makeatletter
\makeatother
\makeatletter
\makeatother
\ifLuaTeX
\usepackage[bidi=basic]{babel}
\else
\usepackage[bidi=default]{babel}
\fi
\babelprovide[main,import]{spanish}
% get rid of language-specific shorthands (see #6817):
\let\LanguageShortHands\languageshorthands
\def\languageshorthands#1{}
\ifLuaTeX
  \usepackage{selnolig}  % disable illegal ligatures
\fi
\usepackage[]{biblatex}
\addbibresource{../../../../references.bib}
\IfFileExists{bookmark.sty}{\usepackage{bookmark}}{\usepackage{hyperref}}
\IfFileExists{xurl.sty}{\usepackage{xurl}}{} % add URL line breaks if available
\urlstyle{same} % disable monospaced font for URLs
\hypersetup{
  pdftitle={Reformas y Modernización de la Gestión Pública en Perú.},
  pdfauthor={Achalma Mendoza Edison},
  pdflang={es},
  colorlinks=true,
  linkcolor={blue},
  filecolor={Maroon},
  citecolor={Blue},
  urlcolor={Blue},
  pdfcreator={LaTeX via pandoc}}

\title{Reformas y Modernización de la Gestión Pública en Perú.}
\usepackage{etoolbox}
\makeatletter
\providecommand{\subtitle}[1]{% add subtitle to \maketitle
  \apptocmd{\@title}{\par {\large #1 \par}}{}{}
}
\makeatother
\subtitle{Hacia una gestión eficiente y transparente en beneficio de la
sociedad.}
\author{Achalma Mendoza Edison}
\date{2021-10-01}

\begin{document}
\maketitle
\ifdefined\Shaded\renewenvironment{Shaded}{\begin{tcolorbox}[interior hidden, frame hidden, sharp corners, breakable, boxrule=0pt, enhanced, borderline west={3pt}{0pt}{shadecolor}]}{\end{tcolorbox}}\fi

\hypertarget{quuxe9-hacer}{%
\section{¿Qué hacer?}\label{quuxe9-hacer}}

\hypertarget{reformar-la-gestiuxf3n-puxfablica}{%
\subsection{¿Reformar la gestión
pública?}\label{reformar-la-gestiuxf3n-puxfablica}}

Para la reforma de la gestión pública se deben realizar acciones
orientadas a incrementar los niveles de eficiencia y eficacia en la
gestión pública a fin de que logre resultados en beneficio de los
ciudadanos.

En este sentido se debe hacer especial mención en términos de tres hitos
que expresan claramente una reforma de la gestión pública en Perú.

El primero guarda relación con el desarrollo de un Plan Estratégico de
Modernización de la Gestión Pública, donde se debe definir una agenda de
cambios en diversos ámbitos de la gestión pública, tales como mejora de
los sistemas de planeación, la instalación de mejores mecanismos de
control de gestión y la medición del quehacer de las instituciones
públicas.

Segundo la instalación de un sistema de evaluación y control de gestión
por parte de la Dirección de Presupuestos del Ministerio de Economía.

Y finalmente la creación del Sistema de Alta Dirección Pública que debe
dotar a las instituciones de gobierno, a través de concursos públicos y
transparentes, de directivos con probada capacidad de gestión y
liderazgo para ejecutar de forma eficaz y eficiente las políticas
públicas definidas por la autoridad.

\hypertarget{reformar-el-estado}{%
\subsection{¿Reformar el estado?}\label{reformar-el-estado}}

Para saber si reformar o qué reformas hacer del Estado debemos tener en
cuenta algunas consideraciones iniciales. En ese sentido comenzamos a
definir al Estado como el principal agente encargado del bienestar de la
sociedad. Pero el Estado es una estructura dantesca que funciona bajo
ciertas organizaciones, instituciones, normas jurídicas, reglas de
juego, etc. En ese sentido el Gobierno, que es la representación del
Estado busca brindar bienestar de la forma más eficiente y eficaz
posible. Cuando los componentes del Estado no funcionan o si funciona,
pero de forma ineficiente se requiere hacer una reforma del aparato
estatal. Una reforma de estado tiene implicancia en el sector económico,
político, social, etc.

En el Perú existe desde inicios del siglo XXI un aparato público muy
endeble y se agravó producto de la pandemia del COVID-19 que trajo como
consecuencia un incremento en la brecha social. Además de incrementarse
la pobreza monetaria.

\hypertarget{reforma-poluxedtica}{%
\subsubsection{Reforma política}\label{reforma-poluxedtica}}

Modificar la forma en que los gobiernos regionales y locales contratan a
los gestores públicos. Que los gobiernos no contraten por afinidad; sino
que los contratos lo realicen la Autoridad Nacional del Servicio Civil
(SERVIR)

\hypertarget{reforma-econuxf3mica}{%
\subsubsection{Reforma económica}\label{reforma-econuxf3mica}}

Las empresas nacionales y extranjeras deben competir en igualdad de
condiciones.

\hypertarget{reforma-de-salud}{%
\subsubsection{Reforma de salud}\label{reforma-de-salud}}

La digitalización de la atención médica. Implica que los doctores de
distintas especialidades puedan atender a las personas de los lugares
más alejados mediante plataformas digitales.

\hypertarget{reforma-agraria}{%
\subsubsection{Reforma agraria}\label{reforma-agraria}}

Incentivar la producción intensiva para la exportación de bienes que se
producen en territorios peruano

\hypertarget{reforma-tributaria}{%
\subsubsection{Reforma tributaria}\label{reforma-tributaria}}

Que se incluya a los personajes influyentes (actores, gamers, cantantes,
faranduleros) que brindan publicidad mediante sus redes sociales al pago
de la renta de cuarta categoría.

Para realizar estas reformas es necesario que el gobierno tenga la
capacidad de articular al sector público, privado y las universidades.
Sin estas articulaciones las reformas del Estado serían esfuerzos
perdidos.

\hypertarget{modernizar-la-gestiuxf3n-puxfablica}{%
\subsection{¿Modernizar la gestión
pública?}\label{modernizar-la-gestiuxf3n-puxfablica}}

Se debe promover e intensificar en el país, una imagen positiva de la
gestión pública, que debe estar al servicio de la sociedad peruana,
dentro de un contexto de excelencia y no ser considerado como un sistema
burocrático, sobredimensionado y conflicto social. La elaboración de los
planes de desarrollo nacional a corto, mediano y largo plazo permitirán
que el sector empresarial privado plantee sus planes tácticos y
estratégicos, no sólo basado en la investigación de mercados y en la
rentabilidad empresarial, sino también en el interés nacional.

La tarea inmediata es generar una cultura organizacional propia para la
gestión pública, basada en la ética, transparencia y eficiencia en su
accionar y en la prestación de sus servicios, constituyéndose en un
paradigma de desarrollo nacional.

A diferencia de una empresa privada, la cual tiene como incentivo la
maximización de su rentabilidad; las entidades públicas no cuentan con
un estímulo natural para mejorar su funcionamiento. En consecuencia, las
personas no pueden ``optar por cambiar'' a la entidad pública con la
cual realiza un trámite u obtiene un servicio, como sí lo puede hacer
con un proveedor privado en caso éste no satisfaga sus expectativas y
necesidades. Por ello, a través de la modernización de la gestión
pública se busca generar incentivos para que las entidades del Estado
mejoren constantemente su funcionamiento e intervenciones (bienes,
servicios y regulaciones) de forma eficiente, orientada a resultados y
teniendo como prioridad a las personas.

La razón de ser de toda entidad y servidor público es brindar una mejor
calidad de bienes, servicios y regulaciones a las personas. Se debe
comprender que no se realiza ``un favor'' al ciudadano, sino es parte
del cumplimiento de su trabajo.

Para ello, en el año 2019, la SGP emitió los Principios de Actuación
para la Modernización de la Gestión Pública, los cuales tienen como
objetivo orientar a las entidades hacia una correcta conceptualización
de lo que es la modernización y hacia una gestión basada en la creación
de valor público. Para ello, se plantean un conjunto de principios de
actuación que todos los servidores civiles se comprometen a cumplir con
el fin de acercarse y recuperar la confianza de las personas. Estos son:

\begin{itemize}
\tightlist
\item
  Diseñar e implementar políticas públicas cuyos resultados generen
  valor.
\item
  Medir los resultados de las intervenciones.
\item
  Pensar de manera sistémica los problemas.
\item
  Responder mejor a las personas.
\item
  Contar con bienes y servicios de calidad.
\item
  Emitir regulaciones de calidad.
\item
  Mejorar la productividad de las entidades públicas.
\item
  Diseñar estructuras organizacionales interconectadas, ágiles y
  adaptables.
\end{itemize}

Con dichos lineamientos se busca que los servidores públicos comprendan
a cabalidad su rol en la satisfacción de las necesidades de las personas
y de la sociedad.

A pesar de dicha implementación y de todas las medidas tomadas en aras
de lograr una modernización en la gestión pública brindada, no se han
observado cambios significativos o los que se efectuaron pasaron
desapercibidos, por lo tanto, se debe complementar a las medidas, un
cambio absoluto en el control del cumplimiento de estas, ya que por
mucho que se quiera cambiar, si es que no se hacen efectivas, al final
solo se quedan en propuestas y los problemas continúan.

Por ello, los puntos que se debe considerar para una modernización más
efectiva en la gestión pública son:

\begin{itemize}
\tightlist
\item
  Incorporar en la gestión pública a los mejores trabajadores; con la
  finalidad
\item
  Capacitación permanente a los trabajadores
\item
  Simplificación administrativa, a fin de generar resultados positivos
  en la mejora de los procedimientos y servicios orientados a los
  ciudadanos y empresas.
\item
  Implementación y uso intensivo de las TIC como soporte a los procesos
  de planificación, producción y gestión de las entidades públicas
  permitiendo a su vez consolidar propuestas de gobierno abierto.
\end{itemize}

\hypertarget{modernizar-el-estado}{%
\subsection{¿Modernizar el estado?}\label{modernizar-el-estado}}

Dado que muchos bienes y servicios están bajo la asignación,
supervisión, regulación y administración del estado, buscamos que se nos
brinden estos de manera más eficiente, eficaz y de manera equitativa,
una manera óptima; pues ¿necesitamos modernizarnos y/o adaptarnos a
nuevas políticas públicas? O qué solución le damos, ante esto muchos nos
hacemos esta pregunta ¿Cómo modernizamos el estado?

Modernizar el estado implica muchos aspectos de mejora y superación
continua de los retos y objetivos del aparato estatal, claro basta
redundar que se deben incluir la misión y visión de este, teniendo estos
la característica de adaptabilidad al contexto de gobernabilidad
democrática y relaciones emergentes entre el estado y los demás agentes
económicos, lo cual implica adecuar la administración pública y hasta la
gestión pública de acuerdo a los planes y políticas de desarrollo y
crecimiento económico nacional, mediante reformas en el aparato estatal
que faciliten el adecuada funcionamiento eficaz de este.

Tras hacer un análisis a nuestro aparato estatal determinamos retos para
poder modernizarlo tales como establecer objetivos claros con orden de
prioridad que muestren resultados al corto y largo plazo, seguir con la
ley de descentralización de poderes, una simplificación administrativa
con menos trabas en procesos burocráticos, el profesionalismo de la
función pública multiplicando la capacidad de servicio del estado con
una adecuada articulación integral de políticas y con transparencia de
estas como su respectiva eficiencia, generalizando modernización de la
gestión pública.

En la actualidad, el estado peruano podrá modernizarse siempre y cuando
cumpla con los retos establecidos en sus planes de desarrollo nacional,
llevará en buen camino la modernización de este con el adecuado uso de
las reformas y objetivos trasados en los planes y políticas de
desarrollo nacional ya establecidos en cada gobierno.

\hypertarget{reformar-y-modernizar-la-gestiuxf3n-puxfablica}{%
\subsection{¿Reformar y modernizar la gestión
pública?}\label{reformar-y-modernizar-la-gestiuxf3n-puxfablica}}

El proceso de modernización y reforma del estado implica cambios
trascendentales en toda la estructura y organización del aparato
estatal. Involucrar a todos los actores constituye un verdadero desafío.
La capacitación del personal y la adecuada publicidad de la estrategia y
los objetivos estratégicos resultan trascendentales para lograr el
compromiso de todos los miembros. El proceso de cambio es un camino
complejo, que encuentra muchos escollos a su paso, en especial la
resistencia de los propios dependientes. La cultura organizacional, los
estilos de liderazgo y la estructura constituyen algunos de los ejemplos
de las potenciales dificultades en el avance.

En este sentido tenemos las principales acciones para reformar y
modernizar la gestión pública:

\begin{itemize}
\tightlist
\item
  Mejorar la calidad de prestación de bienes y servicios por parte del
  estado para cerrar brechas en esta acción.
\item
  Se debe mejorar una convergencia de las actividades y los puntos de
  vista de la sociedad civil y las fuerzas políticas hacia el diseño de
  una visión compartida y los planes a nivel mundial estratégico
\item
  Una descentralización que busca el fortalecimiento de los gobiernos
  locales y regionales como un mecanismo para y lograr la efectiva
  descentralización, este proceso debe ser de forma gradual.
\item
  Se necesita mayor eficacia y eficiencia en el logro de los objetivos y
  en uso de los recursos por parte del estado. Esta es una actividad
  fundamental ya que primero se debe fijar correctamente los objetivos y
  utilizarse los recursos que tiene el estado. para lograr este uso
  eficiente de los recursos del estado, se debe eliminar la duplicidad y
  la superposición de funciones, competencias y atribuciones por parte
  de los diversos organismos que tiene el estado.
\item
  Institucionalizar la evaluación de la gestión por resultados para
  medir las instituciones en base a los resultados ya que por ahora no
  hay una sanción especifica por haber tenido malos resultados y esto
  implica también que las capacidades del estado tienen que verse
  fortalecidas.
\end{itemize}

Esta tarea constituye un verdadero desafío que trasciende el color
político de turno, persiguiendo como único fin el bienestar de toda la
comunidad. Asimismo, implica la capacitación de los recursos humanos y
el logro de la cohesión de objetivos organizacionales e individuales,
brindando transparencia en la gestión y combatiendo la corrupción.


\printbibliography


\end{document}
