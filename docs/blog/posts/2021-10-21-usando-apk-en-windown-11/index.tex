% Options for packages loaded elsewhere
\PassOptionsToPackage{unicode}{hyperref}
\PassOptionsToPackage{hyphens}{url}
\PassOptionsToPackage{dvipsnames,svgnames,x11names}{xcolor}
%
\documentclass[
  letterpaper,
  DIV=11,
  numbers=noendperiod]{scrartcl}

\usepackage{amsmath,amssymb}
\usepackage{iftex}
\ifPDFTeX
  \usepackage[T1]{fontenc}
  \usepackage[utf8]{inputenc}
  \usepackage{textcomp} % provide euro and other symbols
\else % if luatex or xetex
  \usepackage{unicode-math}
  \defaultfontfeatures{Scale=MatchLowercase}
  \defaultfontfeatures[\rmfamily]{Ligatures=TeX,Scale=1}
\fi
\usepackage{lmodern}
\ifPDFTeX\else  
    % xetex/luatex font selection
\fi
% Use upquote if available, for straight quotes in verbatim environments
\IfFileExists{upquote.sty}{\usepackage{upquote}}{}
\IfFileExists{microtype.sty}{% use microtype if available
  \usepackage[]{microtype}
  \UseMicrotypeSet[protrusion]{basicmath} % disable protrusion for tt fonts
}{}
\makeatletter
\@ifundefined{KOMAClassName}{% if non-KOMA class
  \IfFileExists{parskip.sty}{%
    \usepackage{parskip}
  }{% else
    \setlength{\parindent}{0pt}
    \setlength{\parskip}{6pt plus 2pt minus 1pt}}
}{% if KOMA class
  \KOMAoptions{parskip=half}}
\makeatother
\usepackage{xcolor}
\setlength{\emergencystretch}{3em} % prevent overfull lines
\setcounter{secnumdepth}{-\maxdimen} % remove section numbering
% Make \paragraph and \subparagraph free-standing
\ifx\paragraph\undefined\else
  \let\oldparagraph\paragraph
  \renewcommand{\paragraph}[1]{\oldparagraph{#1}\mbox{}}
\fi
\ifx\subparagraph\undefined\else
  \let\oldsubparagraph\subparagraph
  \renewcommand{\subparagraph}[1]{\oldsubparagraph{#1}\mbox{}}
\fi


\providecommand{\tightlist}{%
  \setlength{\itemsep}{0pt}\setlength{\parskip}{0pt}}\usepackage{longtable,booktabs,array}
\usepackage{calc} % for calculating minipage widths
% Correct order of tables after \paragraph or \subparagraph
\usepackage{etoolbox}
\makeatletter
\patchcmd\longtable{\par}{\if@noskipsec\mbox{}\fi\par}{}{}
\makeatother
% Allow footnotes in longtable head/foot
\IfFileExists{footnotehyper.sty}{\usepackage{footnotehyper}}{\usepackage{footnote}}
\makesavenoteenv{longtable}
\usepackage{graphicx}
\makeatletter
\def\maxwidth{\ifdim\Gin@nat@width>\linewidth\linewidth\else\Gin@nat@width\fi}
\def\maxheight{\ifdim\Gin@nat@height>\textheight\textheight\else\Gin@nat@height\fi}
\makeatother
% Scale images if necessary, so that they will not overflow the page
% margins by default, and it is still possible to overwrite the defaults
% using explicit options in \includegraphics[width, height, ...]{}
\setkeys{Gin}{width=\maxwidth,height=\maxheight,keepaspectratio}
% Set default figure placement to htbp
\makeatletter
\def\fps@figure{htbp}
\makeatother

\KOMAoption{captions}{tableheading,figureheading}
\makeatletter
\makeatother
\makeatletter
\makeatother
\makeatletter
\@ifpackageloaded{caption}{}{\usepackage{caption}}
\AtBeginDocument{%
\ifdefined\contentsname
  \renewcommand*\contentsname{Tabla de contenidos}
\else
  \newcommand\contentsname{Tabla de contenidos}
\fi
\ifdefined\listfigurename
  \renewcommand*\listfigurename{Listado de Figuras}
\else
  \newcommand\listfigurename{Listado de Figuras}
\fi
\ifdefined\listtablename
  \renewcommand*\listtablename{Listado de Tablas}
\else
  \newcommand\listtablename{Listado de Tablas}
\fi
\ifdefined\figurename
  \renewcommand*\figurename{Figura}
\else
  \newcommand\figurename{Figura}
\fi
\ifdefined\tablename
  \renewcommand*\tablename{Tabla}
\else
  \newcommand\tablename{Tabla}
\fi
}
\@ifpackageloaded{float}{}{\usepackage{float}}
\floatstyle{ruled}
\@ifundefined{c@chapter}{\newfloat{codelisting}{h}{lop}}{\newfloat{codelisting}{h}{lop}[chapter]}
\floatname{codelisting}{Listado}
\newcommand*\listoflistings{\listof{codelisting}{Listado de Listados}}
\makeatother
\makeatletter
\@ifpackageloaded{caption}{}{\usepackage{caption}}
\@ifpackageloaded{subcaption}{}{\usepackage{subcaption}}
\makeatother
\makeatletter
\@ifpackageloaded{tcolorbox}{}{\usepackage[skins,breakable]{tcolorbox}}
\makeatother
\makeatletter
\@ifundefined{shadecolor}{\definecolor{shadecolor}{rgb}{.97, .97, .97}}
\makeatother
\makeatletter
\makeatother
\makeatletter
\makeatother
\ifLuaTeX
\usepackage[bidi=basic]{babel}
\else
\usepackage[bidi=default]{babel}
\fi
\babelprovide[main,import]{spanish}
% get rid of language-specific shorthands (see #6817):
\let\LanguageShortHands\languageshorthands
\def\languageshorthands#1{}
\ifLuaTeX
  \usepackage{selnolig}  % disable illegal ligatures
\fi
\usepackage[]{biblatex}
\addbibresource{../../../../references.bib}
\IfFileExists{bookmark.sty}{\usepackage{bookmark}}{\usepackage{hyperref}}
\IfFileExists{xurl.sty}{\usepackage{xurl}}{} % add URL line breaks if available
\urlstyle{same} % disable monospaced font for URLs
\hypersetup{
  pdftitle={Cómo usar APK en Windows 11 una guía paso a paso},
  pdfauthor={Edison Achalma Mendoza},
  pdflang={es},
  colorlinks=true,
  linkcolor={blue},
  filecolor={Maroon},
  citecolor={Blue},
  urlcolor={Blue},
  pdfcreator={LaTeX via pandoc}}

\title{Cómo usar APK en Windows 11 una guía paso a paso}
\usepackage{etoolbox}
\makeatletter
\providecommand{\subtitle}[1]{% add subtitle to \maketitle
  \apptocmd{\@title}{\par {\large #1 \par}}{}{}
}
\makeatother
\subtitle{Aprende a instalar y ejecutar aplicaciones de Android en tu PC
con Windows 11}
\author{Edison Achalma Mendoza}
\date{2022-10-21}

\begin{document}
\maketitle
\ifdefined\Shaded\renewenvironment{Shaded}{\begin{tcolorbox}[sharp corners, breakable, boxrule=0pt, borderline west={3pt}{0pt}{shadecolor}, enhanced, interior hidden, frame hidden]}{\end{tcolorbox}}\fi

\hypertarget{windows-11-cuxf3mo-descargar-apk-usando-el-subsistema-de-windows-para-android-y-adb}{%
\section{Windows 11: Cómo descargar APK usando el subsistema de Windows
para Android y
ADB}\label{windows-11-cuxf3mo-descargar-apk-usando-el-subsistema-de-windows-para-android-y-adb}}

Aquí se explica cómo descargar un archivo APK para instalar la
aplicación de Android en su PC con Windows 11 usando el Subsistema de
Windows para Android.~Puede instalar Windows Subsystem para Android
manualmente en su PC con Windows 11 usando su archivo Msixbundle de
nuestra~\href{https://nerdschalk.com/android-apps-on-windows-11-dev-channel-how-to-install-windows-subsystem-for-android-manually-with-msixbundle/}{guía
aquí}~.

\hypertarget{paso-1-habilite-el-modo-de-desarrollador-en-el-subsistema-de-windows}{%
\subsection{Paso 1: habilite el modo de desarrollador en el subsistema
de
Windows}\label{paso-1-habilite-el-modo-de-desarrollador-en-el-subsistema-de-windows}}

Instale~\href{https://nerdschalk.com/android-apps-on-windows-11-dev-channel-how-to-install-windows-subsystem-for-android-manually-with-msixbundle/}{el
Subsistema de Windows para Android}~primero.~Cuando haya terminado, abra
la aplicación `Subsistema de Windows para Android' en su PC.~Para esto,
presione la tecla de Windows y busque Subsistema de Windows para
Android.

Haga clic en Subsistema de Windows para Android.~O haga clic en Abrir.

En el Subsistema de Windows para Android, active el modo Desarrollador.

\hypertarget{paso-2-instale-las-herramientas-de-la-plataforma-sdk}{%
\subsection{Paso 2: Instale las herramientas de la plataforma
SDK}\label{paso-2-instale-las-herramientas-de-la-plataforma-sdk}}

Visite la página de herramientas de la plataforma SDK de
Google~\href{https://developer.android.com/studio/releases/platform-tools.html}{aquí}~.

Haga clic en Descargar SDK Platform-Tools para Windows.

Desplácese hacia abajo hasta el final y seleccione la casilla de
verificación para aceptar los términos y condiciones.~Luego haga clic en
el botón verde para descargar las herramientas de la plataforma.

Se descargará en su PC un archivo zip llamado
platform-tools\_r31.0.3-windows (la versión puede cambiar).

Para su comodidad, cree una nueva carpeta separada llamada carpeta para
aplicaciones en el Explorador de Windows.~Ahora, transfiera el archivo
de herramientas de la plataforma a esta carpeta.

Haga clic con el botón derecho en el archivo de herramientas de la
plataforma y seleccione Extraer todo.

Haga clic en Extraer.

El archivo será extraído.~Abra la carpeta llamada herramientas de la
plataforma.

Tendrá adb.exe y algunos otros archivos aquí.

\hypertarget{paso-3-instale-la-aplicaciuxf3n-de-android}{%
\subsection{Paso 3: Instale la aplicación de
Android}\label{paso-3-instale-la-aplicaciuxf3n-de-android}}

Haga doble clic en la carpeta de herramientas de la plataforma para
abrirla.

Aquí, haga clic en la barra de direcciones y escriba~\textbf{cmd,}~y
luego presione la tecla Intro.

Se abrirá una ventana de comando con su ubicación establecida en la
carpeta de herramientas de la plataforma.~Esto es importante.

Ahora, tenemos la ventana del símbolo del sistema en la carpeta donde
tenemos el archivo adb.exe.~Es decir, nuestra carpeta de herramientas de
plataforma.

Ahora, descargue el archivo APK de la aplicación de Android que desea
instalar.~Por ejemplo, si desea instalar Snapchat,
busque~\textbf{Snapchat APK}~en Google y descargue su archivo APK desde
cualquier sitio web confiable en el que confíe.~Luego, cambie el nombre
del archivo a algo más simple como snapchat.apk.~Ahora, transfiera
snapchat.apk a la carpeta de herramientas de la plataforma.

Ahora podemos instalar la aplicación Snapchat para Android usando
snapchat.apk y adb en su PC.

Abra el Subsistema de Windows para Android y busque la IP donde se puede
conectar con ADB en la opción de modo Desarrollador.

En la ventana del símbolo del sistema, escriba el siguiente comando y
presione Entrar:

adb.exe connect (dirección IP-aquí)

Ejemplo: adb.exe conectar 127.0.0.1:12345

Ahora, escriba el comando de instalación que se proporciona a
continuación y luego presione Enter:

adb.exe install (apk-file-name-here.apk)

Ejemplo: adb.exe instalar Snapchat.apk

La aplicación de Android ahora se instalará en su PC usando ADB y el
archivo APK que proporcionó.

Cuando haya terminado, verá el mensaje de Éxito.

Puede cerrar la ventana CMD ahora.

Ahora puede abrir la aplicación de Android en su PC.

Presione la tecla de Windows y luego escriba el nombre de su
aplicación.~En nuestro caso, es Snapchat.

Así es como se ve Snapchat en Windows 11.

Eso es todo.

\hypertarget{cargar-apk-automuxe1ticamente-con-un-doble-clic}{%
\subsection{Cargar APK automáticamente con un doble
clic}\label{cargar-apk-automuxe1ticamente-con-un-doble-clic}}

Sabemos que usar un comando adb no es la forma más fácil de instalar una
aplicación de Android en su PC.~Afortunadamente, ahora puede hacer doble
clic en un archivo APK para instalarlo.~Consulta el siguiente enlace
para saber cómo configurarlo.


\printbibliography


\end{document}
