% Options for packages loaded elsewhere
\PassOptionsToPackage{unicode}{hyperref}
\PassOptionsToPackage{hyphens}{url}
\PassOptionsToPackage{dvipsnames,svgnames,x11names}{xcolor}
%
\documentclass[
  letterpaper,
  DIV=11,
  numbers=noendperiod]{scrartcl}

\usepackage{amsmath,amssymb}
\usepackage{iftex}
\ifPDFTeX
  \usepackage[T1]{fontenc}
  \usepackage[utf8]{inputenc}
  \usepackage{textcomp} % provide euro and other symbols
\else % if luatex or xetex
  \usepackage{unicode-math}
  \defaultfontfeatures{Scale=MatchLowercase}
  \defaultfontfeatures[\rmfamily]{Ligatures=TeX,Scale=1}
\fi
\usepackage{lmodern}
\ifPDFTeX\else  
    % xetex/luatex font selection
\fi
% Use upquote if available, for straight quotes in verbatim environments
\IfFileExists{upquote.sty}{\usepackage{upquote}}{}
\IfFileExists{microtype.sty}{% use microtype if available
  \usepackage[]{microtype}
  \UseMicrotypeSet[protrusion]{basicmath} % disable protrusion for tt fonts
}{}
\makeatletter
\@ifundefined{KOMAClassName}{% if non-KOMA class
  \IfFileExists{parskip.sty}{%
    \usepackage{parskip}
  }{% else
    \setlength{\parindent}{0pt}
    \setlength{\parskip}{6pt plus 2pt minus 1pt}}
}{% if KOMA class
  \KOMAoptions{parskip=half}}
\makeatother
\usepackage{xcolor}
\setlength{\emergencystretch}{3em} % prevent overfull lines
\setcounter{secnumdepth}{-\maxdimen} % remove section numbering
% Make \paragraph and \subparagraph free-standing
\ifx\paragraph\undefined\else
  \let\oldparagraph\paragraph
  \renewcommand{\paragraph}[1]{\oldparagraph{#1}\mbox{}}
\fi
\ifx\subparagraph\undefined\else
  \let\oldsubparagraph\subparagraph
  \renewcommand{\subparagraph}[1]{\oldsubparagraph{#1}\mbox{}}
\fi


\providecommand{\tightlist}{%
  \setlength{\itemsep}{0pt}\setlength{\parskip}{0pt}}\usepackage{longtable,booktabs,array}
\usepackage{calc} % for calculating minipage widths
% Correct order of tables after \paragraph or \subparagraph
\usepackage{etoolbox}
\makeatletter
\patchcmd\longtable{\par}{\if@noskipsec\mbox{}\fi\par}{}{}
\makeatother
% Allow footnotes in longtable head/foot
\IfFileExists{footnotehyper.sty}{\usepackage{footnotehyper}}{\usepackage{footnote}}
\makesavenoteenv{longtable}
\usepackage{graphicx}
\makeatletter
\def\maxwidth{\ifdim\Gin@nat@width>\linewidth\linewidth\else\Gin@nat@width\fi}
\def\maxheight{\ifdim\Gin@nat@height>\textheight\textheight\else\Gin@nat@height\fi}
\makeatother
% Scale images if necessary, so that they will not overflow the page
% margins by default, and it is still possible to overwrite the defaults
% using explicit options in \includegraphics[width, height, ...]{}
\setkeys{Gin}{width=\maxwidth,height=\maxheight,keepaspectratio}
% Set default figure placement to htbp
\makeatletter
\def\fps@figure{htbp}
\makeatother

\KOMAoption{captions}{tableheading,figureheading}
\makeatletter
\makeatother
\makeatletter
\makeatother
\makeatletter
\@ifpackageloaded{caption}{}{\usepackage{caption}}
\AtBeginDocument{%
\ifdefined\contentsname
  \renewcommand*\contentsname{Tabla de contenidos}
\else
  \newcommand\contentsname{Tabla de contenidos}
\fi
\ifdefined\listfigurename
  \renewcommand*\listfigurename{Listado de Figuras}
\else
  \newcommand\listfigurename{Listado de Figuras}
\fi
\ifdefined\listtablename
  \renewcommand*\listtablename{Listado de Tablas}
\else
  \newcommand\listtablename{Listado de Tablas}
\fi
\ifdefined\figurename
  \renewcommand*\figurename{Figura}
\else
  \newcommand\figurename{Figura}
\fi
\ifdefined\tablename
  \renewcommand*\tablename{Tabla}
\else
  \newcommand\tablename{Tabla}
\fi
}
\@ifpackageloaded{float}{}{\usepackage{float}}
\floatstyle{ruled}
\@ifundefined{c@chapter}{\newfloat{codelisting}{h}{lop}}{\newfloat{codelisting}{h}{lop}[chapter]}
\floatname{codelisting}{Listado}
\newcommand*\listoflistings{\listof{codelisting}{Listado de Listados}}
\makeatother
\makeatletter
\@ifpackageloaded{caption}{}{\usepackage{caption}}
\@ifpackageloaded{subcaption}{}{\usepackage{subcaption}}
\makeatother
\makeatletter
\@ifpackageloaded{tcolorbox}{}{\usepackage[skins,breakable]{tcolorbox}}
\makeatother
\makeatletter
\@ifundefined{shadecolor}{\definecolor{shadecolor}{rgb}{.97, .97, .97}}
\makeatother
\makeatletter
\makeatother
\makeatletter
\makeatother
\ifLuaTeX
\usepackage[bidi=basic]{babel}
\else
\usepackage[bidi=default]{babel}
\fi
\babelprovide[main,import]{spanish}
% get rid of language-specific shorthands (see #6817):
\let\LanguageShortHands\languageshorthands
\def\languageshorthands#1{}
\ifLuaTeX
  \usepackage{selnolig}  % disable illegal ligatures
\fi
\usepackage[]{biblatex}
\addbibresource{../../../../references.bib}
\IfFileExists{bookmark.sty}{\usepackage{bookmark}}{\usepackage{hyperref}}
\IfFileExists{xurl.sty}{\usepackage{xurl}}{} % add URL line breaks if available
\urlstyle{same} % disable monospaced font for URLs
\hypersetup{
  pdftitle={Comandos básicos de Vim para mejorar tu flujo de trabajo},
  pdfauthor={Edison Achalma},
  pdflang={es},
  colorlinks=true,
  linkcolor={blue},
  filecolor={Maroon},
  citecolor={Blue},
  urlcolor={Blue},
  pdfcreator={LaTeX via pandoc}}

\title{Comandos básicos de Vim para mejorar tu flujo de trabajo}
\usepackage{etoolbox}
\makeatletter
\providecommand{\subtitle}[1]{% add subtitle to \maketitle
  \apptocmd{\@title}{\par {\large #1 \par}}{}{}
}
\makeatother
\subtitle{Aprende a utilizar los comandos esenciales de Vim para ser más
productivo en tu programación.}
\author{Edison Achalma}
\date{2022-09-27}

\begin{document}
\maketitle
\ifdefined\Shaded\renewenvironment{Shaded}{\begin{tcolorbox}[boxrule=0pt, borderline west={3pt}{0pt}{shadecolor}, interior hidden, sharp corners, frame hidden, breakable, enhanced]}{\end{tcolorbox}}\fi

Vim es un editor de texto muy poderoso utilizado en sistemas Linux y
Unix. A continuación, se presentan algunos de los comandos y
combinaciones de teclas más utilizados en Vim:

\begin{enumerate}
\def\labelenumi{\arabic{enumi}.}
\item
  Modos de Vim:

  \begin{itemize}
  \item
    Modo de comandos: el modo predeterminado de Vim, en el que se pueden
    ingresar comandos para editar el texto. Para activar el modo comando
    en Vim, debes presionar la tecla ``Esc''. Esto te llevará al modo
    comando desde cualquier otro modo en el que te encuentres, como el
    modo insertar o el modo de reemplazo. Una vez que estés en el modo
    comando, puedes utilizar una variedad de comandos y combinaciones de
    teclas para navegar, editar y guardar tus archivos. Para salir de
    Vim, puedes ingresar el comando ``:q'' seguido de Enter. Si has
    realizado cambios y deseas guardarlos antes de salir, utiliza el
    comando ``:wq'' para escribir y guardar los cambios y salir de Vim.
  \item
    Modo de inserción: el modo en el que se puede ingresar texto normal.
  \item
    Modo de visualización: el modo utilizado para seleccionar y
    manipular bloques de texto.
  \end{itemize}
\item
  Comandos de movimiento de cursor:

  \begin{itemize}
  \tightlist
  \item
    h: mueve el cursor una posición a la izquierda.
  \item
    j: mueve el cursor una posición hacia abajo.
  \item
    k: mueve el cursor una posición hacia arriba.
  \item
    l: mueve el cursor una posición a la derecha.
  \item
    0: mueve el cursor al inicio de la línea.
  \item
    \$: mueve el cursor al final de la línea.
  \item
    w: mueve el cursor a la siguiente palabra.
  \item
    b: mueve el cursor a la palabra anterior.
  \item
    gg: mueve el cursor al inicio del archivo.
  \item
    G: mueve el cursor al final del archivo.
  \item
    :numero: mueve el cursor a la línea con el número especificado.
  \end{itemize}
\item
  Comandos de edición:

  \begin{itemize}
  \tightlist
  \item
    i: entra en el modo de inserción antes del cursor.
  \item
    a: entra en el modo de inserción después del cursor.
  \item
    o: inserta una nueva línea debajo del cursor y entra en el modo de
    inserción.
  \item
    d: elimina el texto seleccionado.
  \item
    y: copia el texto seleccionado.
  \item
    p: pega el texto copiado o eliminado después del cursor.
  \item
    u: deshace la última acción.
  \item
    Ctrl+r: rehace la última acción.
  \item
    :w: guarda el archivo.
  \item
    :q: sale de Vim.
  \item
    :q!: sale de Vim sin guardar los cambios.
  \end{itemize}
\item
  Comandos de búsqueda y reemplazo:

  \begin{itemize}
  \tightlist
  \item
    /: busca el texto especificado hacia adelante.
  \item
    ?: busca el texto especificado hacia atrás.
  \item
    n: busca la siguiente ocurrencia del texto especificado.
  \item
    N: busca la ocurrencia anterior del texto especificado.
  \item
    :s/old/new/g: reemplaza la primera ocurrencia de ``old'' con ``new''
    en la línea actual.
  \item
    :s/old/new/gc: reemplaza todas las ocurrencias de ``old'' con
    ``new'' en la línea actual y pide confirmación antes de cada
    reemplazo.
  \end{itemize}
\end{enumerate}

Estos son solo algunos de los comandos y combinaciones de teclas más
utilizados en Vim. Hay muchos más disponibles, y la lista completa se
puede encontrar en la documentación de Vim.


\printbibliography


\end{document}
