% Options for packages loaded elsewhere
\PassOptionsToPackage{unicode}{hyperref}
\PassOptionsToPackage{hyphens}{url}
\PassOptionsToPackage{dvipsnames,svgnames,x11names}{xcolor}
%
\documentclass[
  a4paper,
]{article}

\usepackage{amsmath,amssymb}
\usepackage{iftex}
\ifPDFTeX
  \usepackage[T1]{fontenc}
  \usepackage[utf8]{inputenc}
  \usepackage{textcomp} % provide euro and other symbols
\else % if luatex or xetex
  \usepackage{unicode-math}
  \defaultfontfeatures{Scale=MatchLowercase}
  \defaultfontfeatures[\rmfamily]{Ligatures=TeX,Scale=1}
\fi
\usepackage{lmodern}
\ifPDFTeX\else  
    % xetex/luatex font selection
\fi
% Use upquote if available, for straight quotes in verbatim environments
\IfFileExists{upquote.sty}{\usepackage{upquote}}{}
\IfFileExists{microtype.sty}{% use microtype if available
  \usepackage[]{microtype}
  \UseMicrotypeSet[protrusion]{basicmath} % disable protrusion for tt fonts
}{}
\makeatletter
\@ifundefined{KOMAClassName}{% if non-KOMA class
  \IfFileExists{parskip.sty}{%
    \usepackage{parskip}
  }{% else
    \setlength{\parindent}{0pt}
    \setlength{\parskip}{6pt plus 2pt minus 1pt}}
}{% if KOMA class
  \KOMAoptions{parskip=half}}
\makeatother
\usepackage{xcolor}
\usepackage[top=2.54cm,right=2.54cm,bottom=2.54cm,left=2.54cm]{geometry}
\setlength{\emergencystretch}{3em} % prevent overfull lines
\setcounter{secnumdepth}{-\maxdimen} % remove section numbering
% Make \paragraph and \subparagraph free-standing
\ifx\paragraph\undefined\else
  \let\oldparagraph\paragraph
  \renewcommand{\paragraph}[1]{\oldparagraph{#1}\mbox{}}
\fi
\ifx\subparagraph\undefined\else
  \let\oldsubparagraph\subparagraph
  \renewcommand{\subparagraph}[1]{\oldsubparagraph{#1}\mbox{}}
\fi


\providecommand{\tightlist}{%
  \setlength{\itemsep}{0pt}\setlength{\parskip}{0pt}}\usepackage{longtable,booktabs,array}
\usepackage{calc} % for calculating minipage widths
% Correct order of tables after \paragraph or \subparagraph
\usepackage{etoolbox}
\makeatletter
\patchcmd\longtable{\par}{\if@noskipsec\mbox{}\fi\par}{}{}
\makeatother
% Allow footnotes in longtable head/foot
\IfFileExists{footnotehyper.sty}{\usepackage{footnotehyper}}{\usepackage{footnote}}
\makesavenoteenv{longtable}
\usepackage{graphicx}
\makeatletter
\def\maxwidth{\ifdim\Gin@nat@width>\linewidth\linewidth\else\Gin@nat@width\fi}
\def\maxheight{\ifdim\Gin@nat@height>\textheight\textheight\else\Gin@nat@height\fi}
\makeatother
% Scale images if necessary, so that they will not overflow the page
% margins by default, and it is still possible to overwrite the defaults
% using explicit options in \includegraphics[width, height, ...]{}
\setkeys{Gin}{width=\maxwidth,height=\maxheight,keepaspectratio}
% Set default figure placement to htbp
\makeatletter
\def\fps@figure{htbp}
\makeatother

\makeatletter
\makeatother
\makeatletter
\makeatother
\makeatletter
\@ifpackageloaded{caption}{}{\usepackage{caption}}
\AtBeginDocument{%
\ifdefined\contentsname
  \renewcommand*\contentsname{Tabla de contenidos}
\else
  \newcommand\contentsname{Tabla de contenidos}
\fi
\ifdefined\listfigurename
  \renewcommand*\listfigurename{Listado de Figuras}
\else
  \newcommand\listfigurename{Listado de Figuras}
\fi
\ifdefined\listtablename
  \renewcommand*\listtablename{Listado de Tablas}
\else
  \newcommand\listtablename{Listado de Tablas}
\fi
\ifdefined\figurename
  \renewcommand*\figurename{Figura}
\else
  \newcommand\figurename{Figura}
\fi
\ifdefined\tablename
  \renewcommand*\tablename{Tabla}
\else
  \newcommand\tablename{Tabla}
\fi
}
\@ifpackageloaded{float}{}{\usepackage{float}}
\floatstyle{ruled}
\@ifundefined{c@chapter}{\newfloat{codelisting}{h}{lop}}{\newfloat{codelisting}{h}{lop}[chapter]}
\floatname{codelisting}{Listado}
\newcommand*\listoflistings{\listof{codelisting}{Listado de Listados}}
\makeatother
\makeatletter
\@ifpackageloaded{caption}{}{\usepackage{caption}}
\@ifpackageloaded{subcaption}{}{\usepackage{subcaption}}
\makeatother
\makeatletter
\@ifpackageloaded{tcolorbox}{}{\usepackage[skins,breakable]{tcolorbox}}
\makeatother
\makeatletter
\@ifundefined{shadecolor}{\definecolor{shadecolor}{rgb}{.97, .97, .97}}
\makeatother
\makeatletter
\makeatother
\makeatletter
\makeatother
\ifLuaTeX
\usepackage[bidi=basic]{babel}
\else
\usepackage[bidi=default]{babel}
\fi
\babelprovide[main,import]{spanish}
% get rid of language-specific shorthands (see #6817):
\let\LanguageShortHands\languageshorthands
\def\languageshorthands#1{}
\ifLuaTeX
  \usepackage{selnolig}  % disable illegal ligatures
\fi
\usepackage[]{biblatex}
\addbibresource{../../../../references.bib}
\IfFileExists{bookmark.sty}{\usepackage{bookmark}}{\usepackage{hyperref}}
\IfFileExists{xurl.sty}{\usepackage{xurl}}{} % add URL line breaks if available
\urlstyle{same} % disable monospaced font for URLs
\hypersetup{
  pdftitle={Día Internacional de los Trabajadores de mayo de 2023.},
  pdfauthor={Comité Central Partido Comunista del Perú},
  pdflang={es},
  colorlinks=true,
  linkcolor={blue},
  filecolor={Maroon},
  citecolor={Blue},
  urlcolor={Blue},
  pdfcreator={LaTeX via pandoc}}

\title{Día Internacional de los Trabajadores de mayo de 2023.}
\usepackage{etoolbox}
\makeatletter
\providecommand{\subtitle}[1]{% add subtitle to \maketitle
  \apptocmd{\@title}{\par {\large #1 \par}}{}{}
}
\makeatother
\subtitle{Mensaje del Primero de Mayo del Partido Comunista del Perú.}
\author{Comité Central Partido Comunista del Perú}
\date{2023-05-01}

\begin{document}
\maketitle
\ifdefined\Shaded\renewenvironment{Shaded}{\begin{tcolorbox}[breakable, borderline west={3pt}{0pt}{shadecolor}, interior hidden, enhanced, frame hidden, sharp corners, boxrule=0pt]}{\end{tcolorbox}}\fi

\hypertarget{en-este-primero-de-mayo-seguir-sembrando-revoluciuxf3n}{%
\section{¡En este primero de mayo: seguir sembrando
revolución!}\label{en-este-primero-de-mayo-seguir-sembrando-revoluciuxf3n}}

\begin{quote}
¡Proletarios de todos los países, uníos!
\end{quote}

En el día del proletariado internacional, persistiendo en la tradición e
impronta revolucionarias enseñadas por nuestro Presidente Gonzalo, el
Partido Comunista del Perú saluda con júbilo comunista a la clase obrera
y al pueblo peruano y del mundo, reafirmando su indeclinable compromiso
de enarbolar, defender y aplicar el marxismo-leninismo-maoísmo, nuestra
arma ideológica estratégica universal y el pensamiento gonzalo, nuestra
arma ideológica estratégica, específica y principal.

Cuando la bipolaridad entre Estados Unidos y China se sigue
desarrollando para un nuevo reparto del mundo y aumenta el peligro de
una tercera guerra mundial, la clase obrera persiste en su lucha,
combate la explotación capitalista y, enfrentando adversidades, brega
decididamente por constituir, reconstituir o fortalecer sus partidos
comunistas y avanza en la unidad de los comunistas, enarbolando el
marxismo-leninismo-maoísmo, principalmente el maoísmo, para que los
partidos comunistas dirijan la revolución y preparen la guerra popular
para derrotar la guerra contrarrevolucionaria y construir la nueva
sociedad.

Nuestro Presidente Gonzalo nos enseñó:

\begin{quote}
Pensamos que la historia del proletariado es la historia de su
ideología: el marxismo-leninismo-maoísmo; es la historia de su partido:
el Partido Comunista; y es la historia de su revolución: la revolución
proletaria mundial, esto es su lucha por instaurar la dictadura del
proletariado, construir el socialismo y marchar al comunismo. A la vez,
la historia del proletariado es confirmación cotidiana de la ley
fundamental: la contradicción, pues toda la vida del proletariado
muestra: la lucha es lo absoluto y la victoria, relativa; y esta se
logra a través de fracasos que también son relativos.
\end{quote}

Del Manifiesto del Partido Comunista hasta hoy han pasado 175 años en
los que la clase ha desarrollado su ideología hasta llegar al
marxismo-leninismo-maoísmo, principalmente maoísmo, el cual es necesidad
vital enarbolarlo hoy. En ese proceso los partidos comunistas en dura
lucha contra el revisionismo y la reacción han devenido en partidos de
nuevo tipo, máquinas de combate que hoy bregan por convertirse en
vanguardia real de la clase y el pueblo y dirigir la guerra popular para
conquistar el poder. Y la revolución proletaria mundial, en medio de
avances y retrocesos, éxitos y fracasos, se desenvuelve desarrollando la
lucha antiimperialista y por el socialismo rumbo al comunismo.

El siglo XX ha sido el siglo de la revolución proletaria, siglo en el
cual la tercera parte de la humanidad vivió bajo el socialismo y la
democracia popular alzando la bandera roja con la hoz del campesino y el
martillo de la clase obrera. Fue un gran campo socialista que auguraba
el futuro paraíso en la Tierra.Mas el revisionismo y la reacción mundial
lo socavaron y destruyeron, restaurando el poder del capital
monopolista.

Estados Unidos primero, ejerciendo de superpotencia hegemónica única a
fines de esa centuria, y luego Estados Unidos y China, disputando en
bipolaridad el dominio mundial en este siglo XXI, encabezan el sistema
imperialista que mantiene al proletariado y a los pueblos en explotación
económica, opresión política, control ideológico. Pero las masas no
cesan de luchar, jamás lo harán. Otra vez se organizan, sacan lección de
los fracasos y se preparan para volver a tomar los cielos por asalto.

Una muestra son las últimas movilizaciones masivas en Europa contra el
descenso de los salarios, el desempleo, el deterioro en salud pública y
educación, así como contra el recorte de pensiones. Mientras en América
latina se registran protestas masivas que, en síntesis, se oponen a la
explotación y opresión del capitalismo neoliberal.

En nuestra patria, las clases dominantes, tras el fin de la heroica
guerra popular y la imposición de 40 años de neoliberalismo, piensan que
su dominio les permite matar, golpear, perseguir y apresar trabajadores
impunemente, cambiar presidentes cuando les conviene, terruquear sin
medida a su antojo, tener presos políticos eternamente y desaparecer sus
restos, impedir referéndum por asamblea constituyente, mantener fiscales
y jueces colocados a su medida, controlar con dinero todos los medios de
comunicación; en fin, mantener su Estado explotador, policiaco y
corrupto por siempre, acallando con plomo la protesta popular.

Nuestro pueblo combatiente no lo está permitiendo. Hartos de explotación
y opresión capitalista, la clase obrera, los campesinos, el pueblo
trabajador han expresado con voz de trueno y andar de gigante el rechazo
al neoliberalismo y su redoblada explotación, exigen sanción a los
asesinos del pueblo empezando por Boluarte, su gobierno y parlamento
golpista y demandan una nueva constitución con asamblea constituyente
que permita democratizar la sociedad, acabar con la persecución política
y ejercer real soberanía sobre los recursos nacionales.

Son ya más de 40 años de aplicación del derecho penal del enemigo con
una legislación antiterrorista usada como arma de guerra contra el
pueblo. Un caso emblemático de esta es la condena de cadena perpetua que
niega el primer derecho constitucional, el derecho a la vida y que en el
Perú desde los años 90 se ha aplicado a miles de personas. Como ejemplo
podemos citar que en México, Brasil, Venezuela, España no existe la pena
de cadena perpetua. En Francia desde los años 90 solo se ha aplicado a
cuatro personas. En otros países donde existe como Japón o Alemania se
revisa a los diez o veinte años.

Pero es en el Perú donde la cadena perpetua se ha aplicado con más saña
contra los presos políticos de la guerra popular por rebelarse contra la
explotación y opresión estatal. Cadena perpetua que llegó a su punto
culminante con la prisión, tortura, aislamiento absoluto, asesinato,
incineración y desaparición de las cenizas de Abimael Guzmán Reinoso,
nuestro por siempre querido y respetado Presidente Gonzalo, el más
grande revolucionario de la historia peruana. Crimen hasta hoy impune
que pretenden seguir aplicando contra su esposa Elena Yparraguirre,
camarada Míriam, y otros dirigentes comunistas.

En particular, a los camaradas Osmán Morote de 78 años y Margot Liendo
de 74 años, que ya cumplen 35 años de prisión efectiva, según ley, se
aplica la revisión de su pena. A ellos les corresponde la libertad
inmediata, pues negársela significa una pena de muerte encubierta para
luego incinerarlos y desaparecer sus cenizas, siniestros planes que la
reacción peruana ha maquinado para que nunca salgan libres los presos
políticos de la guerra popular. La libertad de Osmán y Margot es su
derecho.

La lucha política por una asamblea constituyente con el pueblo y para el
pueblo requiere el fin de toda persecución y la libertad de los presos
políticos, empezando por la derogatoria de toda legislación que impida o
restrinja la participación popular. A una nueva constitución
corresponderá estampar a plenitud el derecho político por excelencia de
rebelarse contra la opresión; el derecho a la vida anulando la cadena
perpetua y garantizando la existencia digna de las masas populares; la
libertad de expresión, reunión, asociación, participación política
acabando con la legislación antiterrorista persecutoria del pueblo;
defensa real de nuestros recursos naturales contra la penetración
imperialista; así como otros derechos y libertades que han sido negados
o recortados bajo la imposición del neoliberalismo.

La actual lucha del pueblo peruano sigue desenvolviéndose y aislando más
al gobierno golpista. Aplicando diversas formas de lucha aprendidas a lo
largo de su historia, la clase obrera y el pueblo se va educando en la
comprensión de que solo acabando con el capitalismo puede alcanzar su
emancipación. Una nueva constitución dentro del marco burgués puede
lograr mejores condiciones para las masas trabajadoras, pero solo una
revolución, la revolución socialista dirigida por el proletariado, puede
lograr la transformación total que el Perú requiere.

\textbf{¡VIVA EL PRIMERO DE MAYO, DÍA DEL PROLETARIADO INTERNACIONAL!}

\textbf{¡GLORIA AL MARXISMO-LENINISMO-MAOÍSMO!}

¡VIVA LA LUCHA DE LA CLASE OBRERA CONTRA EL CAPITALISMO EN EL MUNDO!

¡EL PRESIDENTE GONZALO VIVE EN LAS LUCHAS DEL PROLETARIADO Y EL PUEBLO!

¡LIBERTAD A LOS PRESOS POLÍTICOS DE AYER Y HOY!

¡ABAJO EL CAPITALISMO, HACIA LA REVOLUCIÓN SOCIALISTA!

Mayo de 2023

Comité Central Partido Comunista del Perú

Ediciones Bandera Roja


\printbibliography


\end{document}
