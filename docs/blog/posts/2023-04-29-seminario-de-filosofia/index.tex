% Options for packages loaded elsewhere
\PassOptionsToPackage{unicode}{hyperref}
\PassOptionsToPackage{hyphens}{url}
\PassOptionsToPackage{dvipsnames,svgnames,x11names}{xcolor}
%
\documentclass[
  a4paper,
]{article}

\usepackage{amsmath,amssymb}
\usepackage{lmodern}
\usepackage{iftex}
\ifPDFTeX
  \usepackage[T1]{fontenc}
  \usepackage[utf8]{inputenc}
  \usepackage{textcomp} % provide euro and other symbols
\else % if luatex or xetex
  \usepackage{unicode-math}
  \defaultfontfeatures{Scale=MatchLowercase}
  \defaultfontfeatures[\rmfamily]{Ligatures=TeX,Scale=1}
\fi
% Use upquote if available, for straight quotes in verbatim environments
\IfFileExists{upquote.sty}{\usepackage{upquote}}{}
\IfFileExists{microtype.sty}{% use microtype if available
  \usepackage[]{microtype}
  \UseMicrotypeSet[protrusion]{basicmath} % disable protrusion for tt fonts
}{}
\makeatletter
\@ifundefined{KOMAClassName}{% if non-KOMA class
  \IfFileExists{parskip.sty}{%
    \usepackage{parskip}
  }{% else
    \setlength{\parindent}{0pt}
    \setlength{\parskip}{6pt plus 2pt minus 1pt}}
}{% if KOMA class
  \KOMAoptions{parskip=half}}
\makeatother
\usepackage{xcolor}
\usepackage[lmargin=2.54cm,rmargin=2.54cm,tmargin=2.54cm,bmargin=2.54cm]{geometry}
\setlength{\emergencystretch}{3em} % prevent overfull lines
\setcounter{secnumdepth}{3}
% Make \paragraph and \subparagraph free-standing
\ifx\paragraph\undefined\else
  \let\oldparagraph\paragraph
  \renewcommand{\paragraph}[1]{\oldparagraph{#1}\mbox{}}
\fi
\ifx\subparagraph\undefined\else
  \let\oldsubparagraph\subparagraph
  \renewcommand{\subparagraph}[1]{\oldsubparagraph{#1}\mbox{}}
\fi


\providecommand{\tightlist}{%
  \setlength{\itemsep}{0pt}\setlength{\parskip}{0pt}}\usepackage{longtable,booktabs,array}
\usepackage{calc} % for calculating minipage widths
% Correct order of tables after \paragraph or \subparagraph
\usepackage{etoolbox}
\makeatletter
\patchcmd\longtable{\par}{\if@noskipsec\mbox{}\fi\par}{}{}
\makeatother
% Allow footnotes in longtable head/foot
\IfFileExists{footnotehyper.sty}{\usepackage{footnotehyper}}{\usepackage{footnote}}
\makesavenoteenv{longtable}
\usepackage{graphicx}
\makeatletter
\def\maxwidth{\ifdim\Gin@nat@width>\linewidth\linewidth\else\Gin@nat@width\fi}
\def\maxheight{\ifdim\Gin@nat@height>\textheight\textheight\else\Gin@nat@height\fi}
\makeatother
% Scale images if necessary, so that they will not overflow the page
% margins by default, and it is still possible to overwrite the defaults
% using explicit options in \includegraphics[width, height, ...]{}
\setkeys{Gin}{width=\maxwidth,height=\maxheight,keepaspectratio}
% Set default figure placement to htbp
\makeatletter
\def\fps@figure{htbp}
\makeatother

\makeatletter
\makeatother
\makeatletter
\makeatother
\makeatletter
\@ifpackageloaded{caption}{}{\usepackage{caption}}
\AtBeginDocument{%
\ifdefined\contentsname
  \renewcommand*\contentsname{Table of contents}
\else
  \newcommand\contentsname{Table of contents}
\fi
\ifdefined\listfigurename
  \renewcommand*\listfigurename{List of Figures}
\else
  \newcommand\listfigurename{List of Figures}
\fi
\ifdefined\listtablename
  \renewcommand*\listtablename{List of Tables}
\else
  \newcommand\listtablename{List of Tables}
\fi
\ifdefined\figurename
  \renewcommand*\figurename{Figure}
\else
  \newcommand\figurename{Figure}
\fi
\ifdefined\tablename
  \renewcommand*\tablename{Table}
\else
  \newcommand\tablename{Table}
\fi
}
\@ifpackageloaded{float}{}{\usepackage{float}}
\floatstyle{ruled}
\@ifundefined{c@chapter}{\newfloat{codelisting}{h}{lop}}{\newfloat{codelisting}{h}{lop}[chapter]}
\floatname{codelisting}{Listing}
\newcommand*\listoflistings{\listof{codelisting}{List of Listings}}
\makeatother
\makeatletter
\@ifpackageloaded{caption}{}{\usepackage{caption}}
\@ifpackageloaded{subcaption}{}{\usepackage{subcaption}}
\makeatother
\makeatletter
\@ifpackageloaded{tcolorbox}{}{\usepackage[many]{tcolorbox}}
\makeatother
\makeatletter
\@ifundefined{shadecolor}{\definecolor{shadecolor}{rgb}{.97, .97, .97}}
\makeatother
\makeatletter
\makeatother
\ifLuaTeX
  \usepackage{selnolig}  % disable illegal ligatures
\fi
\usepackage[]{biblatex}
\addbibresource{../../../../references.bib}
\IfFileExists{bookmark.sty}{\usepackage{bookmark}}{\usepackage{hyperref}}
\IfFileExists{xurl.sty}{\usepackage{xurl}}{} % add URL line breaks if available
\urlstyle{same} % disable monospaced font for URLs
\hypersetup{
  pdftitle={Seminario de filosofía del presidente Gonzalo},
  pdfauthor={Comité Central Partido Comunista del Perú},
  colorlinks=true,
  linkcolor={blue},
  filecolor={Maroon},
  citecolor={Blue},
  urlcolor={Blue},
  pdfcreator={LaTeX via pandoc}}

\title{Seminario de filosofía del presidente Gonzalo}
\usepackage{etoolbox}
\makeatletter
\providecommand{\subtitle}[1]{% add subtitle to \maketitle
  \apptocmd{\@title}{\par {\large #1 \par}}{}{}
}
\makeatother
\subtitle{Notas 1987.}
\author{Comité Central Partido Comunista del Perú}
\date{4/29/23}

\begin{document}
\maketitle
\ifdefined\Shaded\renewenvironment{Shaded}{\begin{tcolorbox}[boxrule=0pt, frame hidden, borderline west={3pt}{0pt}{shadecolor}, enhanced, breakable, sharp corners, interior hidden]}{\end{tcolorbox}}\fi

\renewcommand*\contentsname{Contenidos}
{
\hypersetup{linkcolor=}
\setcounter{tocdepth}{3}
\tableofcontents
}
\listoffigures
\listoftables
\hypertarget{seminario-de-filosofuxeda-del-presidente-gonzalo-notas-1987}{%
\section{SEMINARIO DE FILOSOFÍA DEL PRESIDENTE GONZALO (NOTAS
1987)}\label{seminario-de-filosofuxeda-del-presidente-gonzalo-notas-1987}}

Textos de referencia:

\begin{itemize}
\tightlist
\item
  Introducción a la dialéctica F. Engels.
\item
  La familia, la propiedad privada y el Estado F. Engels.
\item
  La transformación del mono en hombre a través del trabajo F. Engels.
\item
  Carlos Marx de Lenin Obras escogidas Tomo II.
\end{itemize}

Muchos han sostenido que lo que forma la mente del hombre son las
matemáticas, ya no se puede pensar así. La lógica otros. Ni la
matemática ni la lógica son sistemas que forman la mente del hombre. Es
la filosofía, proceso del conocimiento a través de distintas etapas y
modos de producción.

Ocupándose de las leyes que rigen el desarrollo del hombre Lenin llegó a
establecer que la filosofía era una necesidad eminentemente política.
``el meollo de la ideología es la filosofía''. Lenin se abocó a estudiar
todo el proceso de la filosofía desde el punto de vista marxista.
Estudió la ciencia de la lógica de Hegel.

\begin{itemize}
\tightlist
\item
  ``Cuadernos filosóficos'' Lenin.
\item
  IV tomo ``Acerca de la práctica'' y ``sobre la contradicción''. Pte.
  Mao.
\end{itemize}

Sin filosofía no hay partido.

\textbf{Proceso de la filosofía:} desechar el criterio de que la
filosofía solo se va a dar a partir del mundo griego. Los estudios
posteriores demuestran que eso es un prejuicio, desprecio por el
pensamiento de otros pueblos. Proceso en China, la India. Conforme la
civilización avanza, los pueblos se esfuerzan por conocer el fondo de
las cosas, el porqué de las cosas, Egipto, Mesopotamia, pueblo Hebreo,
donde hay un proceso de desarrollo, se sigue planteando que es una
pre-filosofía, se niega el proceso de desarrollo desde los tiempos más
iniciales. Las propias religiones: los egipcios plantean que las aguas
son el principio primordial, un símbolo de vida, pero no saben de dónde
viene el Nilo, el Nilo al expandirse deja algunas islas y en ellas se
desarrolla el espíritu. Plantean dos cuestiones: Espíritu y Materia. Lo
importante es que siempre planteaban un principio que es materia.

Los griegos son los que nos plantean una filosofía más desarrollada,
ligada al proceso de mercado, la aparición de la moneda y ligada a la
ciencia. Tales, predice el primer eclipse. Los egipcios sabían
cuestiones matemáticas por práctica, son los griegos los que explican y
demuestran los hechos. Avance en el conocimiento científico y la lucha
de clases de los esclavistas, agudización de la lucha entre comerciantes
y los agricultores, ``democracia griega que tiene un proceso dictatorial
antes de la democracia. Se nos pretende hacer ver (que) la filosofía
está desarrollada al margen de las clases, siglos VII y VI a.c.

\textbf{Escuela materialista.} Arge: origen, el porqué de las cosas es
el origen: el comienzo de las cosas son las aguas, es la ley que todo
deriva. Caos original y orden en las cosas. Esto ya lo habían dichlos
egipcios. Hizo indagaciones y encuentra en las islas conchitas
(fósiles). Otro pensador va a decir que el origen es el aire, siempre un
origen material. Heráclito: plantea que el origen de las cosas es el
fuego, previa: realidad material son pues materialistas. La guerra es el
origen de todas las cosas, la lucha de dos contrarios y de esa pugna
tenemos un proceso constante de desarrollo, todo es un permanente
discurrir, nadie se baña dos veces en la mismas aguas. Aquí tenemos la
dialéctica. Intuiciones geniales. De ellos solo nos han quedado frases,
nada más. Aristóteles historia todo esto. Intuiciones geniales no
fundamentales. La contradicción de la filosofía es contra la religión.
Se desgaja de la religión. Aparece el idealismo. Parménides niega la
dialéctica, surge como contraposición a Heráclito: tiene dos cabezas una
afirma otra niega, no razona. Plantea que el origen de todas las cosas
es el ser: es el ser absoluto, lo abarca todo, las cosas existen porque
participan del ser.

El ser no tiene movimiento, si se desplazase sería el no ser. Los
hombres en esa época no podían refutarle.

El materialismo parte de la materia previa y de un proceso del
conocimiento. Lo primero son los materialistas, los idealistas son
posteriores.

\textbf{Demócrito:} gran materialista. Teoría de los átomos: lo que no
puede partirse. Una mínima instancia material. Todo lo que existe son
pequeñas partículas que no pueden partirse, eternas y en continuo
movimiento. Así refuta las teorías de los idealistas de Parménides de la
divisibilidad infinita que llevaría a lo no existencia. No es hasta 1900
que se refuta la indivisibilidad del átomo.

El conocimiento es un reflejo de los átomos en la cabeza. Los efluvios
se entrecruzan y esto se refleja en nuestra cabeza de ahí tenemos el
error. Plantea que el hombre se desenvuelve socialmente. Parte
integrante de las Polis. Refleja lo que se ve en su propia ciudad. La
esclavitud es nociva porque rebaja al hombre, porque envilece al ser
humano, no le permite dar lo mejor que tiene de sí, es la libertad lo
que corresponde. El hombre debe ser libre, entra al campo de la moral,
sabría que le permitiría vivir libremente. El más grande exponente del
materialismo en la antigüedad.

El materialismo siempre se ha desarrollado con una comprensión y un
respeto al hombre. Su pensamiento era nocivo para la sociedad y los
criterios de la clase dominante, todos los criterios idealistas están
ligados a los comerciantes y los esclavistas. Los sofistas plantean que
el hombre puede ser educado y así elevarse. El hombre es la medida de
todas las cosas. En Sócrates se ve como los griegos eran sumamente
sociales, el individualismo no estaba desarrollado.

\textbf{Platón:} ligado a la aristocracia, muy acaudalado, sistematiza
todo el pensamiento idealista. Sostiene que hay una apariencia y una
realidad, que los sentidos son engañosos, que la apariencia es idea y el
mundo es materia: La realidad de las cosas participan de las ideas.
Plantea una trinidad de las ideas bien, belleza, verdad y estas tres
están sustentadas por el ser. Teoría de la coparticipación de las ideas.
Plantea el comunismo platónico que tiene un antecedente en Egipto que es
un comunismo reaccionario. El entiende que la propiedad genera luchas.
Para él el ordenamiento democrático era nocivo, pensaba en un gobierno
de élites. La educación para él era nociva. Lo entiende porque la
aristocracia estaba siendo atacada, destruida por los comerciantes.
Sociedad: conjunto de trabajadores conforme a los que se van educando se
van seleccionando, trabajadores, guerreros, etc. Y queda un conjunto de
élites (fascismo) la música porque corrompen. Destruyó todo lo que llegó
a su alcance de Demócrito.

\textbf{Aristóteles:} discípulo de Platón. Nos ha informado de todo lo
que pensaban los materialistas, critica a Platón y va a basarse mucho en
los conocimientos científicos y sociales de la época. Aristóteles se
basaba en el conocimiento científico, critica a Platón:

En las cosas existen pero tiene una realidad material y una forma, si no
tuvieran la forma entonces se confundirían. Existen las cosas porque
tienen una materialidad y una forma. Llega al idealismo desde una base
real, ha metido la idea en la realidad. Comienza a manejar los conceptos
y las formas, esencia: una sustancia y una esencia, hay una realidad
primaria, una esencia superior que imprime el movimiento, porque existe
un motor primer móvil, dios, la palabra que se conoce a sí misma. Llega
al idealismo a partir de un nudo de arterias. Las cosas existen
realmente no se pueden negar pero llega al pensamiento que se piensa a
sí mismo, y ese pensarse a sí mismo es lo que ha puesto en movimiento la
realidad.

Como realidad concreta la materia no tiene movimiento, es la idea lo que
se mueve, primen móvile (primer motor). Dialéctica conceptual. Lo
positivo que tiene es que existe la materia. Es otra forma de
platonismo.

\textbf{Escuelas:} Los romanos nunca lo pudieron superar, el
neoplatonismo como decadencia que llega a la mística (Plotino). La
iglesia no puede afiliarse con el platonismo.

\textbf{Medievo.} Se comienza a desenvolver la filosofía como
reivindicación de la Razón. Aparte de los árabes, es a través de ellos
que se empieza a conocer la filosofía griega y se comienza a conocer el
aristotelismo. Los árabes llegan a desarrollar un criterio materialista,
llegan a diferenciar la filosofía de la teología. Filosofía se ocupa de
la tierra y la teología del cielo. Los árabes y los hebreos son los que
tienen influencias.

\textbf{Realistas y nominalistas.} Los realistas aplican las tesis
aristotélicas, realidad de las cosas y las ideas también existen
independientemente. Los nominalistas no son sino bocas vacías, sin
contenido real, son derivaciones extraídas de las cosas. Quienes se
enfrentan son las ideas religiosas.

\textbf{Pedro Abelardo:} comienza a manejar la lógica formal, creador de
la lógica deductiva. Maneja la lógica de forma dialéctica (debate,
discusión). Tiene mucha importancia para el pensamiento francés. Ataca a
la religión. Marx considera que el nominalismo tiene mucha importancia.

\textbf{Duns Scotto} tiene mucha importancia, era franciscano. La raíz
del materialismo moderno está en este personaje: ¿cómo combatir la
religión? La comunión.

¿Cuántas veces y cuántos hombres comulgan? Entonces ya no quedaría
cuerpo de Cristo. Todos los que se oponían eran muertos, época muy
violenta, muy dura. Se nos quiere presentar a los filósofos como hombres
de pupitre, la realidad no ha sido así, la puñalada y el veneno ha sido
la forma de debatir en filosofía.

\textbf{Tomás de Aquino:} Tomismo, (agustinismo --neoplatonismo-),
italiano que entra a dominico. Discípulo de Alberto Magno: plantea que
racionalmente se puede llegar a comprender la religión católica. La
razón no se contrapone a la teología. Se basa deformando a Aristóteles,
no es un desarrollo de Aristóteles, es mucho más bajo. Su obra más
importante es el Ente y La Razón (Bertrand Russell) en vida fue
perseguido por la iglesia y es el fundamento. (OCCAMO lo que hizo junto
con Scotto fue rebatir el tomismo).

El proceso filosófico se empieza a desenvolver con la burguesía
(Francisco Bacon) reivindica la experiencia (``nuevo órgano''). Lo que
hace es desarrollar una lógica inductiva que va a servir a la ciencia.
Plantea que su pensamiento abarca el pensamiento de los hombres
(reconoce a la teología pero aparte).

\textbf{Descartes (1596-1650)} fue discípulo de los jesuitas. Comprendió
que lo que en un pueblo se afirmaba en otro se negaba, que la ciencia no
tenía sólidos fundamentos (coordenadas cartesianas que permiten llevar
la geometría al análisis algebraico). Era estudioso de la física del
mundo, de la materia, lo que retoma es el pensamiento de Demócrito. El
es materialista en ese campo. Plantea la duda metódica (no es el
escepticismo que cuestiona el conocimiento, no confía en el
conocimiento) hay que dudar para llegar a un conocimiento evidente,
plantea el engaño de la vista. Los sentidos engañan, no se puede creer
en los sentidos, pero hay algo que es evidente YO. No puedo dudar que yo
existo, hay una verdad indubitable. Yo dudo luego existo. Cualquiera que
sea lo que pueda presentar a la realidad, existe algo innegable. Yo
pienso, luego existo. Verdad evidente ante cuya existencia no hay duda.
Existo yo y mis pensamientos. La realidad la va viendo a través de sus
pensamientos.

Tengo las ideas, es porque existe dios y es quien ha dado todo. Todo
existe porque existe dios. Cuando desarrolla la ciencia es materialista
pero cuando desarrolla ideas metafísico, pone como pivote de la
filosofía el YO, de aquí para adelante empieza a fundarse el pensamiento
burgués.

Escuela materialista contraria que toma a Demócrito.

Filosofía Alemana: Leibniz, Kant y Hegel.

Siglo XVII- XIX (1830). 150 años más o menos.

El luteranismo: limpia los establos de la iglesia.

Alemania da el más avanzado pensamiento de la escuela idealista.
Leibniz: gran matemático. Desarrolla la lógica, replantea la lógica de
Aristóteles. No difundió sus pensamientos. Desenvuelve un racionalismo.
Es posible un análisis lógico. Lógica con símbolos para manejarla como
análisis matemáticos. Conjunto de axiomas que siguiendo un cálculo se
pueden resolver todas las verdades absolutas. Teoría de las mónadas:
entidades cerradas. Se comunicaban a través de una ventanilla, ideales,
auto-movimiento. Problema de dinámica, pero esto es conceptual porque
son idealistas. Se dedica a analizar el conocimiento humano, liga la
matemática y la física.

\textbf{Kant (1724-1804).} Se centra en el problema del conocimiento.
Crítica de la razón pura. Plantea que existe la realidad pero como
fenómeno, lo que aparece. Lo que la luz muestra. Establece una
diferencia entre los fenómenos. Existen las cosas una parte que aparece
y otra, la cosa en sí, que no aparece. Existe la materia pero que no se
la conoce. Establece relación entre el sujeto que conoce y el objeto
conocido, pero hay una parte que no se conoce. Analizando lascosas
tenemos unas sensaciones que las capto a través de mi sensibilidad.
Mundo, Hombre (alma), Dios Sistema completo del conocimiento Elabora
conceptos

\textbf{Categorías:} sistema lógico de conocimiento. Sólo conozco los
fenómenos escapa a mi conocimiento la cosa en sí, el conocimiento viene
a ser una elaboración de la razón pura (R.P.) elaboración entre el
sujeto y el objeto (cosas) el sujeto es lo más importante. Hay una
realidad que puedo conocer y otra que no puedo conocer. La cosa se deja
conocer.

Posterior a Kant se desarrolla el neokantismo que lo que hace es
disolver la cosa en sí, la cosa en sí es una elaboración de la cosa en
sí. Se pasa de un idealismo a un ultra-idealismo. Kant ha llegado a
conocer por el entendimiento. Crítica de la razón práctica: cuando
analiza el alma llega a pensar en la libertad y ésta sólo se puede
alcanzar en dios. La libertad el alma y el regazo de dios. Ordena la
comprensión del conocimiento y expresa los límites del idealismo (razón)
¿Por qué se plantea que dios existe? Porque para explicar que todo tiene
comienzo y fin se busca la causa y esta causa es dios, pero al plantear
que dios es la causa ¿ cuál es la causa de dios?. El mismo refuta la
existencia de dios.

\textbf{Hegel:} se plantea qué es lo anterior. Lo que pretende Kant es
conocer la realidad a partir de su yo, no centra en lo objetivo. El
problema es partir de lo objetivo, analiza el proceso de la filosofía,
pensaba que todos los filósofos eran anteriores a él. Todos los demás
pueblos para él no existían, no eran nada. Desarrolla una teoría de la
dialéctica que permitía la comprensión de todo el proceso de la materia
(su problema es que era idealista). El proceso se desenvuelve por
contradicción y al desenvolverse genera el problema de cantidad y
calidad, apariencia-realidad. Entendía la dialéctica como proceso de
contradicción entre conceptos, ideas. Va a negar la aplicación de su
propia dialéctica. Plantea que existe una gran idea absoluta. Esa gran
idea es la realidad objetiva, cuyo proceso es contradicción a nivel de
ideas solo. Muy parecido a Aristóteles pero sin partir de la materia.
Esta idea enjuiciaba por el propio proceso de contradicción a la
materia. Siendo el propio espíritu comienza a desenvolverse hasta
generar al hombre y el espíritu se hace autoconciencia, el espíritu se
ha negado. El hombre: sociedad, conocimiento, ciencia, arte, religión,
nación y luego genera Estado. El Estado se convierte en gran
transformación que luego se convierte finalmente en Espíritu, dios. Idea
Absoluta:

Tiene una comprensión de todo el desarrollo materialista pero es
idealista. Dos partes. Su idealismo, desechable y materialismo que es
asumible.

Proceso materialista en Francia Diderot. Materia eterna, no tiene
comienzo ni fin, llega a plantear que hay un auto-movimiento interno que
impulsa la materia, pero no explica porqué. Pero el antecedente de la
filosofía marxista es la filosofía clásica alemana. Muerto Hegel hay una
división, unos comienzan a criticar el idealismo de Hegel, el que
interesa es Feuerbach, critica el idealismo de Hegel pero no diferencia
el materialismo del idealismo de Hegel. Lo que le lleva a desechar a
Hegel. Fenómeno de enajenación ante la religión (la alienación,
enajenación, es una palabra de Hegel) no es una tesis de Marx,
diferencian al Marx joven del amargo. Marx la desecha porque la solución
es la revolución, la emancipación.

Hegel: el trabajo extrae al hombre de su esencia de ser pensante, de ser
nacional.

Marx analiza las causas de la enajenación.

Feuerbach plantea que ante la enajenación el centro es el hombre no
dios. La relación se entiende por el amor, el cáritas (la caridad), de
ver por el otro, maternidad, posición subjetivista, cómo se relaciona un
yo con otro yo. Cristianismo sin Cristo. Lo importante es la crítica
materialista. Marx y Engels llevan a una lucha contra el individualismo
de Feuerbach.

Marx y Engels van a desarrollar el proceso filosófico marxista. Marx
desarrolló y Engels difundió. Las tesis sobre Feuerbach constituyen la
base:

1o defecto de todo materialismo anterior: no haber tenido en cuenta la
práctica. El materialismo anterior se había desarrollado en el empirismo
o ver la realidad como algo pasivo, no se comprende como la materia
actúa y como el hombre a través de su trabajo cambia la realidad
(captación de la realidad). Todo empirismo es una posición burguesa.
Postula: comprender la realidad y transformarla.

2o Práctica y verdad, es en la práctica como prueba de verdad. Marx
critica y Feuerbach, nunca llegó a concebir la captación sensorial como
capacidad transformadora. Este había diluido la esencia religiosa en la
esencia humana, un cristianismo sin Cristo, la incapacidad de comprender
el mundo social. Relaciones sociales.

3o la vida social es esencialmente práctica. La mente humana esta
descarriada por un conjunto de misticismos. Sólo comprendiendo la
práctica se puede barrer con el misticismo. Como no comprenden la
práctica lo llama materialismo contemplativo. Sociedad civil: a lo
máximo que llegó a avanzar fue al estudio de las instituciones, cual es
la raíz que la sustenta. Transformar el mundo: los filósofos no han
hecho más que contemplar el mundo pero el problema es transformarlo.

Con este documento deslinda los campos.

Ajuste de cuentas con sus anteriores pensamientos en una posición nueva.
Se plantean así nuevos criterios para formar la nueva ideología. Se
plantea el proceso económico de la sociedad. Se plantea el comunismo,
cómo la 1a gran revolución en el mundo, pues todas las anteriores fueron
la sustitución de una clase por otra.

Toda la filosofía en su largo recorrido había desarrollado una teoría
sobre la dialéctica, así como sobre el materialismo. Critican con justa
razón el Medievo. Una disputa que quería resolver las cuestiones sin ver
a la realidad. Ellos vieron bien los hitos del desarrollo. Afirman su
posición rotunda materialista. El acceder al materialismo demanda como
un proceso en movimiento derivado de la contradicción.

Althusser niega que Marx y Engels hayan tomado la dialéctica de Hegel.
Plantea que primero se desarrolla la ciencia y luego se produce el
salto.El descubrimiento de Marx y Engels es el materialismo histórico
porque se funda la teoría materialista de la historia y después el
materialismo dialéctico. Según él estaba pendiente el desarrollo de la
filosofía marxista. Es una estupidez de principio a fin.

Platón y Kant son idealistas. Niega el proceso científico que se
desarrolla desde el siglo XVII. Desde finales del siglo XVI se pensaba
que la tierra era algo que cambia, una forma de movimiento. Proceso
dialéctico. Química: no hay una muralla china entre la química orgánica
y la inorgánica. Biología: se descubre la célula, en los animales se ven
formas transaccionales: como eslabones. Teoría de la evolución. Así la
ciencia rompe con la metafísica como procesos, desarrollos. Esto no lo
puede negar Althusser. Así la ciencia demandaba una explicación
dialéctica. Hegel había puesto el proceso dialéctico en la cabeza. Marx
lo que hace es ponerlo en la materia. Antes nunca fue hecho esto. El
materialismo dialéctico es capaz de entrar en el conocimiento y la
transformación al actuar el hombre en la materia. Se cuestiona el
carácter científico del marxismo, la materia se transforma derivado de
la práctica.

La ideología que han generado las clases explotadoras es invertida
porque da una explicación idealista de la historia. Nuestra ideología es
científica porque es un reflejo real verdadero de su práctica y su
carácter de clase. Las teorías de Althusser llevan a un nuevo
surrealismo y lo que cabe es fundir la teoría kantiana y la de Spinoza.
Se toma un racionalismo burgués y un idealismo burgués. Este proceso
tiene un trayecto de 2500 años, tienen un sólido fundamento histórico en
lo que se ha recogido lo mejor y resulta en el
Marxismo-Leninismo-Maoísmo. La aplicación del materialismo dialéctico da
pie al materialismo histórico y a la comprensión científica de la
sociedad.

Ha habido un proceso para demostrar los fundamentos económicos de la
sociedad. ``El marxismo lo que hace es criticar económicamente a la
sociedad'' dicen los que lo atacan. De la base económica y la sociedad
generan la ideología. No se ha dejado el problema de las ideas y la
acción que las sustenta.

Dialéctica: Engels es quien se ocupa de esta cuestión: tres leyes.
Unidad y lucha de la contradicción, el salto y la negación de la
negación. Comprendieron que la 1a era la principal. Si no hubieran
entendido la dialéctica no hubieran llegado a poder desarrollar EL
CAPITAL. Noes un círculo, el marxismo es un proceso dialéctico que
seguirá desarrollándose. Nos deslinda con todos los procesos filosóficos
que son cerrados.

Hegel es inconsecuentemente dialéctico y nosotros somos consecuentemente
dialécticos. Esto es la más grande revolución había en la historia de la
humanidad. Filosofía marxista que sienta las bases del desarrollo, nunca
podrá agotarse el conocimiento, es un proceso que va más y más
acercándose a la verdad y desechando nuevos errores. Negaciones del
marxismo: este fenómeno ha sido constante. En materialismo y
Empiriocriticismo, Lenin enarbola y defiende al marxismo y lo
desarrolla. Teoría del reflejo. Conjunto de reflejos que generan la
conciencia. El reflejo es una característica que es una característica
de la materia, acción y reacción. La conciencia viene a ser un largo
proceso de la característica de la materia. Los átomos. En el año 1900
un físico alemán plantea que existe una cantidad pequeñísima de materia
necesaria para dar un salto, teoría de los cuantos, con esto se abre la
teoría nuclear. Einstein lo que hace es una nueva teoría del espacio
tiempo.

Newton: Hay dos absolutos como entidades inseparables, lo que plantea es
que espacio y tiempo son relativos. Los experimentos demostraban que al
producirse altas velocidades se producía una reducción. El problema es
que el tiempo y el espacio varían según la velocidad, dos absolutos se
convierten en dos relativos. Gravedad dos materias que se mueven en
sentidos más amplios. Física cuántica que rompen el átomo, niegan la
materia, Lenin dice que estamos empezando a conocer las 1a partículas.
La materia en movimiento tiene forma cuantitativa y cualitativa, estamos
viendo nuevas formas de la materia porque la materia es eterno
movimiento. Lenin rechaza que la materia se disolvía.

Carácter partidista de la filosofía y la lucha contra el
Empiriocriticismo. La física cuántica va a dar campo a que se niegue el
materialismo. Si conocemos la velocidad del electrón no conocemos su
ubicación, de esta manera se niega la causalidad: esta tiene dos
acepciones, expresa la correlación entre una causa y un efecto la otra
es el problema de la previsibilidad. Se ha confundido causa efecto con
la previsibilidad, pero la causa efecto sigue existiendo. Al basarse en
la previsibilidad niegan lo anterior. Entonces lo que hemos encontrado
es la casualidad y lo que se ha descubierto es otra forma de la materia.
Nuevas modalidades de la materia, nuevas formas.

Dos líneas paralelas se juntan del lado interiores suman menos de dos
rectos. V. Postulado. Presupuestos. Geometría de las líneas paralelas.
(Triángulo) 180o, durante muchos siglos se consideró que esta era la
única geometría.

Gauss planteó que no tiene demostración, quien cambie este postulado
genera otra geometría. Cuando viene la cónica encuentran que es
inadecuada. Geometría de Reirmer. 180o geometría de latbochesky-bauyeic.
(Dibujo).

Antes hablábamos de un espacio plano otro curvo y otro cóncavo. Así la
materia tiene muchas manifestaciones. Convexo, plano, cóncavo ¿?
(desarrollo del futuro). En vez de cuestionar lo que hacen es confirmar.
La materia es inagotable. Cuantos procesos estarán desarrollándose.
Eternidad de la materia en eterno movimiento (comprender con el menor
problema posible). Hoy en día se concibe a la materia como una
interrupción de la nada. ¿Y qué es la nada? Separa espacio de materia.
Joudan. Una nada es un espacio, y un espacio es una modalidad de la
materia.

\textbf{La cosmogonía:} se descubre que hay unas estrellas que se
desplazan a grandes velocidades: la llamada expansión del universo,
llegan a la concentración puntual del universo. Esto, dicen, demuestra
que ha habido un comienzo por tanto no es eterno, 2o tiene un límite.
Dicen que antes no existía el universo, momento inicial de la creación.
Esto viene de que el entorno que conocemos tiene 15.000.000.000 de años.
Otros llegan a decir que tiene 6.000 millones de años, los hechos
demuestran que de la parte que conocemos ha empezado en esas fechas más
o menos. Lo que se está haciendo es generalizar lo poco que conocemos.
Lo que se afirma de una parte no se puede afirmar del todo. Se pretende
(Russell) introducir por la puerta falsa la divinidad.

El movimiento tiene una faceta cuantitativa y otra cualitativa.

La filosofía burguesa entra en un proceso de clara decadencia. Lukacs
plantea que la contradicción no es materialismo-idealismo, sino
irracionalismo-racionalismo. Que plantea una aguda crisis de la
filosofía burguesa.

Bergson: desarrolla una metafísica llena de lacrimogenia.

Nietzsche: teoría del superhombre, una extraordinaria pluma. Las teorías
que buscan una salida al imperialismo. Teoría moral basada en los
mejores y su dominio. Hombres privilegiados y mentes borreguiles, apunta
contra el cristianismo, intentando restablecer la moral de los señores.
El cristianismo confunde la bondad con la virtud. Son cristianos los más
poderosos, los más fuertes. Esto es puro racismo.

En los años 20 trata de reimpulsarse. Neopositivistas: surge en los
círculos de Viena: el positivismo, respuesta reaccionaria de la
burguesía Comt. Plantea la necesidad de creer en la ciencia positiva,
niega la existencia de leyes en la realidad y plantea que la realidad
son cosas que nosotros elaboramos, el conocimiento. La nueva ciencia es
una religiosidad, el mejor mundo es el burgués y el problema es orden y
progreso.

Neopositivistas: parte de los fenómenos, lleva a un cientificismo. Es el
sujeto el que elabora un sistema de ciencia, derecho, cae en un
desarrollo de la lógica. Elaborar sistemas derivados de la ciencia,
matematicismo.

\textbf{Pitágoras:} planteó que la esencia de las cosas era el número,
que todo podía ser medido, Platón lo desarrolla. Todo el conocimiento lo
reducen a fórmulas. Lo malo esta en sustituir la realidad por las
fórmulas, el hecho es que la matemática sale de la realidad material, el
círculo salió de la rueda; considera la matemática para sustituir la
realidad. (hacer un agujero en la pared con una integral y no con lo que
representa la integral -- un taladro- )-comentario. La lógica: comienzan
a analizar, plantean que el lenguaje es insuficiente y que hay que
sustituirlo por símbolos, para llegar a algo hay que simplificar todo.
Es positivo en el sentido que nos da un desarrollo de la lógica, como
lógica simbólica. Hablan de criterios de verificación, prueba de verdad.
Se termina no analizando la materia, sino los análisis sobre la materia
(análisis lógico).

\textbf{Wittgenstein.} El más consecuente de los neopositivistas. ``yo
no puedo hablar del mundo, se me puede preguntar de cómo yo interpreto
el mundo, el mundo no es cognoscible, lo que se puede hablar es del
conocimiento que se tiene del mundo, no puedo hablar de los otros
sistemas porque no los conozco, lo mejor que puede hacerse es callarse,
no se puede decir nada sobre nada. Se llega a lo inefable. Silencio.
Divinidad, se ha llegado al límite de contemplar''. Dios a la vista.
Científicos y análisis de la ciencia, logicismo. Negación absoluta
del~conocimiento. Russell, Bertrand. Cameades en la antigüedad, Nunme y
Rosses, sus análisis llevan a deshacer el conocimiento.

Todos llegan al agnosticismo. Principia matemática. Creadores de la
lógica moderna. Matematicismo platónico, logicista, planomisticismo
platónico ``Todo lo que he dicho hasta ahora no vale y lo que estoy
diciendo ahora no sé si valdrá''.

Análisis, se quedan en el desmontaje y no llegan a montarlo, no hacen la
síntesis. Sin embargo van descubriendo paradojas que permiten avanzar,
cuando pensamos estamos pensando en términos finitos y han ido limpiando
la filosofía y la ciencia. El conocimiento ha entrado a un momento
crítico, se produce un momento de síntesis, y nuevamente empezaron a
expandirse. Demolición de los conceptos de la ciencia, todo ha entrado
en crisis. El proletariado va a establecer esos nuevos principios. No ha
acabado el proceso de demolición. Hay una clase que está muriendo y sus
principios mueren con ella. El desconcierto es una decantación.

\textbf{Existencialismo:} Heidegger. 1920. Análisis de la existencia, el
creador Dios. La filosofía se debe centrar en la existencia de las
cosas. El hombre es la expresión de la existencia, viene de la nada y
marcha a la nada. No sabe nada de su existencia, de donde viene. En este
tránsito se angustia, cuando esto ocurre caben dos actitudes: enfrentar
o huir de esa angustia. El problema es enfrentar su angustia, enfrentar
su muerte, ser para la muerte, esa es la identidad del hombre, vivir
para la muerte. Sirvió al nazismo, es expresión de una clase que está
agonizando. Expresión de la decadencia filosófica.

\textbf{Sartre:} es de la misma escuela. El hombre es un ser sin
existencia y busca la existencia y busca aferrarse a algo para expresar
su existencia, el hombre reduce todo a la nada, busca aferrarse a las
cosas, pero eso es una salida en falso, en otro ser humano, el uno al
otro se convierte en nada (Auhilar). Otra salida es el amor, pero es la
misma situación, entonces queda dios, pero dios no existe. Entonces
queda su propia libertad, esta es la solución. Solo le queda la
alternativa de vivir o morir. Pesimismo, sin salida la libertad es una
correlación que se da en la sociedad.

\textbf{Marcel:} el hombre viene de dios y va a dios, el problema
entonces es llegar a dios. Todas son expresiones de la clase que ya no
tiene salida. Neotomismo: Maritain. La iglesia se mantiene en el
tomismo. Los pensadores católicos han pensado ajustar el tomismo
teniendo en cuenta el desarrollo de la ciencia en la filosofía. El hecho
de querer tomar una concepción feudal muestra la pobreza ideológica que
tiene la iglesia. Nace muerto pues es una filosofía que ya está muerta.
Sucesores Husserl: aplicación de Deconte. Fenomenología. Trata de
superar los errores de Deconte. García Baca. García Morente, de la
escuela del neotomismo. Presidente Mao decía que uno no puede vacunarse
contra el idealismo si no se le conoce.

\textbf{Marxismo:} la ley principal: Plejanov planteaba que el Marxismo
planteaba el monismo. El materialismo es la base, la directriz es la
dialéctica y de ésta lo principal es la contradicción. Marx --Engels no
llegan a plantear cual es lo medular. Con el C. Stalin se produce una
regresión. El Presidente Mao plantea que la única ley es la
contradicción y las otras son derivaciones. Con el Presidente Mao se
llega al monismo filosófico; la única ley. Esto no implica que el
sistema se haya concluido. Cuestiones referentes a la libertad, por un
lado es conciencia de la necesidad y el otro aspecto es transformación
de la necesidad y este es el principal. Dialéctica: las leyes más
generales del desarrollo del mundo natural, del mundo social y del
conocimiento, entendiendo por tal el reflejo de la realidad material en
la mente del hombre. La dificultad estaría en las leyes. Es el
Presidente Mao el que plantea una única ley, considerando la ley de la
contradicción como ley única .

\textbf{El individualismo.} Plejanov: plantea el monismo, si bien parte
de las leyes y las clases también tiene en cuenta el individuo pues
puede perturbarla. Asumir la ley y llevarla adelante, de la forma más
pura y cumplir el papel que la revolución demanda. Hay particularidades
pero lo principal es que asumen la ley y la llevan adelante.
Marxismo-Leninismo-Maoísmo combatir el individualismo y su raíz el
egoísmo, combatir el yo por delante. El individuo históricamente se va
desenvolviendo, la propiedad privada potenció la individualidad y el
egoísmo, la burguesía potencia al máximo el individualismo hasta el
exceso. El marxismo centrando en la clase rechaza el individualismo, el
egoísmo, en el P. es donde nos imprime una nueva forma de ser, nos va
modelando. La acción misma en la lucha de clases es la principal, el
trabajar colectivamente nos va diluyendo la formación que traemos.

Al hacer la revolución se transforma el mundo y también a los hombres.
La raíz es el egoísmo y es una base del revisionismo y requiere tiempo.
Desarraigar el individualismo va a ser un proceso largo. Al ir generando
nuevas relaciones de producción más desarrolladas va a reflejarse más y
más en la idea en toda la sociedad.

Los comunistas debemos ser trompetas que anuncien el futuro. La
ideología permite que desarrollemos y avancemos en la lucha contra el
egoísmo. Debemos ser los más avanzados. Trabajamos por una meta que no
vamos a ver. Reducir cada vez más el individualismo y el egoísmo. Es en
las luchas donde la acción golpea más el individualismo. La ideología es
lo que permite avanzar.

Lima, marzo-abril 1987


\printbibliography


\end{document}
