% Options for packages loaded elsewhere
\PassOptionsToPackage{unicode}{hyperref}
\PassOptionsToPackage{hyphens}{url}
\PassOptionsToPackage{dvipsnames,svgnames,x11names}{xcolor}
%
\documentclass[
  a4paper,
]{article}

\usepackage{amsmath,amssymb}
\usepackage{iftex}
\ifPDFTeX
  \usepackage[T1]{fontenc}
  \usepackage[utf8]{inputenc}
  \usepackage{textcomp} % provide euro and other symbols
\else % if luatex or xetex
  \usepackage{unicode-math}
  \defaultfontfeatures{Scale=MatchLowercase}
  \defaultfontfeatures[\rmfamily]{Ligatures=TeX,Scale=1}
\fi
\usepackage{lmodern}
\ifPDFTeX\else  
    % xetex/luatex font selection
\fi
% Use upquote if available, for straight quotes in verbatim environments
\IfFileExists{upquote.sty}{\usepackage{upquote}}{}
\IfFileExists{microtype.sty}{% use microtype if available
  \usepackage[]{microtype}
  \UseMicrotypeSet[protrusion]{basicmath} % disable protrusion for tt fonts
}{}
\makeatletter
\@ifundefined{KOMAClassName}{% if non-KOMA class
  \IfFileExists{parskip.sty}{%
    \usepackage{parskip}
  }{% else
    \setlength{\parindent}{0pt}
    \setlength{\parskip}{6pt plus 2pt minus 1pt}}
}{% if KOMA class
  \KOMAoptions{parskip=half}}
\makeatother
\usepackage{xcolor}
\usepackage[top=2.54cm,right=2.54cm,bottom=2.54cm,left=2.54cm]{geometry}
\setlength{\emergencystretch}{3em} % prevent overfull lines
\setcounter{secnumdepth}{-\maxdimen} % remove section numbering
% Make \paragraph and \subparagraph free-standing
\ifx\paragraph\undefined\else
  \let\oldparagraph\paragraph
  \renewcommand{\paragraph}[1]{\oldparagraph{#1}\mbox{}}
\fi
\ifx\subparagraph\undefined\else
  \let\oldsubparagraph\subparagraph
  \renewcommand{\subparagraph}[1]{\oldsubparagraph{#1}\mbox{}}
\fi


\providecommand{\tightlist}{%
  \setlength{\itemsep}{0pt}\setlength{\parskip}{0pt}}\usepackage{longtable,booktabs,array}
\usepackage{calc} % for calculating minipage widths
% Correct order of tables after \paragraph or \subparagraph
\usepackage{etoolbox}
\makeatletter
\patchcmd\longtable{\par}{\if@noskipsec\mbox{}\fi\par}{}{}
\makeatother
% Allow footnotes in longtable head/foot
\IfFileExists{footnotehyper.sty}{\usepackage{footnotehyper}}{\usepackage{footnote}}
\makesavenoteenv{longtable}
\usepackage{graphicx}
\makeatletter
\def\maxwidth{\ifdim\Gin@nat@width>\linewidth\linewidth\else\Gin@nat@width\fi}
\def\maxheight{\ifdim\Gin@nat@height>\textheight\textheight\else\Gin@nat@height\fi}
\makeatother
% Scale images if necessary, so that they will not overflow the page
% margins by default, and it is still possible to overwrite the defaults
% using explicit options in \includegraphics[width, height, ...]{}
\setkeys{Gin}{width=\maxwidth,height=\maxheight,keepaspectratio}
% Set default figure placement to htbp
\makeatletter
\def\fps@figure{htbp}
\makeatother

\makeatletter
\makeatother
\makeatletter
\makeatother
\makeatletter
\@ifpackageloaded{caption}{}{\usepackage{caption}}
\AtBeginDocument{%
\ifdefined\contentsname
  \renewcommand*\contentsname{Tabla de contenidos}
\else
  \newcommand\contentsname{Tabla de contenidos}
\fi
\ifdefined\listfigurename
  \renewcommand*\listfigurename{Listado de Figuras}
\else
  \newcommand\listfigurename{Listado de Figuras}
\fi
\ifdefined\listtablename
  \renewcommand*\listtablename{Listado de Tablas}
\else
  \newcommand\listtablename{Listado de Tablas}
\fi
\ifdefined\figurename
  \renewcommand*\figurename{Figura}
\else
  \newcommand\figurename{Figura}
\fi
\ifdefined\tablename
  \renewcommand*\tablename{Tabla}
\else
  \newcommand\tablename{Tabla}
\fi
}
\@ifpackageloaded{float}{}{\usepackage{float}}
\floatstyle{ruled}
\@ifundefined{c@chapter}{\newfloat{codelisting}{h}{lop}}{\newfloat{codelisting}{h}{lop}[chapter]}
\floatname{codelisting}{Listado}
\newcommand*\listoflistings{\listof{codelisting}{Listado de Listados}}
\makeatother
\makeatletter
\@ifpackageloaded{caption}{}{\usepackage{caption}}
\@ifpackageloaded{subcaption}{}{\usepackage{subcaption}}
\makeatother
\makeatletter
\@ifpackageloaded{tcolorbox}{}{\usepackage[skins,breakable]{tcolorbox}}
\makeatother
\makeatletter
\@ifundefined{shadecolor}{\definecolor{shadecolor}{rgb}{.97, .97, .97}}
\makeatother
\makeatletter
\makeatother
\makeatletter
\makeatother
\ifLuaTeX
\usepackage[bidi=basic]{babel}
\else
\usepackage[bidi=default]{babel}
\fi
\babelprovide[main,import]{spanish}
% get rid of language-specific shorthands (see #6817):
\let\LanguageShortHands\languageshorthands
\def\languageshorthands#1{}
\ifLuaTeX
  \usepackage{selnolig}  % disable illegal ligatures
\fi
\usepackage[]{biblatex}
\addbibresource{../../../../references.bib}
\IfFileExists{bookmark.sty}{\usepackage{bookmark}}{\usepackage{hyperref}}
\IfFileExists{xurl.sty}{\usepackage{xurl}}{} % add URL line breaks if available
\urlstyle{same} % disable monospaced font for URLs
\hypersetup{
  pdftitle={¿Cuáles son las cualidades de los servidores públicos?},
  pdfauthor={Edison Achalma},
  pdflang={es},
  colorlinks=true,
  linkcolor={blue},
  filecolor={Maroon},
  citecolor={Blue},
  urlcolor={Blue},
  pdfcreator={LaTeX via pandoc}}

\title{¿Cuáles son las cualidades de los servidores públicos?}
\usepackage{etoolbox}
\makeatletter
\providecommand{\subtitle}[1]{% add subtitle to \maketitle
  \apptocmd{\@title}{\par {\large #1 \par}}{}{}
}
\makeatother
\subtitle{Cualidades de los servidores públicos.}
\author{Edison Achalma}
\date{2023-05-11}

\begin{document}
\maketitle
\ifdefined\Shaded\renewenvironment{Shaded}{\begin{tcolorbox}[sharp corners, enhanced, frame hidden, interior hidden, borderline west={3pt}{0pt}{shadecolor}, breakable, boxrule=0pt]}{\end{tcolorbox}}\fi

\hypertarget{cuuxe1les-son-las-cualidades-de-los-servidores-puxfablicos}{%
\section{¿Cuáles son las cualidades de los servidores
públicos?}\label{cuuxe1les-son-las-cualidades-de-los-servidores-puxfablicos}}

\hypertarget{amabilidad}{%
\subsection{Amabilidad}\label{amabilidad}}

Con demasiada frecuencia, equiparamos la amabilidad con no ser asertivo
o incluso ser agresivo. Hay tantas personas llenas de ideas brillantes.
Pero aquellos con la amabilidad de crear formas inclusivas y poderosas
de diseñar, dar forma, entregar y evaluar esas ideas brillantes son las
que siempre quiero conectar y aprender.

\hypertarget{juicio-astuto}{%
\subsection{Juicio astuto}\label{juicio-astuto}}

Es la capacidad de ver a través de alguien. El servidor público está en
una posición única: está obligado por la maquinación de un maestro
político. Tiene que entregar a caras desconocidas. Uno tiene que ver a
través de los velos, ya que hay muchas máscaras que usan las personas.
Para un servidor público, esto es muy importante: que entiendan a qué
máscara le están hablando. Si se entiende eso, se resuelve muchos
misterios y desamores. Los siguientes pasos son imaginación, trabajo
duro, trabajo en equipo y una mente abierta.

\hypertarget{habilidades-a-prueba-de-futuro}{%
\subsection{Habilidades a prueba de
futuro}\label{habilidades-a-prueba-de-futuro}}

La cara del gobierno está cambiando: el cambio no es solo en los poderes
geopolíticos, sino en los crecientes poderes de las ciudades, del sector
privado, de la sociedad civil. Se necesita repensar el servicio público
en general. ¿Cómo capacitamos a los departamentos gubernamentales para
estar preparados para un futuro complejo y cambiante y las necesidades y
vidas cambiantes de los ciudadanos? Necesita un replanteamiento, no solo
de habilidades y servicios y programas, sino del tipo de talento y
habilidades para los que reclutamos.

Los servidores públicos deben ser considerablemente más expertos en
datos y tecnología~para comprender los poderes y las oportunidades que
ofrece la tecnología emergente. Necesitan la capacidad de ir más allá de
los programas y políticas ``marcar la casilla'' y adoptar enfoques y
socios innovadores.

\hypertarget{justicia}{%
\subsection{~Justicia}\label{justicia}}

Lo que hace a un bueno a un servidor público, particularmente en
términos de políticos y funcionarios políticos, es la calidad de operar
desde una premisa de ideas y principios, no intereses individuales y
agendas partidistas.

\hypertarget{tenacidad}{%
\subsection{Tenacidad}\label{tenacidad}}

La vocación civil y el compromiso requieren el deseo constante de
mejorar la calidad de vida de la sociedad, la situación actual y las
oportunidades futuras. Todos buscan lo mejor para sus familias, pero un
funcionario busca lo mejor para todas las sociedades y países.

Necesitan tenacidad y consistencia para lograr resultados prácticos para
mejorar la vida de las personas; para construir esa visión de una
sociedad mejor.

\hypertarget{un-sentido-del-deber}{%
\subsection{Un sentido del deber}\label{un-sentido-del-deber}}

Se está atravesando un proceso doloroso para descubrir la corrupción en
el Perú en este momento y, en el fondo, casi siempre hay un servidor
público que no puede resistir alguna forma de soborno. Obviamente, se
debe ofrecer un soborno antes de que pueda ser aceptado, pero aquí es
donde entra en juego la integridad como valor para los servidores
públicos.

Entonces un servidor público siempre debe tener en cuenta al
``servidor'' en el servidor público. En nuestro contexto, una posición
del gobierno a menudo se ve como una posición de estado, y no una de
servir al bien público.

\hypertarget{un-toque-de-rebeliuxf3n.}{%
\subsection{Un toque de rebelión.}\label{un-toque-de-rebeliuxf3n.}}

Los servidores públicos más efectivos son aquellos que desafían
constantemente. Desafían sus propias ideas y prejuicios, desafían la
forma en que ``siempre se ha hecho'' y desafían a los líderes de
pensamiento de alto nivel.

Los realmente efectivos colaboran con socios que también desafiarán el
statu quo. El desafío y la colaboración conducen a la innovación, y eso
es lo que nos ayudará a todos a resolver algunos de los problemas más
apremiantes de nuestro tiempo.

\hypertarget{motivador}{%
\subsection{Motivador}\label{motivador}}

Ser motivador es un factor relevante para que las personas que trabajan
en el sector público vayan a cumplir sus funciones de la mejor manera y
puedan ayudar a que los objetivos institucionales se cumplan en la fecha
programada.

\hypertarget{promueve-el-diuxe1logo}{%
\subsection{Promueve el diálogo}\label{promueve-el-diuxe1logo}}

El diálogo es básico en el sector público para llegar a consensos
rápidamente. Un servidor público con esta característica será capaz de
destrabar muchos cuellos de botella con una llamada o conversación cara
a cara para sacar adelante sus proyectos.

\hypertarget{serenidad}{%
\subsection{Serenidad}\label{serenidad}}

En situaciones extremas y de crisis, el funcionario público moderno debe
saber guardar la calma para no transmitir la desesperación a sus
compañeros. Así, podrá lidiar con las presiones del corto y largo plazo
gracias a una toma de decisiones inteligente.

\hypertarget{enfocado-en-resultados}{%
\subsection{Enfocado en resultados}\label{enfocado-en-resultados}}

Esta es una cualidad imprescindible pues todas las entidades públicas y
sus funciones están hechas para cumplir una serie de objetivos. Si el
servidor público no llega a su trabajo con la idea de alcanzarlos
entonces no ayudará al desarrollo general.

\hypertarget{sabe-delegar}{%
\subsection{Sabe delegar}\label{sabe-delegar}}

Una sola persona no va a poder acumular toda la carga de una entidad.
Por eso, es vital saber delegar, crear una línea horizontal de trabajo
para empoderar a su equipo, motivarla y, como consecuencia, cumplir o
superar los objetivos.


\printbibliography


\end{document}
