% Options for packages loaded elsewhere
\PassOptionsToPackage{unicode}{hyperref}
\PassOptionsToPackage{hyphens}{url}
\PassOptionsToPackage{dvipsnames,svgnames,x11names}{xcolor}
%
\documentclass[
  letterpaper,
  DIV=11,
  numbers=noendperiod]{scrartcl}

\usepackage{amsmath,amssymb}
\usepackage{iftex}
\ifPDFTeX
  \usepackage[T1]{fontenc}
  \usepackage[utf8]{inputenc}
  \usepackage{textcomp} % provide euro and other symbols
\else % if luatex or xetex
  \usepackage{unicode-math}
  \defaultfontfeatures{Scale=MatchLowercase}
  \defaultfontfeatures[\rmfamily]{Ligatures=TeX,Scale=1}
\fi
\usepackage{lmodern}
\ifPDFTeX\else  
    % xetex/luatex font selection
\fi
% Use upquote if available, for straight quotes in verbatim environments
\IfFileExists{upquote.sty}{\usepackage{upquote}}{}
\IfFileExists{microtype.sty}{% use microtype if available
  \usepackage[]{microtype}
  \UseMicrotypeSet[protrusion]{basicmath} % disable protrusion for tt fonts
}{}
\makeatletter
\@ifundefined{KOMAClassName}{% if non-KOMA class
  \IfFileExists{parskip.sty}{%
    \usepackage{parskip}
  }{% else
    \setlength{\parindent}{0pt}
    \setlength{\parskip}{6pt plus 2pt minus 1pt}}
}{% if KOMA class
  \KOMAoptions{parskip=half}}
\makeatother
\usepackage{xcolor}
\setlength{\emergencystretch}{3em} % prevent overfull lines
\setcounter{secnumdepth}{-\maxdimen} % remove section numbering
% Make \paragraph and \subparagraph free-standing
\ifx\paragraph\undefined\else
  \let\oldparagraph\paragraph
  \renewcommand{\paragraph}[1]{\oldparagraph{#1}\mbox{}}
\fi
\ifx\subparagraph\undefined\else
  \let\oldsubparagraph\subparagraph
  \renewcommand{\subparagraph}[1]{\oldsubparagraph{#1}\mbox{}}
\fi


\providecommand{\tightlist}{%
  \setlength{\itemsep}{0pt}\setlength{\parskip}{0pt}}\usepackage{longtable,booktabs,array}
\usepackage{calc} % for calculating minipage widths
% Correct order of tables after \paragraph or \subparagraph
\usepackage{etoolbox}
\makeatletter
\patchcmd\longtable{\par}{\if@noskipsec\mbox{}\fi\par}{}{}
\makeatother
% Allow footnotes in longtable head/foot
\IfFileExists{footnotehyper.sty}{\usepackage{footnotehyper}}{\usepackage{footnote}}
\makesavenoteenv{longtable}
\usepackage{graphicx}
\makeatletter
\def\maxwidth{\ifdim\Gin@nat@width>\linewidth\linewidth\else\Gin@nat@width\fi}
\def\maxheight{\ifdim\Gin@nat@height>\textheight\textheight\else\Gin@nat@height\fi}
\makeatother
% Scale images if necessary, so that they will not overflow the page
% margins by default, and it is still possible to overwrite the defaults
% using explicit options in \includegraphics[width, height, ...]{}
\setkeys{Gin}{width=\maxwidth,height=\maxheight,keepaspectratio}
% Set default figure placement to htbp
\makeatletter
\def\fps@figure{htbp}
\makeatother

\KOMAoption{captions}{tableheading,figureheading}
\makeatletter
\makeatother
\makeatletter
\makeatother
\makeatletter
\@ifpackageloaded{caption}{}{\usepackage{caption}}
\AtBeginDocument{%
\ifdefined\contentsname
  \renewcommand*\contentsname{Tabla de contenidos}
\else
  \newcommand\contentsname{Tabla de contenidos}
\fi
\ifdefined\listfigurename
  \renewcommand*\listfigurename{Listado de Figuras}
\else
  \newcommand\listfigurename{Listado de Figuras}
\fi
\ifdefined\listtablename
  \renewcommand*\listtablename{Listado de Tablas}
\else
  \newcommand\listtablename{Listado de Tablas}
\fi
\ifdefined\figurename
  \renewcommand*\figurename{Figura}
\else
  \newcommand\figurename{Figura}
\fi
\ifdefined\tablename
  \renewcommand*\tablename{Tabla}
\else
  \newcommand\tablename{Tabla}
\fi
}
\@ifpackageloaded{float}{}{\usepackage{float}}
\floatstyle{ruled}
\@ifundefined{c@chapter}{\newfloat{codelisting}{h}{lop}}{\newfloat{codelisting}{h}{lop}[chapter]}
\floatname{codelisting}{Listado}
\newcommand*\listoflistings{\listof{codelisting}{Listado de Listados}}
\makeatother
\makeatletter
\@ifpackageloaded{caption}{}{\usepackage{caption}}
\@ifpackageloaded{subcaption}{}{\usepackage{subcaption}}
\makeatother
\makeatletter
\@ifpackageloaded{tcolorbox}{}{\usepackage[skins,breakable]{tcolorbox}}
\makeatother
\makeatletter
\@ifundefined{shadecolor}{\definecolor{shadecolor}{rgb}{.97, .97, .97}}
\makeatother
\makeatletter
\makeatother
\makeatletter
\makeatother
\ifLuaTeX
\usepackage[bidi=basic]{babel}
\else
\usepackage[bidi=default]{babel}
\fi
\babelprovide[main,import]{spanish}
% get rid of language-specific shorthands (see #6817):
\let\LanguageShortHands\languageshorthands
\def\languageshorthands#1{}
\ifLuaTeX
  \usepackage{selnolig}  % disable illegal ligatures
\fi
\usepackage[]{biblatex}
\addbibresource{../../../../references.bib}
\IfFileExists{bookmark.sty}{\usepackage{bookmark}}{\usepackage{hyperref}}
\IfFileExists{xurl.sty}{\usepackage{xurl}}{} % add URL line breaks if available
\urlstyle{same} % disable monospaced font for URLs
\hypersetup{
  pdftitle={La Economía Peruana de 1970 a 1990 Un Análisis de sus Períodos y Desafíos},
  pdfauthor={Edison Achalma},
  pdflang={es},
  colorlinks=true,
  linkcolor={blue},
  filecolor={Maroon},
  citecolor={Blue},
  urlcolor={Blue},
  pdfcreator={LaTeX via pandoc}}

\title{La Economía Peruana de 1970 a 1990 Un Análisis de sus Períodos y
Desafíos}
\usepackage{etoolbox}
\makeatletter
\providecommand{\subtitle}[1]{% add subtitle to \maketitle
  \apptocmd{\@title}{\par {\large #1 \par}}{}{}
}
\makeatother
\subtitle{Una revisión detallada de los gobiernos y políticas económicas
que marcaron la evolución del Perú.}
\author{Edison Achalma}
\date{2023-05-12}

\begin{document}
\maketitle
\ifdefined\Shaded\renewenvironment{Shaded}{\begin{tcolorbox}[interior hidden, sharp corners, enhanced, frame hidden, boxrule=0pt, borderline west={3pt}{0pt}{shadecolor}, breakable]}{\end{tcolorbox}}\fi

\hypertarget{la-economuxeda-peruana-desde-1970-hasta-1990}{%
\section{LA ECONOMÍA PERUANA DESDE 1970 HASTA
1990}\label{la-economuxeda-peruana-desde-1970-hasta-1990}}

\hypertarget{peruxedodo-de-1970-1975}{%
\subsection{PERÍODO DE 1970 -- 1975}\label{peruxedodo-de-1970-1975}}

PRESIDENTE : General JUAN VELASCO ALVARADO (3 de octubre de 1968)

Las principales medidas que adoptó en el plano económico el gobierno del
General Juan Velasco Alvarado fueron:

\begin{enumerate}
\def\labelenumi{\arabic{enumi}.}
\tightlist
\item
  En 1969 se promulgó la Ley de Reforma Agraria. Los grandes y medianos
  latifundios serían expropiados y transformados, básicamente, en
  cooperativas. La tierra era de quien la trabajaba.
\item
  En 1970 se aprobó industrias, cuyos lineamientos estaban contenidos en
  la Ley General de Industria. En ella se establecían una serie de
  prioridades industriales y se establecían incentivos tributarios para
  los sectores elegidos. Se trató de una ley promocional; el estado
  reservó el monopolio de la industria básica.
\item
  La ley de la minería fue promulgada en 1970, Esta ley tenía como
  objetivo controlar la inversión extranjera de modo que sirviera a los
  intereses del Perú y le otorgaba al estado el monopolio de la
  comercialización de todos los productos mineros. La intención de esta
  ley era regular la inversión.
\item
  Se establecieron las comunidades laborales en industria, minería y
  pesca entre 1970 y 1971. Los trabajadores, organizados en las
  comunidades laborales, tendrían participación directa en la
  administración y beneficios de las empresas. Se establecieron las
  cooperativas de producción en el transporte urbano, en algunas
  industrias y en la agricultura. La comunidad industrial fue
  establecida en las empresas industriales del sector moderno.
\item
  En 1970 se decretó la Ley de Estabilidad Laboral, de acuerdo con la
  cual, luego de tres meses, un trabajador adquiría estabilidad laboral
  y no podía ser despedido.
\item
  En 1973 se crearon las Empresas de Propiedad, que eran empresas
  manejadas por los trabajadores, similares a las que existían en
  Yugoslavia.
\item
  Se estableció una nueva política con respecto al capital extranjero.
  La mayoría de activos extranjeros en el sector primario, en
  comunicaciones y en pesca, fueron expropiados.
\item
  Se creó un gran número de empresas estatales, incrementándose en el
  papel planificador del estado y su injerencia en la economía. Prueba
  de ese deseo del estado de participar más activamente en los diversos
  sectores económicos fue la creación de los ministerios de minería y
  petróleo y el de industria.
\item
  Se organizaron los sectores financieros y comerciales. Se reforzó el
  papel de la Banca de Fomento. Se crearon las corporaciones Financieras
  de Desarrollo (COFIDE), con el objetivo de financiar proyectos de
  desarrollo, y la Comisión Supervisora de Empresas y Valores. El estado
  pasó a controlar el crédito comprando varios bancos comerciales (Banco
  Popular, Banco Internacional, el Banco Continental). El crédito se
  otorgaría selectivamente por regiones y por sectores económicos.
\item
  Se establecieron controles cambiarios y se incrementó la protección a
  la industria, creándose el Registro Nacional de Manufacturas. Este
  registro era un listado de todos los productos manufacturados
  producidos localmente; las importaciones de los sustitutos de los
  mismos quedaron prohibidos.
\item
  Se reforzó el sistema de planificación nacional, que se encargaría de
  establecer metas, formular planes de desarrollo de corto, mediano y
  largo plazo y coordinar la inversión sectorial. El Instituto Nacional
  de Planificación (INP) tendría a su cargo la planificación, buscando
  conciliar el crecimiento económico con la equidad distributiva.
  Demostración de ello fueron el plan inca y el plan Túpac Amaru.
\end{enumerate}

Pero el proceso fue muy contradictorio; llevado adelante sin respetar
las libertades ciudadanas se hizo evidente su naturaleza autoritaria y
vertical.

\hypertarget{peruxedodo-de-1975---1980}{%
\subsection{PERÍODO DE 1975 - 1980}\label{peruxedodo-de-1975---1980}}

PRESIDENTE: General FRANCISCO MORALES BERMÚDEZ CERRUTI (29 de agosto
de~1975)

Las principales medidas que adoptó en el plano económico el gobierno del
General Francisco Morales Bermúdez fueron:

\begin{enumerate}
\def\labelenumi{\arabic{enumi}.}
\tightlist
\item
  El gobierno buscó promover la inversión privada, tratando de recuperar
  la confianza de los inversionistas extranjeros.
\item
  Se disminuyeron los subsidios, para reducir el gasto público.
\item
  Se anuncia el 12 de enero de 1976 un programa de ajuste.
\end{enumerate}

\begin{itemize}
\tightlist
\item
  Eliminación de subsidios para reducir el déficit de las empresas
  públicas.
\item
  Elevar precios en los mercados.
\item
  Devaluación del sol.
\item
  En 1976 las reservas internacionales netas se tornaron negativas.
\item
  El déficit fiscal llegó a 6.3\% del PBI.
\item
  La brecha externa fue del 8.6\% del PBI.
\end{itemize}

\begin{enumerate}
\def\labelenumi{\arabic{enumi}.}
\setcounter{enumi}{3}
\tightlist
\item
  La inflación a fin de periodo fue del 44\% aprox.
\end{enumerate}

\begin{itemize}
\tightlist
\item
  El servicio de la deuda externa fue el 36.3\% de las exportaciones.
\item
  El FMI exigía liberar el tipo de cambio, para reducir el déficit
  fiscal.
\item
  En 1977, el crédito al sector público representaba el 49\%.
\end{itemize}

\begin{enumerate}
\def\labelenumi{\arabic{enumi}.}
\setcounter{enumi}{4}
\tightlist
\item
  El tipo de cambio aumentó para mejorar las exportaciones pero el PBI
  cayó en 1.2\% en 1977 y en 1978 el tipo de cambio se devaluó 15,4\%;
  con un sistema de minidevaluaciones..
\end{enumerate}

\begin{itemize}
\tightlist
\item
  Al utilizarse la estrategia de disminución de la demanda interna,
  surgieron problemas sociales como las famosas huelgas generales,
  debido a los periodos de ajuste y caída de salarios reales.
\item
  La inflación a fin de periodo aumentó de 32.4\% en 1977 a 73.7\% en
  1978. El PBI creció 0.6 en 1977, y cayó en - 3.8\% en 1978. Las
  importaciones disminuyeron 24.5\% en 1978.La brecha externa se cerró:
  pasó de - 7.4\% en 1977 a -1.8\% del PBI en 1978. Crisis de balanza de
  pagos, RIN negativas entre 1976 y 1978, el déficit fiscal se redujo de
  un 5.1\% en 1978 a 0.6\% del PBI en 1979
\end{itemize}

\begin{enumerate}
\def\labelenumi{\arabic{enumi}.}
\setcounter{enumi}{5}
\tightlist
\item
  En1979 los términos de intercambio aumentaron por razones
  internacionales, mejorando los precios internacionales de materias
  primas y se empieza a exportar petróleo.
\end{enumerate}

\begin{itemize}
\tightlist
\item
  Entre 1977 y 1979 el volumen exportado de petróleo aumentó 488\%; y
  los ingresos por exportación, 254\% anual, pasando a ser el segundo
  producto de exportación después del cobre, motivo por el cual, la
  brecha externa (cuenta corriente) mejoró llegando a 11.3\% del PBI en
  1975.
\end{itemize}

\begin{enumerate}
\def\labelenumi{\arabic{enumi}.}
\setcounter{enumi}{6}
\tightlist
\item
  La deuda pública representaba el 16.1\% del PBI en 1973, y pasó al
  42\% en 1979
\end{enumerate}

\hypertarget{peruxedodo-de-1980---1985}{%
\subsection{PERÍODO DE 1980 - 1985}\label{peruxedodo-de-1980---1985}}

En la década del 80, el Perú se convirtió prácticamente inviable azotado
por la violencia terrorista, el fenómeno del niño de 1983, el más fuerte
de las últimas décadas, la hiperinflación y entre otros. Aislado
política, financiera y comercialmente, el Perú era un lugar en el que
ningún joven deseaba vivir. La década del ochenta, en particular, fue la
peor década del siglo XX para el Perú y ALC; de allí su denominación de
``década perdida''.

Segundo gobierno del presidente Fernando Belaunde Terry (1980 -- 1985)

Regida por la Constitución política de 1979, tuvo como régimen
económico: la economía social de mercado, el pluralismo económico y la
libertad de comercio e industria.

Política Económica: ortodoxa

Fernando Belaunde emprendió la difícil tarea de superar la profunda
crisis que heredó del gobierno militar.

Fenómeno del niño de 1983, que ocasiono el alza del déficit fiscal. Bajo
la supervisión del FMI, aplicó medidas~\emph{ortodoxas}, que redujo el
déficit fiscal. En este año, la agricultura cayó en 12\%, la pesca en
40\% y la minería, en 8\%, industria procesadora de recursos primarios,
en 17\%. Se redujo el gasto fiscal y la participación estatal en la
economía, y se promovió el crédito y la inversión extranjera

El gobierno de Belaúnde dejó al país en una profunda crisis económica.
Las inversiones habían caído de 21,2 \% del Producto Bruto Interno
(PBI), en 1982, a 12,2 \% en 1985. En 1982, la economía peruana no
creció y, en 1983, el crecimiento fue negativo: -12,2 \%. Si, en 1980,
el ingreso per cápita era de 1,232 dólares por peruano, en 1985 llegaba
tan sólo a 1,050 dólares. El desastre económico del gobierno de Belaúnde
se debió, principalmente, a una caída de precios de productos que Perú
exportaba (cobre, plata, plomo, café).

La década del ochenta el Perú registró una salida de capitales privados
por las altas tasas de interés internacional y la irrupción violenta de
Sendero Luminoso en 1981 y su subsistencia hasta 1992, trajo consigo que
las reservas internacionales se reduzcan.

\hypertarget{peruxedodo-de-1985-1990}{%
\subsection{PERÍODO DE 1985 -- 1990}\label{peruxedodo-de-1985-1990}}

Primer gobierno del presidente Alan García Pérez~ (1985 -- 1990), regida
también por la Constitución política de 1979.

Política Económica: heterodoxa, (sin ceñirse a los dictados del Fondo
Monetario Internacional FMI). La principal medida económica consistió en
la congelación de~precios~básicos salario, utilidades, tasas
de~interés~y tipo de cambio, pero antes de que esto sea dictaminado, se
había incrementado los precios públicos y se devalúa el dólar en 12 \%;
el dólar MUC (moneda única de cambio), se fija en 13.95 intis y el dólar
financiero en 17.5 intis por dólar norteamericano. Pero estos no fueron
las únicas medidas que se tomaron en el ajuste de 1985, ya que también
se opto por pagar la~deuda externa~con sólo el 10\% de
las~exportaciones~y no acudiendo a intermediarios como lo era
el~FMI~entre otros, es por eso que en 1986 el Perú fue declarado
inelegible a créditos internacionales. Las medidas adoptadas en Julio de
1985, tuvieron los resultados deseados, con ello la heterodoxia ganaba
credibilidad y se fue reforzando la voluntad de aumentar la producción.
Las empresas no invirtieron en el país y sólo se limitaron a aumentar la
producción~y los precios.

En 1987, el peligro de una crisis en la balanza de pagos y en las
reservas internacionales era evidente. A medida que los~límites~al
programa se hacían evidentes y los desequilibrios macroeconómicos se
hacían insostenibles, ya desde Mayo de 1988 se comienzan a sentir
ajustes a la economía los llamados ``paquetazos'' llevando a una gran
recesión económica. Intento de estatizar la banca privada, como
consecuencia nace el movimiento libertad de Vargas Llosa.

Los efectos de la desastrosa política del gobierno se mostraron no sólo
con las colas interminables para conseguir los productos alimenticios,
sino que se produjeron huelgas, desabastecimiento, violencia,
especulación, se incrementó la gasolina en 30 \%; el servicio postal y
telefónico 20 \%; agua potable y alcantarillado~10 \%, el déficit fiscal
aumentó considerablemente, el PBI real cayó y una Inflación acumulada de
2,178.482 \%.

Es por ello que el gobierno de García ha sido catalogado como el peor
gobierno de la historia del Perú.

\hypertarget{las-medidas-de-la-reforma-econuxf3mica-en-la-duxe9cada-del-90}{%
\subsection{LAS MEDIDAS DE LA REFORMA ECONÓMICA EN LA DÉCADA DEL
90:}\label{las-medidas-de-la-reforma-econuxf3mica-en-la-duxe9cada-del-90}}

Durante el resto de 1990 y buena parte de 1991, las principales medidas
del nuevo gobierno se concentraron en el plano económico: continuar con
el ajuste para lograr la estabilización, iniciar la liberalización de la
economía y la reforma estructural del aparato estatal y conseguir la
reinserción del país en el sistema financiero internacional, por lo que
a partir de gestiones efectuadas, el país logró su reinserción con el
otorgamiento de nuevos créditos e incluso donaciones no reembolsables.
(DESCO. Resumen Semanal, n.° 577, 6-12 julio de 1990)

Se inició con las reformas económicas e institucionales, más conocidas
como ajuste estructural, con el apoyo del Banco Mundial. Se liberalizó
el mercado financiero doméstico y el Banco Central dejó de regular la
tasa de interés y el tipo de cambio; se eliminó una serie de monopolios
estatales en la salud, educación, agricultura, minería, transportes,
puertos. En 1992 se constituyó la Comisión de Promoción de la Inversión
Privada (COPRI), que quedó a cargo de ejecutar el proceso de
privatización de las mismas. (Gustavo Guerra-García, 1999).Se estableció
el mercado de tierras; se flexibilizó el mercado de trabajo y se
liberalizó el comercio exterior, ante un país sin reservas, la única
solución era restringir la oferta monetaria y liberar los precios, a
pesar de la incertidumbre que generó en ese momento y el impacto sobre
las personas con menores recursos. Los precios de los productos de
primera necesidad se dispararon, pero con el límite al déficit fiscal y
la impresión inorgánica de dinero, la inflación empezó a bajar.

Las políticas de corte neoliberal que Fujimori aplicó tuvieron un
elevado costo social: trabajadores, campesinos, jubilados, fueron
quienes cargaron con el grueso del ajuste, mediante alzas de precios,
reducción de salarios y pensiones reales, pérdida de empleos y recorte
de derechos laborales, etc. cuando Fujimori asumió el Gobierno, el 44\%
de los peruanos vivían en situación de pobreza. Tras la aplicación del
``Fujishock'' esta cifra creció al 60\%.

\hypertarget{conclusiuxf3n}{%
\section{CONCLUSIÓN}\label{conclusiuxf3n}}

La economía peruana ha sufrido de una deficiente asignación de recursos
y bajos niveles de inversión privada. Durante las décadas de los
sesenta, setenta y ochenta el sistema económico presentaba un cuadro con
precios regulados y excesivo intervencionismo estatal. Asimismo, la
actividad económica se desarrollaba en un ambiente de fuerte
inestabilidad macroeconómica y los altos endeudamientos públicos
terminaban con episodios de insolvencia fiscal. De otro lado, eran
recurrentes las crisis de balanza de pagos, que terminaban en grandes
devaluaciones de la moneda; y, por si fuera poco, se vivieron periodos
inflacionarios e hiperinflacionarios, mientras en la década de los 90 se
inició con los ajustes para estabilizar la economía peruana.

El Perú de hoy es macro económicamente más estable y próspero, pero es
también más desigual y excluyente. La riqueza se ha concentrado y el
Estado se ha debilitado y retrocedido, mientras
los~sempiternos~problemas de la pobreza y la discriminación se han
profundizado.

\hypertarget{bibliografuxeda}{%
\section{BIBLIOGRAFÍA}\label{bibliografuxeda}}

\begin{itemize}
\tightlist
\item
  Carlos PARODI TRECE, ``Perú, 1960-2000: políticas económicas y
  sociales en entornos cambiantes.'' 2001
\item
  SEGUNDO GOBIERNO DE FERNANDO BELAUNDE TERRY: TRABAJO REALIZADO POR
  ESTUDIANTES DE LA UNIVERSIDAD DE LIMA
\item
  Waldo, M. B. (2013). Contexto internacional y desempeño macroeconómico
  en américa latina y el Perú: 1980-2012. Lima, Perú: Departamento de
  Economía -- Pontificia Universidad Católica del Perú
\item
  Alfredo, D., \& Raúl, G.C. (2013).La economía mundial ¿Hacia dónde
  vamos? Lima, Perú: Pontificia Universidad Católica del Perú.
\item
  Efraín, G. O. (1991) Una economía bajo violencia: Perú, 1980-1990.
  Lima, Perú: IEP Instituto de Estudios Peruanos
\item
  Eduardo, M.P. (2013). Los desafíos del Perú. Lima, Perú: Universidad
  del Pacífico.
\item
  Abusada, R., F. Dubois, E. Morón (2000) La Reforma Incompleta:
  Rescatando los noventa. Lima, Perú. Instituto Peruano de
  Economía/Universidad del Pacífico
\item
  Galeón. (n.d). Historia. Obtenida el 11 de julio del 2015
\item
  La década de los noventa y los dos gobiernos de Alberto Fujimori.
  Obtenida el 11 de julio del 2015, de
  http://accion-popular-juventud-arequipa.blogspot.com/2010/03/segundo-gobierno-de-fernando-belaunde.html
\item
  Álvarez Palenzuela, V.A. (coord.) (2008). Historia Universal de la
  Edad Media, Barcelona: Ariel.
\item
  Bloch, M. (1989). Feudal Society: The Growth of Ties of Dependence,
  Londres: Routledge.
\item
  Boutruche, R. (1995). Señorío y feudalismo, 1.Los vínculos de
  dependencia, Madrid: Siglo XXI.
\item
  Calva, J. L. (1988). Los campesinos y su devenir en las economías de
  mercado, México, Siglo XXI.
\item
  Ganshof, F.L. (1975). El feudalismo, Barcelona: Ariel, Barcelona.
\item
  Huizinga, J. (2001). El Otoño de la Edad Media, Madrid: Alianza.
\item
  Little, L. y B. Rosenwein (ed.) (2003). La Edad Media a debate,
  Madrid: Akal.
\item
  Mitre, E. (2004). Introducción a la Historia de la Edad Media europea,
  Madrid: Istmo.
\end{itemize}


\printbibliography


\end{document}
