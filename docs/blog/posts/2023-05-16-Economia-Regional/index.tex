% Options for packages loaded elsewhere
\PassOptionsToPackage{unicode}{hyperref}
\PassOptionsToPackage{hyphens}{url}
\PassOptionsToPackage{dvipsnames,svgnames,x11names}{xcolor}
%
\documentclass[
  a4paper,
]{article}

\usepackage{amsmath,amssymb}
\usepackage{lmodern}
\usepackage{iftex}
\ifPDFTeX
  \usepackage[T1]{fontenc}
  \usepackage[utf8]{inputenc}
  \usepackage{textcomp} % provide euro and other symbols
\else % if luatex or xetex
  \usepackage{unicode-math}
  \defaultfontfeatures{Scale=MatchLowercase}
  \defaultfontfeatures[\rmfamily]{Ligatures=TeX,Scale=1}
\fi
% Use upquote if available, for straight quotes in verbatim environments
\IfFileExists{upquote.sty}{\usepackage{upquote}}{}
\IfFileExists{microtype.sty}{% use microtype if available
  \usepackage[]{microtype}
  \UseMicrotypeSet[protrusion]{basicmath} % disable protrusion for tt fonts
}{}
\makeatletter
\@ifundefined{KOMAClassName}{% if non-KOMA class
  \IfFileExists{parskip.sty}{%
    \usepackage{parskip}
  }{% else
    \setlength{\parindent}{0pt}
    \setlength{\parskip}{6pt plus 2pt minus 1pt}}
}{% if KOMA class
  \KOMAoptions{parskip=half}}
\makeatother
\usepackage{xcolor}
\usepackage[lmargin=2.54cm,rmargin=2.54cm,tmargin=2.54cm,bmargin=2.54cm]{geometry}
\setlength{\emergencystretch}{3em} % prevent overfull lines
\setcounter{secnumdepth}{3}
% Make \paragraph and \subparagraph free-standing
\ifx\paragraph\undefined\else
  \let\oldparagraph\paragraph
  \renewcommand{\paragraph}[1]{\oldparagraph{#1}\mbox{}}
\fi
\ifx\subparagraph\undefined\else
  \let\oldsubparagraph\subparagraph
  \renewcommand{\subparagraph}[1]{\oldsubparagraph{#1}\mbox{}}
\fi


\providecommand{\tightlist}{%
  \setlength{\itemsep}{0pt}\setlength{\parskip}{0pt}}\usepackage{longtable,booktabs,array}
\usepackage{calc} % for calculating minipage widths
% Correct order of tables after \paragraph or \subparagraph
\usepackage{etoolbox}
\makeatletter
\patchcmd\longtable{\par}{\if@noskipsec\mbox{}\fi\par}{}{}
\makeatother
% Allow footnotes in longtable head/foot
\IfFileExists{footnotehyper.sty}{\usepackage{footnotehyper}}{\usepackage{footnote}}
\makesavenoteenv{longtable}
\usepackage{graphicx}
\makeatletter
\def\maxwidth{\ifdim\Gin@nat@width>\linewidth\linewidth\else\Gin@nat@width\fi}
\def\maxheight{\ifdim\Gin@nat@height>\textheight\textheight\else\Gin@nat@height\fi}
\makeatother
% Scale images if necessary, so that they will not overflow the page
% margins by default, and it is still possible to overwrite the defaults
% using explicit options in \includegraphics[width, height, ...]{}
\setkeys{Gin}{width=\maxwidth,height=\maxheight,keepaspectratio}
% Set default figure placement to htbp
\makeatletter
\def\fps@figure{htbp}
\makeatother

\makeatletter
\makeatother
\makeatletter
\makeatother
\makeatletter
\@ifpackageloaded{caption}{}{\usepackage{caption}}
\AtBeginDocument{%
\ifdefined\contentsname
  \renewcommand*\contentsname{Table of contents}
\else
  \newcommand\contentsname{Table of contents}
\fi
\ifdefined\listfigurename
  \renewcommand*\listfigurename{List of Figures}
\else
  \newcommand\listfigurename{List of Figures}
\fi
\ifdefined\listtablename
  \renewcommand*\listtablename{List of Tables}
\else
  \newcommand\listtablename{List of Tables}
\fi
\ifdefined\figurename
  \renewcommand*\figurename{Figure}
\else
  \newcommand\figurename{Figure}
\fi
\ifdefined\tablename
  \renewcommand*\tablename{Table}
\else
  \newcommand\tablename{Table}
\fi
}
\@ifpackageloaded{float}{}{\usepackage{float}}
\floatstyle{ruled}
\@ifundefined{c@chapter}{\newfloat{codelisting}{h}{lop}}{\newfloat{codelisting}{h}{lop}[chapter]}
\floatname{codelisting}{Listing}
\newcommand*\listoflistings{\listof{codelisting}{List of Listings}}
\makeatother
\makeatletter
\@ifpackageloaded{caption}{}{\usepackage{caption}}
\@ifpackageloaded{subcaption}{}{\usepackage{subcaption}}
\makeatother
\makeatletter
\@ifpackageloaded{tcolorbox}{}{\usepackage[many]{tcolorbox}}
\makeatother
\makeatletter
\@ifundefined{shadecolor}{\definecolor{shadecolor}{rgb}{.97, .97, .97}}
\makeatother
\makeatletter
\makeatother
\ifLuaTeX
  \usepackage{selnolig}  % disable illegal ligatures
\fi
\usepackage[]{biblatex}
\addbibresource{../../../../references.bib}
\IfFileExists{bookmark.sty}{\usepackage{bookmark}}{\usepackage{hyperref}}
\IfFileExists{xurl.sty}{\usepackage{xurl}}{} % add URL line breaks if available
\urlstyle{same} % disable monospaced font for URLs
\hypersetup{
  pdftitle={Economía regional},
  pdfauthor={Achalma Mendoza Elmer Edison},
  colorlinks=true,
  linkcolor={blue},
  filecolor={Maroon},
  citecolor={Blue},
  urlcolor={Blue},
  pdfcreator={LaTeX via pandoc}}

\title{Economía regional\thanks{gracias}}
\usepackage{etoolbox}
\makeatletter
\providecommand{\subtitle}[1]{% add subtitle to \maketitle
  \apptocmd{\@title}{\par {\large #1 \par}}{}{}
}
\makeatother
\subtitle{Una leccion detallada de economia regional}
\author{Achalma Mendoza Elmer Edison}
\date{5/16/23}

\begin{document}
\maketitle
\begin{abstract}
hola
\end{abstract}
\ifdefined\Shaded\renewenvironment{Shaded}{\begin{tcolorbox}[boxrule=0pt, frame hidden, sharp corners, borderline west={3pt}{0pt}{shadecolor}, interior hidden, enhanced, breakable]}{\end{tcolorbox}}\fi

\renewcommand*\contentsname{Contenidos}
{
\hypersetup{linkcolor=}
\setcounter{tocdepth}{3}
\tableofcontents
}
\listoffigures
\listoftables
\hypertarget{pauxedses-en-vuxedas-de-desarrollo}{%
\section{Países en vías de
desarrollo}\label{pauxedses-en-vuxedas-de-desarrollo}}

Los países en vías de desarrollo son un grupo de naciones que se
caracterizan por su enfoque en el aprovechamiento del \textbf{potencial}
productivo de su sociedad. El concepto de potencial se refiere a la
\textbf{diferencia existente entre los recursos disponibles en el país y
los recursos que realmente se utilizan.}

Esta disparidad es especialmente notable en países en vías de desarrollo
como Perú, donde a pesar de contar con abundantes recursos, no se
explotan de manera integral. Por ejemplo, el país posee minerales y
minas cuyos precios están en alza, pero no se aprovechan en su
totalidad. Mientras tanto, otras economías están vendiendo y
aprovechando al máximo sus recursos.

Es crucial comprender la importancia de utilizar y vender estos
recursos, ya que los ingresos generados pueden contribuir
significativamente a mejorar las condiciones del país. Si se dejan sin
explotar, se retrasa el proceso de desarrollo. Un ejemplo concreto de
esto es el caso del gas de Camisea en Perú, que solo empezó a
aprovecharse en el año 2004.

Existen potencialidades de los recursos humanos y naturales que no se
están aprovechando en el marco de la política de desarrollo,
especialmente en el ámbito de la educación y la formación de mano de
obra calificada.

El desaprovechamiento de estas potencialidades impide que se destine de
manera adecuada recursos al sector educativo y a la formación de
trabajadores especializados. Esto, a su vez, limita la adopción y
aplicación de tecnologías modernas.

Es esencial comprender la importancia de utilizar de manera eficiente y
efectiva los recursos humanos y naturales disponibles. En el ámbito de
la educación, esto implica invertir en programas educativos de calidad
que promuevan la formación de una fuerza laboral altamente capacitada.
Además, se debe fomentar el desarrollo y la implementación de
tecnologías modernas que impulsen la productividad y la competitividad
en los sectores económicos.

El no aprovechamiento de estas potencialidades tiene consecuencias
negativas, ya que limita el crecimiento económico, la generación de
empleo y el avance tecnológico en el país. Es fundamental establecer
políticas y estrategias que promuevan la inversión en educación, la
mejora de la capacitación laboral y la adopción de tecnologías
innovadoras.

Es importante destacar la relevancia del recurso humano en relación al
empleo, ya que existe un porcentaje de la población en edad de trabajar
que se encuentra desempleada. Por ejemplo, de cada 100 personas en edad
laboral, aproximadamente 30 no están empleadas y no contribuyen a la
producción. Esto implica que, aunque podríamos producir 100 productos
con la totalidad de la fuerza laboral disponible, en realidad solo se
producen 70 productos debido a la falta de incorporación de este recurso
en el proceso productivo.

En otras palabras, \textbf{no estamos aprovechando plenamente las
potencialidades de nuestra producción} debido a la subutilización del
recurso humano disponible. Esta situación impacta negativamente en la
capacidad productiva y en el desarrollo económico.

Además, es importante señalar que una parte significativa de la
población en edad de trabajar no alcanza los niveles de productividad
esperados debido a la falta de formación y capacitación en mano de obra
calificada. Esta improductividad se convierte en un obstáculo para
lograr los niveles de producción deseados y limita el crecimiento
económico.

Para superar esta situación, es fundamental invertir en la formación y
capacitación de la fuerza laboral. Esto implica desarrollar programas
educativos y de formación que promuevan la adquisición de habilidades y
conocimientos especializados requeridos en el mercado laboral. De esta
manera, se puede mejorar la productividad y aprovechar plenamente el
potencial del recurso humano en el proceso productivo.

Tomemos como ejemplo los países árabes, conocidos por su papel como
grandes productores de petróleo y acumuladores de riqueza. Sin embargo,
en la actualidad, el petróleo está perdiendo fuerza como fuente de
energía, lo cual ha llevado a estos países a financiar a otras empresas
para evitar la entrada al mercado de productos como vehículos y
maquinarias eléctricas, que podrían reducir la demanda de petróleo.

Esperamos el crecimiento y prosperidad de estas empresas fabricantes de
vehículos eléctricos, ya que si los automóviles y maquinarias empiezan a
funcionar con sistemas eléctricos, el precio del petróleo se verá
afectado negativamente, lo que podría empobrecer a estas naciones
productoras. Sin embargo, a pesar de esta situación, estos países siguen
explotando al máximo sus recursos petroleros.

Por otro lado, en el caso del Perú, poseemos una abundante reserva de
litio, ocupando un destacado lugar después de Bolivia, Argentina y Chile
en términos de reservas de este recurso. Existe la posibilidad de que en
el futuro nos convirtamos en un gran productor de litio. No obstante, en
un acontecimiento lamentable, las reservas de litio fueron vendidas
durante el gobierno de Vizcarra.

\textbf{El desempleo es una característica propia de los países en vías
de desarrollo}, y en el caso del Perú, somos dependientes de las
fluctuaciones en las relaciones económicas internacionales. Por ejemplo,
cuando el precio del dólar aumenta, se refleja en el incremento del
precio del pollo, mientras que si el precio del dólar disminuye, el
precio del pollo también disminuye.

\hypertarget{a-quuxe9-se-debe-esto}{%
\subsection{¿A qué se debe esto?}\label{a-quuxe9-se-debe-esto}}

Se debe a que, por ejemplo, nosotros no producimos maíz de gallina, lo
produce Colombia. Gran parte de nuestras importaciones de maíz de
gallina proviene de ese país, lo cual tiene un impacto directo en el
precio del pollo en el mercado nacional.

Actualmente, el precio del pollo no ha aumentado significativamente
debido a que todavía existen existencias de maíz en los almacenes de los
agricultores locales. Sin embargo, cuando se agoten estas reservas y se
tenga que importar más maíz, es importante considerar que el precio del
dólar se espera que suba a 3.75, lo que inevitablemente resultará en un
incremento en los costos de producción. Como consecuencia, esto se
traducirá en un aumento en el precio final del pollo.

Es evidente que \textbf{tenemos una dependencia significativa} en
términos de abastecimiento, especialmente en el caso de nuestras
pequeñas empresas. Esto refuerza aún más la necesidad de fortalecer
nuestra capacidad de producción y reducir nuestra dependencia de las
importaciones en sectores clave como la alimentación avícola.

Es fundamental implementar estrategias que fomenten la producción local
de maíz y otros insumos agrícolas necesarios para la industria avícola.
Esto no solo ayudará a reducir la dependencia externa, sino que también
contribuirá al desarrollo económico y la autonomía de nuestras empresas
locales.

\hypertarget{pauxedses-industrializados}{%
\subsection{Países industrializados}\label{pauxedses-industrializados}}

En el contexto de los países industrializados, es fundamental destacar
la importancia de la industrialización en el proceso de desarrollo
económico. La presencia de industrias, tanto a nivel nacional como en el
ámbito global, desempeña un papel fundamental en la generación de
ingresos y el progreso de un país.

Los países industrializados se caracterizan por contar con una sólida
base industrial que impulsa su economía. La industrialización se refiere
al desarrollo y crecimiento de sectores manufactureros y productivos,
donde se transforman materias primas en productos acabados mediante
procesos de producción eficientes y tecnológicamente avanzados.

La presencia de industrias diversificadas y competitivas contribuye a
generar empleo, aumentar la productividad, impulsar la innovación
tecnológica y promover la exportación de productos manufacturados.
Además, las industrias aportan al crecimiento económico de un país al
generar ingresos a través de la producción y venta de bienes y
servicios.

Por otro lado, los países que carecen de un nivel significativo de
industrialización se consideran no industrializados. Estos países pueden
depender en gran medida de sectores primarios, como la agricultura, la
minería o la extracción de recursos naturales, lo que puede limitar su
capacidad para diversificar su economía y alcanzar un desarrollo
sostenible.

En el \textbf{caso de Perú}, es importante destacar las principales
industrias presentes en el país. Entre ellas se encuentran:

\begin{enumerate}
\def\labelenumi{\arabic{enumi}.}
\item
  \textbf{Industria Textil:} Sin embargo, esta industria se ha visto
  afectada debido a la falta de producción de algodón, una materia prima
  fundamental. La apertura de fronteras y la falta de control en los
  ingresos han permitido la entrada de productos textiles chinos a
  precios muy bajos, lo que ha llevado a la destrucción de la industria
  textil nacional.
\item
  \textbf{Industria Pesquera:} Perú cuenta con una importante industria
  pesquera, aprovechando su extensa costa y la abundancia de recursos
  marinos. Esta industria desempeña un papel clave en la generación de
  empleo y en las exportaciones de productos pesqueros.
\item
  \textbf{Industria de Cementos:} La producción y comercialización de
  cemento también es una actividad destacada en el país. La construcción
  de infraestructuras y el desarrollo de proyectos inmobiliarios
  impulsan la demanda de cemento, generando empleo y contribuyendo al
  crecimiento económico.
\end{enumerate}

Perú se caracteriza por ser una economía mixta, lo que significa que
existe participación tanto del Estado como del sector privado. Sin
embargo, es importante destacar que la administración de los recursos
por parte del Estado ha sido cuestionada en algunos casos. Por otro
lado, el sector privado juega un papel importante en la inversión
económica, ya que son las empresas privadas las que generalmente
realizan inversiones significativas en el país.

\hypertarget{las-caracteruxedsticas-de-una-sociedad-en-desarrollo-son}{%
\subsection{Las características de una sociedad en desarrollo
son:}\label{las-caracteruxedsticas-de-una-sociedad-en-desarrollo-son}}

Las características de una sociedad en desarrollo abarcan diferentes
aspectos que reflejan su situación socioeconómica. Entre estas
características se encuentran:

\begin{enumerate}
\def\labelenumi{\arabic{enumi}.}
\item
  Países pobres: Estos países presentan niveles de desarrollo económico
  y social bajos, con altas tasas de pobreza y dificultades para
  satisfacer las necesidades básicas de su población.
\item
  Países en vías de desarrollo: Se trata de naciones que se encuentran
  en un proceso de transición hacia el desarrollo, experimentando
  cambios y mejoras en su economía y calidad de vida, pero aún
  enfrentando desafíos significativos.
\item
  Países dependientes: Estos países tienen una alta dependencia de otras
  naciones en términos económicos, políticos o culturales. Pueden
  depender en gran medida de la exportación de materias primas o tener
  una fuerte influencia de actores externos en sus decisiones y
  políticas.
\item
  Países no industrializados: Son naciones que carecen de una base
  industrial sólida, dependiendo en su mayoría de actividades económicas
  primarias, como la agricultura, la minería o la extracción de recursos
  naturales.
\item
  Países emergentes: Son aquellos países que han experimentado un rápido
  crecimiento económico y han logrado avances significativos en varios
  aspectos, como la reducción de la pobreza, el aumento de la inversión
  extranjera y la mejora de los indicadores sociales.
\end{enumerate}

Las sociedades en desarrollo presentan una serie de desafíos y
características que contribuyen a su situación actual. Estos incluyen:

\begin{itemize}
\tightlist
\item
  Altas tasas de crecimiento poblacional: Esto puede ejercer presión
  sobre los recursos y los sistemas de bienestar social de un país.
\item
  Dependencia de la agricultura: Una gran proporción de la mano de obra
  empleada en el sector agrícola limita la diversificación económica y
  puede generar problemas de empleo y desarrollo.
\item
  Niveles de ingreso y pobreza: La existencia de altos niveles de
  pobreza y escaso o nulo ahorro dificulta el crecimiento económico
  sostenible.
\item
  Paro encubierto masivo (subempleo): La falta de oportunidades de
  empleo adecuadas y la presencia de subempleo son características
  comunes en las sociedades en desarrollo.
\item
  Dependencia de productos de exportación: La economía de estos países
  puede depender en gran medida de uno o unos pocos productos de
  exportación, lo que los hace vulnerables a las fluctuaciones en los
  precios internacionales.
\item
  Desigualdades de riqueza: Existen grandes disparidades entre los
  segmentos más ricos y más pobres de la sociedad, lo que contribuye a
  la desigualdad social y económica.
\item
  Control gubernamental: En algunos casos, el control del gobierno está
  en manos de una minoría rica que se opone a cambios que puedan
  perjudicar sus intereses, lo que puede dificultar el progreso y la
  implementación de políticas de desarrollo equitativas.
\end{itemize}

Estas características y desafíos reflejan la complejidad de las
sociedades en desarrollo y su posición relativa en comparación con otros
países. Estos países suelen tener potencialidades productivas
desaprovechadas y dependencia en diversos aspectos, como la economía, la
cultura, la política y la tecnología. Superar estos desafíos y lograr un
desarrollo sostenible requiere de esfuerzos integrales y políticas
adecuadas que promuevan la equidad, la diversificación económica y la
inclusión social.

Al caracterizar una sociedad, es importante reconocer que existe una
posición ideológica que influye en la interpretación de los factores que
nos permiten afirmar dichas características. Estas caracterizaciones
tienen connotaciones, sentido y una naturaleza ideológica inherente.

En los países con problemas de desarrollo, se manifiestan desigualdades
enormes entre ricos y pobres. Estas desigualdades constituyen un
conjunto complejo de problemas que se traducen y se manifiestan en una
brecha significativa de riqueza y pobreza.

Asimismo, estas economías suelen experimentar estancamientos, es decir,
avanzan en ciertos aspectos y luego se quedan estancadas sin lograr un
progreso sostenido.

En comparación con otros países, nos encontramos rezagados y atrasados
en términos de potencialidades productivas desaprovechadas. Además,
dependemos de estas economías en diversos aspectos, como lo económico,
cultural, político y tecnológico.

\hypertarget{caracteruxedsticas-de-los-pauxedses-con-problemas-de-desarollo}{%
\subsection{Características de los países con problemas de
desarollo}\label{caracteruxedsticas-de-los-pauxedses-con-problemas-de-desarollo}}

En los países con problemas de desarrollo, como el nuestro, existen
características que influyen en nuestra situación económica y social.
Algunas de estas características son las siguientes:

\begin{enumerate}
\def\labelenumi{\arabic{enumi}.}
\item
  Niveles de ingresos y pobreza elevados: En nuestro país, los niveles
  de ingresos son muy bajos para la mayoría de la población, lo que
  genera altos índices de pobreza. Esta situación dificulta la capacidad
  de generar ahorros internos, ya que la prioridad es cubrir las
  necesidades básicas.
\item
  Paro encubierto masivo (subempleo): El subempleo es una forma de
  trabajo informal que prevalece en nuestra sociedad. Muchas personas no
  encuentran empleos formales o bien remunerados, lo que limita sus
  oportunidades de desarrollo económico y social.
\item
  Dependencia de productos de exportación: Nuestra economía está
  \textbf{altamente dependiente de un número reducido de productos que
  exportamos}. En particular, la minería y la pesca son fuentes
  importantes de ingresos, pero esto nos expone a riesgos y volatilidad
  en los mercados internacionales.
\item
  Escasa capacidad de ahorro: Debido a los bajos niveles de ingresos y
  la falta de oportunidades de empleo digno, \textbf{la capacidad de
  ahorro en los países en desarrollo como el nuestro es limitada o
  incluso inexistente.} Esto dificulta la acumulación de capital
  necesario para impulsar el desarrollo económico sostenible.
\end{enumerate}

En este contexto, es crucial que se priorice la distribución del ingreso
de manera más equitativa para reducir la brecha entre los sectores más
desfavorecidos y aquellos con mayores recursos. Una distribución más
justa del ingreso puede contribuir a mejorar las condiciones de vida de
la población y fomentar un crecimiento económico más inclusivo y
sostenible.

La brecha existente entre los pobres y los ricos, tanto a nivel nacional
como internacional, es una realidad que caracteriza a muchas sociedades.

Cuando analizamos un país como \textbf{subdesarrollado}, nos referimos a
la situación estructural e institucional en la que se desenvuelve esa
sociedad.

Por otro lado, cuando hablamos de un **país en vías de desarrollo, nos
referimos al desaprovechamiento del potencial productivo que existe en
las economías con problemas de desarrollo. En estos casos, los recursos
disponibles no se utilizan de manera adecuada, lo que impide su pleno
funcionamiento y producción.

Es importante reconocer que, tecnológicamente, nos enfrentamos a
limitaciones para sustituir muchos productos, lo que nos hace depender
de otros países en ese aspecto. Además, políticamente, tendemos a imitar
modelos económicos extranjeros sin adaptarlos a nuestra realidad, lo que
también contribuye a nuestra dependencia de otros.

Cuando caracterizamos como ``países no industrializados'' pensamos en la
presencia de industrias en el país y para nosotros la principal variable
es el ``desarrollo industrial'' de modo que luego de ello el resto de
las actividades podrían complementarse.

El eje de análisis es la industria y recalcamos que hay una posición
ideológica y que rápidamente vamos a notar esa connotación ideológica,
el sentido que le da a la economía y a la naturaleza misma es el tipo de
gobierno que podemos tener. ``Los países con problemas de desarrollo''
tienen un conjunto complejo, interrelacionado de problemas que afrontan
y se trasladan y expresan en desigualdades notorias.

Tenemos problemas de estancamiento, muchas de estas sociedades pueden
estar en un proceso de avance y por algún motivo se estancan, frenan su
crecimiento y las posibilidades de desarrollo y entonces pierden el
ritmo de mejoras. A nivel del PBI comenzamos a estancarnos, a crecer muy
poco de lo que se estaba creciendo aceleradamente.

Vamos a retrasarnos en relación a otros países, vamos a hacer cosas
incorrectas que nos va permitir avanzar, ir en el camino de la reducción
de la pobreza o acelerar e incrementar el crecimiento de la riqueza del
país (PBI) Las ``potencialidades productivas'' van a ser permanentemente
desaprovechadas por la connotación política ideológica en el sentido de
la naturaleza de los componentes de la sociedad.

Como las medidas del Estado a través del Gobierno no van a ser
coherentes, sino que estas van a superar estos indicadores que
mencionamos.

EJEMPLO: Reducción de la pobreza, promedios altos del crecimiento del
PBI y esto va hacer que nuestro país pierda velocidad ya que cuanto más
es el tiempo, más retrasados vamos a estar en relación al que va a mayor
velocidad.

Es necesario que pensemos que debemos utilizar los recursos con los que
contamos, de modo que pueda servir para superar los problemas que
tenemos.

Sabemos que la ``dependencia económica'' que tenemos es enorme, que casi
ninguna actividad tiene el nivel de autonomía, Ejemplo: agronomía, que
para su producción se necesita de abono e insecticidas que son
tecnologías externas.

Tenemos una dependencia cultural porque imitamos todo, tenemos
``dependencia en la política y en la tecnología'', la aspiración que
tenemos es tener una política redistributiva más equitativa. La igualad
no es posible, muchas economías se han derrumbado porque han querido
llegar a la igualdad.

Hay ``altas tasas de crecimiento en la población'' siempre hay que
basarnos en la información que tenemos, Ej.: década de los 80, la
demografía alta es un problema en los países en vías de desarrollo.

En estos tiempos las economías, las sociedades, países no son
importantes solo porque no crezcan a la velocidad más alta que los
otros, sino que también albergan mucha población.

EJEMPLO: EEUU no tiene mucha población (350 millones aprox.) ya que son
voraces consumidores, tienen niveles de ingresos altos, son importantes
porque son grandes consumidores.

CHINA: son 3 o 4 veces más grandes que EEU, pero se han hecho
importantes en el mundo por su población a consumir, y si fueran más
voraces que EEUU serían más importantes.

Tenemos que pensar en el futuro, no solo reducir el ``crecimiento
poblacional'' sino que también debe de haber una política de crecimiento
poblacional a la inversa de lo que sigue siendo.

En el Perú somos 32 millones pero si tuviéramos niveles de ingreso mucho
más altos seriamos mucho más productivos y nuestro mercado seria
atractivo, EJ.: Chile, con una población de aprox. 18 millones producen
un poco más que nosotros, significa que la mano de obra de Chile es
mucho más productivo que la del Perú.

Cuando hablamos de ``productividad'' lo relacionamos con la mano de
obra.

Cuando hablamos de rendimientos lo vamos a relacionar al capital
financiero o tierra y también bienes de capital en cuanto a maquinarias
y equipos.

Tenemos que trabajar en una política poblacional (demográfica) en el
Perú lo cual hemos descuidado.

Otro indicador para caracterizar a una sociedad menos desarrollada o
``con problemas de desarrollo'' es que hay grandes proporciones de mano
de obra empleada en el sector agrario (área rural)

Ej.: se está dando una escasez de mano de obra en el sector agrario y
esto hay que sustituir y principalmente se está dando la escasez de mano
de obra en el eje de la Sierra, en la Costa hay mano de obra disponible
a los salarios que ofrecen las empresas.

Inclusive los salarios en la Sierra son más altos que en la costa, pero
sin embargo son menos productivos que la mano de obra de la costa, eso
se debe a que el trabajo en la costa es permanente (por eso los
ayacuchanos inmigran).Ica está lleno de trabajadores ayacuchanos.

Aquí vamos a tener un dilema, Ej:En EEUU, el 7\% de la población esta
dedica al Agro, que puede inundar con productos de origen agropecuario
al mundo exterior (indicador que tenemos que tener en cuenta).

Los niveles de ingreso en realidad son de pobreza en países como el
nuestro y esta situación no permite ahorrar suficiente como para
financiar el crecimiento o la capitalización del país, entonces el
ahorro es muy escaso y hasta a veces nula.

Entonces siempre vamos a depender de capitales externos, somos
deficientes de capital y es por eso que permanentemente tenemos que
crear las condiciones favorables para la instalación o desarrollo
empresarial.

Tenemos que hacer que sea fácil la creación de una empresa y que esta
empresa no muera, sino que quede en el tiempo y se desarrolle, pero
sabemos por los datos de la SUNAT que el 80\% de las empresas que se
crean dentro de un año mueren ese mismo año y el 20\% que sería el 100\%
que sobreviven el primer año, el 60\% mueren el tercer año.

Entonces vemos una realidad que de 100 empresas que nacen con entusiasmo
y sacrificio ya sean pequeñas o grandes desaparecen del mercado el 97\%.

Esto nos dice que economías como la nuestra son de alto riesgo ya que
hay mucha inestabilidad ya que las reglas del juego cambian
permanentemente, estos proyectos no se cumplen porque los supuestos con
los que se trabajan no se van a cumplir porque van cambiando.

En vez de crear las condiciones favorables, el estado se convierte en
enemigo de las empresas.

Otro factor a tomar en consideración es que existe un nivel de desempleo
y sub empleo bastante alto y que esta encubierto por los puestos de
trabajo que el mismo trabajador genera (empleo informal).

La informalidad es un indicador clave para determinar en qué tipo de
sociedad nos encontramos cuando esta comienza a formalizarse,
lógicamente es una aspiración de todas las sociedades, no se trata solo
de los impuestos, muchos de nosotros tomamos la informalidad como
sinónimo de evadir impuestos.

Otro indicador de países como los nuestros es que somos países que
tenemos una ``gran dependencia de los pocos productos que exportamos''
eJ.: Perú depende de la minería, ya que esta genera grandes ingresos,
ahora nuestro presupuesto es gasolinera, minero y además pesquero, y
agrícola.

Pero la minería es lo que está sosteniendo el presupuesto nacional
porque el agro se ha resentido, la pesca ha perdido control del mar,
nuestro presupuesto depende de estas 3 actividades, adicionalmente de la
gasolina y la cerveza, tenemos un presupuesto de grandes velocidades.

Cuando analizamos es necesario ver la producción y el efecto espejo del
presupuesto de la Republica.

Qué porcentaje del PBI va al presupuesto de la Republica, el presupuesto
del 2021 está financiado con endeudamiento público.

Otra característica que finalmente debemos ver tiene que ver con la
representación o toma de decisiones que está en realidad en manos de
poca gente que se ha enriquecido, en manos de una minoría rica se ha
puesto al gobierno y estos no quieren ningún tipo de cambio, se oponen
al cambio y no permiten mejorar los indicadores que podríamos generar en
mejores condiciones para ayudar a la población más necesitada.

\hypertarget{clasificamos-en-3-sectores-a-la-economuxeda}{%
\section{CLASIFICAMOS EN 3 SECTORES A LA
ECONOMÍA}\label{clasificamos-en-3-sectores-a-la-economuxeda}}

Debemos saber lo importantes que son las actividades que se realizan en
la sociedad.

SECTOR PRIMARIO: Producción básica (agricultura, ganadería, pesca,
minería, explotación forestal) que están relacionados con los recursos
naturales y su extracción.

SECTOR SECUNDARIO: Producción de bienes (Industria, construcción,
manufacturas) que parten del sector primario, Ej.: cemento, telas\ldots.
TRANSFORMACIÓN

SECTOR TERCIARIO: Producción de servicios. Ej.: Bancos, educación,
comercio, cultura, servicios a domicilio.

Los países más ``exitosos'' y avanzados y con un alto grado de
desarrollo son aquellos países que están ``principalmente en la esfera
de los servicios'' luego vienen la producción de bienes y luego países
como el nuestro que están en el sector primario, estamos en la esfera
extractiva, no hacemos mayor transformación, no procesamos.

Hacemos economía por dos causas:

\begin{itemize}
\tightlist
\item
  Por necesidad, sin economía no existiría provisión ( ya que sin
  economía no hay provisión de alimentos y ropa)
\item
  Por el deseo humano de actividades creativas y productivas para dar
  sentido a su vida.
\end{itemize}

\hypertarget{a-quuxe9-se-debe-que-no-trasnformamos-los-minerales-cobre}{%
\subsection{¿A QUÉ SE DEBE QUE NO TRASNFORMAMOS LOS MINERALES
(cobre)?}\label{a-quuxe9-se-debe-que-no-trasnformamos-los-minerales-cobre}}

Por falta de tecnologías, no tenemos producción a gran escala Por los
altos grados de inversión que requieren

Ej.: La Empresa Altamina (Minería) ha alcanzado la tecnología más grande
del mundo (sector primario) y aquí los costos de producción son
reducidos porque son eficientes.

Nuestro nivel de exigencia para la explotación minera debe ser minimizar
los niveles de contaminación y este es el aspecto tecnológico.

Niveles de Inversión: Las empresas mineras necesitan bastante capital
para su puesta en marcha y las empresas peruanas con excepción de 2
empresas necesitan de inversiones externas, porque los niveles de ahorro
son bajos o incluso nulos y esto hace que importemos capital ya que
ningún inversionista o empresario invierte para perder.

Si pasamos al sector secundario habría empresas en el Perú que financien
a estas trasformaciones en el sector secundario y lógicamente tendríamos
capital suficiente.

A través de la banca llegarían las inversiones, llegaría también la
carga financiera para el sector secundario y por eso mismo es muy
difícil entrar al sector terciario, al sector de los servicios.

\hypertarget{la-globalizaciuxf3n-y-las-economuxedas-regionales}{%
\section{LA GLOBALIZACIÓN Y LAS ECONOMÍAS
REGIONALES}\label{la-globalizaciuxf3n-y-las-economuxedas-regionales}}

Entre el gobierno y la empresa debe estar la ciencia, la producción
científica.

Cuando hablamos de globalización, hablamos de la ``interdependencia
económica creciente'' entre todos los países del mundo, provocado por el
incremento del volumen y la variedad de comercio que se dan de bienes y
servicios a nivel internacional.

También hay ``flujos internacionales de capital'' además de la difusión
acelerada de la información y la generalización de la tecnología.

Interdependencia económica, participan todos países del mundo, las
transacciones de bienes y servicios es internacional y el flujo de
capital es permanente en el mundo, la difusión, la comunicación
acelerada y de forma inmediata y la tecnología que se a estandarizado en
el mundo.

De modo que cuando hablamos de difusión acelerada hablamos de una
situación que nos permite tener cuidado con el recurso tiempo.

El actor principal de estos procesos de globalización son los Estados,
llamados ``Estado Nación''.

Decimos que el mundo es una fábrica global, China es el taller del
mundo, la sociedad ya no es local sino global. Ya no somos pobladores
del País.

La globalización no significa que estemos en igualdad de condiciones
todas las economías.

Las grandes economías están subordinando a los países más débiles, los
ponen en posición de apoyo o de servicio a sus intereses, no hay
igualdad.

Las relaciones internacionales son transversales y fluidas y eso no
significa que estemos en igualdad de condiciones.

Los países hegemónicos que encabezan la conducción del mundo han creado
reglas para subordinar a los países pequeños y con problemas de
desarrollo. En sociedades globalizadas, las comunicaciones juegan un
papel importante Y preponderante.

Los medios de comunicación rompen cualquier barrera que se pueda
presentar y favorecen la transmisión de valores comunes con la cual se
tratan de ``homogenizar'' y tienden a hacer que aparezcan culturas de
masas que se asemejan, homogenizan y se estandarizan de modo que esa
tendencia tenga mecanismos de defensa de tratar de desarrollar o tratar
de volver a recuperar algún nivel de autonomía cultural.

Por lo que la virtualización de la sociedad, la estilización de la
realidad está reflejando los usos y costumbres que se imponen en el
mundo, así como también las empresas extienden sus dominios en el mundo.

Podemos decir que ninguna actividad del hombre escapa del proceso de
globalización.

El principal problema de una globalización unilateral es que existe una
dependencia creada, en casos extremos, estas dependencias podrían
ingresar a un agujero sin salida en caso de distribución y problemas
sociales que están en contra del deseo general de estabilidad, observado
esto a fin de que esto no desestabilice la economía de economías que
tienen problemas de desarrollo pero que son fundamentales porque son los
generadores del sector primario (aquellas que generan insumos y materias
primas principalmente).

Los países que más se han desarrollados, son aquellas que se han
desarrollado en el sector terciario con mayor eficiencia y mayor rapidez
antes que aquellas economías que están en el sector primario.

Las alternativas que podemos proponer son:

Propuestas que se pueden viabilizar e implementar, que sean posibles sin
que eso signifique aislarse del proceso de globalización.

Puede presentarse el riesgo de que un sistema económico orientado
únicamente al mercado mundial pierda de vista las necesidades de la
población. Ósea que tenemos que comenzar con una economía abierta, pero
no con actividades totalmente hacia el exterior ya que descuidaríamos el
consumo interno porque todo lo que producimos estaría saliendo hacia el
exterior, esto también vendría a ser un riesgo porque la población
entraría en una conmoción social por temas de distribución y esto
generaría inestabilidad a los países desarrollados ya que hay una
interdependencia porque pondrían en riesgo el desarrollo industrial por
la falta de provisión.

Si bien es cierto dependemos de los países occidentales, ellos también
dependen de estas pequeñas economías (países con problemas de
desarrollo) y ahí es donde tenemos la oportunidad de aprovechar los
efectos de los precios y materias primas.

EJEMPLO: El Perú tiene oportunidades a un futuro mediano porque la venta
del cobre está en su precio tope (alto).

La globalización que es el capitalismo se ha vuelto global, el que
organiza y restructura el mundo a su conveniencia (a su favor) y tienen
los instrumentos para condicionar a las economías que queremos salir de
estos problemas de desarrollo, queremos ser emergentes y puede ser que
acondicionen los instrumentos a su favor tanto internacionalmente como
al interior porque estos tienen diversas empresas en el mundo entero,
tienen sus capitales en actividades estratégicas. Ej.: En el Perú hay
mucho capital chileno (empresas chilenas).

Esto hace ver las contradicciones que se encuentran al interior de la
globalización, Ej.: Problemas de fuerte centralización de capitales por
el afán de aumentar la cobertura de capital puede hacerlo ineficiente
porque en unas economías puede ser más rentable que otras, no pueden
homogenizar la estructura organizacional de estas grandes empresas
porque puede estar centralizándose, o la importancia de la tecnología
para la agilización en la producción, puede estar imposibilitada por la
falta de las condiciones de los obreros, si bien es cierto hay mano de
obra barata en países en desarrollo, las empresas que se expanden como
efecto del proceso de globalización pueden incorporar en sus empresas
mejoras tecnologías, pero estas también se ven frenadas porque la mano
de obra local, las condiciones de los obreros para estas empresas no son
las esperadas para el manejo de estas tecnologías (maquinas).

Por un lado, los países menos desarrollados necesitamos la presencia
tecnológica y por otro lado las grandes economías (grandes empresas)
quieren imponer mejoras tecnológicas en los países menos desarrollados,
pero no tenemos la mano de obra adecuada para implementar este cambio o
mejora tecnológica al interior de las economías con problemas de
desarrollo.

Se puede ver que en ahí hay una necesidad de mejorar el nivel educativo
adaptándose para el manejo de las mejoras tecnológicas (hay una
paradoja) ya que al empresario le conviene una mano de obra altamente
calificada.

¿Cuáles son las necesidades de la mano de obra? Mano de obra altamente
calificada Altamente capacitado para ese bien o servicio que está
produciendo (no satisface las expectativas).

Este hecho a nuestra escala (nuestro desarrollo) en el ámbito local, en
el ámbito geográfico relativamente pequeño en el que nos movemos.

Esto debemos interpretarlo como una forma de organización en la parte
económica del hombre orientado al ``Desarrollo Regional''.

Este ámbito en el que nos movemos, basamos la mayor parte de nuestra
vida, en la que vivimos, trabajamos, desenvolvemos, por lo que debemos
de organizarnos de una forma más pequeña.

La globalización genera una lógica que tiende a disminuir las autonomías
a aumentar las interdependencias, a acrecentar la fragmentación de las
unidades territoriales, a producir marginación de algunas zonas.

\hypertarget{economia-regional-dentro-del-proceso-de-globalizaciuxf3n}{%
\section{ECONOMIA REGIONAL DENTRO DEL PROCESO DE
GLOBALIZACIÓN}\label{economia-regional-dentro-del-proceso-de-globalizaciuxf3n}}

La globalización es transversal y fluye

Hay un ámbito en el cual el hombre necesita tener una organización más
pequeña (hecha a su medida) que es una economía orientada regionalmente.

Hacer e implementar una economía regional es complementario a la
economía global (parte del proceso de globalización) ya que el hombre se
mueve dentro de un área geográfica relativamente pequeño cubriendo sus
necesidades, eso no significa que las actividades económicas sean solo a
nivel internacional, es aquí donde aparece la Economía Regional como una
posibilidad de mejora a la economía internacional a la economía
globalizada.

Podemos decir que la economía Regional es parte de la globalización, la
Economía Regional no se puede oponer a la globalización (no depende de
nosotros, la globalización está en todas partes, está en todas las
actividades que desarrollamos).

La meta de una economía regional no es oponerse en competencia con el
proceso de globalización.

Podemos hablar de competir con exportaciones de bienes en el mundo, pero
eso no significa que seamos los únicos que conocemos ese producto, ya
que hay otras economías que llegan al mercado porque tienen costos más
bajos, productos de alta calidad cumpliendo con los requisitos exigidos
por el consumidor internacional.

Esto significa que tampoco debemos descuidar el mercado local, *Ya que
si pensamos en la globalización, nos dedicamos a producir todo para la
exportación y el mercado local está abandonado y ahí podemos tener
problemas porque estaríamos desabasteciendo la economía Regional y es
por eso que tenemos que organizarnos para satisfacer las necesidades en
todo sentido a nivel regional, no serán autosuficientes pero pueden
generar la suficiente riqueza como para adquirir lo necesario y cubrir
las necesidades regionales.

Meta es alcanzar recursos óptimos para la población que está en áreas
geográficas y que las personas tengan la posibilidad de conseguir con el
que hacer propio, el sustento propio de su familia, alcanzar en alguna
medida el bienestar de la población a nivel regional y estos son
pequeños ámbitos geográficos en los que nos desenvolvemos.

Por eso tenemos que mirar con especial atención las actividades
económicas que tenemos en la localidad, cómo podemos hacer que estas
crezcan, que mejoren su condición y que estén en la capacidad de
abastecer el mercado local en igual o mejores condiciones que las
empresas externas, deben de hacer incentivos para innovar más y mejor.

Y qué debemos de hacer para que en este mundo globalizado utilizando la
tecnología del celular podamos explotar la información para la
producción local.

Por medio del internet podemos ingresar al mercado global. Los que toman
las decisiones en estas economías Regionales, en los ámbitos geográficos
más pequeños del mundo, así como la mano de obra no está capacitada para
el manejo de la tecnología nueva que se incorpora al proceso de
producción, también la toma de decisiones de los gobernantes en estos
pequeños ámbitos geográficos no están a nivel de la exigencia para
generar estos cambios, no está preparado el nivel de exigencia para
generar estos cambios en los ámbitos geográficos relativamente más
pequeños que podrían generar mejores condiciones de vida para su
población. Esto es elemental, es básico.

SEGUNDA PERCEPCIÓN: percepción de la disociación de la sociedad entre
una cierta forma de racionalización instrumental de la sociedad
industrial con las tecnologías al interior de las economías que quieren
alcanzar algún nivel de desarrollo autónomo.

Tenemos que ver como tenemos que respetar, homogenizar algún nivel de
estándar de producción, debemos de tener cuidado con las entidades,
etnias (organizaciones ancestrales) que nos permitan pensar en el
desarrollo regional.

Las organizaciones deben de reducir las tenciones culturales al interior
creadas con argumentos ancestrales en muchos.

Ejemplo: Ayacucho- Huanta rivalidad por lo que deberíamos aprovechar
esas diferencias para reducir tenciones y tener la oportunidad de
desarrollarnos juntos uniendo fuerzas, construir el reto no está en las
tendencias que se deben uniformizar, el reto está en construir la unión
en las diferencias, en nuestras diferencias estaría el capital, la
riqueza para construir mejoras en nuestra región, en nuestro país, pero
sin embargo esas diferencias las profundizamos en lugar de
aprovecharlas, no podemos pensar que las diferencias pueden
debilitarnos, tiene que servir como parte de nuestra fortaleza y eso
hace que las regiones mejoren.

Aquí está la fuerza del proceso de regionalización, de cómo nos unimos
para mejorar las condiciones en las cuales nos desenvolvemos.

Podemos decir que no es posible oponerse al sistema global por lo que
tenemos que aprovechar y crear las condiciones para que la presencia de
la globalización en sus diversas dimensiones sirvan al desarrollo
Regional, como la dimensión tecnológica, económica, cultural, política,
institucional, pero es necesario en estas dimensiones que fortalezcamos
estas acciones para que canalicemos a favor de las Regiones el efecto de
la globalización, la tecnología debemos de asimilarla, preparar a las
personas para que manejen lo que ingresa de tecnología, en la economía
hay que tener iniciativa propia, hay que ver los nichos para nuestros
productos, en la cultura aprovechar actividades como el turismo en la
que tenemos que vender como es la presencia cultural del Perú diverso,
en lo político ser mucho más sólidos antes que divisiones, tener ideas
sólidas para mantener la continuidad antes que hacer cortes y volver a
comenzar de cero.

Las instituciones tienen que ser mejores y para eso hay que restructurar
y organizar el Sector Publico, el aparato estatal y el aparato privado
para la producción también tienen que cambiar su estructura porque no
puede ser que las empresas hayan hecho las reglas de juego en nuestro
país.

Sabemos que el proceso de globalización hace que las grandes empresas
puedan crear todo el instrumental necesario a su favor, pero para eso
estamos las economías Regionales como país para unirnos o para crear las
facilidades necesarias para mejorar las condiciones de negociación,
también en el ámbito ambiental incorporar criterios propios y locales y
no hacer que se lleven absolutamente todo, Ejemplo: parte forestal de la
selva.

La globalización disminuye la autonomía de nuestras regiones, aumenta
las interdependencias (las grandes economías dependen de nosotros) hay
que mejorar ese fraccionamiento territorial que tenemos haciendo que se
produzca más y mejor, evitando la marginación, respetando los usos y
costumbres de los antepasados.

Podríamos decir que hay algunas formas de situarse frente a la relación
entre el proceso global y el proceso local, como:

\begin{itemize}
\tightlist
\item
  El desarrollo regional y local es una alternativa a los males que
  pueda generar el proceso de globalización a las economías regionales,
  (efectos negativos de la globalización.)
\item
  El proceso de globalización es determinante sobre el desarrollo
  Regional o local
\item
  Es necesario priorizar la articulación de lo regional con lo global al
  interior de cada economía regional comprendiendo la complejidad que
  significa crear las condiciones ante la globalización; es difícil,
  pero tenemos que estar preparados para que la región no enfrente sino
  se articule, se fusione y aproveche la presencia del proceso de
  globalización a favor de las diversas actividades que se desenvuelven
  en la región.
\end{itemize}

Estas son las tres formas de enfrentar a la globalización desde el
ángulo local o Regional.

La economía regional se debe entender como complementaria a la economía
global, a los problemas de una orientación solamente global, hay que
poner el grado de dependencia y los riesgos de inestabilidad.

Por lo contrario una economía regional pone una exigencia de una
independencia económica, de un auto provisión porque se va desarrollar
en su interior, mientras que el proceso de globalización es inestable,
de alto riesgo, de alto nivel de dependencia.

Si hay una organización regional bastante coherente y bien articulado
vamos a reducir esa dependencia y vamos a entrar a un proceso de algún
nivel de independencia economía y vamos a tener la posibilidad de
proveer a nuestra economía regional al menos en una gran parte de sus
necesidades, entones la orientación de una economía descentralizada va
cubrir las necesidades de la población local y ahí está la importancia
del proceso de descentralización de un país.

El proceso de regionalización es un proceso de descentralización, porque
cuando un país es centralizado la dependencia se hace mucho más fuerte y
la presencia del proceso de globalización es mayor, porque dependemos de
la toma de decisiones de autoridades o entes gubernamentales que no son
cercanos a las necesidades de estas pequeñas áreas geográficas en las
cuales nos desenvolvemos.

Ejemplo: el sector salud depende del Gobierno Regional\ldots\ldots.
¿Quién manda en la dirección regional de salud?

El proceso de globalización nace con el hombre.

Las economías orientadas regionalmente, ósea las economías
descentralizadas con miras a mejorar las economías regionales su
característica principal es que la producción está más cerca al
consumidor, el productor está más identificado con el consumidor,
entonces la producción es más transparente para el consumidor, sabe que
el papero se sacrifica mucho para llegar al mercado y comienza a
identificarse con el productor.

Globalmente se puede decir que está mucho más relacionado con los
derechos humanos, mucho más consiente de los procesos más sanos para el
medio ambiente.

El productor está cerca del consumidor y el consumir se identifica con
el productor y sabe cómo es el proceso de producción a eso se refiere
cuando se dice que se transparenta la producción con el consumidor.

Incluso cuando se habla de agricultura digital, se habla de la
transparencia en los costos, de cuán importante se vería en una economía
regional porque es mucho más cercano, la información que debe de manejar
el productor debe ser la que maneja el consumidor, sabe que está pagando
un precio adecuado a las necesidades del productor y no al revés.

También podemos decir que la economía regional está orientada a que la
vivienda, el trabajo la vida, la cultura, la educación, los centros
laborales se acercan más a la familia, se acerca más a las necesidades
de la población, se acorta el transporte, el tiempo necesario para la
producción, se ahorran las emisiones contaminantes que aún se tienen,
ejemplo: la gente que quema, se tiene que ir reduciendo y mejorando esos
procesos de malas costumbres.

La dependencia de los desarrollos globales y nacionales disminuyen con
la complementariedad de las economías regionales, se reducen los grados
de dependencia, los recursos locales se prefieren. Cuando se imponen las
cuestiones ideológicas las cosas cambian un poco. Tenemos que pensar
como acercar esas diferencias.

En el aspecto político globalmente sabemos que hay una creciente
comunicación e interdependencia entre todos los países del mundo y se
puede notar la dependencia que tiene una economía del otro. Pero esto es
peligroso porque en todo caso se hay que pensar en unificar los mercados
y reducir en algo esta dependencia porque podemos ser parte de su
mercado, si son deficitarios en algunos productos esa es la oportunidad
que tenemos que aprovechar, tenemos que incorporar a nuestras
sociedades, como la globalización en costumbres en la cultura, debemos
hacer transformaciones sociales en las que tenemos que adaptarnos y
aceptar que un determinado tipo de sociedad se está imponiendo y permite
que países se desarrollen y mejoren su bienestar, observar el aspecto
político en los diferentes países que existen, el modelo de carácter
global que tienen éxito en el mundo, también aceptar que los modos de
producción o movimientos de capital a gran escala planetaria impulsados
por los países más avanzados son los que predominan y que los
componentes originarios como el nuestro, que somos productores primarios
si bien es cierto seguimos siendo menos desarrollados pero no por eso
dejamos de ser importantes porque sin los recursos que nosotros
proveemos no tendrían la industria y esa industria es la que genera los
servicios, entonces hay que incorporarse a los estándares universalmente
aceptados.

En la práctica necesita esfuerzo y mucha concientización de la población
para participar con sacrificio y alcanzar las metas que se pueda
proponer a nivel del gobierno y para eso necesitamos que los que toman
las decisiones de los gobiernos tanto nacionales, regionales y locales
para el desarrollo regional que podamos considera al interior de los
departamentos y a nivel nacional como américa latina tenemos que
seleccionar y mejorar las condiciones y que el gobierno central o el
gobierno nacional tenga que tener la confianza de la población que está
por encima de cualquier voluntad personal que podamos tener.

No debemos de perder las atribuciones que debe tener el gobierno central
para con la población y buscar alternativas de solución y la
descentralización tiene que convencernos.

RESUMEN

Hablamos de cómo el proceso de globalización y el cómo es que el
desarrollo regional puede servir para mejorar las condiciones del nivel
de vida de la población, considerando de que el problema de
globalización es determinante en los problemas que afronta las regiones
y sobre lo regional es determinante, no podemos frenarlo, no podemos
parar, es más el desarrollo local y regional debemos verlo como una
alternativa a o todos los males con los cuales podría llegar la
globalización, los efectos negativos que podrían llegar con la
globalización y finalmente hacer que lo regional con lo global se
fusionen, sean complementarios ya que esta comprensión al interior entre
lo local y global es tan compleja que tenemos que aprovechar la
velocidad de integración con la cual se da el proceso de globalización,
ahí está la habilidad de las regiones.

El asunto aquí es que unos países más que otros se integraron primero al
proceso de globalización, Ej.: economías sudamericanas como chile
estaban metidos en el mundo de la globalización, su economía se había
abierto, se estaba aplicando una economía de libre mercado, mientras
nosotros íbamos en sentido contrario, cerramos fronteras no pagábamos la
deuda, por lo que estas economías que se integraron primer y abrieron
sus fronteras tienen más ventajas que nosotros.

El proceso de Regionalización es dañino en tanto no estemos preparados
para enfrentar el embate del exterior, pero se hace positivo en tanto
estemos preparados y aprovechemos las oportunidades que nos brinda el
sector externo. Ej.: En el sector agropecuario tenemos ingentes recursos
que nunca lo hemos explotado y nunca hemos exportado, recursos que
todavía no conoce el exterior como el aguaymanto, tenemos que aprovechar
que como efecto de la globalización ingresa la tecnología.

La globalización es una oportunidad única para el proceso de
regionalización, la economía regional es complementaria a la economía
global y que hay tres formas de afrontarlo y que esta es la única manera
en la que la economía regional puede ser complementaria de modo que la
orientación del proceso de globalización en la que hay un
acondicionamiento de los países occidentales a los países menos
desarrollados sea menor porque tendríamos la capacidad de producir
muchos bienes o autoabastecernos de muchos productos que pueda
garantizar la seguridad alimentaria absorbiendo los recursos del
exterior, una auto provisión a fin de que reduzcamos la dependencia.

Nuestra independencia está en la estructura económica descentralizada
que debe de tener el país, ósea que no solo los esfuerzos de la economía
nacional se centren en Lima sino al interior del país y que cada una de
estas regiones deben de estar preparada para recibir los recursos del
gobierno central, priorizar los proyectos con los que cuentan y
dinamizar su desarrollo, pero en realidad vemos que esto no es así, no
estamos preparados para administrar nuestros destinos. (Ej: no
priorizamos los proyectos de inversión, los proyectos se están
postergando y si los proyectos se postergan el desarrollo de nuestra
región se posterga, debido a que nuestras autoridades no están
preparados para este manejo gubernamental).

la regionalización nos permite orientar la economía y acercar al
consumidor a la producción, el proceso de producción se va acercar al
consumidor, el consumidor va saber aproximadamente como se enfrenta a
los problemas el productor , vamos a ver como la población puede buscar
la oportunidad de reducir la parte de la distribución y acceder a los
precios del productor y el productor mejorar sus precios al reducir el
ámbito de la distribución, entonces tanto el productor como el
consumidor se benefician y ahí está la llave para decir que el proceso
de regionalización o la economía regional es una alternativa
complementaria a la regionalización porque tenemos la capacidad de auto
provisión, podemos reducir la presión de la competencia mundial si
tenemos los productos al interior de la región, los embates de la
globalización, sus efectos van a ser de menor impacto.

Podemos vender la idea y crear las condiciones para que los recursos
locales sean de preferencia del consumidor local o regional y para eso
tenemos que tener los rendimientos suficientemente altos para reducir
costos.

El proceso de globalización unifica los mercados, transforma el
comportamiento social, la organización social de la población y la
política al interior y que finalmente hay una creciente dependencia de
las comunicaciones, la comunicación se hace mucho más fluida, va existir
un predominio de diferentes modos de producción sin embargo el
movimiento de los capitales que es a escala mundial va situar a nuestras
regiones en algún momento en ventaja porque las tasas de interés pueden
ser reducidas si es que tenemos el control financiero abierto, no
podemos dejar el mundo financiero abierto como hasta ahora, en las que
existan tasas de interés hasta del 200\% , 250\% tenemos que tener una
banda de control y ahí es cuando podemos discrepar, tiene que haber
algún nivel de intervención del Estado, porque sabemos que el Estado
tiene que intervenir, porque cuando ven que el estado a través del
Gobierno es débil el mundo financiero aprovecha de esas oportunidades
para tener tasas de interés muy altas aun cuando el precio del dinero es
muy bajo (el BCR presta a 0.5\% e incluso al 0.25\% pero que sin embargo
cobran tasas de interés muy altos) es por eso que se tienen que poner
bandas de tasas de interés, porque los ahorros en países como el nuestro
son relativamente bajas o nulas, por lo que necesitamos del flujo de
capitales del exterior y se aprovechan de esto los intermediarios para
lucrar con las tasas de interés como con cualquier producto escaso.

Los gobiernos han perdido control, autoridad frente a las organizaciones
internacionales porque son incapaces de intervenir por lo tanto
necesitamos gobiernos relativamente fuertes en la que lo que proponen se
haga efectivo y no quede en leyes o reglamentos, por lo que hay que
mejorar la condición del sistema público en su conjunto.

OJO:

El PBI es un indicador, es el valor de los bs y ss finales producidos
durante un año No confundamos PBI con PRESUPUESTO PÚBLICO, que son cosas
totalmente distintas

\hypertarget{el-crecimiento-y-la-reducciuxf3n-de-la-pobreza}{%
\section{EL CRECIMIENTO Y LA REDUCCIÓN DE LA
POBREZA}\label{el-crecimiento-y-la-reducciuxf3n-de-la-pobreza}}

Cuando hablamos de crecimiento económico nos referimos a un indicador
principal que es PBI, el análisis es de ¿Cómo el incremento del PBI en
las diferentes economías puede contribuir a la reducción de la pobreza?

No todo crecimiento significa desarrollo, sin embargo el crecimiento
puede contribuir a la reducción de la pobreza porque en alguna medida
puede generar la posibilidad de una mejor redistribución de la riqueza
generada en el país, el crecimiento económico puede hacer posible que el
estado a través del gobierno disponga de mayor cantidad de recursos que
le permita re direccionar o atender el gasto público o las inversiones
del sector público hacia la población más vulnerable.

En países como el nuestro existe la pobreza y pobreza extrema, la
pobreza tiene que ser una tarea que debemos de resolver y los que
tengamos la posibilidad de sumar en la lucha contra la pobreza podamos
contribuir a su solución para mejorar las condiciones en las cuales
viven gran parte de esta población.

Combina el capital humano con el financiero, porque por un lado tenemos
la mano de obra de la parte de la población pobre y pobre extremo que se
descapitaliza permanentemente, la mano de obra es ciega permanentemente
y que su productividad se hace cada vez más bajo y que sin embargo
tenemos la necesidad de financiar la lucha contra la pobreza que está
reduciendo a su mínima expresión al capital humano en productividad.

*Debemos de pensar (cómo podemos financiar) de dónde pueden salir las
finanzas para atender a este segmento de la población.

El mundo está preocupado, sin embargo las grandes organizaciones hasta
qué punto pueden tener la capacidad de desprenderse de la riqueza
acumulada para atender a este gran segmento de la población. Ej.: EEUU y
los países más grandes del mundo (países occidentalizados) están
relacionando su crecimiento con su seguridad nacional y la seguridad
nacional para occidente es de primera prioridad y ven alguna amenaza de
la población pobre que está involucrado dentro de los países menos
desarrollados, es por eso que antes que por factores morales dieron una
iniciativa para apoyar a los países menos desarrollados.

Moralmente el mundo está en la obligación de resolver este problema Pero
la moral de los países ricos no está en atender las necesidades de la
pobreza, en las necesidades de sus trabajadores, y si no está en la
necesidad de generan mayores riquezas, esta es la moral principal de los
que han concentrado gran parte de la riqueza generada en el mundo.

En los países menos desarrollados tendremos una situación caótica, de
difícil cobertura de necesidades básicas por parte del gobierno, de toda
lo organización tanto pública como privada.

INDICADORES

Otra característica de los países con problemas de pobreza son las
migraciones

(Ej. África, que ha evidenciado estas condiciones de pobreza)

Debemos de pensar cuales son las condiciones económicas que permiten que
la población se encuentre entre la espada y la pared y opte por
abandonar la zona en que la que ha nacido y se ha desenvuelto, el PBI no
aumenta sin embargo la población sigue creciendo, las necesidades siguen
aumentando y cada vez los ingresos se reducen, aumenta la cantidad de
población que ingresa en pobreza y extrema pobreza y esto por la
practicas que se pueden tener desde el gobierno y en otros casos porque
se tiene condiciones territoriales adversas, cada vez en estos
territorios hay grande sequias y luego de las sequias se tiene grandes
inundaciones, la gente vive en condiciones inhumanas y acaba con toda
esperanza y lo minúsculo que puede tener esa población en propiedad y
estas son las situaciones que nos quita los sueños de salir de la
pobreza, sin embargo siempre tenemos que ir pensando de que al final del
túnel siempre hay una luz y tenemos que pensar en generar condiciones de
igualdad ante la ley, que la ley nos permita visualizar algo de justicia
para reforzar la posibilidad de que con esfuerzo se puede salir de esas
condiciones de pobreza.

``Las leyes no deben sesgar tanto a favor de las grandes empresas''
(acumuladora de riqueza) las leyes deben de ser herramientas que
incentiven la igualdad de oportunidades de la población que se encuentra
dentro de la pobreza y extrema pobreza.

Otro indicador muy social es la libertad, la gente debe de hacer lo que
mejor sabe y el gobierno debe de ingresar y apoyarlos en sus
iniciativas, ya que en países como el nuestro y otros en la que
visualiza estos casos extremos de hambre y miseria los gobiernos son
generalmente opresores, existe mucha dictadura, muchos cambios de
gobierno, no hay estabilidad, entonces las incitativas de las personas
no son respetabas o son frustradas cuando estas comienzan a gestarse.

Se debe de trabajar en la equidad en la distribución de la riqueza
generada por la sociedad a través del gobierno (prioridades de los que
toman las decisiones) tienen que estar dirigidas a una planificación con
iniciativas que se hagan efectivas y que el estado se haga presente.

*Debemos desaparecer la pobreza extrema y reducir en su mínima expresión
la pobreza, porque no está contribuyendo al crecimiento de la economía,
pero si es parte de la redistribución de la riqueza por lo que no es
conveniente para el crecimiento del país. (Debido a que ellos consumen
más de lo que generan, por lo que reducen lo que nosotros podríamos
consumir o lo que puede estar a disposición de la sociedad)

El pobre y pobre extremo no contribuyen positivamente al PBI.

Decimos que el crecimiento reduce la pobreza ya que al crecer el PBI
genera mayores recursos al gobierno y esta tiene según sus decisiones la
posibilidad de trasladar esos recursos hacia la mejora de las
condiciones de vida de la población de pobreza y extrema pobreza.

El crecimiento se puede generalizar a nivel de todas las actividades,
también aumentara la producción, la contribución del pobre y extremo
pobre para con el PBI pero aun cuando sea así, lo que está generando es
menor de lo que consume por lo que reduce las posibilidades de
crecimiento, por lo que se tiene que mejorar su condición de vida. Por
lo que tenemos que desaparecer la pobreza y pobreza extrema

\hypertarget{cuxf3mo-se-puede-radicar-la-pobreza-en-este-mundo-de-abundancia}{%
\subsection{¿Cómo se puede radicar la pobreza en este mundo de
abundancia?}\label{cuxf3mo-se-puede-radicar-la-pobreza-en-este-mundo-de-abundancia}}

En el mundo hay abundante cantidad de recursos, riquezas y alimentos
pero que sin embargo no llega a la población, por lo que hay que re
direccionar el esfuerzo y los recursos que van.

El crecimiento económico ayuda, las diferencias del grado de desarrollo
de las regiones también hacen la diferencia en el nivel de bienestar
dentro de cada una de estas regiones. Ej: américa latina: Chile,
Colombia, México, Brasil, aquí las diferencias en estas regiones depende
del grado de desarrollo que tienen estas regiones y entre ellas está el
Perú, y estas diferencias que tenemos hace ver las diferencias en el
nivel de bienestar de la población, en la concentración de pobreza y
extrema pobreza.

Ej.2: En Ayacucho somos el 1\% del PBI del Perú

Cuanto más crezca la contribución de riqueza en la sociedad en su
conjunto vamos a tener mejores condiciones de vida y la pobreza y
extrema pobreza se irán reduciendo. Las regiones nos diferenciamos unas
de otros de acuerdo al grado de contribución que tenemos al PBI, las
regiones en las que se produce más (crece más su economía) hay menores
niveles de pobreza y extrema pobreza.

El crecimiento economía cubre la reducción de la pobreza, se tiene que
crecer a niveles sostenibles, a niveles altos.

Caso Perú: - No es posible que nos quedemos con una mano de obra que no
está preparada para conducir el país, porque se tendrían grandes
problemas. - Las epidemias han demostrado que se descapitalizan las
economías - Con una pandemia a nivel mundial han rebasado las
perspectivas de control y manejo de la mano de obra.

Otro indicador que tenemos que tener en cuenta son las inversiones
directas externas, aquellas economías a las cuales ha llegado más
capital externo, mayor flujo de capitales, aquellas economías que han
crecido más y han reducido la extrema pobreza y han reducido la pobreza,
o que han mejorado la situación de los pobres.

Tenemos que crear las condiciones más favorables para la inversión
privada, tenemos que atraer inversiones, capitales a nuestra economía y
ofrecer estabilidad jurídica al país.

Las reglas de juego tienen que ser de uso de largo plazo, no se pueden
cambiar constantemente porque eso contribuye a la inestabilidad, por lo
tanto, los capitales no van a llegar.

Tiene que haber libertad económica para atraer los capitales externos,
hay que crear las condiciones y mejorarlas, aunque eso signifique
descuidar el manejo y la autonomía como nación de parte del Estado.

La clave es incrementar la renta aun cuando esta sea a diferentes
ritmos. (Yn=PBI)

Si Ayacucho quiere ser como Arequipa, Trujillo; Piura tenemos que crecer
más que ellos y para ello necesitamos no solo recursos sino las mejores
decisiones para asignar esos recursos a las actividades económicas.
(Clave)

Tenemos que hacer las cosas correctamente en el sentido correcto, ser
eficiente, eficaz efectivo

Otra Herramienta que tenemos que aprovechar para reducir la pobreza y
desaparecer la extrema pobreza es aprovechar los beneficios de la
tecnología, tecnología adaptada, adecuada a las necesidades de la
población objetiva, ya que esta tecnología van a beneficiarlos al
incorporarlos en sus actividades económicas.

Otra posibilidad es el crecimiento en el país acompañado de las
urbanizaciones que sean resultados de la planificación, de modo que este
se relacione directamente con el aumento de la productividad agrícola,
con la densidad poblacional, con las necesidades del comercio,
necesidades de servicio\ldots. porque si no, vamos a seguir creando
cordones de pobreza.

Tenemos que planificar el crecimiento de la población, mejorar la
productiva agrícola, mejorar las redes de distribución.

\begin{quote}
Objetivo a nivel mundial es erradicar la pobreza extrema y el hambre.
\end{quote}

Tenemos que priorizar las inversiones, identificar las inversiones que
tienen mayor efecto multiplicador y estas inversiones debe estar
centradas en las personas como en infraestructura.

Tenemos que crear sistemas de responsabilidad mutua y no unilateral, en
las que el pobre también asuma su responsabilidad, no solo les enseñemos
que solo tienen derechos que primero son nuestros deberes y
responsabilidades.

Otra estrategia se centra en las inversiones, en los mecanismos de
financiamiento. Los mecanismos de financiación para las zonas más pobres
tienen que ser subsidiaria, no lo debe de asumir el sector privado lo
tiene que asumir el sector público.

\hypertarget{ruxe9gimen-econuxf3mico}{%
\subsection{Régimen económico}\label{ruxe9gimen-econuxf3mico}}

\begin{quote}
El estado debe de participar subsidiariamente
\end{quote}

\hypertarget{rol-econuxf3mico-del-estado}{%
\subsection{Rol Económico del
Estado}\label{rol-econuxf3mico-del-estado}}

El Estado estimula la creación de riqueza y garantiza la libertad de
trabajo y la libertad de empresa, comercio e industria. El ejercicio de
estas libertades no debe ser lesivo a la moral, ni a la salud, ni a la
seguridad pública. El Estado brinda oportunidades de superación a los
sectores que sufren cualquier desigualdad; en tal sentido, promueve las
pequeñas empresas en todas sus modalidades.

\begin{quote}
estamos en una economía de libre mercado, el estado tiene que
participar, no debe de dejarle todo al sector privado debemos de
incorporar al régimen que pueden participar como parte empresarial
cuando el sector privado no participa o cuándo se convierte en
monopolio, podría ingresar el Banco de la Nación para hacer
transferencia de los recursos o financiamiento para la parte pobre y
extremo pobre de la población (tasas bajas del 5 al 8\%) .*el bando de
la nación estaría regulando las tasas de interés, sería el regulador de
las tasas de interés. Pero la constitución restringe estas entradas

Los programas sociales no han hecho otra cosa más que hacer daño a la
población pobre y pobre extremo porque sin hacer nada se les retribuye
con beneficios, porque la población no hace más que aprovecharse del
estado, de la incapacidad que tienen el estado para atender.
\end{quote}

RESUMEN

OTRAS ESTRATEGIAS

Inversión clave en infraestructura, Mecanismos de financiación, Bajar
tasas de interés

En BN podría entrar en acción para realizar préstamos

\hypertarget{cuxf3mo-es-que-se-implementa-la-poluxedtica-social-dentro-de-este-encare-de-reducciuxf3n-de-pobreza-y-extrema-pobreza}{%
\subsection{¿Cómo es que se implementa la política social dentro de este
encare de reducción de pobreza y extrema
pobreza?}\label{cuxf3mo-es-que-se-implementa-la-poluxedtica-social-dentro-de-este-encare-de-reducciuxf3n-de-pobreza-y-extrema-pobreza}}

La estrategia de la lucha contra la pobreza se centra en inversiones
clave, en obras y proyectos debidamente priorizados. Las inversiones
deben de estar dirigidas principalmente a las personas atrapados en la
pobreza y pobreza extremo, tenemos que centrarnos en ahí. La gente debe
de implementar estas inversiones, debe de fijar el orden de estas
prioridades

Tenemos que priorizar al hombre (inversiones clave)

Vemos que en esta pandemia nos hemos descapitalizado en recursos
humanos, en capital humano, debido a que las personas más capacitadas
han sido víctimas de esta pandemia y esto más adelante va afectar en la
Toma de decisiones.

El mejor capital de los jóvenes es su edad, y adquirir lo más antes
posible experiencia y debemos de tomar las mejores decisiones,
decisiones acertadas.

Otra inversión clave es la inversión en infraestructura, esta actividad
genera muchos puesto de trabajo directo y más del indirecto, su
capacidad de expansión en generación de puestos de trabajo y la
necesidad de mano de obra no calificada es importante.

Ej: construcción de una escuela de 2000m2 generara puestos de trabajo

Debemos de Inducirlos en sistemas de responsabilidad mutua (trabajo en
equipo) debemos de enseñarle a los trabajadores indirectos de modo que
ellos puedan ver qué primer son sus deberes, sus responsabilidades y
luego viene sus derechos, que no deben de esperar todo del gobierno, ya
que esto los jala a la pobreza porque cuando esperan del gobierno y este
no llega estos se descapitalizan.

Mecanismos de financiamiento para las actividades en las cuales se
pueden desenvolver ellos, hay que bajar TASAS de interés, el banco de la
nación podría entrar en acción y para eso hay que cambiar la
constitución para que el sector público ingrese a las actividades
financieras, tenemos que crear los mecanismos. (Fondos evolventes)

Y con la reforma se puede hacer, en las condiciones actuales se pueden
encontrar el capital revolvente que se le da a un beneficiario, hay que
buscar mecanismos de financiamiento uno de ellos es el fondo rotatorio,
tenemos que hacer financiamiento por grupos de cultivo como pro compite.

\hypertarget{cuxf3mo-es-que-se-implementa-la-poluxedtica-social-dentro-del-engarce-de-reducciuxf3n-de-la-pobreza-y-extrema-pobreza}{%
\subsection{¿Cómo es que se implementa la política social dentro del
engarce de reducción de la pobreza y extrema
pobreza?}\label{cuxf3mo-es-que-se-implementa-la-poluxedtica-social-dentro-del-engarce-de-reducciuxf3n-de-la-pobreza-y-extrema-pobreza}}

Quienes deciden la política social, la política económica los diversos
sectores como es que se priorizan los sectores, porque nuestra prioridad
es el crecimiento inmediato, por ejemplo: el sector construcción, sector
minero.

No hay entes reguladores, La política económica que maneja, prioriza los
sectores, ya que la política económica es el desenvolvimiento de los
diferentes sectores y su contribución a la riqueza del país

Tiene que haber una política social que permita en el caso de la pobreza
que factores de producción mover y que acciones hacer.

Socialmente la pregunta que tenemos que hacernos es ¿Quienes deciden la
política?

Tenemos que analizar la información de los medios de comunicación de
impacto, como es que los medios de comunicación si bien es cierto
brindan noticias, hace conocer lo que ocurre día a día, cuando hablamos
de información ya no es lo que ocurre, información son los datos
procesados y en se sentido los medios de comunicación no brindan
información en el país y lo que hay que hacer es que los medios de
comunicación brinden realmente información que es todo el resultado de
un esfuerzo de procesamiento de datos.

Si hacemos que la población se informe permanentemente se tendrá
sensibilizado a toda la población que está directamente relacionado a
este cordón de pobreza con la que se cuenta.

Tenemos problemas de esfuerzo para lograr nuestros logros

Hemos hablado de los mecanismos de financiamiento

Cuando hablamos de cómo financiarnos a través de tasas de interés
relativamente bajas, es que en el mundo financiero en la constitución de
la república no le permite al estado intervenir ni siquiera como ente
regulador\ldots. por lo que debemos de cambiar las reglas de juego y
esto pasaría por una reforma constitucional y esto depende del
presidente y congreso de la república.

Se podría incorporar que cuando la tasas de interés de la banca, de las
entidades financieras y de seguros estará de acorde a las tasa de
interés mundial aplicadas a nivel mundial de no ser así el estado está
en la obligación de intervenir como ente regulador o genera las
condiciones para dar acceso al crédito al financiamiento a las pequeñas
y medianas empresas. También se puede mencionar al BN, Y a través del BN
intervenir en el financiamiento de las actividades de las pequeñas
empresas, el BN da créditos a tasas de interés que se encuentran en la
banda de 0.25 a 0.5 anuales.

Pero sin embargo los bancos (interbank, BCP, continental) se les da a
estas tasas y estos bancos nos cobran mayor interés.

La ley (desarrollo de la constitución) de la banca y seguros juega a
favor de estos, pero se debe de dar la posibilidad de que el BN cumpla
esa función de intermediario no solo con un mercado cautivo que son los
empleados públicos.

Riesgo: ser una economía con participación estatal

En el sector público debemos de tener a los mejores profesionales, el
reto es que tenemos que mejorar la educación y la formación de nuestra
población 1. Debemos de mejorar la formación de nuestra población, de
nuestros profesionales 2. Debemos de hacer una reforma constitucional
que cree las condiciones necesarias.

\begin{quote}
En la actualidad somos un país entrelazado con el mundo internacional,
somos un país con un buen crecimiento, lo único que debemos de hacer son
algunos ajustes para que no tengamos problemas monopólicos, abusos
monopólicos, organizaciones oligopólicas\ldots debemos de corregir esto,
pero esto no significa que el estado deba de cumplir con las actividades
financieras, empresariales y todas las actividades económicas porque ahí
la propiedad privada pasa a segundo plano.
\end{quote}

\begin{quote}
la ley servir permite reconocer la meritocracia,
\end{quote}

Inversiones estratégicas que se deben de hacer para reducir la pobreza y
principalmente nos centrábamos en dos:

Las inversiones clave teníamos que hacerlos en las personas y la
infraestructura. La atención se debe de centrar en las personas, en la
infraestructura podemos edificar priorizar, es cuestión de gestión, se
trata de incorporar personas en diversas actividades para las cuales
quizá no estén capacitados o perfeccionarlos va ser bastante difícil por
lo que hay que tratar de entrar con una metodología nueva, porque son
personas que no están acostumbradas a las capacitaciones, como por
grupos de actividad, oficios comunes o por cercanía a determinadas
actividades y también se debe de buscar mecanismos de financiamiento.

En la parte de la infraestructura depende del don de la capacidad,
calidad de los funcionarios y nos centrábamos en las inversiones que
están dirigidos a las personas que están en pobreza y extrema pobreza,
hay que buscar algunas formas especiales para capacitarlos y atraerlos a
las capacitaciones, para elegir algunos oficios y actividades a las
cuales se van a dedicarse a futuro y no queden truncas, se tiene que
aprovechar las oportunidades y darles todas las oportunidades para que
ejerzan ese oficio y vean los benéficos y aclarar que no es solo por
responsabilidad del aparato estatal o sector privado interesarse en
mejor su situación sino que los principales interesados deben de ser
ellos del aparato estatal . Lo otro es que hay que buscar mecanismos de
financiamiento para hacer actividades propias y para tratar de financiar
las iniciativas que ellos tienen.

Ej: el BN puede cumplir esa función, hacer una pequeña reforma en la
constitución en la que se puede tener no solo\ldots{} Sino que también
participa en la actividad financiera, que promueva y participe, se
podría modificar la ley de banca y seguros y poner las bandas de tasas
de interés, el estado puede ingresar a través del BN sin que esto
signifique crear otros bancos para estos fines.

Normalmente cuando vemos a los pobres debemos de hacer un análisis de
qué tipos de capital se puede hacer inversiones con ellos.

Los pobres carecen de algunos tipos de capitales y hay que invertir para
genera, sabemos que el capital humano en el segmento de la población de
pobreza y extrema pobreza el capital humano tienen bajísimos niveles de
productividad, decimos que debemos de hacer inversiones para mejorar las
capacidades y participación directa en trabajos asequibles que no
necesitan demasiado entrenamiento y que pueden ser absorbidos por los
proyectos de inversión del sector público ej. Las obras públicas,
podrían crearse programas sociales que identifique a esta gente, en la
cual podamos capacitarlos e incorporarlos.

De los factores de producción debemos de darle la capacidad empresarial,
debemos de darles esa capacidad a los pobres ya que estos no cuentan con
esto, no saben identificar oportunidades. Debemos de enseñarles de que
hay cierto tipo de actividades que ellos pueden ejercer, las decisiones
que tienen que tomar y que factores tienen que considerar como el
capital disponible, las cosas que sabe hacer. Tienen que adaptarse, no
solo trata de copiarse sino adaptarse en el mismo rubro que pueden
mejorar esos ingresos.

Las zonas más pobres de país son los que menos captan de los
conocimientos existentes y son los adversos a decepcionar porque siempre
buscamos culpables.

Otro capital es el capital natural, los pobres y pobres extremos tienen
sus terrenos, pero las tierras cultivables no siempre los mejoran los
suelos no están acondicionados, tiene que mejorar los suelos, ponerlos
en mejores condiciones para hacerlos cultivables.

Otro capital es el capital institucional del sector público, el sector
público tiene que generar legislaciones comerciales que permitan que la
población participe en mejores condiciones, un sistema judicial
adecuado, el majo comercial tiene que cambiar, el sistema judicial tiene
que mejorar. Tiene que haber una reforma profunda a mejorar las
condiciones, los servicios gubernamentales tienen que mejorar, no se
pueden cobrar la misma tarifa entre los pobres y extremo pobres que al
resto de la población.

Las políticas que respaldan la división del trabajo deben de
implementarse, identificarse con priorización de parte del estado

Otro capital es el capital intelectual, eso significa que debemos
priorizar el saber práctico antes que el saber teórico, debemos de
preparar a la gente en el saber practico. En el Perú esto es lo que
falta, porque tenemos profesionales pero solo en el marco teórico,
debemos de invertir en capital intelectual científico y tecnológico

Y la atención se tiene que centra en las personas, en la infraestructura
podemos edificar, priorizar es cuestión de gestión pero se trata de
incorporar personas en diversas actividades para las cuales no están
capacitadas o es muy nuevo para ellos o es perfeccionarlos es bastante
difícil por lo que hay que incorporar metodologías nuevas porque son
personas que no están acostumbrados a las capacitaciones, ya sea por
grupos de actividad, por oficios comunes por cercanía a determinadas
actividades

RESUMEN

La política social en el país es bastante influida por los medios de
comunicación de impacto, y que en alguna medida modifiquen su
comportamiento.

Y Como es que nosotros participamos para superar la pobreza y extrema
pobreza

El razonamiento que debemos de hacer es que la pobreza extrema ha
desaparecido en los países más desarrollados y está desapareciendo en
los países de renta media, países con ingresos medios en la que la clase
media es la clase dominante en número en determinados países, por tanto
es necesario en alguna medida alcanzar la prosperidad de los países más
desarrollados, de modo que nos permita desaparecer la extrema pobreza y
reducirla a su mínima expresión, el crecimiento es necesario pero, ¿cuál
es el nivel de crecimiento que necesitamos? Podríamos decir que si
hubiera tuviéramos voluntad política es posible desaparecer la pobreza,
en todo caso llamaríamos pobreza a cierto nivel de comodidad y bienestar
de parte de la población con acceso a todos los servicios y que debe de
haber algún nivel de restricción, prácticamente hablaríamos de una clase
media baja.

Por eso decíamos que la pobreza extrema ha desaparecido en los países
más desarrollados, lo que tienen son pobres pero que estos están
reducidos en un mínimo porcentaje, pero es una pobreza con algún nivel
de ingreso que está por encima de los pobres de los países con problemas
con desarrollo.

Las políticas globales más importantes para reducir la pobreza en
realidad son combatir la crisis de la deuda ósea no se debe incrementar
en lo posible la presión de la deuda externa ante el presupuesto público
o relacionándolo con el PBI. (En Perú nos endeudamos para gasto público)

Uno de los factores que debemos de tener en cuenta para reducir la
pobreza es la crisis de la deuda pública, también existe una deuda
privada pero esta deuda lo paga el sector privado.

La política comercial es uno de los factores para reducir la pobreza ya
que permite a que los países donde impera el libre comercio donde las
barreras arancelarias se acercan a cero, son los que crecen más rápido y
estos son los que permiten reducir la pobreza porque la gente más pobre
está en el sector agropecuario y esta gente si encuentra ventanillas
para las exportaciones van reducir la pobreza rápidamente en esas zonas,
es una herramienta para reducir la pobreza y mejorar la política
comercial, por lo que debemos de mejorar la política comercial, los
tratados de libre comercio.

La ciencia aplicada al desarrollo en el que nos encontramos fuertemente
retrasados, el gobierno debe comenzar a financiar y ver el orden de
prioridades a las cuales debe estar dirigido la investigación, la
ciencia aplicada debe de estar aplicado en las zonas urbano marginales,
en el área rural ya que el gran avances tecnológico de por si, como
parte de la globalización se incorpora a las empresas sin embargo en
estas zonas de poco desarrollo donde está la pobreza y extrema pobreza
no es fácilmente que ingrese la tecnológica porque necesita de
inversión, por lo que el gobierno debería de priorizar y canalizar los
fondos para investigaciones de aplicación en estas zonas urbano
marginales y en el sector agropecuario lo cual sería un gran avance, un
gran aporte para que la población de extrema pobreza y pobreza puedan
reducirse.

La gestión del medio ambiente debe de ser con transferencias que
permitan a que los practican y protegen el medioambiente se les pueda
reconocer ese esfuerzo.

Debemos de entender que el desarrollo de una sociedad, la mejora de
condiciones de vida, la situación de desarrollo de por si proporciona
recursos a la población para construir la esperanza, darle alguna
posibilidad de salida a su situación de pobreza y extrema pobreza y que
en algún momento podría llegar la prosperidad que dependerá a de su
esfuerzo y además la seguridad de que puede desenvolverse en las
diversas actividades económicas sin temor de que pueda ser víctima de la
delincuencia o simplemente de la inestabilidad de la normatividad que el
estado genera, es decir que las reglas de juego no cambie
permanentemente.

Por lo que ya hemos visto que políticas globales podemos implementar: -
La deuda - Política comercial - Ciencia aplicada - Gestión ambiental Es
necesario imprimir la esperanza y eso pasa por la confianza en el
gobierno en la confianza de los tomadores de decisiones eso no llevaría
a la prosperidad y seguridad del país

Y si es así ¿en que invertimos? podemos invertir en desarrollo
agropecuario, porque ya tenemos identificado la ubicación de la pobreza
y extrema pobreza por sector, el sector agropecuario es en el que mayor
porcentaje de la población pobre y extrema pobre tiene, después tenemos
la población urbano marginal por lo que podemos hacer inversiones en
desarrollo agropecuario, también debemos de invertir en la salud
primaria de atenciones para enfermedades comunes, debemos de mejorar la
calidad de salud.

Otra inversión que hay que hacer es en el control demográfico y que la
población y más pobre sean conscientes de que no debe de tener más hijos
y eso es responsabilidad compartida no solo es responsabilidad del
gobierno sino también de la familia, debe de haber una política de
control de natalidad apuntada a la población pobre porque son los que
menos posibilidades tiene.

Lo otro es que debemos de incentivar el ahorro, el pobre no va ahorrar,
pero la clase media para arriba si para que el gobierno central tenga
capital para invertir.

Debemos de considerar actividades propias de ellos y hay que hacer
inversiones directas del estado demostrativas, el factor humano va
participar en ellos que esté debidamente preparado y este consiente que
está trabando para le gente pobre y no es para una empresa y esto en
alguna media es una restricción porque el empleado público no valora el
trabajo que se debe de hacer desde el estado para con la población.

En la actualidad en el mundo por ejemplo ecuador Bolivia en el Perú está
dando buenos resultados los programas sociales con transferencias
condicionadas (ej. Juntos)

Debemos de pensar en que todos los programas sociales que tenemos
deberían de cambiar su reglamento y todos deben de ser condicionadas,
debemos de poner algunas exigencias mínimas que cumplan las familias
para ser beneficiaras de estas transferencias económicas que se hacen,
no se pueden entregar así por así ya que esto sería condenarlos a la
pobreza eterna, ya que siempre estarían en esta condición, los programas
sociales no resuelven los problemas de la pobreza solo las atenúan, y
son dañinos ya que mucha gente que está en esa condición lo ven como un
medio de vida y se acostumbra a que el gobierno les hagan esas
transferencias y otras más y no piensan en trabajar, en generar,
esforzarse más para generar más riqueza y ahí ese tiene responsabilidad
del gobierno por no haber diseñado bien los programas sociales, los
gobiernos deben de tener la capacidad de desaparecer, retirar los
programas sociales que no funcionan y transformarlos porque no tiene
sentido una transferencia presupuestal que no mejora la situación del
pobre y extremo pobre, por lo que sería bueno evaluar los programas
sociales\ldots. porque los programas con transferencias condicionadas
exitosa en el país en Bolivia en Ecuador han incremento la tasa de
asistencia de los niños a sus centros educativos, la deserción es mucho
menor que antes, también han incrementado su asistencia al control de la
salud.

Por lo que se debe de evaluar los programas sociales considerando
principalmente la estabilidad, la capacidad para mantenerlo en el
tiempo, lo que realmente funciona, los programas sociales deben de ser
estable siempre y cuando funcionen, si estas no cumplen las meta para
las cuales han sido diseñadas y creadas simplemente es deben de retiras,
los programas sociales se deben de adaptar a las nuevas exigencias que
permanentemente se dan, se tienen que adaptar a las condiciones
necesarias.

Las normativas existentes para el cumplimiento de los objetivos en los
programas sociales deberían de ser cumplidas e implementadas, pero no se
cumplen las normativas.

Debemos de ejecutar y cumplir las regulaciones que se dan de parte del
gobierno, no podemos tener entes reguladoras que son parte de las
grandes empresas y estas empresas capturan a estos órganos reguladores
como parte de ellos y no como parte del gobierno, porque incluso
sobreviven de las multas o las trasferencia que hace el sector privado a
estas reguladoras, entonces en que momento los castigan, no tiene la
capacidad, no pueden actuar porque lo que manda en esos sectores es el
presupuesto.

Por lo que se tiene que ver el interés político que es importante y que
generalmente solo ayuda a determinados grupos de poder, es ahí donde
debemos de reducir los sesgos, porque como se dice las campañas
electorales son caras, parte de eso pagos son esas decisiones de
gobierno para favorecer determinados grupos por lo que se hacen dueños o
atienden a través de ONGs u organismos paralelos al gobierno y las
instituciones privadas que son apoyo publicó privado y por ahí está la
herramienta para favorecer a los grupos de interés.

Aquí terminamos con el tema de globalización y el creciendo y reducción
de la pobreza.

\hypertarget{gobernabilidad-e-instituciones}{%
\subsection{GOBERNABILIDAD E
INSTITUCIONES}\label{gobernabilidad-e-instituciones}}

La gobernabilidad del estado depende de la fortaleza de sus
instituciones, la gobernabilidad del estado es el manejo del estado a
través del gobierno de los diferente sistemas tanto a nivel del poder
ejecutivo, legislativo y judicial y cada uno de estos poderes tienen un
conjunto de instituciones que los representa y la fortaleza del estado
depende de que estas instituciones sean eficientes lleguen a alcanzar
las metas, objetivos y logros que se espera tener, tienen que ser
eficaces porque los recursos con los que cuenta tiene fines para
alcanzar logros, es más deben de ser efectivos para traducir confianza
en la población, por tanto la gestión gubernamental se rige bajo un
marco presupuestario adecuado y que su uso debe pasar por una
planificación que esté en manos expertas y que tengan un enfoque técnico
de la función pública.

El estado es una representación jurídica es algo imaginario y se expresa
a través del gobierno el presidente de la republica representa al estado
es el jefe de estado y a su vez es el jefe de gobierno, es el enlace
entre el mundo inmaterial y el mundo real.

El gobierno a través de esto diferentes poderes llega brindando sus
servicios a la población y estos poderes tienen que estar representados
en el territorio nacional en el que se enmarca el estado.

Tenemos tres niveles del gobierno

\begin{itemize}
\tightlist
\item
  Gobierno Central
\item
  Gobierno Regional
\item
  Gobierno Local
\end{itemize}

Si las instituciones que representan al estado a través del gobierno
están bien instituidos, debidamente organizadas, sus servicios son
efectivos, entonces la población reconocerá como muy buenas, como que el
gobierno tiene presencia y para ello cuenta con un marco presupuestal
necesario para atender las necesidades de la población a la cual brinda
sus servicios. El presupuesto es antes de supuesto es una propuesta de
ejecución de gastos, una propuesta posibilidad a lo que el gobierno
asume que puede hacer y se hace efectivo el presupuesto cuando esta
puede ser disponible y es ahí cuando se convierte tangible.

El estado a través del gobierno cuenta con un marco presupuestal, y ese
presupuesto tiene que estar de acuerdo a la realidad de las necesidades
de la población beneficiaria y debe de proveer la funcionalidad del
gobierno ya que sin presupuesto no existe gobierno, ya que donde no hay
presupuesto no está la presencia del estado a través del gobierno
dependiendo del marco presupuestal hablaremos de presencia adecuada o
inadecuada ya que lo importante está en el orden de prioridades que debe
de hacer el estado a través del gobierno considerando las necesidades de
las diversas actividades, para lo cual tiene que haber una
planificación, debemos de crear estrategias y para la planificación lo
más importante es el orden de prioridades.

Pero la priorización no está en la voluntad de los políticos,

En la actualidad se está hablando de sistema de planificación, la
planificación se está sobreponiendo al presupuesto, porque finalmente
primero se debe de planificar y luego asignar el presupuesto, pero esto
no es así ya que primero se asigna el presupuesto y luego se planifica.

Por eso es necesario que la planificación esté en manos de expertos en
función pública y que conozcan cómo funciona, la aspiración del sistema
de planificación es que todo lo que ejecute el gobierno central debe de
estar previsto y que esté en orden de prioridades y que los políticos se
ciñan a los planes, que no cambien las reglas de juego por lo que debe
de haber continuidad.

La gobernabilidad depende de que sus instituciones sean fuertes,
respetables y cumplan que tengan una asignación presupuestal adecuada y
que se rijan con una planificación hecha por expertos en la materia,
crear un enfoque de función pública, no se puede improvisar tiene que
ser debidamente planificado tiene que ser previsto.

Las instituciones se fortalecen en tanto cuenten con las capacidades de
expertos que responden a su visión a su misión y alcanzan los logros
esperados.

*La gobernabilidad del estado depende de la fortaleza de sus
instituciones, de la gestión estatal, gubernamental depende en su
integridad del nivel que ha alcanzado las instituciones públicas y está
basado y debe regirse dentro de un marco presupuestario adecuado y debe
de estar en manos de expertos en planificación y que conozcan del
enfoque técnico de la función pública.

En el mundo podemos decir que hay países que han alcanzado crecimiento
sostenido sin contar con instituciones sólidas, lo cierto es que los
países que han alcanzado el desarrollo y se encuentran en la etapa
intermedia entre países desarrollados y no desarrollados han alcanzado
este nivel de desarrollo y reducido la pobreza porque cuentan con
instituciones sólidas, bien organizadas y calidad de servicio.

Por lo que es una preocupación para el Perú, ya que si queremos reducir
la pobreza debemos tener instituciones que permitan que el estado a
través del gobierno llegan a la población objetiva, porque resulta que
algunas herramientas, algunos esfuerzos que se implementan parara llegar
a esa población pobre y extrema pobreza se diluye en el trayecto entre
la disponibilidad presupuestal, la disponibilidad para cumplir la
normativa y ejecutar, porque en el intermedio están las instituciones
públicas y estas en el caso peruano no traducen la normatividad
correctamente y no cumplen.

\hypertarget{en-el-peruxfa-la-calidad-de-las-instituciones-estuxe1-a-la-altura-de-las-metas-estuxe1n-las-instituciones-puxfablicas-organizadas-para-alcanzar-las-metas-que-se-les-propone}{%
\subsection{¿En el Perú la calidad de las instituciones está a la altura
de las metas, están las instituciones públicas organizadas para alcanzar
las metas que se les
propone?}\label{en-el-peruxfa-la-calidad-de-las-instituciones-estuxe1-a-la-altura-de-las-metas-estuxe1n-las-instituciones-puxfablicas-organizadas-para-alcanzar-las-metas-que-se-les-propone}}

Aun no, porque estamos en proceso, ya que las instituciones públicas
reflejan los intereses personales de los empleados públicos más no para
atender las necesidades y los intereses de la población a la cual está
dirigida esa institución pública. (ej: trabajadores del sector salud)
tenemos que saber darnos cuenta de que el aparato estatal, las
instituciones que representan al gobierno no están bien consolidadas, no
están bien organizadas y el personal no reflejan los objetivos que la
institución buscan alcanzar (El ministerio público investiga al que odia
y no al que realmente debe de investigar, la razón de ser del ministerio
público primero es que el fiscal debe pensar bajo la presunción de
inocencia, pero este hace la investigación pensando que ya es culpable)

En el Perú, en la década de los 90 se adoptó un enfoque basado en la
creación de organismos para la reforma estatal, incluso entre el 2000 y
2010 se vio que las instituciones estaban placados de tintes políticos y
se había desnaturalizado las instituciones, por lo que se adoptó un
enfoque que permita organismos, y generen un nivel de reforma estatal,
que hayan generado resultados positivos para algunas instituciones (ej.:
en el ministerio de economía y finanzas se incorporó el siaf, mejor
gestión fiscal, mejor ejecución de los gastos y no por eso se ha supero
que los diferentes ministerios lo ejecuten bien) la segunda fase debe de
concentrarse en mejorar la calidad del gasto fiscal, no solo en los
resultados que se esperan que las instituciones cumplan con la
normativa, la segunda fase se tiene que concentrar efectivamente en la
mejora de la calidad de gasto fiscal, y esta mejora pasa por lograr que
el país consolide la gestión fiscal, desarrolle estrategias de reforma
estatal futura, deberíamos empeñaros en un orden de prioridades para
mejorar la calidad del gasto, no se trata solo de gastar.

El ministerio de economía y finanzas tiene que mejorar la segunda etapa
de la reforma tiene que mejorar la calidad de vida.

La gestión fiscal a cumplir debe de ser estratégico, la reforma estatal
a futuro considerando algo que sume a las estrategias que ya se están
utilizando.

La futura reforma estatal tiene que estar basado en tres elementos
básicos (desafíos)

\begin{enumerate}
\def\labelenumi{\arabic{enumi}.}
\tightlist
\item
  El incremental ismo estratégico, hay que apuntalar que en toda
  institución pública su funcionamiento tiene que ser estratégico, todos
  del aparato público tenemos que aprender a generar estrategias
  (debemos darnos cuenta que es lo que buscamos), esto es aumentar el
  uso de las estrategias, todo trabajador del sector público debe saber
  generar o crear estrategias.
\item
  La reforma estatal de las instituciones del gobierno central
  (ministerios, opedes), LOPE (ley orgánica del poder ejecutivo, que
  establece las reglas de organización, gestión y competencias de uno de
  los principales poderes del Estado) se ha hecho en el año 2009 y se
  publicó en el 2010 pero no se ha desarrollado esa ley, pero hay que
  hacer una reforma estatal del gobierno central, ósea de los
  ministerios y los organismos públicos descentralizados (opedes) deben
  pasar por una reforma de modo que se adapten o adecuen a la nueva ley
  orgánica del poder ejecutivo pero a su vez esa ley hay que
  modernizarlas, adaptarlas a las nuevas condiciones y exigencias del
  país a fin de que la instituciones públicas se consoliden en el
  gobierno central, una vez que se esté haciendo eso se debe también
  hacer la reforma estatal más allá del poder ejecutivo, eso significa
  el poder judicial y el congreso ya que ahí también hay que
  restructurar, reorganizar y hay que cambiar la normativa existente,
  adaptarlas a las nuevas exigencias del país. Entonces hay que hacer
  cambios, hay que hacer reformas en el aparato estatal que ya no es el
  gobierno central.
\item
  Ingresar a los otros poderes (poder judicial y legislativo).
\end{enumerate}

En la perspectiva del futuro, en el nuevo contexto que se va encontrar
el Perú, es necesario pensar en la reinserción en los mercados
mundiales, tenemos que cambiar la política externa del país, nuestras
representaciones del exterior tienen que ser embajadores especializados
en inserción en mercados, no podemos solo tener una burocracia que solo
estén dedicado a la buena presencia y protocolos de etiqueta, tiene que
identificar mercados.

La descentralización tiene que mejorar no puede ser entendido que
cualquiera pueda hacer y deshacer y destrozar la institucionalidad de
los pliegos presupuestales, la descentralización tiene que refinarse,
tener que quitar algunas potestades al presidente regional, garantizar
la continuidad de los funcionarios, debe de haber un nivel de selección
del personal basada en la experiencia, continuidad del funcionario.

Si queremos afrontar en el futuro el nuevo contexto que se nos va
presentar tenemos que considerar la reinserción en los mercados, tenemos
que revisar el grado de competitividad de las empresas privadas y
mejorar las condiciones de la descentralización ya que estas necesitan
mucha dedicación para tener instituciones serias y para hacer esto
tenemos que afrontar desafíos, una de ellas es elevar la calidad del
crecimiento económico pensando en la reducción de la pobreza (esto es
importante porque si el crecimiento es de largo plazo a una tasa de 6 a
7\% anual durante 10 años de crecimiento la pobreza se va reducir y va
ser una reducción acelerada de la pobreza) y esto pasa por un
crecimiento sostenido.

Otro desafío es lograr que las instituciones públicas tengan mayor
credibilidad, lograr que la gente comience a creer en lo que pueden
hacer las instituciones públicas y para eso hay que demostrar con hechos
que lo que se dice se ejecuta.

Finalmente, otro desafío es mejor la eficiencia de los servicios, el
cumplimiento de los servicios, la calidad de los servicios tiene que
mejorar, no podemos seguir con trabajadores que no brindan un buen
servicio.

Necesitamos una nueva estrategia de reforma del sector público
considerando estos tres desafíos. (Calidad del crecimiento económico,
superar la incredulidad de estado y finalmente la eficiencia de los
servicios públicos)

Debemos considerar que una mejor gobernabilidad del aparato estatal pasa
por manejar los recursos de forma eficaz, la gobernabilidad debe ser
manejada con los mejores tomadores de decisiones (decisiones
trascendentales, decisiones de gobierno) debemos ser eficaces tomar
buenas decisiones y hacerlas correctamente.

Para mejorar la gobernabilidad debemos tomar las mejores decisiones,
debemos implementar políticas fiscales solidas que permitan que en el
tiempo se consolide ese tipo de decisiones para captar más ingresos y
una buena direccionalidad, esto va permitir mejorar los servicios a los
ciudadanos. (Aquí está el secreto para tomar las decisiones).

La población debe percibir que se cumpla con la ley, el control de la
corrupción se ha deteriorado, la contraloría hace un esfuerzo para
controlarlo y la población debe de estar involucrada, el estado debe ser
efectivo en su lucha contra la corrupción.

Debemos hacer una segunda reforma del estado que pasa por una
restructuración y reorganización, debemos mejorar la eficiencia del
estado, los servicios del sector público.

Hemos hablado de la participación de las instituciones públicas y su
influencia y determinación en la gobernabilidad del aparato estatal.

RESUMEN

Hemos hablado de Como elevar la calidad del crecimiento económico y que
eso nos puede permitir reducir la pobreza rápidamente y cómo podemos
lograr que las instituciones públicas tengan mayor credibilidad en el
país y como mejorar la eficiencia de los servicios para hacer frente a
estos desafíos que nos pone el contexto mundial como una estrategia del
sector público.

Debemos mejorar el manejo de los recursos públicos de manera más eficaz,
implementar políticas fiscal más sólidas y mejorar los servicios que
brinda el sector público a los ciudadanos y la percepción del
cumplimiento de las leyes, de acá a un tiempo las diferentes
instituciones del sector público como del privado no estamos cumpliendo
con la normativa existente, parece que no existe un control
gubernamental de modo que se deteriora la imagen del estado a través del
gobierno y la realidad nos dice que la corrupción se a incremento porque
el control de la misma por falta de cumplimento de sus obligaciones y
las leyes de parte del gobierno hace que se fomente la corrupción, por
lo que tenemos que superar haciendo que haya una efectividad del estado.

Es necesario mejorar y comenzar a implementar una reforma del aparato
estatal en búsqueda de mejorar e incrementar la eficacia a nivel
estatal.

Debemos recalcar que en los 3 niveles y poderes del gobierno no existe
voluntad de iniciar un control del manejo gubernamental y el cumplimento
de las leyes para controlar la corrupción, porque desde el interior de
estas instituciones representativas neurálgicas o su razón de estar del
gobierno no hay voluntad de cumplimiento de las normas, ya que pareciera
no existir el gobierno y los pequeños funcionarios hacen de las
decisiones lo que más les parece y les conviene para fines personales,
en la década de los 90 la reforma del estado ha sido drástico pero poco
sistemático, no se visualizó el futuro inmediato del aparato estatal.
Solo se vio el corto plazo, ya que el objetivo solo fue reducir el
tamaño del estado y no lo proyectaron, no se vio lo inmediato que sería
este proceso de racionalización de los recursos humanos.

CLASE

Mayores esfuerzos para el proceso de fiscalización, trasparencia y
coordinación

Cuando hablamos de fiscalización hablábamos del control que hacemos a lo
que el gobierno central, regional o local podía actuar, debemos de
verificar si se está cumpliendo o no de acuerdo a los parámetros, si se
han cumplido las normas pre establecidas para estos fines.

Cuando fiscalizamos revisamos si ha cumplido o no, como estaba fijado,
si se ha alcanzado las metas en el tiempo, en que cosas se han fallado.

La trasparencia es que la gente conozca en lo posible todos los
procedimientos. Debemos de hacer una reforma estatal de segunda
generación, esta reforma estatal que se puede hacer a partir del 28 de
julio, puede basarse en estrategias nuevas y efectivas, efectivas en el
sentido de que debe de ser sistemático, se debe de mirar el futuro.

Otra acción que debemos de considerar son las reformas en el gobierno
central Otra reforma debe de ser en el poder ejecutivo y poder judicial,
la reforma tiene que ser vertical y transversal, debemos de hacer las
reformas en base a 4 áreas:

\begin{enumerate}
\def\labelenumi{\arabic{enumi}.}
\item
  un nuevo rol de congreso (así como está el congreso no funciona,
  debemos de hacer una nueva reforma creando nuevos reglamentos y nuevas
  directivas y todo lo demás dejarlo sin efecto, comenzar nuevamente de
  cero pero sin dejar de actuar con lo que hay pero sin dejar de actuar
  con lo que hay).
\item
  Crear un nuevo rol de las instituciones presupuestarias, el sistema
  nacional de presupuesto ha variado y mejorado mucho pero viene
  cargando los vicios anteriores, por lo que debemos dejar sin efectos
  esos e implementar un nuevo rol de todas las unidades estructurales
  que tienen que ver con el presupuesto público y crear una nueva
  cultura de presupuesto.
\item
  Debemos de hacer reformas en todo lo que es administración pública, en
  todos los procedimientos y los sistemas administrativos diversos, hay
  que modificar casi todo para lo cual se necesita un congreso dinámico
  con gente capaz y ganas de hacer las cosas bien.
\item
  Debemos ver una reforma en la nueva ley de descentralización e incluir
  en esta descentralización al poder judicial, el poder judicial es un
  ente muy centralizado, incluso el dinero que ese ejecuta en Ayacucho
  se gasta y después rinden, no tienen autonomía en su ejecución, no hay
  capacidad de fiscalización no hay capacidad de control en el poder
  judicial.
\end{enumerate}

La reforma estatal debe de basarse en 3 razones:

\begin{itemize}
\tightlist
\item
  Estrategias nuevas y efectivas
\item
  Reformas en el gobierno central
\item
  Reforma en el resto de los poderes: legislativo y judicial
\item
  Las reformas se deben de hacer en 4 áreas
\end{itemize}

\begin{enumerate}
\def\labelenumi{\arabic{enumi}.}
\tightlist
\item
  En el congreso, debemos de hacer nuevas leyes, hay que darle un nuevo
  rol al congreso.
\item
  Debemos darles un nuevo rol a las instituciones presupuestarias,
  debemos de darle otra imagen, otras leyes, todo lo positivo debemos de
  acumularlo e incorporar en proyecciones, debemos de hacer reforma en
  todo lo que es el aparato estatal, todos los sistemas administrativos,
  la administración pública.
\item
  Finalmente se debe de profundizar la descentralización, mejorar la ley
  de descentralización, reformar estos gobiernos regionales, hay que
  restructurarlos, reorganizarlos y darle una estructura organizacional
  y ahí también incorporar al poder judicial, debemos de descentralizar
  el poder judicial.
\end{enumerate}

Esto es en cuanto a las reformas, las áreas en las que hay que hacer las
reformas.

\begin{enumerate}
\def\labelenumi{\arabic{enumi}.}
\setcounter{enumi}{3}
\tightlist
\item
  Lo otro es las reformas en las áreas institucionales y lo otro en
  áreas en generales a nivel nacional (macro).
\end{enumerate}

Las otras áreas institucionales en cada uno de estos poderes, en c/u de
estos niveles de gobiernos debemos de hacer una reforma institucional es
la organización del estado.

La ley orgánica del poder ejecutivo se hizo en el 2009, la cual esta
desactualizada\ldots y si vamos hacer reformas la ley orgánica tiene que
modificarse totalmente y mejorar, por lo que debemos de cambiar algunos
artículos y adecuarlos a las nuevas condiciones y exigencias.

Un área de reforma institucional es la organización del estado, nuevas
leyes como: la ley orgánica del poder ejecutivo, ley orgánica de
descentralización, ley orgánica de gobierno regional, ley orgánica de
municipalidades debemos de modificarlas y si modificamos todas estas
leyes todas las demás normas se modifican.

Un, reforma institucional que debemos de hacer en lo que es calidad del
gasto público, aquí tiene que ingresar un ajuste en las leyes,
reglamentos de la ejecución de gasto presupuestal, y esto es calidad del
gasto público que tiene que haber un orden de prioridades y que sin
autorización y si no se encuentran en el plan estratégico institucional
de mediano y largo plazo no se deben de ejecutar ningún proyecto.
Debemos de poner por encima del sistema de presupuesto al sistema de
planificación y por ahí podemos mejorar la calidad del gasto, debe de
haber un orden de prioridades, la calidad de gasto para que sea
mejorado, para que realmente funcione tendríamos que hace reformas en la
ley de recursos humanos u oficinas de personal, debemos de cambiar la
ley de adquisiciones, contrataciones y adquisiciones del estado,
cambiarlo lo más rápido, cambar los criterios de inversión pública, los
criterios del proceso de centralización, elecciones, la calidad del
gasto estaría garantizado, sabríamos cual se hace primero y cual se hace
después, estará en función a la disponibilidad presupuestal.

Otra área de reforma institucional es el manejo fiscal que es la
administración financiera del país, debemos de hacer un profundo
monitoreo, seguimiento y evaluación de la administración financiera para
mejorar la calidad del gasto, porque si queremos mejorar la calidad del
gasto debemos de saber con cuanto de fondos disponemos y cuáles son las
alternativas para financiar esos proyectos, esas obras que vamos hacer,
no confundamos la inversión con el gasto.

Y finalmente el área de reforma institucional son las reformas del
estado, el rol del congreso, instituciones presupuestarias y el combate
a la corrupción.

Debemos de pensar que debemos tener un combate a la corrupción desde el
arma legal, debemos de crear leyes que realmente sirva, ya que hemos
creado leyes contra la corrupción que en realidad no controlan nada, más
bien adormecen la lucha contra la corrupción por lo que hay que mejorar
el conjunto de leyes.

Otra cosa que debemos de tener en consideración la plena gobernabilidad
democrática para el Perú implica vigencia de los derechos humanos, pero
de los derechos humanos se deben de entender que no es para todos, sino
para las personas correctas, ósea para persona correctas ya que estas
más funcionan para personas torcidas para personas que han asesinado,
etc. Para ellos están hechos los derechos, más les hacen caso las
instituciones al ladrón.

Debemos de pensar que debemos de mejorar estas circunstancias,
lamentablemente hemos enseñado y aprendido a enseñar a la gente primero
sus derechos luego sus deberes cuando debería de ser al revés.

No podemos inculcar a la gente que solo tiene derechos sino que también
tiene deberes ya que los derechos humanos y su vigencia no debe de ver
discriminación debe de haber acceso a la justicia, debemos de prevenir,
la gestión de conflictos, la población no puede estar en constante
enfrentamiento.

Debemos de hacer gestión del conflicto ya que después del 28 de julio
puede venirse una avalancha de descontentos si gana keiko o gana
castillo\ldots. es una situación un poco difícil, podemos entrar en un
tramo de violencia por lo que debemos de implementar gestión de
conflictos, seguridad ciudadana, confianza en los políticos, un gobierno
más descentralizado y para eso necesitamos tener gobernadores regionales
más capacitados que entiendan el comportamiento del país en su conjunto
y en ese manejo como se inserta.

Finalmente hay que incorporar la eficiencia y la transparencia para que
recuperemos la credibilidad.

La gobernabilidad podemos definirlo como la capacidad técnica y política
con la cual el estado a través del gobierno cuenta para dar solución a
las demandas de la población.

La gobernabilidad es la capacidad de manejo de todos los instrumentos
con los que cuenta el aparato estatal para atender las necesidades de la
población.

La población en su mayoría no siempre tiene razón y para eso están los
gobernantes, para eso está el estado, para descifrar las necesidades de
la población y atenderlos, no todo lo que dice la población es cierto y
prioritarios, por lo que debemos de aprender a priorizar ya que el
pueblo es como un niño y este es nuestro problema, que no sabemos
diferenciar.

La gobernabilidad pasa por esa capacidad que tiene el estado a través
del gobierno, capacidad de manejo que tiene de todas esas herramientas
que tiene que principalmente está plasmado en la leyes y eso traducido a
la ejecución y haya una decisión políticas para resolver ese problema
entonces podemos atender las demandas del gobierno, en tanto no se
atiendan las demandas priorizadas de la población por parte del gobierno
simplemente no abra credibilidad y abra desgobierno no hay presencia del
Estado.

Por lo que debemos de tomar enserio la gobernabilidad, usar bien las
herramientas y para eso debemos de dominar bien las leyes (ley de
adquisiciones, ley de presupuestos)

Para que la gobernabilidad funcione y las políticas públicas se hagan
efectivas tiene que partir por la voluntad política para su
implementación y esa voluntad política tiene que tener como sustento la
capacidad técnica del manejo gubernamental, necesitamos gente en el
aparato estatal que realmente sepa de gestión pública, no gente con
títulos o maestrías sino gente que sepa cómo se hace

La existencia de políticas públicas y la creación de más política
públicas no significa que la gobernabilidad va mejorar, hay que
preocuparnos en mejorar la capacidad técnica y política. Hay que mejorar
la toma de decisiones, pero lo principal es la capacidad técnica de los
que van a tomar las decisiones debajo de el en la ejecución, su
prioridad es contar con gente eminentemente práctica.

Para la buena gobernabilidad debemos de considerar algunos principios
básicos:

Dijimos que debemos de hacer reforma en áreas institucionales, en base a
4 grandes áreas o cuando decíamos que debe basarse la reforma estatal en
3 pilares como lo son:

\begin{itemize}
\tightlist
\item
  las estrategias nuevas y efectivas,
\item
  la reforma del resto de los poderes
\item
  la reforma del gobierno central
\end{itemize}

Podemos decir que tenemos algunos principios que cumplir dentro del
manejo gubernamental, por ejemplo para que haya gobernabilidad tiene que
haber percepción de legitimidad de parte de la población, la población
tiene que sentir que el gobierno que está en el poder refleja sus
interese (la percepción de legitimidad)

Cuando decimos que el gobierno no nos representa, estamos desconociendo
la gobernabilidad, estamos desconociendo al máximo representante del
estado en el gobierno (presidente de la república) le quitamos la
legitimidad y si gran parte de la población desconoce a ese gobierno se
pierde la percepción de legitimidad, pierde manejo gubernamental, la
población no cree en lo que puede hacer o en lo promete hacer y este es
un principio básico el manejo gubernamental de la existencia de la
gobernabilidad estatal.

Otro principio es la importancia que se le debe de dar al ciudadano, la
importancia central al papel del ciudadano, a la función que cumple el
ciudadano. De ahí es que la gente dice que el gobierno no está
atendiendo a los intereses del pueblo, oses que hay promesas, acciones
incumplidas por parte del gobierno y que la ciudadanía lo está haciendo
notar, por lo que el gobierno debe de cumplir, por lo que los ciudadanos
viene a ser el centro de atención, pero no necesariamente se tiene que
atender todas sus necesidades pero si se le tiene que hacer sentir al
ciudadano que es participe de la acciones del gobierno, hacerlo parte de
él.

ej carreteras de la selva: el gobierno le falta dinero, el gobierno
brilla por su ausencia no hay presencia del gobierno en el quehacer de
la población, lo que tenemos presencia de instituciones que no
funcionan.

\hypertarget{los-4-principios-buxe1sicos-de-la-buena-gobernabilidad}{%
\subsection{Los 4 principios básicos de la buena
gobernabilidad:}\label{los-4-principios-buxe1sicos-de-la-buena-gobernabilidad}}

Decimos que una institución es una estructura de orden social que hace
que la sociedad funcione por eso es que siempre nos vamos a preguntar y
vamos a preguntar a nuestros gobernantes que tipo de sociedad queremos,
cual es la imagen de sociedad que queremos, por eso que ingresa un
conjunto de variables en esa imagen de sociedad que queremos, pero lo
que hay es lo que podemos observar, lo que está ahí, la estructura en la
que las personas botan las basuras, las calles malogradas, etc. y esa es
la estructura social, por lo que tenemos que hacer un esfuerzo de cómo
se debe de entablar la normativa existente para que en esta sociedad las
normas de conducta se cumplan y las costumbres sean positivas, y esa son
normas que queremos imponer cuando en realidad deben de ir a ritmo de
los cambios que se generan en la normativa y la norma de conducta de la
sociedad no está bien implementada, divulgada por lo que hacemos lo que
queremos.

Y con esto concluimos con gobernabilidad e instituciones, está por
encima de todo la legitimidad del gobierno en los 3 niveles, la
legitimidad de los 3 poderes y para ganar esa percepción debemos de
mejorar la confianza, la importancia del ciudadano de su participación,
pero tampoco permitir que haga todo la ciudadanía, entonces siempre hay
que pensar que la población puede pedir todo pero que no siempre tienen
la razón y para esto están los técnicos y especialistas porque si bien
es cierto hay demandas insatisfechas pues hay que priorizar ya que de
ahí sale la calidad del gasto.

\hypertarget{la-prestaciuxf3n-de-servicios-publicos}{%
\section{LA PRESTACIÓN DE SERVICIOS
PUBLICOS}\label{la-prestaciuxf3n-de-servicios-publicos}}

Cuando hablamos de prestación de servicios públicos nos estamos
refiriendo al orden social pre establecido que pueden ser de servicios
públicos o privados, la sociedad se prepara para recibir los servicios
que brinda el sector público y el sector privado, generalmente los
servicios que se reciben del sector público son gratuitos o de precios
bastante bajos, sin embargo los servicios del sector privado si
responden a fines de lucro, los servicios privados van a generar
beneficios a los duelos o promotores de estos servicios, en tanto en el
sector publico precisamente porque están financiados con presupuesto
públicos, sus precios son relativamente bajos, pero ya se han
introducido de que debe de pagarse un costo, un precio relativamente
bajo subsidiado pero ya se debe de asumir.

Por eso se ha creado el texto único de procedimiento administrativos en
la cual se costea los servicios que brindan las instituciones, ej.: si
queremos sacar una constancia (o el pronunciamiento para titular
terrenos rurales vamos a solicitar una constancia negativa de zona no
catastrófica a la dirección regional de agricultura).

Entendemos que existen servicios que brindan tanto del sector privado y
publico Cuando hablamos de servicios para mejorar las condiciones de la
pobreza también podemos hablar de la inversión en infraestructura que se
efectúa en el Perú, los servicios de infraestructura a nivel del país
son muy escasos y volátiles en el entendido de que estas puedan volver a
ser como antes si es que no tienen un mantenimiento adecuado, es
insuficiente el servicio que brinda el sector público para ofrecer la
infraestructura necesaria para lograr un crecimiento dinámica del país
que se ha sostenido de la economía y hay que facilitar la reducción de
la pobreza, ósea la inversión en infraestructura es escasa e
insuficiente para lograr un crecimiento económico alto y que sea en el
mediano y largo plazo y que esta puede afectar positivamente a la
reducción de la pobreza.

Podemos decir que la inversión en el caso del Perú es muy baja por ej.
En obras de transportes e hídricas (trenes, estaciones, rieles,
aeropuertos, la infraestructura de riegos), esto se traduce en que los
costos económicos para el sector privado en realidad son muy altos y que
hay una inadecuada provisión de servicios públicos particular para los
pobres, no llega el servicio público, los pobres no son beneficiarios
directos de estos proyectos de inversión del sector privado, los
servicios públicos tienen que generalizarse y hacer un esfuerzo en
incorporar al mayor número posible de usuarios.

Para crear una infraestructura adecuada el Perú debería de invertir
aprox. el 4\% del PBI anual, es muy alto, está por encima de lo que
actualmente se invierte, tendríamos que duplicar, la diferencia en
infraestructura rural en diferencia con otras economías es bastante alto
dada las restricciones presupuestales, la insuficiencia de recursos
presupuestales \ldots\ldots sería bueno continuar o propiciar la
participación de la asociación publico privada, cómo hacer que el sector
privado se interés por obras que son propias del sector público, sin
embargo el sector público puede renunciar sus derechos y asignárselo al
sector privado y esta podría ser a través de las asociaciones público
privadas (apps) debemos de ver como esa asociación funciona y el
gobierno se ahorraría la asignación presupuestal correspondiente, porque
el gobierno no debería de asignar presupuesto, solo evaluar el
presupuesto, sincerar el presupuesto y aprobar e invitar al sector
privado interesado y a cambio de la inversión que va efectuar el sector
privado darle la oportunidad para que maneje esas vías o canales hasta
que recupere su inversión y tengan un nivel de utilidad, pero las apps
han perdido mucha fuerza porque habido mal manejo y en vez de que el
sector privado pusiera el capital lo a apuesto el gobierno y sin embargo
se les ha asignado para que manejen estas obras, por lo que se ha
generado una desconfianza en las asociaciones público privadas (apps)
esta asociación que hasta ahora se ha practicado debería de mejorar el
diseño de las concesiones para ofrecer seguridad financiera y
contractual. Lo cual significa que ambas partes deberían de estar en la
capacidad de financiar y firma los contratos respectivos y pre
establecer acuerdos en la que la prioridad es ganar.

Debemos de hacer que las concesiones sean atractivas y que la población
vuelva a confiar, lo que se debe de hacer es que la población sea
consiente.

Debemos de asignar apropiadamente los recursos y reducir los riesgos en
la que ambas partes ganen tanto el sector privado como el gobierno.

Resolviendo las preocupaciones sociales de la población podría
consolidarse y desarrollarse mercados financieros locales, capital local
incorporado en la parte financiera (ej. Cooperativas).

La infraestructura productiva es bastante limitada en el caso del Perú,
es deteriorada e insuficiente, contribuye a la falta de competitividad
de la industria y del sector agrario, porque los costos del transporte
se elevan y esto hace que pierdan la oportunidad en el mercado por lo
que ese debe de elevar y fomentar la inversión en infraestructura básica
para mejorar la competitividad.

El crecimiento económico debería de ser a mayor velocidad, pero no es
posible en tanto la mejora de la competitiva se retrase por lo que la
preocupación del desarrollo regional debe de ser la reducción de la
pobreza en la región, se debe de incrementar para ello los niveles de
inversión en infraestructura.

La competitividad frente al agro frente a la industria de la zona
urbana, el área rural está en permanente desventaja.

Hay una necesidad de incrementar el nivel de inversión en
infraestructura.

Debemos de aumentar los beneficios de las asociaciones público privado,
debemos de hacerlo más atractivo debemos de darles mayores niveles de
rentabilidad, mayor tiempo para que administren el abastecimiento por
ejemplo de agua.

Debemos de ir a las poblaciones rurales a las zonas urbanas marginales y
explicar por qué se deben de hacer concesiones, debemos de superar ese
problema explicando el porqué del rechazo social, rechazo de la
población a las concesiones y hacer entender a la población que si bien
es cierto han tenido la razón en su momento, en la actualidad ya no está
de acuerdo que las necesidades institucionales y a las necesidades da la
población pobre por lo que se debe de recuperar la confianza en las
inversiones publico privados.

Debemos de resolver los problemas de conflicto de rechazo social a las
concesiones haciendo que estas sean efectivas y que realmente comprendan
que la empresa que está cobrando el peaje, esos fondos los van a
reinvertir en el mantenimiento y además tiene que ganar cierto nivel de
utilidades y que eso tampoco está prohibido ya que las empresas están
hechas para ganar, es más, estas instituciones publico privadas podría
explicar el rechazo social a las concesiones. Por lo que hay que
resolver esos problemas y hacer entender a la población de sus
necesidades y la necesidad ya es concesionarlo porque ellos no están en
la capacidad de hacer un mantenimiento de las pistas.

Otra razón por la que la infraestructura productiva es bastante limitada
es que los productores no se atreven en muchos casos a atraer a los
mercados locales a los inversionistas institucionales también locales.

Otra situación que debemos de considerar es el acceso insuficiente y con
calidad de los servicios del aparato estatal, del proceso de
centralización es bastante avanzada pero es incompleto por lo que los
servicios que brinda los gobiernos regionales son pésimos, los efectos
en la eficiencia y eficacia del gasto público en realidad deja mucho que
desear porque las regiones no destinan los fondos del presupuesto
público a proyectos debidamente priorizados sino a proyectos debidamente
politizados.

Por lo que debemos de cerrar estas indiferencias en la infraestructura

\begin{quote}
Tenemos sobrevaloración y sobredimensionamiento de obras
\end{quote}

TAREA: TITULO 3 DE LA CONSITUCION DE LA REPUBLICA DEL REGIMEN ECONOMICOS
DEL Articulo 58 Hasta el 89

RESUMEN:

Cuando hablamos de prestación de servicios públicos decíamos que las
inversiones que se han hecho infraestructura para brindar los servicios
del sector público o del aparato estatal era insuficiente y que debimos
concentrar el esfuerzo en obras especialmente de transporte, obras
hídricas y que esto se traducían en costos económicos para el sector
privado y que también la provisión de los servicios públicos era
inadecuada para los pobres y extremo pobres.

Para crear una infraestructura adecuada decíamos que en el Perú se debía
de invertir el 4\% del PBI de forma anual y eso para alcanzar aprox. en
8 a 10 años y cerrar las diferencias que tenemos entre el área urbana y
rural y las diferencias que tenemos con el exterior, para nivelarnos
internacionalmente con el grado de competitividad que ellos tienen,
porque esta infraestructura que no es adecuada hace que sea
incompetentes en las diversas actividades, el aparato productivo de
muchos productos que exportamos de por si es altamente competitivo en la
zona de producción, lo que pierde competitividad es entre la zona de
producción y la zona de entrega al exterior, por eso decimos que es
importante ya que estamos en una etapa difícil económicamente, el
presupuesto está en déficit permanente en estos 2 últimos años, es
importante la participación de las asociaciones público privadas, ya que
estas nos podrían ayudar para dar el mejoramiento de toda la
infraestructura vía concesiones, por lo que deberíamos de diseñar el
formato de las negociaciones con el sector privado para ofrecerles
seguridad financiera y contractual, por lo que debemos de garantizar los
contratos y este es justamente le candado que cualquiera de nosotros
quisiera.

Por lo que será necesario consignar con todos riesgos pero considerando
la seguridad financiera y contractual que el estado a través del
gobierno ofrece, de modo que incluso podemos desarrollar los mercados
financieros y locales, a esto de las concesiones podríamos para
garantizar la ejecución de los grande proyectos, pero principalmente que
las asociaciones público privados (las apps) que por nuestra restricción
presupuestal estamos diciéndole al sector privado que pongan partes del
financiamiento que podría ser del 60\%, 50\% en el capital, y a mayor
capital que ponga el sector privado habría que darle mayor concesiones,
mayores beneficios ya que estaría arriesgando más, en el caso del Perú
las cosas han ocurrido a la inversa, el Gobierno ha dado todo el
presupuesto en concesión, es por eso que las apps han perdido
legitimidad en el caso peruano por lo que hay que recuperar ese
prestigio perdido, hay que hacer que las concesiones realmente funcionen
Ej. Ayacucho-San Clemente, el tramo a la selva deberíamos de concesiona,
pero que se cobre un monto relativamente bajo porque este necesita
mantenimiento, cobrar peajes por tramos.

El déficit que tenemos lo vamos a financiar con endeudamiento público y
quien nos va financiar eso, teniendo en cuenta que tenemos que tener
proyectos no sobrevaluados ni dimensionados, una vez que hayamos hecho
los proyectos con precios reales vamos a buscar créditos, para lo cual
nos puede prestar una entidad externa, el Banco mundial, fondos o de
gobierno a gobierno en al que las empresas de los países europeos
ingresas a través de sus gobiernos y como nos van a prestar dinero para
cubrir nuestros déficit pero ellos van a realzar nuestro proyectos
(pistas) en la que ellos nos prestan y hacen la obra, que es una forma
de ingresar y mejora la calidad del gasto así como la ejecución, ya que
los créditos exteriores siempre viene condicionadas.

La cual es una alternativa muy viable y que es factible y para poner en
duda esto, los convenios contratos de gobierno a gobierno lo que tenemos
que hacer es:

\begin{itemize}
\tightlist
\item
  Proyectos sincerados
\item
  Los gobiernos tiene que tener negociaciones trasparentes (no
  sobrevalorados)
\end{itemize}

pero es interesante en tanto podamos hacer negociaciones de deuda
pública con determinado tipo de empresas u origen de endeudamiento o
capitales para financiar proyectos de inversión Ej: Club de parís (grupo
de bancos para financiar las obras del mundo).

Por lo que tenemos dos formar de captar fondos de manera adecuada y
financiar el presupuesto y mejorar las condiciones de servicios:

\begin{itemize}
\tightlist
\item
  Las concesiones
\item
  Convenio contratos de gobierno a gobierno
\end{itemize}

Ya que nuestra infraestructura está deteriorada ya que en los últimos
años no hemos hecho mantenimientos, lo que existía es insuficiente,
reduce la competitividad de la industria, hay que darle fomento a la
infraestructura, debemos de elevar el destino del presupuesto, ya que
sería interesante llegar hasta el 4\% del PBI para comenzar a cerrar las
diferencias, hay que darle mayores beneficios atractivos a las empresas
que inviertan en proyectos de infraestructura simplemente reduciendo la
corrupción podemos incentivar mucho más, por lo que debemos de cambiar
la mentalidad de las personas que es muy difícil y buscar explicar para
resolver los problemas de rechazo social a los proyectos de inversión
porque todas ellas son cuestionadas por lo que hay que reducir los
indicios de corrupción, hay que atraer las inversiones de las pensiones
locales y las AFPS, la ONP también puede convertirse como una AFP para
lo cual deberíamos de sacar una ley para que comience a invertir y
compita con las AFPS para elevar las tasas, hay que mejorar las empresas
aseguradoras para asegurar los riesgos de los sectores principalmente en
el sector agrario en vez de dar los seguros agrarios, sería bueno dar a
cambio infraestructura, hacer en las zonas más afectadas trabajos de
investigación o infraestructura que evite o reduzca los riesgos (Ej:
cercos para reducir los efectos de las heladas, instalaciones de agua
que podrían ser a través de mangueras) en todo caso cerras estas
diferencias en la infraestructura en el mediano y largo plazo, el gasto
público debe ser descentralizado, pero hay que mejorar la eficiencia y
eficacia del gasto público en las regiones y gobiernos locales.

\hypertarget{desarrollo-territorial}{%
\subsection{DESARROLLO TERRITORIAL}\label{desarrollo-territorial}}

El desarrollo territorial se debe de generar evitando la dispersión en
proyectos atomizados y fomentando la coinversión público-privada
aprovechando las leyes de la asociación Público-Privada y seguir
fortalecimiento los mecanismos de descentralizados de la toma de
decisiones.

La experiencia existente en el Perú debe de ser de formas de
participativa concertadas a nivel de los gobiernos locales y Regionales,
los proyectos para fomento el desarrollo deben de ser altamente
productivos y principalmente deben de estar en las áreas rurales.

Debemos de hacer un orden de prioridades de modo que podamos identificar
los proyectos más importantes en la que la población se sienta parte
integrante de las actividades que ejecuta el gobierno en sus diferentes
niveles.

Las políticas públicas son herramientas del Estado al servicio de la
sociedad.

El territorio de la república del Perú está integrado por regiones,
departamentos, provincias, distritos y centros poblados.

En estas circunscripciones geográficas se constituye y se organiza el
Estado y el Gobierno para un buen manejo se va dividir en niveles
Nacionales, Regionales y Locales.

Por un lado tenemos una división territorial basada en departamentos
provincias, distritos y centros poblados y considerando esta división
del país territorialmente el Gobierno se organiza y constituye a nivel
nacional, regional y local (3 niveles de gobierno) y es aquí donde
podemos hablar de niveles de gobierno.

\hypertarget{cuxf3mo-planteamos-el-desarrollo-considerando-e-territorio}{%
\subsection{¿Cómo planteamos el desarrollo considerando e
territorio?}\label{cuxf3mo-planteamos-el-desarrollo-considerando-e-territorio}}

El país territorialmente está dividido en 4 áreas: Departamentos o
Regiones, Provincias, Distritos y centros poblados y sobre estas el
gobierno se organiza, el que está en el gobierno central (nacional) es
el que manda a todos, a nivel de regiones hay una autoridad regional, en
las provincias existen las autoridades provinciales y a nivel local
están las autoridades locales de modo que se puede notar la presencia
del estado a través del Gobierno.

Políticamente el gobierno se divide en 3 poderes: Poder Ejecutivo
(Gobierno Nacional), Poder Legislativo y el Poder Judicial que imparte
justicia, estos poderes son independientes.

El Gobierno Nacional, es el que manda en todo el territorio nacional, su
mandato es transversal; el Gobierno Regional manda en su departamento
pero este no puede mandar en el Gobierno Nacional pero si puede
compartir responsabilidades y funciones (responsabilidades compartidas).

El Gobierno Nacional se representa a través de los diferentes
ministerios o los organismos públicos descentralizados (OPDES); los
ministerios que son los representantes del poder ejecutivo en el
Gobierno Nacional son los entes rectores, son las instituciones que van
a generar las normativas de aplicación a nivel Nacional y esa normativa
van a ser aplicadas a través de los gobiernos Regionales o Locales , y
cada uno de estos niveles de Gobierno conforme a sus competencias y
autonomías propias van a actuar para el bien común, para el bien de la
población, sin embargo se debe de respetar y preservar la unidad e
integridad del Estado como Nación (Se refiere principalmente a los
gobiernos regionales que se encuentran en las fronteras, los
gobernadores regionales no pueden regalar el territorio ni el presidente
de la república, nadie puede entregar parte de nuestro territorio).

Sin embargo en los últimos años se ha comenzado a notar que los
gobiernos Regionales y Locales hacen lo que creen por conveniente, ósea
que la normativa que genera el Gobierno Central no se cumple a cabalidad
y los Gobiernos Regionales principalmente se convierten en un obstáculo
para el cumplimiento de la normativa y En ese entender el Gobierno
Nacional está considerando reabsorber las funciones de los gobiernos
Regionales y Locales (Ej: sector Salud).

Debemos de defender el proceso de descentralización y mejorar a la hora
de elegir a nuestros representantes.

El gobierno Nacional tiene jurisdicción en todo el territorio nacional,
los gobiernos Regionales en su ámbito, los gobiernos locales dependen de
si es provincial en toda la provincia, si es distrital en todo el
distrito y si es en un centro poblado, en todo ese centro poblado; su
circunscripción territorial corresponde a cada nivel de gobierno según
el área geográfica.

Si hablamos de los gobiernos locales tenemos que decir que el nivel
provincial manda en los distritos, Ej: el alcalde de la municipalidad
provincial de Huamanga en teoría está sobre los alcaldes distritales y
en la práctica el alcalde no tiene mando ni jurisdicción sobre los
distritos por desconocimiento y falta de liderazgo.

La preferencia básica del gobierno en sus distintos niveles es el
interés público.

El desarrollo territorial se debe de generar considerando evitar la
dispersión de los proyectos, no se debe de atomizar los proyectos, Ej.:
en la región de Ayacucho no debemos de hacer pequeños proyectos por
todas partes, debemos de hacer proyectos reales que tengan un efecto
multiplicador que afecten a otros sectores, de modo que colateralmente
va desarrollar otras actividades, y si no hay dinero el financiamiento
seria fomentado por las asociación publico privado y fortalecer la toma
de decisiones, hay que fortalecer la descentralización, debemos de hacer
proyectos concertados.

La propuesta parte de la experiencia, en el Perú existen plataformas
participativas, concertación local ya que uno de los principios es poner
en el centro de atención a la ciudadanía, el pueblo nunca va tomar las
decisiones ya que se toman decisiones en nombre del pueblo que se tienen
que asumir en su momento.

Debemos de fomentar los proyectos que sean productivos principalmente en
el área rural y ahí viene el secreto del desarrollo rural vía
procedimiento de desarrollo territorial.

Las políticas públicas debemos de tomarlas en cuenta para afianzar las
decisiones que estamos tomando, para respaldar nuestra decisión porque
tenemos que ver el interés público.

La política pública busca el interés público y el bien común, las
políticas públicas con el bien común se ven bien interrelacionados con
el interés público por lo que se puede actuar con la normativa
existente, porque las herramientas del Estado están al servicio de la
sociedad, están al servicio de la población en su conjunto pero lo que
pasa es que la normativa existente están hechas solo pensando en el
interés privado porque nuestros gobernantes se han convertido en
gestores de las empresas privadas y no en gestores públicos.

Por lo que según las teorías económicas tenemos que crear las
condiciones favorables para que se desarrolle el sector privado, no dice
que hagamos las cosas para favorecer a la empresa privada sino que se
tiene que crear las condiciones (ej. Arreglar pistas, veredas,
carreteras, etc).

Cuándo estamos en el sector público y estamos en el área de
contrataciones en el momento de las calificaciones somos enemigo del
sector privado porque tenemos que ser meticulosos y ver que cumplan todo
los requisitos y lo que se va hacer, que alcancen los objetivos del
proyecto.

Cuando hablamos del enfoque territorial es necesario implementar
intervenciones diferenciadas según lo diferentes tipos de carencias,
patrones de desarrollo de cada territorio, se debe de estudiar cada zona
detenidamente y encontrar las carencias principales para que ellos con
esfuerzo propio puedan salir de la situación de pobreza y extrema
pobreza en la que se encuentra ya que las autoridades están justamente
para intervenir de acuerdo a esas carencias diferenciadas y que
seguramente 2 o 3 necesidades que podrían implementarse podrían servir
para reducir la pobreza existente en la zona.

Debemos de identificar el piso ecológico en el que se encuentran,
identificar qué tipo de cultivos se va a dar con ese rio, debemos de
trabajar en un cambio de los patrones de desarrollo de ese territorio,
debemos de ver el tipo de cultivos, ver el circuito de la distribución,
el acceso hacia el mercado, por lo que si resolvemos el problema del
agua, el tipo de cultivo, el acceso a esa zona para mejorar la
carretera, efectivamente podría salir de la pobreza esa población.

Por lo que cada zona de intervención con este enfoque territorial es un
caso totalmente distinto, no se pueden hacer intervenciones
generalizadas por eso es que no tiene efecto la intervención del Estado,
por lo que debemos de diferenciarlos, ver los patrones al interior e
intervenir recién.

Cada zona de intervención previamente debe de ser analizada e
investigada de modos que cuando se ponga el esfuerzo estatal esta tenga
un efecto real en la que el sector privado incluso pueda intervenir.

Cada centro poblado en su área tiene sus propias valoraciones, tiene
diversas prioridades en la que se tiene que ver lo que realmente
necesita la población, en la que tenemos que ingresar trabajando con la
población para que la población involucrada se incorpore e identifique
con la intervención y el trabajo que efectúa el sector público y
hacerles entender que la política del gobierno es ayudarlos pero no es
hacer todo por ellos, ya que la población tiene que asumir sus
responsabilidades.

El sentido de un enfoque territorial para el desarrollo es que va
permitir que un área geográfica mejore su condición económica, es tener
una mirada de coordinación y articulación del gobierno sea de cualquier
nivel, que puede ser del gobierno central o nacional, regional o local
con participación del sector privado y de los productores. El gobierno
tiene que hacer ese nivel de coordinación, tiene que tener esa mirada de
articular y juntar los esfuerzos a fin de que participe el sector
privado y los productores y a los productores el gobierno va apoyándolos
con subsidios de modo que sea transitorio y no permanente, el productor
tiene que tener beneficios tiene que tener utilidades, toda actividad
económica tiene fines de lucro, de obtener beneficios y utilidades.

El desarrollo territorial se debe de entender como un proceso de
construcción social de todo el entorno en el que se desenvuelve la
población perteneciente a una determinante área geográfica y que hay una
interacción entre las características geográficas del área porque el
ambiente va afectar, el piso ecológico en la que se desenvuelve, las
iniciativas individuales y experiencias tanto individuales como
colectivas deben de ser valoradas e incorporadas en esta política de
mejora del desarrollo territorial con enfoque territorial, cuando el
estado a través del gobierno llega para intervenir en un área geográfica
previamente elegido se debe aprovechar las experiencias de los
pobladores, experiencias individuales por lo que se debe de incorporar
en el trabajo esas iniciativas y experiencias individuales así como
colectivas y ver la operación de las fuerzas económicas, cuales son las
perspectivas de cada uno de ellos, que buscan para el futuro

Debemos de ver hasta qué punto cada uno de ellos están aptos a optar
tecnologías nuevas, que nivel de permeabilidad hay en esa poblacion para
captar el avance tecnológico que podría incorporarse y podría ser
transferido a la zona de trabajo y cuan fácil creen ellos que es
incorporar tecnologías nuevas y si bien es cierto captan rápidamente las
tecnologías pero estas no son aplicadas adecuadamente.

Debemos de analizar la situación sociopolítica, aquí entra el análisis
de conflictos, debemos de ver hasta qué punto la poblacion tiene
protestas coherentes o protestas permanentes o si es una poblacion
pacifica que absorbe la tecnología y acepta la intervención del Estado y
contribuye en mejorar su situación.

Debemos de analizar en el territorio en el cual vamos a intervenir como
está organizado la poblacion y que tipo de conflictos tiene, por lo que
primer debemos de resolver esos problemas antes de ingresar con la
intervenir del Estado.

Otro factor que es muy importante es la cultura, la cultura existente en
la zona hay que aprovecharla, canalizarlo para el bien (Ej.: aprovechar
los carnavales para reuniones más productivas).

Los factores ambientales, incorporar si son más vulnerables o no para la
quema de rastrojos, la poda de los arboles deben de seguir en el mismo
lugar y no trasladarlos o quemarlos.

En el territorio tenemos que ver qué acciones podemos iniciar con éxito
para contribuir y hacer que la población sea par, de eso trata el
desarrollo territorial y el enfoque territorial participativo en
realidad corresponde a un proceso planificado y aplicado a un territorio
que está constituido por una población x o socialmente constituido y que
es una zona de interrelación de diversas instituciones gubernamentales o
privadas que son las provincias, distrititos y centros poblados y en la
que está el sector público con el sector privado y también los factores
sociales, la interculturalidad que tenemos a nivel nacional. Por lo que
debemos de hacer consensos institucionales, debemos de crear consensos
de las comunidades que van a participar de repente hay similitudes y
podemos intervenir varias comunidades a la vez, varios espacios
geográficos y ver las zonas de mayor concentración de los centros
poblados, como operan la actividad económica, las tradiciones
culturales, la historia de esos pueblos y aquí hay factores sociales que
no siempre es puro ingreso.

El núcleo central de la propuesta del desarrollo territorial con enfoque
territorial es:

\begin{enumerate}
\def\labelenumi{\arabic{enumi}.}
\tightlist
\item
  El fortalecimiento de entidades público-privadas y para ello debemos
  de recuperar la confianza en la población y cuando sea necesario la
  constitución de acciones de índole provincial, departamental con
  coordinación previa del presupuesto público (intervención del Estado),
  y si va participar el sector privado darle las facilidades para su
  intervención, implementación e instalación.
\item
  La preparación de estas entidades de programas de inversión que deben
  de ser ordenadas en torno de ejes de prioridades de desarrollo
  previamente identificados. Ej.: El sector agrario directamente
  relacionado es el eje del riego y provisión del agua, recursos
  hídricos y las posibilidades de donde llevar, otro eje son las
  cosechas de agua, los reservorios que es de almacenamiento de épocas
  de lluvia y en épocas de estiaje se empieza a consumir esa agua dado
  que no hay riego permanente, otro eje son los caminos vecinales,
  debemos de ver los programas de inversión y que posibilidades de que
  el aparato estatal intervenga con éxito.
\item
  El cofinanciamiento de los programas a través de mecanismos
  financieros competitivos. Podemos dar créditos con altas tasas de
  interés, pero debemos de ver la rentabilidad del agro, no debemos de
  prestarle poco interés, debemos focalizar el desarrollo territorial y
  en ese enfoque vamos a fijar áreas de intervención y eso va ayudar a
  mejorar las condiciones económicas de la población de ese piso
  ecológico.
\end{enumerate}

Estas propuestas en realidad son propuestas que se inscriben en la
visión territorial del desarrollo rural principalmente, en las zonas
urbanas también se puede usar el enfoque territorial porque tenemos que
estudiar zonas puntuales, zonas en la que hay mucha concentración de
pobreza y es necesario profundizar el desarrollo descentralizado en el
país, la descentralización es un vehículo de aplicación de la estrategia
nacional del desarrollo rural ya que la descentralización es la vía
mediante el cual podemos llegar a la población más necesitada del área
rural, porque las regiones están más cerca de su población ya que el
gobierno nacional está alejado y a través de los gobiernos regionales y
locales está tratando de llegar directamente al beneficiario sin embargo
en la práctica podemos ver que los gobiernos regionales y locales no
están en la capacidad de implementar las políticas nacionales,
simplemente los actores políticos de este tipo de gobiernos desconocen y
no tiene la voluntad de implementar políticas con enfoque territorial de
modo que podrían ser exitosos.

Tenemos que hacer un desarrollo con enfoque territorial y para esto
debemos de visualizar la descentralización basada en cuencas
hidrográficas por donde están los ríos centrales de determinadas zonas
en el Perú y todo lo que abarca ese rio va ser un enfoque, una zona de
intervención (Ej. Cuenca del cachi).

El enfoque del desarrollo territorial está centrado en el desarrollo
productivo y los elementos básicos para el desarrollo productivo son
multisectoriales ya que va necesitar la intervención de varios sectores,
de varias direcciones regionales y varias unidades locales, debemos de
construir todo lo necesario para la construcción de alianzas
público-privadas, y si las necesidades son de mayor financiamiento
presupuestal debemos de ver la capacidad de endeudamiento de los
gobiernos regionales, la planificación estratégica es importante como
instrumento para determinar las diversas fuentes de financiamiento, ya
que si previamente se han hecho los estudios estaríamos garantizando el
éxito de la intervención en el territorio elegido por lo que tendríamos
una conglomeración de inversiones que generen expectativa.

La construcción de mecanismos de apoyo al desarrollo territorial a
partir de la realidad de las zonas a intervenir cuyos elementos
centrales del mecanismo son:

\begin{enumerate}
\def\labelenumi{\arabic{enumi}.}
\tightlist
\item
  Organización de los actores, a los de la población involucrada de cómo
  se organizan (por terreno, sexo) como se distribuyen el agua (por
  tamaño de terreno por tipo de cultivo)
\item
  La provisión de recursos, de cómo se proveen de agua, como es que
  llegan los fertilizantes e insecticidas, semillas y quienes son los
  proveedores. Debemos de ver los canales de provisión de recursos para
  reducir los costos de producción en la zona
\item
  El establecimiento de procedimientos, en cada segmento de producción
  la población conformado por pequeños agricultores de la zona deben de
  conocer cuáles son las serlas de juego, cuales son los procedimientos
  que se deben de cumplir para determinados tramites o para ser
  beneficiarios de algunas atenciones de parte del gobierno.
\end{enumerate}

\hypertarget{ley-de-bases-de-la-descentralizaciuxf3n}{%
\section{LEY DE BASES DE LA
DESCENTRALIZACIÓN}\label{ley-de-bases-de-la-descentralizaciuxf3n}}

Artículo 29.- Conformación de las regiones

29.1. La conformación y creación de regiones requiere que se integren o
fusionen dos o más circunscripciones departamentales colindantes, y que
la propuesta sea aprobada por las poblaciones involucradas mediante
referéndum.

29.2. El primer referéndum para dicho fin se realiza dentro del segundo
semestre del año 2004, y sucesivamente hasta quedar debidamente
conformadas todas las regiones del país. El Jurado Nacional de
Elecciones convoca la consulta popular, y la Oficina Nacional de
Procesos Electorales (ONPE) organiza y conduce el proceso
correspondiente. 29.3. Las provincias y distritos contiguos a una futura
región, podrán cambiar de circunscripción por única vez en el mismo
proceso de consulta a que se refiere el numeral precedente.

29.4. En ambos casos, el referéndum surte efecto cuando alcanza un
resultado favorable de cincuenta por ciento (50\%) más uno de electores
de la circunscripción consultada. La ONPE comunica los resultados
oficiales al Poder Ejecutivo a efecto que proponga las iniciativas
legislativas correspondientes al Congreso de la República.

29.5. Las regiones son creadas por ley en cada caso, y sus autoridades
son elegidas en la siguiente elección regional.

29.6. La capital de la República no integra ninguna región.

29.7. No procede un nuevo referéndum para la misma consulta, sino hasta
después de seis (6) años.

Este artículo da la posibilidad de que las regiones pueden crecer,
pueden juntarse las regiones o pueden anexarse a determinadas regiones
de modo que puedan integrar zonas muy cercanas, zonas de relación
directa o que pueden juntarse áreas geográficas contiguas pero
complementarias económicamente Ej.: Ica-Ayacucho que están en el mismo
corredor y que los obligaría a juntarse las dos carreteras libertadores
y la de puquio y a raíz de eso pueden integrarse más, en el 2006 hubo un
referéndum en la que también participo Ayacucho (Puquio quiso anexarse a
Ica), pero es casi imposible que dos regiones se junten porque los
gobernadores ganaron las elecciones y perderían el poder si esto llegara
a pasar, el egoísmo provinciano hace que no nos juntemos, pero que
atreves de referéndum podemos juntar 2 zonas contiguas pero
complementarias económicamente.

Artículo 18.- Planes de desarrollo

18.1. El Poder Ejecutivo elabora y aprueba los planes nacionales y
sectoriales de desarrollo, teniendo en cuenta la visión y orientaciones
nacionales y los planes de desarrollo de nivel regional y local, que
garanticen la estabilidad macroeconómica.

18.2. Los planes y presupuestos participativos son de carácter
territorial y expresan los aportes e intervenciones tanto del sector
público como privado, de las sociedades regionales y locales y de la
cooperación internacional.

18.3. La planificación y promoción del desarrollo debe propender y
optimizar las inversiones con iniciativa privada, la inversión pública
con participación de la comunidad y la competitividad a todo nivel.

En la práctica el plan Nacional de Desarrollo lo hace el centro de
planeamiento y en base a este plan de desarrollo Nacional cada sector a
nivel del poder Ejecutivo hace su propio plan de Desarrollo Nacional
sectorial y del plan de desarrollo Nacional los Gobiernos Regionales
hacen el plan de desarrollo Regional y de este plan de desarrollo
Regional se saca los planes de desarrollo sectorial (DIRESA,EDUCACION
AGRICULTA Y TRANSPORTE) y adicionalmente a ese plan de desarrollo
Regional tiene que engancharse del plan de desarrollo sectorial
ministerial; este artículo se refiere a que el país se debe de manejar
en base a un plan de desarrollo de modo que no se salga del control
nacional y es por eso que debemos de mantener la estabilidad
macroeconómica porque este plan no se puede salir del marco presupuestal
por lo que tiene que estar directamente ligado con el presupuesto
Nacional y a esto la participación del sector privado, cuando se hace el
plan de desarrollo Nacional se supone que está involucrado el Gobierno,
el Estado a través del Gobierno, el sector privado y todo el potencial
que tiene el país, por lo que el artículo se está refiriendo a la
relación que debe de existir entre planes y presupuesto, como es que
desde el gobierno central de ese plan a los sectores a nivel de poder
ejecutivo, Gobiernos regionales, sectores y los Gobiernos locales se
debe de manejar a través de un plan de desarrollo, nada debe de ser
improvisado, tiene que haber un orden de prioridades, por lo que
ordenadamente debemos de generar las inversiones necesarias para el
desarrollo del país. El primer reto que debemos de romper es que los
gobernantes, los que toman las decisiones tienen que saber de la
importancia del plan de desarrollo. Debemos de relacionar planes con
presupuestos.

Artículo 21.- Fiscalización y control

21.1. Los gobiernos regionales y locales son fiscalizados por el Consejo
Regional y el Concejo Municipal respectivamente, conforme a sus
atribuciones propias.

21.2. Son fiscalizados también por los ciudadanos de su jurisdicción,
conforme a Ley.

21.3. Están sujetos al control y supervisión permanente de la
Contraloría General de la República en el marco del Sistema Nacional de
Control. El auditor interno o funcionario equivalente de los gobiernos
regionales y locales, para los fines de control concurrente y posterior,
dependen funcional y orgánicamente de la Contraloría General de la
República.

21.4. La Contraloría General de la República se organiza con una
estructura descentralizada para cumplir su función de control, y
establece criterios mínimos y comunes para la gestión y control de los
gobiernos regionales y locales, acorde a la realidad y tipologías de
cada una de dichas instancias.

Cuando hablamos de fiscalización, hablamos de las actividades económico
financieras que realizan el aparato estatal o el Estado a través del
Gobierno y que debe de estar basado principalmente en la legalidad,
eficiencia y en la economía de modo que se cumplan con los objetivos de
la institución, y cuando hablamos de control nos estamos refiriendo a
cuanto de esto se ha cumplido en su ejecución, por un lado se debe de
controlar cuan eficiente ha sido las decisiones que tomaron,
técnicamente hasta qué punto fue correcto la decisión de haber ingresado
a ese local considerándolo como almacén, hasta qué punto esa decisión
permitió cumplir con el objetivo ( esta es la parte de fiscalización),
en la parte de control vemos si legalmente se ha ceñido a las normas, si
no hay daños y perjuicios a la institución y en esa búsqueda de la
fiscalización y el control en el Perú hay un sistema de control que
representa según el lugar la oficina de control institucional y a nivel
nacional la contraloría general de la república. Y quienes fiscalizan,
quienes pueden decir que es ilegal, que es antieconómico, que no es
eficiente esa actividad o decisión es la población, la ciudadanía, la
población puede detectar y dentro de esta ciudadanía también están los
trabajadores, por lo que la contraloría general de la república tiene
mucha responsabilidad pero esta no es tan grande por lo que al azar
elige e intervine porque no tiene tanta gente para intervenir.

Artículo 23.- Consejo Nacional de Descentralización

23.1. Créase el Consejo Nacional de Descentralización (CND) como
organismo independiente y descentralizado, adscrito a la Presidencia del
Consejo de Ministros, y con calidad de Pliego Presupuestario, cuyo
titular es el Presidente de dicho Consejo.

23.2. El Consejo Nacional de Descentralización será presidido por un
representante del Presidente de la República y estará conformado por dos
(2) representantes de la Presidencia del Consejo de Ministros, dos (2)
representantes del Ministerio de Economía y Finanzas, dos (2)
representantes de los gobiernos regionales, un (1) representante de los
gobiernos locales provinciales y un (1) representante de los gobiernos
locales distritales. (1)(2)(3) (1) De conformidad con el Artículo 1 de
la Resolución Suprema N° 393-2002-PCM, publicado el 05- 09-2002, se
designa al señor Luis Alberto Thaís Díaz, como Presidente del Consejo
Nacional de Descentralización (CND) por el plazo señalado en el numeral
23.4) del presente artículo. (2) De conformidad con el Artículo 1 de la
Resolución Suprema N° 394-2002-PCM, publicado el 05- 09-2002, se
designan a los representantes de la Presidencia del Consejo de Ministros
y del Ministerio de Economía y Finanzas, por los plazos a que se refiere
el numeral.

El consejo nacional de descentralización funciono a los inicios del
proceso de descentralización hasta el 2007 y fue aquí que se desactivo,
este fue el jefe de todos los presidentes Regionales, ya que todos los
presidentes regionales tenían que coordinar con él para hacer cualquier
cosa, era el jefe supremo de los presidentes regionales en tanto se
implementara el proceso de regionalización, este era el intermediario de
los gobiernos regionales con todos los ministerios, y hoy en día es una
dirección más, es el que hace seguimientos de los indicadores de las
regiones, llama a los presidentes de las regiones para reuniones, para
que estén con el presidente y lo respalden, ya perdió su razón de ser,
por intermedio de una ley lo desactivaron en el 2007, el consejo
nacional de descentralización incluso se involucraba en la designación
de directores regionales, es un artículo que ya perdió vigencia, que ya
no funciona que se creó con buenas intenciones, ha perdido vigencia
porque fue un obstáculo que no permitía a los presidentes regionales
hablar directamente con los ministros o el presidente de la república,
era un obstáculo para el proceso de regionalización que en nada ayudo
que más bien le quito velocidad al proceso de descentralización.

\hypertarget{ley-organica-de-gobiernos-regionales}{%
\section{LEY ORGANICA DE GOBIERNOS
REGIONALES}\label{ley-organica-de-gobiernos-regionales}}

\hypertarget{ruxe9gimen-normativo}{%
\subsection{RÉGIMEN NORMATIVO}\label{ruxe9gimen-normativo}}

Artículo 36.- Generalidades

Las normas y disposiciones del Gobierno Regional se adecuan al
ordenamiento jurídico nacional, no pueden invalidar ni dejar sin efecto
normas de otro Gobierno Regional ni de los otros niveles de gobierno.
Las normas y disposiciones de los gobiernos regionales se rigen por los
principios de exclusividad, territorialidad, legalidad y simplificación
administrativa. En los regímenes normativos las resoluciones y las
normas que emanan de los gobiernos regionales o consejos regionales de
cada gobierno regional solo es aplicable en ese territorio, no es
extensivo a otras regiones, no se aplican en otras regiones, pero cuando
es del gobierno Nacional es aplica para todo el país, esto también
funciona en los gobiernos locales por extensión ya que si saca una
ordenanza regional o una resolución de alcaldía el distrito de Carmen
alto no es aplicable para san juan y otros distritos, sino
exclusivamente en ese territorio, los gobiernos regionales van a sacar
normativas para su aplicación en el territorio Regional principalmente
considerando las funciones exclusivas que han sido delegadas a los
gobiernos regionales, no pueden sacar resoluciones que afecten a los
gobiernos locales porque estas ya serian funciones compartidas y si son
funciones compartidas lo que hay que hacer es que estas se pongan de
acuerdo y saquen una sola normativa complementaria, y sobre la legalidad
de la normativa que se genera en los gobiernos Regionales estas tiene
que tener concatenación con la normativa nacional, ya que nada que no
esté normado podría ser habilitada como normas, y cuando hablamos de
simplificación administrativa se refiere a que todos los gobiernos
regionales tienen la capacidad de generar normativas a fin de agilizar
los trámites, simplificación de procedimientos administrativos significa
que debemos de agilizar y reducir los pasos que se siguen para
determinados resultados.

Artículo 37.- Normas y disposiciones regionales

Los Gobiernos Regionales, a través de sus órganos de gobierno, dictan
las normas y disposiciones siguientes:

\begin{enumerate}
\def\labelenumi{\alph{enumi})}
\tightlist
\item
  El Consejo Regional: Ordenanzas Regionales y Acuerdos del Consejo
  Regional.
\item
  La Presidencia Regional: Decretos Regionales y Resoluciones
  Regionales. Los órganos internos y desconcentrados emiten Resoluciones
  conforme a sus funciones y nivel que señale el Reglamento respectivo.
\end{enumerate}

Aquí nos dice cuáles son los niveles de normas de mayor jerarquía en el
ámbito de los gobiernos regionales, Ej.: a nivel nacional la
constitución (madre de todas las normas), después viene las leyes (que
es de aplicación nacional) y quienes generan las leyes son el congreso,
excepcionalmente el congreso cuando esta apurado el gobierno central o
el presidente de la republica quiere leyes rápidas y que no hayan
discusiones excepcionalmente el congreso lo que hace es delegarle las
atribuciones del congreso para que saque las leyes que favorezcan al
gobierno central al presidente de la Republica por lo que con sus
ministros hace las leyes, firma el presidente y sale la ley, por lo que
esto se llama Decretos legislativos porque fueron delegados con rango de
ley, el presidente también puede sacar decretos supremos, resoluciones
supremas y hasta aquí llega el presidente de la república y esto es de
aplicación nacional.

Los gobiernos regionales solo pueden sacar ordenanzas Regionales y esta
la sacan los consejos regionales y por extensión podemos decir que en
las municipalidades el consejo municipal, los regidores pueden sacar una
resolución que se llama ordenanza municipal.

El consejo regional también puede delegar al presidente Regional,
entonces este puede sacar ordenanzas por decreto, los presidentes
regionales sacan decretos regionales, resoluciones regionales y también
pueden sacar las resoluciones ejecutivas regionales

Los órganos de control institucional pueden sacar resoluciones de
sanción o reglamentos o procedimientos que ayuden a la buena
administración regional, pero que tiene que estar ceñidos a los
reglamentos de nivel nacional o Regionales, debemos de buscar normas que
no tengan duplicidad.

Las ordenanzas regionales son de mayor peso que las normas que emana el
gobernador Regional, pero lo que pasa cuando se saca un decreto regional
este pasa por encima del Consejo regional en el entendido de que el
presidente regional a través del gerente general ha enviado una
propuesta de ordenanza regional al consejo regional y el consejo
regional demora en responder por desacuerdos y etc, y en tanto se
aprueba eso aprueban un decreto regional y si nuca sale esa ordenanza
regional esta otra es válida, pero podría ser dejada sin efecto por una
ordenanza regional, por eso es que los gobiernos regionales son sinónimo
de leyes que emana el congreso y es igual en los municipios.

Las leyes normalmente son trasversales, por lo que sería bueno hacer las
modificaciones necesarias o dejar sin efecto muchas leyes que hacen daño
al país.

Artículo 52.- Funciones en materia pesquera

\begin{enumerate}
\def\labelenumi{\alph{enumi})}
\tightlist
\item
  Formular, aprobar, ejecutar, evaluar, dirigir, controlar y administrar
  los planes y políticas en materia pesquera y producción acuícola de la
  región.
\item
  Administrar, supervisar y fiscalizar la gestión de actividades y
  servicios pesqueros bajo su jurisdicción.
\item
  Desarrollar acciones de vigilancia y control para garantizar el uso
  sostenible de los recursos bajo su jurisdicción.
\item
  Promover la provisión de recursos financieros privados a las empresas
  y organizaciones de la región, con énfasis en las medianas, PYMES y
  unidades productivas orientadas a la exportación.
\item
  Desarrollar e implementar sistemas de información y poner a
  disposición de la población información útil referida a la gestión del
  sector.
\item
  Promover, controlar y administrar el uso de los servicios de
  infraestructura de desembarque y procesamiento pesquero de su
  competencia, en armonía con las políticas y normas del sector, a
  excepción del control y vigilancia de las normas sanitarias
  sectoriales, en todas las etapas de las actividades pesqueras.
\item
  Verificar el cumplimiento y correcta aplicación de los dispositivos
  legales sobre control y fiscalización de insumos químicos con fines
  pesqueros y acuícolas, de acuerdo a la Ley de la materia. Dictar las
  medidas correctivas y sancionar de acuerdo con los dispositivos
  vigentes.
\item
  Promover la investigación e información acerca de los servicios
  tecnológicos para la preservación y protección del medio ambiente.
\end{enumerate}

Cuando hablamos de pesquería señalamos que no tenemos pesca marítima en
la Región de Ayacucho, sino acuicultura y lo que están haciendo los
biólogos pesqueros es ir a los ríos, hacen canales y ahí están
produciendo las truchas, otros en las lagunas y en esto de las funciones
en materia pesquera en realidad el gobierno central ha vuelto a
centralizar las funciones en el ministerio de la producción, pero ellos
están interesados en la gran pesca y no en la pequeña pesca artesanal,
el gobierno central es el encargado del manejo del mar en todo el ámbito
nacional y esto sí ha sido reabsorbido, sin embargo los gobiernos
regionales que no tienen límites marítimos lo ha dejado a la voluntad
del pesquero, se ha dejado a voluntad, pero estos deberían de cuidar sus
funciones y la limpieza de los ríos, que las minas no contaminen mucho
el agua y que esta sea apta para el consumo y para la crianza de truchas
principalmente, su función debe de ser más ambiental, vemos que la
Dirección Regional no tiene la importancia que debería de tener, porque
los mismas Gobiernos regionales no se han interesado, porque a los
políticos esto no les genera muchos votos, económicamente no les sirve
para la coima en este país de corruptos debido a que existe una
corrupción masiva, por lo que a través de estas funciones tenemos que
garantizar los recursos acuícolas, los recursos que se encuentran bajo
el agua a través de esta función. Ya que lo ideal sería que nuestras
ríos y lagunas en el caso de Ayacucho estén debidamente identificadas en
todo su potencial.

\hypertarget{lineamienos-prioritarios-de-la-politica-general-del-gobienro-al-2021}{%
\subsection{LINEAMIENOS PRIORITARIOS DE LA POLITICA GENERAL DEL GOBIENRO
AL
2021}\label{lineamienos-prioritarios-de-la-politica-general-del-gobienro-al-2021}}

Estos lineamientos se hicieron pensando que al 2021 deberíamos de llegar
en óptimas condiciones como país superando muchas barreras y que como
habíamos alcanzado un crecimiento bastante alto y que el gobierno había
comenzado a responder a algunas expectativas de la población se ha
generado mucha esperanza en la población peruana pues se espera una gran
ventaja para los gobiernos y el optimismo de la población.

Cuando hablamos de crecimiento económico decimos que el PBI va crecer,
que la producción aumenta, los ingresos aumentan, el desempleo
disminuye, la tasa de crecimiento de los precios está totalmente
controlado y en muchos casos tiende a la baja, por lo que estaríamos en
una etapa de prosperidad que crece permanentemente y decimos que hay
optimismo de parte de los inversionistas y de los consumidores, esto
significa que para que haya crecimiento los inversionistas tiene que ser
optimistas en querer seguir invirtiendo, y lo consumidores están
satisfechos porque cada vez consumen más ya que tiene los ingresos
suficientes.

Por lo que se fijaron 6 objetivos nacionales para el plan bicentenario:

\begin{enumerate}
\def\labelenumi{\arabic{enumi}.}
\tightlist
\item
  El país tiene una población con los derechos fundamentales y dignidad
  de las personas.
\item
  Oportunidades y acceso a los servicios.
\item
  Estado y Gobernabilidad.
\item
  Economía, competitividad y empleo
\item
  El desarrollo regional e infraestructura
\item
  Los recursos naturales y ambiente
\end{enumerate}

En este contexto de los objetivos nacionales al bicentenario, tenemos
que entender que el sector público no financiero está constituido por el
holding de empresas del estado y el Gobierno General (palacio de
Gobierno).

El gobierno General está compuesto por: el gobierno Central, los
gobiernos Regionales y gobiernos Locales llamados también niveles de
gobierno. El gobierno Central está constituido por un conjunto de
entidades como son los ministerios, oficinas y organismos que son
dependencias o instrumentos de la autoridad central del país.

Si hablamos a niveles de cuentas Fiscales del Perú se incluyen
ministerios, instituciones públicas (aquí debemos de mencionar a los
organismos públicos descentralizados-OPEDES) y universidades.

Al 2021 hay varias consideraciones sobre las cuales se han construido
estos objetivos nacionales y se ha fijado logros, por ejemplo:

\begin{enumerate}
\def\labelenumi{\arabic{enumi}.}
\tightlist
\item
  El país debería de ser considerado un país de desarrollo intermedio en
  rápido crecimiento económico.
\item
  Un país plenamente integrado a través de los tratados de libre
  comercio, un país comprometido con organismos internacionales y
  multilaterales (OEA, Paramento andino, etc.)
\end{enumerate}

Estas son las consideraciones al 2021, son los supuestos básicos que
deberíamos de cumplir para llegar al 2021, debemos de ir creciendo a esa
misma velocidad, debemos de ir mejorando y ampliando los tratados de
libre comercio, participar en los organismos internacionales y apuntalar
los organismos cercanos como el parlamento andino.

Y los logros que esperamos alcanzar al 2021 deberían de ser:

\begin{enumerate}
\def\labelenumi{\arabic{enumi}.}
\tightlist
\item
  El Perú debe de tener una población de 33 millones sin pobreza
  extrema, sin desempleo, sin desnutrición, sin analfabetismo, sin
  mortalidad infantil (por lo que se puede observar que a la fecha no
  hemos avanzado los logros, nuestro crecimiento ha sido mínimo, no
  hemos alcanzado el crecimiento de 6\% o 7\%)
\item
  El ingreso per cápita en el 2021 debería de estar entre 8000 y 10000
  dólares (no se ha alcanzado)
\item
  Tener un PBI duplicado entre 2010 y 2021 (no hemos alcanzado el logro)
\item
  Volumen de exportaciones cuadruplicado
\item
  La tasa anual de crecimiento mínimo debe de ser del 6\%
\item
  Tasa de inversión anual cercana al 25\%
\item
  Mejoras en la tributación, se esperaba que cada años después del 2011
  se incrementara en 5 puntos en relación al PBI
\item
  Reducción de la pobreza a menos del 10\% de la población (en la
  actualidad se ha incrementado)
\end{enumerate}

\hypertarget{ejes-de-la-poluxedtica-general-de-gobierno}{%
\subsection{ejes de la política general de
Gobierno}\label{ejes-de-la-poluxedtica-general-de-gobierno}}

los ejes de la política general de Gobierno son 5: 1. la integridad y
lucha contra la corrupción 2. es el fortalecimiento institucional para
la gobernabilidad. 3. el crecimiento económico equitativo competitivo y
sostenible. 4. Desarrollo social y bienestar de la población 5.
Descentralización efectiva para el desarrollo

\hypertarget{los-lineamientos-que-se-basan-un}{%
\subsection{Los lineamientos que se basan
un}\label{los-lineamientos-que-se-basan-un}}

\begin{enumerate}
\def\labelenumi{\arabic{enumi}.}
\tightlist
\item
  integridad y lucha contra la corrupción debemos una prioridad es la de
  combatir la corrupción y las actividades ilícitas en todas sus formas
  esto que las actividades ilícitas en todas sus formas en realidad.
\end{enumerate}

permisos necesarios entidades gubernamentales esto es que el Gobierno
central en su momento de transferir su talento todos sus conocimientos
muy bien lo otro es el

\begin{enumerate}
\def\labelenumi{\arabic{enumi}.}
\setcounter{enumi}{1}
\tightlist
\item
  fortalecimiento institucional de la gobernabilidad cuando hablamos del
  fortalecimiento institucional para la gobernabilidad se trata de
  construir consensos políticos y tener acuerdos consensos en las cuales
  coincidan con los consensos sociales de las organizaciones sociales o
  la participación de sus componentes para el desarrollo en democracia.
\end{enumerate}

Fortalecer las capacidades del Estado para atender efectivamente las
necesidades ciudadanas se cubrirá parte de las necesidades de las
ciudades considerando sus condiciones de su debilidad sus condiciones de
vulnerabilidad y diversidad cultural con las que cuenta los países como
el Perú

\begin{enumerate}
\def\labelenumi{\arabic{enumi}.}
\setcounter{enumi}{2}
\tightlist
\item
  crecimiento económico equitativo competitivo y sostenible se refiere a
  tratar de recuperar la estabilidad fiscal en las finanzas públicas o
  sea tener un mínimo necesario para financiar el presupuesto público se
  debe sumar los adicionales de actividades propias del país
\end{enumerate}

potenciar la inversión pública y privada descentralizada es decir que la
inversión del parte del Gobierno también tenga criterios de
descentralización y la propiedad privada por supuesto se debe seguir a
esos principios de la propiedad privada.

la finanza pública en cuanto a su equilibrio dependerá de la estabilidad
fiscal dependerá del estándar de captación vía impuestos de los ingresos
suficientemente financiados para atender las necesidades de la población

para contribuir al crecimiento económico equilibrado competitivo y
sostenible se debe fomentar la agricultura se debe acelerar el proceso
de reconstrucción con cambios puntualizando principalmente la
prevención. Provincias o en los distritos estas deben ser incorporadas
al circuito de distribución nacional y el mercado exterior de modo que
podemos asegurar el aprovechamiento podría ser sostenible de los
recursos naturales y del patrimonio cultural esto está íntimamente
relacionado al turismo y debería ser una actividad que deberíamos
aprovecharlo porque dinamiza toda la actividad.

\begin{enumerate}
\def\labelenumi{\arabic{enumi}.}
\setcounter{enumi}{3}
\item
  Desarrollo Social y bienestar de la población ejemplo es reducir la
  anemia infantil en los niños principalmente de 6 meses a 6 años hay
  que reducir la presencia de la anemia en los niños hay que brindar
  servicios de salud de calidad. y los establecimientos de salud tengan
  la capacidad necesaria para resolver los problemas al interior de cada
  establecimiento.
\item
  Descentralización efectiva para el desarrollo hay que promover desde
  los diferentes ámbitos territoriales del Perú o del país alianzas
  estratégicas entre regiones entre niveles de Gobierno pueden que
  pueden tener proyectos birregionales.
\end{enumerate}

\hypertarget{la-descentralizaciuxf3n.}{%
\section{LA DESCENTRALIZACIÓN.}\label{la-descentralizaciuxf3n.}}

había sido concebida para lograr objetivos en el desarrollo económico y
que no solo el desarrollo económico sino también la competitividad la
modernización la simplificación de sistemas administrativos y
simplificación de procedimientos administrativos ir a la asignación de
competencias de los servicios públicos en los niveles más cercanos a la
población que es la expresión o la razón de ser de los gobiernos
regionales los gobiernos locales.

\hypertarget{los-desafuxedos-de-la-descentralizaciuxf3n}{%
\subsection{los desafíos de la
descentralización}\label{los-desafuxedos-de-la-descentralizaciuxf3n}}

dado estas condiciones que tenemos para mejorar el nivel de organización
y contribuir al desarrollo del país 1 de los grandes desafíos:

\begin{enumerate}
\def\labelenumi{\arabic{enumi}.}
\item
  Es reducir las diferencias que se presentan entre las regiones porque
  tenemos unas regiones más desarrolladas que otras unos que tienen más
  recursos naturales en explotación en la actualidad y otras que todavía
  no las tenemos no se han descubierto simplemente no tenemos no
  contamos con esos recursos.
\item
  podría significar ejemplo que muchas municipalidades o gobiernos
  regionales estemos cada vez menos favorecidos, hay regiones que por el
  tema de canon minero. Ejemplo: canon gasífero tienen ingentes recursos
  y pese a que somos regiones cercanas o municipios vecinos no tenemos
  esa disponibilidad de recursos entonces la diferencia entre la
  población de estos municipios o de estas regiones comienza a ampliarse
  en vez de reducirse.
\end{enumerate}

RETOS

\begin{enumerate}
\def\labelenumi{\arabic{enumi}.}
\setcounter{enumi}{2}
\tightlist
\item
  ahí el reto es reducir estas diferencias, pero Adicionalmente a ese es
  el reto a ese gran objetivo de reducir las brechas el reto es que las
  regiones que cuentan con estos recursos acepten que esos recursos no
  solo son de ellos sino del país lo que pasa es que ha sido
  implementado esto un primer momento precisamente pensando en que deben
  ser los primeros favorecidos, pero eso debió ser solo por un periodo
  de tiempo hubiéramos puesto 5 años 8 años y punto. después ya debe ser
  distribuciones equitativas cada región creen que son dueños de esos
  recursos solo en esa región por ellos e incluso que todo el canon se
  distribuya solamente entre las provincias distritos entre las
  municipalidades de esa región esa concepción por. Ejemplo: ya se ha
  impregnado en la mentalidad de los pobladores lamentablemente entonces
  tenemos que desatar o desvendar de la mentalidad de la población esa
  forma de pensar.
\end{enumerate}

ahora último con esta incertidumbre de las elecciones incluso han salido
algunas propuestas algunas opinó y mucha gente pensando que el sur debe
ser un país como que alguna vez por ahí alguien se le ocurría habían
propuesto están proponiendo un mapa considerando solamente a las a la
parte sur desde Ayacucho está concentrado la minería no puede ser,
estamos llegando a extremos llegamos de egoísmo de discriminación al
interior del país y lamentablemente el gobierno central ha visto
rebasado su autoridad y no hay carácter de toma de decisiones, se trata
de que hay cosas que el gobernante tiene que tomar decisiones y se
respetan esas decisiones. El reto o desafío es reducir las deficiencias
las brechas.

\begin{enumerate}
\def\labelenumi{\arabic{enumi}.}
\setcounter{enumi}{3}
\item
  deficiente calidad de la información que harán posible qué exista
  menos transparencia en la rendición de cuentas y cada vez menos
  participa la ciudadanía en las rendiciones porque estos mecanismos de
  rendición de cuentas no han traído resultados positivos porque
  simplemente ha llegado a ser una exposición del presidente del
  gobernador regional o del alcalde y después no hay ningún resultado
  porque participa pues gente allegado a los presidentes regionales
  gobernadores regionales o alcaldes y no hay ningún resultado ningún
  tipo de análisis o entrega de información.
\item
  Hay un reto de mejorar cierto la deficiente calidad de información que
  entregan los gobernadores o los alcaldes municipales los alcaldes
  provinciales o distritales, y una iniciativa. Ejemplo: en vez de hacer
  esas presentaciones a través de la Defensoría del Pueblo la población
  puede entregar preguntas y que respondan esas preguntas, pero
  previamente se debe publicar toda la información por, ejemplo número
  de proyectos presupuesto avance físico y financiero de los proyectos
  en ejecución presupuesto de cada 1, y las modalidades de contrato con
  las cuales están ejecutando.
\item
  Es la descentralización que pueda acabar amenazando la responsabilidad
  fiscal sea los gobiernos regionales en la actualidad en realidad solo
  están recibiendo presupuesto y no están ejecutando que alcanzamos
  niveles reducidos de ejecución de gastos y menos calidad del gasto,
  perdemos el objetivo de competitividad regional y local que en
  realidad ya entró en riesgo y estamos fracasando o no se cumplan estos
  objetivos de mejorar la competitividad tanto regional como local
  porque falta responsabilidad fiscal de los gobernadores y las
  municipalidades.
\item
  Otro reto es la atomización del gasto público y la falta de
  intergubernamental a nivel de las regiones hay que implementarlo con
  las intervenciones bajo la concepción de desarrollo territorial hay
  que identificar los proyectos con efectos con las externalidades más
  amplias hay que hacer proyectos conjuntos entre una o más regiones y
  que haya capacidades desprendimiento de las autoridades regionales
  municipales a nivel incluso de las municipalidades distritales y los
  centros poblados.
\item
  El reto también sería de tratar de incorporar empresas o actividades
  con economías de escala y nivelarnos con las otras.
\item
  Otro desafío sería una fragmentación del proceso político en el país y
  la toma de decisiones en materia fiscal en este momento estamos
  fragmentados en dos y lo peor estamos no hay mayoría somos mitad y
  mitad.
\item
  Otro desafío o reto que tenemos es la adopción de impuestos a nivel
  regional y local que podría quitar incentivos a las inversiones y
  alejar a las inversiones porque ha comenzado hace dos años y está
  paralizado que los gobiernos regionales y locales se comienza a
  incrementar los impuestos según actividades económicas.
\item
  La calidad de gasto y la prestación de servicios.
\end{enumerate}

Sistema: conjunto de estructuras que están interrelacionados entre sí.
sistema administrativos tipos de gestión

Estructura: es un conjunto de relaciones que ordenan la distribución y
composición de las partes.

\hypertarget{la-regionalizacion}{%
\section{LA REGIONALIZACION}\label{la-regionalizacion}}

EL RACIONALISMO: es una teoría que postula la razón humana como fuente
de conocimiento que se opone en el sentido filosófico a la experiencia.

RACIONALIZACION: se refiere a una combinación de factores productivos
disponibles que permite formular una nueva combinación considerando la
ley de rendimientos decrecientes.

También es un proceso de mejora constante. Ejemplo: En una
administración científica. Permanentemente tenemos que ver que está
mejorando.

Ejemplo: reubicar tus cosas del cuarto.

Tenemos que adecuar a los objetivos de la empresa y a sus condicionantes
a las empresas para lograr el máximo aprovechamiento de los recursos.

Ejemplo: La reorganización de la empresa ante una reducción de la
demanda.

La racionalización: es un proceso que se organiza sistemáticamente un
trabajo para obtener mayores rendimientos al menor costo.

RESUMEN

La organización sistemática del trabajo para obtener el mayor
rendimiento de productividad al mínimo costo.

\hypertarget{tipos-de-racionalizaciuxf3n}{%
\subsection{TIPOS DE
RACIONALIZACIÓN}\label{tipos-de-racionalizaciuxf3n}}

\begin{enumerate}
\def\labelenumi{\arabic{enumi}.}
\tightlist
\item
  Racionalización administrativa: se refiere a conjunto de métodos,
  teorías que se derivan de los conocimientos tecnológicos y científicos
  aplicados a la gestión de las organizaciones y que alcancen los
  óptimos niveles de eficiencia en el marco de la eficacia. La
  racionalización administrativa en la actualidad se llama modernización
  del estado Racionalización administrativa: establece orden
  simplicidad, oportunidad en la organización y operaciones que se
  realizan para el cual se utiliza técnicas y métodos para estudiar,
  diseñar, simplificar estructuras, funciones, redacción de manuales,
  organigramas, procedimientos y uso de recursos. \#\# QUE ES REGIÓN es
  la extensión del territorio suficientemente homogénea de
  diversificación relativa que permite compatibilizar las acciones de
  desarrollo considerando su vocación actual o potencial y la naturaleza
  propia y especifica de sus recursos naturales y humanos. La región es
  aquella unidad geográfica económica definida alrededor de un núcleo o
  eje urbano en la cual converge flujos comerciales humanos y de
  geometría vial siendo a su vez ese núcleo punto de partida de
  relaciones extra regionales.
\end{enumerate}

Región administrativa: es el ámbito territorial en el cual se establece
y opera una administración regional, dicho ámbito territorial está
enmarcado dentro de la división político y administrativo del país y
guarda relación con las regiones económicas que deben desarrollarse en
el territorio nacional.

Región económica: es una zona con problemas económicas y sociales
comunes inducida por condiciones naturales o de otra clase. Ejemplo: la
cuenca de un rio o una zona sin adecuado abastecimiento de agua para la
agricultura.

\hypertarget{regiuxf3n-homoguxe9nea}{%
\subsubsection{REGIÓN HOMOGÉNEA}\label{regiuxf3n-homoguxe9nea}}

Corresponde a una región continua en que cada una de las zonas o partes
presentan características lo más próximo posible a los demás es
esencialmente morfológicas y estéticas. Región de planificación: es el
espacio geográfico definido en términos de elaboración de planes tendrá
diferentes niveles según el objetivo del país.

\hypertarget{regiuxf3n--plan}{%
\subsubsection{REGIÓN- PLAN}\label{regiuxf3n--plan}}

Es un espacio que refleja el objetivo de un plan determinado y que puede
ser una estrategia de desarrollo multinacional, nacional, interregional,
regional, subregional o local. Se podría decir que es el espacio en el
cual diversas partes proceden de una misma decisión. Ejemplo: las
filiales proceden de una matriz.

\hypertarget{regionalismo}{%
\subsection{REGIONALISMO}\label{regionalismo}}

Es el sentimiento de identificación socioeconómica de los asentamientos
humanos respectos a otros ámbitos geográficos con el cual se siente
vinculado por razones históricas, sociales o económicas.

También se define como una alternativa de inversión desde el punto de
vista económica y se debe ver como un proceso abierto plural no
uniformizarte.

\hypertarget{regionalizaciuxf3n}{%
\subsection{REGIONALIZACIÓN}\label{regionalizaciuxf3n}}

Es el proceso de redistribución y reorganimiento espacial que busca la
articulación económica social y geopolítica, ecológica y administrativa
en ámbitos denominados regiones con la finalidad de alcanzar el
desarrollo auto sostenido de la región.

Las regiones se constituyen sobre la base de áreas contiguas histórica,
económica, administrativa y culturalmente y conforman unidades
geoeconómicas.

La regionalización es una inserción del país en el circuito capitalista
y ellos se encuentran en el plan de regionalización que como fundamento
tiene en la descentralización.

\hypertarget{criterios-de-identificaciuxf3n-de-regiones}{%
\subsection{CRITERIOS DE IDENTIFICACIÓN DE
REGIONES}\label{criterios-de-identificaciuxf3n-de-regiones}}

\begin{enumerate}
\def\labelenumi{\arabic{enumi}.}
\tightlist
\item
  Geometría vial: cuantas carreteras entras allí y cuantas conexiones.
  Ejemplo: fluidos comerciales se refiere a centros de acopio a la
  distribución (huamanga).
\end{enumerate}

flujos financieros: se refiere a los montos de dinero y capitales que
ingresan traslado de encajes ejemplo: vraem (dinero y capitales se
movían allí). el flujo principal está en huamanga

\begin{enumerate}
\def\labelenumi{\arabic{enumi}.}
\setcounter{enumi}{1}
\tightlist
\item
  Flujos de población: la migración donde se concentra más la población,
  mayor recepción de personas es huamanga. Por factor enológicos
\item
  centros administrativos:
\item
  zonas deprimidas: zonas de extrema pobreza y pobreza
\end{enumerate}

\hypertarget{regiuxf3n-poluxedtica}{%
\subsection{REGIÓN POLÍTICA}\label{regiuxf3n-poluxedtica}}

Es una zona geográfica designada como una unidad administrativa
gubernamental de una zona o de una dependencia territorial de una
combinación de una o más naciones con uno o más territorios
dependientes.

\hypertarget{regiuxf3n-polarizada}{%
\subsection{REGIÓN POLARIZADA}\label{regiuxf3n-polarizada}}

Es el espacio que se encuentra bajo la influencia de un polo y que se
manifiesta esta influencia por el intercambio de flujos y por la
atracción ejercida entre el polo y el área.

No es un espacio homogéneo sino es heterogéneo cuando intervienen los
distintos aspectos dentro de él son complementarios intercambian entre
si y son dominantes con las regiones vecinas. Este concepto tiene base
en la funcionalidad y el dinamismo de una zona la interrelación y flujos
entre el polo y los centros circundantes o el área de influencia.

el espacio total de la región gira sobre la influencia que tiene una
determinada zona de que lo podemos llamar o un polo de desarrollo
alrededor de ello es que vamos a tener que este determinar considerando
la zona de influencia de ese polo de desarrollo la extensión de la
región.

\hypertarget{reorganizaciuxf3n}{%
\subsection{reorganización}\label{reorganizaciuxf3n}}

Se entiende por la reorganización de una entidad privada o pública
cuando las acciones son de transformación sustancial qué significa
modificaciones en su finalidad y objetivos. Ejemplo: el Perú en los
últimos años solo se ha dado el proceso de reorganización a nivel de
Gobierno entre el período de 1993 y el año 97 pues hemos tenido 4 años
de reorganización como parte de un proceso más complejo que ha incluido
la reestructuración.

cuándo le reorganizamos lo que hacemos es primero en la práctica y
teoría que lo puedes encontrar definiciones grandes de 5 o 10 páginas
cambiar fines y objetivos de una institución esto pasa porque hay que
transformarlo hay que cambiar prácticamente la razón de ser de una
institución esto es que en la práctica lo que se hace un nuevo
reglamento de organización y funciones se hace una nueva estructura
organizacional ahí en la nueva estructura organizacional al hacer el
reglamento vas a cambiar los fines y objetivos institucionales.

Entonces en realidad es readecuar las instituciones públicas a las
exigencias actuales a las exigencias de la población dado que hay una
transformación permanente y la estructura social la gente estamos
cambiando. Ejemplo: mi forma de pensar es totalmente distinto a los que
han nacido después de 10 años de mí los que nacieron en los 80 es otra
forma lo que ustedes lógicamente que nacieron pues al acabar el 90 antes
del siglo 20 ustedes nacieron el 97, 98 también no tiene una forma
totalmente distinta de pensar.

\begin{itemize}
\tightlist
\item
  hay que ir adecuando a esas formas de actuar de la población a las
  necesidades que exige la población y los de generaciones anteriores
  tenemos que ir adecuándonos.
\item
  las instituciones igual tenemos que adecuar a las condiciones nuevas
  que exige el desarrollo del país.
\item
  La reorganización implica tomar en consideración como base un nuevo
  ordenamiento principalmente del potencial humano de la gente.
\end{itemize}

\hypertarget{economuxeda-regional}{%
\subsection{Economía regional}\label{economuxeda-regional}}

es el estudio del comportamiento económico del hombre en el espacio
(tierra) analiza por tanto los procesos económicos a nivel espacial y
trata de conocer y de plantearse algunas preguntas entorno a la
estructura del que hacer económico.

Se ocupa la economía regional:

\begin{itemize}
\tightlist
\item
  de descubrir las causas que determinan la distribución de las
  actividades económicas en el espacio.
\item
  de la delimitación de subsistemas económicos para analizar en cada 1
  su dinámica interna.
\item
  Del Estudio de las interrelaciones entre dos o más regiones. Ejemplo:
  intercambio de bienes y servicios, la transmisión de los siglos
  económicos.
\item
  De la construcción de sistemas de equilibrio óptimo interregional.
\item
  Estudio de la política regional acciones realizadas para conseguir una
  correcta distribución de recursos.
\end{itemize}

la economía regional va estudiar el comportamiento económico del hombre
en un determinado espacio geográfico y se encargar de analizar e
investigar de generar acciones necesarias y todos los procesos
económicos en su ámbito espacial en su ámbito geográfico.

diferentes variables de la economía regional en realidad cada región
deberíamos saber cuál es el nivel de influencia. Ejemplo: que tiene el
gasto público en el PBI en cuanto porciento incide presupuesto en
infraestructura.

\hypertarget{centralismo}{%
\subsection{CENTRALISMO}\label{centralismo}}

centralismos es una tendencia a mantener las dependencias como ubicación
en los centros urbanos de carácter metropolitano las instituciones
normalmente que son centralizadas van a estar concentradas en los
centros urbanos más desarrollados va estar el poder centralizado es el
que va a determinar, Institucionalmente está representado por una sola
institución y esa es la sede central.

centralismo mantiene las dependencias con ubicación en los centros
urbanos de carácter metropolitano por decir Lima o Ayacucho o las
capitales de las regiones.

la centralización viene a ser la concentración de autoridad en un nivel
jerárquico particular lo cual reúne en una sola persona el cargo o
ámbito de poder tomar decisiones máximo alguien con el nivel más alto de
autoridad concentra todo el poder para tomar decisiones más importantes
están a cargo de esas personas que concentran poder.

Ejemplo: el presidente de la República el gobernador regional en la zona
del ámbito de su influencia.

A nivel de gobierno esta descentralizado los gobiernos regionales y
municipalidades le ha delegado: ejemplo el INC es una organización
centralizada desconcentrada porque hay casi en todo departamento.

\hypertarget{desarrollo.}{%
\subsection{DESARROLLO.}\label{desarrollo.}}

el desarrollo es un proceso que lleva a una modificación sustantiva de
las estructuras económicas políticas e ideológicas de una sociedad,
incluye la perspectiva internacional si el país crece habrá cambios
sustanciales en las organizaciones económicas a nivel de las empresas
públicas y privadas.

El desarrollo se concebí como un cambio radical del ordenamiento
tradicional en lo económico, social, político y cultural de la sociedad.

\hypertarget{desarrollo-regional}{%
\subsubsection{Desarrollo regional}\label{desarrollo-regional}}

Es el incremento del bienestar e un territorio en particular diferente
al integrado por las jurisdicciones político administrativo de un país,
cuando el desarrollo nacional le enfoca como el desarrollo de conjunto
de regiones que conforman el país

\hypertarget{objetivo-principal}{%
\subsubsection{Objetivo principal}\label{objetivo-principal}}

\begin{itemize}
\tightlist
\item
  Distribución del desequilibrio entre las poblaciones de las diversas
  regiones.
\item
  La equidad entre sus pobladores, tener metas y objetivos claros.
\end{itemize}

\hypertarget{la-descentralizaciuxf3n}{%
\subsection{la descentralización}\label{la-descentralizaciuxf3n}}

la descentralización es un principio organizativo según el cual a partir
de una entidad central se generan entidades con personería jurídica
sujetos a la política general de una entidad central que por razones
naturaleza diferencial de las funciones y actividades que deben cumplir
se les otorga autonomía operativa suficiente para asegurar el mejor
cumplimiento ellas.

la descentralización en concepto en principio son delegación de
funciones y atribuciones qué hace el Gobierno central a las regiones a
las instituciones que creen por conveniente porque de ese modo se puede
llegar a la población más alejada que no tiene beneficios del Gobierno y
estas jurídicamente van a tener autonomía o personería propia no van a
depender de otras entidades.

\hypertarget{descentralizaciuxf3n-administrativa}{%
\subsubsection{descentralización
administrativa}\label{descentralizaciuxf3n-administrativa}}

la descentralización administrativa es la delegación de autoridad y
deberes a oficinas regionales o defunciones especificas a fin de que
ellos tengan suficiente dependencia para la toma de decisiones sin que
en cada caso deban consultar con la oficina o el poder central. ejemplo
la dirección regional agraria toma sus decisiones directamente,
presidente regional no tiene que pedirle permiso en absoluto al ministro
de Agricultura para hacer acciones propias a nivel de la región.

\hypertarget{la-desconcentraciuxf3n}{%
\subsection{la desconcentración}\label{la-desconcentraciuxf3n}}

la desconcentración es un principio también organizativo según la cual
se genera una delegación de funciones atribuciones y decisiones desde un
nivel de autoridad superior hacia niveles de menor jerarquía funcional o
territorial sin embargo la misma persona jurídica o la autoridad que
delega sigue siendo responsable.

la desconcentración a nivel administración de gestión pública o privada
de principio es un principio organizativo lo cual genera de una
delegación de funciones atribuciones y decisiones desde un nivel de
autoridad superior a otros de menor jerarquía funcional o territorial
pero dentro de la misma autoridad o dentro del ámbito de la misma
persona la autoridad que delega sigue siendo responsable la autoridad
que delega sigue siendo responsable en consecuencia puedes revocar la
delegación o revisar las decisiones.

\hypertarget{desconcentraciuxf3n-administrativa}{%
\subsubsection{Desconcentración
administrativa}\label{desconcentraciuxf3n-administrativa}}

una desconcentración administrativa para quedar mejor la
desconcentración es un proceso técnico proceso técnico administrativo
mediante el cual las autoridades proceden las a delegar funciones desde
un nivel de autoridad funcionales o territoriales de mejor jerarquía.

\hypertarget{acciuxf3n-administrativa}{%
\subsubsection{Acción administrativa}\label{acciuxf3n-administrativa}}

la acción administrativa es la labor desarrollada en 1 o varios puestos
de trabajo que puede ser de naturaleza física o intelectual también se
le conoce como la decisión que adopta una autoridad al momento de
resolver asuntos de índole político, técnico o administrativo.

Es el responsable de las acciones administrativas no solo se refiere al
administrador eso al responsable es la máxima autoridad de mayor
jerarquía responsable de las acciones administrativas que van efectuando
acciones administrativas.

el Gobierno actúa en representación del Estado.

\hypertarget{el-estado}{%
\subsection{el estado}\label{el-estado}}

el estado es una construcción social es la institucionalización jurídica
y política de la sociedad tiene como elementos constitutivos el poder
político, el territorio y el pueblo (población).

el estado está constituido por el poder político que son los 3 niveles:

poder político sea los elegidos políticos representados por los
políticos el estado está constituido por los políticos Gobierno central
Poder Ejecutivo el poder legislativo emanan la elección de la población.

\hypertarget{elemento-del-estado}{%
\subsubsection{Elemento del estado}\label{elemento-del-estado}}

el territorio el área geográfica el ámbito geográfico de ese país el
Perú todavía no es una nación que somos un conjunto de etnias enteras o
poblaciones de diferentes orígenes como los pueblos amazónicos andinos
afro peruanos afro asiáticos.

El estado tiene por objeto dirigir controlar y administrar las
instituciones y regular una sociedad política y ejercer autoridad que la
población puede manifestarse, pero la autoridad está para poder poner el
orden institucional controlar administrar lo que hace la población.

resumen

El estado es una construcción social y es la institucionalidad política
jurídica de la sociedad y sus elementos consecutivos son los poderes de
poder político, el territorio y el estado.

En el caso peruano según la constitución es uno, y indivisible, su
gobierno es unitario (por el territorio es solo), representativo
(elegido por las elecciones) y descentralizado (por naturaleza está
constituido por regiones).

En las áreas de suscripciones de la índole nacional se organice en
gobierno a nivel central, regional y local.

Según la constitución y la ley se preserva la unidad e integridad del
territorio de la nación (varias naciones, samuces, ashánincas) todo ello
está integrado al estado.

El objeto del estado es dirigir, controlar y administrar instituciones
del estado, así como regular una sociedad política y ejercer autoridad
(no puede perder autoridad el estado porque si no comienza una crisis
social generalizada) por eso el estado debe poner los juegos en reglas,
los reglamentos, las leyes para su control.

Para que un gobierno pueda subsistir, existir deben de desarrollarse
algunos poderes: el Poder ejecutivo, por ejemplo, (debe coordinar con
los otros poderes para aprobar las leyes como es el poder legislativo y
además para que estas leyes se ejecuten tiene que ingresar el poder
judicial).

\hypertarget{organizaciuxf3n-del-pauxeds}{%
\section{ORGANIZACIÓN DEL PAÍS}\label{organizaciuxf3n-del-pauxeds}}

Cuando hablamos de organización del país tenemos que ver como se
visualiza el país tenemos que construir una visión país, tenemos que ver
el largo plazo.

Hay que observar que un país no se puede gobernar pensando solo en cinco
años, sino pensar en 20, 40, 100 años de modo que haya continuidad en su
implementación porque la visión país nos hace ver ``qué es lo que
buscamos, que es lo que esperamos'' para nuestra población utilizando
diversos recursos con las que contamos y tenemos que adecuarnos a las
necesidades de mundo como país y hacer lo necesario de insertarnos en
las mejores condiciones.

De modo que cuando construyamos la visión país (que el Perú tiene una
visión, el ceplan pero nosotros sabemos y podemos hacer una visión país,
regional, sectorial, actividad, proyecto).

Cuando queremos construir una visión debemos considerar algunas
características del proceso de construcción de estabilización, cuando
vas a construir una visión tienes que darle un enfoque
multidisciplinario tiene que ser holístico, partir de la interacción
temática entre distintas disciplinas involucradas en la elaboración del
documento.

La característica del proceso de construcción de una visión tiene que
ser un enfoque multidisciplinario tienes que explorar diversos
escenarios tienes que hacer que participen en lo posible todos los
actores sociales o sus representantes y al final tienes que pulirlo y
darle los lineamientos y el proceso metodológico de modo que una
construcción bien hecha.

Ya has hecho la visión Perú o de la institución tenemos que pensar en un
país en el aspecto publico vas hacer gestión pública.

Cuando hablamos de gestión nos referimos a la capacidad de articular
procesos, agentes y recursos con las que cuentas para alcanzar o
perseguir los objetivos institucionales. La gestión tiene la capacidad
de generar una relación adecuada entre lo que es la estructura la
estrategia los sistemas las capacidades los objetivos supremos de la
organización.

Podemos decir que gestión es la capacidad de juntar de articular y hacer
que coordinadamente funcione los diversos procesos, agentes, y recursos
con los que cuentas para alcanzar los objetivos.

En la administración la gestión consiste la administración de una
institución entonces normalmente el administrador de las instituciones
son los que están involucrados en la gestión.

La gestión es un procedimiento que comprende una tramitación importante
para lograr algo o solucionar un problema esto es de tipo administrativo
o requiere de documento. También podemos decir gestión es la agrupación
de la actividades u operaciones vinculadas con la administración ósea
como se utilizan los diversos recursos. Gestión se utiliza conocer
proyector en forma general cualquier acción necesaria de procedimientos
de proyección ejemplo: para ver cómo funciona una institución o
proyectar que va hacer una institución implantar una nueva institución o
generar una dirección también es gestión.

El propósito principal de la gestión es lograr el incremento de buenos
resultados de una empresa o del sector publico ósea busca cada vez
mejorar, obtener buenos resultados.

Tenemos 4 factores que participan en la búsqueda de buenos resultados de
una gestión:

\begin{enumerate}
\def\labelenumi{\arabic{enumi}.}
\tightlist
\item
  Estrategia
\item
  Estructura de la institución
\item
  Cultura de la organización, social, ancestral
\item
  La ejecución de la mesa
\end{enumerate}

La gestión pública es la aplicación de todos los procesos e instrumentos
que posee la administración pública para el logro de objetivos de
desarrollo o bienestar de la población.

La administración pública con el ejercicio de la función administrativa
del gobierno, puesta en marcha.

Los fines, objetivos y metas debe estar apoyadas a través de la gestión
políticas, recursos, programas.

resumen

Gestión es la fusión la articulación de las capacidades que tenemos para
articular los procesos, agentes, recursos para alcanzar los objetivos
institucionales.

En la gestión pública tienes que utilizar la gestión política, recursos,
programas para alcanzar el logro de los fines de objetivos y metas
institucionales a través del conjunto de procesos y acciones necesarias
que vas a poner en marcha.

\hypertarget{componentes-de-la-gestiuxf3n-puxfablica-como-estuxe1-dividido}{%
\subsection{componentes de la gestión pública como está
dividido}\label{componentes-de-la-gestiuxf3n-puxfablica-como-estuxe1-dividido}}

La gestión pública por un lado tiene al gobierno y por otro lado a la
administración pública.

El gobierno es el conjunto de personas, tienen la capacidad de regir el
destino de un país en este caso el presidente de la república, los
ministros los gobernadores regionales los alcaldes y los jefes de los
proyectos nacionales.

La administración pública es el conjunto de personas que utilizan todos
los materiales de maquinaria y equipos, muebles e inmuebles con los que
cuenta esa institución implementa procesos e instrumentos que se aplican
para ejercer el gobierno.

La gestión pública es el conjunto de herramientas de los instrumentos y
procesos puestos en ejecución.

La gestión son guías para orientar al interior de la administración
pública al interior son las guías para orientar la acción como se actúa
la previsión la visualización o uso de empleo de los recursos y el
esfuerzo de los trabajadores para alcanzar los fines que se desea.

Es el correcto manejo de los recursos de los que dispone una determinada
organización pueden ser los organismos públicos las ONG de modo que
siempre se enfoca en el mejor uso o si se quiere uso eficiente de todos
los recursos considerando que se debe maximizar los rendimientos.

La gestión es un conjunto de trámites que se llevan a cabo para resolver
algunos asuntos concretos o particular u concretar un proyecto en si.

También se dice que es gestión a la dirección, administración de una
empresa de una institución de una compañía de negocios.

Ejemplo: el administrador de excelencia de gestores.

\hypertarget{tipos-de-gestiuxf3n}{%
\subsection{Tipos de gestión}\label{tipos-de-gestiuxf3n}}

\begin{enumerate}
\def\labelenumi{\arabic{enumi}.}
\tightlist
\item
  Tecnológica
\item
  Social
\item
  De proyectos
\item
  De conocimientos
\item
  De ambiente
\item
  Estrategia
\item
  Administrativa
\item
  Gerencial
\item
  Financiera, gerencia publica y gestión de riesgo
\end{enumerate}

En el sector privado tenemos gestión empresarial, gestión educativa y
gestión humana de calidad comercial y cultural.

\hypertarget{gobierno}{%
\subsection{GOBIERNO}\label{gobierno}}

El gobierno viene a ser la autoridad que controla y administra las
instituciones del estado, ejerce el poder del estado con un ordenamiento
jurídico y está al servicio del estado es fraccional no es duradero.

El gobierno actúa en representación del estado, es la autoridad que
entra en acción controla y administra lo que representa al estado las
instituciones y las instituciones es representación mental.

El estado está institucionalizado políticamente y jurídicamente el
estado está constituido en el área geográfica está representado por el
poder político y por toda la población. cuando hablamos de Gobierno el
Gobierno actúa en su representación porque está a un giro de autoridad
para representar la construcción social del cual hablábamos del estado.

el Gobierno ejerce el poder de determinar el poder de la toma de
decisiones basado en un ordenamiento jurídico y el ordenamiento jurídico
viene desde el estado, el Gobierno siempre por definición debe estar al
servicio del Estado.

El Gobierno está al servicio de la construcción social que representa a
la población el Gobierno es transitorio es fraccional.

El Gobierno es fraccional mientras sea democrático mucho más fraccional
lo que hay que hacer es garantizar la continuidad de las políticas de
Estado por eso no se habla de políticas e gobierno.

Si es política de Gobierno a lo más 5 años es política de Estado será en
el tiempo por eso se habla políticas públicas las políticas públicas no
son para 2 años 3 años si no es para el largo plazo hasta alcanzar los
logros esperados.

\hypertarget{cuuxe1l-es-el-rol-del-estado-el-estado}{%
\subsubsection{CUÁL ES EL ROL DEL ESTADO EL
ESTADO}\label{cuuxe1l-es-el-rol-del-estado-el-estado}}

esa construcción social esa representación institucionalizada
jurídicamente por ejemplo el rol del estado es garantizar la seguridad
interna (encabeza la policía) y externa (fuerzas aéreo, la marina y el
ejército nacional).

garantizar la seguridad interna y externa y asegurar que se imparta
justicia bueno eso será pues la decisión política y el Poder Judicial
son los que ponen en marcha y acción las normas que deben cumplirse
correctamente tanto en el sector público como en el sector privada. el
rol del Estado es garantizar o velar por la propiedad es un rol
importantísimo que se debe respetar la propiedad privada.

La propiedad privada es inviolable, pero velar por la propiedad y puede
ser propiedad social y propiedad privada.

La prosperidad del sector privado y buscaremos la prosperidad de la
propiedad colectiva de las famosas cooperativas las asociaciones que el
mismo estado de implementarlo y conducir. estamos en una economía como
el Perú es regular los mercados por eso la necesidad de los entes
reguladores o a través del mercado mismo con participación del Gobierno
se puede regular.

Ejemplo: debería haber grifos reguladores de Petroperú a nivel nacional,
pero Petroperú como que sea retirado y más bien lo han vendido los
grifos de propiedad de Petroperú a las empresas privadas debieron haber
reservado entonces hubiéramos tenido.

hay que regular los mercados, debe haber participación, pero
lamentablemente como las leyes se han hecho a favor de los grandes
empresarios ante la Constitución entonces los entes reguladores no
funcionan, pero el rol del Estado es regular los mercados se debe
promover la igualdad de oportunidades.

¿es posible garantizar la igualdad de oportunidades sí o no o que sea
derecho de oportunidad de igualdad? no podemos garantizar para la
igualdad de oportunidades cierto porque tenemos diversas cualidades
diversos niveles de conocimiento diversas formas de pensar diversas
expectativas.

Ejemplo: no todos tienen pensado estudiar economía y el estado
garantizaría la igualdad de oportunidades no garantizaría, puede
promover y crear las condiciones de fortalecer a la gente.

¿sería un derecho a la igualdad de oportunidades debería ser un derecho
en todo caso, no podemos decir que sea un derecho para que en alguna
medida todos tengamos la posibilidad de promover la promoción de la
igualdad de oportunidades.

\hypertarget{el-estado-quienes-lo-constituyen}{%
\subsubsection{el estado quienes lo
constituyen}\label{el-estado-quienes-lo-constituyen}}

\begin{itemize}
\tightlist
\item
  el pueblo o la población.
\item
  el poder político le pone su área geográfica o sea el país un
  territorio. y quienes viven pues los hombres.
\item
  todo está en función a los hombres por eso que a veces algunos dicen,
  pero eso hacemos los hombres se supone que el mundo prácticamente está
  a la disposición del hombre somos los racionales los que disponemos
  los que hacemos que el mundo esté a nuestro servicio y todo lo que hay
  en este mundo en la tierra está a nuestro servicio.
\item
  poder político tenemos el territorio una fracción de la tierra y la
  población que vive en esa atracción, cuando hablamos de proteger el
  medio ambiente todos esté de acuerdo en que debemos cuidar las cosas
  más de lo necesario a la tierra.
\item
  pero eso no significa que solo seamos los dañinos por ejemplo los
  animales la vaca es uno de los animales más contaminadores del medio
  ambiente no hay nadie se ha ido contra la vaca hay que seguir criando
  la vaquita. para la carne etc.
\end{itemize}

El rol del Estado es promover la infraestructura física eso significa
que debemos hacer proyectos infraestructura pues debemos mejorar o vemos
la mejora permanente de las condiciones de vida del hombre o facilitar
los procesos de producción en las diversas actividades en
infraestructura vial, carreteras estamos haciendo más fácil más rápido
el traslado de los recursos que genera el área rural digamos hacia el
área urbana.

Otras funciones o que sean necesarios en general hacia el logro del
bienestar de repente pueden priorizar el estado a la educación, la salud
y también promoveremos o podemos garantizar la salud.

\begin{itemize}
\tightlist
\item
  ¿La salud es un producto o es un bien físico? es un servicio
\item
  la salud es un derecho que debe ser promotor del estado debe crear las
  condiciones necesarias para brindar servicios de salud, educación
  igual no debería incorporarse cómo del estado debemos incorporado que
  el rol del Estado es generar servicio de educación y de calidad, la
  función del Gobierno estaría en brindar servicios de calidad en la
  educación pública.
\item
  El estado debería ingresar para mejorar la calidad de educación y
  salud en todos los otros sectores en ese entender tenemos que estar
  preocupados permanentemente en la modernización del Estado
  modernización del Estado es un proceso:
\end{itemize}

\begin{enumerate}
\def\labelenumi{\arabic{enumi}.}
\tightlist
\item
  político
\item
  técnico
\end{enumerate}

el poder político tiene que decidir que se debe proceder con un proceso
de modernización porque sí no hay decisión política no hay forma de
hacer.

\begin{itemize}
\tightlist
\item
  el proceso de modernización es agilizar y es proceso político decisión
  política una vez que el político ha decidido qué cosas vamos a
  descentralizar el MEEF tiene que ser un proceso técnico.
\end{itemize}

\hypertarget{en-quuxe9-se-debe-centrar-el-estado-hacia-la-modernizaciuxf3n}{%
\subsubsection{en qué se debe centrar el estado hacia la
modernización}\label{en-quuxe9-se-debe-centrar-el-estado-hacia-la-modernizaciuxf3n}}

\begin{enumerate}
\def\labelenumi{\arabic{enumi}.}
\tightlist
\item
  debe centrarse en la transformación de actitudes que la gente vea con
  diferencia el cargo público que vea con diferencia las funciones el
  nivel la representación al Gobierno y el hombre debe estar con actitud
  positiva que debe de estar identificado con el manejo gubernamental.
\item
  fortalecer las aptitudes hay que hacer que la gente esté debidamente
  preparada para alcanzar los objetivos institucionales tiene que estar
  bien capacitado y si no está capacitado hay que capacitarlo.
\item
  Hay que transformar para que el estado de modernice, agilización de
  procesos.
\item
  Tenemos que identificar plenamente los sistemas funcionales y
  administrativos debemos identificar y hacer las relaciones y
  estructuras administrativas correctamente, como cada 1 de las unidades
  estructuradas que deben entrelazar deben complementarse deben sumar al
  mismo objetivo de modo que no tenga contradicciones debe haber
  relaciones y estructuras administrativas todo esto con el fin de hacer
  compatibles con todo los niveles de Gobierno además con los planes
  nacionales e institucionales, Plan Nacional y el planeamiento
  estratégico nacional.
\end{enumerate}

¿Porque es necesario impulsar un proceso de modernización?

en la actualidad tenemos el sector público muy pesado muy amplios
debemos impulsar hace tiempo se debió hacer un nuevo proceso de
modernización del Estado la modernización.

por qué es necesario hacer un proceso de modernización?

primero hay que emprender un proceso de reforma integral de la gestión a
nivel gerencial a nivel operacional hay que hacer un proceso de reforma
integral a nivel de gestión gerencial nivel de los de las gerencias y
también la parte operativa de modo que se pueda afrontar las debilidades
del aparato estatal, Porque hay muchos aspectos en la actualidad en la
que el aparato estatal no responde a las expectativas de la población.

Pasar de una administración pública que obtenga resultados para el
ciudadano siempre delante de nosotros tiene que estar la población el
servicio es el objetivo principal tienes que saber desde ahora que
cuando trabajes para el estado.

la gestión pública es la realización de acciones orientadas a mejorar a
incrementar los niveles de eficiencia y eficacia de la gestión pública a
fin de lograr resultados en beneficios de la población en beneficio de
la ciudadanía.

entonces la gestión pública debe ser con enfoque de resultado.

La planificación en realidad estás hablando de un proceso de
racionalizar a mediano a largo plazo por eso es importante porque dice
dónde cuándo un medio que sirve a la acción de desarrollo es
indispensable para entrar en acción necesaria para el desarrollo.

Este proceso de racionalizador en medio para la acción que es la
planificación para el desarrollo es a través de la escogencia la
realización de mejores métodos la planificación es un proceso
racionalizador es un medio necesario indispensable a la acción del
desarrollo a través de la priorización la realización de los mejores
métodos para satisfacer determinadas políticas y lograr sus objetivos

si hablamos a nivel de sociedad tendríamos que decir que la
planificación es un proceso social, así como hemos dicho un proceso
racionalizador podríamos decir que es un proceso social que permite
ordenar las actividades destinadas a satisfacer las necesidades físicas,
sociales necesidades físicas sociales políticas y culturales de una
sociedad.

procesos de planificación social siempre se debe buscar una
planificación participativa esto sí es importantísimo que sea
participativo debemos hacer que la gente se incorpore que los
representantes más representativos de la población que contribuyan que
lo hagan suyo el plan que va a resultar porque si la población no lo
siente como suyo lo que puede hacer que fracase ese proceso de
planificación.

Desde el Gobierno podemos decir es un instrumento Del Gobierno la
planificación es un instrumento de Gobierno dirigido a superar
progresivamente a superar progresivamente el carácter espontáneo del
proceso económico, político y social para sustituirlo por un desarrollo
orgánico y armónico y alcanzar los objetivos prefijados

la planificación te permite tomar decisiones adecuadas te permite hacer
que el desarrollo económico, social, político y cultural sea armónico y
secuencial.

\hypertarget{realidad-nacional}{%
\section{REALIDAD NACIONAL}\label{realidad-nacional}}

cuando hablamos de realidad nacional nos estamos refiriendo al conjunto
de recursos humanos naturales y financieros elementos institucionales y
relaciones creadas por los diferentes grupos sociales en el país en sus
actividades económicas, sociales, políticas y culturales a lo largo de
la historia y los vigentes dentro del territorio nacional, así como la
relaciones que se generan entre estos y el exterior.

Los elementos institucionales están relacionados por los grupos sociales
distintos a nivel del país y que cada 1 de ellos tienen diversas
actividades económicas, políticas, sociales y culturales, pero todo ese
cruce de relaciones no es casual entonces siempre hay que tener en
cuenta su pasado la historia no hay que agarrar a la población solo
mirando el futuro siempre hay que considerar el pasado la historia. Cómo
piensa cuáles son sus expectativas es que cree de las otras zonas del
país no se está refiriendo a lo que está fuera del país también está
incluido, pero fuera de su ámbito.

nuestra realidad nacional no debemos confundir son toda la estructura
como digamos la realidad nacional está constituido por toda la
estructura económico social político cultural de los pueblos que están
interrelacionados entre sí los elementos fundamentales en nuestra región
en la que el actor principal es la población y la naturaleza lo
acondiciona.

Nuestra realidad nacional no solo son los recursos también hay que
considerar las actividades culturales sociales económicas políticas está
incluido sobre el hombre sobre la naturaleza y las finanzas es nuestra
realidad nacional no se trata yo voy a describir la realidad nacional es
probablemente imposible describir la realidad nacional pero algunas
partes de la realidad nacional.

desenvolvimiento histórico se debe considerar como un pasado que ha dado
resultado el producto actual es el desenvolvimiento histórico.


\printbibliography


\end{document}
