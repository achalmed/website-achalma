% Options for packages loaded elsewhere
\PassOptionsToPackage{unicode}{hyperref}
\PassOptionsToPackage{hyphens}{url}
\PassOptionsToPackage{dvipsnames,svgnames,x11names}{xcolor}
%
\documentclass[
  letterpaper,
  DIV=11,
  numbers=noendperiod]{scrartcl}

\usepackage{amsmath,amssymb}
\usepackage{iftex}
\ifPDFTeX
  \usepackage[T1]{fontenc}
  \usepackage[utf8]{inputenc}
  \usepackage{textcomp} % provide euro and other symbols
\else % if luatex or xetex
  \usepackage{unicode-math}
  \defaultfontfeatures{Scale=MatchLowercase}
  \defaultfontfeatures[\rmfamily]{Ligatures=TeX,Scale=1}
\fi
\usepackage{lmodern}
\ifPDFTeX\else  
    % xetex/luatex font selection
\fi
% Use upquote if available, for straight quotes in verbatim environments
\IfFileExists{upquote.sty}{\usepackage{upquote}}{}
\IfFileExists{microtype.sty}{% use microtype if available
  \usepackage[]{microtype}
  \UseMicrotypeSet[protrusion]{basicmath} % disable protrusion for tt fonts
}{}
\makeatletter
\@ifundefined{KOMAClassName}{% if non-KOMA class
  \IfFileExists{parskip.sty}{%
    \usepackage{parskip}
  }{% else
    \setlength{\parindent}{0pt}
    \setlength{\parskip}{6pt plus 2pt minus 1pt}}
}{% if KOMA class
  \KOMAoptions{parskip=half}}
\makeatother
\usepackage{xcolor}
\setlength{\emergencystretch}{3em} % prevent overfull lines
\setcounter{secnumdepth}{-\maxdimen} % remove section numbering
% Make \paragraph and \subparagraph free-standing
\ifx\paragraph\undefined\else
  \let\oldparagraph\paragraph
  \renewcommand{\paragraph}[1]{\oldparagraph{#1}\mbox{}}
\fi
\ifx\subparagraph\undefined\else
  \let\oldsubparagraph\subparagraph
  \renewcommand{\subparagraph}[1]{\oldsubparagraph{#1}\mbox{}}
\fi


\providecommand{\tightlist}{%
  \setlength{\itemsep}{0pt}\setlength{\parskip}{0pt}}\usepackage{longtable,booktabs,array}
\usepackage{calc} % for calculating minipage widths
% Correct order of tables after \paragraph or \subparagraph
\usepackage{etoolbox}
\makeatletter
\patchcmd\longtable{\par}{\if@noskipsec\mbox{}\fi\par}{}{}
\makeatother
% Allow footnotes in longtable head/foot
\IfFileExists{footnotehyper.sty}{\usepackage{footnotehyper}}{\usepackage{footnote}}
\makesavenoteenv{longtable}
\usepackage{graphicx}
\makeatletter
\def\maxwidth{\ifdim\Gin@nat@width>\linewidth\linewidth\else\Gin@nat@width\fi}
\def\maxheight{\ifdim\Gin@nat@height>\textheight\textheight\else\Gin@nat@height\fi}
\makeatother
% Scale images if necessary, so that they will not overflow the page
% margins by default, and it is still possible to overwrite the defaults
% using explicit options in \includegraphics[width, height, ...]{}
\setkeys{Gin}{width=\maxwidth,height=\maxheight,keepaspectratio}
% Set default figure placement to htbp
\makeatletter
\def\fps@figure{htbp}
\makeatother

\KOMAoption{captions}{tableheading,figureheading}
\makeatletter
\makeatother
\makeatletter
\makeatother
\makeatletter
\@ifpackageloaded{caption}{}{\usepackage{caption}}
\AtBeginDocument{%
\ifdefined\contentsname
  \renewcommand*\contentsname{Tabla de contenidos}
\else
  \newcommand\contentsname{Tabla de contenidos}
\fi
\ifdefined\listfigurename
  \renewcommand*\listfigurename{Listado de Figuras}
\else
  \newcommand\listfigurename{Listado de Figuras}
\fi
\ifdefined\listtablename
  \renewcommand*\listtablename{Listado de Tablas}
\else
  \newcommand\listtablename{Listado de Tablas}
\fi
\ifdefined\figurename
  \renewcommand*\figurename{Figura}
\else
  \newcommand\figurename{Figura}
\fi
\ifdefined\tablename
  \renewcommand*\tablename{Tabla}
\else
  \newcommand\tablename{Tabla}
\fi
}
\@ifpackageloaded{float}{}{\usepackage{float}}
\floatstyle{ruled}
\@ifundefined{c@chapter}{\newfloat{codelisting}{h}{lop}}{\newfloat{codelisting}{h}{lop}[chapter]}
\floatname{codelisting}{Listado}
\newcommand*\listoflistings{\listof{codelisting}{Listado de Listados}}
\makeatother
\makeatletter
\@ifpackageloaded{caption}{}{\usepackage{caption}}
\@ifpackageloaded{subcaption}{}{\usepackage{subcaption}}
\makeatother
\makeatletter
\@ifpackageloaded{tcolorbox}{}{\usepackage[skins,breakable]{tcolorbox}}
\makeatother
\makeatletter
\@ifundefined{shadecolor}{\definecolor{shadecolor}{rgb}{.97, .97, .97}}
\makeatother
\makeatletter
\makeatother
\makeatletter
\makeatother
\ifLuaTeX
\usepackage[bidi=basic]{babel}
\else
\usepackage[bidi=default]{babel}
\fi
\babelprovide[main,import]{spanish}
% get rid of language-specific shorthands (see #6817):
\let\LanguageShortHands\languageshorthands
\def\languageshorthands#1{}
\ifLuaTeX
  \usepackage{selnolig}  % disable illegal ligatures
\fi
\usepackage[]{biblatex}
\addbibresource{../../../../references.bib}
\IfFileExists{bookmark.sty}{\usepackage{bookmark}}{\usepackage{hyperref}}
\IfFileExists{xurl.sty}{\usepackage{xurl}}{} % add URL line breaks if available
\urlstyle{same} % disable monospaced font for URLs
\hypersetup{
  pdftitle={Impulsando el Desarrollo Económico Regional Claves para un Futuro Próspero},
  pdfauthor={Edison Achalma},
  pdflang={es},
  colorlinks=true,
  linkcolor={blue},
  filecolor={Maroon},
  citecolor={Blue},
  urlcolor={Blue},
  pdfcreator={LaTeX via pandoc}}

\title{Impulsando el Desarrollo Económico Regional Claves para un Futuro
Próspero\thanks{gracias}}
\usepackage{etoolbox}
\makeatletter
\providecommand{\subtitle}[1]{% add subtitle to \maketitle
  \apptocmd{\@title}{\par {\large #1 \par}}{}{}
}
\makeatother
\subtitle{Estrategias y perspectivas para fortalecer la economía
regional en un mundo en constante cambio, leccion detallada de economia
regional.}
\author{Edison Achalma}
\date{2023-05-16}

\begin{document}
\maketitle
\ifdefined\Shaded\renewenvironment{Shaded}{\begin{tcolorbox}[breakable, sharp corners, interior hidden, enhanced, borderline west={3pt}{0pt}{shadecolor}, frame hidden, boxrule=0pt]}{\end{tcolorbox}}\fi

\hypertarget{pauxedses-en-vuxedas-de-desarrollo}{%
\section{Países en vías de
desarrollo}\label{pauxedses-en-vuxedas-de-desarrollo}}

Los países en vías de desarrollo son un grupo de naciones que se
caracterizan por su enfoque en el aprovechamiento del \textbf{potencial}
productivo de su sociedad. El concepto de potencial se refiere a la
\textbf{diferencia existente entre los recursos disponibles en el país y
los recursos que realmente se utilizan.}

Esta disparidad es especialmente notable en países en vías de
desarrollo, donde a pesar de contar con abundantes recursos, no se
explotan de manera integral. Por ejemplo, poseemos minerales cuyos
precios están en alza, pero no aprovechamos en su totalidad. Mientras
tanto, otras economías están vendiendo y aprovechando al máximo sus
recursos.

Es crucial comprender la importancia de utilizar y vender estos
recursos, ya que los ingresos generados pueden contribuir
significativamente a mejorar las condiciones del nuestro país. Si se
dejan sin explotar, se retrasa el \textbf{proceso de desarrollo}. Un
ejemplo concreto de esto es el caso del gas de Camisea, que solo empezó
a aprovecharse en el año 2004.

Existen potencialidades de los recursos humanos y naturales que no se
están aprovechando en el marco de la política de desarrollo,
especialmente en el ámbito de la educación y la formación de mano de
obra calificada.

El desaprovechamiento de estas potencialidades impide que se destine de
manera adecuada recursos al sector educativo y a la formación de
trabajadores especializados. Esto, a su vez, limita la adopción y
aplicación de tecnologías modernas.

Es esencial comprender la importancia de utilizar de manera eficiente y
efectiva los recursos humanos y naturales disponibles. En el ámbito de
la educación, esto implica invertir en programas educativos de calidad
que promuevan la formación de una fuerza laboral altamente capacitada.
Además, se debe fomentar el desarrollo y la implementación de
tecnologías modernas que impulsen la productividad y la competitividad
en los sectores económicos.

El no aprovechamiento de estas potencialidades tiene consecuencias
negativas, ya que limita el crecimiento económico, la generación de
empleo y el avance tecnológico en el país. Es fundamental establecer
políticas y estrategias que promuevan la inversión en educación, la
mejora de la capacitación laboral y la adopción de tecnologías
innovadoras.

Es importante destacar la relevancia del recurso humano en relación al
empleo, ya que existe un porcentaje de la población en edad de trabajar
que se encuentra desempleada. Por ejemplo, de cada 100 personas en edad
laboral, aproximadamente 30 no están empleadas y no contribuyen a la
producción. Esto implica que, aunque podríamos producir 100 productos
con la totalidad de la fuerza laboral disponible, en realidad solo se
producen 70 productos debido a la falta de incorporación de este recurso
en el proceso productivo.

En otras palabras, \textbf{no estamos aprovechando plenamente las
potencialidades de nuestra producción} debido a la subutilización del
recurso humano disponible. Esta situación impacta negativamente en la
capacidad productiva y en el desarrollo económico.

Además, es importante señalar que una parte significativa de la
población en edad de trabajar no alcanza los niveles de productividad
esperados debido a la falta de formación y capacitación en mano de obra
calificada. Esta improductividad se convierte en un obstáculo para
lograr los niveles de producción deseados y limita el crecimiento
económico.

Para superar esta situación, es fundamental invertir en la formación y
capacitación de la fuerza laboral. Esto implica desarrollar programas
educativos y de formación que promuevan la adquisición de habilidades y
conocimientos especializados requeridos en el mercado laboral. De esta
manera, se puede mejorar la productividad y aprovechar plenamente el
potencial del recurso humano en el proceso productivo.

Tomemos como ejemplo los países árabes, conocidos por su papel como
grandes productores de petróleo y acumuladores de riqueza. Sin embargo,
en la actualidad, el petróleo está perdiendo fuerza como fuente de
energía, lo cual ha llevado a estos países a financiar a otras empresas
para evitar la entrada al mercado de productos como vehículos y
maquinarias eléctricas, que podrían reducir la demanda de petróleo.

Esperamos el crecimiento y prosperidad de estas empresas fabricantes de
vehículos eléctricos, ya que si los automóviles y maquinarias empiezan a
funcionar con sistemas eléctricos, el precio del petróleo se verá
afectado negativamente, lo que podría empobrecer a estas naciones
productoras. Sin embargo, a pesar de esta situación, estos países siguen
explotando al máximo sus recursos petroleros.

Por otro lado, en el caso del Perú, poseemos una abundante reserva de
litio, ocupando un destacado lugar después de Bolivia, Argentina y Chile
en términos de reservas de este recurso. Existe la posibilidad de que en
el futuro nos convirtamos en un gran productor de litio. No obstante, en
un acontecimiento lamentable, las reservas de litio fueron vendidas
durante el gobierno de Vizcarra.

\textbf{El desempleo es una característica propia de los países en vías
de desarrollo}, y en el caso del Perú, somos dependientes de las
fluctuaciones en las relaciones económicas internacionales. Por ejemplo,
cuando el precio del dólar aumenta, se refleja en el incremento del
precio del pollo, mientras que si el precio del dólar disminuye, el
precio del pollo también disminuye.

\hypertarget{a-quuxe9-se-debe-esto}{%
\subsection{¿A qué se debe esto?}\label{a-quuxe9-se-debe-esto}}

Se debe a que, por ejemplo, nosotros no producimos maíz de gallina, lo
produce Colombia. Gran parte de nuestras importaciones de maíz de
gallina proviene de ese país, lo cual tiene un impacto directo en el
precio del pollo en el mercado nacional.

Actualmente, el precio del pollo no ha aumentado significativamente
debido a que todavía existen existencias de maíz en los almacenes de los
agricultores locales. Sin embargo, cuando se agoten estas reservas y se
tenga que importar más maíz, es importante considerar que el precio del
dólar se espera que suba a 3.75, lo que inevitablemente resultará en un
incremento en los costos de producción. Como consecuencia, esto se
traducirá en un aumento en el precio final del pollo.

Es evidente que \textbf{tenemos una dependencia significativa} en
términos de abastecimiento, especialmente en el caso de nuestras
pequeñas empresas. Esto refuerza aún más la necesidad de fortalecer
nuestra capacidad de producción y reducir nuestra dependencia de las
importaciones en sectores clave como la alimentación avícola.

Es fundamental implementar estrategias que fomenten la producción local
de maíz y otros insumos agrícolas necesarios para la industria avícola.
Esto no solo ayudará a reducir la dependencia externa, sino que también
contribuirá al desarrollo económico y la autonomía de nuestras empresas
locales.

\hypertarget{pauxedses-industrializados}{%
\subsection{Países industrializados}\label{pauxedses-industrializados}}

En el contexto de los países industrializados, es fundamental destacar
la importancia de la industrialización en el proceso de desarrollo
económico. La presencia de industrias, tanto a nivel nacional como en el
ámbito global, desempeña un papel fundamental en la generación de
ingresos y el progreso de un país.

Los países industrializados se caracterizan por contar con una sólida
base industrial que impulsa su economía. La industrialización se refiere
al desarrollo y crecimiento de sectores manufactureros y productivos,
donde se transforman materias primas en productos acabados mediante
procesos de producción eficientes y tecnológicamente avanzados.

La presencia de industrias diversificadas y competitivas contribuye a
generar empleo, aumentar la productividad, impulsar la innovación
tecnológica y promover la exportación de productos manufacturados.
Además, las industrias aportan al crecimiento económico de un país al
generar ingresos a través de la producción y venta de bienes y
servicios.

Por otro lado, los países que carecen de un nivel significativo de
industrialización se consideran no industrializados. Estos países pueden
depender en gran medida de sectores primarios, como la agricultura, la
minería o la extracción de recursos naturales, lo que puede limitar su
capacidad para diversificar su economía y alcanzar un desarrollo
sostenible.

En el \textbf{caso de Perú}, es importante destacar las principales
industrias presentes en el país. Entre ellas se encuentran:

\begin{enumerate}
\def\labelenumi{\arabic{enumi}.}
\item
  \textbf{Industria Textil:} Sin embargo, esta industria se ha visto
  afectada debido a la falta de producción de algodón, una materia prima
  fundamental. La apertura de fronteras y la falta de control en los
  ingresos han permitido la entrada de productos textiles chinos a
  precios muy bajos, lo que ha llevado a la destrucción de la industria
  textil nacional.
\item
  \textbf{Industria Pesquera:} Perú cuenta con una importante industria
  pesquera, aprovechando su extensa costa y la abundancia de recursos
  marinos. Esta industria desempeña un papel clave en la generación de
  empleo y en las exportaciones de productos pesqueros.
\item
  \textbf{Industria de Cementos:} La producción y comercialización de
  cemento también es una actividad destacada en el país. La construcción
  de infraestructuras y el desarrollo de proyectos inmobiliarios
  impulsan la demanda de cemento, generando empleo y contribuyendo al
  crecimiento económico.
\end{enumerate}

Perú se caracteriza por ser una economía mixta, lo que significa que
existe participación tanto del Estado como del sector privado. Sin
embargo, es importante destacar que la administración de los recursos
por parte del Estado ha sido cuestionada en algunos casos. Por otro
lado, el sector privado juega un papel importante en la inversión
económica, ya que son las empresas privadas las que generalmente
realizan inversiones significativas en el país.

\hypertarget{las-caracteruxedsticas-de-una-sociedad-en-desarrollo-son}{%
\subsection{Las características de una sociedad en desarrollo
son:}\label{las-caracteruxedsticas-de-una-sociedad-en-desarrollo-son}}

Las características de una sociedad en desarrollo abarcan diferentes
aspectos que reflejan su situación socioeconómica. Entre estas
características se encuentran:

\begin{enumerate}
\def\labelenumi{\arabic{enumi}.}
\item
  \textbf{Países pobres}: Estos países presentan niveles de desarrollo
  económico y social bajos, con altas tasas de pobreza y dificultades
  para satisfacer las necesidades básicas de su población.
\item
  \textbf{Países en vías de desarrollo}: Se trata de naciones que se
  encuentran en un proceso de transición hacia el desarrollo,
  experimentando cambios y mejoras en su economía y calidad de vida,
  pero aún enfrentando desafíos significativos.
\item
  \textbf{Países dependientes}: Estos países tienen una alta dependencia
  de otras naciones en términos económicos, políticos o culturales.
  Pueden depender en gran medida de la exportación de materias primas o
  tener una fuerte influencia de actores externos en sus decisiones y
  políticas.
\item
  \textbf{Países no industrializados}: Son naciones que carecen de una
  base industrial sólida, dependiendo en su mayoría de actividades
  económicas primarias, como la agricultura, la minería o la extracción
  de recursos naturales.
\item
  \textbf{Países emergentes}: Son aquellos países que han experimentado
  un rápido crecimiento económico y han logrado avances significativos
  en varios aspectos, como la reducción de la pobreza, el aumento de la
  inversión extranjera y la mejora de los indicadores sociales.
\end{enumerate}

Las sociedades en desarrollo presentan una serie de desafíos y
características que contribuyen a su situación actual. Estos incluyen:

\begin{itemize}
\tightlist
\item
  Altas tasas de crecimiento poblacional: Esto puede ejercer presión
  sobre los recursos y los sistemas de bienestar social de un país.
\item
  Dependencia de la agricultura: Una gran proporción de la mano de obra
  empleada en el sector agrícola limita la diversificación económica y
  puede generar problemas de empleo y desarrollo.
\item
  Niveles de ingreso y pobreza: La existencia de altos niveles de
  pobreza y escaso o nulo ahorro dificulta el crecimiento económico
  sostenible.
\item
  Paro encubierto masivo (subempleo): La falta de oportunidades de
  empleo adecuadas y la presencia de subempleo son características
  comunes en las sociedades en desarrollo.
\item
  Dependencia de productos de exportación: La economía de estos países
  puede depender en gran medida de uno o unos pocos productos de
  exportación, lo que los hace vulnerables a las fluctuaciones en los
  precios internacionales.
\item
  Desigualdades de riqueza: Existen grandes disparidades entre los
  segmentos más ricos y más pobres de la sociedad, lo que contribuye a
  la desigualdad social y económica.
\item
  Control gubernamental: En algunos casos, el control del gobierno está
  en manos de una minoría rica que se opone a cambios que puedan
  perjudicar sus intereses, lo que puede dificultar el progreso y la
  implementación de políticas de desarrollo equitativas.
\end{itemize}

Estas características y desafíos reflejan la complejidad de las
sociedades en desarrollo y su posición relativa en comparación con otros
países. Estos países suelen tener potencialidades productivas
desaprovechadas y dependencia en diversos aspectos, como la economía, la
cultura, la política y la tecnología. Superar estos desafíos y lograr un
desarrollo sostenible requiere de esfuerzos integrales y políticas
adecuadas que promuevan la equidad, la diversificación económica y la
inclusión social.

Al caracterizar una sociedad, es importante reconocer que existe una
posición ideológica que influye en la interpretación de los factores que
nos permiten afirmar dichas características. Estas caracterizaciones
tienen connotaciones, sentido y una naturaleza ideológica inherente.

En los países con problemas de desarrollo, se manifiestan desigualdades
enormes entre ricos y pobres. Estas desigualdades constituyen un
conjunto complejo de problemas que se traducen y se manifiestan en una
brecha significativa de riqueza y pobreza.

Asimismo, estas economías suelen experimentar estancamientos, es decir,
avanzan en ciertos aspectos y luego se quedan estancadas sin lograr un
progreso sostenido.

En comparación con otros países, nos encontramos rezagados y atrasados
en términos de potencialidades productivas desaprovechadas. Además,
dependemos de estas economías en diversos aspectos, como lo económico,
cultural, político y tecnológico.

\hypertarget{caracteruxedsticas-de-los-pauxedses-con-problemas-de-desarollo}{%
\subsection{Características de los países con problemas de
desarollo}\label{caracteruxedsticas-de-los-pauxedses-con-problemas-de-desarollo}}

En los países con problemas de desarrollo, como el nuestro, existen
características que influyen en nuestra situación económica y social.
Algunas de estas características son las siguientes:

\begin{enumerate}
\def\labelenumi{\arabic{enumi}.}
\item
  Niveles de ingresos y pobreza elevados: En nuestro país, los niveles
  de ingresos son muy bajos para la mayoría de la población, lo que
  genera altos índices de pobreza. Esta situación dificulta la capacidad
  de generar ahorros internos, ya que la prioridad es cubrir las
  necesidades básicas.
\item
  Paro encubierto masivo (subempleo): El subempleo es una forma de
  trabajo informal que prevalece en nuestra sociedad. Muchas personas no
  encuentran empleos formales o bien remunerados, lo que limita sus
  oportunidades de desarrollo económico y social.
\item
  Dependencia de productos de exportación: Nuestra economía está
  \textbf{altamente dependiente de un número reducido de productos que
  exportamos}. En particular, la minería y la pesca son fuentes
  importantes de ingresos, pero esto nos expone a riesgos y volatilidad
  en los mercados internacionales.
\item
  Escasa capacidad de ahorro: Debido a los bajos niveles de ingresos y
  la falta de oportunidades de empleo digno, \textbf{la capacidad de
  ahorro en los países en desarrollo como el nuestro es limitada o
  incluso inexistente.} Esto dificulta la acumulación de capital
  necesario para impulsar el desarrollo económico sostenible.
\end{enumerate}

En este contexto, es crucial que se priorice la distribución del ingreso
de manera más equitativa para reducir la brecha entre los sectores más
desfavorecidos y aquellos con mayores recursos. Una distribución más
justa del ingreso puede contribuir a mejorar las condiciones de vida de
la población y fomentar un crecimiento económico más inclusivo y
sostenible.

La brecha existente entre los pobres y los ricos, tanto a nivel nacional
como internacional, es una realidad que caracteriza a muchas sociedades.

Cuando analizamos un país como \textbf{subdesarrollado}, nos referimos a
la situación estructural e institucional en la que se desenvuelve esa
sociedad.

Por otro lado, cuando hablamos de un \textbf{país en vías de
desarrollo}, nos referimos al desaprovechamiento del potencial
productivo que existe en las economías con problemas de desarrollo. En
estos casos, los recursos disponibles no se utilizan de manera adecuada,
lo que impide su pleno funcionamiento y producción.

Es importante reconocer que, \textbf{tecnológicamente, nos enfrentamos a
limitaciones para sustituir muchos productos}, lo que nos hace depender
de otros países en ese aspecto. Además, políticamente, tendemos a imitar
modelos económicos extranjeros sin adaptarlos a nuestra realidad, lo que
también contribuye a nuestra dependencia de otros.

Cuando caracterizamos a un país como ``no industrializado'', nos
referimos a la ausencia de industrias en su territorio. Para nosotros,
\textbf{el desarrollo industrial es una variable fundamental}, ya que
consideramos que a partir de él se pueden complementar otras actividades
económicas. En este análisis, es importante destacar que existe una
posición ideológica que influye en cómo se percibe la economía y la
propia naturaleza del país, así como el tipo de gobierno que puede
surgir.

Los países con problemas de desarrollo enfrentan una serie compleja de
problemas interrelacionados, los cuales se manifiestan en notorias
desigualdades. Además, experimentan situaciones de estancamiento, donde
sociedades que estaban en proceso de avance y crecimiento frenan su
desarrollo, perdiendo el impulso de mejoras continuas.

En términos del Producto Interno Bruto (PIB), \textbf{observamos un
estancamiento}, con un crecimiento reducido en comparación con el ritmo
acelerado que se experimentaba previamente.

Corremos el riesgo de quedarnos rezagados en comparación con otros
países si seguimos tomando decisiones incorrectas que no nos permitan
avanzar en la reducción de la pobreza o impulsar un crecimiento
económico más sólido (medido a través del Producto Interno Bruto - PIB).
Las potencialidades productivas de nuestro país seguirán siendo
desaprovechadas debido a la influencia política e ideológica en la forma
en que se gestionan los componentes de nuestra sociedad.

Si las medidas adoptadas por el Estado a través del Gobierno no son
coherentes y no se alinean con los indicadores mencionados, será difícil
alcanzar los objetivos deseados. Es fundamental que se implementen
políticas y acciones que promuevan la estabilidad económica, la
generación de empleo, la inversión en sectores estratégicos y el fomento
de la innovación. De esta manera, podremos potenciar nuestras
capacidades productivas y mejorar la calidad de vida de la población en
general.

Tomemos como ejemplo la reducción de la pobreza y el crecimiento del
Producto Interno Bruto (PIB). Si no tomamos medidas adecuadas, corremos
el riesgo de perder velocidad y quedarnos rezagados en comparación con
aquellos países que están experimentando un crecimiento más acelerado.

Es fundamental utilizar eficientemente los recursos disponibles para
superar los desafíos que enfrentamos. Reconocemos que \textbf{nuestra
dependencia económica es significativa}, ya que muchas de nuestras
actividades requieren tecnologías externas, como en el caso de la
agronomía que necesita fertilizantes e insecticidas.

También somos conscientes de nuestra \textbf{dependencia cultural,
política y tecnológica}. Tenemos una aspiración de lograr una
distribución más equitativa en nuestra política, pero es importante
tener en cuenta que alcanzar la igualdad absoluta puede ser un objetivo
inalcanzable y ha llevado al colapso económico en otras economías.

Además, \textbf{enfrentamos altas tasas de crecimiento poblacional}, lo
cual es un desafío común en los países en vías de desarrollo. Es
esencial basarnos en información precisa y actualizada para comprender y
abordar este problema demográfico.

En la actualidad, la importancia de las economías, sociedades y países
no solo radica en su capacidad para crecer a una velocidad más alta que
otros, sino también en su capacidad para albergar y atender a una gran
población de manera efectiva. Es necesario adoptar estrategias que
impulsen el desarrollo sostenible, promoviendo un crecimiento inclusivo
y equitativo que beneficie a todos los sectores de la sociedad.

Tomemos como ejemplo a Estados Unidos (EE.UU.), cuya población es de
aproximadamente 350 millones de habitantes. Aunque su número de
población no es muy alto en comparación con otros países, su importancia
radica en su papel como consumidores voraces. Sus niveles de ingresos
altos y su capacidad de consumo los convierten en un actor relevante en
la economía global.

Por otro lado, China, con una población tres o cuatro veces más grande
que la de EE.UU., ha logrado un papel destacado en el escenario mundial
debido a su enorme población y su potencial como consumidores. Si China
tuviera una capacidad de consumo aún mayor que la de EE.UU., su
importancia sería aún mayor.

Es necesario considerar el futuro y no solo centrarse en la reducción
del crecimiento poblacional, sino también en implementar políticas que
fomenten un crecimiento poblacional adecuado y equilibrado. En el caso
del Perú, con una población de aproximadamente 32 millones de
habitantes, si lográramos niveles de ingresos más altos, podríamos
aumentar nuestra productividad y hacer que nuestro mercado sea más
atractivo. Por ejemplo, Chile, con una población de aproximadamente 18
millones de habitantes, logra producir más que nosotros, lo que indica
que \textbf{la mano de obra en Chile es más productiva que en el Perú}.

\textbf{Cuando nos referimos a la ``productividad'', estamos hablando
principalmente de la eficiencia y el rendimiento de la mano de obra} en
una sociedad.

Por otro lado, \textbf{cuando mencionamos los ``rendimientos'', nos
referimos a la capacidad de generación de capital financiero}, tierra y
también bienes de capital como maquinarias y equipos.

Es fundamental trabajar en una política demográfica o poblacional en el
Perú, aspecto que ha sido descuidado hasta el momento. Este es un
indicador relevante para caracterizar a una sociedad menos desarrollada
o con problemas de desarrollo. Un ejemplo de ello es la
\textbf{existencia de grandes proporciones de mano de obra empleada en
el sector agrario}, especialmente en áreas rurales.

Por ejemplo, actualmente estamos experimentando escasez de mano de obra
en el sector agrario, y esta situación debe ser abordada. En particular,
se observa una escasez de mano de obra en las regiones de la Sierra,
mientras que en la Costa hay disponibilidad de trabajadores dispuestos a
aceptar los salarios ofrecidos por las empresas.

Es interesante notar que los salarios en la Sierra son más altos que en
la costa, pero a pesar de ello, la mano de obra de la costa suele ser
más productiva. Esto se debe en parte a que en la costa existe trabajo
permanente, lo que lleva a que muchos migrantes de regiones como
Ayacucho se dirijan hacia la zona de Ica en búsqueda de empleo. En la
actualidad, Ica cuenta con una gran cantidad de trabajadores
provenientes de Ayacucho.

Aquí nos enfrentamos a un dilema. Por ejemplo, en Estados Unidos, solo
el 7\% de la población se dedica al sector agrícola, lo que les permite
abastecer al mundo con productos agropecuarios (un indicador relevante a
considerar).

En países como el nuestro, los niveles de ingreso se sitúan en realidad
en rangos de pobreza. Esta situación no permite ahorrar lo suficiente
para financiar el crecimiento o la inversión en el país. Como resultado,
el ahorro es escaso e incluso nulo en ocasiones.

Como consecuencia, siempre dependemos de capitales externos y carecemos
de capital interno. Por lo tanto, es necesario crear constantemente las
condiciones favorables para la instalación y el desarrollo de empresas.

Debemos facilitar la creación de empresas y asegurarnos de que estas
puedan mantenerse a lo largo del tiempo y prosperar. Sin embargo, según
los datos de la SUNAT, el 80\% de las empresas que se crean en un año
desaparecen ese mismo año. Del 20\% restante que logra sobrevivir el
primer año, el 60\% desaparece antes de alcanzar el tercer año.

Esto nos muestra una realidad preocupante: de cada 100 empresas que
nacen con entusiasmo y sacrificio, ya sean pequeñas o grandes, el 97\%
desaparece del mercado. Esto indica que economías como la nuestra
conllevan un alto riesgo, ya que existe una gran inestabilidad debido a
los constantes cambios en las reglas del juego. Los proyectos no se
cumplen porque los supuestos con los que se trabaja van cambiando.

En lugar de crear las condiciones favorables, el Estado se convierte en
un obstáculo para las empresas, lo que agrava aún más la situación.

Otro factor relevante a considerar es el \textbf{alto nivel de desempleo
y subempleo}, el cual a menudo está encubierto por empleos generados por
los propios trabajadores (empleo informal).

\textbf{La informalidad} juega un papel importante como indicador para
determinar el nivel de desarrollo de una sociedad. La formalización del
empleo es una aspiración común en todas las sociedades. Sin embargo, es
importante destacar que la informalidad no se limita solo a la evasión
de impuestos, aunque a menudo se perciba de esa manera.

Otro indicador destacado en países como el nuestro es la gran
dependencia de un reducido número de productos de exportación. Por
ejemplo, en el caso de Perú, la minería genera importantes ingresos y
constituye una parte fundamental de nuestro presupuesto nacional. Además
de la minería, también dependemos de actividades como la pesca, la
agricultura y los ingresos derivados de la venta de gasolina y cerveza.

Sin embargo, la minería es el sector que actualmente sostiene en mayor
medida nuestro presupuesto, ya que la agricultura ha sufrido
contratiempos y la pesca ha perdido control en las aguas marítimas. Por
lo tanto, nuestra economía se encuentra altamente dependiente de estas
tres actividades, junto con los ingresos generados por la venta de
gasolina y cerveza. Esto nos expone a ciertos riesgos debido a la
limitada diversificación de nuestra economía.

Al analizar la producción y el impacto en el presupuesto de la
República, es importante considerar el porcentaje del Producto Interno
Bruto (PBI) destinado a dicho presupuesto. En el caso del presupuesto
del año 2021, este se encuentra financiado principalmente mediante
endeudamiento público.

Además, otra característica relevante a tener en cuenta es la
concentración del poder de toma de decisiones en manos de una minoría
enriquecida. Esta élite gobernante se opone a cualquier tipo de cambio
que pueda mejorar los indicadores y condiciones de vida de la población
más necesitada. Es crucial abordar esta situación y buscar una
representación más amplia y equitativa que promueva el bienestar de toda
la sociedad.

\hypertarget{clasificamos-en-tres-sectores-a-la-economuxeda}{%
\subsection{Clasificamos en tres sectores a la
economía}\label{clasificamos-en-tres-sectores-a-la-economuxeda}}

La economía se clasifica en tres sectores que reflejan la importancia de
las diferentes actividades realizadas en una sociedad.

El \textbf{sector primario} engloba las actividades de producción básica
que están directamente relacionadas con los recursos naturales y su
extracción. Incluye la agricultura, ganadería, pesca, minería y
explotación forestal. Estas actividades son fundamentales para obtener
materias primas y alimentos.

El \textbf{sector secundario} se dedica a la producción de bienes a
partir de las materias primas obtenidas en el sector primario. Aquí se
encuentran la industria, la construcción y la manufactura. Es en este
sector donde se lleva a cabo la transformación de los recursos en
productos acabados, como cemento, telas y otros bienes manufacturados.

Por último, el \textbf{sector terciario} se centra en la producción de
servicios. Aquí se incluyen actividades como la banca, la educación, el
comercio, la cultura y los servicios a domicilio. Los países más
exitosos y desarrollados suelen tener una economía principalmente
orientada hacia este sector de servicios, seguido por la producción de
bienes. En contraste, países como el nuestro se ubican en el sector
primario, con una economía basada en la extracción de recursos
naturales, sin realizar una gran cantidad de transformación o
procesamiento.

\textbf{Existen dos motivos fundamentales por los cuales se lleva a cabo
la actividad económica}. Por un lado, la \textbf{necesidad} humana de
obtener provisiones básicas, como alimentos y ropa, impulsa la economía
para garantizar su disponibilidad. Por otro lado, el \textbf{deseo
humano de participar en actividades creativas y productivas} también
juega un papel importante en la economía, ya que permite dar sentido y
propósito a la vida de las personas.

\hypertarget{a-quuxe9-se-debe-que-no-transformamos-los-minerales}{%
\subsection{¿A qué se debe que no transformamos los
minerales?}\label{a-quuxe9-se-debe-que-no-transformamos-los-minerales}}

\begin{quote}
La falta de transformación de los minerales, como el cobre, en nuestro
país se debe a diversos factores, entre ellos la falta de tecnologías
adecuadas y la alta inversión requerida.
\end{quote}

En cuanto a la falta de tecnologías, no contamos con las herramientas
necesarias ni la capacidad de producción a gran escala para realizar la
transformación de minerales de manera eficiente. Sin embargo, existen
ejemplos como la empresa Altamina en el sector primario de la minería,
que ha logrado alcanzar tecnología de punta a nivel mundial y ha logrado
reducir los costos de producción gracias a su eficiencia.

En términos de inversión, las empresas mineras requieren de grandes
cantidades de capital para poner en marcha sus operaciones. En el caso
de Perú, con excepción de unas pocas empresas, la mayoría depende de
inversionistas externos debido a los bajos niveles de ahorro e incluso
la ausencia de ahorro en el país. Esto nos lleva a importar capital, ya
que \textbf{ningún inversionista o empresario invertiría si hay riesgo
de pérdidas}.

Si lográramos avanzar hacia el sector secundario, podríamos contar con
empresas nacionales que financien las transformaciones necesarias. En
este sentido, la inversión llegaría a través del sector bancario y
proporcionaría el respaldo financiero necesario para el desarrollo del
sector secundario. Sin embargo, este paso hacia el sector secundario es
difícil de lograr, lo cual también dificulta la entrada al sector
terciario, es decir, al sector de servicios.

\hypertarget{la-globalizaciuxf3n-y-las-economuxedas-regionales}{%
\section{La globalización y las economías
regionales}\label{la-globalizaciuxf3n-y-las-economuxedas-regionales}}

\begin{quote}
Entre el gobierno y la empresa debe estar la ciencia, la produccion
científica.
\end{quote}

La globalización ha generado una \textbf{creciente interdependencia
económica} entre todos los países del mundo, impulsada por el aumento
del comercio internacional de bienes y servicios. Además, se han
intensificado los flujos internacionales de capital y la difusión
acelerada de la información, junto con la generalización de la
tecnología.

Esta interdependencia económica implica la participación de todos los
países en las transacciones internacionales y un \textbf{flujo constante
de capital a nivel mundial}. La \textbf{difusión acelerada de
información} y la \textbf{tecnología estandarizada} han transformado la
forma en que nos comunicamos y nos relacionamos, lo que nos obliga a ser
conscientes de la importancia del tiempo como recurso.

En este contexto, \textbf{los Estados-Nación desempeñan un papel
fundamental como actores principales en los procesos de globalización}.
Se considera que el mundo es una fábrica global, donde China, por
ejemplo, se destaca como el taller del mundo. La sociedad ya no se
limita a lo local, sino que se ha vuelto global, y nuestra identidad
como ciudadanos trasciende las fronteras nacionales.

\textbf{La globalización no implica una igualdad de condiciones entre
las economías}. Las economías más poderosas subordinan a los países más
débiles, colocándolos en posiciones de apoyo o de servicio a sus
intereses, lo que resulta en una falta de equidad.

\textbf{Las relaciones internacionales son transversales y fluidas},
pero eso no significa que exista igualdad en las condiciones.
\textbf{Los países hegemónicos, que lideran la conducción mundial, han
establecido reglas que subordinan a las naciones más pequeñas y con
problemas de desarrollo}. En las sociedades globalizadas, los medios de
comunicación desempeñan un papel importante y predominante.

Los medios de comunicación rompen barreras y facilitan la transmisión de
valores comunes, lo que puede llevar a una ``homogeneización'' cultural
y la aparición de culturas de masas que se asemejan y estandarizan. Ante
esta tendencia, surge la necesidad de desarrollar mecanismos de defensa
para preservar la autonomía cultural y recuperarla en cierta medida.

La virtualización de la sociedad y \textbf{la estilización de la
realidad reflejan los usos y costumbres impuestos en el mundo}.
Asimismo, las empresas amplían su dominio a nivel mundial, extendiendo
su influencia.

Podemos afirmar que ninguna actividad humana escapa al proceso de
globalización.

\textbf{El principal problema de una globalización unilateral radica en
la creación de dependencias}, que en casos extremos podrían llevar a un
callejón sin salida en términos de distribución y problemas sociales,
contrarios al deseo general de estabilidad. Es importante tener en
cuenta esto para evitar desestabilizar la economía de aquellos países
que enfrentan dificultades en su desarrollo, pero que desempeñan un
papel fundamental como generadores del sector primario, es decir,
aquellos que producen insumos y materias primas principalmente.

\textbf{Los países que han logrado un mayor desarrollo} son aquellos que
han puesto mayor énfasis y \textbf{han avanzado de manera más eficiente
en el sector terciario} antes que en las economías basadas en el sector
primario.

Las alternativas que podemos proponer son las siguientes:

\begin{enumerate}
\def\labelenumi{\arabic{enumi}.}
\item
  Implementar propuestas que sean viables y factibles sin aislarnos del
  proceso de globalización.
\item
  \textbf{Establecer una economía abierta}, pero no orientada únicamente
  hacia el mercado mundial, ya que esto podría descuidar las necesidades
  de consumo interno. Es importante mantener un equilibrio entre las
  exportaciones y el consumo interno para evitar conmociones sociales y
  asegurar la estabilidad en los países desarrollados, considerando la
  interdependencia económica y la necesidad de mantener un suministro
  adecuado para el desarrollo industrial.
\item
  Reconocer que, si bien \textbf{dependemos de los países occidentales},
  ellos también dependen de las economías más pequeñas y con problemas
  de desarrollo. Esto brinda oportunidades para aprovechar los efectos
  de los precios y las materias primas a nuestro favor.
\end{enumerate}

Por ejemplos el Perú, tiene perspectivas favorables en el futuro debido
al precio alto del cobre y las oportunidades que esto brinda.

\textbf{La globalización, que es impulsada por el capitalismo}, ha
adquirido una dimensión global, organiza y reestructura el mundo según
sus propios intereses. Estos actores \textbf{poseen los instrumentos
para condicionar las economías que buscan superar sus problemas de
desarrollo} y aspiran a convertirse en economías emergentes. A nivel
internacional, cuentan con diversas empresas estratégicas en todo el
mundo, como se evidencia en el caso de Perú, donde hay una presencia
significativa de capital chileno.

Esto pone de manifiesto \textbf{las contradicciones que se encuentran
dentro del proceso de globalización}. Por ejemplo, la fuerte
centralización de capitales en busca de una mayor cobertura puede
conducir a ineficiencias, ya que lo que es rentable en unas economías
puede no serlo en otras. Las grandes empresas también enfrentan desafíos
al intentar homogeneizar su estructura organizativa, ya que la
centralización excesiva puede ser contraproducente. Asimismo, la
importancia de la tecnología para agilizar la producción puede verse
obstaculizada por las condiciones de los trabajadores. \textbf{Aunque
exista mano de obra barata en los países en desarrollo, la
implementación de tecnologías avanzadas puede verse limitada por la
falta de manejo adecuado de las tecnologías de los trabajadores
locales}.

Por un lado, los países menos desarrollados necesitamos la presencia de
tecnología en nuestras economías, y las grandes economías y empresas
desean imponer mejoras tecnológicas. Sin embargo, la mano de obra local
no siempre cuenta con las habilidades y condiciones necesarias para
llevar a cabo estos cambios o mejoras tecnológicas.

En este sentido, \textbf{se hace evidente la necesidad de mejorar el
nivel educativo y adaptarnos a las demandas tecnológicas}. Existe una
paradoja en la que, si bien los empresarios se benefician de una mano de
obra altamente calificada, las condiciones actuales no siempre permiten
su desarrollo. Esto plantea la necesidad de abordar la brecha educativa
y promover la formación de trabajadores capacitados en el manejo de las
tecnologías avanzadas, como parte de los esfuerzos por superar los
problemas de desarrollo en las economías menos desarrolladas.

\hypertarget{cuuxe1les-son-las-necesidades-de-la-mano-de-obra}{%
\subsection{¿Cuáles son las necesidades de la mano de
obra?}\label{cuuxe1les-son-las-necesidades-de-la-mano-de-obra}}

Las necesidades de la mano de obra se centran en contar con trabajadores
altamente capacitados y especializados en la producción de bienes o
servicios específicos. Sin embargo, a menudo nos encontramos con la
falta de satisfacción de estas expectativas en términos de desarrollo
local y geográfico en el que operamos.

Es importante interpretar este fenómeno como una forma de organización
económica orientada hacia el desarrollo regional, teniendo en cuenta el
ámbito en el que nos movemos, que es relativamente pequeño en
comparación con la escala global.

Dentro de este ámbito en el que vivimos, trabajamos y nos desenvolvemos,
\textbf{es crucial organizarnos de manera más localizada}. La
globalización, por su parte, genera una lógica que tiende a reducir las
autonomías, aumentar las interdependencias y fomentar la fragmentación
de las unidades territoriales, lo que puede resultar en la marginación
de ciertas áreas.

\hypertarget{la-economuxeda-regional-en-el-contexto-de-la-globalizaciuxf3n}{%
\section{La economía regional en el contexto de la
globalización}\label{la-economuxeda-regional-en-el-contexto-de-la-globalizaciuxf3n}}

La globalización es un fenómeno transversal y en constante flujo, que
afecta a todas las áreas de actividad humana. Sin embargo, también
existe la necesidad de contar con una organización económica a nivel
regional, más adaptada y ajustada a las características y dimensiones
locales.

\textbf{La implementación de una economía regional es complementaria a
la economía global y forma parte del proceso de globalización}. Aunque
el comercio y las actividades económicas se expanden a nivel
internacional, es en el ámbito regional donde surgen oportunidades de
mejora y desarrollo.

Es importante destacar que \textbf{la economía regional no se opone a la
globalización}, ya que esta última se encuentra presente en todas
nuestras actividades. La meta de la economía regional no es competir
directamente con la globalización, sino más bien encontrar su lugar
dentro de ella.

Si bien podemos hablar de competir a través de exportaciones a nivel
mundial, no debemos olvidar que otros países también conocen esos
productos y pueden ofrecerlos a precios más bajos y con alta calidad,
cumpliendo con los requisitos exigidos por los consumidores
internacionales.

Por lo tanto, no debemos descuidar el mercado local. Enfocarnos
únicamente en la producción para la exportación puede desabastecer la
economía regional y generar problemas. Es necesario organizarnos para
satisfacer las necesidades en todos los niveles, tanto regional como
local. Aunque no seamos autosuficientes, podemos generar la suficiente
riqueza para adquirir lo necesario y cubrir nuestras necesidades
regionales. El enfoque debe ser encontrar un equilibrio entre el mercado
global y el mercado local, de manera que podamos aprovechar las
oportunidades de la globalización sin descuidar nuestras propias
necesidades económicas regionales.

\hypertarget{alcanzar-el-bienestar-en-uxe1mbitos-regionales-y-la-importancia-de-la-innovaciuxf3n}{%
\subsection{Alcanzar el bienestar en ámbitos regionales y la importancia
de la
innovación}\label{alcanzar-el-bienestar-en-uxe1mbitos-regionales-y-la-importancia-de-la-innovaciuxf3n}}

\textbf{El objetivo fundamental es lograr recursos óptimos para la
población que reside en áreas geográficas específicas}, brindando a las
personas la oportunidad de asegurar su propio sustento y el de sus
familias, así como buscar el bienestar en el ámbito regional en el que
se desenvuelven.

Es crucial prestar especial atención a las actividades económicas
locales y buscar formas de fomentar su crecimiento y mejorar sus
condiciones para que puedan abastecer el mercado local en igual o
mejores condiciones que las empresas externas. Para lograr esto, se
deben implementar incentivos que promuevan la innovación y la mejora
continua.

En este mundo globalizado, es fundamental aprovechar la tecnología y la
información a través de dispositivos móviles para impulsar la producción
a nivel local. Internet nos brinda la posibilidad de acceder al mercado
global y expandir nuestras oportunidades.

Sin embargo, es importante tener en cuenta que tanto la mano de obra
como la toma de decisiones en estos ámbitos regionales más pequeños
pueden no estar completamente capacitadas para manejar la nueva
tecnología que se incorpora al proceso de producción. Asimismo, los
líderes y gobernantes en estas regiones pueden enfrentar desafíos para
generar los cambios necesarios y cumplir con las exigencias actuales.

Es fundamental elevar el nivel de exigencia y preparación en estos
ámbitos geográficos relativamente pequeños, ya que esto puede generar
mejores condiciones de vida para la población. Es necesario reconocer
que esto es un aspecto elemental y básico para alcanzar el desarrollo
deseado en estas regiones.

\hypertarget{percepciuxf3n-de-la-disociaciuxf3n-social-y-la-importancia-de-la-regionalizaciuxf3n}{%
\subsection{Percepción de la disociación social y la importancia de la
regionalización}\label{percepciuxf3n-de-la-disociaciuxf3n-social-y-la-importancia-de-la-regionalizaciuxf3n}}

Una percepción que surge es la disociación de la sociedad entre la
racionalización instrumental propia de la sociedad industrial y las
tecnologías utilizadas dentro de las economías que aspiran a alcanzar un
nivel de desarrollo autónomo.

Es crucial considerar cómo podemos respetar y homogeneizar ciertos
estándares de producción, al mismo tiempo que debemos tener en cuenta
las entidades étnicas y organizaciones ancestrales que juegan un papel
importante en el desarrollo regional.

Las organizaciones deben trabajar en la reducción de las tensiones
culturales internas generadas a través de argumentos ancestrales en
muchas ocasiones.

Tomemos como ejemplo la rivalidad entre Ayacucho y Huanta. En lugar de
profundizar esas diferencias, deberíamos aprovecharlas como una
oportunidad para reducir tensiones y unir fuerzas. El verdadero desafío
no radica en buscar la uniformidad, sino en construir la unidad a través
de nuestras diferencias. En esas diferencias se encuentra el capital y
la riqueza necesarios para impulsar mejoras en nuestra región y en
nuestro país.

Es importante reconocer que estas diferencias no nos debilitan, sino que
deberían convertirse en parte de nuestra fortaleza. Son estas
diferencias las que pueden propiciar el desarrollo de las regiones.

Aquí radica la fuerza del proceso de regionalización: cómo nos unimos
para mejorar las condiciones en las que nos desenvolvemos. Es a través
de la colaboración y la búsqueda de objetivos comunes que podremos
generar avances significativos en nuestras regiones.

\hypertarget{aprovechando-los-beneficios-de-la-globalizaciuxf3n-para-el-desarrollo-regional}{%
\subsection{Aprovechando los beneficios de la globalización para el
desarrollo
regional}\label{aprovechando-los-beneficios-de-la-globalizaciuxf3n-para-el-desarrollo-regional}}

Es importante reconocer que \textbf{oponerse al sistema global no es una
opción viable}. En cambio, \textbf{debemos aprovechar y crear las
condiciones para que la presencia de la globalización en sus diversas
dimensiones beneficie al desarrollo regional}. Estas dimensiones abarcan
aspectos tecnológicos, económicos, culturales, políticos e
institucionales. Sin embargo, es fundamental fortalecer nuestras
acciones en estas áreas para dirigir el efecto de la globalización a
favor de las regiones.

En el ámbito tecnológico, \textbf{es crucial asimilar y preparar a las
personas para manejar los avances tecnológicos} que ingresan. En cuanto
a la economía, debemos tener iniciativa propia y buscar nichos para
nuestros productos. \textbf{En el ámbito cultural, podemos aprovechar
actividades como el turismo} para destacar la diversidad cultural de
nuestro país. Desde el punto de vista político, es importante fortalecer
nuestra unidad y mantener ideas sólidas para garantizar continuidad en
lugar de comenzar desde cero.

\textbf{Las instituciones también deben mejorar, lo cual implica
reestructurar y organizar tanto el sector público como el sector
privado}. No podemos permitir que las empresas dicten las reglas del
juego en nuestro país.

Es cierto que el proceso de globalización otorga a las grandes empresas
ventajas significativas. Sin embargo, como economías regionales y como
país, debemos unirnos o crear las facilidades necesarias para mejorar
las condiciones de negociación. Además, en el ámbito ambiental, es
crucial incorporar criterios propios y locales en lugar de permitir que
se agoten nuestros recursos naturales. Por ejemplo, en la selva, debemos
proteger la parte forestal.

\textbf{La globalización tiende a disminuir la autonomía de nuestras
regiones y aumentar las interdependencias}. No obstante, debemos
trabajar en superar el fraccionamiento territorial existente, fomentando
una producción más sólida y de mayor calidad. Es esencial evitar la
marginación y, al mismo tiempo, respetar los usos y costumbres de
nuestros antepasados.

\hypertarget{diversas-perspectivas-sobre-la-relaciuxf3n-entre-globalizaciuxf3n-y-desarrollo-regional}{%
\subsection{Diversas perspectivas sobre la relación entre globalización
y desarrollo
regional}\label{diversas-perspectivas-sobre-la-relaciuxf3n-entre-globalizaciuxf3n-y-desarrollo-regional}}

Existen diferentes enfoques para abordar la relación entre el proceso
global y el proceso local/regional:

\begin{enumerate}
\def\labelenumi{\arabic{enumi}.}
\item
  \textbf{El desarrollo regional y local como alternativa a los efectos
  negativos de la globalización} en las economías regionales: Se plantea
  que el desarrollo a nivel regional puede contrarrestar los problemas
  generados por la globalización.
\item
  La determinación del desarrollo regional o local por parte del proceso
  de globalización: \textbf{El proceso de globalización tiene un papel
  determinante en el desarrollo regional o local}.
\item
  La necesidad de \textbf{articular lo regional con lo global}:
  Priorizar la integración de lo regional con lo global en cada economía
  regional, comprendiendo la complejidad de crear condiciones favorables
  frente a la globalización. Prepararse para aprovechar la presencia del
  proceso de globalización en beneficio de las diversas actividades
  regionales.
\end{enumerate}

Estas son tres formas de abordar la globalización desde una perspectiva
local o regional.

\textbf{La economía regional se considera complementaria a la economía
global}, y se resalta la importancia de reconocer los problemas que
surgen de una orientación exclusivamente global, como la dependencia y
la inestabilidad.

Por el contrario, \textbf{una economía regional plantea la necesidad de
buscar cierto grado de independencia económica} y autoabastecimiento, ya
que se desarrolla internamente. Mientras que el proceso de globalización
se caracteriza por su inestabilidad y alto nivel de dependencia.

\textbf{Una organización regional coherente y bien articulada reduce
esta dependencia y permite alcanzar cierto nivel de independencia
económica}. De esta manera, se puede proveer a la economía regional en
gran medida de sus necesidades. Por lo tanto, la orientación hacia una
economía descentralizada tiene como objetivo satisfacer las necesidades
de la población local. Aquí radica la importancia del proceso de
descentralización en un país.

\textbf{El proceso de regionalización implica descentralización}, ya que
cuando un país se encuentra centralizado, la dependencia se vuelve más
fuerte y la influencia del proceso de globalización es mayor. Esto se
debe a que las decisiones son tomadas por autoridades o entidades
gubernamentales que no están cercanas a las necesidades de las pequeñas
áreas geográficas en las que nos encontramos.

Por ejemplo, el sector de la salud depende del Gobierno Regional. Surge
entonces la pregunta: ¿Quién tiene el control en la dirección regional
de salud en Ayacucho?

\begin{quote}
El proceso de globalización es inherente al ser humano desde sus
inicios.
\end{quote}

Las economías orientadas regionalmente, es decir, las economías
descentralizadas con el objetivo de mejorar las economías regionales, se
caracterizan por tener una producción más cercana al consumidor.
\textbf{El productor se identifica más con el consumidor}, lo que hace
que la producción sea más transparente para él. El consumidor sabe que
el productor se esfuerza mucho para llegar al mercado y comienza a
establecer una conexión con él.

A nivel global, esto se relaciona mucho más con los derechos humanos y
la conciencia sobre prácticas más saludables para el medio ambiente.

Cuando se habla de \textbf{agricultura digital}, por ejemplo, se hace
referencia a la transparencia en los costos, lo cual es especialmente
relevante en una economía regional, donde la cercanía entre productor y
consumidor es mayor. La información que el productor maneja debe ser la
misma que el consumidor tiene, de manera que este último esté consciente
de que está pagando un precio justo que cubre las necesidades del
productor, y no al revés.

Además, en la economía regional, se busca acercar la vivienda, el
trabajo, la vida cotidiana, la cultura, la educación y los centros
laborales a la familia, respondiendo así a las necesidades de la
población. Esto implica una reducción en el transporte, en el tiempo
necesario para la producción y en las emisiones contaminantes que aún
persisten, como el caso de la quema de residuos, lo cual debe ser
mejorado y reducido a través de cambios en malas prácticas.

\textbf{La dependencia de los desarrollos globales y nacionales
disminuye gracias a la complementariedad de las economías regionales}.
Se reducen los niveles de dependencia y se priorizan los recursos
locales. Sin embargo, es importante considerar que las cuestiones
ideológicas pueden influir en este aspecto, por lo que se debe buscar la
manera de acercar esas diferencias y encontrar puntos de encuentro.

\textbf{En el ámbito político}, a nivel global, es evidente la creciente
comunicación e interdependencia entre los países, lo que revela la
dependencia económica que existe entre ellos. Sin embargo, esta
situación plantea riesgos, por lo que es necesario considerar la
unificación de los mercados y reducir esta dependencia. Es importante
aprovechar las oportunidades que se presentan cuando otros países tienen
déficits en determinados productos, ya que podemos convertirnos en
proveedores para su mercado.

Además, debemos incorporar en nuestras sociedades los aspectos
culturales y las costumbres que llegan a través de la globalización. Es
necesario realizar transformaciones sociales que nos permitan adaptarnos
y aceptar que ciertos modelos de sociedad han logrado un desarrollo y
una mejora en el bienestar. Observar los aspectos políticos en
diferentes países y analizar los modelos de carácter global que han
tenido éxito es fundamental. Asimismo, es importante reconocer que los
movimientos de capital a gran escala, impulsados por países más
avanzados, predominan en la escena mundial. Aunque seamos productores
primarios y estemos menos desarrollados, no dejamos de ser importantes,
ya que sin los recursos que proporcionamos, la industria no podría
funcionar y esta industria es la que genera servicios. Por tanto, es
necesario adherirse a los estándares universalmente aceptados.

En la práctica, se requiere un esfuerzo significativo y una gran
concientización de la población para participar activamente y lograr las
metas establecidas a nivel gubernamental. Para ello, es fundamental que
quienes toman decisiones en los gobiernos, ya sean a nivel nacional,
regional o local, tengan en cuenta el desarrollo regional y consideren
las condiciones específicas de los departamentos y del país en general.
Es crucial que el gobierno central o nacional genere confianza en la
población, por encima de cualquier interés personal que puedan tener.

No debemos perder de vista las atribuciones que debe tener el gobierno
central para atender las necesidades de la población y buscar
alternativas de solución. La descentralización debe convencernos de su
importancia y de los beneficios que puede traer consigo.

\hypertarget{resumen}{%
\subsection{Resumen}\label{resumen}}

Hablamos sobre \textbf{el proceso de globalización y cómo el desarrollo
regional puede mejorar las condiciones de vida de la población}. Es
importante reconocer que la globalización plantea desafíos
significativos para las regiones, pero no podemos detenerla. En cambio,
debemos considerar el desarrollo local y regional como una alternativa
para enfrentar los posibles efectos negativos de la globalización y
lograr una fusión entre lo regional y lo global. La comprensión de esta
relación compleja entre lo local y lo global nos brinda la oportunidad
de aprovechar la rapidez con la que se está dando el proceso de
globalización, lo cual es una habilidad clave de las regiones.

\textbf{Algunos países se integraron antes al proceso de globalización
que otros}. Por ejemplo, economías sudamericanas como Chile estaban más
abiertas al mundo, implementando economías de libre mercado, mientras
que nosotros íbamos en sentido contrario, cerrando fronteras y sin pagar
deudas. Esto ha resultado en ventajas para aquellas economías que se
integraron primero.

\textbf{El proceso de regionalización puede ser perjudicial si no
estamos preparados para enfrentar los desafíos externos}. Sin embargo,
puede ser beneficioso si estamos preparados y aprovechamos las
oportunidades que nos brinda el mercado externo. Por ejemplo, en el
sector agropecuario, tenemos recursos abundantes que aún no hemos
explotado ni exportado, como el aguaymanto. Debemos aprovechar la
llegada de la tecnología como resultado de la globalización.

La globalización ofrece una oportunidad única para el proceso de
regionalización. La economía regional es complementaria a la economía
global, y hay tres formas de abordar esta relación para lograr que la
economía regional sea complementaria. Esto permitiría reducir la
dependencia de los países menos desarrollados con respecto a los países
occidentales y tener la capacidad de producir muchos bienes o
autoabastecernos de productos que \textbf{garanticen la seguridad
alimentaria} y reduzcan la dependencia externa.

\textbf{La independencia de nuestras regiones radica en una estructura
económica descentralizada}, donde los esfuerzos económicos no se centren
únicamente en Lima, sino que se distribuyan en todo el país. Cada región
debe estar preparada para recibir los recursos del gobierno central,
priorizar sus proyectos y estimular su desarrollo. Sin embargo, en
realidad, observamos que no estamos preparados para administrar nuestro
propio destino, ya que no se priorizan los proyectos de inversión y se
posterga el desarrollo regional debido a la falta de preparación de
nuestras autoridades para la gestión gubernamental.

La regionalización nos permite orientar la economía y acercar la
producción al consumidor. Esto significa que el consumidor tiene una
mejor comprensión de cómo el productor enfrenta los desafíos. Al reducir
la distribución, tanto el productor como el consumidor se benefician al
obtener precios más accesibles. La clave para afirmar que \textbf{el
proceso de regionalización o la economía regional es una alternativa
complementaria a la globalización} radica en la capacidad de
autoabastecimiento y en la reducción de la presión de la competencia
mundial mediante la disponibilidad de productos regionales.

Es importante crear las condiciones para que los recursos locales sean
preferidos por los consumidores locales o regionales, lo cual requiere
rendimientos suficientemente altos para reducir costos.

\textbf{La globalización unifica los mercados y transforma el
comportamiento social}, la organización social y la política a nivel
interno. Además, genera una creciente dependencia en las comunicaciones,
las cuales se vuelven más fluidas. A pesar de esto, el movimiento de
capitales a escala mundial puede en algún momento beneficiar a nuestras
regiones, ya que las tasas de interés pueden ser reducidas si tenemos un
control financiero abierto. No podemos permitir que el sistema
financiero siga operando como hasta ahora, con tasas de interés del
200\% o 250\%. Es necesario establecer límites y alguna forma de
intervención estatal, ya que sabemos que el Estado debe intervenir.
Cuando el Gobierno es débil, las instituciones financieras aprovechan
estas oportunidades para imponer tasas de interés muy altas, incluso
cuando el costo del dinero es muy bajo. Por lo tanto, se deben
establecer límites a las tasas de interés, ya que \textbf{los ahorros en
países como el nuestro son relativamente bajos o nulos}, y necesitamos
el flujo de capitales extranjeros. Los intermediarios se aprovechan de
esta situación para lucrarse con las tasas de interés como con cualquier
producto escaso.

\textbf{Los gobiernos han perdido control y autoridad frente a las
organizaciones internacionales debido a su incapacidad para intervenir}.
Por lo tanto, necesitamos gobiernos relativamente fuertes para que las
propuestas se hagan efectivas y no se queden en meras leyes o
reglamentos. Es fundamental mejorar el sistema público en su conjunto.

\begin{quote}
Es importante destacar que el Producto Interno Bruto (PIB) es un
indicador que representa el valor de los bienes y servicios finales
producidos durante un año, y no debe confundirse con el presupuesto
público, que son conceptos completamente diferentes.
\end{quote}

\hypertarget{el-crecimiento-econuxf3mico-y-la-reducciuxf3n-de-la-pobreza}{%
\section{El crecimiento económico y la reducción de la
pobreza}\label{el-crecimiento-econuxf3mico-y-la-reducciuxf3n-de-la-pobreza}}

El crecimiento económico, medido principalmente a través del Producto
Interno Bruto (PIB), puede desempeñar un papel importante en la
reducción de la pobreza. \textbf{No todo crecimiento económico implica
desarrollo}, pero puede proporcionar oportunidades para una mejor
distribución de la riqueza generada en el país. Esto se logra mediante
la capacidad del Estado y el gobierno para disponer de mayores recursos
y dirigir el gasto público e inversiones hacia las personas más
vulnerables de la sociedad.

En nuestro país, la pobreza y la pobreza extrema son realidades que
debemos abordar. Es una tarea que debemos asumir y aquellos que tenemos
la posibilidad de contribuir a combatir la pobreza debemos hacerlo para
mejorar las condiciones de vida de una gran parte de la población.

Es fundamental \textbf{combinar el capital humano con el financiero}.
Por un lado, tenemos una mano de obra proveniente de la población pobre
y extremadamente pobre, que se encuentra en una situación de desventaja
constante, con baja productividad y falta de capital humano. Por otro
lado, es necesario obtener los recursos financieros necesarios para
atender a este segmento de la población.

El mundo está preocupado por esta situación, sin embargo, las grandes
organizaciones y los países más desarrollados enfrentan el desafío de
equilibrar su crecimiento económico con sus prioridades de seguridad
nacional. Algunos países occidentalizados, como Estados Unidos, ven una
posible amenaza en la pobreza que afecta a los países menos
desarrollados y han tomado iniciativas para apoyarlos, aunque su
principal motivación está relacionada con la seguridad nacional. Es
importante reconocer que, \textbf{aunque existe una obligación moral a
nivel global para abordar este problema, la moral de los países ricos a
menudo se centra en generar mayor riqueza y no en atender las
necesidades de los trabajadores o la pobreza.}

En los países menos desarrollados, la situación puede volverse caótica y
resultar en una difícil cobertura de las necesidades básicas por parte
del gobierno y de las organizaciones públicas y privadas.

\hypertarget{indicadores}{%
\subsection{Indicadores}\label{indicadores}}

\textbf{La migración es otra característica común en los países con
problemas de pobreza}, como es el caso de África, donde se evidencian
condiciones de pobreza significativas. Es importante reflexionar sobre
las condiciones económicas que llevan a las personas a verse atrapadas
entre la espada y la pared, optando por abandonar su lugar de origen y
buscar nuevas oportunidades en otros lugares. Aunque el Producto Interno
Bruto (PIB) no se incrementa, la población sigue creciendo y las
necesidades aumentan, mientras que los ingresos se reducen. Esto resulta
en un aumento de la población en situación de pobreza y extrema pobreza.
Estas circunstancias pueden deberse a prácticas gubernamentales
inadecuadas o a \textbf{condiciones territoriales adversas}, como
sequías seguidas de inundaciones, que obligan a las personas a vivir en
condiciones inhumanas y sin esperanzas. \textbf{Estas situaciones
dificultan la superación de la pobreza}, pero es importante recordar que
siempre hay una luz al final del túnel. Debemos trabajar en la
generación de condiciones de igualdad ante la ley, para que la justicia
refuerce la posibilidad de superar la pobreza mediante el esfuerzo.

\textbf{Es necesario que las leyes no favorezcan en exceso a las grandes
empresas acumuladoras de riqueza}, sino que sean herramientas que
fomenten la igualdad de oportunidades para las personas en situación de
pobreza y extrema pobreza. \textbf{Otro indicador relevante es la
libertad}, ya que las personas deben poder ejercer sus habilidades y
capacidades, mientras el gobierno las respalda en sus iniciativas. En
países como el nuestro y otros que experimentan casos extremos de hambre
y miseria, los gobiernos suelen ser opresores, con frecuentes cambios de
gobierno y falta de estabilidad. Esto impide que las iniciativas de las
personas se desarrollen o se ven frustradas en su gestación.

\textbf{Es fundamental trabajar en la equidad en la distribución de la
riqueza generada por la sociedad a través del gobierno}. Las prioridades
de quienes toman decisiones deben enfocarse en una planificación
efectiva y en la implementación de iniciativas que mejoren las
condiciones de vida de la población en situación de pobreza y extrema
pobreza. \textbf{Debemos eliminar la pobreza extrema y reducir al mínimo
la pobreza}, ya que esta no contribuye al crecimiento económico y, de
hecho, obstaculiza la redistribución de la riqueza, lo cual no es
beneficioso para el desarrollo del país. Esto se debe a que los
individuos en situación de pobreza y extrema pobreza consumen más de lo
que generan, reduciendo así los recursos disponibles para la sociedad.

Es importante destacar que \textbf{las personas en situación de pobreza
y extrema pobreza no contribuyen de manera positiva al PIB}. Si bien el
crecimiento económico puede reducir la pobreza al generar más recursos
para el gobierno, es necesario mejorar las condiciones de vida de estas
personas, ya que su contribución al PIB es inferior a su consumo,
limitando las posibilidades de crecimiento. Por lo tanto, es imperativo
erradicar la pobreza y la pobreza extrema, para abrir camino hacia un
crecimiento más sólido y equitativo.

\hypertarget{cuxf3mo-erradicar-la-pobreza-en-un-mundo-de-abundancia}{%
\subsection{¿Cómo erradicar la pobreza en un mundo de
abundancia?}\label{cuxf3mo-erradicar-la-pobreza-en-un-mundo-de-abundancia}}

En nuestro mundo abundan los recursos, la riqueza y los alimentos, pero
lamentablemente no llegan de manera equitativa a toda la población. Por
lo tanto, es necesario redirigir esfuerzos y recursos hacia donde más se
necesitan.

El crecimiento económico desempeña un papel fundamental en esta tarea.
Las diferencias en el grado de desarrollo de las regiones también
influyen en el nivel de bienestar dentro de cada una de ellas. Por
ejemplo, en América Latina, países como Chile, Colombia, México y Brasil
muestran diferencias significativas en su desarrollo, y dentro de este
contexto se encuentra Perú. Estas disparidades tienen un impacto directo
en el bienestar de la población y en la concentración de la pobreza y la
extrema pobreza.

A medida que la contribución de la riqueza en la sociedad en su conjunto
aumenta, se mejoran las condiciones de vida y se reducen la pobreza y la
extrema pobreza. Las regiones se diferencian entre sí en función de su
contribución al Producto Interno Bruto (PIB). \textbf{Aquellas regiones
con un mayor crecimiento económico experimentan niveles más bajos de
pobreza y extrema pobreza.}

El crecimiento económico es esencial para abordar la reducción de la
pobreza, pero debe ser sostenible y a niveles elevados. Algunos ejemplos
muestran por qué es importante enfocarse en el crecimiento económico:

\begin{itemize}
\tightlist
\item
  No podemos permitirnos tener una fuerza laboral no preparada para
  liderar el país, ya que esto podría generar grandes problemas.
\item
  Las epidemias demuestran cómo se descapitalizan las economías.
\item
  En situaciones de pandemia a nivel mundial, las perspectivas de
  control y gestión de la mano de obra pueden verse rebasadas.
\end{itemize}

\textbf{Otro indicador} relevante a considerar son las
\textbf{inversiones directas externas}. Aquellos países que han atraído
más capital extranjero y han experimentado un mayor flujo de capitales
han logrado reducir la pobreza extrema y mejorar la situación de los más
pobres.

Para lograr esto, \textbf{debemos crear condiciones favorables para la
inversión privada y atraer capital a nuestra economía}. Es importante
ofrecer estabilidad jurídica al país y establecer reglas de juego a
largo plazo, evitando cambios constantes que generen inestabilidad y
ahuyenten a los inversionistas.

La libertad económica también es fundamental para atraer inversión
extranjera. Debemos mejorar las condiciones existentes, incluso si eso
significa ceder cierta autonomía como nación en beneficio de la
economía.

\textbf{La clave para incrementar los ingresos radica en un crecimiento
diferenciado, aunque sea a diferentes ritmos (Yn=PBI)}. Si deseamos que
Ayacucho alcancen el nivel de desarrollo de Arequipa, Trujillo o Piura,
debemos crecer más que ellos. Para lograrlo, necesitamos no solo
recursos, sino también tomar las mejores decisiones en la asignación de
esos recursos a actividades económicas.

Es importante hacer las cosas correctamente, de manera eficiente, eficaz
y efectiva. Además, debemos \textbf{aprovechar los beneficios de la
tecnología}, adaptándola a las necesidades de la población para
incorporarla en sus actividades económicas.

\textbf{El crecimiento económico debe ir de la mano de un desarrollo
urbanístico planificado} que esté directamente relacionado con el
aumento de la productividad agrícola, la densidad poblacional y las
necesidades del comercio y los servicios. De lo contrario, seguiremos
creando zonas de pobreza concentrada.

Es esencial planificar el crecimiento de la población, mejorar la
productividad agrícola y fortalecer las redes de distribución. \textbf{A
nivel mundial, el objetivo es erradicar la pobreza extrema y el hambre},
por lo que debemos priorizar las inversiones, identificar aquellas que
tengan un mayor efecto multiplicador y enfocarlas tanto en las personas
como en la infraestructura.

Para lograr resultados duraderos, es necesario crear sistemas de
responsabilidad mutua en los que los pobres también asuman su parte de
responsabilidad, no solo como beneficiarios de derechos, sino también
como agentes activos que cumplen con sus deberes y responsabilidades.

Otra estrategia clave se centra en los mecanismos de financiamiento. En
las zonas más pobres, estos mecanismos deben ser subsidiarios y asumidos
por el sector público, en lugar de depender exclusivamente del sector
privado.

\hypertarget{el-rol-econuxf3mico-del-estado}{%
\subsection{El Rol Económico del
Estado}\label{el-rol-econuxf3mico-del-estado}}

\begin{quote}
El estado debe de participar subsidiariamente
\end{quote}

En un contexto de economía de libre mercado, es fundamental que el
Estado desempeñe un papel activo y subsidiario para garantizar el
bienestar de la sociedad. Si bien se promueve la libertad de trabajo,
empresa, comercio e industria, se debe velar por no perjudicar la moral,
la salud ni la seguridad pública. Asimismo, el Estado tiene la
responsabilidad de proporcionar oportunidades de superación a los
sectores que enfrentan desigualdades, fomentando el desarrollo de
pequeñas empresas en todas sus formas.

En esta economía de mercado, resulta necesario que el Estado participe
activamente y no deje todo en manos del sector privado. En situaciones
donde el sector privado no se involucre o se convierta en un monopolio,
el Banco de la Nación podría intervenir como una entidad empresarial
para transferir recursos y proporcionar financiamiento a las personas en
situación de pobreza extrema, estableciendo tasas de interés bajas,
entre el 5\% y el 8\%. De esta manera, el Banco de la Nación cumpliría
el papel de regulador en términos de tasas de interés. Sin embargo,
existen restricciones constitucionales que limitan este tipo de
intervenciones.

Es importante reconocer que los programas sociales no deben generar
dependencia ni perjudicar a la población pobre y pobre extrema. En lugar
de ello, se debe buscar que estos programas fomenten la autonomía y la
capacitación de las personas, brindándoles las herramientas necesarias
para su desarrollo. El Estado debe fortalecer su capacidad de atención y
evitar que la población se aproveche de su incapacidad para brindar un
adecuado apoyo.

\hypertarget{cuxf3mo-es-que-se-implementa-la-poluxedtica-social-dentro-de-este-encare-de-reducciuxf3n-de-pobreza-y-extrema-pobreza}{%
\subsection{¿Cómo es que se implementa la política social dentro de este
encare de reducción de pobreza y extrema
pobreza?}\label{cuxf3mo-es-que-se-implementa-la-poluxedtica-social-dentro-de-este-encare-de-reducciuxf3n-de-pobreza-y-extrema-pobreza}}

Dentro del enfoque de reducción de la pobreza y extrema pobreza, se
implementan diversas estrategias con el objetivo de promover el
bienestar de la población y generar condiciones favorables para su
desarrollo. Algunas de estas estrategias clave incluyen la inversión en
infraestructura, mecanismos de financiación y la reducción de las tasas
de interés.

El Banco de la Nación puede desempeñar un papel fundamental al ofrecer
préstamos y facilitar el acceso al financiamiento para actividades
productivas. Es importante recalcar la necesidad de priorizar estas
inversiones en las personas que se encuentran atrapadas en la pobreza y
la pobreza extrema, estableciendo un orden de prioridades claro.

La pandemia ha evidenciado la importancia del \textbf{capital humano} y
la necesidad de proteger y fortalecer los recursos humanos capacitados.
Los jóvenes representan un valioso capital en términos de su edad y
potencial para adquirir experiencia. Tomar decisiones acertadas y
fomentar su desarrollo es fundamental para asegurar un futuro próspero.

La \textbf{inversión en infraestructura} también desempeña un papel
relevante en la generación de empleo. Por ejemplo, la construcción de
una escuela de tamaño considerable puede generar numerosos puestos de
trabajo directos e indirectos. Es esencial \textbf{fomentar la
responsabilidad mutua} y el trabajo en equipo, enseñando a los
trabajadores indirectos sobre sus deberes y responsabilidades, antes de
sus derechos. Esto evita la dependencia del gobierno y contribuye a
evitar el deterioro económico.

Asimismo, es necesario \textbf{establecer mecanismos de financiamiento}
adecuados, incluyendo la reducción de las tasas de interés. En este
sentido, se requiere una reforma constitucional para permitir que el
sector público participe en actividades financieras. La creación de
fondos rotatorios y sistemas de financiamiento, como Pro Compite, son
medidas que pueden contribuir al desarrollo económico y social.

\hypertarget{quiuxe9n-implementa-la-poluxedtica-social-en-la-reducciuxf3n-de-la-pobreza-y-la-extrema-pobreza}{%
\subsection{¿Quién Implementa la Política Social en la Reducción de la
Pobreza y la Extrema
Pobreza?}\label{quiuxe9n-implementa-la-poluxedtica-social-en-la-reducciuxf3n-de-la-pobreza-y-la-extrema-pobreza}}

En el marco de la reducción de la pobreza y la extrema pobreza, surge la
interrogante sobre quiénes son los responsables de implementar la
política social y económica, y cómo se priorizan los diferentes sectores
en este contexto. Poque nuestra prioridad es el crecimiento inmediato
que conlleva la consideración de sectores como la construcción y la
minería.

Es importante destacar la falta de entes reguladores en este ámbito. La
política económica, que impulsa el desarrollo de diversos sectores y su
contribución a la riqueza del país, debe estar acompañada de una
política social que defina los factores de producción y las acciones
necesarias para abordar la pobreza.

Desde una perspectiva social, surge la pregunta sobre quiénes toman las
decisiones en cuanto a la política.

Es necesario analizar la información proporcionada por los medios de
comunicación de mayor impacto. Si bien estos medios ofrecen noticias y
reportan los eventos diarios, es importante tener en cuenta que la
información implica el procesamiento de datos. Por tanto, se requiere
que los medios de comunicación brinden una información sólida, basada en
un análisis riguroso de los datos.

Al fomentar la constante información a la población, se logra generar
conciencia sobre la situación de pobreza que prevalece en el país. Sin
embargo, enfrentamos dificultades para alcanzar nuestros objetivos
debido a los desafíos y obstáculos existentes.

En cuanto a los mecanismos de financiamiento, se plantea la necesidad de
tasas de interés relativamente bajas. No obstante, la Constitución de la
República limita la intervención del Estado en el ámbito financiero,
incluso como ente regulador. Para cambiar esta situación, se requeriría
una reforma constitucional, lo cual depende del presidente y del
congreso.

Una alternativa podría ser establecer que las tasas de interés de la
banca y las entidades financieras se ajusten a los estándares
internacionales. En caso de incumplimiento, el Estado estaría obligado a
intervenir como ente regulador o a crear las condiciones necesarias para
facilitar el acceso al crédito y financiamiento para las pequeñas y
medianas empresas. Además, el Banco de la Nación (BN) podría desempeñar
un papel relevante al ofrecer financiamiento a tasas de interés entre
0.25\% y 0.5\% anuales para las actividades de las pequeñas empresas.

Es importante señalar que los bancos comerciales, como Interbank, BCP y
Continental, cobran intereses más altos en comparación con las tasas
mencionadas anteriormente. La legislación vigente en el ámbito bancario
y de seguros favorece a estos bancos, por lo que es necesario brindar al
BN la posibilidad de actuar como intermediario financiero no solo para
un mercado cautivo, como son los empleados públicos.

\begin{quote}
Riesgo: Participación Estatal en la Economía
\end{quote}

En el ámbito público, es fundamental contar con profesionales altamente
capacitados. El desafío radica en mejorar la educación y la formación de
nuestra población. Para lograrlo, es necesario abordar dos aspectos
clave: 1) mejorar la formación de los ciudadanos y profesionales, y 2)
llevar a cabo una reforma constitucional que cree las condiciones
propicias.

En la actualidad, nuestro país se encuentra estrechamente vinculado con
la comunidad internacional y ha experimentado un buen crecimiento
económico. Sin embargo, es crucial realizar ajustes para prevenir
problemas relacionados con monopolios, abusos monopolísticos y
organizaciones oligopólicas. Estas correcciones no implican que el
Estado deba asumir todas las actividades financieras, empresariales y
económicas, ya que en esos casos, la propiedad privada quedaría en
segundo plano.

La ley del servicio público busca reconocer y promover la meritocracia,
fomentando la valoración del mérito y la capacidad en el ámbito laboral.
Esto implica garantizar que los profesionales y ciudadanos más
capacitados ocupen los puestos más relevantes en la administración
pública y en la toma de decisiones económicas.

\hypertarget{estrategias-de-inversiuxf3n-para-reducir-la-pobreza}{%
\subsection{Estrategias de Inversión para Reducir la
Pobreza}\label{estrategias-de-inversiuxf3n-para-reducir-la-pobreza}}

La reducción de la pobreza requiere la implementación de
\textbf{inversiones estratégicas}, centrándose principalmente en dos
áreas clave: \textbf{las personas y la infraestructura}. En relación a
las personas, es necesario desarrollar enfoques innovadores para
capacitarlas en actividades, así como perfeccionar sus habilidades. Esto
implica establecer metodologías que consideren la agrupación por
actividades, oficios comunes o proximidad a determinadas áreas, y
también buscar mecanismos de financiamiento para apoyar sus iniciativas.

En cuanto a la infraestructura, su desarrollo depende de la capacidad y
calidad de los funcionarios encargados. Las inversiones deben estar
dirigidas a las personas en situación de pobreza y extrema pobreza, y se
deben encontrar formas especiales de capacitarlas y atraerlas hacia
programas de formación que les permitan elegir oficios y actividades
para su futuro desarrollo. Es importante que estas oportunidades sean
accesibles y que los beneficiarios comprendan los beneficios que pueden
obtener. Es crucial aclarar que no solo es responsabilidad del Estado o
del sector privado mejorar su situación, sino que los propios individuos
deben estar interesados en mejorar su bienestar.

En este sentido, el Banco de la Nación (BN) puede desempeñar un papel
importante al brindar apoyo financiero a través de una posible reforma
constitucional que permita su participación en actividades financieras.
También se puede considerar la modificación de la ley de banca y seguros
para establecer límites en las tasas de interés, lo cual permitiría al
Estado intervenir a través del BN sin la necesidad de crear nuevos
bancos. Esto evitaría que los pobres sean afectados por tasas de interés
más altas impuestas por otros bancos comerciales.

Es fundamental reconocer los diferentes tipos de capital con los que
cuentan los pobres y en qué aspectos es necesario invertir. Por ejemplo,
el capital humano en este segmento de la población muestra bajos niveles
de productividad, por lo que se requiere invertir en mejoras de
capacidades y proporcionar oportunidades de empleo en trabajos
accesibles que no demanden una capacitación exhaustiva. Las inversiones
en capital natural también son importantes, ya que los terrenos
cultivables necesitan mejoras en los suelos para ser productivos.

Asimismo, el capital institucional del sector público juega un papel
crucial. Se requiere una legislación comercial que permita la
participación de la población en condiciones favorables, así como un
sistema judicial eficiente. Además, los servicios gubernamentales deben
mejorar, evitando la aplicación de las mismas tarifas a los pobres y
extremadamente pobres que a la población en general.

Es esencial implementar políticas que promuevan la división del trabajo
y que sean identificadas y priorizadas por el Estado. Asimismo, se debe
dar prioridad al desarrollo del capital intelectual, enfocándose en el
conocimiento práctico antes que en el teórico. Esto implica invertir en
capital intelectual científico y tecnológico, y preparar a la población
en estas áreas.

\hypertarget{resumen-1}{%
\subsection{Resumen}\label{resumen-1}}

La influencia de los medios de comunicación en la política social es un
factor significativo a considerar.

Para superar la pobreza y la extrema pobreza, es importante aspirar a
alcanzar la prosperidad de los países más desarrollados, siguiendo el
ejemplo de aquellos donde la pobreza extrema ha desaparecido. En países
de ingresos medios, donde la clase media es dominante, esta
desapareciendo la pobreza a su mínima expresión. Aunque el crecimiento
económico es esencial, pero ¿cuál es el nivel necesario de crecimiento?.
Si existe una voluntad política adecuada, es posible erradicar la
pobreza, aunque este término se podría referir a un nivel de comodidad y
bienestar que incluya acceso a servicios básicos, incluso si hay ciertas
restricciones. En esencia, se podría hablar de una clase media baja.

Es importante destacar que en los países más desarrollados la pobreza
extrema ha desaparecido, aunque aún haya personas pobres, su número es
mínimo en comparación con los países en desarrollo. Estos últimos
enfrentan desafíos más complejos relacionados con la pobreza.

A nivel global, combatir la crisis de la deuda es uno de los aspectos
más relevantes para reducir la pobreza. Es fundamental evitar el
incremento de la presión de la deuda externa sobre el presupuesto
público o su relación con el PBI. En el caso específico del Perú, el
endeudamiento público se ha utilizado para gastos gubernamentales.

\textbf{La política comercial} también desempeña un papel importante en
la reducción de la pobreza. Los países que fomentan el libre comercio y
reducen las barreras arancelarias experimentan un crecimiento más
rápido, lo que a su vez contribuye a la disminución de la pobreza. En
particular, el sector agropecuario, donde se encuentra gran parte de la
población más pobre, puede beneficiarse de las oportunidades de
exportación. Mejorar la política comercial, incluyendo los tratados de
libre comercio, es fundamental en este sentido.

\textbf{La aplicación de la ciencia al desarrollo} es otro factor clave.
En este sentido, es importante que el gobierno destine fondos para
investigaciones que se centren en las áreas urbanas marginales y en el
sector agropecuario, donde el avance tecnológico es menos accesible
debido a la falta de inversión. Priorizar y canalizar recursos hacia
investigaciones aplicadas en estas áreas sería un gran avance para
reducir la pobreza y la extrema pobreza en la población.

\textbf{La gestión de medio ambiente} Es fundamental reconocer el
esfuerzo de aquellos que practican y protegen el medio ambiente mediante
transferencias.

El desarrollo de una sociedad, la mejora de las condiciones de vida y el
progreso son aspectos que proporcionan recursos a la población,
generando esperanza y posibilidades de superar la pobreza y la extrema
pobreza. Es esencial que las personas puedan participar en diversas
actividades económicas sin temor a la delincuencia o a la inestabilidad
normativa generada por el Estado. Esto implica que las reglas de juego
no cambien constantemente.

Debemos implemetar políticas globales que fomenten el desarrollo y la
reducción de la pobreza, como: la deuda, política comercial, ciencia
aplicada y gestión ambiental. Es necesario imprimir esperanza y tener
confianza en el gobierno y en los tomadores de decisiones. Estos
aspectos son cruciales para lograr la prosperidad y la seguridad del
país.

En cuanto a las inversiones, se pueden destinar recursos al desarrollo
agropecuario, ya que este sector alberga a la mayor parte de la
población pobre y en extrema pobreza. Además, es importante invertir en
la atención primaria de salud para abordar enfermedades comunes y
mejorar su calidad.

El \textbf{control demográfico} también requiere inversión, y es
necesario concienciar a las personas más pobres de que no deben tener
más hijos. Esta responsabilidad debe ser compartida entre el gobierno y
las familias, implementando políticas de control de natalidad dirigidas
a la población con menos posibilidades.

\textbf{Incentivar el ahorro} es otro aspecto relevante, ya que las
personas de clase media hacia arriba tienden a ahorrar, lo que
proporciona al gobierno central capital para invertir.

Es importante considerar actividades específicas para los sectores más
afectados por la pobreza, y el Estado debe realizar inversiones directas
demostrativas. Es fundamental contar con un personal capacitado y
consciente de que están trabajando en beneficio de las personas pobres,
y no solo para una empresa. Sin embargo, esto puede ser una restricción,
ya que los empleados públicos a veces no valoran adecuadamente el
trabajo realizado por el Estado en beneficio de la población.

En la actualidad, programas sociales con transferencias condicionadas,
han demostrado buenos resultados en países como Ecuador, Bolivia y Perú
con ``Programa Juntos''.

Es necesario replantear los reglamentos de todos los programas sociales
existentes para que sean condicionados. Se deben establecer requisitos
mínimos que las familias deben cumplir para ser beneficiarias de las
transferencias económicas. No se pueden entregar sin restricciones, ya
que esto perpetuaría la pobreza. Los programas sociales no resuelven los
problemas de la pobreza, solo los atenúan, y a menudo se convierten en
una forma de vida para muchas personas, lo que desincentiva el trabajo y
la generación de riqueza. Los gobiernos deben tener la capacidad de
eliminar o transformar programas sociales que no funcionen, evaluándolos
adecuadamente.

Es importante \textbf{evaluar los programas sociales} teniendo en cuenta
su estabilidad y capacidad para mantenerse a largo plazo, así como su
efectividad. Los programas deben adaptarse a las nuevas exigencias y
condiciones necesarias que surjan.

Las normativas existentes para el cumplimiento de los objetivos de los
programas sociales deben ser implementadas y respetadas. No se deben
tolerar entidades reguladoras que estén bajo la influencia de grandes
empresas y actúen en su beneficio en lugar del gobierno. Es necesario
reducir los sesgos y evitar que las decisiones políticas solo favorezcan
a ciertos grupos de poder. La transparencia y la imparcialidad son
fundamentales en este sentido.

\hypertarget{gobernabilidad-e-instituciones}{%
\subsection{Gobernabilidad e
instituciones}\label{gobernabilidad-e-instituciones}}

\textbf{La gobernabilidad de un país se basa en la solidez de sus
instituciones.} Implica la eficiente y eficaz gestión de los diferentes
sistemas gubernamentales, como el poder ejecutivo, legislativo y
judicial. Cada uno de estos poderes está representado por un conjunto de
instituciones que juegan un papel crucial en el logro de metas y
objetivos. Para que un Estado sea fuerte, \textbf{estas instituciones
deben ser eficaces y eficientes,} alcanzando los resultados esperados y
generando confianza en la población.

\textbf{La gestión gubernamental se rige por un marco presupuestario
adecuado,} que requiere una planificación experta y un enfoque técnico
en la función pública. Los recursos asignados deben destinarse
estratégicamente para lograr resultados concretos y traducir esa
confianza en acciones tangibles.

\textbf{El Estado es una entidad jurídica,} una representación
imaginaria que se materializa a través del gobierno. El presidente de la
república actúa como jefe de estado y de gobierno, siendo el enlace
entre el mundo abstracto y el mundo real.

El gobierno, a través de los diferentes poderes, brinda servicios a la
población. Estos poderes deben estar representados en todo el territorio
nacional en el que se encuentra el Estado. Hay tres niveles de gobierno:
central, regional y local.

Cuando las instituciones que representan al Estado están bien
establecidas, organizadas y ofrecen servicios efectivos, la población
percibe al gobierno como eficiente y presente. Para lograrlo, es
esencial contar con un marco presupuestario adecuado que atienda las
necesidades de la población. El presupuesto es una propuesta de
asignación de gastos y se convierte en una realidad tangible cuando se
ejecuta y se hace disponible.

El gobierno opera dentro de un \textbf{marco presupuestario} y este debe
reflejar las necesidades reales de la población. Además, debe
proporcionar los recursos necesarios para el funcionamiento del
gobierno, \textbf{ya que sin presupuesto no puede haber gobierno.} La
presencia adecuada o inadecuada del Estado a través del gobierno depende
en gran medida del marco presupuestario y de cómo se establecen las
prioridades para abordar las diversas necesidades. La planificación
estratégica desempeña un papel clave en este proceso.

Es fundamental que la priorización no se base únicamente en la voluntad
política, sino en la evaluación objetiva de las necesidades de la
población y las diversas actividades. Se deben crear estrategias y
establecer un orden de prioridades que reflejen de manera adecuada las
demandas y los desafíos del país.

\begin{quote}
Al fortalecer la gobernabilidad y las instituciones estatales, se
sientan las bases para una gestión eficiente, una representación
adecuada en todos los niveles de gobierno y un uso efectivo de los
recursos presupuestarios. Esto contribuye a generar confianza en la
población y a garantizar que el gobierno cumpla con su responsabilidad
de atender las necesidades del país de manera eficaz y transparente.
\end{quote}

++++++++

En la actualidad se está hablando de sistema de planificación, la
planificación se está sobreponiendo al presupuesto, porque finalmente
primero se debe de planificar y luego asignar el presupuesto, pero esto
no es así ya que primero se asigna el presupuesto y luego se planifica.

Por eso es necesario que la planificación esté en manos de expertos en
función pública y que conozcan cómo funciona, la aspiración del sistema
de planificación es que todo lo que ejecute el gobierno central debe de
estar previsto y que esté en orden de prioridades y que los políticos se
ciñan a los planes, que no cambien las reglas de juego por lo que debe
de haber continuidad.

La gobernabilidad depende de que sus instituciones sean fuertes,
respetables y cumplan que tengan una asignación presupuestal adecuada y
que se rijan con una planificación hecha por expertos en la materia,
crear un enfoque de función pública, no se puede improvisar tiene que
ser debidamente planificado tiene que ser previsto.

Las instituciones se fortalecen en tanto cuenten con las capacidades de
expertos que responden a su visión a su misión y alcanzan los logros
esperados.

La gobernabilidad del estado depende de la fortaleza de sus
instituciones, de la gestión estatal, gubernamental depende en su
integridad del nivel que ha alcanzado las instituciones públicas y está
basado y debe regirse dentro de un marco presupuestario adecuado y debe
de estar en manos de expertos en planificación y que conozcan del
enfoque técnico de la función pública.

En el mundo podemos decir que hay países que han alcanzado crecimiento
sostenido sin contar con instituciones sólidas, lo cierto es que los
países que han alcanzado el desarrollo y se encuentran en la etapa
intermedia entre países desarrollados y no desarrollados han alcanzado
este nivel de desarrollo y reducido la pobreza porque cuentan con
instituciones sólidas, bien organizadas y calidad de servicio.

Por lo que es una preocupación para el Perú, ya que si queremos reducir
la pobreza debemos tener instituciones que permitan que el estado a
través del gobierno llegan a la población objetiva, porque resulta que
algunas herramientas, algunos esfuerzos que se implementan parara llegar
a esa población pobre y extrema pobreza se diluye en el trayecto entre
la disponibilidad presupuestal, la disponibilidad para cumplir la
normativa y ejecutar, porque en el intermedio están las instituciones
públicas y estas en el caso peruano no traducen la normatividad
correctamente y no cumplen.

\hypertarget{en-el-peruxfa-la-calidad-de-las-instituciones-estuxe1-a-la-altura-de-las-metas-estuxe1n-las-instituciones-puxfablicas-organizadas-para-alcanzar-las-metas-que-se-les-propone}{%
\subsection{¿En el Perú la calidad de las instituciones está a la altura
de las metas, están las instituciones públicas organizadas para alcanzar
las metas que se les
propone?}\label{en-el-peruxfa-la-calidad-de-las-instituciones-estuxe1-a-la-altura-de-las-metas-estuxe1n-las-instituciones-puxfablicas-organizadas-para-alcanzar-las-metas-que-se-les-propone}}

Aun no, porque estamos en proceso, ya que las instituciones públicas
reflejan los intereses personales de los empleados públicos más no para
atender las necesidades y los intereses de la población a la cual está
dirigida esa institución pública. (ej: trabajadores del sector salud)
tenemos que saber darnos cuenta de que el aparato estatal, las
instituciones que representan al gobierno no están bien consolidadas, no
están bien organizadas y el personal no reflejan los objetivos que la
institución buscan alcanzar (El ministerio público investiga al que odia
y no al que realmente debe de investigar, la razón de ser del ministerio
público primero es que el fiscal debe pensar bajo la presunción de
inocencia, pero este hace la investigación pensando que ya es culpable)

En el Perú, en la década de los 90 se adoptó un enfoque basado en la
creación de organismos para la reforma estatal, incluso entre el 2000 y
2010 se vio que las instituciones estaban placados de tintes políticos y
se había desnaturalizado las instituciones, por lo que se adoptó un
enfoque que permita organismos, y generen un nivel de reforma estatal,
que hayan generado resultados positivos para algunas instituciones (ej.:
en el ministerio de economía y finanzas se incorporó el siaf, mejor
gestión fiscal, mejor ejecución de los gastos y no por eso se ha supero
que los diferentes ministerios lo ejecuten bien) la segunda fase debe de
concentrarse en mejorar la calidad del gasto fiscal, no solo en los
resultados que se esperan que las instituciones cumplan con la
normativa, la segunda fase se tiene que concentrar efectivamente en la
mejora de la calidad de gasto fiscal, y esta mejora pasa por lograr que
el país consolide la gestión fiscal, desarrolle estrategias de reforma
estatal futura, deberíamos empeñaros en un orden de prioridades para
mejorar la calidad del gasto, no se trata solo de gastar.

El ministerio de economía y finanzas tiene que mejorar la segunda etapa
de la reforma tiene que mejorar la calidad de vida.

La gestión fiscal a cumplir debe de ser estratégico, la reforma estatal
a futuro considerando algo que sume a las estrategias que ya se están
utilizando.

La futura reforma estatal tiene que estar basado en tres elementos
básicos (desafíos)

\begin{enumerate}
\def\labelenumi{\arabic{enumi}.}
\tightlist
\item
  El incremental ismo estratégico, hay que apuntalar que en toda
  institución pública su funcionamiento tiene que ser estratégico, todos
  del aparato público tenemos que aprender a generar estrategias
  (debemos darnos cuenta que es lo que buscamos), esto es aumentar el
  uso de las estrategias, todo trabajador del sector público debe saber
  generar o crear estrategias.
\item
  La reforma estatal de las instituciones del gobierno central
  (ministerios, opedes), LOPE (ley orgánica del poder ejecutivo, que
  establece las reglas de organización, gestión y competencias de uno de
  los principales poderes del Estado) se ha hecho en el año 2009 y se
  publicó en el 2010 pero no se ha desarrollado esa ley, pero hay que
  hacer una reforma estatal del gobierno central, ósea de los
  ministerios y los organismos públicos descentralizados (opedes) deben
  pasar por una reforma de modo que se adapten o adecuen a la nueva ley
  orgánica del poder ejecutivo pero a su vez esa ley hay que
  modernizarlas, adaptarlas a las nuevas condiciones y exigencias del
  país a fin de que la instituciones públicas se consoliden en el
  gobierno central, una vez que se esté haciendo eso se debe también
  hacer la reforma estatal más allá del poder ejecutivo, eso significa
  el poder judicial y el congreso ya que ahí también hay que
  restructurar, reorganizar y hay que cambiar la normativa existente,
  adaptarlas a las nuevas exigencias del país. Entonces hay que hacer
  cambios, hay que hacer reformas en el aparato estatal que ya no es el
  gobierno central.
\item
  Ingresar a los otros poderes (poder judicial y legislativo).
\end{enumerate}

En la perspectiva del futuro, en el nuevo contexto que se va encontrar
el Perú, es necesario pensar en la reinserción en los mercados
mundiales, tenemos que cambiar la política externa del país, nuestras
representaciones del exterior tienen que ser embajadores especializados
en inserción en mercados, no podemos solo tener una burocracia que solo
estén dedicado a la buena presencia y protocolos de etiqueta, tiene que
identificar mercados.

La descentralización tiene que mejorar no puede ser entendido que
cualquiera pueda hacer y deshacer y destrozar la institucionalidad de
los pliegos presupuestales, la descentralización tiene que refinarse,
tener que quitar algunas potestades al presidente regional, garantizar
la continuidad de los funcionarios, debe de haber un nivel de selección
del personal basada en la experiencia, continuidad del funcionario.

Si queremos afrontar en el futuro el nuevo contexto que se nos va
presentar tenemos que considerar la reinserción en los mercados, tenemos
que revisar el grado de competitividad de las empresas privadas y
mejorar las condiciones de la descentralización ya que estas necesitan
mucha dedicación para tener instituciones serias y para hacer esto
tenemos que afrontar desafíos, una de ellas es elevar la calidad del
crecimiento económico pensando en la reducción de la pobreza (esto es
importante porque si el crecimiento es de largo plazo a una tasa de 6 a
7\% anual durante 10 años de crecimiento la pobreza se va reducir y va
ser una reducción acelerada de la pobreza) y esto pasa por un
crecimiento sostenido.

Otro desafío es lograr que las instituciones públicas tengan mayor
credibilidad, lograr que la gente comience a creer en lo que pueden
hacer las instituciones públicas y para eso hay que demostrar con hechos
que lo que se dice se ejecuta.

Finalmente, otro desafío es mejor la eficiencia de los servicios, el
cumplimiento de los servicios, la calidad de los servicios tiene que
mejorar, no podemos seguir con trabajadores que no brindan un buen
servicio.

Necesitamos una nueva estrategia de reforma del sector público
considerando estos tres desafíos. (Calidad del crecimiento económico,
superar la incredulidad de estado y finalmente la eficiencia de los
servicios públicos)

Debemos considerar que una mejor gobernabilidad del aparato estatal pasa
por manejar los recursos de forma eficaz, la gobernabilidad debe ser
manejada con los mejores tomadores de decisiones (decisiones
trascendentales, decisiones de gobierno) debemos ser eficaces tomar
buenas decisiones y hacerlas correctamente.

Para mejorar la gobernabilidad debemos tomar las mejores decisiones,
debemos implementar políticas fiscales solidas que permitan que en el
tiempo se consolide ese tipo de decisiones para captar más ingresos y
una buena direccionalidad, esto va permitir mejorar los servicios a los
ciudadanos. (Aquí está el secreto para tomar las decisiones).

La población debe percibir que se cumpla con la ley, el control de la
corrupción se ha deteriorado, la contraloría hace un esfuerzo para
controlarlo y la población debe de estar involucrada, el estado debe ser
efectivo en su lucha contra la corrupción.

Debemos hacer una segunda reforma del estado que pasa por una
restructuración y reorganización, debemos mejorar la eficiencia del
estado, los servicios del sector público.

Hemos hablado de la participación de las instituciones públicas y su
influencia y determinación en la gobernabilidad del aparato estatal.

RESUMEN

Hemos hablado de Como elevar la calidad del crecimiento económico y que
eso nos puede permitir reducir la pobreza rápidamente y cómo podemos
lograr que las instituciones públicas tengan mayor credibilidad en el
país y como mejorar la eficiencia de los servicios para hacer frente a
estos desafíos que nos pone el contexto mundial como una estrategia del
sector público.

Debemos mejorar el manejo de los recursos públicos de manera más eficaz,
implementar políticas fiscal más sólidas y mejorar los servicios que
brinda el sector público a los ciudadanos y la percepción del
cumplimiento de las leyes, de acá a un tiempo las diferentes
instituciones del sector público como del privado no estamos cumpliendo
con la normativa existente, parece que no existe un control
gubernamental de modo que se deteriora la imagen del estado a través del
gobierno y la realidad nos dice que la corrupción se a incremento porque
el control de la misma por falta de cumplimento de sus obligaciones y
las leyes de parte del gobierno hace que se fomente la corrupción, por
lo que tenemos que superar haciendo que haya una efectividad del estado.

Es necesario mejorar y comenzar a implementar una reforma del aparato
estatal en búsqueda de mejorar e incrementar la eficacia a nivel
estatal.

Debemos recalcar que en los 3 niveles y poderes del gobierno no existe
voluntad de iniciar un control del manejo gubernamental y el cumplimento
de las leyes para controlar la corrupción, porque desde el interior de
estas instituciones representativas neurálgicas o su razón de estar del
gobierno no hay voluntad de cumplimiento de las normas, ya que pareciera
no existir el gobierno y los pequeños funcionarios hacen de las
decisiones lo que más les parece y les conviene para fines personales,
en la década de los 90 la reforma del estado ha sido drástico pero poco
sistemático, no se visualizó el futuro inmediato del aparato estatal.
Solo se vio el corto plazo, ya que el objetivo solo fue reducir el
tamaño del estado y no lo proyectaron, no se vio lo inmediato que sería
este proceso de racionalización de los recursos humanos.

CLASE

Mayores esfuerzos para el proceso de fiscalización, trasparencia y
coordinación

Cuando hablamos de fiscalización hablábamos del control que hacemos a lo
que el gobierno central, regional o local podía actuar, debemos de
verificar si se está cumpliendo o no de acuerdo a los parámetros, si se
han cumplido las normas pre establecidas para estos fines.

Cuando fiscalizamos revisamos si ha cumplido o no, como estaba fijado,
si se ha alcanzado las metas en el tiempo, en que cosas se han fallado.

La trasparencia es que la gente conozca en lo posible todos los
procedimientos. Debemos de hacer una reforma estatal de segunda
generación, esta reforma estatal que se puede hacer a partir del 28 de
julio, puede basarse en estrategias nuevas y efectivas, efectivas en el
sentido de que debe de ser sistemático, se debe de mirar el futuro.

Otra acción que debemos de considerar son las reformas en el gobierno
central Otra reforma debe de ser en el poder ejecutivo y poder judicial,
la reforma tiene que ser vertical y transversal, debemos de hacer las
reformas en base a 4 áreas:

\begin{enumerate}
\def\labelenumi{\arabic{enumi}.}
\item
  un nuevo rol de congreso (así como está el congreso no funciona,
  debemos de hacer una nueva reforma creando nuevos reglamentos y nuevas
  directivas y todo lo demás dejarlo sin efecto, comenzar nuevamente de
  cero pero sin dejar de actuar con lo que hay pero sin dejar de actuar
  con lo que hay).
\item
  Crear un nuevo rol de las instituciones presupuestarias, el sistema
  nacional de presupuesto ha variado y mejorado mucho pero viene
  cargando los vicios anteriores, por lo que debemos dejar sin efectos
  esos e implementar un nuevo rol de todas las unidades estructurales
  que tienen que ver con el presupuesto público y crear una nueva
  cultura de presupuesto.
\item
  Debemos de hacer reformas en todo lo que es administración pública, en
  todos los procedimientos y los sistemas administrativos diversos, hay
  que modificar casi todo para lo cual se necesita un congreso dinámico
  con gente capaz y ganas de hacer las cosas bien.
\item
  Debemos ver una reforma en la nueva ley de descentralización e incluir
  en esta descentralización al poder judicial, el poder judicial es un
  ente muy centralizado, incluso el dinero que ese ejecuta en Ayacucho
  se gasta y después rinden, no tienen autonomía en su ejecución, no hay
  capacidad de fiscalización no hay capacidad de control en el poder
  judicial.
\end{enumerate}

La reforma estatal debe de basarse en 3 razones:

\begin{itemize}
\tightlist
\item
  Estrategias nuevas y efectivas
\item
  Reformas en el gobierno central
\item
  Reforma en el resto de los poderes: legislativo y judicial
\item
  Las reformas se deben de hacer en 4 áreas
\end{itemize}

\begin{enumerate}
\def\labelenumi{\arabic{enumi}.}
\tightlist
\item
  En el congreso, debemos de hacer nuevas leyes, hay que darle un nuevo
  rol al congreso.
\item
  Debemos darles un nuevo rol a las instituciones presupuestarias,
  debemos de darle otra imagen, otras leyes, todo lo positivo debemos de
  acumularlo e incorporar en proyecciones, debemos de hacer reforma en
  todo lo que es el aparato estatal, todos los sistemas administrativos,
  la administración pública.
\item
  Finalmente se debe de profundizar la descentralización, mejorar la ley
  de descentralización, reformar estos gobiernos regionales, hay que
  restructurarlos, reorganizarlos y darle una estructura organizacional
  y ahí también incorporar al poder judicial, debemos de descentralizar
  el poder judicial.
\end{enumerate}

Esto es en cuanto a las reformas, las áreas en las que hay que hacer las
reformas.

\begin{enumerate}
\def\labelenumi{\arabic{enumi}.}
\setcounter{enumi}{3}
\tightlist
\item
  Lo otro es las reformas en las áreas institucionales y lo otro en
  áreas en generales a nivel nacional (macro).
\end{enumerate}

Las otras áreas institucionales en cada uno de estos poderes, en c/u de
estos niveles de gobiernos debemos de hacer una reforma institucional es
la organización del estado.

La ley orgánica del poder ejecutivo se hizo en el 2009, la cual esta
desactualizada\ldots y si vamos hacer reformas la ley orgánica tiene que
modificarse totalmente y mejorar, por lo que debemos de cambiar algunos
artículos y adecuarlos a las nuevas condiciones y exigencias.

Un área de reforma institucional es la organización del estado, nuevas
leyes como: la ley orgánica del poder ejecutivo, ley orgánica de
descentralización, ley orgánica de gobierno regional, ley orgánica de
municipalidades debemos de modificarlas y si modificamos todas estas
leyes todas las demás normas se modifican.

Un, reforma institucional que debemos de hacer en lo que es calidad del
gasto público, aquí tiene que ingresar un ajuste en las leyes,
reglamentos de la ejecución de gasto presupuestal, y esto es calidad del
gasto público que tiene que haber un orden de prioridades y que sin
autorización y si no se encuentran en el plan estratégico institucional
de mediano y largo plazo no se deben de ejecutar ningún proyecto.
Debemos de poner por encima del sistema de presupuesto al sistema de
planificación y por ahí podemos mejorar la calidad del gasto, debe de
haber un orden de prioridades, la calidad de gasto para que sea
mejorado, para que realmente funcione tendríamos que hace reformas en la
ley de recursos humanos u oficinas de personal, debemos de cambiar la
ley de adquisiciones, contrataciones y adquisiciones del estado,
cambiarlo lo más rápido, cambar los criterios de inversión pública, los
criterios del proceso de centralización, elecciones, la calidad del
gasto estaría garantizado, sabríamos cual se hace primero y cual se hace
después, estará en función a la disponibilidad presupuestal.

Otra área de reforma institucional es el manejo fiscal que es la
administración financiera del país, debemos de hacer un profundo
monitoreo, seguimiento y evaluación de la administración financiera para
mejorar la calidad del gasto, porque si queremos mejorar la calidad del
gasto debemos de saber con cuanto de fondos disponemos y cuáles son las
alternativas para financiar esos proyectos, esas obras que vamos hacer,
no confundamos la inversión con el gasto.

Y finalmente el área de reforma institucional son las reformas del
estado, el rol del congreso, instituciones presupuestarias y el combate
a la corrupción.

Debemos de pensar que debemos tener un combate a la corrupción desde el
arma legal, debemos de crear leyes que realmente sirva, ya que hemos
creado leyes contra la corrupción que en realidad no controlan nada, más
bien adormecen la lucha contra la corrupción por lo que hay que mejorar
el conjunto de leyes.

Otra cosa que debemos de tener en consideración la plena gobernabilidad
democrática para el Perú implica vigencia de los derechos humanos, pero
de los derechos humanos se deben de entender que no es para todos, sino
para las personas correctas, ósea para persona correctas ya que estas
más funcionan para personas torcidas para personas que han asesinado,
etc. Para ellos están hechos los derechos, más les hacen caso las
instituciones al ladrón.

Debemos de pensar que debemos de mejorar estas circunstancias,
lamentablemente hemos enseñado y aprendido a enseñar a la gente primero
sus derechos luego sus deberes cuando debería de ser al revés.

No podemos inculcar a la gente que solo tiene derechos sino que también
tiene deberes ya que los derechos humanos y su vigencia no debe de ver
discriminación debe de haber acceso a la justicia, debemos de prevenir,
la gestión de conflictos, la población no puede estar en constante
enfrentamiento.

Debemos de hacer gestión del conflicto ya que después del 28 de julio
puede venirse una avalancha de descontentos si gana keiko o gana
castillo\ldots. es una situación un poco difícil, podemos entrar en un
tramo de violencia por lo que debemos de implementar gestión de
conflictos, seguridad ciudadana, confianza en los políticos, un gobierno
más descentralizado y para eso necesitamos tener gobernadores regionales
más capacitados que entiendan el comportamiento del país en su conjunto
y en ese manejo como se inserta.

Finalmente hay que incorporar la eficiencia y la transparencia para que
recuperemos la credibilidad.

La gobernabilidad podemos definirlo como la capacidad técnica y política
con la cual el estado a través del gobierno cuenta para dar solución a
las demandas de la población.

La gobernabilidad es la capacidad de manejo de todos los instrumentos
con los que cuenta el aparato estatal para atender las necesidades de la
población.

La población en su mayoría no siempre tiene razón y para eso están los
gobernantes, para eso está el estado, para descifrar las necesidades de
la población y atenderlos, no todo lo que dice la población es cierto y
prioritarios, por lo que debemos de aprender a priorizar ya que el
pueblo es como un niño y este es nuestro problema, que no sabemos
diferenciar.

La gobernabilidad pasa por esa capacidad que tiene el estado a través
del gobierno, capacidad de manejo que tiene de todas esas herramientas
que tiene que principalmente está plasmado en la leyes y eso traducido a
la ejecución y haya una decisión políticas para resolver ese problema
entonces podemos atender las demandas del gobierno, en tanto no se
atiendan las demandas priorizadas de la población por parte del gobierno
simplemente no abra credibilidad y abra desgobierno no hay presencia del
Estado.

Por lo que debemos de tomar enserio la gobernabilidad, usar bien las
herramientas y para eso debemos de dominar bien las leyes (ley de
adquisiciones, ley de presupuestos)

Para que la gobernabilidad funcione y las políticas públicas se hagan
efectivas tiene que partir por la voluntad política para su
implementación y esa voluntad política tiene que tener como sustento la
capacidad técnica del manejo gubernamental, necesitamos gente en el
aparato estatal que realmente sepa de gestión pública, no gente con
títulos o maestrías sino gente que sepa cómo se hace

La existencia de políticas públicas y la creación de más política
públicas no significa que la gobernabilidad va mejorar, hay que
preocuparnos en mejorar la capacidad técnica y política. Hay que mejorar
la toma de decisiones, pero lo principal es la capacidad técnica de los
que van a tomar las decisiones debajo de el en la ejecución, su
prioridad es contar con gente eminentemente práctica.

Para la buena gobernabilidad debemos de considerar algunos principios
básicos:

Dijimos que debemos de hacer reforma en áreas institucionales, en base a
4 grandes áreas o cuando decíamos que debe basarse la reforma estatal en
3 pilares como lo son:

\begin{itemize}
\tightlist
\item
  las estrategias nuevas y efectivas,
\item
  la reforma del resto de los poderes
\item
  la reforma del gobierno central
\end{itemize}

Podemos decir que tenemos algunos principios que cumplir dentro del
manejo gubernamental, por ejemplo para que haya gobernabilidad tiene que
haber percepción de legitimidad de parte de la población, la población
tiene que sentir que el gobierno que está en el poder refleja sus
interese (la percepción de legitimidad)

Cuando decimos que el gobierno no nos representa, estamos desconociendo
la gobernabilidad, estamos desconociendo al máximo representante del
estado en el gobierno (presidente de la república) le quitamos la
legitimidad y si gran parte de la población desconoce a ese gobierno se
pierde la percepción de legitimidad, pierde manejo gubernamental, la
población no cree en lo que puede hacer o en lo promete hacer y este es
un principio básico el manejo gubernamental de la existencia de la
gobernabilidad estatal.

Otro principio es la importancia que se le debe de dar al ciudadano, la
importancia central al papel del ciudadano, a la función que cumple el
ciudadano. De ahí es que la gente dice que el gobierno no está
atendiendo a los intereses del pueblo, oses que hay promesas, acciones
incumplidas por parte del gobierno y que la ciudadanía lo está haciendo
notar, por lo que el gobierno debe de cumplir, por lo que los ciudadanos
viene a ser el centro de atención, pero no necesariamente se tiene que
atender todas sus necesidades pero si se le tiene que hacer sentir al
ciudadano que es participe de la acciones del gobierno, hacerlo parte de
él.

ej carreteras de la selva: el gobierno le falta dinero, el gobierno
brilla por su ausencia no hay presencia del gobierno en el quehacer de
la población, lo que tenemos presencia de instituciones que no
funcionan.

\hypertarget{los-4-principios-buxe1sicos-de-la-buena-gobernabilidad}{%
\subsection{Los 4 principios básicos de la buena
gobernabilidad:}\label{los-4-principios-buxe1sicos-de-la-buena-gobernabilidad}}

Decimos que una institución es una estructura de orden social que hace
que la sociedad funcione por eso es que siempre nos vamos a preguntar y
vamos a preguntar a nuestros gobernantes que tipo de sociedad queremos,
cual es la imagen de sociedad que queremos, por eso que ingresa un
conjunto de variables en esa imagen de sociedad que queremos, pero lo
que hay es lo que podemos observar, lo que está ahí, la estructura en la
que las personas botan las basuras, las calles malogradas, etc. y esa es
la estructura social, por lo que tenemos que hacer un esfuerzo de cómo
se debe de entablar la normativa existente para que en esta sociedad las
normas de conducta se cumplan y las costumbres sean positivas, y esa son
normas que queremos imponer cuando en realidad deben de ir a ritmo de
los cambios que se generan en la normativa y la norma de conducta de la
sociedad no está bien implementada, divulgada por lo que hacemos lo que
queremos.

Y con esto concluimos con gobernabilidad e instituciones, está por
encima de todo la legitimidad del gobierno en los 3 niveles, la
legitimidad de los 3 poderes y para ganar esa percepción debemos de
mejorar la confianza, la importancia del ciudadano de su participación,
pero tampoco permitir que haga todo la ciudadanía, entonces siempre hay
que pensar que la población puede pedir todo pero que no siempre tienen
la razón y para esto están los técnicos y especialistas porque si bien
es cierto hay demandas insatisfechas pues hay que priorizar ya que de
ahí sale la calidad del gasto.

\hypertarget{la-prestaciuxf3n-de-servicios-puxfablicos}{%
\section{La prestación de servicios
públicos}\label{la-prestaciuxf3n-de-servicios-puxfablicos}}

Cuando hablamos de prestación de servicios públicos nos estamos
refiriendo al orden social pre establecido que pueden ser de servicios
públicos o privados, la sociedad se prepara para recibir los servicios
que brinda el sector público y el sector privado, generalmente los
servicios que se reciben del sector público son gratuitos o de precios
bastante bajos, sin embargo los servicios del sector privado si
responden a fines de lucro, los servicios privados van a generar
beneficios a los duelos o promotores de estos servicios, en tanto en el
sector publico precisamente porque están financiados con presupuesto
públicos, sus precios son relativamente bajos, pero ya se han
introducido de que debe de pagarse un costo, un precio relativamente
bajo subsidiado pero ya se debe de asumir.

Por eso se ha creado el texto único de procedimiento administrativos en
la cual se costea los servicios que brindan las instituciones, ej.: si
queremos sacar una constancia (o el pronunciamiento para titular
terrenos rurales vamos a solicitar una constancia negativa de zona no
catastrófica a la dirección regional de agricultura).

Entendemos que existen servicios que brindan tanto del sector privado y
publico Cuando hablamos de servicios para mejorar las condiciones de la
pobreza también podemos hablar de la inversión en infraestructura que se
efectúa en el Perú, los servicios de infraestructura a nivel del país
son muy escasos y volátiles en el entendido de que estas puedan volver a
ser como antes si es que no tienen un mantenimiento adecuado, es
insuficiente el servicio que brinda el sector público para ofrecer la
infraestructura necesaria para lograr un crecimiento dinámica del país
que se ha sostenido de la economía y hay que facilitar la reducción de
la pobreza, ósea la inversión en infraestructura es escasa e
insuficiente para lograr un crecimiento económico alto y que sea en el
mediano y largo plazo y que esta puede afectar positivamente a la
reducción de la pobreza.

Podemos decir que la inversión en el caso del Perú es muy baja por ej.
En obras de transportes e hídricas (trenes, estaciones, rieles,
aeropuertos, la infraestructura de riegos), esto se traduce en que los
costos económicos para el sector privado en realidad son muy altos y que
hay una inadecuada provisión de servicios públicos particular para los
pobres, no llega el servicio público, los pobres no son beneficiarios
directos de estos proyectos de inversión del sector privado, los
servicios públicos tienen que generalizarse y hacer un esfuerzo en
incorporar al mayor número posible de usuarios.

Para crear una infraestructura adecuada el Perú debería de invertir
aprox. el 4\% del PBI anual, es muy alto, está por encima de lo que
actualmente se invierte, tendríamos que duplicar, la diferencia en
infraestructura rural en diferencia con otras economías es bastante alto
dada las restricciones presupuestales, la insuficiencia de recursos
presupuestales \ldots\ldots sería bueno continuar o propiciar la
participación de la asociación publico privada, cómo hacer que el sector
privado se interés por obras que son propias del sector público, sin
embargo el sector público puede renunciar sus derechos y asignárselo al
sector privado y esta podría ser a través de las asociaciones público
privadas (apps) debemos de ver como esa asociación funciona y el
gobierno se ahorraría la asignación presupuestal correspondiente, porque
el gobierno no debería de asignar presupuesto, solo evaluar el
presupuesto, sincerar el presupuesto y aprobar e invitar al sector
privado interesado y a cambio de la inversión que va efectuar el sector
privado darle la oportunidad para que maneje esas vías o canales hasta
que recupere su inversión y tengan un nivel de utilidad, pero las apps
han perdido mucha fuerza porque habido mal manejo y en vez de que el
sector privado pusiera el capital lo a apuesto el gobierno y sin embargo
se les ha asignado para que manejen estas obras, por lo que se ha
generado una desconfianza en las asociaciones público privadas (apps)
esta asociación que hasta ahora se ha practicado debería de mejorar el
diseño de las concesiones para ofrecer seguridad financiera y
contractual. Lo cual significa que ambas partes deberían de estar en la
capacidad de financiar y firma los contratos respectivos y pre
establecer acuerdos en la que la prioridad es ganar.

Debemos de hacer que las concesiones sean atractivas y que la población
vuelva a confiar, lo que se debe de hacer es que la población sea
consiente.

Debemos de asignar apropiadamente los recursos y reducir los riesgos en
la que ambas partes ganen tanto el sector privado como el gobierno.

Resolviendo las preocupaciones sociales de la población podría
consolidarse y desarrollarse mercados financieros locales, capital local
incorporado en la parte financiera (ej. Cooperativas).

La infraestructura productiva es bastante limitada en el caso del Perú,
es deteriorada e insuficiente, contribuye a la falta de competitividad
de la industria y del sector agrario, porque los costos del transporte
se elevan y esto hace que pierdan la oportunidad en el mercado por lo
que ese debe de elevar y fomentar la inversión en infraestructura básica
para mejorar la competitividad.

El crecimiento económico debería de ser a mayor velocidad, pero no es
posible en tanto la mejora de la competitiva se retrase por lo que la
preocupación del desarrollo regional debe de ser la reducción de la
pobreza en la región, se debe de incrementar para ello los niveles de
inversión en infraestructura.

La competitividad frente al agro frente a la industria de la zona
urbana, el área rural está en permanente desventaja.

Hay una necesidad de incrementar el nivel de inversión en
infraestructura.

Debemos de aumentar los beneficios de las asociaciones público privado,
debemos de hacerlo más atractivo debemos de darles mayores niveles de
rentabilidad, mayor tiempo para que administren el abastecimiento por
ejemplo de agua.

Debemos de ir a las poblaciones rurales a las zonas urbanas marginales y
explicar por qué se deben de hacer concesiones, debemos de superar ese
problema explicando el porqué del rechazo social, rechazo de la
población a las concesiones y hacer entender a la población que si bien
es cierto han tenido la razón en su momento, en la actualidad ya no está
de acuerdo que las necesidades institucionales y a las necesidades da la
población pobre por lo que se debe de recuperar la confianza en las
inversiones publico privados.

Debemos de resolver los problemas de conflicto de rechazo social a las
concesiones haciendo que estas sean efectivas y que realmente comprendan
que la empresa que está cobrando el peaje, esos fondos los van a
reinvertir en el mantenimiento y además tiene que ganar cierto nivel de
utilidades y que eso tampoco está prohibido ya que las empresas están
hechas para ganar, es más, estas instituciones publico privadas podría
explicar el rechazo social a las concesiones. Por lo que hay que
resolver esos problemas y hacer entender a la población de sus
necesidades y la necesidad ya es concesionarlo porque ellos no están en
la capacidad de hacer un mantenimiento de las pistas.

Otra razón por la que la infraestructura productiva es bastante limitada
es que los productores no se atreven en muchos casos a atraer a los
mercados locales a los inversionistas institucionales también locales.

Otra situación que debemos de considerar es el acceso insuficiente y con
calidad de los servicios del aparato estatal, del proceso de
centralización es bastante avanzada pero es incompleto por lo que los
servicios que brinda los gobiernos regionales son pésimos, los efectos
en la eficiencia y eficacia del gasto público en realidad deja mucho que
desear porque las regiones no destinan los fondos del presupuesto
público a proyectos debidamente priorizados sino a proyectos debidamente
politizados.

Por lo que debemos de cerrar estas indiferencias en la infraestructura

\begin{quote}
Tenemos sobrevaloración y sobredimensionamiento de obras
\end{quote}

TAREA: TITULO 3 DE LA CONSITUCION DE LA REPUBLICA DEL REGIMEN ECONOMICOS
DEL Articulo 58 Hasta el 89

RESUMEN:

Cuando hablamos de prestación de servicios públicos decíamos que las
inversiones que se han hecho infraestructura para brindar los servicios
del sector público o del aparato estatal era insuficiente y que debimos
concentrar el esfuerzo en obras especialmente de transporte, obras
hídricas y que esto se traducían en costos económicos para el sector
privado y que también la provisión de los servicios públicos era
inadecuada para los pobres y extremo pobres.

Para crear una infraestructura adecuada decíamos que en el Perú se debía
de invertir el 4\% del PBI de forma anual y eso para alcanzar aprox. en
8 a 10 años y cerrar las diferencias que tenemos entre el área urbana y
rural y las diferencias que tenemos con el exterior, para nivelarnos
internacionalmente con el grado de competitividad que ellos tienen,
porque esta infraestructura que no es adecuada hace que sea
incompetentes en las diversas actividades, el aparato productivo de
muchos productos que exportamos de por si es altamente competitivo en la
zona de producción, lo que pierde competitividad es entre la zona de
producción y la zona de entrega al exterior, por eso decimos que es
importante ya que estamos en una etapa difícil económicamente, el
presupuesto está en déficit permanente en estos 2 últimos años, es
importante la participación de las asociaciones público privadas, ya que
estas nos podrían ayudar para dar el mejoramiento de toda la
infraestructura vía concesiones, por lo que deberíamos de diseñar el
formato de las negociaciones con el sector privado para ofrecerles
seguridad financiera y contractual, por lo que debemos de garantizar los
contratos y este es justamente le candado que cualquiera de nosotros
quisiera.

Por lo que será necesario consignar con todos riesgos pero considerando
la seguridad financiera y contractual que el estado a través del
gobierno ofrece, de modo que incluso podemos desarrollar los mercados
financieros y locales, a esto de las concesiones podríamos para
garantizar la ejecución de los grande proyectos, pero principalmente que
las asociaciones público privados (las apps) que por nuestra restricción
presupuestal estamos diciéndole al sector privado que pongan partes del
financiamiento que podría ser del 60\%, 50\% en el capital, y a mayor
capital que ponga el sector privado habría que darle mayor concesiones,
mayores beneficios ya que estaría arriesgando más, en el caso del Perú
las cosas han ocurrido a la inversa, el Gobierno ha dado todo el
presupuesto en concesión, es por eso que las apps han perdido
legitimidad en el caso peruano por lo que hay que recuperar ese
prestigio perdido, hay que hacer que las concesiones realmente funcionen
Ej. Ayacucho-San Clemente, el tramo a la selva deberíamos de concesiona,
pero que se cobre un monto relativamente bajo porque este necesita
mantenimiento, cobrar peajes por tramos.

El déficit que tenemos lo vamos a financiar con endeudamiento público y
quien nos va financiar eso, teniendo en cuenta que tenemos que tener
proyectos no sobrevaluados ni dimensionados, una vez que hayamos hecho
los proyectos con precios reales vamos a buscar créditos, para lo cual
nos puede prestar una entidad externa, el Banco mundial, fondos o de
gobierno a gobierno en al que las empresas de los países europeos
ingresas a través de sus gobiernos y como nos van a prestar dinero para
cubrir nuestros déficit pero ellos van a realzar nuestro proyectos
(pistas) en la que ellos nos prestan y hacen la obra, que es una forma
de ingresar y mejora la calidad del gasto así como la ejecución, ya que
los créditos exteriores siempre viene condicionadas.

La cual es una alternativa muy viable y que es factible y para poner en
duda esto, los convenios contratos de gobierno a gobierno lo que tenemos
que hacer es:

\begin{itemize}
\tightlist
\item
  Proyectos sincerados
\item
  Los gobiernos tiene que tener negociaciones trasparentes (no
  sobrevalorados)
\end{itemize}

pero es interesante en tanto podamos hacer negociaciones de deuda
pública con determinado tipo de empresas u origen de endeudamiento o
capitales para financiar proyectos de inversión Ej: Club de parís (grupo
de bancos para financiar las obras del mundo).

Por lo que tenemos dos formar de captar fondos de manera adecuada y
financiar el presupuesto y mejorar las condiciones de servicios:

\begin{itemize}
\tightlist
\item
  Las concesiones
\item
  Convenio contratos de gobierno a gobierno
\end{itemize}

Ya que nuestra infraestructura está deteriorada ya que en los últimos
años no hemos hecho mantenimientos, lo que existía es insuficiente,
reduce la competitividad de la industria, hay que darle fomento a la
infraestructura, debemos de elevar el destino del presupuesto, ya que
sería interesante llegar hasta el 4\% del PBI para comenzar a cerrar las
diferencias, hay que darle mayores beneficios atractivos a las empresas
que inviertan en proyectos de infraestructura simplemente reduciendo la
corrupción podemos incentivar mucho más, por lo que debemos de cambiar
la mentalidad de las personas que es muy difícil y buscar explicar para
resolver los problemas de rechazo social a los proyectos de inversión
porque todas ellas son cuestionadas por lo que hay que reducir los
indicios de corrupción, hay que atraer las inversiones de las pensiones
locales y las AFPS, la ONP también puede convertirse como una AFP para
lo cual deberíamos de sacar una ley para que comience a invertir y
compita con las AFPS para elevar las tasas, hay que mejorar las empresas
aseguradoras para asegurar los riesgos de los sectores principalmente en
el sector agrario en vez de dar los seguros agrarios, sería bueno dar a
cambio infraestructura, hacer en las zonas más afectadas trabajos de
investigación o infraestructura que evite o reduzca los riesgos (Ej:
cercos para reducir los efectos de las heladas, instalaciones de agua
que podrían ser a través de mangueras) en todo caso cerras estas
diferencias en la infraestructura en el mediano y largo plazo, el gasto
público debe ser descentralizado, pero hay que mejorar la eficiencia y
eficacia del gasto público en las regiones y gobiernos locales.

\hypertarget{desarrollo-territorial}{%
\subsection{Desarrollo territorial}\label{desarrollo-territorial}}

El desarrollo territorial se debe de generar evitando la dispersión en
proyectos atomizados y fomentando la coinversión público-privada
aprovechando las leyes de la asociación Público-Privada y seguir
fortalecimiento los mecanismos de descentralizados de la toma de
decisiones.

La experiencia existente en el Perú debe de ser de formas de
participativa concertadas a nivel de los gobiernos locales y Regionales,
los proyectos para fomento el desarrollo deben de ser altamente
productivos y principalmente deben de estar en las áreas rurales.

Debemos de hacer un orden de prioridades de modo que podamos identificar
los proyectos más importantes en la que la población se sienta parte
integrante de las actividades que ejecuta el gobierno en sus diferentes
niveles.

Las políticas públicas son herramientas del Estado al servicio de la
sociedad.

El territorio de la república del Perú está integrado por regiones,
departamentos, provincias, distritos y centros poblados.

En estas circunscripciones geográficas se constituye y se organiza el
Estado y el Gobierno para un buen manejo se va dividir en niveles
Nacionales, Regionales y Locales.

Por un lado tenemos una división territorial basada en departamentos
provincias, distritos y centros poblados y considerando esta división
del país territorialmente el Gobierno se organiza y constituye a nivel
nacional, regional y local (3 niveles de gobierno) y es aquí donde
podemos hablar de niveles de gobierno.

\hypertarget{cuxf3mo-planteamos-el-desarrollo-considerando-e-territorio}{%
\subsection{¿Cómo planteamos el desarrollo considerando e
territorio?}\label{cuxf3mo-planteamos-el-desarrollo-considerando-e-territorio}}

El país territorialmente está dividido en 4 áreas: Departamentos o
Regiones, Provincias, Distritos y centros poblados y sobre estas el
gobierno se organiza, el que está en el gobierno central (nacional) es
el que manda a todos, a nivel de regiones hay una autoridad regional, en
las provincias existen las autoridades provinciales y a nivel local
están las autoridades locales de modo que se puede notar la presencia
del estado a través del Gobierno.

Políticamente el gobierno se divide en 3 poderes: Poder Ejecutivo
(Gobierno Nacional), Poder Legislativo y el Poder Judicial que imparte
justicia, estos poderes son independientes.

El Gobierno Nacional, es el que manda en todo el territorio nacional, su
mandato es transversal; el Gobierno Regional manda en su departamento
pero este no puede mandar en el Gobierno Nacional pero si puede
compartir responsabilidades y funciones (responsabilidades compartidas).

El Gobierno Nacional se representa a través de los diferentes
ministerios o los organismos públicos descentralizados (OPDES); los
ministerios que son los representantes del poder ejecutivo en el
Gobierno Nacional son los entes rectores, son las instituciones que van
a generar las normativas de aplicación a nivel Nacional y esa normativa
van a ser aplicadas a través de los gobiernos Regionales o Locales , y
cada uno de estos niveles de Gobierno conforme a sus competencias y
autonomías propias van a actuar para el bien común, para el bien de la
población, sin embargo se debe de respetar y preservar la unidad e
integridad del Estado como Nación (Se refiere principalmente a los
gobiernos regionales que se encuentran en las fronteras, los
gobernadores regionales no pueden regalar el territorio ni el presidente
de la república, nadie puede entregar parte de nuestro territorio).

Sin embargo en los últimos años se ha comenzado a notar que los
gobiernos Regionales y Locales hacen lo que creen por conveniente, ósea
que la normativa que genera el Gobierno Central no se cumple a cabalidad
y los Gobiernos Regionales principalmente se convierten en un obstáculo
para el cumplimiento de la normativa y En ese entender el Gobierno
Nacional está considerando reabsorber las funciones de los gobiernos
Regionales y Locales (Ej: sector Salud).

Debemos de defender el proceso de descentralización y mejorar a la hora
de elegir a nuestros representantes.

El gobierno Nacional tiene jurisdicción en todo el territorio nacional,
los gobiernos Regionales en su ámbito, los gobiernos locales dependen de
si es provincial en toda la provincia, si es distrital en todo el
distrito y si es en un centro poblado, en todo ese centro poblado; su
circunscripción territorial corresponde a cada nivel de gobierno según
el área geográfica.

Si hablamos de los gobiernos locales tenemos que decir que el nivel
provincial manda en los distritos, Ej: el alcalde de la municipalidad
provincial de Huamanga en teoría está sobre los alcaldes distritales y
en la práctica el alcalde no tiene mando ni jurisdicción sobre los
distritos por desconocimiento y falta de liderazgo.

La preferencia básica del gobierno en sus distintos niveles es el
interés público.

El desarrollo territorial se debe de generar considerando evitar la
dispersión de los proyectos, no se debe de atomizar los proyectos, Ej.:
en la región de Ayacucho no debemos de hacer pequeños proyectos por
todas partes, debemos de hacer proyectos reales que tengan un efecto
multiplicador que afecten a otros sectores, de modo que colateralmente
va desarrollar otras actividades, y si no hay dinero el financiamiento
seria fomentado por las asociación publico privado y fortalecer la toma
de decisiones, hay que fortalecer la descentralización, debemos de hacer
proyectos concertados.

La propuesta parte de la experiencia, en el Perú existen plataformas
participativas, concertación local ya que uno de los principios es poner
en el centro de atención a la ciudadanía, el pueblo nunca va tomar las
decisiones ya que se toman decisiones en nombre del pueblo que se tienen
que asumir en su momento.

Debemos de fomentar los proyectos que sean productivos principalmente en
el área rural y ahí viene el secreto del desarrollo rural vía
procedimiento de desarrollo territorial.

Las políticas públicas debemos de tomarlas en cuenta para afianzar las
decisiones que estamos tomando, para respaldar nuestra decisión porque
tenemos que ver el interés público.

La política pública busca el interés público y el bien común, las
políticas públicas con el bien común se ven bien interrelacionados con
el interés público por lo que se puede actuar con la normativa
existente, porque las herramientas del Estado están al servicio de la
sociedad, están al servicio de la población en su conjunto pero lo que
pasa es que la normativa existente están hechas solo pensando en el
interés privado porque nuestros gobernantes se han convertido en
gestores de las empresas privadas y no en gestores públicos.

Por lo que según las teorías económicas tenemos que crear las
condiciones favorables para que se desarrolle el sector privado, no dice
que hagamos las cosas para favorecer a la empresa privada sino que se
tiene que crear las condiciones (ej. Arreglar pistas, veredas,
carreteras, etc).

Cuándo estamos en el sector público y estamos en el área de
contrataciones en el momento de las calificaciones somos enemigo del
sector privado porque tenemos que ser meticulosos y ver que cumplan todo
los requisitos y lo que se va hacer, que alcancen los objetivos del
proyecto.

Cuando hablamos del enfoque territorial es necesario implementar
intervenciones diferenciadas según lo diferentes tipos de carencias,
patrones de desarrollo de cada territorio, se debe de estudiar cada zona
detenidamente y encontrar las carencias principales para que ellos con
esfuerzo propio puedan salir de la situación de pobreza y extrema
pobreza en la que se encuentra ya que las autoridades están justamente
para intervenir de acuerdo a esas carencias diferenciadas y que
seguramente 2 o 3 necesidades que podrían implementarse podrían servir
para reducir la pobreza existente en la zona.

Debemos de identificar el piso ecológico en el que se encuentran,
identificar qué tipo de cultivos se va a dar con ese rio, debemos de
trabajar en un cambio de los patrones de desarrollo de ese territorio,
debemos de ver el tipo de cultivos, ver el circuito de la distribución,
el acceso hacia el mercado, por lo que si resolvemos el problema del
agua, el tipo de cultivo, el acceso a esa zona para mejorar la
carretera, efectivamente podría salir de la pobreza esa población.

Por lo que cada zona de intervención con este enfoque territorial es un
caso totalmente distinto, no se pueden hacer intervenciones
generalizadas por eso es que no tiene efecto la intervención del Estado,
por lo que debemos de diferenciarlos, ver los patrones al interior e
intervenir recién.

Cada zona de intervención previamente debe de ser analizada e
investigada de modos que cuando se ponga el esfuerzo estatal esta tenga
un efecto real en la que el sector privado incluso pueda intervenir.

Cada centro poblado en su área tiene sus propias valoraciones, tiene
diversas prioridades en la que se tiene que ver lo que realmente
necesita la población, en la que tenemos que ingresar trabajando con la
población para que la población involucrada se incorpore e identifique
con la intervención y el trabajo que efectúa el sector público y
hacerles entender que la política del gobierno es ayudarlos pero no es
hacer todo por ellos, ya que la población tiene que asumir sus
responsabilidades.

El sentido de un enfoque territorial para el desarrollo es que va
permitir que un área geográfica mejore su condición económica, es tener
una mirada de coordinación y articulación del gobierno sea de cualquier
nivel, que puede ser del gobierno central o nacional, regional o local
con participación del sector privado y de los productores. El gobierno
tiene que hacer ese nivel de coordinación, tiene que tener esa mirada de
articular y juntar los esfuerzos a fin de que participe el sector
privado y los productores y a los productores el gobierno va apoyándolos
con subsidios de modo que sea transitorio y no permanente, el productor
tiene que tener beneficios tiene que tener utilidades, toda actividad
económica tiene fines de lucro, de obtener beneficios y utilidades.

El desarrollo territorial se debe de entender como un proceso de
construcción social de todo el entorno en el que se desenvuelve la
población perteneciente a una determinante área geográfica y que hay una
interacción entre las características geográficas del área porque el
ambiente va afectar, el piso ecológico en la que se desenvuelve, las
iniciativas individuales y experiencias tanto individuales como
colectivas deben de ser valoradas e incorporadas en esta política de
mejora del desarrollo territorial con enfoque territorial, cuando el
estado a través del gobierno llega para intervenir en un área geográfica
previamente elegido se debe aprovechar las experiencias de los
pobladores, experiencias individuales por lo que se debe de incorporar
en el trabajo esas iniciativas y experiencias individuales así como
colectivas y ver la operación de las fuerzas económicas, cuales son las
perspectivas de cada uno de ellos, que buscan para el futuro

Debemos de ver hasta qué punto cada uno de ellos están aptos a optar
tecnologías nuevas, que nivel de permeabilidad hay en esa poblacion para
captar el avance tecnológico que podría incorporarse y podría ser
transferido a la zona de trabajo y cuan fácil creen ellos que es
incorporar tecnologías nuevas y si bien es cierto captan rápidamente las
tecnologías pero estas no son aplicadas adecuadamente.

Debemos de analizar la situación sociopolítica, aquí entra el análisis
de conflictos, debemos de ver hasta qué punto la poblacion tiene
protestas coherentes o protestas permanentes o si es una poblacion
pacifica que absorbe la tecnología y acepta la intervención del Estado y
contribuye en mejorar su situación.

Debemos de analizar en el territorio en el cual vamos a intervenir como
está organizado la poblacion y que tipo de conflictos tiene, por lo que
primer debemos de resolver esos problemas antes de ingresar con la
intervenir del Estado.

Otro factor que es muy importante es la cultura, la cultura existente en
la zona hay que aprovecharla, canalizarlo para el bien (Ej.: aprovechar
los carnavales para reuniones más productivas).

Los factores ambientales, incorporar si son más vulnerables o no para la
quema de rastrojos, la poda de los arboles deben de seguir en el mismo
lugar y no trasladarlos o quemarlos.

En el territorio tenemos que ver qué acciones podemos iniciar con éxito
para contribuir y hacer que la población sea par, de eso trata el
desarrollo territorial y el enfoque territorial participativo en
realidad corresponde a un proceso planificado y aplicado a un territorio
que está constituido por una población x o socialmente constituido y que
es una zona de interrelación de diversas instituciones gubernamentales o
privadas que son las provincias, distrititos y centros poblados y en la
que está el sector público con el sector privado y también los factores
sociales, la interculturalidad que tenemos a nivel nacional. Por lo que
debemos de hacer consensos institucionales, debemos de crear consensos
de las comunidades que van a participar de repente hay similitudes y
podemos intervenir varias comunidades a la vez, varios espacios
geográficos y ver las zonas de mayor concentración de los centros
poblados, como operan la actividad económica, las tradiciones
culturales, la historia de esos pueblos y aquí hay factores sociales que
no siempre es puro ingreso.

El núcleo central de la propuesta del desarrollo territorial con enfoque
territorial es:

\begin{enumerate}
\def\labelenumi{\arabic{enumi}.}
\tightlist
\item
  El fortalecimiento de entidades público-privadas y para ello debemos
  de recuperar la confianza en la población y cuando sea necesario la
  constitución de acciones de índole provincial, departamental con
  coordinación previa del presupuesto público (intervención del Estado),
  y si va participar el sector privado darle las facilidades para su
  intervención, implementación e instalación.
\item
  La preparación de estas entidades de programas de inversión que deben
  de ser ordenadas en torno de ejes de prioridades de desarrollo
  previamente identificados. Ej.: El sector agrario directamente
  relacionado es el eje del riego y provisión del agua, recursos
  hídricos y las posibilidades de donde llevar, otro eje son las
  cosechas de agua, los reservorios que es de almacenamiento de épocas
  de lluvia y en épocas de estiaje se empieza a consumir esa agua dado
  que no hay riego permanente, otro eje son los caminos vecinales,
  debemos de ver los programas de inversión y que posibilidades de que
  el aparato estatal intervenga con éxito.
\item
  El cofinanciamiento de los programas a través de mecanismos
  financieros competitivos. Podemos dar créditos con altas tasas de
  interés, pero debemos de ver la rentabilidad del agro, no debemos de
  prestarle poco interés, debemos focalizar el desarrollo territorial y
  en ese enfoque vamos a fijar áreas de intervención y eso va ayudar a
  mejorar las condiciones económicas de la población de ese piso
  ecológico.
\end{enumerate}

Estas propuestas en realidad son propuestas que se inscriben en la
visión territorial del desarrollo rural principalmente, en las zonas
urbanas también se puede usar el enfoque territorial porque tenemos que
estudiar zonas puntuales, zonas en la que hay mucha concentración de
pobreza y es necesario profundizar el desarrollo descentralizado en el
país, la descentralización es un vehículo de aplicación de la estrategia
nacional del desarrollo rural ya que la descentralización es la vía
mediante el cual podemos llegar a la población más necesitada del área
rural, porque las regiones están más cerca de su población ya que el
gobierno nacional está alejado y a través de los gobiernos regionales y
locales está tratando de llegar directamente al beneficiario sin embargo
en la práctica podemos ver que los gobiernos regionales y locales no
están en la capacidad de implementar las políticas nacionales,
simplemente los actores políticos de este tipo de gobiernos desconocen y
no tiene la voluntad de implementar políticas con enfoque territorial de
modo que podrían ser exitosos.

Tenemos que hacer un desarrollo con enfoque territorial y para esto
debemos de visualizar la descentralización basada en cuencas
hidrográficas por donde están los ríos centrales de determinadas zonas
en el Perú y todo lo que abarca ese rio va ser un enfoque, una zona de
intervención (Ej. Cuenca del cachi).

El enfoque del desarrollo territorial está centrado en el desarrollo
productivo y los elementos básicos para el desarrollo productivo son
multisectoriales ya que va necesitar la intervención de varios sectores,
de varias direcciones regionales y varias unidades locales, debemos de
construir todo lo necesario para la construcción de alianzas
público-privadas, y si las necesidades son de mayor financiamiento
presupuestal debemos de ver la capacidad de endeudamiento de los
gobiernos regionales, la planificación estratégica es importante como
instrumento para determinar las diversas fuentes de financiamiento, ya
que si previamente se han hecho los estudios estaríamos garantizando el
éxito de la intervención en el territorio elegido por lo que tendríamos
una conglomeración de inversiones que generen expectativa.

La construcción de mecanismos de apoyo al desarrollo territorial a
partir de la realidad de las zonas a intervenir cuyos elementos
centrales del mecanismo son:

\begin{enumerate}
\def\labelenumi{\arabic{enumi}.}
\tightlist
\item
  Organización de los actores, a los de la población involucrada de cómo
  se organizan (por terreno, sexo) como se distribuyen el agua (por
  tamaño de terreno por tipo de cultivo)
\item
  La provisión de recursos, de cómo se proveen de agua, como es que
  llegan los fertilizantes e insecticidas, semillas y quienes son los
  proveedores. Debemos de ver los canales de provisión de recursos para
  reducir los costos de producción en la zona
\item
  El establecimiento de procedimientos, en cada segmento de producción
  la población conformado por pequeños agricultores de la zona deben de
  conocer cuáles son las serlas de juego, cuales son los procedimientos
  que se deben de cumplir para determinados tramites o para ser
  beneficiarios de algunas atenciones de parte del gobierno.
\end{enumerate}

\hypertarget{ley-de-bases-de-la-descentralizaciuxf3n}{%
\section{Ley de bases de la
descentralización}\label{ley-de-bases-de-la-descentralizaciuxf3n}}

Artículo 29.- Conformación de las regiones

29.1. La conformación y creación de regiones requiere que se integren o
fusionen dos o más circunscripciones departamentales colindantes, y que
la propuesta sea aprobada por las poblaciones involucradas mediante
referéndum.

29.2. El primer referéndum para dicho fin se realiza dentro del segundo
semestre del año 2004, y sucesivamente hasta quedar debidamente
conformadas todas las regiones del país. El Jurado Nacional de
Elecciones convoca la consulta popular, y la Oficina Nacional de
Procesos Electorales (ONPE) organiza y conduce el proceso
correspondiente. 29.3. Las provincias y distritos contiguos a una futura
región, podrán cambiar de circunscripción por única vez en el mismo
proceso de consulta a que se refiere el numeral precedente.

29.4. En ambos casos, el referéndum surte efecto cuando alcanza un
resultado favorable de cincuenta por ciento (50\%) más uno de electores
de la circunscripción consultada. La ONPE comunica los resultados
oficiales al Poder Ejecutivo a efecto que proponga las iniciativas
legislativas correspondientes al Congreso de la República.

29.5. Las regiones son creadas por ley en cada caso, y sus autoridades
son elegidas en la siguiente elección regional.

29.6. La capital de la República no integra ninguna región.

29.7. No procede un nuevo referéndum para la misma consulta, sino hasta
después de seis (6) años.

Este artículo da la posibilidad de que las regiones pueden crecer,
pueden juntarse las regiones o pueden anexarse a determinadas regiones
de modo que puedan integrar zonas muy cercanas, zonas de relación
directa o que pueden juntarse áreas geográficas contiguas pero
complementarias económicamente Ej.: Ica-Ayacucho que están en el mismo
corredor y que los obligaría a juntarse las dos carreteras libertadores
y la de puquio y a raíz de eso pueden integrarse más, en el 2006 hubo un
referéndum en la que también participo Ayacucho (Puquio quiso anexarse a
Ica), pero es casi imposible que dos regiones se junten porque los
gobernadores ganaron las elecciones y perderían el poder si esto llegara
a pasar, el egoísmo provinciano hace que no nos juntemos, pero que
atreves de referéndum podemos juntar 2 zonas contiguas pero
complementarias económicamente.

Artículo 18.- Planes de desarrollo

18.1. El Poder Ejecutivo elabora y aprueba los planes nacionales y
sectoriales de desarrollo, teniendo en cuenta la visión y orientaciones
nacionales y los planes de desarrollo de nivel regional y local, que
garanticen la estabilidad macroeconómica.

18.2. Los planes y presupuestos participativos son de carácter
territorial y expresan los aportes e intervenciones tanto del sector
público como privado, de las sociedades regionales y locales y de la
cooperación internacional.

18.3. La planificación y promoción del desarrollo debe propender y
optimizar las inversiones con iniciativa privada, la inversión pública
con participación de la comunidad y la competitividad a todo nivel.

En la práctica el plan Nacional de Desarrollo lo hace el centro de
planeamiento y en base a este plan de desarrollo Nacional cada sector a
nivel del poder Ejecutivo hace su propio plan de Desarrollo Nacional
sectorial y del plan de desarrollo Nacional los Gobiernos Regionales
hacen el plan de desarrollo Regional y de este plan de desarrollo
Regional se saca los planes de desarrollo sectorial (DIRESA,EDUCACION
AGRICULTA Y TRANSPORTE) y adicionalmente a ese plan de desarrollo
Regional tiene que engancharse del plan de desarrollo sectorial
ministerial; este artículo se refiere a que el país se debe de manejar
en base a un plan de desarrollo de modo que no se salga del control
nacional y es por eso que debemos de mantener la estabilidad
macroeconómica porque este plan no se puede salir del marco presupuestal
por lo que tiene que estar directamente ligado con el presupuesto
Nacional y a esto la participación del sector privado, cuando se hace el
plan de desarrollo Nacional se supone que está involucrado el Gobierno,
el Estado a través del Gobierno, el sector privado y todo el potencial
que tiene el país, por lo que el artículo se está refiriendo a la
relación que debe de existir entre planes y presupuesto, como es que
desde el gobierno central de ese plan a los sectores a nivel de poder
ejecutivo, Gobiernos regionales, sectores y los Gobiernos locales se
debe de manejar a través de un plan de desarrollo, nada debe de ser
improvisado, tiene que haber un orden de prioridades, por lo que
ordenadamente debemos de generar las inversiones necesarias para el
desarrollo del país. El primer reto que debemos de romper es que los
gobernantes, los que toman las decisiones tienen que saber de la
importancia del plan de desarrollo. Debemos de relacionar planes con
presupuestos.

Artículo 21.- Fiscalización y control

21.1. Los gobiernos regionales y locales son fiscalizados por el Consejo
Regional y el Concejo Municipal respectivamente, conforme a sus
atribuciones propias.

21.2. Son fiscalizados también por los ciudadanos de su jurisdicción,
conforme a Ley.

21.3. Están sujetos al control y supervisión permanente de la
Contraloría General de la República en el marco del Sistema Nacional de
Control. El auditor interno o funcionario equivalente de los gobiernos
regionales y locales, para los fines de control concurrente y posterior,
dependen funcional y orgánicamente de la Contraloría General de la
República.

21.4. La Contraloría General de la República se organiza con una
estructura descentralizada para cumplir su función de control, y
establece criterios mínimos y comunes para la gestión y control de los
gobiernos regionales y locales, acorde a la realidad y tipologías de
cada una de dichas instancias.

Cuando hablamos de fiscalización, hablamos de las actividades económico
financieras que realizan el aparato estatal o el Estado a través del
Gobierno y que debe de estar basado principalmente en la legalidad,
eficiencia y en la economía de modo que se cumplan con los objetivos de
la institución, y cuando hablamos de control nos estamos refiriendo a
cuanto de esto se ha cumplido en su ejecución, por un lado se debe de
controlar cuan eficiente ha sido las decisiones que tomaron,
técnicamente hasta qué punto fue correcto la decisión de haber ingresado
a ese local considerándolo como almacén, hasta qué punto esa decisión
permitió cumplir con el objetivo ( esta es la parte de fiscalización),
en la parte de control vemos si legalmente se ha ceñido a las normas, si
no hay daños y perjuicios a la institución y en esa búsqueda de la
fiscalización y el control en el Perú hay un sistema de control que
representa según el lugar la oficina de control institucional y a nivel
nacional la contraloría general de la república. Y quienes fiscalizan,
quienes pueden decir que es ilegal, que es antieconómico, que no es
eficiente esa actividad o decisión es la población, la ciudadanía, la
población puede detectar y dentro de esta ciudadanía también están los
trabajadores, por lo que la contraloría general de la república tiene
mucha responsabilidad pero esta no es tan grande por lo que al azar
elige e intervine porque no tiene tanta gente para intervenir.

Artículo 23.- Consejo Nacional de Descentralización

23.1. Créase el Consejo Nacional de Descentralización (CND) como
organismo independiente y descentralizado, adscrito a la Presidencia del
Consejo de Ministros, y con calidad de Pliego Presupuestario, cuyo
titular es el Presidente de dicho Consejo.

23.2. El Consejo Nacional de Descentralización será presidido por un
representante del Presidente de la República y estará conformado por dos
(2) representantes de la Presidencia del Consejo de Ministros, dos (2)
representantes del Ministerio de Economía y Finanzas, dos (2)
representantes de los gobiernos regionales, un (1) representante de los
gobiernos locales provinciales y un (1) representante de los gobiernos
locales distritales. (1)(2)(3) (1) De conformidad con el Artículo 1 de
la Resolución Suprema N° 393-2002-PCM, publicado el 05- 09-2002, se
designa al señor Luis Alberto Thaís Díaz, como Presidente del Consejo
Nacional de Descentralización (CND) por el plazo señalado en el numeral
23.4) del presente artículo. (2) De conformidad con el Artículo 1 de la
Resolución Suprema N° 394-2002-PCM, publicado el 05- 09-2002, se
designan a los representantes de la Presidencia del Consejo de Ministros
y del Ministerio de Economía y Finanzas, por los plazos a que se refiere
el numeral.

El consejo nacional de descentralización funciono a los inicios del
proceso de descentralización hasta el 2007 y fue aquí que se desactivo,
este fue el jefe de todos los presidentes Regionales, ya que todos los
presidentes regionales tenían que coordinar con él para hacer cualquier
cosa, era el jefe supremo de los presidentes regionales en tanto se
implementara el proceso de regionalización, este era el intermediario de
los gobiernos regionales con todos los ministerios, y hoy en día es una
dirección más, es el que hace seguimientos de los indicadores de las
regiones, llama a los presidentes de las regiones para reuniones, para
que estén con el presidente y lo respalden, ya perdió su razón de ser,
por intermedio de una ley lo desactivaron en el 2007, el consejo
nacional de descentralización incluso se involucraba en la designación
de directores regionales, es un artículo que ya perdió vigencia, que ya
no funciona que se creó con buenas intenciones, ha perdido vigencia
porque fue un obstáculo que no permitía a los presidentes regionales
hablar directamente con los ministros o el presidente de la república,
era un obstáculo para el proceso de regionalización que en nada ayudo
que más bien le quito velocidad al proceso de descentralización.

\hypertarget{ley-orguxe1nica-de-gobiernos-regionales}{%
\section{Ley orgánica de gobiernos
regionales}\label{ley-orguxe1nica-de-gobiernos-regionales}}

\hypertarget{ruxe9gimen-normativo}{%
\subsection{Régimen normativo}\label{ruxe9gimen-normativo}}

Artículo 36.- Generalidades

Las normas y disposiciones del Gobierno Regional se adecuan al
ordenamiento jurídico nacional, no pueden invalidar ni dejar sin efecto
normas de otro Gobierno Regional ni de los otros niveles de gobierno.
Las normas y disposiciones de los gobiernos regionales se rigen por los
principios de exclusividad, territorialidad, legalidad y simplificación
administrativa. En los regímenes normativos las resoluciones y las
normas que emanan de los gobiernos regionales o consejos regionales de
cada gobierno regional solo es aplicable en ese territorio, no es
extensivo a otras regiones, no se aplican en otras regiones, pero cuando
es del gobierno Nacional es aplica para todo el país, esto también
funciona en los gobiernos locales por extensión ya que si saca una
ordenanza regional o una resolución de alcaldía el distrito de Carmen
alto no es aplicable para san juan y otros distritos, sino
exclusivamente en ese territorio, los gobiernos regionales van a sacar
normativas para su aplicación en el territorio Regional principalmente
considerando las funciones exclusivas que han sido delegadas a los
gobiernos regionales, no pueden sacar resoluciones que afecten a los
gobiernos locales porque estas ya serian funciones compartidas y si son
funciones compartidas lo que hay que hacer es que estas se pongan de
acuerdo y saquen una sola normativa complementaria, y sobre la legalidad
de la normativa que se genera en los gobiernos Regionales estas tiene
que tener concatenación con la normativa nacional, ya que nada que no
esté normado podría ser habilitada como normas, y cuando hablamos de
simplificación administrativa se refiere a que todos los gobiernos
regionales tienen la capacidad de generar normativas a fin de agilizar
los trámites, simplificación de procedimientos administrativos significa
que debemos de agilizar y reducir los pasos que se siguen para
determinados resultados.

Artículo 37.- Normas y disposiciones regionales

Los Gobiernos Regionales, a través de sus órganos de gobierno, dictan
las normas y disposiciones siguientes:

\begin{enumerate}
\def\labelenumi{\alph{enumi})}
\tightlist
\item
  El Consejo Regional: Ordenanzas Regionales y Acuerdos del Consejo
  Regional.
\item
  La Presidencia Regional: Decretos Regionales y Resoluciones
  Regionales. Los órganos internos y desconcentrados emiten Resoluciones
  conforme a sus funciones y nivel que señale el Reglamento respectivo.
\end{enumerate}

Aquí nos dice cuáles son los niveles de normas de mayor jerarquía en el
ámbito de los gobiernos regionales, Ej.: a nivel nacional la
constitución (madre de todas las normas), después viene las leyes (que
es de aplicación nacional) y quienes generan las leyes son el congreso,
excepcionalmente el congreso cuando esta apurado el gobierno central o
el presidente de la republica quiere leyes rápidas y que no hayan
discusiones excepcionalmente el congreso lo que hace es delegarle las
atribuciones del congreso para que saque las leyes que favorezcan al
gobierno central al presidente de la Republica por lo que con sus
ministros hace las leyes, firma el presidente y sale la ley, por lo que
esto se llama Decretos legislativos porque fueron delegados con rango de
ley, el presidente también puede sacar decretos supremos, resoluciones
supremas y hasta aquí llega el presidente de la república y esto es de
aplicación nacional.

Los gobiernos regionales solo pueden sacar ordenanzas Regionales y esta
la sacan los consejos regionales y por extensión podemos decir que en
las municipalidades el consejo municipal, los regidores pueden sacar una
resolución que se llama ordenanza municipal.

El consejo regional también puede delegar al presidente Regional,
entonces este puede sacar ordenanzas por decreto, los presidentes
regionales sacan decretos regionales, resoluciones regionales y también
pueden sacar las resoluciones ejecutivas regionales

Los órganos de control institucional pueden sacar resoluciones de
sanción o reglamentos o procedimientos que ayuden a la buena
administración regional, pero que tiene que estar ceñidos a los
reglamentos de nivel nacional o Regionales, debemos de buscar normas que
no tengan duplicidad.

Las ordenanzas regionales son de mayor peso que las normas que emana el
gobernador Regional, pero lo que pasa cuando se saca un decreto regional
este pasa por encima del Consejo regional en el entendido de que el
presidente regional a través del gerente general ha enviado una
propuesta de ordenanza regional al consejo regional y el consejo
regional demora en responder por desacuerdos y etc, y en tanto se
aprueba eso aprueban un decreto regional y si nuca sale esa ordenanza
regional esta otra es válida, pero podría ser dejada sin efecto por una
ordenanza regional, por eso es que los gobiernos regionales son sinónimo
de leyes que emana el congreso y es igual en los municipios.

Las leyes normalmente son trasversales, por lo que sería bueno hacer las
modificaciones necesarias o dejar sin efecto muchas leyes que hacen daño
al país.

Artículo 52.- Funciones en materia pesquera

\begin{enumerate}
\def\labelenumi{\alph{enumi})}
\tightlist
\item
  Formular, aprobar, ejecutar, evaluar, dirigir, controlar y administrar
  los planes y políticas en materia pesquera y producción acuícola de la
  región.
\item
  Administrar, supervisar y fiscalizar la gestión de actividades y
  servicios pesqueros bajo su jurisdicción.
\item
  Desarrollar acciones de vigilancia y control para garantizar el uso
  sostenible de los recursos bajo su jurisdicción.
\item
  Promover la provisión de recursos financieros privados a las empresas
  y organizaciones de la región, con énfasis en las medianas, PYMES y
  unidades productivas orientadas a la exportación.
\item
  Desarrollar e implementar sistemas de información y poner a
  disposición de la población información útil referida a la gestión del
  sector.
\item
  Promover, controlar y administrar el uso de los servicios de
  infraestructura de desembarque y procesamiento pesquero de su
  competencia, en armonía con las políticas y normas del sector, a
  excepción del control y vigilancia de las normas sanitarias
  sectoriales, en todas las etapas de las actividades pesqueras.
\item
  Verificar el cumplimiento y correcta aplicación de los dispositivos
  legales sobre control y fiscalización de insumos químicos con fines
  pesqueros y acuícolas, de acuerdo a la Ley de la materia. Dictar las
  medidas correctivas y sancionar de acuerdo con los dispositivos
  vigentes.
\item
  Promover la investigación e información acerca de los servicios
  tecnológicos para la preservación y protección del medio ambiente.
\end{enumerate}

Cuando hablamos de pesquería señalamos que no tenemos pesca marítima en
la Región de Ayacucho, sino acuicultura y lo que están haciendo los
biólogos pesqueros es ir a los ríos, hacen canales y ahí están
produciendo las truchas, otros en las lagunas y en esto de las funciones
en materia pesquera en realidad el gobierno central ha vuelto a
centralizar las funciones en el ministerio de la producción, pero ellos
están interesados en la gran pesca y no en la pequeña pesca artesanal,
el gobierno central es el encargado del manejo del mar en todo el ámbito
nacional y esto sí ha sido reabsorbido, sin embargo los gobiernos
regionales que no tienen límites marítimos lo ha dejado a la voluntad
del pesquero, se ha dejado a voluntad, pero estos deberían de cuidar sus
funciones y la limpieza de los ríos, que las minas no contaminen mucho
el agua y que esta sea apta para el consumo y para la crianza de truchas
principalmente, su función debe de ser más ambiental, vemos que la
Dirección Regional no tiene la importancia que debería de tener, porque
los mismas Gobiernos regionales no se han interesado, porque a los
políticos esto no les genera muchos votos, económicamente no les sirve
para la coima en este país de corruptos debido a que existe una
corrupción masiva, por lo que a través de estas funciones tenemos que
garantizar los recursos acuícolas, los recursos que se encuentran bajo
el agua a través de esta función. Ya que lo ideal sería que nuestras
ríos y lagunas en el caso de Ayacucho estén debidamente identificadas en
todo su potencial.

\hypertarget{lineamientos-prioritarios-de-la-poluxedtica-general-del-gobierno-al-2021}{%
\subsection{Lineamientos prioritarios de la política general del
gobierno al
2021}\label{lineamientos-prioritarios-de-la-poluxedtica-general-del-gobierno-al-2021}}

Estos lineamientos se hicieron pensando que al 2021 deberíamos de llegar
en óptimas condiciones como país superando muchas barreras y que como
habíamos alcanzado un crecimiento bastante alto y que el gobierno había
comenzado a responder a algunas expectativas de la población se ha
generado mucha esperanza en la población peruana pues se espera una gran
ventaja para los gobiernos y el optimismo de la población.

Cuando hablamos de crecimiento económico decimos que el PBI va crecer,
que la producción aumenta, los ingresos aumentan, el desempleo
disminuye, la tasa de crecimiento de los precios está totalmente
controlado y en muchos casos tiende a la baja, por lo que estaríamos en
una etapa de prosperidad que crece permanentemente y decimos que hay
optimismo de parte de los inversionistas y de los consumidores, esto
significa que para que haya crecimiento los inversionistas tiene que ser
optimistas en querer seguir invirtiendo, y lo consumidores están
satisfechos porque cada vez consumen más ya que tiene los ingresos
suficientes.

Por lo que se fijaron 6 objetivos nacionales para el plan bicentenario:

\begin{enumerate}
\def\labelenumi{\arabic{enumi}.}
\tightlist
\item
  El país tiene una población con los derechos fundamentales y dignidad
  de las personas.
\item
  Oportunidades y acceso a los servicios.
\item
  Estado y Gobernabilidad.
\item
  Economía, competitividad y empleo
\item
  El desarrollo regional e infraestructura
\item
  Los recursos naturales y ambiente
\end{enumerate}

En este contexto de los objetivos nacionales al bicentenario, tenemos
que entender que el sector público no financiero está constituido por el
holding de empresas del estado y el Gobierno General (palacio de
Gobierno).

El gobierno General está compuesto por: el gobierno Central, los
gobiernos Regionales y gobiernos Locales llamados también niveles de
gobierno. El gobierno Central está constituido por un conjunto de
entidades como son los ministerios, oficinas y organismos que son
dependencias o instrumentos de la autoridad central del país.

Si hablamos a niveles de cuentas Fiscales del Perú se incluyen
ministerios, instituciones públicas (aquí debemos de mencionar a los
organismos públicos descentralizados-OPEDES) y universidades.

Al 2021 hay varias consideraciones sobre las cuales se han construido
estos objetivos nacionales y se ha fijado logros, por ejemplo:

\begin{enumerate}
\def\labelenumi{\arabic{enumi}.}
\tightlist
\item
  El país debería de ser considerado un país de desarrollo intermedio en
  rápido crecimiento económico.
\item
  Un país plenamente integrado a través de los tratados de libre
  comercio, un país comprometido con organismos internacionales y
  multilaterales (OEA, Paramento andino, etc.)
\end{enumerate}

Estas son las consideraciones al 2021, son los supuestos básicos que
deberíamos de cumplir para llegar al 2021, debemos de ir creciendo a esa
misma velocidad, debemos de ir mejorando y ampliando los tratados de
libre comercio, participar en los organismos internacionales y apuntalar
los organismos cercanos como el parlamento andino.

Y los logros que esperamos alcanzar al 2021 deberían de ser:

\begin{enumerate}
\def\labelenumi{\arabic{enumi}.}
\tightlist
\item
  El Perú debe de tener una población de 33 millones sin pobreza
  extrema, sin desempleo, sin desnutrición, sin analfabetismo, sin
  mortalidad infantil (por lo que se puede observar que a la fecha no
  hemos avanzado los logros, nuestro crecimiento ha sido mínimo, no
  hemos alcanzado el crecimiento de 6\% o 7\%)
\item
  El ingreso per cápita en el 2021 debería de estar entre 8000 y 10000
  dólares (no se ha alcanzado)
\item
  Tener un PBI duplicado entre 2010 y 2021 (no hemos alcanzado el logro)
\item
  Volumen de exportaciones cuadruplicado
\item
  La tasa anual de crecimiento mínimo debe de ser del 6\%
\item
  Tasa de inversión anual cercana al 25\%
\item
  Mejoras en la tributación, se esperaba que cada años después del 2011
  se incrementara en 5 puntos en relación al PBI
\item
  Reducción de la pobreza a menos del 10\% de la población (en la
  actualidad se ha incrementado)
\end{enumerate}

\hypertarget{ejes-de-la-poluxedtica-general-de-gobierno}{%
\subsection{Ejes de la política general de
Gobierno}\label{ejes-de-la-poluxedtica-general-de-gobierno}}

los ejes de la política general de Gobierno son 5: 1. la integridad y
lucha contra la corrupción 2. es el fortalecimiento institucional para
la gobernabilidad. 3. el crecimiento económico equitativo competitivo y
sostenible. 4. Desarrollo social y bienestar de la población 5.
Descentralización efectiva para el desarrollo

\hypertarget{los-lineamientos-que-se-basan-un}{%
\subsection{Los lineamientos que se basan
un}\label{los-lineamientos-que-se-basan-un}}

\begin{enumerate}
\def\labelenumi{\arabic{enumi}.}
\tightlist
\item
  integridad y lucha contra la corrupción debemos una prioridad es la de
  combatir la corrupción y las actividades ilícitas en todas sus formas
  esto que las actividades ilícitas en todas sus formas en realidad.
\end{enumerate}

permisos necesarios entidades gubernamentales esto es que el Gobierno
central en su momento de transferir su talento todos sus conocimientos
muy bien lo otro es el

\begin{enumerate}
\def\labelenumi{\arabic{enumi}.}
\setcounter{enumi}{1}
\tightlist
\item
  fortalecimiento institucional de la gobernabilidad cuando hablamos del
  fortalecimiento institucional para la gobernabilidad se trata de
  construir consensos políticos y tener acuerdos consensos en las cuales
  coincidan con los consensos sociales de las organizaciones sociales o
  la participación de sus componentes para el desarrollo en democracia.
\end{enumerate}

Fortalecer las capacidades del Estado para atender efectivamente las
necesidades ciudadanas se cubrirá parte de las necesidades de las
ciudades considerando sus condiciones de su debilidad sus condiciones de
vulnerabilidad y diversidad cultural con las que cuenta los países como
el Perú

\begin{enumerate}
\def\labelenumi{\arabic{enumi}.}
\setcounter{enumi}{2}
\tightlist
\item
  crecimiento económico equitativo competitivo y sostenible se refiere a
  tratar de recuperar la estabilidad fiscal en las finanzas públicas o
  sea tener un mínimo necesario para financiar el presupuesto público se
  debe sumar los adicionales de actividades propias del país
\end{enumerate}

potenciar la inversión pública y privada descentralizada es decir que la
inversión del parte del Gobierno también tenga criterios de
descentralización y la propiedad privada por supuesto se debe seguir a
esos principios de la propiedad privada.

la finanza pública en cuanto a su equilibrio dependerá de la estabilidad
fiscal dependerá del estándar de captación vía impuestos de los ingresos
suficientemente financiados para atender las necesidades de la población

para contribuir al crecimiento económico equilibrado competitivo y
sostenible se debe fomentar la agricultura se debe acelerar el proceso
de reconstrucción con cambios puntualizando principalmente la
prevención. Provincias o en los distritos estas deben ser incorporadas
al circuito de distribución nacional y el mercado exterior de modo que
podemos asegurar el aprovechamiento podría ser sostenible de los
recursos naturales y del patrimonio cultural esto está íntimamente
relacionado al turismo y debería ser una actividad que deberíamos
aprovecharlo porque dinamiza toda la actividad.

\begin{enumerate}
\def\labelenumi{\arabic{enumi}.}
\setcounter{enumi}{3}
\item
  Desarrollo Social y bienestar de la población ejemplo es reducir la
  anemia infantil en los niños principalmente de 6 meses a 6 años hay
  que reducir la presencia de la anemia en los niños hay que brindar
  servicios de salud de calidad. y los establecimientos de salud tengan
  la capacidad necesaria para resolver los problemas al interior de cada
  establecimiento.
\item
  Descentralización efectiva para el desarrollo hay que promover desde
  los diferentes ámbitos territoriales del Perú o del país alianzas
  estratégicas entre regiones entre niveles de Gobierno pueden que
  pueden tener proyectos birregionales.
\end{enumerate}

\hypertarget{la-descentralizaciuxf3n}{%
\section{La descentralización}\label{la-descentralizaciuxf3n}}

había sido concebida para lograr objetivos en el desarrollo económico y
que no solo el desarrollo económico sino también la competitividad la
modernización la simplificación de sistemas administrativos y
simplificación de procedimientos administrativos ir a la asignación de
competencias de los servicios públicos en los niveles más cercanos a la
población que es la expresión o la razón de ser de los gobiernos
regionales los gobiernos locales.

\hypertarget{los-desafuxedos-de-la-descentralizaciuxf3n}{%
\subsection{Los desafíos de la
descentralización}\label{los-desafuxedos-de-la-descentralizaciuxf3n}}

dado estas condiciones que tenemos para mejorar el nivel de organización
y contribuir al desarrollo del país 1 de los grandes desafíos:

\begin{enumerate}
\def\labelenumi{\arabic{enumi}.}
\item
  Es reducir las diferencias que se presentan entre las regiones porque
  tenemos unas regiones más desarrolladas que otras unos que tienen más
  recursos naturales en explotación en la actualidad y otras que todavía
  no las tenemos no se han descubierto simplemente no tenemos no
  contamos con esos recursos.
\item
  podría significar ejemplo que muchas municipalidades o gobiernos
  regionales estemos cada vez menos favorecidos, hay regiones que por el
  tema de canon minero. Ejemplo: canon gasífero tienen ingentes recursos
  y pese a que somos regiones cercanas o municipios vecinos no tenemos
  esa disponibilidad de recursos entonces la diferencia entre la
  población de estos municipios o de estas regiones comienza a ampliarse
  en vez de reducirse.
\end{enumerate}

RETOS

\begin{enumerate}
\def\labelenumi{\arabic{enumi}.}
\setcounter{enumi}{2}
\tightlist
\item
  ahí el reto es reducir estas diferencias, pero Adicionalmente a ese es
  el reto a ese gran objetivo de reducir las brechas el reto es que las
  regiones que cuentan con estos recursos acepten que esos recursos no
  solo son de ellos sino del país lo que pasa es que ha sido
  implementado esto un primer momento precisamente pensando en que deben
  ser los primeros favorecidos, pero eso debió ser solo por un periodo
  de tiempo hubiéramos puesto 5 años 8 años y punto. después ya debe ser
  distribuciones equitativas cada región creen que son dueños de esos
  recursos solo en esa región por ellos e incluso que todo el canon se
  distribuya solamente entre las provincias distritos entre las
  municipalidades de esa región esa concepción por. Ejemplo: ya se ha
  impregnado en la mentalidad de los pobladores lamentablemente entonces
  tenemos que desatar o desvendar de la mentalidad de la población esa
  forma de pensar.
\end{enumerate}

ahora último con esta incertidumbre de las elecciones incluso han salido
algunas propuestas algunas opinó y mucha gente pensando que el sur debe
ser un país como que alguna vez por ahí alguien se le ocurría habían
propuesto están proponiendo un mapa considerando solamente a las a la
parte sur desde Ayacucho está concentrado la minería no puede ser,
estamos llegando a extremos llegamos de egoísmo de discriminación al
interior del país y lamentablemente el gobierno central ha visto
rebasado su autoridad y no hay carácter de toma de decisiones, se trata
de que hay cosas que el gobernante tiene que tomar decisiones y se
respetan esas decisiones. El reto o desafío es reducir las deficiencias
las brechas.

\begin{enumerate}
\def\labelenumi{\arabic{enumi}.}
\setcounter{enumi}{3}
\item
  deficiente calidad de la información que harán posible qué exista
  menos transparencia en la rendición de cuentas y cada vez menos
  participa la ciudadanía en las rendiciones porque estos mecanismos de
  rendición de cuentas no han traído resultados positivos porque
  simplemente ha llegado a ser una exposición del presidente del
  gobernador regional o del alcalde y después no hay ningún resultado
  porque participa pues gente allegado a los presidentes regionales
  gobernadores regionales o alcaldes y no hay ningún resultado ningún
  tipo de análisis o entrega de información.
\item
  Hay un reto de mejorar cierto la deficiente calidad de información que
  entregan los gobernadores o los alcaldes municipales los alcaldes
  provinciales o distritales, y una iniciativa. Ejemplo: en vez de hacer
  esas presentaciones a través de la Defensoría del Pueblo la población
  puede entregar preguntas y que respondan esas preguntas, pero
  previamente se debe publicar toda la información por, ejemplo número
  de proyectos presupuesto avance físico y financiero de los proyectos
  en ejecución presupuesto de cada 1, y las modalidades de contrato con
  las cuales están ejecutando.
\item
  Es la descentralización que pueda acabar amenazando la responsabilidad
  fiscal sea los gobiernos regionales en la actualidad en realidad solo
  están recibiendo presupuesto y no están ejecutando que alcanzamos
  niveles reducidos de ejecución de gastos y menos calidad del gasto,
  perdemos el objetivo de competitividad regional y local que en
  realidad ya entró en riesgo y estamos fracasando o no se cumplan estos
  objetivos de mejorar la competitividad tanto regional como local
  porque falta responsabilidad fiscal de los gobernadores y las
  municipalidades.
\item
  Otro reto es la atomización del gasto público y la falta de
  intergubernamental a nivel de las regiones hay que implementarlo con
  las intervenciones bajo la concepción de desarrollo territorial hay
  que identificar los proyectos con efectos con las externalidades más
  amplias hay que hacer proyectos conjuntos entre una o más regiones y
  que haya capacidades desprendimiento de las autoridades regionales
  municipales a nivel incluso de las municipalidades distritales y los
  centros poblados.
\item
  El reto también sería de tratar de incorporar empresas o actividades
  con economías de escala y nivelarnos con las otras.
\item
  Otro desafío sería una fragmentación del proceso político en el país y
  la toma de decisiones en materia fiscal en este momento estamos
  fragmentados en dos y lo peor estamos no hay mayoría somos mitad y
  mitad.
\item
  Otro desafío o reto que tenemos es la adopción de impuestos a nivel
  regional y local que podría quitar incentivos a las inversiones y
  alejar a las inversiones porque ha comenzado hace dos años y está
  paralizado que los gobiernos regionales y locales se comienza a
  incrementar los impuestos según actividades económicas.
\item
  La calidad de gasto y la prestación de servicios.
\end{enumerate}

Sistema: conjunto de estructuras que están interrelacionados entre sí.
sistema administrativos tipos de gestión

Estructura: es un conjunto de relaciones que ordenan la distribución y
composición de las partes.

\hypertarget{la-regionalizaciuxf3n}{%
\section{La regionalización}\label{la-regionalizaciuxf3n}}

EL RACIONALISMO: es una teoría que postula la razón humana como fuente
de conocimiento que se opone en el sentido filosófico a la experiencia.

RACIONALIZACION: se refiere a una combinación de factores productivos
disponibles que permite formular una nueva combinación considerando la
ley de rendimientos decrecientes.

También es un proceso de mejora constante. Ejemplo: En una
administración científica. Permanentemente tenemos que ver que está
mejorando.

Ejemplo: reubicar tus cosas del cuarto.

Tenemos que adecuar a los objetivos de la empresa y a sus condicionantes
a las empresas para lograr el máximo aprovechamiento de los recursos.

Ejemplo: La reorganización de la empresa ante una reducción de la
demanda.

La racionalización: es un proceso que se organiza sistemáticamente un
trabajo para obtener mayores rendimientos al menor costo.

RESUMEN

La organización sistemática del trabajo para obtener el mayor
rendimiento de productividad al mínimo costo.

\hypertarget{tipos-de-racionalizaciuxf3n}{%
\subsection{Tipos de
racionalización}\label{tipos-de-racionalizaciuxf3n}}

\begin{enumerate}
\def\labelenumi{\arabic{enumi}.}
\tightlist
\item
  Racionalización administrativa: se refiere a conjunto de métodos,
  teorías que se derivan de los conocimientos tecnológicos y científicos
  aplicados a la gestión de las organizaciones y que alcancen los
  óptimos niveles de eficiencia en el marco de la eficacia. La
  racionalización administrativa en la actualidad se llama modernización
  del estado Racionalización administrativa: establece orden
  simplicidad, oportunidad en la organización y operaciones que se
  realizan para el cual se utiliza técnicas y métodos para estudiar,
  diseñar, simplificar estructuras, funciones, redacción de manuales,
  organigramas, procedimientos y uso de recursos. \#\# QUE ES REGIÓN es
  la extensión del territorio suficientemente homogénea de
  diversificación relativa que permite compatibilizar las acciones de
  desarrollo considerando su vocación actual o potencial y la naturaleza
  propia y especifica de sus recursos naturales y humanos. La región es
  aquella unidad geográfica económica definida alrededor de un núcleo o
  eje urbano en la cual converge flujos comerciales humanos y de
  geometría vial siendo a su vez ese núcleo punto de partida de
  relaciones extra regionales.
\end{enumerate}

Región administrativa: es el ámbito territorial en el cual se establece
y opera una administración regional, dicho ámbito territorial está
enmarcado dentro de la división político y administrativo del país y
guarda relación con las regiones económicas que deben desarrollarse en
el territorio nacional.

Región económica: es una zona con problemas económicas y sociales
comunes inducida por condiciones naturales o de otra clase. Ejemplo: la
cuenca de un rio o una zona sin adecuado abastecimiento de agua para la
agricultura.

\hypertarget{regiuxf3n-homoguxe9nea}{%
\subsubsection{Región homogénea}\label{regiuxf3n-homoguxe9nea}}

Corresponde a una región continua en que cada una de las zonas o partes
presentan características lo más próximo posible a los demás es
esencialmente morfológicas y estéticas. Región de planificación: es el
espacio geográfico definido en términos de elaboración de planes tendrá
diferentes niveles según el objetivo del país.

\hypertarget{regiuxf3n-plan}{%
\subsubsection{Región-Plan}\label{regiuxf3n-plan}}

Es un espacio que refleja el objetivo de un plan determinado y que puede
ser una estrategia de desarrollo multinacional, nacional, interregional,
regional, subregional o local. Se podría decir que es el espacio en el
cual diversas partes proceden de una misma decisión. Ejemplo: las
filiales proceden de una matriz.

\hypertarget{regionalismo}{%
\subsection{Regionalismo}\label{regionalismo}}

Es el sentimiento de identificación socioeconómica de los asentamientos
humanos respectos a otros ámbitos geográficos con el cual se siente
vinculado por razones históricas, sociales o económicas.

También se define como una alternativa de inversión desde el punto de
vista económica y se debe ver como un proceso abierto plural no
uniformizarte.

\hypertarget{regionalizaciuxf3n}{%
\subsection{Regionalización}\label{regionalizaciuxf3n}}

Es el proceso de redistribución y reorganimiento espacial que busca la
articulación económica social y geopolítica, ecológica y administrativa
en ámbitos denominados regiones con la finalidad de alcanzar el
desarrollo auto sostenido de la región.

Las regiones se constituyen sobre la base de áreas contiguas histórica,
económica, administrativa y culturalmente y conforman unidades
geoeconómicas.

La regionalización es una inserción del país en el circuito capitalista
y ellos se encuentran en el plan de regionalización que como fundamento
tiene en la descentralización.

\hypertarget{criterios-de-identificaciuxf3n-de-regiones}{%
\subsection{Criterios de identificación de
regiones}\label{criterios-de-identificaciuxf3n-de-regiones}}

\begin{enumerate}
\def\labelenumi{\arabic{enumi}.}
\tightlist
\item
  Geometría vial: cuantas carreteras entras allí y cuantas conexiones.
  Ejemplo: fluidos comerciales se refiere a centros de acopio a la
  distribución (huamanga).
\end{enumerate}

flujos financieros: se refiere a los montos de dinero y capitales que
ingresan traslado de encajes ejemplo: vraem (dinero y capitales se
movían allí). el flujo principal está en huamanga

\begin{enumerate}
\def\labelenumi{\arabic{enumi}.}
\setcounter{enumi}{1}
\tightlist
\item
  Flujos de población: la migración donde se concentra más la población,
  mayor recepción de personas es huamanga. Por factor enológicos
\item
  centros administrativos:
\item
  zonas deprimidas: zonas de extrema pobreza y pobreza
\end{enumerate}

\hypertarget{regiuxf3n-poluxedtica}{%
\subsection{Región política}\label{regiuxf3n-poluxedtica}}

Es una zona geográfica designada como una unidad administrativa
gubernamental de una zona o de una dependencia territorial de una
combinación de una o más naciones con uno o más territorios
dependientes.

\hypertarget{regiuxf3n-polarizada}{%
\subsection{Región polarizada}\label{regiuxf3n-polarizada}}

Es el espacio que se encuentra bajo la influencia de un polo y que se
manifiesta esta influencia por el intercambio de flujos y por la
atracción ejercida entre el polo y el área.

No es un espacio homogéneo sino es heterogéneo cuando intervienen los
distintos aspectos dentro de él son complementarios intercambian entre
si y son dominantes con las regiones vecinas. Este concepto tiene base
en la funcionalidad y el dinamismo de una zona la interrelación y flujos
entre el polo y los centros circundantes o el área de influencia.

el espacio total de la región gira sobre la influencia que tiene una
determinada zona de que lo podemos llamar o un polo de desarrollo
alrededor de ello es que vamos a tener que este determinar considerando
la zona de influencia de ese polo de desarrollo la extensión de la
región.

\hypertarget{reorganizaciuxf3n}{%
\subsection{Reorganización}\label{reorganizaciuxf3n}}

Se entiende por la reorganización de una entidad privada o pública
cuando las acciones son de transformación sustancial qué significa
modificaciones en su finalidad y objetivos. Ejemplo: el Perú en los
últimos años solo se ha dado el proceso de reorganización a nivel de
Gobierno entre el período de 1993 y el año 97 pues hemos tenido 4 años
de reorganización como parte de un proceso más complejo que ha incluido
la reestructuración.

cuándo le reorganizamos lo que hacemos es primero en la práctica y
teoría que lo puedes encontrar definiciones grandes de 5 o 10 páginas
cambiar fines y objetivos de una institución esto pasa porque hay que
transformarlo hay que cambiar prácticamente la razón de ser de una
institución esto es que en la práctica lo que se hace un nuevo
reglamento de organización y funciones se hace una nueva estructura
organizacional ahí en la nueva estructura organizacional al hacer el
reglamento vas a cambiar los fines y objetivos institucionales.

Entonces en realidad es readecuar las instituciones públicas a las
exigencias actuales a las exigencias de la población dado que hay una
transformación permanente y la estructura social la gente estamos
cambiando. Ejemplo: mi forma de pensar es totalmente distinto a los que
han nacido después de 10 años de mí los que nacieron en los 80 es otra
forma lo que ustedes lógicamente que nacieron pues al acabar el 90 antes
del siglo 20 ustedes nacieron el 97, 98 también no tiene una forma
totalmente distinta de pensar.

\begin{itemize}
\tightlist
\item
  hay que ir adecuando a esas formas de actuar de la población a las
  necesidades que exige la población y los de generaciones anteriores
  tenemos que ir adecuándonos.
\item
  las instituciones igual tenemos que adecuar a las condiciones nuevas
  que exige el desarrollo del país.
\item
  La reorganización implica tomar en consideración como base un nuevo
  ordenamiento principalmente del potencial humano de la gente.
\end{itemize}

\hypertarget{economuxeda-regional}{%
\subsection{Economía regional}\label{economuxeda-regional}}

es el estudio del comportamiento económico del hombre en el espacio
(tierra) analiza por tanto los procesos económicos a nivel espacial y
trata de conocer y de plantearse algunas preguntas entorno a la
estructura del que hacer económico.

Se ocupa la economía regional:

\begin{itemize}
\tightlist
\item
  de descubrir las causas que determinan la distribución de las
  actividades económicas en el espacio.
\item
  de la delimitación de subsistemas económicos para analizar en cada 1
  su dinámica interna.
\item
  Del Estudio de las interrelaciones entre dos o más regiones. Ejemplo:
  intercambio de bienes y servicios, la transmisión de los siglos
  económicos.
\item
  De la construcción de sistemas de equilibrio óptimo interregional.
\item
  Estudio de la política regional acciones realizadas para conseguir una
  correcta distribución de recursos.
\end{itemize}

la economía regional va estudiar el comportamiento económico del hombre
en un determinado espacio geográfico y se encargar de analizar e
investigar de generar acciones necesarias y todos los procesos
económicos en su ámbito espacial en su ámbito geográfico.

diferentes variables de la economía regional en realidad cada región
deberíamos saber cuál es el nivel de influencia. Ejemplo: que tiene el
gasto público en el PBI en cuanto porciento incide presupuesto en
infraestructura.

\hypertarget{centralismo}{%
\subsection{Centralismo}\label{centralismo}}

centralismos es una tendencia a mantener las dependencias como ubicación
en los centros urbanos de carácter metropolitano las instituciones
normalmente que son centralizadas van a estar concentradas en los
centros urbanos más desarrollados va estar el poder centralizado es el
que va a determinar, Institucionalmente está representado por una sola
institución y esa es la sede central.

centralismo mantiene las dependencias con ubicación en los centros
urbanos de carácter metropolitano por decir Lima o Ayacucho o las
capitales de las regiones.

la centralización viene a ser la concentración de autoridad en un nivel
jerárquico particular lo cual reúne en una sola persona el cargo o
ámbito de poder tomar decisiones máximo alguien con el nivel más alto de
autoridad concentra todo el poder para tomar decisiones más importantes
están a cargo de esas personas que concentran poder.

Ejemplo: el presidente de la República el gobernador regional en la zona
del ámbito de su influencia.

A nivel de gobierno esta descentralizado los gobiernos regionales y
municipalidades le ha delegado: ejemplo el INC es una organización
centralizada desconcentrada porque hay casi en todo departamento.

\hypertarget{desarrollo}{%
\subsection{Desarrollo}\label{desarrollo}}

el desarrollo es un proceso que lleva a una modificación sustantiva de
las estructuras económicas políticas e ideológicas de una sociedad,
incluye la perspectiva internacional si el país crece habrá cambios
sustanciales en las organizaciones económicas a nivel de las empresas
públicas y privadas.

El desarrollo se concebí como un cambio radical del ordenamiento
tradicional en lo económico, social, político y cultural de la sociedad.

\hypertarget{desarrollo-regional}{%
\subsubsection{Desarrollo regional}\label{desarrollo-regional}}

Es el incremento del bienestar e un territorio en particular diferente
al integrado por las jurisdicciones político administrativo de un país,
cuando el desarrollo nacional le enfoca como el desarrollo de conjunto
de regiones que conforman el país

\hypertarget{objetivo-principal}{%
\subsubsection{Objetivo principal}\label{objetivo-principal}}

\begin{itemize}
\tightlist
\item
  Distribución del desequilibrio entre las poblaciones de las diversas
  regiones.
\item
  La equidad entre sus pobladores, tener metas y objetivos claros.
\end{itemize}

\hypertarget{la-descentralizaciuxf3n-1}{%
\subsection{La descentralización}\label{la-descentralizaciuxf3n-1}}

la descentralización es un principio organizativo según el cual a partir
de una entidad central se generan entidades con personería jurídica
sujetos a la política general de una entidad central que por razones
naturaleza diferencial de las funciones y actividades que deben cumplir
se les otorga autonomía operativa suficiente para asegurar el mejor
cumplimiento ellas.

la descentralización en concepto en principio son delegación de
funciones y atribuciones qué hace el Gobierno central a las regiones a
las instituciones que creen por conveniente porque de ese modo se puede
llegar a la población más alejada que no tiene beneficios del Gobierno y
estas jurídicamente van a tener autonomía o personería propia no van a
depender de otras entidades.

\hypertarget{descentralizaciuxf3n-administrativa}{%
\subsubsection{Descentralización
administrativa}\label{descentralizaciuxf3n-administrativa}}

la descentralización administrativa es la delegación de autoridad y
deberes a oficinas regionales o defunciones especificas a fin de que
ellos tengan suficiente dependencia para la toma de decisiones sin que
en cada caso deban consultar con la oficina o el poder central. ejemplo
la dirección regional agraria toma sus decisiones directamente,
presidente regional no tiene que pedirle permiso en absoluto al ministro
de Agricultura para hacer acciones propias a nivel de la región.

\hypertarget{la-desconcentraciuxf3n}{%
\subsection{La desconcentración}\label{la-desconcentraciuxf3n}}

la desconcentración es un principio también organizativo según la cual
se genera una delegación de funciones atribuciones y decisiones desde un
nivel de autoridad superior hacia niveles de menor jerarquía funcional o
territorial sin embargo la misma persona jurídica o la autoridad que
delega sigue siendo responsable.

la desconcentración a nivel administración de gestión pública o privada
de principio es un principio organizativo lo cual genera de una
delegación de funciones atribuciones y decisiones desde un nivel de
autoridad superior a otros de menor jerarquía funcional o territorial
pero dentro de la misma autoridad o dentro del ámbito de la misma
persona la autoridad que delega sigue siendo responsable la autoridad
que delega sigue siendo responsable en consecuencia puedes revocar la
delegación o revisar las decisiones.

\hypertarget{desconcentraciuxf3n-administrativa}{%
\subsubsection{Desconcentración
administrativa}\label{desconcentraciuxf3n-administrativa}}

una desconcentración administrativa para quedar mejor la
desconcentración es un proceso técnico proceso técnico administrativo
mediante el cual las autoridades proceden las a delegar funciones desde
un nivel de autoridad funcionales o territoriales de mejor jerarquía.

\hypertarget{acciuxf3n-administrativa}{%
\subsubsection{Acción administrativa}\label{acciuxf3n-administrativa}}

la acción administrativa es la labor desarrollada en 1 o varios puestos
de trabajo que puede ser de naturaleza física o intelectual también se
le conoce como la decisión que adopta una autoridad al momento de
resolver asuntos de índole político, técnico o administrativo.

Es el responsable de las acciones administrativas no solo se refiere al
administrador eso al responsable es la máxima autoridad de mayor
jerarquía responsable de las acciones administrativas que van efectuando
acciones administrativas.

el Gobierno actúa en representación del Estado.

\hypertarget{el-estado}{%
\subsection{El estado}\label{el-estado}}

el estado es una construcción social es la institucionalización jurídica
y política de la sociedad tiene como elementos constitutivos el poder
político, el territorio y el pueblo (población).

el estado está constituido por el poder político que son los 3 niveles:

poder político sea los elegidos políticos representados por los
políticos el estado está constituido por los políticos Gobierno central
Poder Ejecutivo el poder legislativo emanan la elección de la población.

\hypertarget{elemento-del-estado}{%
\subsubsection{Elemento del estado}\label{elemento-del-estado}}

el territorio el área geográfica el ámbito geográfico de ese país el
Perú todavía no es una nación que somos un conjunto de etnias enteras o
poblaciones de diferentes orígenes como los pueblos amazónicos andinos
afro peruanos afro asiáticos.

El estado tiene por objeto dirigir controlar y administrar las
instituciones y regular una sociedad política y ejercer autoridad que la
población puede manifestarse, pero la autoridad está para poder poner el
orden institucional controlar administrar lo que hace la población.

resumen

El estado es una construcción social y es la institucionalidad política
jurídica de la sociedad y sus elementos consecutivos son los poderes de
poder político, el territorio y el estado.

En el caso peruano según la constitución es uno, y indivisible, su
gobierno es unitario (por el territorio es solo), representativo
(elegido por las elecciones) y descentralizado (por naturaleza está
constituido por regiones).

En las áreas de suscripciones de la índole nacional se organice en
gobierno a nivel central, regional y local.

Según la constitución y la ley se preserva la unidad e integridad del
territorio de la nación (varias naciones, samuces, ashánincas) todo ello
está integrado al estado.

El objeto del estado es dirigir, controlar y administrar instituciones
del estado, así como regular una sociedad política y ejercer autoridad
(no puede perder autoridad el estado porque si no comienza una crisis
social generalizada) por eso el estado debe poner los juegos en reglas,
los reglamentos, las leyes para su control.

Para que un gobierno pueda subsistir, existir deben de desarrollarse
algunos poderes: el Poder ejecutivo, por ejemplo, (debe coordinar con
los otros poderes para aprobar las leyes como es el poder legislativo y
además para que estas leyes se ejecuten tiene que ingresar el poder
judicial).

\hypertarget{organizaciuxf3n-del-pauxeds}{%
\section{Organización del país}\label{organizaciuxf3n-del-pauxeds}}

Cuando hablamos de organización del país tenemos que ver como se
visualiza el país tenemos que construir una visión país, tenemos que ver
el largo plazo.

Hay que observar que un país no se puede gobernar pensando solo en cinco
años, sino pensar en 20, 40, 100 años de modo que haya continuidad en su
implementación porque la visión país nos hace ver ``qué es lo que
buscamos, que es lo que esperamos'' para nuestra población utilizando
diversos recursos con las que contamos y tenemos que adecuarnos a las
necesidades de mundo como país y hacer lo necesario de insertarnos en
las mejores condiciones.

De modo que cuando construyamos la visión país (que el Perú tiene una
visión, el ceplan pero nosotros sabemos y podemos hacer una visión país,
regional, sectorial, actividad, proyecto).

Cuando queremos construir una visión debemos considerar algunas
características del proceso de construcción de estabilización, cuando
vas a construir una visión tienes que darle un enfoque
multidisciplinario tiene que ser holístico, partir de la interacción
temática entre distintas disciplinas involucradas en la elaboración del
documento.

La característica del proceso de construcción de una visión tiene que
ser un enfoque multidisciplinario tienes que explorar diversos
escenarios tienes que hacer que participen en lo posible todos los
actores sociales o sus representantes y al final tienes que pulirlo y
darle los lineamientos y el proceso metodológico de modo que una
construcción bien hecha.

Ya has hecho la visión Perú o de la institución tenemos que pensar en un
país en el aspecto publico vas hacer gestión pública.

Cuando hablamos de gestión nos referimos a la capacidad de articular
procesos, agentes y recursos con las que cuentas para alcanzar o
perseguir los objetivos institucionales. La gestión tiene la capacidad
de generar una relación adecuada entre lo que es la estructura la
estrategia los sistemas las capacidades los objetivos supremos de la
organización.

Podemos decir que gestión es la capacidad de juntar de articular y hacer
que coordinadamente funcione los diversos procesos, agentes, y recursos
con los que cuentas para alcanzar los objetivos.

En la administración la gestión consiste la administración de una
institución entonces normalmente el administrador de las instituciones
son los que están involucrados en la gestión.

La gestión es un procedimiento que comprende una tramitación importante
para lograr algo o solucionar un problema esto es de tipo administrativo
o requiere de documento. También podemos decir gestión es la agrupación
de la actividades u operaciones vinculadas con la administración ósea
como se utilizan los diversos recursos. Gestión se utiliza conocer
proyector en forma general cualquier acción necesaria de procedimientos
de proyección ejemplo: para ver cómo funciona una institución o
proyectar que va hacer una institución implantar una nueva institución o
generar una dirección también es gestión.

El propósito principal de la gestión es lograr el incremento de buenos
resultados de una empresa o del sector publico ósea busca cada vez
mejorar, obtener buenos resultados.

Tenemos 4 factores que participan en la búsqueda de buenos resultados de
una gestión:

\begin{enumerate}
\def\labelenumi{\arabic{enumi}.}
\tightlist
\item
  Estrategia
\item
  Estructura de la institución
\item
  Cultura de la organización, social, ancestral
\item
  La ejecución de la mesa
\end{enumerate}

La gestión pública es la aplicación de todos los procesos e instrumentos
que posee la administración pública para el logro de objetivos de
desarrollo o bienestar de la población.

La administración pública con el ejercicio de la función administrativa
del gobierno, puesta en marcha.

Los fines, objetivos y metas debe estar apoyadas a través de la gestión
políticas, recursos, programas.

resumen

Gestión es la fusión la articulación de las capacidades que tenemos para
articular los procesos, agentes, recursos para alcanzar los objetivos
institucionales.

En la gestión pública tienes que utilizar la gestión política, recursos,
programas para alcanzar el logro de los fines de objetivos y metas
institucionales a través del conjunto de procesos y acciones necesarias
que vas a poner en marcha.

\hypertarget{componentes-de-la-gestiuxf3n-puxfablica-como-estuxe1-dividido}{%
\subsection{Componentes de la gestión pública como está
dividido}\label{componentes-de-la-gestiuxf3n-puxfablica-como-estuxe1-dividido}}

La gestión pública por un lado tiene al gobierno y por otro lado a la
administración pública.

El gobierno es el conjunto de personas, tienen la capacidad de regir el
destino de un país en este caso el presidente de la república, los
ministros los gobernadores regionales los alcaldes y los jefes de los
proyectos nacionales.

La administración pública es el conjunto de personas que utilizan todos
los materiales de maquinaria y equipos, muebles e inmuebles con los que
cuenta esa institución implementa procesos e instrumentos que se aplican
para ejercer el gobierno.

La gestión pública es el conjunto de herramientas de los instrumentos y
procesos puestos en ejecución.

La gestión son guías para orientar al interior de la administración
pública al interior son las guías para orientar la acción como se actúa
la previsión la visualización o uso de empleo de los recursos y el
esfuerzo de los trabajadores para alcanzar los fines que se desea.

Es el correcto manejo de los recursos de los que dispone una determinada
organización pueden ser los organismos públicos las ONG de modo que
siempre se enfoca en el mejor uso o si se quiere uso eficiente de todos
los recursos considerando que se debe maximizar los rendimientos.

La gestión es un conjunto de trámites que se llevan a cabo para resolver
algunos asuntos concretos o particular u concretar un proyecto en si.

También se dice que es gestión a la dirección, administración de una
empresa de una institución de una compañía de negocios.

Ejemplo: el administrador de excelencia de gestores.

\hypertarget{tipos-de-gestiuxf3n}{%
\subsection{Tipos de gestión}\label{tipos-de-gestiuxf3n}}

\begin{enumerate}
\def\labelenumi{\arabic{enumi}.}
\tightlist
\item
  Tecnológica
\item
  Social
\item
  De proyectos
\item
  De conocimientos
\item
  De ambiente
\item
  Estrategia
\item
  Administrativa
\item
  Gerencial
\item
  Financiera, gerencia publica y gestión de riesgo
\end{enumerate}

En el sector privado tenemos gestión empresarial, gestión educativa y
gestión humana de calidad comercial y cultural.

\hypertarget{gobierno}{%
\subsection{Gobierno}\label{gobierno}}

El gobierno viene a ser la autoridad que controla y administra las
instituciones del estado, ejerce el poder del estado con un ordenamiento
jurídico y está al servicio del estado es fraccional no es duradero.

El gobierno actúa en representación del estado, es la autoridad que
entra en acción controla y administra lo que representa al estado las
instituciones y las instituciones es representación mental.

El estado está institucionalizado políticamente y jurídicamente el
estado está constituido en el área geográfica está representado por el
poder político y por toda la población. cuando hablamos de Gobierno el
Gobierno actúa en su representación porque está a un giro de autoridad
para representar la construcción social del cual hablábamos del estado.

el Gobierno ejerce el poder de determinar el poder de la toma de
decisiones basado en un ordenamiento jurídico y el ordenamiento jurídico
viene desde el estado, el Gobierno siempre por definición debe estar al
servicio del Estado.

El Gobierno está al servicio de la construcción social que representa a
la población el Gobierno es transitorio es fraccional.

El Gobierno es fraccional mientras sea democrático mucho más fraccional
lo que hay que hacer es garantizar la continuidad de las políticas de
Estado por eso no se habla de políticas e gobierno.

Si es política de Gobierno a lo más 5 años es política de Estado será en
el tiempo por eso se habla políticas públicas las políticas públicas no
son para 2 años 3 años si no es para el largo plazo hasta alcanzar los
logros esperados.

\hypertarget{cuuxe1l-es-el-rol-del-estado}{%
\subsubsection{¿Cuál es el rol del
estado?}\label{cuuxe1l-es-el-rol-del-estado}}

esa construcción social esa representación institucionalizada
jurídicamente por ejemplo el rol del estado es garantizar la seguridad
interna (encabeza la policía) y externa (fuerzas aéreo, la marina y el
ejército nacional).

garantizar la seguridad interna y externa y asegurar que se imparta
justicia bueno eso será pues la decisión política y el Poder Judicial
son los que ponen en marcha y acción las normas que deben cumplirse
correctamente tanto en el sector público como en el sector privada. el
rol del Estado es garantizar o velar por la propiedad es un rol
importantísimo que se debe respetar la propiedad privada.

La propiedad privada es inviolable, pero velar por la propiedad y puede
ser propiedad social y propiedad privada.

La prosperidad del sector privado y buscaremos la prosperidad de la
propiedad colectiva de las famosas cooperativas las asociaciones que el
mismo estado de implementarlo y conducir. estamos en una economía como
el Perú es regular los mercados por eso la necesidad de los entes
reguladores o a través del mercado mismo con participación del Gobierno
se puede regular.

Ejemplo: debería haber grifos reguladores de Petroperú a nivel nacional,
pero Petroperú como que sea retirado y más bien lo han vendido los
grifos de propiedad de Petroperú a las empresas privadas debieron haber
reservado entonces hubiéramos tenido.

hay que regular los mercados, debe haber participación, pero
lamentablemente como las leyes se han hecho a favor de los grandes
empresarios ante la Constitución entonces los entes reguladores no
funcionan, pero el rol del Estado es regular los mercados se debe
promover la igualdad de oportunidades.

¿es posible garantizar la igualdad de oportunidades sí o no o que sea
derecho de oportunidad de igualdad? no podemos garantizar para la
igualdad de oportunidades cierto porque tenemos diversas cualidades
diversos niveles de conocimiento diversas formas de pensar diversas
expectativas.

Ejemplo: no todos tienen pensado estudiar economía y el estado
garantizaría la igualdad de oportunidades no garantizaría, puede
promover y crear las condiciones de fortalecer a la gente.

¿sería un derecho a la igualdad de oportunidades debería ser un derecho
en todo caso, no podemos decir que sea un derecho para que en alguna
medida todos tengamos la posibilidad de promover la promoción de la
igualdad de oportunidades.

\hypertarget{el-estado-quienes-lo-constituyen}{%
\subsubsection{¿El estado quienes lo
constituyen?}\label{el-estado-quienes-lo-constituyen}}

\begin{itemize}
\tightlist
\item
  el pueblo o la población.
\item
  el poder político le pone su área geográfica o sea el país un
  territorio. y quienes viven pues los hombres.
\item
  todo está en función a los hombres por eso que a veces algunos dicen,
  pero eso hacemos los hombres se supone que el mundo prácticamente está
  a la disposición del hombre somos los racionales los que disponemos
  los que hacemos que el mundo esté a nuestro servicio y todo lo que hay
  en este mundo en la tierra está a nuestro servicio.
\item
  poder político tenemos el territorio una fracción de la tierra y la
  población que vive en esa atracción, cuando hablamos de proteger el
  medio ambiente todos esté de acuerdo en que debemos cuidar las cosas
  más de lo necesario a la tierra.
\item
  pero eso no significa que solo seamos los dañinos por ejemplo los
  animales la vaca es uno de los animales más contaminadores del medio
  ambiente no hay nadie se ha ido contra la vaca hay que seguir criando
  la vaquita. para la carne etc.
\end{itemize}

El rol del Estado es promover la infraestructura física eso significa
que debemos hacer proyectos infraestructura pues debemos mejorar o vemos
la mejora permanente de las condiciones de vida del hombre o facilitar
los procesos de producción en las diversas actividades en
infraestructura vial, carreteras estamos haciendo más fácil más rápido
el traslado de los recursos que genera el área rural digamos hacia el
área urbana.

Otras funciones o que sean necesarios en general hacia el logro del
bienestar de repente pueden priorizar el estado a la educación, la salud
y también promoveremos o podemos garantizar la salud.

\begin{itemize}
\tightlist
\item
  ¿La salud es un producto o es un bien físico? es un servicio
\item
  la salud es un derecho que debe ser promotor del estado debe crear las
  condiciones necesarias para brindar servicios de salud, educación
  igual no debería incorporarse cómo del estado debemos incorporado que
  el rol del Estado es generar servicio de educación y de calidad, la
  función del Gobierno estaría en brindar servicios de calidad en la
  educación pública.
\item
  El estado debería ingresar para mejorar la calidad de educación y
  salud en todos los otros sectores en ese entender tenemos que estar
  preocupados permanentemente en la modernización del Estado
  modernización del Estado es un proceso:
\end{itemize}

\begin{enumerate}
\def\labelenumi{\arabic{enumi}.}
\tightlist
\item
  político
\item
  técnico
\end{enumerate}

el poder político tiene que decidir que se debe proceder con un proceso
de modernización porque sí no hay decisión política no hay forma de
hacer.

\begin{itemize}
\tightlist
\item
  el proceso de modernización es agilizar y es proceso político decisión
  política una vez que el político ha decidido qué cosas vamos a
  descentralizar el MEEF tiene que ser un proceso técnico.
\end{itemize}

\hypertarget{en-quuxe9-se-debe-centrar-el-estado-hacia-la-modernizaciuxf3n}{%
\subsubsection{¿En qué se debe centrar el estado hacia la
modernización?}\label{en-quuxe9-se-debe-centrar-el-estado-hacia-la-modernizaciuxf3n}}

\begin{enumerate}
\def\labelenumi{\arabic{enumi}.}
\tightlist
\item
  debe centrarse en la transformación de actitudes que la gente vea con
  diferencia el cargo público que vea con diferencia las funciones el
  nivel la representación al Gobierno y el hombre debe estar con actitud
  positiva que debe de estar identificado con el manejo gubernamental.
\item
  fortalecer las aptitudes hay que hacer que la gente esté debidamente
  preparada para alcanzar los objetivos institucionales tiene que estar
  bien capacitado y si no está capacitado hay que capacitarlo.
\item
  Hay que transformar para que el estado de modernice, agilización de
  procesos.
\item
  Tenemos que identificar plenamente los sistemas funcionales y
  administrativos debemos identificar y hacer las relaciones y
  estructuras administrativas correctamente, como cada 1 de las unidades
  estructuradas que deben entrelazar deben complementarse deben sumar al
  mismo objetivo de modo que no tenga contradicciones debe haber
  relaciones y estructuras administrativas todo esto con el fin de hacer
  compatibles con todo los niveles de Gobierno además con los planes
  nacionales e institucionales, Plan Nacional y el planeamiento
  estratégico nacional.
\end{enumerate}

\hypertarget{por-quuxe9-es-necesario-impulsar-un-proceso-de-modernizaciuxf3n}{%
\subsubsection{¿Por qué es necesario impulsar un proceso de
modernización?}\label{por-quuxe9-es-necesario-impulsar-un-proceso-de-modernizaciuxf3n}}

en la actualidad tenemos el sector público muy pesado muy amplios
debemos impulsar hace tiempo se debió hacer un nuevo proceso de
modernización del Estado la modernización.

por qué es necesario hacer un proceso de modernización?

primero hay que emprender un proceso de reforma integral de la gestión a
nivel gerencial a nivel operacional hay que hacer un proceso de reforma
integral a nivel de gestión gerencial nivel de los de las gerencias y
también la parte operativa de modo que se pueda afrontar las debilidades
del aparato estatal, Porque hay muchos aspectos en la actualidad en la
que el aparato estatal no responde a las expectativas de la población.

Pasar de una administración pública que obtenga resultados para el
ciudadano siempre delante de nosotros tiene que estar la población el
servicio es el objetivo principal tienes que saber desde ahora que
cuando trabajes para el estado.

la gestión pública es la realización de acciones orientadas a mejorar a
incrementar los niveles de eficiencia y eficacia de la gestión pública a
fin de lograr resultados en beneficios de la población en beneficio de
la ciudadanía.

entonces la gestión pública debe ser con enfoque de resultado.

La planificación en realidad estás hablando de un proceso de
racionalizar a mediano a largo plazo por eso es importante porque dice
dónde cuándo un medio que sirve a la acción de desarrollo es
indispensable para entrar en acción necesaria para el desarrollo.

Este proceso de racionalizador en medio para la acción que es la
planificación para el desarrollo es a través de la escogencia la
realización de mejores métodos la planificación es un proceso
racionalizador es un medio necesario indispensable a la acción del
desarrollo a través de la priorización la realización de los mejores
métodos para satisfacer determinadas políticas y lograr sus objetivos

si hablamos a nivel de sociedad tendríamos que decir que la
planificación es un proceso social, así como hemos dicho un proceso
racionalizador podríamos decir que es un proceso social que permite
ordenar las actividades destinadas a satisfacer las necesidades físicas,
sociales necesidades físicas sociales políticas y culturales de una
sociedad.

procesos de planificación social siempre se debe buscar una
planificación participativa esto sí es importantísimo que sea
participativo debemos hacer que la gente se incorpore que los
representantes más representativos de la población que contribuyan que
lo hagan suyo el plan que va a resultar porque si la población no lo
siente como suyo lo que puede hacer que fracase ese proceso de
planificación.

Desde el Gobierno podemos decir es un instrumento Del Gobierno la
planificación es un instrumento de Gobierno dirigido a superar
progresivamente a superar progresivamente el carácter espontáneo del
proceso económico, político y social para sustituirlo por un desarrollo
orgánico y armónico y alcanzar los objetivos prefijados

la planificación te permite tomar decisiones adecuadas te permite hacer
que el desarrollo económico, social, político y cultural sea armónico y
secuencial.

\hypertarget{realidad-nacional}{%
\section{Realidad nacional}\label{realidad-nacional}}

cuando hablamos de realidad nacional nos estamos refiriendo al conjunto
de recursos humanos naturales y financieros elementos institucionales y
relaciones creadas por los diferentes grupos sociales en el país en sus
actividades económicas, sociales, políticas y culturales a lo largo de
la historia y los vigentes dentro del territorio nacional, así como la
relaciones que se generan entre estos y el exterior.

Los elementos institucionales están relacionados por los grupos sociales
distintos a nivel del país y que cada 1 de ellos tienen diversas
actividades económicas, políticas, sociales y culturales, pero todo ese
cruce de relaciones no es casual entonces siempre hay que tener en
cuenta su pasado la historia no hay que agarrar a la población solo
mirando el futuro siempre hay que considerar el pasado la historia. Cómo
piensa cuáles son sus expectativas es que cree de las otras zonas del
país no se está refiriendo a lo que está fuera del país también está
incluido, pero fuera de su ámbito.

nuestra realidad nacional no debemos confundir son toda la estructura
como digamos la realidad nacional está constituido por toda la
estructura económico social político cultural de los pueblos que están
interrelacionados entre sí los elementos fundamentales en nuestra región
en la que el actor principal es la población y la naturaleza lo
acondiciona.

Nuestra realidad nacional no solo son los recursos también hay que
considerar las actividades culturales sociales económicas políticas está
incluido sobre el hombre sobre la naturaleza y las finanzas es nuestra
realidad nacional no se trata yo voy a describir la realidad nacional es
probablemente imposible describir la realidad nacional pero algunas
partes de la realidad nacional.

desenvolvimiento histórico se debe considerar como un pasado que ha dado
resultado el producto actual es el desenvolvimiento histórico.


\printbibliography


\end{document}
