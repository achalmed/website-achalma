% Options for packages loaded elsewhere
\PassOptionsToPackage{unicode}{hyperref}
\PassOptionsToPackage{hyphens}{url}
\PassOptionsToPackage{dvipsnames,svgnames,x11names}{xcolor}
%
\documentclass[
  letterpaper,
  DIV=11,
  numbers=noendperiod]{scrartcl}

\usepackage{amsmath,amssymb}
\usepackage{iftex}
\ifPDFTeX
  \usepackage[T1]{fontenc}
  \usepackage[utf8]{inputenc}
  \usepackage{textcomp} % provide euro and other symbols
\else % if luatex or xetex
  \usepackage{unicode-math}
  \defaultfontfeatures{Scale=MatchLowercase}
  \defaultfontfeatures[\rmfamily]{Ligatures=TeX,Scale=1}
\fi
\usepackage{lmodern}
\ifPDFTeX\else  
    % xetex/luatex font selection
\fi
% Use upquote if available, for straight quotes in verbatim environments
\IfFileExists{upquote.sty}{\usepackage{upquote}}{}
\IfFileExists{microtype.sty}{% use microtype if available
  \usepackage[]{microtype}
  \UseMicrotypeSet[protrusion]{basicmath} % disable protrusion for tt fonts
}{}
\makeatletter
\@ifundefined{KOMAClassName}{% if non-KOMA class
  \IfFileExists{parskip.sty}{%
    \usepackage{parskip}
  }{% else
    \setlength{\parindent}{0pt}
    \setlength{\parskip}{6pt plus 2pt minus 1pt}}
}{% if KOMA class
  \KOMAoptions{parskip=half}}
\makeatother
\usepackage{xcolor}
\setlength{\emergencystretch}{3em} % prevent overfull lines
\setcounter{secnumdepth}{-\maxdimen} % remove section numbering
% Make \paragraph and \subparagraph free-standing
\ifx\paragraph\undefined\else
  \let\oldparagraph\paragraph
  \renewcommand{\paragraph}[1]{\oldparagraph{#1}\mbox{}}
\fi
\ifx\subparagraph\undefined\else
  \let\oldsubparagraph\subparagraph
  \renewcommand{\subparagraph}[1]{\oldsubparagraph{#1}\mbox{}}
\fi


\providecommand{\tightlist}{%
  \setlength{\itemsep}{0pt}\setlength{\parskip}{0pt}}\usepackage{longtable,booktabs,array}
\usepackage{calc} % for calculating minipage widths
% Correct order of tables after \paragraph or \subparagraph
\usepackage{etoolbox}
\makeatletter
\patchcmd\longtable{\par}{\if@noskipsec\mbox{}\fi\par}{}{}
\makeatother
% Allow footnotes in longtable head/foot
\IfFileExists{footnotehyper.sty}{\usepackage{footnotehyper}}{\usepackage{footnote}}
\makesavenoteenv{longtable}
\usepackage{graphicx}
\makeatletter
\def\maxwidth{\ifdim\Gin@nat@width>\linewidth\linewidth\else\Gin@nat@width\fi}
\def\maxheight{\ifdim\Gin@nat@height>\textheight\textheight\else\Gin@nat@height\fi}
\makeatother
% Scale images if necessary, so that they will not overflow the page
% margins by default, and it is still possible to overwrite the defaults
% using explicit options in \includegraphics[width, height, ...]{}
\setkeys{Gin}{width=\maxwidth,height=\maxheight,keepaspectratio}
% Set default figure placement to htbp
\makeatletter
\def\fps@figure{htbp}
\makeatother

\KOMAoption{captions}{tableheading,figureheading}
\makeatletter
\makeatother
\makeatletter
\makeatother
\makeatletter
\@ifpackageloaded{caption}{}{\usepackage{caption}}
\AtBeginDocument{%
\ifdefined\contentsname
  \renewcommand*\contentsname{Tabla de contenidos}
\else
  \newcommand\contentsname{Tabla de contenidos}
\fi
\ifdefined\listfigurename
  \renewcommand*\listfigurename{Listado de Figuras}
\else
  \newcommand\listfigurename{Listado de Figuras}
\fi
\ifdefined\listtablename
  \renewcommand*\listtablename{Listado de Tablas}
\else
  \newcommand\listtablename{Listado de Tablas}
\fi
\ifdefined\figurename
  \renewcommand*\figurename{Figura}
\else
  \newcommand\figurename{Figura}
\fi
\ifdefined\tablename
  \renewcommand*\tablename{Tabla}
\else
  \newcommand\tablename{Tabla}
\fi
}
\@ifpackageloaded{float}{}{\usepackage{float}}
\floatstyle{ruled}
\@ifundefined{c@chapter}{\newfloat{codelisting}{h}{lop}}{\newfloat{codelisting}{h}{lop}[chapter]}
\floatname{codelisting}{Listado}
\newcommand*\listoflistings{\listof{codelisting}{Listado de Listados}}
\makeatother
\makeatletter
\@ifpackageloaded{caption}{}{\usepackage{caption}}
\@ifpackageloaded{subcaption}{}{\usepackage{subcaption}}
\makeatother
\makeatletter
\@ifpackageloaded{tcolorbox}{}{\usepackage[skins,breakable]{tcolorbox}}
\makeatother
\makeatletter
\@ifundefined{shadecolor}{\definecolor{shadecolor}{rgb}{.97, .97, .97}}
\makeatother
\makeatletter
\makeatother
\makeatletter
\makeatother
\ifLuaTeX
\usepackage[bidi=basic]{babel}
\else
\usepackage[bidi=default]{babel}
\fi
\babelprovide[main,import]{spanish}
% get rid of language-specific shorthands (see #6817):
\let\LanguageShortHands\languageshorthands
\def\languageshorthands#1{}
\ifLuaTeX
  \usepackage{selnolig}  % disable illegal ligatures
\fi
\usepackage[]{biblatex}
\addbibresource{../../../../references.bib}
\IfFileExists{bookmark.sty}{\usepackage{bookmark}}{\usepackage{hyperref}}
\IfFileExists{xurl.sty}{\usepackage{xurl}}{} % add URL line breaks if available
\urlstyle{same} % disable monospaced font for URLs
\hypersetup{
  pdftitle={Seminario de filosofía},
  pdfauthor={Achalma Mendoza Edison},
  pdflang={es},
  colorlinks=true,
  linkcolor={blue},
  filecolor={Maroon},
  citecolor={Blue},
  urlcolor={Blue},
  pdfcreator={LaTeX via pandoc}}

\title{Seminario de filosofía\thanks{Universidad Nacional de San
Cristóbal de Huamanga}}
\usepackage{etoolbox}
\makeatletter
\providecommand{\subtitle}[1]{% add subtitle to \maketitle
  \apptocmd{\@title}{\par {\large #1 \par}}{}{}
}
\makeatother
\subtitle{Notas sobre filosofía marxista}
\author{Achalma Mendoza Edison}
\date{2023-05-19}

\begin{document}
\maketitle
\ifdefined\Shaded\renewenvironment{Shaded}{\begin{tcolorbox}[boxrule=0pt, enhanced, breakable, borderline west={3pt}{0pt}{shadecolor}, sharp corners, interior hidden, frame hidden]}{\end{tcolorbox}}\fi

\hypertarget{introducciuxf3n}{%
\section{Introducción}\label{introducciuxf3n}}

Textos de referencia:

\begin{enumerate}
\def\labelenumi{\arabic{enumi}.}
\item
  ``Introducción a la dialéctica'' de Friedrich Engels.
\item
  ``La familia, la propiedad privada y el Estado'' de Friedrich Engels.
\item
  ``La transformación del mono en hombre a través del trabajo'' de
  Friedrich Engels.
\item
  ``Carlos Marx'' de Lenin, incluido en las Obras escogidas, Tomo II.
\end{enumerate}

La afirmación de que las matemáticas o la lógica son los únicos sistemas
que forman la mente del hombre es errónea. En realidad, es la filosofía
la que desempeña un papel fundamental en este proceso. La filosofía
abarca distintas etapas y modos de producción, y es a través de ella que
se desarrolla el conocimiento humano.

Un referente importante en este sentido es Lenin, quien comprendió que
la filosofía era una necesidad política. Él afirmaba que ``el meollo de
la ideología es la filosofía''. Lenin se dedicó a estudiar el proceso
filosófico desde una perspectiva marxista, profundizando en temas como
la ciencia de la lógica de Hegel.

Para respaldar esta afirmación, podemos mencionar algunos textos
relevantes que abordan la relación entre filosofía y política. Por
ejemplo, los ``Cuadernos filosóficos'' de Lenin, así como el IV tomo de
``Acerca de la práctica'' y ``sobre la contradicción'' del Presidente
Mao, son obras que exploran la importancia de la filosofía en el
contexto político.

Es importante destacar que la filosofía no solo es relevante a nivel
individual, sino también a nivel colectivo. En este sentido, se puede
afirmar que \textbf{sin filosofía no hay partido}. Esto implica que un
partido político o cualquier movimiento social necesita tener una base
filosófica sólida para comprender el mundo, desarrollar una estrategia
política coherente y luchar por sus ideales.

El \textbf{proceso de la filosofía} es un tema que requiere revisar y
descartar el criterio limitado de que esta disciplina solo se desarrolló
en el mundo griego. Es importante superar ese prejuicio y el desprecio
hacia el pensamiento de otros pueblos. De hecho, los estudios
posteriores demuestran que la filosofía también tuvo su desarrollo en
China, la India y otras civilizaciones.

A medida que las civilizaciones avanzan, los pueblos se esfuerzan por
comprender la esencia de las cosas, el porqué de las situaciones.
Ejemplos de esto se encuentran en las civilizaciones de Egipto,
Mesopotamia y el pueblo Hebreo. A pesar de ello, se ha considerado
erróneamente que estos son meros precursores de la filosofía, negando su
propio proceso de desarrollo desde épocas antiguas. Incluso en el ámbito
religioso, los egipcios, por ejemplo, planteaban el agua como principio
primordial, un símbolo de vida, pero desconocían su origen. Sin embargo,
en las islas formadas por la expansión del Nilo, desarrollaron una
comprensión más profunda de la espiritualidad y la materia, planteando
la dualidad entre espíritu y materia, y siempre estableciendo que el
principio primordial es la materia.

Los griegos, por su parte, nos brindaron un desarrollo filosófico más
completo, estrechamente vinculado al surgimiento del mercado y al uso de
la moneda, así como a avances científicos. Tales de Mileto, por ejemplo,
predijo el primer eclipse. Aunque los egipcios poseían conocimientos
matemáticos por práctica, fueron los griegos quienes explicaron y
demostraron los hechos de manera más sistemática. Esto fue posible
gracias al avance en el conocimiento científico y a las luchas de clases
en la sociedad esclavista, donde se agudizó el conflicto entre
comerciantes y agricultores. Incluso la llamada ``democracia griega''
fue precedida por un proceso dictatorial. Por lo tanto, es incorrecto
afirmar que la filosofía se desarrolló al margen de las clases sociales
en los siglos VII y VI a.C

\hypertarget{escuela-materialista}{%
\section{Escuela materialista}\label{escuela-materialista}}

La \textbf{escuela materialista} plantea un enfoque fundamentado en el
origen y el porqué de las cosas. Los pensadores, como Arge, reconocieron
que el comienzo de todo está en el agua, una ley que engloba todo. Estas
ideas ya habían sido expuestas por los egipcios. A través de
investigaciones, Arge descubrió conchas fósiles en las islas, lo que
respaldaba la concepción materialista del origen.

Otro filósofo, Heráclito, propuso que el fuego es el origen de todas las
cosas. Su perspectiva se basaba en la realidad material. Argumentaba que
la guerra y la lucha de opuestos son el fundamento del constante
desarrollo, en el que todo está en un flujo constante. Esta noción nos
introduce al concepto de la dialéctica, con sus intuiciones geniales.
Aunque se han conservado algunas de sus frases, no se ha transmitido
mucho más de su pensamiento.

Aristóteles hizo una revisión histórica de estas ideas, reconociendo su
genialidad pero considerándolas no fundamentales. La filosofía comenzó a
separarse de la religión y surgió el idealismo como contraposición.
Parménides, en particular, negó la dialéctica y planteó que el origen de
todas las cosas es el ser absoluto. Según él, el ser abarca todo, y las
cosas existen porque participan de esa esencia.

El idealismo se contrapone al materialismo y se desarrolló
posteriormente. Mientras que el materialismo parte de la materia
primaria y del proceso del conocimiento, los idealistas adoptaron una
postura distinta.

\textbf{Demócrito} fue un destacado filósofo materialista cuya teoría de
los átomos tuvo un impacto significativo. Según él, todo lo que existe
está compuesto por partículas indivisibles y eternas en constante
movimiento. Esta idea refutaba las teorías de los idealistas, como
Parménides, quienes sostenían la infinita divisibilidad que llevaría a
la inexistencia. Cabe mencionar que la refutación definitiva de la
indivisibilidad del átomo no se produjo hasta el año 1900.

Demócrito también abordó el tema del conocimiento, argumentando que este
es un reflejo de los átomos en nuestra mente. Los efluvios se
entrecruzan y generan errores en nuestra percepción. Además, destacó que
el ser humano se desenvuelve en un contexto social, siendo un elemento
integral de las polis (ciudades-estado). Lo que se observa en su propia
ciudad se refleja en su pensamiento. En cuanto a la esclavitud,
Demócrito la consideraba perjudicial, ya que degrada al ser humano y no
le permite manifestar su mejor versión. La libertad, en cambio, es lo
que corresponde al ser humano. Con estas reflexiones, Demócrito
incursionó en el campo de la moral, explorando qué permitiría a las
personas vivir en libertad. Es considerado uno de los mayores exponentes
del materialismo en la antigüedad.

Es importante destacar que el materialismo siempre ha estado vinculado a
una comprensión y un respeto hacia el ser humano. Su pensamiento
desafiaba los criterios de la clase dominante y las ideas idealistas,
las cuales estaban estrechamente ligadas a los comerciantes y los
esclavistas. Por otro lado, los sofistas argumentaban que el ser humano
podía ser educado y, de esta manera, elevarse. Además, postulaban que el
hombre es la medida de todas las cosas. En el caso de Sócrates, se
evidencia la fuerte dimensión social de los griegos, donde el
individualismo aún no estaba plenamente desarrollado.

\textbf{Platón}, filósofo estrechamente vinculado a la aristocracia y de
gran fortuna personal, desarrolló un sistema filosófico completo de
carácter idealista. En su pensamiento, sostiene la existencia de una
apariencia y una realidad, argumentando que nuestros sentidos nos
engañan y que la apariencia corresponde al mundo material, mientras que
la realidad está en el ámbito de las ideas.

Platón introduce la noción de una trinidad de ideas: el bien, la belleza
y la verdad, las cuales se sustentan en el ser. A través de su teoría de
la coparticipación de las ideas, argumenta que la realidad de las cosas
participa de estas ideas. Además, presenta la idea de un comunismo
platónico, que tiene sus antecedentes en el antiguo Egipto, aunque con
un enfoque reaccionario. Según Platón, la propiedad genera conflictos y
él consideraba que el ordenamiento democrático era perjudicial,
proponiendo en su lugar un gobierno de élites.

En cuanto a la educación, Platón la consideraba dañina. Esta perspectiva
se debe a que la aristocracia, a la cual pertenecía, estaba siendo
atacada y socavada por los comerciantes. En su visión de la sociedad, se
concibe como un conjunto de trabajadores que se van educando y
seleccionando en diferentes roles, como trabajadores, guerreros, entre
otros. Al final, quedaba un grupo de élites que detentaba el poder (una
idea que puede relacionarse con el fascismo). La música también es
criticada por Platón, ya que consideraba que podía corromper a las
personas. En su afán por establecer su propio sistema filosófico, Platón
destruyó todo lo que llegó a su alcance de las ideas de Demócrito.

\textbf{Aristóteles}, discípulo de Platón, desempeñó un papel
fundamental al informarnos sobre las ideas de los materialistas. A
diferencia de su maestro, Aristóteles se basó en los conocimientos
científicos y sociales de su época para desarrollar su propia filosofía.
A través de sus críticas a Platón, Aristóteles estableció una
perspectiva más centrada en el conocimiento científico.

Según Aristóteles, las cosas existen tanto en su realidad material como
en su forma. Consideraba que la forma era necesaria para distinguir y
definir las cosas, ya que sin ella se confundirían entre sí. De esta
manera, Aristóteles aborda el idealismo partiendo de una base real, al
incorporar la idea dentro de la realidad misma. Comienza a emplear
conceptos y formas, y introduce el concepto de esencia, que implica la
existencia de una sustancia primaria y superior que da origen al
movimiento. En este sentido, Dios, considerado el primer motor inmóvil,
representa la palabra que se conoce a sí misma. Aristóteles llega al
idealismo a través de un razonamiento complejo. Aunque reconoce la
existencia real de las cosas, sostiene que es el acto de pensar en sí
mismo lo que pone en movimiento la realidad.

En cuanto a la materia concreta, Aristóteles argumenta que carece de
movimiento intrínseco, y es la idea o la forma lo que impulsa el
movimiento. Esto introduce una dialéctica conceptual en su filosofía.
Una de las características positivas de su enfoque es que reconoce la
importancia de la materia en la existencia de las cosas. No obstante, en
algunos aspectos, Aristóteles se asemeja al platonismo al postular la
existencia de ideas y formas que trascienden la realidad material.

\hypertarget{escuelas}{%
\section{Escuelas}\label{escuelas}}

El neoplatonismo, fue una corriente filosófica que los romanos nunca
pudieron superar y que eventualmente desembocó en la mística. Sin
embargo, la iglesia no pudo afiliarse completamente con el platonismo
debido a sus diferencias fundamentales.

\textbf{Medievo:}

Durante la Edad Media, la filosofía comenzó a desarrollarse como una
reivindicación de la Razón. Además de los árabes, quienes jugaron un
papel fundamental en la difusión del conocimiento filosófico griego y
aristotélico, se comenzó a conocer el aristotelismo. Los árabes incluso
llegaron a desarrollar un enfoque materialista y distinguieron
claramente la filosofía de la teología, asignando a la filosofía el
dominio terrenal y a la teología el dominio celestial. Tanto los árabes
como los hebreos tuvieron una influencia significativa en este período.

\textbf{Realistas y nominalistas:}

Los realistas defendían las tesis aristotélicas y afirmaban que las
cosas y las ideas existían independientemente entre sí. Por otro lado,
los nominalistas sostenían que las ideas no tenían contenido real y eran
meras abstracciones derivadas de las cosas. Estas diferencias teóricas
también estaban vinculadas a disputas de índole religiosa.

\textbf{Pedro Abelardo:}

Pedro Abelardo desempeñó un papel importante al introducir la lógica
formal y ser el creador de la lógica deductiva. Utilizó la lógica de
manera dialéctica, promoviendo debates y discusiones. Su influencia fue
significativa en el pensamiento francés y también fue conocido por sus
ataques a la religión. Marx consideró que el nominalismo de Abelardo era
de gran importancia en la filosofía.

\textbf{Duns Scotto:}

Duns Scotto, un fraile franciscano, tuvo una gran relevancia en el
desarrollo del materialismo moderno. Se preguntó cómo se podría combatir
la religión y planteó su famoso argumento sobre la comunión. Argumentó
que si todos los hombres comulgaran, el cuerpo de Cristo se agotaría. En
una época violenta y dura, aquellos que se oponían eran perseguidos y
asesinados. Esto pone de manifiesto que los filósofos no eran
simplemente eruditos tranquilos, sino que el debate filosófico a menudo
se libraba a través de puñaladas y veneno.

\textbf{Tomás de Aquino:}

Tomás de Aquino, asociado con el tomismo, fue un filósofo italiano que
se unió a los dominicos. Como discípulo de Alberto Magno, argumentó que
se podía comprender racionalmente la religión católica y que la razón no
estaba en oposición a la teología. Sin embargo, su enfoque distorsionaba
las ideas de Aristóteles en lugar de desarrollarlas, y su obra más
importante, ``Ente y La Razón'' fue perseguida por la iglesia en vida.
El tomismo fue desafiado posteriormente por Occam y Scotto.

El desarrollo filosófico comenzó a tomar forma con la burguesía,
destacando figuras como Francisco Bacon, quien enfatizó la importancia
de la experiencia y propuso una lógica inductiva que sería útil para la
ciencia. Reconoció la influencia de la teología, pero también planteó
que su pensamiento iba más allá de ella.

\textbf{Descartes (1596-1650)} fue un discípulo de los jesuitas. Durante
sus estudios, se dio cuenta de que lo que se afirmaba en un lugar podía
negarse en otro, lo que llevó a cuestionar los fundamentos sólidos de la
ciencia. En su trabajo, introdujo las coordenadas cartesianas, que
permitieron la unificación de la geometría y el análisis algebraico.
Descartes se enfocó en el estudio de la física y la materia, retomando
el pensamiento de Demócrito. En este campo, adoptó una postura
materialista.

Descartes planteó el método de la duda, que no debe confundirse con el
escepticismo, ya que no se trata de cuestionar todo conocimiento, sino
de dudar para llegar a un conocimiento evidente. También señaló la
capacidad engañosa de los sentidos, afirmando que no se puede confiar en
ellos. Sin embargo, encontró una verdad indudable en la existencia del
``yo''. Descartes afirmó: ``Dudo, luego existo''. La existencia del
``yo'' y el hecho de pensar son verdades evidentes que no se pueden
poner en duda. A través de los pensamientos, Descartes percibe la
realidad.

Descartes también argumentó que sus ideas existen porque Dios existe y
es quien les ha dado origen. Para él, todo existe porque Dios existe. Si
bien en sus estudios científicos adoptó una perspectiva materialista, en
sus reflexiones metafísicas puso al ``yo'' como el punto central de la
filosofía. A partir de este punto, comienza a establecerse el
pensamiento burgués.

En la filosofía alemana, figuras como Leibniz, Kant y Hegel desempeñaron
un papel importante.

Durante los siglos XVII al XIX (hasta 1830 aproximadamente), el
luteranismo surgió como una corriente que buscaba reformar la iglesia.
Alemania fue el lugar donde se desarrolló el pensamiento idealista más
avanzado. Leibniz, reconocido como un gran matemático, trabajó en el
desarrollo de la lógica y replanteó la lógica aristotélica. Sin embargo,
sus pensamientos no se difundieron ampliamente. Leibniz promovió un
racionalismo que consideraba posible un análisis lógico mediante el uso
de símbolos, similar al análisis matemático. Propuso la teoría de las
mónadas, entidades cerradas que se comunicaban a través de una
``ventanilla'', representando un movimiento autónomo e idealista. Aunque
abordó problemas relacionados con la dinámica, estos eran principalmente
conceptuales debido a su enfoque idealista. Leibniz también se dedicó al
análisis del conocimiento humano y estableció conexiones entre las
matemáticas y la física.

\textbf{Kant (1724-1804)} se enfoca en el problema del conocimiento a
través de su obra ``Crítica de la razón pura''. Plantea que existe una
realidad fenoménica, lo que aparece ante nosotros, lo que la luz revela.
Establece una distinción entre los fenómenos y la cosa en sí misma, que
no es accesible al conocimiento humano. Aunque reconocemos la existencia
de la materia, no podemos conocerla directamente. Kant establece una
relación entre el sujeto que conoce y el objeto conocido, pero existe
una parte que permanece desconocida. Al analizar las cosas, captamos
sensaciones a través de nuestra sensibilidad.

Kant desarrolla un sistema completo del conocimiento que abarca el
mundo, el hombre (el alma) y Dios. Elabora conceptos y establece las
categorías como un sistema lógico del conocimiento. Reconoce que solo
podemos conocer los fenómenos y que la cosa en sí misma escapa a nuestro
entendimiento. El conocimiento es una construcción de la razón pura que
se realiza a través de la interacción entre el sujeto y los objetos
(cosas). Kant enfatiza la importancia del sujeto en este proceso. Existe
una realidad que podemos conocer y otra que permanece inaccesible,
aunque se deje conocer en parte.

Después de Kant, se desarrolla el neokantismo, que tiende a disolver la
noción de la cosa en sí misma y se adentra en un ultra-idealismo. Kant
ha logrado un conocimiento a través de la facultad del entendimiento. En
su obra ``Crítica de la razón práctica'', cuando analiza el alma, Kant
plantea la idea de la libertad, que solo puede alcanzarse en Dios. La
libertad, el alma y la existencia de Dios están intrínsecamente
relacionadas. Kant establece los límites del idealismo (la razón) y
ordena la comprensión del conocimiento. ¿Por qué se plantea la
existencia de Dios? Para explicar que todo tiene un comienzo y un fin,
se busca una causa, y esa causa es Dios. Sin embargo, al plantear que
Dios es la causa, surge la pregunta: ¿cuál es la causa de Dios? Kant
mismo cuestiona la existencia de Dios al plantear este argumento.

\textbf{Hegel:} Mientras Kant intenta conocer la realidad a partir del
yo, Hegel se centra en lo objetivo. Analiza el proceso de la filosofía y
considera que todos los filósofos anteriores a él carecen de
importancia. Desarrolla una teoría de la dialéctica que busca comprender
todo el proceso de la materia, aunque su enfoque es idealista. Según él,
el proceso se desarrolla a través de contradicciones que generan
problemas de cantidad y calidad, apariencia y realidad. Hegel ve la
dialéctica como un proceso de contradicción entre conceptos e ideas. Sin
embargo, contradice su propia dialéctica al plantear la existencia de
una Gran Idea Absoluta. Esta Gran Idea representa la realidad objetiva,
cuyo proceso de contradicción se limita a nivel de las ideas solamente.
Es similar a Aristóteles, pero sin partir de la materia. Esta idea juzga
a la materia por su propio proceso de contradicción. A medida que el
espíritu se desenvuelve, surge el hombre, y el espíritu se convierte en
autoconciencia, negándose a sí mismo. El hombre abarca la sociedad, el
conocimiento, la ciencia, el arte, la religión y la nación, y finalmente
genera el Estado. El Estado se convierte en una gran transformación que,
finalmente, se identifica con el Espíritu, es decir, Dios.

Hegel tiene una comprensión general del desarrollo materialista, pero su
enfoque es idealista. Hay dos partes en su filosofía: su idealismo, que
es desechable, y su materialismo, que es aceptable.

En Francia, \textbf{Diderot} desarrolla un enfoque materialista basado
en la idea de una materia eterna, sin principio ni fin. Diderot incluso
plantea la existencia de un movimiento interno en la materia, aunque no
explica por qué. Sin embargo, el antecedente directo del marxismo es la
filosofía clásica alemana. Después de la muerte de Hegel, surge una
división, y algunos filósofos comienzan a criticar su idealismo. Uno de
los críticos importantes es Feuerbach, quien critica el idealismo de
Hegel, pero no logra distinguir claramente entre el materialismo y el
idealismo de Hegel. Esto lleva a Feuerbach a desechar a Hegel. La noción
de alienación ante la religión, que es un concepto presente en Hegel, no
es una tesis de Marx. Marx y Feuerbach difieren en este punto, ya que
Marx considera que la solución a la alienación es la revolución y la
emancipación.

Hegel plantea que el trabajo aliena al hombre de su esencia como ser
pensante y como ser nacional. Marx analiza las causas de la alienación.
Feuerbach sostiene que, frente a la alienación, el centro de atención
debe ser el hombre y no Dios. La relación entre los individuos se basa
en el amor, la caridad y la preocupación por el otro, la maternidad y
una posición subjetivista que aborda cómo un ``yo'' se relaciona con
otro ``yo''. Esto se asemeja a un cristianismo sin Cristo. Lo importante
es la crítica materialista. Marx y Engels luchan contra el
individualismo de Feuerbach.

Marx y Engels desarrollan el proceso filosófico marxista. Marx lo
desarrolla y Engels lo difunde. Las tesis sobre Feuerbach constituyen la
base de su pensamiento:

\begin{enumerate}
\def\labelenumi{\arabic{enumi}.}
\item
  Critican el defecto de todo materialismo anterior por no haber tenido
  en cuenta la práctica. El materialismo anterior se desarrolló en el
  empirismo, que veía la realidad como algo pasivo y no comprendía cómo
  la materia actúa y cómo el hombre, a través de su trabajo, cambia la
  realidad. El empirismo es una posición burguesa. Marx postula la
  comprensión de la realidad y su transformación.
\item
  Consideran que la práctica es la prueba de la verdad. Marx critica a
  Feuerbach, quien nunca llegó a concebir la percepción sensorial como
  una capacidad transformadora. Feuerbach había diluido lo religioso en
  la esencia humana, lo que Marx considera como incapacidad para
  comprender el mundo social y las relaciones sociales.
\item
  Sostienen que la vida social es esencialmente práctica. La mente
  humana se desvía debido a una serie de misticismos. Solo al comprender
  la práctica se puede superar el misticismo. Marx califica a aquellos
  que no comprenden la práctica como materialistas contemplativos. Marx
  llega a estudiar las instituciones de la sociedad civil, pero solo
  superficialmente, sin llegar a la raíz que las sustenta. El problema
  radica en transformar el mundo, no solo en contemplarlo.
\end{enumerate}

Con este documento, Marx establece las diferencias claras entre su
enfoque y el de Feuerbach.

\textbf{Ajuste de cuentas con pensamientos anteriores y adopción de una
nueva posición}. Se establecen nuevos criterios para formar una nueva
ideología, centrándose en el proceso económico de la sociedad y
planteando el comunismo como la primera gran revolución mundial, a
diferencia de las anteriores que solo implicaron el reemplazo de una
clase por otra.

A lo largo de su evolución, la filosofía ha desarrollado teorías sobre
la dialéctica y el materialismo, y ha criticado con razón la Edad Media,
que intentaba resolver cuestiones sin tener en cuenta la realidad. Los
filósofos marxistas han comprendido los hitos del desarrollo y han
afirmado de manera contundente su posición materialista. Acceder al
materialismo requiere un proceso en constante movimiento derivado de la
contradicción.

Althusser niega que Marx y Engels hayan tomado la dialéctica de Hegel.
Sostiene que primero se desarrolla la ciencia y luego se produce un
salto. El descubrimiento de Marx y Engels es el materialismo histórico,
que establece la teoría materialista de la historia, seguida por el
materialismo dialéctico. Según Althusser, el desarrollo de la filosofía
marxista estaba pendiente. Sin embargo, esta afirmación carece de
fundamentos sólidos y puede considerarse errónea de principio a fin.

\textbf{Platón y Kant son filósofos idealistas} que niegan el proceso
científico que se ha desarrollado desde el siglo XVII. A partir de
finales del siglo XVI, se comenzó a comprender que la Tierra es un ente
en constante cambio y movimiento, un proceso dialéctico. En campos como
la química y la biología se han realizado descubrimientos
significativos, como el reconocimiento de la estructura celular y la
teoría de la evolución. Estos avances científicos rompen con la
metafísica y demandan una explicación dialéctica, algo que Althusser no
puede negar.

Hegel introdujo el proceso dialéctico en la esfera del pensamiento, pero
Marx lo llevó al ámbito material. El materialismo dialéctico permite
comprender y transformar la realidad a través de la acción humana sobre
la materia, algo que nunca se había logrado antes. Por lo tanto, se
cuestiona la crítica de Althusser sobre la naturaleza científica del
marxismo, ya que el materialismo dialéctico implica una transformación
material basada en la práctica.

\textbf{La ideología generada por las clases explotadoras es invertida},
ya que ofrece una explicación idealista de la historia. Por el
contrario, la ideología marxista se considera científica, ya que refleja
de manera verdadera y real la práctica y el carácter de clase. Las
teorías de Althusser conducen a un nuevo surrealismo al fusionar el
racionalismo burgués de Kant con el idealismo burgués de Spinoza. Sin
embargo, \textbf{el marxismo-leninismo-maoísmo se basa en un sólido
fundamento histórico que ha sido construido a lo largo de 2500 años,
incorporando los mejores elementos de la tradición filosófica}. La
aplicación del materialismo dialéctico permite una comprensión
científica de la sociedad y el materialismo histórico.

El marxismo ha llevado a cabo un proceso para demostrar los fundamentos
económicos de la sociedad. Aquellos que lo critican afirman que el
marxismo simplemente realiza una crítica económica de la sociedad. Sin
embargo, no se ha dejado de lado el problema de las ideas y la acción
que las sustenta, ya que se reconoce la influencia de la base económica
en la generación de la ideología.

\textbf{La dialéctica} es un concepto abordado por Engels, quien
establece tres leyes fundamentales: la unidad, lucha de contrarios, y la
negación de la negación. Comprender la dialéctica fue crucial para el
desarrollo de ``El Capital''. El marxismo no es un sistema cerrado, sino
un proceso dialéctico en constante evolución. Nos distanciamos de todas
las corrientes filosóficas que son estáticas y cerradas.

\textbf{Hegel utiliza la dialéctica de manera inconsistente, mientras
que nosotros, los marxistas, la empleamos de manera coherente}. Esta es
la revolución más importante en la historia de la humanidad. La
filosofía marxista establece las bases para el desarrollo continuo del
conocimiento, que nunca puede agotarse. Es un proceso que se acerca cada
vez más a la verdad al eliminar nuevos errores. A lo largo de la
historia, el marxismo ha enfrentado negaciones constantes. Lenin
defendió y desarrolló el marxismo en su obra ``Materialismo y
Empiriocriticismo'', donde introduce la teoría del reflejo, que explica
cómo la conciencia es producto de los reflejos generados por la materia
en acción y reacción. La conciencia es el resultado de un largo proceso
que tiene sus raíces en las características mismas de la materia.

En el ámbito de la física, se planteó en 1900 la teoría cuántica, que
postula que existe una cantidad ínfima de materia necesaria para dar un
salto. Esto condujo al desarrollo de la teoría nuclear. Por otro lado,
Einstein propuso una nueva teoría del espacio-tiempo con su teoría de la
relatividad.

\textbf{Newton} plantea la existencia de dos entidades absolutas e
inseparables: el espacio y el tiempo, pero con los experimentos demostró
que a altas velocidades se produce una reducción en ellos. Esto
significa que el tiempo y el espacio son relativos, pasando de ser
absolutos a relativos. Además, la gravedad implica el movimiento de dos
masas en direcciones más amplias. La física cuántica desafía la noción
de átomo (rompe el átomo) y niega la materia, aunque Lenin argumenta que
apenas estamos comenzando a comprender las primeras partículas. La
materia en movimiento tiene una forma tanto cuantitativa como
cualitativa, y estamos descubriendo nuevas formas de la materia debido a
su movimiento constante. Lenin rechaza la idea de que la materia se
disuelve.

\textbf{Existe un carácter partidista en la filosofía y una lucha contra
el Empiriocriticismo}. La física cuántica ha llevado a negar el
materialismo, ya que al conocer la velocidad de un electrón,
desconocemos su ubicación, lo cual desafía la causalidad en términos de
correlación entre causa y efecto y también en relación a la
previsibilidad. Sin embargo, la causa y el efecto siguen existiendo,
aunque se niegue debido a la previsibilidad. Por lo tanto, lo que se ha
descubierto es la casualidad y una nueva forma de la materia. Se han
encontrado nuevas modalidades y formas de la materia.

En cuanto a la geometría, durante muchos siglos se consideró que la suma
de los ángulos de un triángulo era de 180 grados, siendo esta la única
geometría aceptada. Sin embargo, Gauss planteó que este postulado no
tenía demostración y que su modificación daría lugar a otra geometría.
La geometría de Reimann, por ejemplo, muestra que la suma de los ángulos
de un triángulo puede ser diferente a 180 grados. Esto demuestra que la
materia puede tener múltiples manifestaciones, desde convexa y plana
hasta cóncava, y probablemente muchas otras en el futuro.

En lugar de cuestionar, lo que se debe hacer es confirmar. La materia es
inagotable y están ocurriendo innumerables procesos en ella. La
eternidad de la materia se manifiesta en su movimiento constante. Hoy en
día, se concibe la materia como una interrupción de la nada, pero ¿qué
es la nada? La nada es en sí misma un espacio, y el espacio es una forma
o modalidad de la materia, según la perspectiva de Joudan.

\textbf{La cosmogonía} nos revela el descubrimiento de estrellas que se
desplazan a altas velocidades, lo cual se conoce como la expansión del
universo. Este fenómeno sugiere que el universo tuvo un comienzo y no es
eterno, además de tener un límite. Algunos argumentan que antes no
existía el universo y que hubo un momento inicial de creación. Sin
embargo, basar estas afirmaciones en la parte que conocemos es una
generalización injustificada. Afirmar algo de una parte no implica que
se pueda afirmar de la totalidad. Se intenta introducir la divinidad por
la puerta falsa, como pretende Russell.

El movimiento tiene tanto una dimensión cuantitativa como cualitativa, y
es en este contexto que la filosofía burguesa entra en un proceso de
clara decadencia. Lukacs plantea que la contradicción no radica en el
materialismo versus idealismo, sino en el irracionalismo frente al
racionalismo, lo que refleja una profunda crisis en la filosofía
burguesa.

\textbf{Bergson} desarrolla una metafísica cargada de sentimentalismo,
mientras que \textbf{Nietzsche} expone la teoría del superhombre. Sus
teorías buscan encontrar una salida al imperialismo y se basan en una
moral centrada en los mejores y su dominio. Nietzsche critica al
cristianismo y busca restablecer la moral de los señores, confundiendo
la bondad con la virtud. Es importante señalar que estas ideas promueven
el racismo.

En los años 20 se intenta revitalizar la filosofía. Surge el
neopositivismo en los círculos de Viena, como una respuesta reaccionaria
de la burguesía. Plantea la necesidad de creer en la ciencia positiva y
niega la existencia de leyes en la realidad, sosteniendo que la realidad
es algo que nosotros mismos construimos a través del conocimiento. Esta
corriente considera a la nueva ciencia como una forma de religiosidad,
donde el mejor mundo es el mundo burgués y el progreso es el objetivo
principal.

Los neopositivistas se centran en los fenómenos y promueven un
cientificismo que coloca al sujeto como el elaborador de un sistema
científico y jurídico, pero caen en un desarrollo excesivo de la lógica
y en la elaboración de sistemas basados en el matematicismo derivado de
la ciencia.

\textbf{Pitágoras} planteó que la esencia de las cosas era el número y
que todo podía ser medido. Platón desarrolló esta idea y redujo todo el
conocimiento a fórmulas. Sin embargo, el problema radica en sustituir la
realidad por estas fórmulas. Es importante recordar que las matemáticas
surgen de la realidad material, por ejemplo, el círculo surge de la
rueda. Considerar las matemáticas como una sustitución de la realidad
puede llevarnos a perder de vista lo que representan esas fórmulas en el
mundo físico.

\begin{quote}
hacer un agujero en la pared con una integral y no con lo que representa
la integral --un taladro-
\end{quote}

En cuanto a \textbf{la lógica}, se comienza a analizar y se plantea que
el lenguaje es insuficiente, por lo que es necesario reemplazarlo por
símbolos. Esta simplificación es positiva, ya que nos proporciona un
desarrollo de la lógica simbólica. Se habla de criterios de verificación
y pruebas de verdad. Sin embargo, se termina analizando no la materia en
sí, sino los análisis sobre la materia, es decir, se enfoca en el
análisis lógico.

\textbf{Wittgenstein} es considerado el neopositivista más consecuente.
Él sostiene que no puede hablar del mundo, solo puede hablar de cómo
interpreta el mundo. Afirma que el mundo no es cognoscible y que lo
único de lo que podemos hablar es del conocimiento que tenemos del
mundo. Llega a considerar lo inefable y aboga por el silencio. Esta
postura agnóstica y negación absoluta del conocimiento es compartida por
otros científicos y analistas de la ciencia, como Russell y los antiguos
filósofos Cameades, Numme y Rosses.

Todos estos pensadores llegan al agnosticismo (Principia matemática) y
sus análisis conducen a deshacer el conocimiento. A pesar de descubrir
paradojas que permiten avanzar, se centran en el desmontaje y no logran
hacer la síntesis. Sin embargo, sus contribuciones han limpiado la
filosofía y la ciencia al ponerlas en crisis. En este momento crítico
del conocimiento, el proletariado se convertirá en el establecedor de
nuevos principios. Aunque el proceso de demolición aún no ha terminado,
hay una clase que está muriendo y sus principios están desapareciendo.
Este desconcierto es parte del proceso de decantación hacia nuevas
perspectivas.

\hypertarget{existencialismo}{%
\section{Existencialismo}\label{existencialismo}}

\textbf{Heidegger}, en 1920, se centra en el análisis de la existencia y
sostiene que Dios es el creador. Según esta corriente filosófica, el
hombre es la expresión de la existencia, surge de la nada y se dirige
hacia la nada. El individuo no sabe nada acerca de su existencia ni de
su origen. En este tránsito, experimenta angustia y enfrenta dos
posibilidades: enfrentarla o huir de ella. El problema reside en
enfrentar la angustia y la muerte, aceptar que la muerte es parte de su
identidad. Sin embargo, este enfoque sirvió al nazismo y es una
expresión de una clase agonizante, reflejando una decadencia filosófica.

\textbf{Sartre}, perteneciente a la misma escuela existencialista,
plantea que el hombre es un ser sin existencia que busca aferrarse a
algo para expresar su existencia. El individuo reduce todo a la nada y
busca aferrarse a las cosas, pero esto es una ilusión. Incluso en las
relaciones humanas, el uno al otro se convierte en nada. Otra opción es
el amor, pero la situación es la misma. Por lo tanto, solo queda Dios,
pero según Sartre, Dios no existe. La única salida es la propia
libertad, la cual se convierte en la única alternativa para vivir o
morir. Esta perspectiva existencialista se caracteriza por un pesimismo
y la falta de una solución clara, ya que la libertad se ve como una
correlación dentro de la sociedad.

\textbf{Marcel}, otro exponente del existencialismo, sostiene que el
hombre viene de Dios y se dirige hacia Dios, por lo que el problema
radica en alcanzar a Dios. Estas diferentes corrientes existencialistas
son expresiones de una clase que no encuentra una salida.

El neotomismo, representado por Maritain, busca ajustar el tomismo
católico teniendo en cuenta el desarrollo de la ciencia y la filosofía.
Sin embargo, al intentar adoptar una concepción feudal, muestra la
pobreza ideológica de la iglesia, ya que se trata de una filosofía
obsoleta desde su origen. Los sucesores de Husserl, como Deconte y
García Baca, intentan superar los errores del neotomismo, pero siguen
anclados en su influencia. El presidente Mao destaca la importancia de
conocer el idealismo para poder combatirlo.

En \textbf{el marxismo}, la ley principal es la contradicción, y
Plejanov plantea que el materialismo es la base y la dialéctica es la
guía, siendo la contradicción el elemento central. Sin embargo, no se
establece claramente cuál es el punto medular de la filosofía marxista.
Bajo el liderazgo de Stalin, se produce una regresión en este sentido,
pero es el presidente Mao quien plantea la ley de la contradicción como
la única ley, llegando a un monismo filosófico. Aunque se ha alcanzado
una ley única, esto no significa que el sistema esté completo. En cuanto
a la libertad, se considera tanto la conciencia de la necesidad como la
transformación de esa necesidad, siendo esta última el aspecto
principal. La dialéctica abarca las leyes más generales del desarrollo
del mundo natural, social y del conocimiento. La dificultad radica en
comprender y aplicar estas leyes, y es el presidente Mao quien establece
la ley de la contradicción como la única ley fundamental.

\hypertarget{individualismo}{%
\section{Individualismo}\label{individualismo}}

Según Plejanov, también se aborda desde el monismo, ya que si bien se
tienen en cuenta las leyes y las clases, se reconoce que el individuo
puede perturbarlas. La clave radica en asumir y llevar adelante las
leyes de la forma más pura posible para cumplir el papel exigido por la
revolución. El marxismo-leninismo-maoísmo combate el individualismo y su
raíz, el egoísmo, y rechaza la idea del ``yo'' como lo principal.
Históricamente, el individuo se ha desarrollado a medida que la
propiedad privada ha potenciado la individualidad y el egoísmo,
especialmente bajo la influencia de la burguesía. El marxismo, al
centrarse en la clase, rechaza el individualismo y el egoísmo, y en el
partido, se moldea una nueva forma de ser. La acción colectiva en la
lucha de clases es lo primordial y al trabajar en equipo, se diluye la
formación individualista.

La revolución transforma tanto el mundo como a los hombres. La raíz del
egoísmo es una de las bases del revisionismo y su erradicación requerirá
tiempo. Deshacerse del individualismo será un proceso largo. A medida
que se generan relaciones de producción más avanzadas, esto se reflejará
cada vez más en la conciencia de toda la sociedad.

Como comunistas, debemos ser los heraldos del futuro. La ideología nos
permite avanzar en la lucha contra el egoísmo y debemos ser los más
avanzados en esta tarea. Trabajamos por una meta que puede que no
lleguemos a ver personalmente. Reducir cada vez más el individualismo y
el egoísmo es fundamental. En la lucha, la acción golpea directamente al
individualismo. La ideología es lo que nos impulsa a avanzar.


\printbibliography


\end{document}
