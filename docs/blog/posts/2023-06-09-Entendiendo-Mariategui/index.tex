% Options for packages loaded elsewhere
\PassOptionsToPackage{unicode}{hyperref}
\PassOptionsToPackage{hyphens}{url}
\PassOptionsToPackage{dvipsnames,svgnames,x11names}{xcolor}
%
\documentclass[
  letterpaper,
  DIV=11,
  numbers=noendperiod]{scrartcl}

\usepackage{amsmath,amssymb}
\usepackage{iftex}
\ifPDFTeX
  \usepackage[T1]{fontenc}
  \usepackage[utf8]{inputenc}
  \usepackage{textcomp} % provide euro and other symbols
\else % if luatex or xetex
  \usepackage{unicode-math}
  \defaultfontfeatures{Scale=MatchLowercase}
  \defaultfontfeatures[\rmfamily]{Ligatures=TeX,Scale=1}
\fi
\usepackage{lmodern}
\ifPDFTeX\else  
    % xetex/luatex font selection
\fi
% Use upquote if available, for straight quotes in verbatim environments
\IfFileExists{upquote.sty}{\usepackage{upquote}}{}
\IfFileExists{microtype.sty}{% use microtype if available
  \usepackage[]{microtype}
  \UseMicrotypeSet[protrusion]{basicmath} % disable protrusion for tt fonts
}{}
\makeatletter
\@ifundefined{KOMAClassName}{% if non-KOMA class
  \IfFileExists{parskip.sty}{%
    \usepackage{parskip}
  }{% else
    \setlength{\parindent}{0pt}
    \setlength{\parskip}{6pt plus 2pt minus 1pt}}
}{% if KOMA class
  \KOMAoptions{parskip=half}}
\makeatother
\usepackage{xcolor}
\setlength{\emergencystretch}{3em} % prevent overfull lines
\setcounter{secnumdepth}{-\maxdimen} % remove section numbering
% Make \paragraph and \subparagraph free-standing
\ifx\paragraph\undefined\else
  \let\oldparagraph\paragraph
  \renewcommand{\paragraph}[1]{\oldparagraph{#1}\mbox{}}
\fi
\ifx\subparagraph\undefined\else
  \let\oldsubparagraph\subparagraph
  \renewcommand{\subparagraph}[1]{\oldsubparagraph{#1}\mbox{}}
\fi


\providecommand{\tightlist}{%
  \setlength{\itemsep}{0pt}\setlength{\parskip}{0pt}}\usepackage{longtable,booktabs,array}
\usepackage{calc} % for calculating minipage widths
% Correct order of tables after \paragraph or \subparagraph
\usepackage{etoolbox}
\makeatletter
\patchcmd\longtable{\par}{\if@noskipsec\mbox{}\fi\par}{}{}
\makeatother
% Allow footnotes in longtable head/foot
\IfFileExists{footnotehyper.sty}{\usepackage{footnotehyper}}{\usepackage{footnote}}
\makesavenoteenv{longtable}
\usepackage{graphicx}
\makeatletter
\def\maxwidth{\ifdim\Gin@nat@width>\linewidth\linewidth\else\Gin@nat@width\fi}
\def\maxheight{\ifdim\Gin@nat@height>\textheight\textheight\else\Gin@nat@height\fi}
\makeatother
% Scale images if necessary, so that they will not overflow the page
% margins by default, and it is still possible to overwrite the defaults
% using explicit options in \includegraphics[width, height, ...]{}
\setkeys{Gin}{width=\maxwidth,height=\maxheight,keepaspectratio}
% Set default figure placement to htbp
\makeatletter
\def\fps@figure{htbp}
\makeatother

% Preámbulo
\usepackage[utf8]{inputenc} % Codificación de entrada en UTF-8
\usepackage{comment} % Permite comentar secciones del código
\usepackage{marvosym} % Agrega símbolos adicionales
\usepackage{graphicx} % Permite insertar imágenes
%\usepackage[spanish]{babel} % Configuración para escribir en español
%\selectlanguage{spanish} % Selecciona el idioma español
\usepackage{csquotes}
\usepackage[style=apa,sortcites=true,sorting=nyt,backend=biber]{biblatex}
\usepackage{mathptmx} % Fuente de texto matemática
\usepackage{amssymb} % Símbolos adicionales de matemáticas
\usepackage{lipsum} % Crea texto aleatorio
\usepackage{amsthm}
\usepackage{float}
\usepackage{rotating, graphicx}
\usepackage{multirow}
\usepackage{tabularx}
\usepackage{tcolorbox}

\KOMAoption{captions}{tableheading,figureheading}
\makeatletter
\makeatother
\makeatletter
\makeatother
\makeatletter
\@ifpackageloaded{caption}{}{\usepackage{caption}}
\AtBeginDocument{%
\ifdefined\contentsname
  \renewcommand*\contentsname{Tabla de contenidos}
\else
  \newcommand\contentsname{Tabla de contenidos}
\fi
\ifdefined\listfigurename
  \renewcommand*\listfigurename{Listado de Figuras}
\else
  \newcommand\listfigurename{Listado de Figuras}
\fi
\ifdefined\listtablename
  \renewcommand*\listtablename{Listado de Tablas}
\else
  \newcommand\listtablename{Listado de Tablas}
\fi
\ifdefined\figurename
  \renewcommand*\figurename{Figura}
\else
  \newcommand\figurename{Figura}
\fi
\ifdefined\tablename
  \renewcommand*\tablename{Tabla}
\else
  \newcommand\tablename{Tabla}
\fi
}
\@ifpackageloaded{float}{}{\usepackage{float}}
\floatstyle{ruled}
\@ifundefined{c@chapter}{\newfloat{codelisting}{h}{lop}}{\newfloat{codelisting}{h}{lop}[chapter]}
\floatname{codelisting}{Listado}
\newcommand*\listoflistings{\listof{codelisting}{Listado de Listados}}
\makeatother
\makeatletter
\@ifpackageloaded{caption}{}{\usepackage{caption}}
\@ifpackageloaded{subcaption}{}{\usepackage{subcaption}}
\makeatother
\makeatletter
\@ifpackageloaded{tcolorbox}{}{\usepackage[skins,breakable]{tcolorbox}}
\makeatother
\makeatletter
\@ifundefined{shadecolor}{\definecolor{shadecolor}{rgb}{.97, .97, .97}}
\makeatother
\makeatletter
\makeatother
\makeatletter
\makeatother
\ifLuaTeX
\usepackage[bidi=basic]{babel}
\else
\usepackage[bidi=default]{babel}
\fi
\babelprovide[main,import]{spanish}
% get rid of language-specific shorthands (see #6817):
\let\LanguageShortHands\languageshorthands
\def\languageshorthands#1{}
\ifLuaTeX
  \usepackage{selnolig}  % disable illegal ligatures
\fi
\usepackage[]{biblatex}
\addbibresource{../../../../references.bib}
\IfFileExists{bookmark.sty}{\usepackage{bookmark}}{\usepackage{hyperref}}
\IfFileExists{xurl.sty}{\usepackage{xurl}}{} % add URL line breaks if available
\urlstyle{same} % disable monospaced font for URLs
\hypersetup{
  pdftitle={Naturaleza Humana y su impacto político, sesgos científicos. Más allá de los Límites Impuestos},
  pdfauthor={Edison Achalma Mendoza},
  pdflang={es},
  colorlinks=true,
  linkcolor={blue},
  filecolor={Maroon},
  citecolor={Blue},
  urlcolor={Blue},
  pdfcreator={LaTeX via pandoc}}

\title{Naturaleza Humana y su impacto político, sesgos científicos. Más
allá de los Límites Impuestos}
\usepackage{etoolbox}
\makeatletter
\providecommand{\subtitle}[1]{% add subtitle to \maketitle
  \apptocmd{\@title}{\par {\large #1 \par}}{}{}
}
\makeatother
\subtitle{Explorando las implicaciones políticas de nuestras
concepciones sobre la naturaleza humana.}
\author{Edison Achalma Mendoza}
\date{2023-06-09}

\begin{document}
\maketitle
\ifdefined\Shaded\renewenvironment{Shaded}{\begin{tcolorbox}[interior hidden, breakable, borderline west={3pt}{0pt}{shadecolor}, boxrule=0pt, sharp corners, frame hidden, enhanced]}{\end{tcolorbox}}\fi

\hypertarget{entendiendo-a-mariuxe1tegui-una-perspectiva-profunda}{%
\section{Entendiendo a Mariátegui: una perspectiva
profunda}\label{entendiendo-a-mariuxe1tegui-una-perspectiva-profunda}}

Para comprender a Mariátegui, es esencial acercarse a su figura con
respeto y desde una posición de clase clara y precisa. De lo contrario,
resultará imposible apreciar la riqueza y vigencia de su pensamiento.
Aunque Mariátegui falleció hace años, su legado intelectual sigue vivo y
pujante, en contraste con otros pensamientos que, aunque pertenecientes
a personas aún vivas, carecen de vitalidad.

\hypertarget{el-mariuxe1tegui-proletario}{%
\subsection{El Mariátegui
proletario}\label{el-mariuxe1tegui-proletario}}

Un aspecto fundamental para entender a Mariátegui es reconocerlo como un
intelectual proletario. A pesar de las interpretaciones erróneas que se
han difundido, Mariátegui afirmó ser un marxista convicto y confeso, sin
temor ni ambigüedades. Esto implica que Mariátegui adoptó una posición
clara del lado de los explotados, una postura que se tradujo en acción y
escritos comprometidos. Además, Mariátegui desarrolló una concepción del
mundo basada en el marxismo-leninismo, que consideraba la forma más
avanzada de su tiempo. Su filiación con Marx y Lenin era evidente, y
esto se reflejaba en su pensamiento.

\hypertarget{el-muxe9todo-de-mariuxe1tegui-el-materialismo-dialuxe9ctico}{%
\subsection{El método de Mariátegui: el materialismo
dialéctico}\label{el-muxe9todo-de-mariuxe1tegui-el-materialismo-dialuxe9ctico}}

Mariátegui también se destacaba por su método de análisis, basado en el
materialismo dialéctico. Sus trabajos son testimonio fehaciente de esta
influencia. Es fundamental comprender que la posición proletaria de
Mariátegui, su ideología marxista-leninista y su enfoque basado en el
materialismo dialéctico son los pilares para entender su figura. Quienes
no tomen en cuenta estos tres puntos fundamentales no podrán comprender
su pensamiento y, en muchos casos, lo tergiversarán con intenciones
deshonestas.

\hypertarget{el-legado-de-mariuxe1tegui}{%
\subsection{El legado de Mariátegui}\label{el-legado-de-mariuxe1tegui}}

José Carlos Mariátegui fue un destacado marxista-leninista
latinoamericano, una figura en la que debemos sentirnos orgullosos. Su
influencia trasciende nuestras fronteras, aunque lamentablemente en
nuestro país no se le reconoce y valora lo suficiente. Mariátegui no es
un simple repetidor de fórmulas, sino alguien que fusionó el
marxismo-leninismo con la realidad peruana, iluminando nuestro
pensamiento con una vigencia que perdura. Sus ``Siete Ensayos de
Interpretación de la Realidad Peruana'' son un documento inquebrantable
y fundamental.

\hypertarget{enfrentando-las-cruxedticas}{%
\subsection{Enfrentando las
críticas}\label{enfrentando-las-cruxedticas}}

A pesar de los esfuerzos por silenciar, mistificar y tergiversar la
figura de Mariátegui, su influencia se mantiene intacta. Críticos
reaccionarios, como Víctor Andrés Belaúnde, han intentado desacreditarlo
sin éxito. Mariátegui poseía una garra marxista y genial que le permitió
fusionar el marxismo-leninismo con la realidad peruana, algo que pocos
pueden lograr. Aquellos que temen a Mariátegui tienen razones para
hacerlo, ya que su figura representa un criterio fundamental para
distinguir entre auténticos revolucionarios y otros.

\hypertarget{un-libro-inmortal-los-fundamentos-de-mariuxe1tegui}{%
\section{Un libro inmortal: Los fundamentos de
Mariátegui}\label{un-libro-inmortal-los-fundamentos-de-mariuxe1tegui}}

El legado de José Carlos Mariátegui perdura a través de su libro
inmortal. Mientras el trabajo del señor Víctor Andrés Belaúnde ha caído
en el olvido, el librito de Mariátegui sigue vivo, siendo una obra que
merece ser leída tanto por su relevancia histórica como por su visión
popular en nuestra patria. En este texto, exploraremos los puntos clave
y conceptos importantes presentes en la obra de Mariátegui.

\hypertarget{la-visiuxf3n-econuxf3mica-y-estructura-social}{%
\subsection{La visión económica y estructura
social}\label{la-visiuxf3n-econuxf3mica-y-estructura-social}}

Mariátegui nos brinda un análisis fundamental de la economía peruana en
su libro. Comprender la estructura económica de una sociedad y las
relaciones de explotación que existen en ella es vital para comprenderla
en su totalidad. Mariátegui nos muestra que el Perú es un país
semifeudal y semicolonial, y lo respalda con su esquema del proceso
económico nacional. Estas ideas siguen siendo desarrolladas en el
pensamiento marxista peruano actual, bajo la influencia del pensamiento
de Mao.

\hypertarget{la-evoluciuxf3n-de-las-ideas-y-la-literatura-peruana}{%
\subsection{La evolución de las ideas y la literatura
peruana}\label{la-evoluciuxf3n-de-las-ideas-y-la-literatura-peruana}}

Además de analizar las relaciones de explotación en nuestra patria,
Mariátegui también aborda la evolución de las ideas y la literatura en
el Perú. En particular, destaca el problema literario y su carácter
netamente clasista. Estudiar cómo ha evolucionado la literatura peruana
es crucial para comprender su desarrollo histórico. Mariátegui logra
fusionar el marxismo-leninismo con la realidad concreta de nuestra
patria, generando un análisis profundo y realista de la realidad
peruana.

\hypertarget{refutando-las-cruxedticas}{%
\subsection{Refutando las críticas}\label{refutando-las-cruxedticas}}

A lo largo del tiempo, se han hecho intentos de refutar los fundamentos
de Mariátegui, pero ninguno ha tenido éxito. Aquellos que lo critican
suelen hacer esquemas elementales que no pueden igualar el edificio
teórico que él construyó en tan corta edad. Es importante destacar que
las críticas que menosprecian su figura, como las del sujeto Ravines,
carecen de comprensión y reflejan una falta de entendimiento de
Mariátegui y su obra.

\hypertarget{la-importancia-de-la-posiciuxf3n-de-clase-la-ideologuxeda-y-el-muxe9todo}{%
\subsection{La importancia de la posición de clase, la ideología y el
método}\label{la-importancia-de-la-posiciuxf3n-de-clase-la-ideologuxeda-y-el-muxe9todo}}

El problema central no radica en aspectos superficiales o externos, sino
en tres elementos fundamentales en la obra de Mariátegui: su posición de
clase, su ideología y su método. Aquellos que adoptan una posición en
favor del proletariado, el campesinado y las clases explotadas en
nuestro país son quienes tienen la capacidad de comprender a Mariátegui
en su totalidad. Por otro lado, aquellos que se sitúan a medio camino
entre los explotadores y los explotados no podrán entender su
pensamiento. Es por ello que se generan críticas vacías y desinformadas,
que no logran alcanzar la altura de Mariátegui, quien hace más de 30
años dejó una huella imborrable en nuestra historia.

\hypertarget{mariuxe1tegui-un-combatiente-proletario-y-organizador-extraordinario}{%
\section{Mariátegui: Un combatiente proletario y organizador
extraordinario}\label{mariuxe1tegui-un-combatiente-proletario-y-organizador-extraordinario}}

\hypertarget{la-misiuxf3n-de-mariuxe1tegui-y-su-compromiso}{%
\subsection{La misión de Mariátegui y su
compromiso}\label{la-misiuxf3n-de-mariuxe1tegui-y-su-compromiso}}

Mariátegui llegó a nuestra patria desde Europa con una misión clara:
trabajar por la formación del socialismo en el Perú. Él vivió, trabajó y
se desvivió por esta causa, siendo un ferviente defensor de los ideales
proletarios. Su compromiso fue inquebrantable y su columna vertebral
siempre se mantuvo recta, sin ceder a acomodamientos. Mariátegui fue un
combatiente marxista ejemplar y el primer militante de esta ideología en
nuestra patria.

\hypertarget{el-sindicalismo-clasista-y-la-cgtp}{%
\subsection{El sindicalismo clasista y la
CGTP}\label{el-sindicalismo-clasista-y-la-cgtp}}

Mariátegui dejó un legado en la organización del proletariado en nuestro
país. Realizó un trabajo de preparación en el ámbito sindical y sentó
las bases del sindicalismo clasista. Aunque ya existían disputas
sindicales previas en el país, Mariátegui fue el creador de la
Confederación General de Trabajadores del Perú (CGTP) y su principal
ideólogo y mentor. La CGTP fue una institución fundamental para el
movimiento obrero, y Mariátegui fue quien la constituyó orgánicamente y
estableció sus fundamentos y documentos constitutivos.

\hypertarget{la-labor-preparatoria-y-la-estructuraciuxf3n-de-la-cgtp}{%
\subsection{La labor preparatoria y la estructuración de la
CGTP}\label{la-labor-preparatoria-y-la-estructuraciuxf3n-de-la-cgtp}}

Mariátegui comprendió la importancia de la estructuración de una central
sindical para el proletariado. No solo lo comprendió intelectualmente,
sino que también sintió la necesidad de cumplir con la tarea que esta
comprensión le exigía. Llevó a cabo una labor preparatoria para la
formación de la CGTP, siguiendo un enfoque marxista. Cualquier
institución o organismo consta de dos elementos constitutivos: una parte
ideológica, que implica la movilización del pensamiento, la formulación
de un programa y la valoración de un estatuto, y una parte orgánica, que
implica la creación de aparatos organizativos. Mariátegui entendió
profundamente esta dinámica y, siguiendo su enfoque marxista, fue quien
dio vida a la CGTP en nuestra patria.

\hypertarget{la-cgtp-bases-organizativas-y-lucha-proletaria}{%
\section{La CGTP: Bases organizativas y lucha
proletaria}\label{la-cgtp-bases-organizativas-y-lucha-proletaria}}

\hypertarget{la-orientaciuxf3n-clasista-y-proletaria-de-los-estatutos-de-la-cgtp}{%
\subsection{La orientación clasista y proletaria de los estatutos de la
CGTP}\label{la-orientaciuxf3n-clasista-y-proletaria-de-los-estatutos-de-la-cgtp}}

Mariátegui, al redactar los estatutos de la CGTP, creó un documento
sindical que reflejaba una orientación clasista y proletaria. Estos
estatutos, aún vigentes, esperan ver su plena realización. Sin embargo,
es irónico que ciertos elementos posteriores a Mariátegui hayan impuesto
desorientación en el movimiento sindical de nuestro país. Al analizar
los estatutos de la CGTP, encontramos un prólogo u orientación redactada
por Mariátegui, que explica cómo el proletariado ve el mundo y reconoce
la lucha ineludible entre la burguesía y el proletariado. Además,
plantea la importancia de seguir una ideología de clase para la
formación de un organismo sindical.

\hypertarget{bases-generales-de-organizaciuxf3n-y-desarrollo}{%
\subsection{Bases generales de organización y
desarrollo}\label{bases-generales-de-organizaciuxf3n-y-desarrollo}}

Mariátegui estableció bases generales para la constitución orgánica de
la CGTP. No buscaba la rigidez que limita y estanca, sino que
proporcionó lineamientos básicos que permitieran el desarrollo y la
iniciativa del pueblo. Reconoció la importancia de dejar espacio para la
iniciativa individual, fomentando que las personas piensen por sí
mismas, comprendan y aprendan, sin ser perpetuos menores. Su enfoque se
centró en el pueblo, entendiendo que no necesitaba una guía constante,
ya que el pueblo no es ciego. Mariátegui abogaba por bases generales de
organización que dieran autonomía y empoderamiento al pueblo trabajador.

\hypertarget{las-formas-de-lucha-y-la-importancia-de-la-huelga}{%
\subsection{Las formas de lucha y la importancia de la
huelga}\label{las-formas-de-lucha-y-la-importancia-de-la-huelga}}

Mariátegui abordó las formas de lucha en los estatutos de la CGTP,
destacando la importancia de la huelga. Esta elección no fue casual, ya
que consideraba fundamental informar a los trabajadores sobre las
diferentes formas de lucha y su relación con los objetivos que se desean
alcanzar. Es relevante resaltar esto, ya que algunos medios de
comunicación, como La Prensa, han intentado desacreditar la huelga como
un método inadecuado o extremista. Mariátegui entendía que la
movilización de las masas a través de la huelga era esencial para que el
pueblo abriera los ojos, comprendiera la realidad y se liberara de las
ataduras del pasado. La movilización masiva es una herramienta valiosa
que permite al pueblo generar líderes y tomar conciencia de sus
derechos.

\hypertarget{la-propaganda-y-la-agitaciuxf3n-la-voz-propia-del-pueblo}{%
\subsection{La propaganda y la agitación: La voz propia del
pueblo}\label{la-propaganda-y-la-agitaciuxf3n-la-voz-propia-del-pueblo}}

Mariátegui también abordó el tema de la propaganda y la agitación en los
estatutos de la CGTP. Reconoció la necesidad de que el pueblo tenga su
propia voz y exprese sus propias palabras, sin depender de otros para
hacero por él. Mariátegui comprendía que el lenguaje del pueblo no sería
necesariamente sofisticado ni refinado, y que pueden existir errores,
pero lo fundamental es que puedan expresar sus sentimientos, necesidades
y luchar consecuentemente por lo que desean, incluso en caso de derrotas
temporales. Mariátegui planteó la importancia de una prensa proletaria
en el Perú, que aún no se ha logrado en toda su magnitud, y cómo la
propaganda y la agitación son herramientas vitales para el pueblo.

\hypertarget{la-organizaciuxf3n-del-campesinado-seguxfan-josuxe9-carlos-mariuxe1tegui}{%
\section{La organización del campesinado según José Carlos
Mariátegui}\label{la-organizaciuxf3n-del-campesinado-seguxfan-josuxe9-carlos-mariuxe1tegui}}

\hypertarget{la-situaciuxf3n-del-campesinado-peruano-y-la-lucha-contra-la-feudalidad}{%
\subsection{La situación del campesinado peruano y la lucha contra la
feudalidad}\label{la-situaciuxf3n-del-campesinado-peruano-y-la-lucha-contra-la-feudalidad}}

Mariátegui reconoció que en el Perú existían campesinos que sufrían la
opresión de la feudalidad. Identificó al latifundio y la servidumbre
como las expresiones de esta opresión, en la que los campesinos eran
aplastados y explotados. Mariátegui afirmó que el problema fundamental
del campesino peruano era el problema de la tierra y su conquista. Para
ello, era necesario comprender la causa histórica del problema: la
semifeudalidad arraigada en el país.

\hypertarget{formas-organizativas-propuestas-por-mariuxe1tegui}{%
\subsection{Formas organizativas propuestas por
Mariátegui}\label{formas-organizativas-propuestas-por-mariuxe1tegui}}

Mariátegui planteó la importancia de la organización del campesinado
como un paso fundamental para su liberación. Propuso la formación de
sindicatos campesinos y ligas campesinas como formas de organización.
**Reconoció que sin organización, el pueblo es frágil y no puede luchar
efectivamente. Además, Mariátegui destacó la necesidad de construir una
alianza obrero-campesina, entendiendo que esta alianza era fundamental
en cualquier proceso revolucionario.

\textbf{El poder y el papel del campesinado armado}

Mariátegui, siguiendo los principios revolucionarios, comprendió la
importancia del poder en el proceso de cambio. En este sentido, planteó
que el problema de la revolución era el problema del poder. Mariátegui
fue más allá y propuso una medida sorprendente para la organización del
campesinado: el armamento. Consideró necesario organizar la fuerza
armada del campesinado como una forma organizativa esencial. Además,
Mariátegui abogó por la formación de soviets, una estructura de poder
popular que permitiría la participación directa del campesinado en la
toma de decisiones.

\textbf{Conclusión}

José Carlos Mariátegui, en su análisis sobre el campesinado peruano,
planteó la necesidad de su organización para enfrentar la feudalidad y
conquistar la tierra. Propuso la formación de sindicatos y ligas
campesinas, así como la construcción de una alianza obrero-campesina.
Mariátegui también resaltó la importancia del poder y propuso la
organización del campesinado armado, junto con la formación de soviets.
Estas ideas, aunque a menudo ignoradas o tergiversadas, son
fundamentales para comprender el pensamiento revolucionario de
Mariátegui y su visión de la organización del campesinado en la lucha
por la justicia social y la transformación radical.


\printbibliography


\end{document}
