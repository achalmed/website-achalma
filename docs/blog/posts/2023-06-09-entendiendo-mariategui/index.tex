% Options for packages loaded elsewhere
\PassOptionsToPackage{unicode}{hyperref}
\PassOptionsToPackage{hyphens}{url}
\PassOptionsToPackage{dvipsnames,svgnames,x11names}{xcolor}
%
\documentclass[
  letterpaper,
  DIV=11,
  numbers=noendperiod]{scrartcl}

\usepackage{amsmath,amssymb}
\usepackage{iftex}
\ifPDFTeX
  \usepackage[T1]{fontenc}
  \usepackage[utf8]{inputenc}
  \usepackage{textcomp} % provide euro and other symbols
\else % if luatex or xetex
  \usepackage{unicode-math}
  \defaultfontfeatures{Scale=MatchLowercase}
  \defaultfontfeatures[\rmfamily]{Ligatures=TeX,Scale=1}
\fi
\usepackage{lmodern}
\ifPDFTeX\else  
    % xetex/luatex font selection
\fi
% Use upquote if available, for straight quotes in verbatim environments
\IfFileExists{upquote.sty}{\usepackage{upquote}}{}
\IfFileExists{microtype.sty}{% use microtype if available
  \usepackage[]{microtype}
  \UseMicrotypeSet[protrusion]{basicmath} % disable protrusion for tt fonts
}{}
\makeatletter
\@ifundefined{KOMAClassName}{% if non-KOMA class
  \IfFileExists{parskip.sty}{%
    \usepackage{parskip}
  }{% else
    \setlength{\parindent}{0pt}
    \setlength{\parskip}{6pt plus 2pt minus 1pt}}
}{% if KOMA class
  \KOMAoptions{parskip=half}}
\makeatother
\usepackage{xcolor}
\setlength{\emergencystretch}{3em} % prevent overfull lines
\setcounter{secnumdepth}{-\maxdimen} % remove section numbering
% Make \paragraph and \subparagraph free-standing
\ifx\paragraph\undefined\else
  \let\oldparagraph\paragraph
  \renewcommand{\paragraph}[1]{\oldparagraph{#1}\mbox{}}
\fi
\ifx\subparagraph\undefined\else
  \let\oldsubparagraph\subparagraph
  \renewcommand{\subparagraph}[1]{\oldsubparagraph{#1}\mbox{}}
\fi


\providecommand{\tightlist}{%
  \setlength{\itemsep}{0pt}\setlength{\parskip}{0pt}}\usepackage{longtable,booktabs,array}
\usepackage{calc} % for calculating minipage widths
% Correct order of tables after \paragraph or \subparagraph
\usepackage{etoolbox}
\makeatletter
\patchcmd\longtable{\par}{\if@noskipsec\mbox{}\fi\par}{}{}
\makeatother
% Allow footnotes in longtable head/foot
\IfFileExists{footnotehyper.sty}{\usepackage{footnotehyper}}{\usepackage{footnote}}
\makesavenoteenv{longtable}
\usepackage{graphicx}
\makeatletter
\def\maxwidth{\ifdim\Gin@nat@width>\linewidth\linewidth\else\Gin@nat@width\fi}
\def\maxheight{\ifdim\Gin@nat@height>\textheight\textheight\else\Gin@nat@height\fi}
\makeatother
% Scale images if necessary, so that they will not overflow the page
% margins by default, and it is still possible to overwrite the defaults
% using explicit options in \includegraphics[width, height, ...]{}
\setkeys{Gin}{width=\maxwidth,height=\maxheight,keepaspectratio}
% Set default figure placement to htbp
\makeatletter
\def\fps@figure{htbp}
\makeatother

\KOMAoption{captions}{tableheading,figureheading}
\makeatletter
\makeatother
\makeatletter
\makeatother
\makeatletter
\@ifpackageloaded{caption}{}{\usepackage{caption}}
\AtBeginDocument{%
\ifdefined\contentsname
  \renewcommand*\contentsname{Tabla de contenidos}
\else
  \newcommand\contentsname{Tabla de contenidos}
\fi
\ifdefined\listfigurename
  \renewcommand*\listfigurename{Listado de Figuras}
\else
  \newcommand\listfigurename{Listado de Figuras}
\fi
\ifdefined\listtablename
  \renewcommand*\listtablename{Listado de Tablas}
\else
  \newcommand\listtablename{Listado de Tablas}
\fi
\ifdefined\figurename
  \renewcommand*\figurename{Figura}
\else
  \newcommand\figurename{Figura}
\fi
\ifdefined\tablename
  \renewcommand*\tablename{Tabla}
\else
  \newcommand\tablename{Tabla}
\fi
}
\@ifpackageloaded{float}{}{\usepackage{float}}
\floatstyle{ruled}
\@ifundefined{c@chapter}{\newfloat{codelisting}{h}{lop}}{\newfloat{codelisting}{h}{lop}[chapter]}
\floatname{codelisting}{Listado}
\newcommand*\listoflistings{\listof{codelisting}{Listado de Listados}}
\makeatother
\makeatletter
\@ifpackageloaded{caption}{}{\usepackage{caption}}
\@ifpackageloaded{subcaption}{}{\usepackage{subcaption}}
\makeatother
\makeatletter
\@ifpackageloaded{tcolorbox}{}{\usepackage[skins,breakable]{tcolorbox}}
\makeatother
\makeatletter
\@ifundefined{shadecolor}{\definecolor{shadecolor}{rgb}{.97, .97, .97}}
\makeatother
\makeatletter
\makeatother
\makeatletter
\makeatother
\ifLuaTeX
\usepackage[bidi=basic]{babel}
\else
\usepackage[bidi=default]{babel}
\fi
\babelprovide[main,import]{spanish}
% get rid of language-specific shorthands (see #6817):
\let\LanguageShortHands\languageshorthands
\def\languageshorthands#1{}
\ifLuaTeX
  \usepackage{selnolig}  % disable illegal ligatures
\fi
\usepackage[]{biblatex}
\addbibresource{../../../../references.bib}
\IfFileExists{bookmark.sty}{\usepackage{bookmark}}{\usepackage{hyperref}}
\IfFileExists{xurl.sty}{\usepackage{xurl}}{} % add URL line breaks if available
\urlstyle{same} % disable monospaced font for URLs
\hypersetup{
  pdftitle={El legado revolucionario de Mariátegui. Siguiendo su camino hacia un futuro transformado},
  pdfauthor={Edison Achalma Mendoza},
  pdflang={es},
  colorlinks=true,
  linkcolor={blue},
  filecolor={Maroon},
  citecolor={Blue},
  urlcolor={Blue},
  pdfcreator={LaTeX via pandoc}}

\title{El legado revolucionario de Mariátegui. Siguiendo su camino hacia
un futuro transformado}
\usepackage{etoolbox}
\makeatletter
\providecommand{\subtitle}[1]{% add subtitle to \maketitle
  \apptocmd{\@title}{\par {\large #1 \par}}{}{}
}
\makeatother
\subtitle{Explorando el pensamiento y la acción de José Carlos
Mariátegui para inspirar el cambio social}
\author{Edison Achalma Mendoza}
\date{2023-06-09}

\begin{document}
\maketitle
\ifdefined\Shaded\renewenvironment{Shaded}{\begin{tcolorbox}[frame hidden, borderline west={3pt}{0pt}{shadecolor}, boxrule=0pt, enhanced, sharp corners, interior hidden, breakable]}{\end{tcolorbox}}\fi

\hypertarget{entendiendo-a-mariuxe1tegui-una-perspectiva-profunda}{%
\section[Entendiendo a Mariátegui: una perspectiva
profunda]{\texorpdfstring{Entendiendo a Mariátegui: una perspectiva
profunda\footnote{Este blog está inspirado en la conferencia del doctor
  Abimael Guzmán Reynoso dictada en el año 1968 en la Universidad
  Nacional de San Cristóbal de Huamanga de Ayacucho. Si bien el
  contenido se basa en el legado revolucionario de José Carlos
  Mariátegui, reconocemos la influencia de dicha conferencia en la
  exploración de las ideas presentadas.}}{Entendiendo a Mariátegui: una perspectiva profunda}}\label{entendiendo-a-mariuxe1tegui-una-perspectiva-profunda}}

Para comprender a Mariátegui, es esencial acercarse a su figura con
respeto y desde una posición de clase clara y precisa. De lo contrario,
resultará imposible apreciar la riqueza y vigencia de su pensamiento.
Aunque Mariátegui falleció hace años, su legado intelectual sigue vivo y
pujante, en contraste con otros pensamientos que, aunque pertenecientes
a personas aún vivas, carecen de vitalidad.

\hypertarget{el-mariuxe1tegui-proletario}{%
\subsection{El Mariátegui
proletario}\label{el-mariuxe1tegui-proletario}}

Un aspecto fundamental para entender a Mariátegui es reconocerlo como un
intelectual proletario. A pesar de las interpretaciones erróneas que se
han difundido, Mariátegui afirmó ser un marxista convicto y confeso, sin
temor ni ambigüedades. Esto implica que Mariátegui adoptó una posición
clara del lado de los explotados, una postura que se tradujo en acción y
escritos comprometidos. Además, Mariátegui desarrolló una concepción del
mundo basada en el marxismo-leninismo, que consideraba la forma más
avanzada de su tiempo. Su filiación con Marx y Lenin era evidente, y
esto se reflejaba en su pensamiento.

\hypertarget{el-muxe9todo-de-mariuxe1tegui-el-materialismo-dialuxe9ctico}{%
\subsection{El método de Mariátegui: el materialismo
dialéctico}\label{el-muxe9todo-de-mariuxe1tegui-el-materialismo-dialuxe9ctico}}

Mariátegui también se destacaba por su método de análisis, basado en el
materialismo dialéctico. Sus trabajos son testimonio fehaciente de esta
influencia. Es fundamental comprender que la posición proletaria de
Mariátegui, su ideología marxista-leninista y su enfoque basado en el
materialismo dialéctico son los pilares para entender su figura. Quienes
no tomen en cuenta estos tres puntos fundamentales no podrán comprender
su pensamiento y, en muchos casos, lo tergiversarán con intenciones
deshonestas.

\hypertarget{el-legado-de-mariuxe1tegui}{%
\subsection{El legado de Mariátegui}\label{el-legado-de-mariuxe1tegui}}

José Carlos Mariátegui fue un destacado marxista-leninista
latinoamericano, una figura en la que debemos sentirnos orgullosos. Su
influencia trasciende nuestras fronteras, aunque lamentablemente en
nuestro país no se le reconoce y valora lo suficiente. Mariátegui no es
un simple repetidor de fórmulas, sino alguien que fusionó el
marxismo-leninismo con la realidad peruana, iluminando nuestro
pensamiento con una vigencia que perdura. Sus ``Siete Ensayos de
Interpretación de la Realidad Peruana'' son un documento inquebrantable
y fundamental.

\hypertarget{enfrentando-las-cruxedticas}{%
\subsection{Enfrentando las
críticas}\label{enfrentando-las-cruxedticas}}

A pesar de los esfuerzos por silenciar, mistificar y tergiversar la
figura de Mariátegui, su influencia se mantiene intacta. Críticos
reaccionarios, como Víctor Andrés Belaúnde, han intentado desacreditarlo
sin éxito. Mariátegui poseía una garra marxista y genial que le permitió
fusionar el marxismo-leninismo con la realidad peruana, algo que pocos
pueden lograr. Aquellos que temen a Mariátegui tienen razones para
hacerlo, ya que su figura representa un criterio fundamental para
distinguir entre auténticos revolucionarios y otros.

\hypertarget{un-libro-inmortal-los-fundamentos-de-mariuxe1tegui}{%
\section{Un libro inmortal: Los fundamentos de
Mariátegui}\label{un-libro-inmortal-los-fundamentos-de-mariuxe1tegui}}

El legado de José Carlos Mariátegui perdura a través de su libro
inmortal. Mientras el trabajo del señor Víctor Andrés Belaúnde ha caído
en el olvido, el librito de Mariátegui sigue vivo, siendo una obra que
merece ser leída tanto por su relevancia histórica como por su visión
popular en nuestra patria. En este texto, exploraremos los puntos clave
y conceptos importantes presentes en la obra de Mariátegui.

\hypertarget{la-visiuxf3n-econuxf3mica-y-estructura-social}{%
\subsection{La visión económica y estructura
social}\label{la-visiuxf3n-econuxf3mica-y-estructura-social}}

Mariátegui nos brinda un análisis fundamental de la economía peruana en
su libro. Comprender la estructura económica de una sociedad y las
relaciones de explotación que existen en ella es vital para comprenderla
en su totalidad. Mariátegui nos muestra que el Perú es un país
semifeudal y semicolonial, y lo respalda con su esquema del proceso
económico nacional. Estas ideas siguen siendo desarrolladas en el
pensamiento marxista peruano actual, bajo la influencia del pensamiento
de Mao.

\hypertarget{la-evoluciuxf3n-de-las-ideas-y-la-literatura-peruana}{%
\subsection{La evolución de las ideas y la literatura
peruana}\label{la-evoluciuxf3n-de-las-ideas-y-la-literatura-peruana}}

Además de analizar las relaciones de explotación en nuestra patria,
Mariátegui también aborda la evolución de las ideas y la literatura en
el Perú. En particular, destaca el problema literario y su carácter
netamente clasista. Estudiar cómo ha evolucionado la literatura peruana
es crucial para comprender su desarrollo histórico. Mariátegui logra
fusionar el marxismo-leninismo con la realidad concreta de nuestra
patria, generando un análisis profundo y realista de la realidad
peruana.

\hypertarget{refutando-las-cruxedticas}{%
\subsection{Refutando las críticas}\label{refutando-las-cruxedticas}}

A lo largo del tiempo, se han hecho intentos de refutar los fundamentos
de Mariátegui, pero ninguno ha tenido éxito. Aquellos que lo critican
suelen hacer esquemas elementales que no pueden igualar el edificio
teórico que él construyó en tan corta edad. Es importante destacar que
las críticas que menosprecian su figura, como las del sujeto Ravines,
carecen de comprensión y reflejan una falta de entendimiento de
Mariátegui y su obra.

\hypertarget{la-importancia-de-la-posiciuxf3n-de-clase-la-ideologuxeda-y-el-muxe9todo}{%
\subsection{La importancia de la posición de clase, la ideología y el
método}\label{la-importancia-de-la-posiciuxf3n-de-clase-la-ideologuxeda-y-el-muxe9todo}}

El problema central no radica en aspectos superficiales o externos, sino
en tres elementos fundamentales en la obra de Mariátegui: su posición de
clase, su ideología y su método. Aquellos que adoptan una posición en
favor del proletariado, el campesinado y las clases explotadas en
nuestro país son quienes tienen la capacidad de comprender a Mariátegui
en su totalidad. Por otro lado, aquellos que se sitúan a medio camino
entre los explotadores y los explotados no podrán entender su
pensamiento. Es por ello que se generan críticas vacías y desinformadas,
que no logran alcanzar la altura de Mariátegui, quien hace más de 30
años dejó una huella imborrable en nuestra historia.

\hypertarget{mariuxe1tegui-un-combatiente-proletario-y-organizador-extraordinario}{%
\section{Mariátegui: Un combatiente proletario y organizador
extraordinario}\label{mariuxe1tegui-un-combatiente-proletario-y-organizador-extraordinario}}

\hypertarget{la-misiuxf3n-de-mariuxe1tegui-y-su-compromiso}{%
\subsection{La misión de Mariátegui y su
compromiso}\label{la-misiuxf3n-de-mariuxe1tegui-y-su-compromiso}}

Mariátegui llegó a nuestra patria desde Europa con una misión clara:
trabajar por la formación del socialismo en el Perú. Él vivió, trabajó y
se desvivió por esta causa, siendo un ferviente defensor de los ideales
proletarios. Su compromiso fue inquebrantable y su columna vertebral
siempre se mantuvo recta, sin ceder a acomodamientos. Mariátegui fue un
combatiente marxista ejemplar y el primer militante de esta ideología en
nuestra patria.

\hypertarget{el-sindicalismo-clasista-y-la-cgtp}{%
\subsection{El sindicalismo clasista y la
CGTP}\label{el-sindicalismo-clasista-y-la-cgtp}}

Mariátegui dejó un legado en la organización del proletariado en nuestro
país. Realizó un trabajo de preparación en el ámbito sindical y sentó
las bases del sindicalismo clasista. Aunque ya existían disputas
sindicales previas en el país, Mariátegui fue el creador de la
Confederación General de Trabajadores del Perú (CGTP) y su principal
ideólogo y mentor. La CGTP fue una institución fundamental para el
movimiento obrero, y Mariátegui fue quien la constituyó orgánicamente y
estableció sus fundamentos y documentos constitutivos.

\hypertarget{la-labor-preparatoria-y-la-estructuraciuxf3n-de-la-cgtp}{%
\subsection{La labor preparatoria y la estructuración de la
CGTP}\label{la-labor-preparatoria-y-la-estructuraciuxf3n-de-la-cgtp}}

Mariátegui comprendió la importancia de la estructuración de una central
sindical para el proletariado. No solo lo comprendió intelectualmente,
sino que también sintió la necesidad de cumplir con la tarea que esta
comprensión le exigía. Llevó a cabo una labor preparatoria para la
formación de la CGTP, siguiendo un enfoque marxista. Cualquier
institución o organismo consta de dos elementos constitutivos: una parte
ideológica, que implica la movilización del pensamiento, la formulación
de un programa y la valoración de un estatuto, y una parte orgánica, que
implica la creación de aparatos organizativos. Mariátegui entendió
profundamente esta dinámica y, siguiendo su enfoque marxista, fue quien
dio vida a la CGTP en nuestra patria.

\hypertarget{la-cgtp-bases-organizativas-y-lucha-proletaria}{%
\section{La CGTP: Bases organizativas y lucha
proletaria}\label{la-cgtp-bases-organizativas-y-lucha-proletaria}}

\hypertarget{la-orientaciuxf3n-clasista-y-proletaria-de-los-estatutos-de-la-cgtp}{%
\subsection{La orientación clasista y proletaria de los estatutos de la
CGTP}\label{la-orientaciuxf3n-clasista-y-proletaria-de-los-estatutos-de-la-cgtp}}

Mariátegui, al redactar los estatutos de la CGTP, creó un documento
sindical que reflejaba una orientación clasista y proletaria. Estos
estatutos, aún vigentes, esperan ver su plena realización. Sin embargo,
es irónico que ciertos elementos posteriores a Mariátegui hayan impuesto
desorientación en el movimiento sindical de nuestro país. Al analizar
los estatutos de la CGTP, encontramos un prólogo u orientación redactada
por Mariátegui, que explica cómo el proletariado ve el mundo y reconoce
la lucha ineludible entre la burguesía y el proletariado. Además,
plantea la importancia de seguir una ideología de clase para la
formación de un organismo sindical.

\hypertarget{bases-generales-de-organizaciuxf3n-y-desarrollo}{%
\subsection{Bases generales de organización y
desarrollo}\label{bases-generales-de-organizaciuxf3n-y-desarrollo}}

Mariátegui estableció bases generales para la constitución orgánica de
la CGTP. No buscaba la rigidez que limita y estanca, sino que
proporcionó lineamientos básicos que permitieran el desarrollo y la
iniciativa del pueblo. Reconoció la importancia de dejar espacio para la
iniciativa individual, fomentando que las personas piensen por sí
mismas, comprendan y aprendan, sin ser perpetuos menores. Su enfoque se
centró en el pueblo, entendiendo que no necesitaba una guía constante,
ya que el pueblo no es ciego. Mariátegui abogaba por bases generales de
organización que dieran autonomía y empoderamiento al pueblo trabajador.

\hypertarget{las-formas-de-lucha-y-la-importancia-de-la-huelga}{%
\subsection{Las formas de lucha y la importancia de la
huelga}\label{las-formas-de-lucha-y-la-importancia-de-la-huelga}}

Mariátegui abordó las formas de lucha en los estatutos de la CGTP,
destacando la importancia de la huelga. Esta elección no fue casual, ya
que consideraba fundamental informar a los trabajadores sobre las
diferentes formas de lucha y su relación con los objetivos que se desean
alcanzar. Es relevante resaltar esto, ya que algunos medios de
comunicación, como La Prensa, han intentado desacreditar la huelga como
un método inadecuado o extremista. Mariátegui entendía que la
movilización de las masas a través de la huelga era esencial para que el
pueblo abriera los ojos, comprendiera la realidad y se liberara de las
ataduras del pasado. La movilización masiva es una herramienta valiosa
que permite al pueblo generar líderes y tomar conciencia de sus
derechos.

\hypertarget{la-propaganda-y-la-agitaciuxf3n-la-voz-propia-del-pueblo}{%
\subsection{La propaganda y la agitación: La voz propia del
pueblo}\label{la-propaganda-y-la-agitaciuxf3n-la-voz-propia-del-pueblo}}

Mariátegui también abordó el tema de la propaganda y la agitación en los
estatutos de la CGTP. Reconoció la necesidad de que el pueblo tenga su
propia voz y exprese sus propias palabras, sin depender de otros para
hacero por él. Mariátegui comprendía que el lenguaje del pueblo no sería
necesariamente sofisticado ni refinado, y que pueden existir errores,
pero lo fundamental es que puedan expresar sus sentimientos, necesidades
y luchar consecuentemente por lo que desean, incluso en caso de derrotas
temporales. Mariátegui planteó la importancia de una prensa proletaria
en el Perú, que aún no se ha logrado en toda su magnitud, y cómo la
propaganda y la agitación son herramientas vitales para el pueblo.

\hypertarget{la-organizaciuxf3n-del-campesinado-seguxfan-josuxe9-carlos-mariuxe1tegui}{%
\section{La organización del campesinado según José Carlos
Mariátegui}\label{la-organizaciuxf3n-del-campesinado-seguxfan-josuxe9-carlos-mariuxe1tegui}}

\hypertarget{la-situaciuxf3n-del-campesinado-peruano-y-la-lucha-contra-la-feudalidad}{%
\subsection{La situación del campesinado peruano y la lucha contra la
feudalidad}\label{la-situaciuxf3n-del-campesinado-peruano-y-la-lucha-contra-la-feudalidad}}

Mariátegui reconoció que en el Perú existían campesinos que sufrían la
opresión de la feudalidad. Identificó al latifundio y la servidumbre
como las expresiones de esta opresión, en la que los campesinos eran
aplastados y explotados. Mariátegui afirmó que el problema fundamental
del campesino peruano era el problema de la tierra y su conquista. Para
ello, era necesario comprender la causa histórica del problema: la
semifeudalidad arraigada en el país.

\hypertarget{formas-organizativas-propuestas-por-mariuxe1tegui}{%
\subsection{Formas organizativas propuestas por
Mariátegui}\label{formas-organizativas-propuestas-por-mariuxe1tegui}}

Mariátegui planteó la importancia de la organización del campesinado
como un paso fundamental para su liberación. Propuso la formación de
sindicatos campesinos y ligas campesinas como formas de organización.
\textbf{Reconoció que sin organización, el pueblo es frágil y no puede
luchar efectivamente.} Además, Mariátegui destacó la necesidad de
construir una alianza obrero-campesina, entendiendo que esta alianza es
fundamental en cualquier proceso revolucionario.

\hypertarget{el-poder-y-el-papel-del-campesinado-armado}{%
\subsection{El poder y el papel del campesinado
armado}\label{el-poder-y-el-papel-del-campesinado-armado}}

Mariátegui, siguiendo los principios revolucionarios, comprendió la
importancia del poder en el proceso de cambio. En este sentido, planteó
que el problema de la revolución era el problema del poder. Mariátegui
fue más allá y propuso una medida sorprendente para la organización del
campesinado: el armamento. Consideró necesario organizar la fuerza
armada del campesinado como una forma organizativa esencial. Además,
Mariátegui abogó por la formación de soviets, una estructura de poder
popular que permitiría la participación directa del campesinado en la
toma de decisiones.

\hypertarget{la-importancia-del-partido-seguxfan-josuxe9-carlos-mariuxe1tegui}{%
\section{La importancia del Partido según José Carlos
Mariátegui}\label{la-importancia-del-partido-seguxfan-josuxe9-carlos-mariuxe1tegui}}

\hypertarget{el-partido-proletario-como-cerebro-de-la-lucha-revolucionaria}{%
\subsection{El partido proletario como cerebro de la lucha
revolucionaria}\label{el-partido-proletario-como-cerebro-de-la-lucha-revolucionaria}}

Mariátegui reconoció que el proletariado tenía formas de organización
efectivas, como los sindicatos, la alianza obrera y el armamento obrero.
Sin embargo, comprendió que estas estructuras no serían suficientes sin
la existencia de un partido que las guiara. Por lo tanto, Mariátegui
planteó la necesidad de formar un partido proletario en el país.

\hypertarget{aclarando-la-creaciuxf3n-del-partido-comunista-del-peruxfa}{%
\subsection{Aclarando la creación del Partido Comunista del
Perú}\label{aclarando-la-creaciuxf3n-del-partido-comunista-del-peruxfa}}

Es importante desmentir la afirmación errónea de que Mariátegui no creó
el Partido Comunista del Perú, sino el Partido Socialista. Esta
afirmación es una ofensa a la memoria de Mariátegui. Él no era un hombre
sectario ni estrecho en sus ideas. Sin embargo, afirmar que Mariátegui
creó el Partido Socialista y no el comunista es negar su legado y su
compromiso revolucionario. Mariátegui creó el Partido Comunista, que
inicialmente se denominó Partido Socialista debido a su amplitud y
respeto por diversas corrientes ideológicas.

\hypertarget{la-afiliaciuxf3n-a-la-iii-internacional-y-los-principios-leninistas}{%
\subsection{La afiliación a la III Internacional y los principios
leninistas}\label{la-afiliaciuxf3n-a-la-iii-internacional-y-los-principios-leninistas}}

Mariátegui afilió el Partido Socialista a la III Internacional y lo
sometió a los principios planteados por Lenin en ese momento histórico.
Esta afiliación demuestra que Mariátegui no era un ignorante que
confundía los términos. Él fue consciente de la importancia de vincular
su partido con la corriente comunista internacional y adoptar los
principios revolucionarios leninistas.

\hypertarget{la-conspiraciuxf3n-y-el-legado-de-mariuxe1tegui}{%
\subsection{La conspiración y el legado de
Mariátegui}\label{la-conspiraciuxf3n-y-el-legado-de-mariuxe1tegui}}

Existen intentos de conspirar y arrebatarle a Mariátegui su legado
revolucionario. Sin embargo, la figura de Mariátegui sigue siendo
inmensa y su contribución a la formación del socialismo peruano es
innegable. Estamos en un proceso de redescubrimiento de su figura y de
revalorización de su pensamiento.

\hypertarget{somos-los-leguxedtimos-herederos-de-mariuxe1tegui-la-vigencia-de-su-pensamiento}{%
\section{Somos los legítimos herederos de Mariátegui: La vigencia de su
pensamiento}\label{somos-los-leguxedtimos-herederos-de-mariuxe1tegui-la-vigencia-de-su-pensamiento}}

\hypertarget{la-manipulaciuxf3n-de-la-figura-de-mariuxe1tegui}{%
\subsection{La manipulación de la figura de
Mariátegui}\label{la-manipulaciuxf3n-de-la-figura-de-mariuxe1tegui}}

Tras la muerte prematura de Mariátegui, diversos personajes se han
levantado enarbolando su nombre, pero con el propósito de renegar
sistemáticamente de su pensamiento y traicionar su legado. Estos
supuestos herederos de Mariátegui han generado confusión y han utilizado
su figura para ocultar sus propias claudicaciones y traiciones a lo
largo de más de 30 años en el país.

\hypertarget{contradicciones-en-la-interpretaciuxf3n-de-mariuxe1tegui}{%
\subsection{Contradicciones en la interpretación de
Mariátegui}\label{contradicciones-en-la-interpretaciuxf3n-de-mariuxe1tegui}}

Los autodenominados seguidores de Mariátegui sostienen ideas que
contradicen su pensamiento original. Mientras Mariátegui afirmaba que el
Perú era un país semifeudal y semicolonial, estos individuos
descaradamente afirman que el Perú es un país dependiente. Alegan que el
pensamiento de Mariátegui sigue siendo vigente, pero tergiversan su
análisis económico y niegan la realidad de la sociedad peruana.

\hypertarget{la-penetraciuxf3n-del-imperialismo-y-la-veracidad-de-mariuxe1tegui}{%
\subsection{La penetración del imperialismo y la veracidad de
Mariátegui}\label{la-penetraciuxf3n-del-imperialismo-y-la-veracidad-de-mariuxe1tegui}}

Mariátegui preveía que a medida que el imperialismo se infiltrara más en
el país, este se convertiría en una semicolonia y estaría en riesgo de
perder su soberanía de forma definitiva. Si observamos la realidad
actual, es innegable que el imperialismo ha penetrado más en el Perú, lo
que respalda las afirmaciones de Mariátegui. Sus palabras adquieren
mayor relevancia y deben ser tomadas en cuenta.

\hypertarget{la-traiciuxf3n-a-los-principios-de-mariuxe1tegui}{%
\subsection{La traición a los principios de
Mariátegui}\label{la-traiciuxf3n-a-los-principios-de-mariuxe1tegui}}

Mariátegui planteaba la importancia de un frente obrero y campesino, la
necesidad de armar a los trabajadores y la creación de soviets. Sin
embargo, estos supuestos seguidores promueven alianzas con la burguesía
y priorizan las elecciones como medio para acceder al poder. Sus
acciones contradicen directamente los principios revolucionarios de
Mariátegui.

Los individuos que se autodenominan seguidores de Mariátegui, pero
traicionan su pensamiento y tergiversan su legado, son enemigos del
Amauta. Utilizan su figura para fines personales y políticos,
prostituyendo su pensamiento y negando su versión revolucionaria. Es
importante reconocer estas manipulaciones y combatir a quienes se oponen
a Mariátegui y niegan su legado. La figura de Mariátegui debe ser
celebrada y comprendida en su verdadera esencia, como un pensador
revolucionario que dejó una profunda huella en la historia del Perú.

\hypertarget{los-superadores-de-mariuxe1tegui-un-anuxe1lisis-cruxedtico}{%
\section{Los superadores de Mariátegui: Un análisis
crítico}\label{los-superadores-de-mariuxe1tegui-un-anuxe1lisis-cruxedtico}}

\hypertarget{los-superadores-de-mariuxe1tegui-y-sus-muxe9todos-cuestionables}{%
\subsection{Los superadores de Mariátegui y sus métodos
cuestionables:}\label{los-superadores-de-mariuxe1tegui-y-sus-muxe9todos-cuestionables}}

Los críticos de Mariátegui afirman que su obra está desactualizada y que
los avances de la ciencia histórica y los estudios sobre la historia
peruana han dejado obsoleto su pensamiento. Estos ``superadores de
bolsillo'' se han aferrado a la acumulación de datos como un símbolo de
intelectualidad burguesa, creyendo erróneamente que más datos equivalen
a una mejor comprensión de la realidad. Sin embargo, Mariátegui enfatizó
la importancia de la interpretación en su obra ``Siete Ensayos de
Interpretación \ldots{}'', no los ``Siete ensayos de la acumulación de
datos''.

\hypertarget{la-interpretaciuxf3n-desde-una-posiciuxf3n-de-clase-y-el-muxe9todo-materialista-dialuxe9ctico}{%
\subsection{La interpretación desde una posición de clase y el método
materialista
dialéctico:}\label{la-interpretaciuxf3n-desde-una-posiciuxf3n-de-clase-y-el-muxe9todo-materialista-dialuxe9ctico}}

El problema central radica en la falta de comprensión por parte de los
superadores de Mariátegui acerca del conocimiento en la sociedad
burguesa y proletaria. Estos críticos intentan interpretar el Perú desde
una perspectiva marxista, pero sus mentes están atrapadas en
concepciones burguesas, lo que conduce a confusiones y contradicciones
en sus argumentos. Carecen de una base sólida de pensamiento de clase y
del método materialista dialéctico, esencial para comprender
adecuadamente las leyes y dinámicas sociales.

\hypertarget{el-caruxe1cter-capitalista-del-peruxfa-y-la-visiuxf3n-de-mariuxe1tegui}{%
\subsection{El carácter capitalista del Perú y la visión de
Mariátegui:}\label{el-caruxe1cter-capitalista-del-peruxfa-y-la-visiuxf3n-de-mariuxe1tegui}}

Uno de los puntos de desacuerdo más frecuentes es el carácter
capitalista del Perú. Mientras Mariátegui sostenía que el país era
semifeudal, algunos críticos argumentan que ha evolucionado hacia una
economía capitalista. Sin embargo, estos críticos no comprenden el
mecanismo dialéctico de la revolución ni han considerado las enseñanzas
de Lenin sobre el desarrollo desigual y combinado de las fuerzas
productivas.

\hypertarget{la-revoluciuxf3n-peruana-democruxe1tica-nacional-o-socialista}{%
\subsection{La revolución peruana: democrática nacional o
socialista:}\label{la-revoluciuxf3n-peruana-democruxe1tica-nacional-o-socialista}}

Otro punto de conflicto surge en torno a la etapa de la revolución
peruana planteada por Mariátegui. Él sostenía que la primera etapa era
democrática nacional o democrática popular, mientras que los superadores
argumentan que la revolución debe ser socialista desde su inicio. La
falta de un análisis riguroso y documentado por parte de los superadores
deja en evidencia sus inconsistencias y la falta de una comprensión
profunda de las condiciones específicas del país.

\hypertarget{la-visiuxf3n-tergiversada-de-mariuxe1tegui}{%
\subsection{La visión tergiversada de
Mariátegui:}\label{la-visiuxf3n-tergiversada-de-mariuxe1tegui}}

Además de los superadores, existen aquellos que tergiversan parcialmente
las ideas de Mariátegui para respaldar sus propios argumentos. Algunos
interpretan su opinión sobre la religión y el mito como una muestra de
misticismo y humanismo, ignorando su fundamentación marxista y su
enfoque científico.

\hypertarget{estudiar-y-difundir-la-importancia-del-pensamiento-de-mariuxe1tegui-para-los-revolucionarios}{%
\section{Estudiar y difundir: La importancia del pensamiento de
Mariátegui para los
revolucionarios}\label{estudiar-y-difundir-la-importancia-del-pensamiento-de-mariuxe1tegui-para-los-revolucionarios}}

\hypertarget{estudiar-a-mariuxe1tegui-una-tarea-pendiente}{%
\subsection{Estudiar a Mariátegui: Una tarea
pendiente:}\label{estudiar-a-mariuxe1tegui-una-tarea-pendiente}}

El pensamiento de Mariátegui es ampliamente mencionado en nuestro país,
pero lamentablemente es poco leído. Es fundamental realizar un examen
retrospectivo y cuestionarnos si realmente hemos estudiado y comprendido
sus diez obras fundamentales. ¿Conocemos sus planteamientos políticos y
su visión antiimperialista? Es momento de reflexionar y dedicar más
tiempo al estudio de Mariátegui.

\hypertarget{la-importancia-de-difundir-su-pensamiento}{%
\subsection{La importancia de difundir su
pensamiento:}\label{la-importancia-de-difundir-su-pensamiento}}

Mariátegui es una figura luminosa en nuestra historia patria, sin igual
en su talla intelectual. En contraste, personajes como Riva Agüero o
Víctor Andrés Belaúnde carecen de un pensamiento sólido y profundo. Es
imperativo difundir el pensamiento de Mariátegui para contrarrestar las
ideas superficiales y vacías de aquellos que intentan minimizar su
importancia.

\hypertarget{la-defensa-de-mariuxe1tegui}{%
\subsection{La defensa de
Mariátegui:}\label{la-defensa-de-mariuxe1tegui}}

Mariátegui es una fuente de luz que no podemos permitir que sea opacada.
Debemos defenderlo de los ataques abiertos y solapados que recibe. No
podemos permitir que distorsionen sus ideas fundamentales ni que
silencien su voz. Al defender a Mariátegui, estamos defendiendo los
intereses de nuestro pueblo y luchando contra la reacción.

\hypertarget{desarrollar-el-pensamiento-de-mariuxe1tegui}{%
\subsection{Desarrollar el pensamiento de
Mariátegui:}\label{desarrollar-el-pensamiento-de-mariuxe1tegui}}

No se trata de superar a Mariátegui, sino de desarrollar su legado.
Debemos tomar su ideología, su método y sus fuentes como base para
profundizar en los problemas actuales. Por ejemplo, podemos analizar
cómo podemos interpretar la economía peruana de 1968 a la luz de su
ensayo de 1928. Esta tarea no es exclusiva de los intelectuales, sino
que también involucra a los obreros y campesinos, ya que Mariátegui
abordó temas relevantes para ellos en un lenguaje claro y preciso.

\hypertarget{mariuxe1tegui-como-ejemplo}{%
\subsection{Mariátegui como ejemplo:}\label{mariuxe1tegui-como-ejemplo}}

Mariátegui se une a otras grandes figuras de nuestra historia, como
Túpac Amaru, como un ejemplo a seguir. Su legado nos inspira a seguir
luchando por la liberación de nuestro pueblo. Mariátegui no solo
representa una figura histórica, sino también una guía para la acción
revolucionaria en el presente.

\hypertarget{el-legado-ejemplar-de-mariuxe1tegui-un-revolucionario-proletario}{%
\section{El legado ejemplar de Mariátegui: Un revolucionario
proletario}\label{el-legado-ejemplar-de-mariuxe1tegui-un-revolucionario-proletario}}

\hypertarget{mariuxe1tegui-un-referente-histuxf3rico-indiscutible}{%
\subsection{Mariátegui: Un referente histórico
indiscutible:}\label{mariuxe1tegui-un-referente-histuxf3rico-indiscutible}}

José Carlos Mariátegui se ha convertido en una figura histórica
destacada en nuestro país. Aunque su presencia es reciente, su dimensión
histórica es perfecta y sobresaliente. En el ámbito de los ideólogos,
Mariátegui sobrepasa a cualquier otro, dejando en desventaja a los
reaccionarios.

\hypertarget{la-singularidad-de-mariuxe1tegui}{%
\subsection{La singularidad de
Mariátegui:}\label{la-singularidad-de-mariuxe1tegui}}

Mariátegui es una figura única que no surge todos los días. Aunque otros
puedan llevar su apellido, la importancia radica en el ejemplo que él
representa. Debemos elevarlo como un modelo ejemplar, un guía de la
revolución en nuestra patria, que está experimentando profundos cambios
y seguirá transformándose. La historia no puede ser detenida, solo
desviada temporalmente.

\hypertarget{el-ejemplo-de-mariuxe1tegui}{%
\subsection{El ejemplo de
Mariátegui:}\label{el-ejemplo-de-mariuxe1tegui}}

Mariátegui es un ejemplo de revolucionario proletario, sin más ni menos.
No debemos exagerar su importancia, ni quitarle méritos. Al referirnos a
él como ejemplo de revolucionario, simplemente destacamos su cualidad
proletaria. Si eliminamos este adjetivo, Mariátegui perdería su
singularidad y se convertiría en uno más del montón.

\hypertarget{siguiendo-el-camino-de-mariuxe1tegui-un-teoriquito-pequeuxf1ito}{%
\section{Siguiendo el camino de Mariátegui: Un teoriquito
pequeñito}\label{siguiendo-el-camino-de-mariuxe1tegui-un-teoriquito-pequeuxf1ito}}

\hypertarget{el-desarrollo-teuxf3rico-de-mariuxe1tegui}{%
\subsection{El desarrollo teórico de
Mariátegui:}\label{el-desarrollo-teuxf3rico-de-mariuxe1tegui}}

Mariátegui representa un gran teórico del Perú y América Latina. A
través de su labor y vida, encontramos un enfoque marxista-leninista en
el análisis de nuestros problemas. Siguiendo este camino, no buscamos
igualarnos a él, sino más bien transitar por sus pasos. A través de
contribuciones modestas, como un prologuito, podemos convertirnos en
teoriquitos pequeñitos que, al unirse, formarán un gran río de verdad.

\hypertarget{intelectuales-revolucionarios}{%
\subsection{Intelectuales
revolucionarios:}\label{intelectuales-revolucionarios}}

La responsabilidad de continuar este legado recae principalmente en los
intelectuales, pero no en cualquier intelectual, sino en aquellos que
sean revolucionarios. Para ser un intelectual revolucionario, es
necesario fusionarse con las masas explotadas, trabajar y sentir como
ellas. Este proceso implica dejar de lado privilegios y comodidades para
realmente entender y representar los intereses del pueblo.

\hypertarget{la-acciuxf3n-consecuente-de-mariuxe1tegui}{%
\subsection{La acción consecuente de
Mariátegui:}\label{la-acciuxf3n-consecuente-de-mariuxe1tegui}}

Mariátegui se destacó por su coherencia y dedicación. Cuando asumía una
tarea, la cumplía sin dejar de lado sus responsabilidades personales.
Era un combatiente, comprendiendo que ser marxista-leninista implica ser
un agente activo del cambio. Siguiendo su camino, podemos enfrentar
dificultades, pero es posible avanzar y continuar su legado.

Es fundamental entronizar el pensamiento de Mariátegui, defenderlo y
seguir su ejemplo. El destino de nuestro pueblo está en juego, y solo
mediante la adopción de su enfoque teórico y la acción consecuente
podremos avanzar. Siguiendo la senda trazada por Mariátegui, podremos
contribuir, aunque sea modestamente, a la construcción de un futuro
mejor para nuestra patria.


\printbibliography


\end{document}
