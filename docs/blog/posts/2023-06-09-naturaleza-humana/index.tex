% Options for packages loaded elsewhere
\PassOptionsToPackage{unicode}{hyperref}
\PassOptionsToPackage{hyphens}{url}
\PassOptionsToPackage{dvipsnames,svgnames,x11names}{xcolor}
%
\documentclass[
  letterpaper,
  DIV=11,
  numbers=noendperiod]{scrartcl}

\usepackage{amsmath,amssymb}
\usepackage{iftex}
\ifPDFTeX
  \usepackage[T1]{fontenc}
  \usepackage[utf8]{inputenc}
  \usepackage{textcomp} % provide euro and other symbols
\else % if luatex or xetex
  \usepackage{unicode-math}
  \defaultfontfeatures{Scale=MatchLowercase}
  \defaultfontfeatures[\rmfamily]{Ligatures=TeX,Scale=1}
\fi
\usepackage{lmodern}
\ifPDFTeX\else  
    % xetex/luatex font selection
\fi
% Use upquote if available, for straight quotes in verbatim environments
\IfFileExists{upquote.sty}{\usepackage{upquote}}{}
\IfFileExists{microtype.sty}{% use microtype if available
  \usepackage[]{microtype}
  \UseMicrotypeSet[protrusion]{basicmath} % disable protrusion for tt fonts
}{}
\makeatletter
\@ifundefined{KOMAClassName}{% if non-KOMA class
  \IfFileExists{parskip.sty}{%
    \usepackage{parskip}
  }{% else
    \setlength{\parindent}{0pt}
    \setlength{\parskip}{6pt plus 2pt minus 1pt}}
}{% if KOMA class
  \KOMAoptions{parskip=half}}
\makeatother
\usepackage{xcolor}
\setlength{\emergencystretch}{3em} % prevent overfull lines
\setcounter{secnumdepth}{-\maxdimen} % remove section numbering
% Make \paragraph and \subparagraph free-standing
\ifx\paragraph\undefined\else
  \let\oldparagraph\paragraph
  \renewcommand{\paragraph}[1]{\oldparagraph{#1}\mbox{}}
\fi
\ifx\subparagraph\undefined\else
  \let\oldsubparagraph\subparagraph
  \renewcommand{\subparagraph}[1]{\oldsubparagraph{#1}\mbox{}}
\fi


\providecommand{\tightlist}{%
  \setlength{\itemsep}{0pt}\setlength{\parskip}{0pt}}\usepackage{longtable,booktabs,array}
\usepackage{calc} % for calculating minipage widths
% Correct order of tables after \paragraph or \subparagraph
\usepackage{etoolbox}
\makeatletter
\patchcmd\longtable{\par}{\if@noskipsec\mbox{}\fi\par}{}{}
\makeatother
% Allow footnotes in longtable head/foot
\IfFileExists{footnotehyper.sty}{\usepackage{footnotehyper}}{\usepackage{footnote}}
\makesavenoteenv{longtable}
\usepackage{graphicx}
\makeatletter
\def\maxwidth{\ifdim\Gin@nat@width>\linewidth\linewidth\else\Gin@nat@width\fi}
\def\maxheight{\ifdim\Gin@nat@height>\textheight\textheight\else\Gin@nat@height\fi}
\makeatother
% Scale images if necessary, so that they will not overflow the page
% margins by default, and it is still possible to overwrite the defaults
% using explicit options in \includegraphics[width, height, ...]{}
\setkeys{Gin}{width=\maxwidth,height=\maxheight,keepaspectratio}
% Set default figure placement to htbp
\makeatletter
\def\fps@figure{htbp}
\makeatother

\KOMAoption{captions}{tableheading,figureheading}
\makeatletter
\makeatother
\makeatletter
\makeatother
\makeatletter
\@ifpackageloaded{caption}{}{\usepackage{caption}}
\AtBeginDocument{%
\ifdefined\contentsname
  \renewcommand*\contentsname{Tabla de contenidos}
\else
  \newcommand\contentsname{Tabla de contenidos}
\fi
\ifdefined\listfigurename
  \renewcommand*\listfigurename{Listado de Figuras}
\else
  \newcommand\listfigurename{Listado de Figuras}
\fi
\ifdefined\listtablename
  \renewcommand*\listtablename{Listado de Tablas}
\else
  \newcommand\listtablename{Listado de Tablas}
\fi
\ifdefined\figurename
  \renewcommand*\figurename{Figura}
\else
  \newcommand\figurename{Figura}
\fi
\ifdefined\tablename
  \renewcommand*\tablename{Tabla}
\else
  \newcommand\tablename{Tabla}
\fi
}
\@ifpackageloaded{float}{}{\usepackage{float}}
\floatstyle{ruled}
\@ifundefined{c@chapter}{\newfloat{codelisting}{h}{lop}}{\newfloat{codelisting}{h}{lop}[chapter]}
\floatname{codelisting}{Listado}
\newcommand*\listoflistings{\listof{codelisting}{Listado de Listados}}
\makeatother
\makeatletter
\@ifpackageloaded{caption}{}{\usepackage{caption}}
\@ifpackageloaded{subcaption}{}{\usepackage{subcaption}}
\makeatother
\makeatletter
\@ifpackageloaded{tcolorbox}{}{\usepackage[skins,breakable]{tcolorbox}}
\makeatother
\makeatletter
\@ifundefined{shadecolor}{\definecolor{shadecolor}{rgb}{.97, .97, .97}}
\makeatother
\makeatletter
\makeatother
\makeatletter
\makeatother
\ifLuaTeX
\usepackage[bidi=basic]{babel}
\else
\usepackage[bidi=default]{babel}
\fi
\babelprovide[main,import]{spanish}
% get rid of language-specific shorthands (see #6817):
\let\LanguageShortHands\languageshorthands
\def\languageshorthands#1{}
\ifLuaTeX
  \usepackage{selnolig}  % disable illegal ligatures
\fi
\usepackage[]{biblatex}
\addbibresource{../../../../references.bib}
\IfFileExists{bookmark.sty}{\usepackage{bookmark}}{\usepackage{hyperref}}
\IfFileExists{xurl.sty}{\usepackage{xurl}}{} % add URL line breaks if available
\urlstyle{same} % disable monospaced font for URLs
\hypersetup{
  pdftitle={Naturaleza Humana y su impacto político, sesgos científicos. Más allá de los Límites Impuestos},
  pdfauthor={Edison Achalma},
  pdflang={es},
  colorlinks=true,
  linkcolor={blue},
  filecolor={Maroon},
  citecolor={Blue},
  urlcolor={Blue},
  pdfcreator={LaTeX via pandoc}}

\title{Naturaleza Humana y su impacto político, sesgos científicos. Más
allá de los Límites Impuestos}
\usepackage{etoolbox}
\makeatletter
\providecommand{\subtitle}[1]{% add subtitle to \maketitle
  \apptocmd{\@title}{\par {\large #1 \par}}{}{}
}
\makeatother
\subtitle{Explorando las implicaciones políticas de nuestras
concepciones sobre la naturaleza humana.}
\author{Edison Achalma}
\date{2023-06-09}

\begin{document}
\maketitle
\ifdefined\Shaded\renewenvironment{Shaded}{\begin{tcolorbox}[enhanced, sharp corners, interior hidden, borderline west={3pt}{0pt}{shadecolor}, boxrule=0pt, frame hidden, breakable]}{\end{tcolorbox}}\fi

\hypertarget{introducciuxf3n}{%
\section{Introducción}\label{introducciuxf3n}}

\hypertarget{el-ser-humano-es-inherentemente-bueno-o-malo}{%
\subsection{¿El ser humano es inherentemente bueno o
malo?}\label{el-ser-humano-es-inherentemente-bueno-o-malo}}

Rousseau sostenía que los seres humanos somos buenos por
naturaleza,\footnote{La afirmación ``Rousseau sostenía que los seres
  humanos somos buenos por naturaleza'' se basa en las ideas del
  filósofo Jean-Jacques Rousseau, específicamente en su obra ``Discurso
  sobre el origen y los fundamentos de la desigualdad entre los
  hombres'' (1755). En este libro, Rousseau argumenta que el hombre en
  su estado natural es esencialmente bueno, pero que la sociedad y la
  civilización corrompen su bondad innata.} mientras que Thomas Hobbes
afirmaba que somos malos, tan malos que utilizaba la famosa frase
\emph{`el hombre es el lobo del hombre',} sugiriendo que representamos
un peligro para nosotros mismos.\footnote{La afirmación sobre Thomas
  Hobbes y su famosa frase ``el hombre es el lobo del hombre'' se basa
  en las ideas presentadas en su obra principal, ``Leviatán'' (1651). En
  este libro, Hobbes argumenta que los seres humanos tienen una
  naturaleza egoísta y competitiva, y que sin un gobierno fuerte que
  imponga la ley y el orden, la sociedad se sumiría en el caos y la
  violencia.} Por otro lado, \textbf{Marx argumentaba que no existe una
naturaleza humana determinada}, sino que somos producto de un conjunto
de circunstancias y condiciones que nos moldean y modifican nuestra
personalidad y comportamiento.\footnote{La afirmación sobre la visión de
  Marx acerca de la naturaleza humana se encuentra en varias de sus
  obras, pero una de las más relevantes es ``El Manifiesto Comunista''
  (1848). En este texto, Marx y Engels argumentan que la naturaleza
  humana no es fija ni determinada, sino que está moldeada por las
  relaciones sociales y económicas en un determinado sistema.}

Entonces profundizaremos en la importancia de elegir conscientemente
nuestra definición de naturaleza humana y entender las implicaciones de
nuestras creencias. Dependiendo de la idea de naturaleza humana que
adoptemos, aceptaremos un conjunto de creencias que pueden permitir o
limitar ciertas formas de ser, vivir y existir en el mundo.

\hypertarget{definiciones-de-naturaleza-humana}{%
\section{Definiciones de naturaleza
humana}\label{definiciones-de-naturaleza-humana}}

\textbf{¿Qué es la naturaleza humana?}. Existen múltiples respuestas
populares a esta pregunta, y, por supuesto, el contenido de la
definición varía según nuestras creencias y nuestra cultura.

\hypertarget{respuestas-metafuxedsicas-y-antimetafuxedsicas}{%
\subsection{Respuestas metafísicas y
antimetafísicas}\label{respuestas-metafuxedsicas-y-antimetafuxedsicas}}

Aquellos que creen en la existencia de una naturaleza humana la
justifican con argumentos no empíricos, es decir, con cosas que no
pueden ser comprobadas en la realidad. Por ejemplo, los argumentos de
Rousseau y Hobbes son de naturaleza metafísica, ya que no pueden ser
respaldados por estadísticas o estudios biológicos. Por lo tanto, si
queremos creer en la existencia de una naturaleza humana, ya sea buena o
mala, debemos conformarnos con elegir el argumento \textbf{metafísico}
más convincente, aunque no podamos comprobarlo en la realidad.

Existen otras formas de responder a la pregunta sobre la naturaleza
humana, como la versión \textbf{antimetafísica}, que se asemeja más a la
respuesta que nos da Marx. En este caso, vamos a explorar la respuesta
que ofrece David Hume, quien, precisamente para oponerse a los
argumentos metafísicos sobre la naturaleza humana, decidió describir
nuestro aparato cognitivo y emocional. Observó al ser humano, recopiló
datos de la realidad y dedujo conclusiones lógicas. A partir de esta
descripción de cómo pensamos y sentimos, Hume define la naturaleza
humana como ese conjunto de cualidades, descripciones y condiciones.
Esto implica que nuestras capacidades intelectuales y afectivas son las
que determinan nuestra naturaleza humana, y a su vez, las experiencias a
las que estamos expuestos modifican estas capacidades tanto
intelectuales como afectivas.\footnote{La visión de David Hume sobre la
  naturaleza humana se sustenta en su obra principal, el ``Tratado de la
  naturaleza humana'' (1739-1740). En este tratado, Hume lleva a cabo un
  análisis empírico de la mente humana, examinando detalladamente los
  procesos de pensamiento, sentimiento y percepción que caracterizan
  nuestra experiencia del mundo. A partir de esta rigurosa
  investigación, Hume concluye que la naturaleza humana se comprende
  como un conjunto de cualidades, descripciones y condiciones, las
  cuales son moldeadas por nuestras capacidades intelectuales y
  emocionales, así como por las experiencias a las que nos vemos
  expuestos.}

\hypertarget{la-perspectiva-de-david-hume-y-karl-marx}{%
\subsection{La perspectiva de David Hume y Karl
Marx}\label{la-perspectiva-de-david-hume-y-karl-marx}}

Las respuestas de Marx y Hume son anti-metafísicas, ya que se basan en
hechos que podemos más o menos comprobar en la realidad, en hechos
humanos o sociales. Por otro lado, las respuestas de Rousseau y Hobbes
son hipotéticas, ya que se basan en reflexiones e ideas hipotéticas.

Hobbes parte de la hipotética idea de un estado de naturaleza que se
asemeja a una guerra en la que todos peleamos contra todos. Por otro
lado, Rousseau parte de la hipotética idea del buen salvaje, imaginando
a los primeros pobladores como buenos salvajes. Según Rousseau, es la
sociedad la que nos corrompe y provoca que seamos menos buenas personas,
pero esta corrupción es producto de la sociedad, no de nuestra
naturaleza. Estos argumentos metafísicos se basan en cosas que no
podemos comprobar en la realidad.

En contraste, las respuestas de Marx y Hume podríamos llamarlas más
realistas, aunque el término correcto sería anti-metafísicas. Estas
respuestas intentan basarse en hechos de nuestra realidad. En el caso de
Hume, nos quedamos con la idea de que los seres humanos son el resultado
de nuestras capacidades intelectuales y afectivas, lo que determina
nuestra naturaleza.

\hypertarget{la-visiuxf3n-de-sigmund-freud}{%
\subsection{La visión de Sigmund
Freud}\label{la-visiuxf3n-de-sigmund-freud}}

Sigmund Freud comparte esta idea con David Hume y Karl Marx, aunque con
algunas variantes. Freud nos dice que todos somos víctimas de la
naturaleza humana y que somos el resultado de nuestros traumas y deseos
reprimidos. Según Freud, somos agentes pasivos en nuestras vidas, donde
nuestras experiencias determinan absolutamente todo lo que somos. Desde
este punto de vista, no tenemos mucha participación en la construcción
de nuestra identidad, lo cual tampoco es una opción muy favorable, ya
que no podemos hacernos cargo de nosotros mismos.\footnote{La
  perspectiva de Sigmund Freud sobre la naturaleza humana se encuentra
  en su obra fundamental ``La interpretación de los sueños'' (1899). En
  este libro, Freud explora los procesos inconscientes y los mecanismos
  psicológicos que influyen en nuestra vida mental y emocional. Según
  Freud, nuestra naturaleza humana está determinada por los traumas y
  los deseos reprimidos que existen en nuestro inconsciente. Desde esta
  perspectiva, se considera que somos víctimas de nuestra propia
  naturaleza, y nuestras experiencias juegan un papel fundamental en la
  configuración de nuestra identidad. Freud sostiene que, en gran
  medida, somos agentes pasivos en la determinación de nuestras vidas y
  que carecemos de control total sobre nosotros mismos.}

\begin{quote}
Lo importante de este tipo de pensamiento anti-metafísico es que nos
permite definir la naturaleza humana sin necesidad de determinar ninguna
esencia, lo cual es positivo.
\end{quote}

\hypertarget{implicaciones-de-las-creencias-sobre-la-naturaleza-humana}{%
\section{Implicaciones de las creencias sobre la naturaleza
humana:}\label{implicaciones-de-las-creencias-sobre-la-naturaleza-humana}}

\hypertarget{luxedmites-y-convenciones-de-las-esencias-humanas}{%
\subsection{Límites y convenciones de las esencias
humanas}\label{luxedmites-y-convenciones-de-las-esencias-humanas}}

Observemos que cuando Rousseau y Hobbes determinaron la naturaleza
humana como buena o mala, estaban estableciendo una esencia. Si creemos
en Hobbes, afirmamos que somos esencialmente malos y estamos inclinados
naturalmente a hacer daño a otros seres vivos. Por otro lado, si creemos
en Rousseau, afirmamos que nuestra naturaleza es esencialmente buena y,
por lo tanto, nos inclinamos a realizar actos buenos. Si reflexionamos
sobre estas afirmaciones, nos damos cuenta de que al determinar una
esencia, estamos definiendo cómo debe ser una cosa o una persona,
imponiendo límites y delimitando la capacidad de acción. Es importante
recordar que todo límite es una convención y no debemos olvidarlo.

\hypertarget{justificaciuxf3n-de-acciones-y-responsabilidad-uxe9tica}{%
\subsection{Justificación de acciones y responsabilidad
ética}\label{justificaciuxf3n-de-acciones-y-responsabilidad-uxe9tica}}

Por ejemplo, si considero que la naturaleza humana o la naturaleza de
los seres humanos es mala, juzgaré mis actos malos como una inclinación
natural en mí. Por lo tanto, podré justificarlos con menos
responsabilidad ética, lo que limitará mi capacidad para evitar las
malas acciones. Me veré obligado a realizarlas por naturaleza. Este
escenario es desfavorable para aquellos de nosotros que creemos en
nuestra capacidad para limitar efectivamente nuestros impulsos y
reacciones ante nuestras emociones, sensaciones y sentimientos.

Por tanto, dependiendo de la definición de la naturaleza humana que
decidamos creer, delimitamos nuestras posibilidades de ser y actuar en
el mundo.

\begin{quote}
Recordemos que la respuesta metafísica sostiene que el ser humano tiene
una esencia buena o mala, mientras que la respuesta anti-metafísica
afirma que no tenemos una esencia, sino que somos el resultado de todo
lo que nos sucede en la vida.
\end{quote}

\hypertarget{importancia-de-elegir-una-definiciuxf3n-que-aumente-nuestras-capacidades.}{%
\subsection{Importancia de elegir una definición que aumente nuestras
capacidades.}\label{importancia-de-elegir-una-definiciuxf3n-que-aumente-nuestras-capacidades.}}

Personalmente, creo que debemos elegir el concepto de naturaleza humana
que nos brinde mayor potencia, es decir, que incremente nuestras
capacidades de ser y actuar en lugar de limitarlas. Por lo tanto, es
crucial preguntarnos qué idea estoy aceptando como mi naturaleza humana,
ya que estas ideas que formulamos como ``naturales'' en nosotros o en
nuestra especie son las que delimitan nuestras posibilidades de ser y
actuar.

\hypertarget{influencia-en-la-vida-cotidiana-y-las-relaciones-sociales}{%
\subsection{Influencia en la vida cotidiana y las relaciones
sociales}\label{influencia-en-la-vida-cotidiana-y-las-relaciones-sociales}}

Por ejemplo, si considero que la naturaleza humana implica diferencias
de género justificadas por la división de sexos, como la creencia de que
los hombres son menos sensibles que las mujeres y que las mujeres son
menos racionales que los hombres, delimito mi potencialidad de ser.
Dependiendo del sexo con el que nazca, si naciera mujer y considerara
verdadera la afirmación anterior, me sentiría y actuaría como menos
racional que todos los hombres. Tomaría mis decisiones de vida basándome
en mis pasiones y emociones, sin tener claridad en mis razones, porque
mi naturaleza sería ser sensible y no racional. Por otro lado, si
naciera hombre y considerara que por naturaleza los hombres no somos
sensibles, sino racionales, viviría toda mi vida sin prestar atención a
mis emociones, reprimiendo mis sentimientos para demostrar mi supuesta
superioridad racional y ser congruente con mi propia idea de mi
naturaleza.

En cualquier caso, si justifico mi naturaleza y la de los demás a partir
de estas creencias, mi vida está limitada y restringida por ellas.
Ninguno de los dos escenarios es realmente deseable: ser una mujer
irracional o un hombre insensible. Ambos privan a cada género de una
parte muy importante de la vida.

\hypertarget{integraciuxf3n-de-pensamiento-y-sentimiento}{%
\subsection{Integración de pensamiento y
sentimiento}\label{integraciuxf3n-de-pensamiento-y-sentimiento}}

Sabemos que todos los seres humanos tenemos el potencial de desarrollar
y mejorar tanto nuestra racionalidad como nuestros sentimientos, y no
tenemos por qué elegir uno u otro. De hecho, podríamos afirmar que esta
separación entre sentimientos y pensamientos no es tan clara como nos
gusta pensar. Los sentimientos generan pensamientos y viceversa. Existe
un término muy hermoso, acuñado por el sociólogo Fals Borda, que es
``sentipensar''. Significa precisamente estas acciones, ideas y
emociones que se entrelazan entre pensamientos y sentimientos. Esta idea
del sentir pensante se adopta cuando Fals Borda entrevista a unos
pescadores colombianos y les pregunta algo como: ``¿Cuándo deciden salir
a pescar?''. La respuesta de ellos es: ``Los sentipensamos, sentimos y
pensamos al mismo tiempo, y cuando estamos de acuerdo en las emociones y
en los pensamientos, ahí vamos a pescar''.

Es interesante comenzar a integrar este tipo de términos, como el
sentipensar, en nuestra vida cotidiana cuando no tenemos tan clara esta
división entre el pensamiento y los sentimientos. Esta visión es muy
cuestionable y parece ser incorrecta, aunque haya sido así en el
desarrollo de la humanidad.

\hypertarget{cuestionamiento-de-las-afirmaciones-sobre-la-naturaleza-humana}{%
\subsection{Cuestionamiento de las afirmaciones sobre la naturaleza
humana}\label{cuestionamiento-de-las-afirmaciones-sobre-la-naturaleza-humana}}

Ahora, imagina que considero que la naturaleza humana implica que las
personas son inherentemente egoístas y solo se preocupan por su propio
beneficio. Si adopto esta creencia como verdadera, entonces justificaré
mis acciones egoístas como algo natural y esperado. Esto limitará mi
capacidad para actuar de manera altruista o preocuparme por el bienestar
de los demás, ya que me sentiré obligado a actuar de acuerdo con mi
supuesta naturaleza egoísta. En consecuencia, estaré coartado por esta
idea y no podré experimentar el pleno potencial de ser una persona
generosa y empática.

Sigamos con los ejemplos. Si considero, por ejemplo, que todos los seres
humanos somos malos y pecadores por naturaleza, estoy asumiendo que las
personas toman malas decisiones de manera inherente, sin importar cuáles
sean los pecados en cuestión. Esto implica que todos estamos alejados
del bien debido a nuestra condición de pecadores, según la teoría, y que
esto es parte de nuestra naturaleza. Este tipo de argumentación nos
lleva a justificar nuestras malas acciones y las de los demás como algo
natural e inevitable, lo que nos exime en gran medida de la
responsabilidad ética de nuestros actos. Comienza un juego de
autojustificación a nivel personal.

Estos argumentos sobre la naturaleza humana pueden afectarnos en
múltiples formas. Por un lado, pueden llevarnos a llevar una vida
construida en torno a evitar el pecado a toda costa, generando un
comportamiento represivo hacia nosotros mismos y hacia los demás
pecadores. Esto puede conducir a altos niveles de violencia, ya que
podemos justificar actos violentos como inevitables debido a nuestra
supuesta naturaleza violenta. Por ejemplo, \textbf{podemos llegar a
justificar el maltrato a niños o animales con el fin de educarlos, entre
otras cosas aún peores.} Esta justificación se basa en la supuesta
presencia de violencia en nuestra naturaleza.

\begin{quote}
Es importante señalar que el término \textbf{``violencia''} es complejo
y merece una reflexión más profunda. No toda violencia es necesariamente
negativa, y su papel en nuestra vida cotidiana es un tema que merece ser
explorado en futuras discusiones.
\end{quote}

A nivel social, este tipo de justificaciones basadas en la idea de que
la naturaleza humana es mala pueden llevarnos a justificar altos niveles
de represión social en aras de mantener el orden. Según esta
perspectiva, el Estado tendría la obligación de protegernos de todas las
amenazas, incluso si la amenaza somos nosotros mismos. Así, se pueden
tomar decisiones que limiten nuestras libertades individuales en pos del
supuesto bienestar de los ciudadanos. Thomas Hobbes, por ejemplo,
utiliza la noción de que los seres humanos son malvados y ambiciosos
para justificar un sistema político conocido como monarquía absoluta, en
el cual se limita, mediante la fuerza, la naturaleza malvada de las
personas. Es importante observar cómo se utilizan estas argumentaciones
sobre la naturaleza humana en el ámbito político, ya que pueden moldear
la organización y el orden social en el que vivimos. Dependiendo de la
perspectiva que aceptemos, se puede favorecer un sistema con mayor
represión o uno que garantice mayores libertades.

Es crucial cuestionar estas afirmaciones sobre la naturaleza humana,
como la de Thomas Hobbes, y plantearnos si realmente somos capaces de
modificar nuestra supuesta naturaleza malvada, o si somos totalmente
incapaces de hacerlo. También debemos cuestionar si esta es
verdaderamente nuestra naturaleza, ya que hemos visto que existen
diversas respuestas posibles que difieren de las afirmaciones de Hobbes.

\hypertarget{implicaciones-poluxedticas-y-el-significado-de-lo-poluxedtico-en-la-filosofuxeda}{%
\subsection{Implicaciones políticas y el significado de lo político en
la
filosofía}\label{implicaciones-poluxedticas-y-el-significado-de-lo-poluxedtico-en-la-filosofuxeda}}

Es importante recordar que al definir la naturaleza humana, también nos
estamos definiendo a nosotros mismos. Elegir una perspectiva sobre la
naturaleza humana implica tomar una posición política. No existe ninguna
definición de naturaleza humana que no tenga implicaciones políticas.
Sería interesante dedicar un artículo completo a explorar el significado
de lo político en la filosofía, ya que va mucho más allá de la política
que conocemos.

Debemos comprender que lo político es mucho más profundo que la simple
noción de participar en elecciones, tener partidos políticos o ver
políticos haciendo promesas en la televisión. En términos filosóficos,
lo político implica que cuando una libertad entra en contacto con otra,
se crea una situación política. En ese momento, debemos mediar, negociar
y aprender a establecer límites. Lo político atraviesa todos los ámbitos
de nuestra vida. No debemos limitarlo a la idea tradicional de la
política pública o social. Es un concepto mucho más amplio y esencial en
nuestras interacciones cotidianas.

Por ahora, dejemos aquí esta reflexión sobre lo político. En futuros
publicaciones, profundizaremos más en este tema y exploraremos su
importancia en nuestras vidas.

\textbf{¿Por qué elegir nuestra concepción de la naturaleza humana se
convierte en un posicionamiento político?} La elección de nuestra
naturaleza humana implica establecer límites a nuestra propia
naturaleza, y es en este punto donde surge la noción de lo político. Por
ejemplo, al seleccionar una definición de naturaleza humana, se generan
sesgos interpretativos en ámbitos que se consideran objetivos, como la
ciencia. Nuestra concepción de la naturaleza humana puede influir en la
interpretación de los resultados de investigaciones científicas.

\hypertarget{implicaciones-cientuxedficas}{%
\subsection{Implicaciones
científicas}\label{implicaciones-cientuxedficas}}

Un ejemplo de sesgo interpretativo se puede encontrar en el trabajo del
neurocientífico Broca, descrito en su libro ``El cerebro de Broca''. En
su estudio, Broca comparó 292 cerebros masculinos con 140 cerebros
femeninos y encontró que el cerebro de las mujeres pesaba, en promedio,
181 gramos menos que el de los hombres. Sin embargo, Broca interpretó
esta diferencia en masa cerebral como evidencia de la supuesta
inferioridad de las mujeres, sin tener en cuenta que la variación podría
explicarse fácilmente por diferencias de estatura y tamaño corporal.
Broca no intentó contextualizar estos datos ni cuestionar su propia
conclusión, sino que afirmó que esto confirmaba las diferencias de
género, llegando incluso a declarar que ``las mujeres son menos
inteligentes que los hombres''

Este ejemplo ilustra cómo la percepción individual de la naturaleza
humana puede influir en nuestros actos, conclusiones y experimentos en
el mundo. La ciencia tampoco está exenta de esta influencia
interpretativa. Aunque posee un grado de objetividad, está sujeta a
métodos y formas interpretativas que permiten la apertura a la
subjetividad. Por lo tanto, es importante reconocer que incluso la
ciencia interpreta el mundo a partir de nuestras concepciones sobre la
naturaleza humana.

El caso de Broca demuestra cómo nuestra noción de la naturaleza humana
moldea y perfila nuestras acciones y conclusiones. En su caso, la
concepción de la naturaleza humana implicaba que los hombres eran más
inteligentes que las mujeres, y utilizó los datos de su experimento para
afirmar y reafirmar esta creencia. No se le ocurrió cuestionar la
veracidad de la afirmación de que las mujeres son menos inteligentes que
los hombres, ya que esta idea le parecía natural y encajaba con su
concepción de la naturaleza. No consideró que la diferencia en el tamaño
cerebral pudiera estar relacionada con la complexión física de las
personas en lugar de ser determinada únicamente por el género.

\begin{quote}
Esto nos muestra que incluso la ciencia, que a menudo se considera el
lenguaje más objetivo, interpreta el mundo a partir de nuestras
concepciones de la naturaleza humana. Es esencial reconocer que nuestras
creencias sobre la naturaleza humana pueden influir en nuestra
comprensión de los resultados científicos y en nuestras acciones en el
mundo.
\end{quote}

En la actualidad, podemos hablar del neurosexismo, un tipo de sesgo
ideológico que incluso existe dentro de la neurología, la ciencia y la
medicina, así como en muchos otros ámbitos. Gracias a este tipo de
análisis, podemos afirmar con total certeza que ninguno de los dos sexos
es más inteligente o sensible que el otro. Si están interesados en
profundizar en este tema del neurosexismo, les recomiendo un libro muy
importante llamado ``El género del cerebro'' de Gina Ripo. Es un
análisis profundo realizado por una destacada neuróloga que observa cómo
gran parte de la historia de la neurociencia ha estado influenciada por
cuestiones de género.

\hypertarget{conclusiuxf3n}{%
\section{Conclusión}\label{conclusiuxf3n}}

A partir de los ejemplos que acabamos de mencionar, podemos observar que
la definición de la naturaleza humana que adoptamos de manera personal,
así como la definición que adopta nuestra sociedad y cultura, establecen
los límites de lo que podemos ser y lo que nos permitimos ser. Nuestras
creencias personales y las demandas y expectativas sociales influyen en
nuestras acciones y en la forma en que nos desenvolvemos en la vida.
Detrás de todos estos límites impuestos por la sociedad y por nosotros
mismos se encuentra la concepción de la naturaleza humana.

Por lo tanto, es de vital importancia revisar nuestras creencias, ya que
moldean nuestra vida y la hacen posible. Es saludable asumir la
responsabilidad de nuestras propias creencias, es decir,
\textbf{comprender por qué creemos lo que creemos y cómo justificamos
esas creencias.} Esto nos permitirá tomar un mayor control sobre nuestra
propia vida y actuar de manera más consciente.


\printbibliography


\end{document}
