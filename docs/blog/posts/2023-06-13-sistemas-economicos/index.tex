% Options for packages loaded elsewhere
\PassOptionsToPackage{unicode}{hyperref}
\PassOptionsToPackage{hyphens}{url}
\PassOptionsToPackage{dvipsnames,svgnames,x11names}{xcolor}
%
\documentclass[
  letterpaper,
  DIV=11,
  numbers=noendperiod]{scrartcl}

\usepackage{amsmath,amssymb}
\usepackage{iftex}
\ifPDFTeX
  \usepackage[T1]{fontenc}
  \usepackage[utf8]{inputenc}
  \usepackage{textcomp} % provide euro and other symbols
\else % if luatex or xetex
  \usepackage{unicode-math}
  \defaultfontfeatures{Scale=MatchLowercase}
  \defaultfontfeatures[\rmfamily]{Ligatures=TeX,Scale=1}
\fi
\usepackage{lmodern}
\ifPDFTeX\else  
    % xetex/luatex font selection
\fi
% Use upquote if available, for straight quotes in verbatim environments
\IfFileExists{upquote.sty}{\usepackage{upquote}}{}
\IfFileExists{microtype.sty}{% use microtype if available
  \usepackage[]{microtype}
  \UseMicrotypeSet[protrusion]{basicmath} % disable protrusion for tt fonts
}{}
\makeatletter
\@ifundefined{KOMAClassName}{% if non-KOMA class
  \IfFileExists{parskip.sty}{%
    \usepackage{parskip}
  }{% else
    \setlength{\parindent}{0pt}
    \setlength{\parskip}{6pt plus 2pt minus 1pt}}
}{% if KOMA class
  \KOMAoptions{parskip=half}}
\makeatother
\usepackage{xcolor}
\setlength{\emergencystretch}{3em} % prevent overfull lines
\setcounter{secnumdepth}{-\maxdimen} % remove section numbering
% Make \paragraph and \subparagraph free-standing
\ifx\paragraph\undefined\else
  \let\oldparagraph\paragraph
  \renewcommand{\paragraph}[1]{\oldparagraph{#1}\mbox{}}
\fi
\ifx\subparagraph\undefined\else
  \let\oldsubparagraph\subparagraph
  \renewcommand{\subparagraph}[1]{\oldsubparagraph{#1}\mbox{}}
\fi


\providecommand{\tightlist}{%
  \setlength{\itemsep}{0pt}\setlength{\parskip}{0pt}}\usepackage{longtable,booktabs,array}
\usepackage{calc} % for calculating minipage widths
% Correct order of tables after \paragraph or \subparagraph
\usepackage{etoolbox}
\makeatletter
\patchcmd\longtable{\par}{\if@noskipsec\mbox{}\fi\par}{}{}
\makeatother
% Allow footnotes in longtable head/foot
\IfFileExists{footnotehyper.sty}{\usepackage{footnotehyper}}{\usepackage{footnote}}
\makesavenoteenv{longtable}
\usepackage{graphicx}
\makeatletter
\def\maxwidth{\ifdim\Gin@nat@width>\linewidth\linewidth\else\Gin@nat@width\fi}
\def\maxheight{\ifdim\Gin@nat@height>\textheight\textheight\else\Gin@nat@height\fi}
\makeatother
% Scale images if necessary, so that they will not overflow the page
% margins by default, and it is still possible to overwrite the defaults
% using explicit options in \includegraphics[width, height, ...]{}
\setkeys{Gin}{width=\maxwidth,height=\maxheight,keepaspectratio}
% Set default figure placement to htbp
\makeatletter
\def\fps@figure{htbp}
\makeatother

\KOMAoption{captions}{tableheading,figureheading}
\makeatletter
\makeatother
\makeatletter
\makeatother
\makeatletter
\@ifpackageloaded{caption}{}{\usepackage{caption}}
\AtBeginDocument{%
\ifdefined\contentsname
  \renewcommand*\contentsname{Tabla de contenidos}
\else
  \newcommand\contentsname{Tabla de contenidos}
\fi
\ifdefined\listfigurename
  \renewcommand*\listfigurename{Listado de Figuras}
\else
  \newcommand\listfigurename{Listado de Figuras}
\fi
\ifdefined\listtablename
  \renewcommand*\listtablename{Listado de Tablas}
\else
  \newcommand\listtablename{Listado de Tablas}
\fi
\ifdefined\figurename
  \renewcommand*\figurename{Figura}
\else
  \newcommand\figurename{Figura}
\fi
\ifdefined\tablename
  \renewcommand*\tablename{Tabla}
\else
  \newcommand\tablename{Tabla}
\fi
}
\@ifpackageloaded{float}{}{\usepackage{float}}
\floatstyle{ruled}
\@ifundefined{c@chapter}{\newfloat{codelisting}{h}{lop}}{\newfloat{codelisting}{h}{lop}[chapter]}
\floatname{codelisting}{Listado}
\newcommand*\listoflistings{\listof{codelisting}{Listado de Listados}}
\makeatother
\makeatletter
\@ifpackageloaded{caption}{}{\usepackage{caption}}
\@ifpackageloaded{subcaption}{}{\usepackage{subcaption}}
\makeatother
\makeatletter
\@ifpackageloaded{tcolorbox}{}{\usepackage[skins,breakable]{tcolorbox}}
\makeatother
\makeatletter
\@ifundefined{shadecolor}{\definecolor{shadecolor}{rgb}{.97, .97, .97}}
\makeatother
\makeatletter
\makeatother
\makeatletter
\makeatother
\ifLuaTeX
\usepackage[bidi=basic]{babel}
\else
\usepackage[bidi=default]{babel}
\fi
\babelprovide[main,import]{spanish}
% get rid of language-specific shorthands (see #6817):
\let\LanguageShortHands\languageshorthands
\def\languageshorthands#1{}
\ifLuaTeX
  \usepackage{selnolig}  % disable illegal ligatures
\fi
\usepackage[]{biblatex}
\addbibresource{../../../../references.bib}
\IfFileExists{bookmark.sty}{\usepackage{bookmark}}{\usepackage{hyperref}}
\IfFileExists{xurl.sty}{\usepackage{xurl}}{} % add URL line breaks if available
\urlstyle{same} % disable monospaced font for URLs
\hypersetup{
  pdftitle={Introducción a los Sistemas Económicos. Cómo se distribuyen los recursos y se producen},
  pdfauthor={Achalma Mendoza Edison},
  pdflang={es},
  colorlinks=true,
  linkcolor={blue},
  filecolor={Maroon},
  citecolor={Blue},
  urlcolor={Blue},
  pdfcreator={LaTeX via pandoc}}

\title{Introducción a los Sistemas Económicos. Cómo se distribuyen los
recursos y se producen}
\usepackage{etoolbox}
\makeatletter
\providecommand{\subtitle}[1]{% add subtitle to \maketitle
  \apptocmd{\@title}{\par {\large #1 \par}}{}{}
}
\makeatother
\subtitle{Explorando los fundamentos de los sistemas económicos y sus
implicaciones en la sociedad}
\author{Achalma Mendoza Edison}
\date{2023-06-13}

\begin{document}
\maketitle
\ifdefined\Shaded\renewenvironment{Shaded}{\begin{tcolorbox}[sharp corners, boxrule=0pt, interior hidden, borderline west={3pt}{0pt}{shadecolor}, enhanced, breakable, frame hidden]}{\end{tcolorbox}}\fi

\hypertarget{introducciuxf3n-a-los-sistemas-econuxf3micos-cuxf3mo-se-distribuyen-los-recursos-se-producen-y-se-distribuyen-los-bienes-y-servicios}{%
\section{Introducción a los Sistemas Económicos: Cómo se distribuyen los
recursos, se producen y se distribuyen los bienes y
servicios}\label{introducciuxf3n-a-los-sistemas-econuxf3micos-cuxf3mo-se-distribuyen-los-recursos-se-producen-y-se-distribuyen-los-bienes-y-servicios}}

Los sistemas económicos desempeñan un papel crucial en la organización
de una sociedad. Determinan cómo se distribuyen los recursos, cómo se
producen los bienes y servicios, quién se encarga de hacerlo y cómo se
distribuyen esos bienes y servicios a los consumidores. En este
artículo, exploraremos los conceptos clave relacionados con los sistemas
económicos y analizaremos detalladamente cada uno de ellos.

\begin{enumerate}
\def\labelenumi{\arabic{enumi}.}
\tightlist
\item
  Distribución de recursos para la producción de bienes y servicios: El
  sistema económico de una sociedad se refiere a la forma en que se
  distribuyen los recursos para la producción de bienes y servicios. Los
  recursos pueden incluir capital, tierra, trabajo y tecnología. Hay
  diferentes enfoques en la distribución de estos recursos:
\end{enumerate}

\begin{itemize}
\item
  Economía de mercado: En una economía de mercado, los recursos son
  distribuidos principalmente por la interacción de la oferta y la
  demanda en un mercado. Los precios juegan un papel importante en la
  asignación de recursos, ya que reflejan las preferencias y
  valoraciones de los consumidores y las empresas.
\item
  Economía planificada: En una economía planificada, el gobierno tiene
  un papel central en la distribución de recursos. El Estado determina
  qué bienes y servicios se producirán y en qué cantidad, y asigna los
  recursos de acuerdo con un plan económico.
\item
  Economía mixta: La mayoría de las economías modernas son mixtas, lo
  que significa que combinan elementos de una economía de mercado y una
  economía planificada. En este sistema, tanto el mercado como el
  gobierno intervienen en la asignación de recursos y la producción de
  bienes y servicios.
\end{itemize}

\begin{enumerate}
\def\labelenumi{\arabic{enumi}.}
\setcounter{enumi}{1}
\tightlist
\item
  Satisfacción de las necesidades de los consumidores: En cualquier
  sistema económico, la satisfacción de las necesidades de los
  consumidores es un objetivo fundamental. Para determinar qué bienes y
  servicios satisfarán las necesidades de los consumidores y en qué
  cantidad, se utilizan diferentes enfoques:
\end{enumerate}

\begin{itemize}
\item
  Oferta y demanda: El mecanismo de oferta y demanda en un mercado
  determina qué bienes y servicios son demandados por los consumidores.
  Los productores responden a esta demanda ofreciendo productos que sean
  rentables.
\item
  Preferencias del consumidor: Las preferencias y elecciones
  individuales de los consumidores también influyen en la determinación
  de qué bienes y servicios se producirán y en qué cantidad. Los
  consumidores expresan sus preferencias a través de sus decisiones de
  compra.
\end{itemize}

\begin{enumerate}
\def\labelenumi{\arabic{enumi}.}
\setcounter{enumi}{2}
\tightlist
\item
  Producción de bienes y servicios: La forma en que se producen los
  bienes y servicios, quién se encarga de hacerlo y qué recursos se
  utilizan para ello, varía según el sistema económico:
\end{enumerate}

\begin{itemize}
\item
  Propiedad y control: En una economía de mercado, la propiedad y el
  control de los medios de producción (recursos y empresas) están en
  manos de individuos y empresas privadas. En una economía planificada,
  el gobierno tiene propiedad y control sobre la producción.
\item
  Asignación de recursos: En una economía de mercado, la asignación de
  recursos se realiza a través del sistema de precios y la competencia
  entre las empresas. En una economía planificada, el gobierno asigna
  los recursos según el plan económico establecido.
\end{itemize}

\begin{enumerate}
\def\labelenumi{\arabic{enumi}.}
\setcounter{enumi}{3}
\tightlist
\item
  Distribución de bienes y servicios a los consumidores: La forma en que
  se distribuyen los bienes y servicios a los consumidores también varía
  según el sistema económico:
\end{enumerate}

\begin{itemize}
\item
  Mercado: En una economía de mercado, la distribución de bienes y
  servicios se realiza a través de transacciones voluntarias en el
  mercado. Los consumidores adquieren los bienes y servicios que desean
  a través de intercambios comerciales.
\item
  Planificación centralizada: En una economía planificada, el gobierno
  controla la distribución de bienes y servicios y los asigna a los
  consumidores según el plan económico establecido.
\end{itemize}

\hypertarget{tipos-de-sistemas-capitalismo-socialismo-y-comunismo}{%
\section{Tipos de Sistemas: Capitalismo, Socialismo y
Comunismo}\label{tipos-de-sistemas-capitalismo-socialismo-y-comunismo}}

\hypertarget{comunismo}{%
\subsection{Comunismo}\label{comunismo}}

El comunismo es una sociedad en la que las personas son propietarias de
todos los recursos de una nación, independientemente de su clase social.
Sin embargo, es importante destacar que el comunismo tal como lo
describiste no se ha logrado implementar en su forma pura en la
práctica.

Sistema económico-político ideal, donde todos contribuyen según su
capacidad y reciben según su necesidad, es una afirmación que proviene
de la teoría marxista y busca la igualdad en la distribución de
recursos. Sin embargo, es importante mencionar que el comunismo como se
ha intentado aplicar en países como China, Corea del Norte y Cuba ha
presentado desafíos y no ha alcanzado completamente los objetivos
propuestos.

\hypertarget{socialismo}{%
\subsection{Socialismo}\label{socialismo}}

El socialismo se caracteriza por la propiedad y operación estatal de las
industrias básicas, como los servicios públicos, la industria postal, el
transporte, la salud y algunos sectores manufactureros. A diferencia del
comunismo, en el socialismo existe cierto grado de propiedad privada de
las empresas por parte de los individuos.

En este sistema, los ciudadanos dependen del gobierno para obtener
muchos bienes y servicios. Sin embargo, es importante destacar que
existen diferentes enfoques dentro del socialismo. Algunos países
socialistas democráticos, como Suecia, India, Francia e Israel,
reconocen las libertades individuales básicas y sus ciudadanos
participan en el proceso de elección de representantes políticos.

\hypertarget{capitalismo-o-sistema-de-libre-empresa}{%
\subsection{Capitalismo o Sistema de Libre
Empresa}\label{capitalismo-o-sistema-de-libre-empresa}}

El capitalismo es un sistema en el que las personas son propietarias y
operadoras de la mayoría de las empresas que proporcionan bienes y
servicios. En este sistema, la competencia, la oferta y la demanda son
los factores determinantes para qué bienes y servicios se producen, cómo
se producen y cómo se distribuyen.

Algunos ejemplos de países que adoptan el sistema capitalista son
Estados Unidos, Canadá, Japón y Australia. Sin embargo, es importante
destacar que existen diferentes modelos de capitalismo en todo el mundo,
algunos con mayores regulaciones gubernamentales y otros con un enfoque
más liberal.

\hypertarget{diferencias-entre-sistemas-econuxf3micos}{%
\section{Diferencias entre sistemas
económicos}\label{diferencias-entre-sistemas-econuxf3micos}}

\begin{longtable}[]{@{}
  >{\raggedright\arraybackslash}p{(\columnwidth - 6\tabcolsep) * \real{0.0758}}
  >{\raggedright\arraybackslash}p{(\columnwidth - 6\tabcolsep) * \real{0.2695}}
  >{\raggedright\arraybackslash}p{(\columnwidth - 6\tabcolsep) * \real{0.3200}}
  >{\raggedright\arraybackslash}p{(\columnwidth - 6\tabcolsep) * \real{0.3347}}@{}}
\toprule\noalign{}
\begin{minipage}[b]{\linewidth}\raggedright
\end{minipage} & \begin{minipage}[b]{\linewidth}\raggedright
Capitalismo
\end{minipage} & \begin{minipage}[b]{\linewidth}\raggedright
Socialismo
\end{minipage} & \begin{minipage}[b]{\linewidth}\raggedright
Comunismo
\end{minipage} \\
\midrule\noalign{}
\endhead
\bottomrule\noalign{}
\endlastfoot
Tipos de propiedad de las empresas & Los individuos son propietarios y
operadores de todas las empresas. & El gobierno es propietario y
operador de grandes industrias; las personas son propietarias de
pequeñas empresas & La propiedad de las empresas es común, no hay
propiedad privada. \\
Competencia & Se alienta con las fuerzas de mercado y regulaciones
gubernamentales. & Restringida en las grandes industrias; se alienta en
las pequeñas empresas & No se promueve la competencia en términos de
propiedad de empresas ya que todas son propiedad común y no hay empresas
privadas. \\
Utilidades & Los individuos son libres de mantener sus utilidades y
usarlas como ellos deseen. & Las utilidades que generan las pequeñas
empresas suelen reinvertirse en ellas; las utilidades de las industrias
propiedad del gobierno pasan al gobierno & No se aplica la noción de
utilidades individuales ya que la producción y distribución de bienes y
servicios se realiza en beneficio de la comunidad en general. \\
Disponibilidad y precio de productos & Los consumidores tienen una
amplia variedad de bienes y servicios a su disposición; los precios
dependen de la oferta y demanda. & Los consumidores tienen cierta
diversidad de bienes y servicios; los precios dependen de la oferta y
demanda & El concepto de ``precio'' tal como se entiende en los sistemas
capitalistas es diferente. \footnote{En el comunismo, el concepto de
  ``precio'' tal como se entiende en los sistemas capitalistas es
  diferente. En el comunismo, donde no existe la propiedad privada ni el
  intercambio de bienes y servicios a través de un mercado competitivo,
  no se utiliza el mecanismo de precios como una forma de asignar
  recursos y determinar el valor de los productos. En lugar de precios
  determinados por la oferta y la demanda, en el comunismo se busca una
  distribución equitativa de los bienes y servicios, priorizando las
  necesidades de la comunidad en general. La planificación centralizada
  y la administración colectiva de los recursos son características
  centrales de este sistema, y la asignación de recursos se basa en
  criterios de utilidad social y no en términos monetarios o de precios.
  Por lo tanto, es más apropiado hablar de la disponibilidad y
  distribución de bienes y servicios en el comunismo, en lugar de
  precios específicos para cada producto. El objetivo principal es
  garantizar que todos los miembros de la sociedad tengan acceso a los
  bienes y servicios necesarios para satisfacer sus necesidades básicas.} \\
Opciones de empleo & Variedad ilimitada de carreras & Cierta variedad de
carreras; muchas personas trabajan para el gobierno \footnote{Según el
  marco teórico del socialismo, no es cierto afirmar que ``muchas
  personas trabajan para el gobierno'' en términos absolutos. El
  socialismo no implica automáticamente un alto porcentaje de empleo en
  el sector público o para el gobierno. En el socialismo, el énfasis se
  coloca en la propiedad y control colectivo de los medios de
  producción, así como en la justa distribución de los recursos. El
  objetivo principal del socialismo es crear una sociedad más equitativa
  y reducir las desigualdades sociales y económicas. Esto se puede
  lograr a través de la propiedad pública, cooperativas o formas mixtas
  de propiedad. En un sistema socialista, es posible que exista una
  mayor participación y regulación estatal en ciertos sectores
  estratégicos de la economía, como la salud, la educación, los
  servicios públicos o la infraestructura. Sin embargo, esto no implica
  necesariamente que ``muchas personas trabajen para el gobierno''. El
  empleo en el socialismo puede ser diverso y abarcar una amplia gama de
  sectores económicos, incluyendo el sector privado y la economía
  cooperativa. Es importante destacar que el socialismo puede
  manifestarse de diferentes maneras en diferentes países y contextos.
  Las políticas y prácticas específicas pueden variar, y es posible que
  algunos países socialistas hayan tenido una mayor participación del
  gobierno en el empleo. Sin embargo, en términos generales, el marco
  teórico del socialismo no implica automáticamente un alto empleo
  gubernamental para muchas personas.} & El empleo en el comunismo se
basa en principios de cooperación, igualdad y satisfacción de las
necesidades de la comunidad. \footnote{El empleo en el comunismo se basa
  en principios de cooperación, igualdad y satisfacción de las
  necesidades de la comunidad. En un estado comunista ideal, el objetivo
  es eliminar la explotación y las desigualdades laborales presentes en
  otros sistemas económicos. En el comunismo, se espera que todas las
  personas contribuyan a la sociedad según sus capacidades y
  habilidades. El trabajo se concibe como una actividad que beneficia a
  toda la comunidad, en lugar de ser una forma de generar beneficios
  individuales o acumulación de riqueza. En términos de elección de
  carrera, aunque no hay una definición específica en el marco teórico
  del comunismo, se espera que la planificación centralizada y la
  organización colectiva de la economía permitan satisfacer las
  necesidades de la sociedad en general. En teoría, se busca la
  asignación de recursos y empleo en función de las necesidades sociales
  y la capacidad de cada individuo, en lugar de las fuerzas del mercado
  o los intereses individuales.} \\
\end{longtable}

\hypertarget{capitalismo}{%
\section{Capitalismo}\label{capitalismo}}

El capitalismo es un sistema económico caracterizado por la propiedad
privada de los medios de producción y la libertad de mercado.

\hypertarget{capitalismo-puro}{%
\subsection{Capitalismo puro}\label{capitalismo-puro}}

El capitalismo puro, también conocido como sistema libre de mercado, se
refiere a un modelo en el cual las decisiones económicas se toman sin
intervención gubernamental directa en los asuntos económicos. En este
sistema, los recursos y la producción están en manos de individuos y
empresas privadas, quienes determinan qué bienes y servicios producir y
cómo distribuirlos en el mercado. A menudo se le atribuye a Adam Smith,
autor de la obra ``La riqueza de las naciones'' publicada en 1776, el
ser considerado el padre del capitalismo debido a su análisis y defensa
de los principios fundamentales de este sistema económico.

La ``mano invisible de la competencia'' es un concepto introducido por
Adam Smith en su obra. Se refiere a la idea de que, en un sistema de
libre competencia, las interacciones entre los compradores y vendedores,
guiadas por sus propios intereses egoístas, conducen a una asignación
eficiente de los recursos en la economía. Según esta teoría, la
competencia en el mercado actúa como un regulador natural, ajustando los
precios y las cantidades producidas en función de la oferta y la
demanda, sin necesidad de intervención gubernamental directa.

\hypertarget{capitalismo-modificado}{%
\subsection{Capitalismo modificado}\label{capitalismo-modificado}}

En contraste con el capitalismo puro, el capitalismo modificado implica
la intervención y regulación gubernamental en cierta medida. En este
caso, el gobierno desempeña un papel más activo en la economía y
establece regulaciones para salvaguardar el interés público, promover la
equidad y prevenir abusos. Estas regulaciones pueden abarcar diversos
aspectos, como la protección del medio ambiente, la seguridad laboral,
la competencia justa y el etiquetado de sustancias peligrosas, entre
otros.

Es importante destacar que cada país puede tener su propio conjunto de
leyes y regulaciones que se aplican al funcionamiento de las empresas y
los mercados. Tomemos como ejemplo a Estados Unidos y Canadá, donde
existen leyes que regulan aspectos como el etiquetado de sustancias
peligrosas. Sin embargo, es importante reconocer que estas regulaciones
específicas pueden variar y evolucionar con el tiempo, ya que los
gobiernos buscan equilibrar los intereses económicos y sociales en el
contexto de sus respectivas sociedades.

\hypertarget{economuxedas-mixtas}{%
\section{Economías mixtas}\label{economuxedas-mixtas}}

\hypertarget{economuxedas-mixtas-una-combinaciuxf3n-de-sistemas-econuxf3micos}{%
\subsection{Economías mixtas: una combinación de sistemas
económicos}\label{economuxedas-mixtas-una-combinaciuxf3n-de-sistemas-econuxf3micos}}

En la realidad, ningún país practica una forma pura de comunismo,
socialismo o capitalismo. En cambio, muchas naciones adoptan un enfoque
de economía mixta, que combina elementos de dos o más sistemas
económicos para abordar las necesidades y los desafíos específicos de su
sociedad.

\hypertarget{suecia-y-su-modelo-econuxf3mico}{%
\subsection{Suecia y su modelo
económico}\label{suecia-y-su-modelo-econuxf3mico}}

Si bien se ha descrito a Suecia como un país socialista en algunos
contextos, es más preciso considerar su sistema económico como una
economía mixta con una fuerte intervención del Estado en algunos
sectores clave. En Suecia, las personas pueden ser propietarias y
operadoras de muchas empresas, lo que implica la existencia de una
iniciativa privada y empresarial. Sin embargo, el gobierno sueco también
desempeña un papel activo en la provisión de servicios públicos y la
regulación de ciertos sectores de la economía.

\hypertarget{estados-unidos-y-su-enfoque-capitalista}{%
\subsection{Estados Unidos y su enfoque
capitalista}\label{estados-unidos-y-su-enfoque-capitalista}}

Estados Unidos se caracteriza por ser una economía capitalista, donde la
propiedad privada y la libertad de mercado son fundamentales. Sin
embargo, también existen casos en los que el gobierno federal tiene la
propiedad y opera ciertos servicios públicos, como el servicio postal y
eléctrico. Estos ejemplos de propiedad estatal en un sistema mayormente
capitalista son excepciones a la norma y no representan la totalidad de
la economía del país.

\hypertarget{reino-unido-muxe9xico-y-la-privatizaciuxf3n}{%
\subsection{Reino Unido, México y la
privatización}\label{reino-unido-muxe9xico-y-la-privatizaciuxf3n}}

Tanto el Reino Unido como México han llevado a cabo procesos de
privatización, que implican la venta de empresas estatales a empresas
privadas. Este enfoque busca transferir la propiedad y la operación de
ciertas industrias del sector público al sector privado. Sin embargo, es
importante tener en cuenta que la privatización no significa
necesariamente un cambio hacia un sistema económico completamente
capitalista, ya que estos países aún mantienen una regulación
gubernamental en diversos aspectos económicos.

\hypertarget{transiciuxf3n-en-naciones-antiguamente-comunistas}{%
\subsection{Transición en naciones antiguamente
comunistas}\label{transiciuxf3n-en-naciones-antiguamente-comunistas}}

Tras la caída del comunismo, naciones como Rusia, Hungría, Polonia y
otras han experimentado una transición hacia una economía más orientada
al capitalismo. Esto ha implicado la adopción de la propiedad privada y
la apertura de mercados, en contraste con el enfoque anterior de
propiedad estatal y planificación centralizada. Sin embargo, es
importante destacar que estas economías aún tienen elementos de una
economía mixta, ya que el gobierno mantiene un cierto grado de
intervención y regulación en la economía.

\hypertarget{sistema-de-libre-empresa}{%
\section{Sistema de libre empresa}\label{sistema-de-libre-empresa}}

\hypertarget{el-sistema-de-libre-empresa-una-oportunidad-para-el-uxe9xito-y-el-fracaso-empresarial}{%
\subsection{El sistema de libre empresa: una oportunidad para el éxito y
el fracaso
empresarial}\label{el-sistema-de-libre-empresa-una-oportunidad-para-el-uxe9xito-y-el-fracaso-empresarial}}

El sistema de libre empresa, también conocido como economía de mercado o
capitalismo, brinda a las empresas la oportunidad de tener éxito o
fracaso en función de la demanda y las decisiones empresariales. En este
sistema, el mercado y la competencia son los principales impulsores de
la actividad económica.

\hypertarget{derechos-individuales-y-empresariales-en-la-libre-empresa}{%
\subsection{Derechos individuales y empresariales en la libre
empresa}\label{derechos-individuales-y-empresariales-en-la-libre-empresa}}

Para que funcione el sistema de libre empresa, es necesario garantizar
varios derechos individuales y empresariales fundamentales. Estos
derechos incluyen:

\begin{enumerate}
\def\labelenumi{\arabic{enumi}.}
\item
  Derecho a la propiedad privada y transferencia intergeneracional: Los
  individuos deben tener el derecho a poseer propiedades privadas y
  transferirlas a sus herederos. Esto implica que las personas pueden
  ser propietarias de activos, como tierras, viviendas y empresas.
\item
  Derecho a la libre elección y búsqueda de utilidades: Los individuos y
  las empresas deben contar con el derecho de tomar decisiones autónomas
  en su actividad económica, incluyendo la elección de productos,
  servicios y la búsqueda de utilidades. Esto implica que los
  emprendedores y empresarios tienen la libertad de emplear los recursos
  y factores productivos de la manera que consideren más adecuada para
  maximizar sus beneficios.
\item
  Derecho a la autonomía empresarial: Los individuos y las empresas
  tienen el derecho de tomar decisiones que determinen la forma en que
  operan. Esto incluye la libertad para establecer estrategias de
  negocios, fijar precios, seleccionar proveedores, establecer políticas
  internas y gestionar sus recursos de manera autónoma.
\item
  Derecho a la libre elección y autodeterminación personal: Los
  individuos tienen el derecho de elegir su carrera profesional, dónde
  vivir y qué comprar, mientras que las empresas tienen la libertad de
  decidir dónde ubicarse, qué producir y cómo operar en el mercado.
\end{enumerate}

\begin{quote}
Es importante destacar que, si bien el sistema de libre empresa otorga
una amplia libertad y autonomía a los individuos y las empresas, también
implica la necesidad de cumplir con ciertas normas y regulaciones
establecidas por el Estado. Estas regulaciones tienen como objetivo
proteger los derechos de las partes involucradas y garantizar un entorno
de competencia justa y equitativa.
\end{quote}

\hypertarget{fuerza-de-la-oferta-y-la-demanda}{%
\section{Fuerza de la oferta y la
demanda}\label{fuerza-de-la-oferta-y-la-demanda}}

\hypertarget{entendiendo-la-demanda-el-papel-de-los-consumidores}{%
\subsection{Entendiendo la demanda: el papel de los
consumidores}\label{entendiendo-la-demanda-el-papel-de-los-consumidores}}

La demanda se refiere al número de bienes o servicios que los
consumidores están dispuestos a comprar a diferentes precios y en un
momento específico. La demanda está determinada por varios factores,
como el precio del producto, el nivel de ingresos de los consumidores,
las preferencias individuales y los factores externos que puedan influir
en la demanda, como las tendencias de mercado o las políticas
gubernamentales.

Cuando el precio de un producto disminuye, la cantidad demandada tiende
a aumentar, ya que los consumidores encuentran el producto más
asequible. Por el contrario, cuando el precio de un producto aumenta, la
cantidad demandada tiende a disminuir, ya que los consumidores pueden
optar por buscar alternativas más económicas o reducir su consumo.

\hypertarget{explorando-la-oferta-el-papel-de-las-empresas}{%
\subsection{Explorando la oferta: el papel de las
empresas}\label{explorando-la-oferta-el-papel-de-las-empresas}}

La oferta se refiere al número de productos (bienes o servicios) que las
empresas están dispuestas a vender a diferentes precios y en un momento
específico. La oferta está influenciada por diversos factores, como el
costo de producción, la disponibilidad de recursos, la tecnología
utilizada, la competencia en el mercado y las expectativas
empresariales.

Cuando el precio de un producto aumenta, las empresas tienen un
incentivo para ofrecer más productos en el mercado, ya que pueden
obtener mayores ganancias. Por el contrario, cuando el precio de un
producto disminuye, las empresas pueden reducir su oferta, ya que la
rentabilidad puede verse afectada. La relación entre el precio y la
cantidad ofrecida se conoce como la ley de la oferta, que establece que,
en general, a medida que el precio aumenta, la cantidad ofrecida también
tiende a aumentar.

\hypertarget{precio-de-equilibrio-cuando-la-oferta-y-la-demanda-se-encuentran}{%
\subsection{Precio de equilibrio: cuando la oferta y la demanda se
encuentran}\label{precio-de-equilibrio-cuando-la-oferta-y-la-demanda-se-encuentran}}

El precio de equilibrio es aquel en el cual el número de productos que
las empresas están dispuestas a vender es igual a la cantidad de
productos que los consumidores están dispuestos a comprar, en un momento
específico. En este punto, la oferta y la demanda se equilibran, lo que
significa que no hay exceso de oferta o demanda insatisfecha en el
mercado.

El precio de equilibrio se determina mediante la interacción de la
oferta y la demanda en el mercado. Si el precio actual es más alto que
el precio de equilibrio, es probable que la cantidad demandada sea menor
que la cantidad ofrecida, lo que puede generar un exceso de oferta. En
cambio, si el precio actual es más bajo que el precio de equilibrio, es
probable que la cantidad demandada sea mayor que la cantidad ofrecida,
lo que puede generar una escasez de producto.

\begin{quote}
Es importante destacar que el precio de equilibrio puede cambiar a lo
largo del tiempo debido a diversos factores, como cambios en la oferta o
la demanda, avances tecnológicos, políticas gubernamentales o eventos
económicos.
\end{quote}

\hypertarget{naturaleza-de-la-competencia}{%
\section{Naturaleza de la
competencia}\label{naturaleza-de-la-competencia}}

\hypertarget{comprendiendo-la-competencia-rivalidad-entre-empresas}{%
\subsection{Comprendiendo la competencia: rivalidad entre
empresas}\label{comprendiendo-la-competencia-rivalidad-entre-empresas}}

La competencia se refiere a la rivalidad existente entre las empresas
por la venta de productos o servicios a los consumidores. Es un
componente fundamental de los mercados y juega un papel crucial en la
determinación de los precios, la calidad de los productos y la
innovación.

\hypertarget{competencia-perfecta-muchas-empresas-y-productos-estandarizados}{%
\subsection{Competencia perfecta: muchas empresas y productos
estandarizados}\label{competencia-perfecta-muchas-empresas-y-productos-estandarizados}}

La competencia perfecta es una estructura de mercado en la que existen
numerosas empresas pequeñas que venden un producto estandarizado. En
este tipo de competencia, no hay una empresa dominante y todas las
empresas tienen acceso a la misma información y tecnología. Los
consumidores tienen plena libertad para elegir entre diferentes opciones
y no hay barreras significativas para la entrada de nuevas empresas al
mercado.

\hypertarget{competencia-monopolista-variedad-y-diferenciaciuxf3n-de-productos}{%
\subsection{Competencia monopolista: variedad y diferenciación de
productos}\label{competencia-monopolista-variedad-y-diferenciaciuxf3n-de-productos}}

La competencia monopolista es una estructura de mercado en la que hay
menos empresas que en la competencia perfecta y existe cierta
diferenciación entre los productos. En este caso, cada empresa busca
diferenciarse de las demás a través de características únicas de sus
productos, como marca, diseño, calidad o servicios adicionales. Ejemplos
comunes de competencia monopolista son la industria de los refrescos o
la venta de aspirinas, donde diferentes marcas compiten por atraer a los
consumidores ofreciendo pequeñas diferencias en los productos.

\hypertarget{oligopolio-pocas-empresas-y-control-de-precios}{%
\subsection{Oligopolio: pocas empresas y control de
precios}\label{oligopolio-pocas-empresas-y-control-de-precios}}

El oligopolio es una estructura de mercado en la que solo unas pocas
empresas dominan la venta de un producto o servicio. En esta situación,
las empresas tienden a tener un alto grado de interdependencia y sus
acciones estratégicas pueden afectar significativamente a las demás.
Esto puede llevar a una competencia basada en factores distintos al
precio, como la publicidad, la calidad del producto o la diferenciación.
Las empresas en un oligopolio también pueden colaborar para fijar
precios o limitar la competencia, lo que puede resultar en precios
relativamente estables y similares entre las empresas del sector.

\hypertarget{monopolio-una-empresa-en-control}{%
\subsection{Monopolio: una empresa en
control}\label{monopolio-una-empresa-en-control}}

El monopolio ocurre cuando una sola empresa tiene el control exclusivo
sobre la oferta de un producto o servicio en un mercado determinado. En
esta situación, la empresa puede establecer los precios y las
condiciones de venta sin enfrentar competencia directa. Los monopolios
pueden surgir debido a barreras legales, tecnológicas o económicas que
dificultan o impiden la entrada de nuevas empresas al mercado. Para
evitar abusos de poder, los monopolios suelen estar sujetos a
regulaciones gubernamentales.


\printbibliography


\end{document}
