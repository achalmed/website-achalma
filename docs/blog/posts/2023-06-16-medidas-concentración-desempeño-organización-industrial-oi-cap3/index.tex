% Options for packages loaded elsewhere
\PassOptionsToPackage{unicode}{hyperref}
\PassOptionsToPackage{hyphens}{url}
\PassOptionsToPackage{dvipsnames,svgnames,x11names}{xcolor}
%
\documentclass[
  letterpaper,
  DIV=11,
  numbers=noendperiod]{scrartcl}

\usepackage{amsmath,amssymb}
\usepackage{iftex}
\ifPDFTeX
  \usepackage[T1]{fontenc}
  \usepackage[utf8]{inputenc}
  \usepackage{textcomp} % provide euro and other symbols
\else % if luatex or xetex
  \usepackage{unicode-math}
  \defaultfontfeatures{Scale=MatchLowercase}
  \defaultfontfeatures[\rmfamily]{Ligatures=TeX,Scale=1}
\fi
\usepackage{lmodern}
\ifPDFTeX\else  
    % xetex/luatex font selection
\fi
% Use upquote if available, for straight quotes in verbatim environments
\IfFileExists{upquote.sty}{\usepackage{upquote}}{}
\IfFileExists{microtype.sty}{% use microtype if available
  \usepackage[]{microtype}
  \UseMicrotypeSet[protrusion]{basicmath} % disable protrusion for tt fonts
}{}
\makeatletter
\@ifundefined{KOMAClassName}{% if non-KOMA class
  \IfFileExists{parskip.sty}{%
    \usepackage{parskip}
  }{% else
    \setlength{\parindent}{0pt}
    \setlength{\parskip}{6pt plus 2pt minus 1pt}}
}{% if KOMA class
  \KOMAoptions{parskip=half}}
\makeatother
\usepackage{xcolor}
\setlength{\emergencystretch}{3em} % prevent overfull lines
\setcounter{secnumdepth}{-\maxdimen} % remove section numbering
% Make \paragraph and \subparagraph free-standing
\ifx\paragraph\undefined\else
  \let\oldparagraph\paragraph
  \renewcommand{\paragraph}[1]{\oldparagraph{#1}\mbox{}}
\fi
\ifx\subparagraph\undefined\else
  \let\oldsubparagraph\subparagraph
  \renewcommand{\subparagraph}[1]{\oldsubparagraph{#1}\mbox{}}
\fi


\providecommand{\tightlist}{%
  \setlength{\itemsep}{0pt}\setlength{\parskip}{0pt}}\usepackage{longtable,booktabs,array}
\usepackage{calc} % for calculating minipage widths
% Correct order of tables after \paragraph or \subparagraph
\usepackage{etoolbox}
\makeatletter
\patchcmd\longtable{\par}{\if@noskipsec\mbox{}\fi\par}{}{}
\makeatother
% Allow footnotes in longtable head/foot
\IfFileExists{footnotehyper.sty}{\usepackage{footnotehyper}}{\usepackage{footnote}}
\makesavenoteenv{longtable}
\usepackage{graphicx}
\makeatletter
\def\maxwidth{\ifdim\Gin@nat@width>\linewidth\linewidth\else\Gin@nat@width\fi}
\def\maxheight{\ifdim\Gin@nat@height>\textheight\textheight\else\Gin@nat@height\fi}
\makeatother
% Scale images if necessary, so that they will not overflow the page
% margins by default, and it is still possible to overwrite the defaults
% using explicit options in \includegraphics[width, height, ...]{}
\setkeys{Gin}{width=\maxwidth,height=\maxheight,keepaspectratio}
% Set default figure placement to htbp
\makeatletter
\def\fps@figure{htbp}
\makeatother

\KOMAoption{captions}{tableheading,figureheading}
\makeatletter
\makeatother
\makeatletter
\makeatother
\makeatletter
\@ifpackageloaded{caption}{}{\usepackage{caption}}
\AtBeginDocument{%
\ifdefined\contentsname
  \renewcommand*\contentsname{Tabla de contenidos}
\else
  \newcommand\contentsname{Tabla de contenidos}
\fi
\ifdefined\listfigurename
  \renewcommand*\listfigurename{Listado de Figuras}
\else
  \newcommand\listfigurename{Listado de Figuras}
\fi
\ifdefined\listtablename
  \renewcommand*\listtablename{Listado de Tablas}
\else
  \newcommand\listtablename{Listado de Tablas}
\fi
\ifdefined\figurename
  \renewcommand*\figurename{Figura}
\else
  \newcommand\figurename{Figura}
\fi
\ifdefined\tablename
  \renewcommand*\tablename{Tabla}
\else
  \newcommand\tablename{Tabla}
\fi
}
\@ifpackageloaded{float}{}{\usepackage{float}}
\floatstyle{ruled}
\@ifundefined{c@chapter}{\newfloat{codelisting}{h}{lop}}{\newfloat{codelisting}{h}{lop}[chapter]}
\floatname{codelisting}{Listado}
\newcommand*\listoflistings{\listof{codelisting}{Listado de Listados}}
\makeatother
\makeatletter
\@ifpackageloaded{caption}{}{\usepackage{caption}}
\@ifpackageloaded{subcaption}{}{\usepackage{subcaption}}
\makeatother
\makeatletter
\@ifpackageloaded{tcolorbox}{}{\usepackage[skins,breakable]{tcolorbox}}
\makeatother
\makeatletter
\@ifundefined{shadecolor}{\definecolor{shadecolor}{rgb}{.97, .97, .97}}
\makeatother
\makeatletter
\makeatother
\makeatletter
\makeatother
\ifLuaTeX
\usepackage[bidi=basic]{babel}
\else
\usepackage[bidi=default]{babel}
\fi
\babelprovide[main,import]{spanish}
% get rid of language-specific shorthands (see #6817):
\let\LanguageShortHands\languageshorthands
\def\languageshorthands#1{}
\ifLuaTeX
  \usepackage{selnolig}  % disable illegal ligatures
\fi
\usepackage[]{biblatex}
\addbibresource{../../../../references.bib}
\IfFileExists{bookmark.sty}{\usepackage{bookmark}}{\usepackage{hyperref}}
\IfFileExists{xurl.sty}{\usepackage{xurl}}{} % add URL line breaks if available
\urlstyle{same} % disable monospaced font for URLs
\hypersetup{
  pdftitle={Medidas de concentracion},
  pdfauthor={Edison Achalma},
  pdflang={es},
  colorlinks=true,
  linkcolor={blue},
  filecolor={Maroon},
  citecolor={Blue},
  urlcolor={Blue},
  pdfcreator={LaTeX via pandoc}}

\title{Medidas de concentracion}
\usepackage{etoolbox}
\makeatletter
\providecommand{\subtitle}[1]{% add subtitle to \maketitle
  \apptocmd{\@title}{\par {\large #1 \par}}{}{}
}
\makeatother
\subtitle{Explorando los pilares fundamentales para comprender el
funcionamiento y éxito de la industria moderna}
\author{Edison Achalma}
\date{2023-06-16}

\begin{document}
\maketitle
\ifdefined\Shaded\renewenvironment{Shaded}{\begin{tcolorbox}[frame hidden, interior hidden, sharp corners, borderline west={3pt}{0pt}{shadecolor}, boxrule=0pt, enhanced, breakable]}{\end{tcolorbox}}\fi

MEDICION DE LA ESTRUCTURA Y DESEMPEÑO DE LA ORGANIZACIÓN INDUSTRIAL

En este capitulo aprenderemos:

\begin{itemize}
\item
  Aspectos conceptuales Y prácticos sobre la medición de la
  concentración y poder de las empresas en el mercado
\item
  Medición de la estructura de mercado
\item
  Medición del desempeño empresarial
\end{itemize}

\hypertarget{concentraciuxf3n-del-mercdo}{%
\section{CONCENTRACIÓN DEL MERCDO}\label{concentraciuxf3n-del-mercdo}}

\textbf{DEFINICIÓN:}

La Concentración es una categoría fundamental de la Organización
industrial, que estudia, analiza y describe de cómo la decisión de
producción y la provisión de bienes y servicios en el mercado esta
concentrado en un número REDUCIDO de empresas. Se puede definir también,
cómo el Grado de producción y ventas, concentrada en manos de un
reducido número de grandes empresas que imponen poder en el mercado.

\textbf{¿A que responde la concentración del mercado?}

\begin{itemize}
\item
  Responde a decisiones de fijación de precios
\item
  A los grandes niveles y magnitudes de inversión
\item
  Las capacidades productivas de un conjunto de empresas.
\end{itemize}

El resultante de esta acumulación se conoce como monopolio, oligopolio o
grupo de empresas monopólicas, sociedades de empresas o trust.

Las medidas de concentración en organización industrial van desde un
«índice discreto» y el «índice de importancia relativa de las empresas»
hasta indicadores más sutiles de concentración.

\textbf{¿Como conocer los niveles de concentración?}

La concentración de un bien o servicio, se visualiza mediante las curvas
de concentración, que describe la relación en entre el porcentaje (\%)
acumulado de producción o ventas y el número acumulado porcentual (\%)
de empresas ordenadas de acuerdo con su tamaño, de las más eficientes a
la menos eficientes.

Ejemplo: Los licores son ejemplos de productos sustitutos: Supongamos
que el país se han identificado las cinco (5) empresas más grandes
productores de licores: Backus y Johnston con 100 mil docenas de
cerveza, Pisco Perú con 60,000 docenas de Pisco, Cartavio Perú con
50,000 docenas de Ron, Tres Cruces con 40,000 docenas de cerveza, Piscos
Chinchano con 20,000 docenas y se estima que las pequeñas empresas
producen aproximadamente 10,000 docenas de diversos licores. Calcular la
participación porcentual relativa y acumulada de las empresas en la
industria de la licorería y luego graficar la curva de concentración de
la industria de licores?

SOLUCIÓN DEL PROBLEMA:

Se disponen de los datos en la forma más conveniente, luego se hallan
las distribuciones porcentuales de Producción y número de empresas en
términos relativos y después en forma acumulada , graficándose con estas
últimas la curva de concentración. En la abscisa de registran el \%
acumulado de empresas y en la ordenada el \% acumulado de producción .

TABLAAAAA

Otros indicadores distintos de la concentración : Promedio de la
producción Industrial y ratio de participación porcentual de la
producción de cada empresa en la producción industrial.

La Competencia perfecta es el ``estado natural'' al que tienden los
mercados, donde la concentración es nula. En cambio la concentración
extrema es el monopolio perfecto.

\textbf{¿Por qué surge concentración de mercado?}

\begin{itemize}
\item
  Innovación tecnológica (empresarial y geográficamente) y economías de
  escala.
\item
  Enfoque de crecimiento estocástico (procesos de Gibrat - 1931).
  (Michael Porter, P. Dracker)
\item
  La ``destrucción creativa'' (Schumpeter - 1942). (Invención,
  innovasión, Inversión e Investigación)
\end{itemize}

\hypertarget{el-indice-del-recuxedproco-de-nuxfamero-de-empresas}{%
\subsection{EL INDICE DEL RECÍPROCO DE NÚMERO DE
EMPRESAS}\label{el-indice-del-recuxedproco-de-nuxfamero-de-empresas}}

Este índice nos indica de una manera sencilla cuál es la estructura del
mercado dada la cantidad de empresas en cada instante del tiempo.

La forma más simple de medir la concentración industrial es usar el
recíproco del número de empresas (1/n):

\begin{itemize}
\item
  Sí 1/n→0 \ldots.. El mercado tiene una estructura de ``competencia
  perfecta''
\item
  1/n = 1 \ldots.. El mercado tiene una estructura de ``monopolio''
\item
  Sí 0 \textless{} 1/n \textless{} 0,5 \ldots. Puede ser competencia
  monopolística
\item
  Sí 0.50 ≤ 1/n \textless{} 1, \ldots{} Entonces tiene la estructura de
  oligopolio
\end{itemize}

\hypertarget{producciuxf3n-total-intercambiada-o-transada-en-el-mercado}{%
\subsection{PRODUCCIÓN TOTAL INTERCAMBIADA O TRANSADA EN EL
MERCADO}\label{producciuxf3n-total-intercambiada-o-transada-en-el-mercado}}

Es la oferta agregada total de producción industrial de un mercado
relevante, por sus características de precios, cantidad, calidad, usos y
satisfacción de una necesidad en el tiempo previsto. Supongamos que
tenemos n empresas. Las ordenamos en orden decreciente de acuerdo con su
nivel de producción y las denominamos según su posición en esta
ordenación. La empresa 1 será la mayor y la empresa n la menor.
Conocemos la producción de cada empresa (qi) y, en consecuencia, la
cantidad total intercambiada en el Mercado es igual a:

Q = q1 + q2 + q3 + \ldots{} + qn (12) Q = σn i=1 qi (13)

A partir de esta información se pueden determinar las cuotas
porcentuales de ventas de cada empresa en el mercado industrial. La
cuota de mercado de la empresa i (si ) se define como el cociente entre
la producción de la empresa (q i) y la producción total de la industria
(Q). Es decir:

si = qi σn i=1qi * 100 = qi Q * 100 (3)

A partir de las formulas anteriores, vamos a definir los índices de
concentración:

\hypertarget{el-uxedndice-de-concentraciuxf3n-discreta}{%
\subsection{EL ÍNDICE DE CONCENTRACIÓN
DISCRETA:}\label{el-uxedndice-de-concentraciuxf3n-discreta}}

Es un índice que mide la participación de la producción de un grupo de
empresas (Las más grandes), dentro de la producción total de la
industrial. Es decir, El índice de concentración o índice de acumulación
discreto. consiste en observar la proporción de la variable de dimensión
(por ejemplo, cifra de negocio) que poseen las n empresa más grandes de
la industria estudiada. Ordenadas de conformidad con sus eficiencias de
las más grandes a las más pequeñas, en función a la cantidad producida
de las empresas. El mismo que se calcula mediante la siguiente fórmula:

ICDi = q1 ∗ 100 + q2 ∗ 100 + q 3 ∗ 100 + \ldots+ qn ∗ 100 Q Q Q Q

ICDi = σn i=1 si

\hypertarget{el-uxedndice-de-herfindahl-ih-ih-ux3c3n-i1-s2-i}{%
\subsection{EL ÍNDICE DE HERFINDAHL IH: (IH = σn i=1 s2 i
)}\label{el-uxedndice-de-herfindahl-ih-ih-ux3c3n-i1-s2-i}}

Es una medida empleada en economía, que informa sobre la concentración
económica de la producción total de las empresas más grandes del mercado
relevante, mide la competencia de las empresas más grandes en un sistema
económico. Un IH elevado nos explica que el mercado es muy concentrado y
poco competitivo.

IH = σn i=1 s2 i

El índice se calcula elevando al cuadrado la cuota de mercado (o índice
de concentración) que cada empresa posee y sumando esas cantidades. Los
resultados van desde cero (competencia perfecta) a 10.000 (control
monopólico)

Por ejemplo, considérese un monopolio que controle la totalidad (100\%)
del mercado. 100 elevado al cuadrado (1002 ) es 10.000, dando un índice
de 10 mil. Dos empresas que compartan igualmente el mercado: 50\% del
mercado cada una: es 2.500 cada una. Sumando esas cantidades nos da un
índice de 5 mil. Cuatro empresas con control del mercado de 30\%, 30\%,
20\% y 20\% respectivamente nos da 302 + 302 +202 + 202 = 2600.

Lo anterior se puede resumir y expresar matemáticamente de la siguiente
manera: IH = σ n i=1 s2 i . Varía:

0 ≤ IH ≤ 10. 000, O sea : 0 ≤ σn i=1 s2 i ≤ 10.000

El IH se expresa como el cuadrado de la suma de las cuotas de mercado.

\begin{itemize}
\item
  Un monopolio pleno alcanzaría el máximo. O sea IH = 10.000. Caso de
  una sola empresa, 100\% del mercado, IHH = 1002 , es decir diez mil
  (10.000).
\item
  Un duopolio con dos empresas iguales alcanzaría 5.000. Caso de dos
  empresas con el 50\% de cuota de mercado cada una, IHH = 502 x 2 =
  2.500 x 2 = 5.000
\item
  Cinco empresas iguales, se alcanzaría el valor 1.600. Con el 20\% de
  cuota de mercado cada una, resulta IHH = 202 x 4 = 1.600
\item
  Cien empresas iguales proporcionarían el valor 100. Sus cuotas de
  mercado serían del 1\%, de manera que: IHH = 12 x 100 = 100
\end{itemize}

\hypertarget{valor-institucional}{%
\subsection{VALOR INSTITUCIONAL}\label{valor-institucional}}

El IH ha obtenido una gran difusión y respaldo como resultado de su
utilización en el control de las operaciones de concentración de
empresas en una determinad industria en los EEUU. En este sentido se
tiene que:

\begin{itemize}
\item
  Sí el IH tiende a cero (0), entonces es de competencia perfecta.
\item
  Si el IH está por debajo de 1000 puntos el sector no se considera
  concentrado
\item
  Sí el IH se encuentra entre1000 y 1800 puntos se considera una
  concentración moderada puede ser el caso de competencia monopolística
\item
  SÍ 1800 \textless{} IH \textless{} 10,000es 1800 puntos el sector se
  considera concentrado, y sería el caso de competencia oligopolística.
\item
  Sí el IH es igual a 10.000, entonces la empresa es un monopolio
  perfecto.
\end{itemize}

Por esta razón, dado un número de empresas n, el índice toma un valor
mayor cuanto más asimétricas sean las empresas. El valor mínimo lo toma
cuando todas las empresas tienen la misma cuota y el valor máximo toma
cuando toda la producción se concentra en una empresa. En el primer caso
vale (1/n) y en el segundo vale 1.

Una forma alternativa de hallar el índice de Hirfindahl, es calculando
la varianza que existe entre las cuotas de mercado

2 1 IH = nσ +n

por tanto, el índice de Herfindahl crece cuando el número de empresas
cae y si aumenta la varianza de las cuotas de mercado.

2 1 Tarea: demostrar que el IH = nσ + n

\hypertarget{indice-de-herfindahl}{%
\subsection{INDICE DE HERFINDAHL}\label{indice-de-herfindahl}}

TABLAAA

En la tabla tenemos 5 situaciones diferentes de mercado, todas con
concentraciones diferentes, sin embargo, el índice C4 resulta en el
valor máximo siempre. En cambio, el índice HHI da un peso diferente a
situaciones distintas. Así, el mercado más concentrado es S5, mientras
que el menos es S3, pero todos son mercados concentrados. La única
referencia internacional para este tipo de indicadores es el realizado
por el Departamento de Justicia de USA y en donde se clasifica a los
mercados en:

\begin{itemize}
\tightlist
\item
  Industrias poco concentradas: 0 \textless{} HHI \textless{} 10,000. O
  sea, hasta diez empresas de igual tamaño.
\end{itemize}

\hypertarget{indice-de-lerner}{%
\subsection{INDICE DE LERNER}\label{indice-de-lerner}}

Es un indicador que busca medir el poder (monopólico) de una empresa a
través de la diferencia entre el precio que una empresa carga por sus
productos y el costo marginal de producción. En una economía
perfectamente competitiva, donde las empresas son idénticas las
cantidades vendidas deberían ser iguales, para un nivel de precios que
es igual al costo marginal.

P −CMg IL =

En la practica, tanto el índice de Lerner y el índice de Herfindahl nos
permite medir el ``grado de relación de la concentración de n
empresas'', y en general son utilizados para realizar estudios del grado
de competitividad en una economía.

El índice de Lerner varía entre 0 \textless{} \textless{} 1 P

\hypertarget{uxedndice-de-inestabilidad}{%
\subsection{ÍNDICE DE INESTABILIDAD}\label{uxedndice-de-inestabilidad}}

Es un índice que mide la forma como va cambiando las estructuras del
mercado en el tiempo. Es decir, mide la forma de como van alterándose
las proporciones de cuotas de participación entre el período t-1 y el
período t . Lo que se calcula con La siguiente formula:

II = σN i=1 S i(t) − S i(t−1)

Si el índice de inestabilidad tiende a cero (0), entonces todas las
empresas en el tiempo mantienen su cuota de participación en el mercado.
En cambio, si el índice de inestabilidad tiende a uno (1), entonces hay
una máxima inestabilidad en el mercado. O sea, todas las empresas
presentes en el mercado en el período t-1, tienen cuotas de mercado
bajas o nulos en el periodo presente t.

\hypertarget{uxedndice-de-entropia}{%
\subsection{ÍNDICE DE ENTROPIA}\label{uxedndice-de-entropia}}

Otra forma de medir la concentración es a través del índice de entropía,
el cual es igual a la sumatoria de las cuotas de mercado multiplicadas
por sus respectivos logaritmos. Lo que se estima mediante la siguiente
formula:

IE = σn i=1 si ln si

Este índice tiene mucha aplicación en las ciencias físicas, el concepto
de entropía se refiere al grado de desorden que tiene un sistema físico.
En este sentido si hacemos una analogía para el análisis de la industria
o de un mercado, el Índice de entropía muestra el grado de imperfección
o concentración habido en el mismo. En el caso de un monopolio este
índice sería cero (0), y mientras más competitivo sea el mercado dicho
índice será mayor (en valores absolutos).

\hypertarget{el-inverso-del-uxedndice-de-herfindahl}{%
\subsection{EL Inverso del índice de
Herfindahl}\label{el-inverso-del-uxedndice-de-herfindahl}}

El inverso del índice de Herfindahl representa el número hipotético de
empresas del mismo tamaño que compartirían el mercado, y se le conoce
como el número de equivalentes en el mercado (NEQ). En un mercado en el
cual las empresas son de igual tamaño, la varianza de su participación
en el mismo es cero (0). Por lo tanto, se tiene que:

entonces, n = 1/H, demostrándose que para mercados equidistribuidos, el
índice de Herfindahl inverso representa el número de empresas que lo
comparten. Para este caso, la interpretación del índice es que cuanto
más cercano sea el valor obtenido al número efectivo de participantes,
más balanceada es la distribución del mercado y más cerca están los
participantes del tamaño óptimo de operación. Cuanto más difieran ambos
valores mayor es la probabilidad de que se estén presentando
ineficiencias dentro del mismo.

\hypertarget{indice-de-rosembluth-hall-y-tideman}{%
\subsection{INDICE DE ROSEMBLUTH, HALL Y
TIDEMAN}\label{indice-de-rosembluth-hall-y-tideman}}

Es un índice que mide el grado de concentración industrial y
consecuentemente el nivel de competencia en el mercado. Su valor se
calcula mediante la siguiente formula:

1 RHT = {[}2 σN i=1 iSi {]} −1

Donde:

\begin{itemize}
\item
  S i es la cuota de participación de la empresa i--ésima, ordenadas de
  mayor a menor
\item
  i, es el rango de la empresa i-ésima en la industria
\item
  N, es el número de empresas en el mercado.
\end{itemize}

El índice de Rosembluth varía entre: (1/N) ≤ RHT ≤ 1.

Sí RHT es igual a uno (1) es Monopolio. Si RHT tiende a uno (1) sin
serlo, entonces existe una alta concentración con baja competencia. Sí
RHT tiende a cero, entonces hay una baja concentración industrial y una
alta competencia en el mercado, finalmente, sí RHT = 0, entonces la
concentración es nula y el mercado es de competencia es perfecta.

\hypertarget{el-coeficiente-de-gini}{%
\subsection{EL COEFICIENTE DE GINI}\label{el-coeficiente-de-gini}}

El coeficiente de Gini fue diseñado con el fin de estimar la
concentración poblacional. Sin embargo, su uso ha sido adoptado a la
concentración del mercado. El coeficiente de Gini, se calcula mediante
un análisis geométricamente. En un eje de coordenadas cartesianas se
elabora la Curva de Lorenz utilizando en el eje vertical para el número
de empresas en proporción al total y en el eje horizontal la variable
(Ventas o Producción) para la cual se desea calcular la concentración,
también en proporción al total de empresas o entidades participantes. De
esta manera, si la variable para la cual se desea estimar la
concentración representa los recursos del sistema, cada punto (x,y) de
la curva de Lorenz podrá ser interpretado como el ``y'' porcentaje del
total de las empresas que concentra el ``x'' porcentaje de ventas o
producción en el mercado. La recta de 45 grados representa la
distribución perfecta de las ventas o niveles de producción del mercado
realizados por el número de empresas existentes, ya que para todos los
puntos de la recta, x es igual a y.

EL COEFICIENTE O INDICE DE GINI

N+1 −2 σN I=1 iSi G =

Otra forma de calcular el coeficiente de Gini, es a través de la sigte
formla:

1 G = 1 - N(RHT)

Donde:

\begin{itemize}
\item
  S i es la cuota de participación en el mercado de i-ésima empresa
\item
  i, es el rango de la empresa i-ésima en la industria
\item
  N es el número de empresas en el mercado industrial.
\end{itemize}

El índice de Gini varía entre 0 ≤ G ≤ 1.

\begin{itemize}
\item
  Sí G tiende a cero, entonces hay una baja concentración industrial y
  una alta competencia en el mercado.
\item
  Sí G tiende a uno (1), entonces existe una alta concentración
  industrial y una baja competencia en el mercado.
\end{itemize}

También se puede decir que el valor del índice de Gini fluctúa entre
(1/n) ≤ G ≤ 1.

Otra manera de calcular el coeficiente de Gini es a través de la
siguiente expresión matemática:

1 ׬ X −f X dX IG = 0 (12) 5

\hypertarget{indice-de-hannah-y-kay}{%
\subsection{INDICE DE HANNAH Y KAY}\label{indice-de-hannah-y-kay}}

En un índice que se calcula sobre la totalidad de empresas que operan en
la industria. Pues, considera mucha información sobre el conjunto de
puntos de la curva de concentración, estimándose su valor mediante la
siguiente formula:

1 IHK = σ N i=1{[}si ∝ {]} ∝−1

Donde ∝\textgreater0, es la ponderación que otorga a la diferentes
empresas de acuerdo con su cuota de participación relativa en el
mercado. En este sentido, cuanto mayor sea ∝, la concentración será más
alta. Por el contrario si ∝ tiende a cero (0) entonces el IHK = (1/N),
en este caso todas las empresas son iguales. En cambio, si IHK tiende al
infinito, entonces el grado de concentración es igual a su cuota de
participación (si).

En condiciones adecuadas el índice de Hannah Kay varía entre:

1 (1/N) ≤ σN i=1{[}si ∝ {]} ∝−1 ≤ 1.

\hypertarget{indice-de-dominaciuxf3n}{%
\subsection{INDICE DE DOMINACIÓN}\label{indice-de-dominaciuxf3n}}

Es un índice que mide que tan dominado está un mercado por la empresa
más grande. El coeficiente se calcula con la siguiente formula:

2 S2ID = σn i=1 iIH

Donde:

\begin{itemize}
\item
  Si es la cuota de participación en el mercado de la i-ésima empresa.
\item
  n es el número de empresas participantes en el mercado
\item
  IH es el índice de herfintadal
\end{itemize}

Los valores más altos del ID nos indica la mayor participación de la
empresa dominante, el mismo que se utiliza para autorizar la ejecución
de fusiones.

INDICE DE DOMINACIÓN

Los índices de dominación se aplican al análisis de las fusiones, el
cual nos permite distinguir situaciones de concentración de aquellas
empresas que tienen el poder monopólico de aquellas que no tienen.

Supóngase un índice de concentración común para evaluar estructuras,
como es el índice de Herfindahl (H). Este, al igual que los otros
índices tradicionales de concentración, tiene el inconveniente de que
con cualquier fusión aumenta su valor; En efecto es posible que se den
fusiones con el objetivo de aumentar la competitividad de las empresas.
Esto podría ocurrir en el caso de dos empresas de moderado tamaño, en un
mercado dominado por una empresa que concentra la mayor parte de la
producción de la industria; en cuyo caso la fusión contribuiría a una
mayor capacidad de respuesta ante decisiones unilaterales por parte de
la empresa grande. Por lo advertido, se requiere un indicador que no
penalice indiscriminadamente cualquier fusión, sino que el resultado
tome en cuenta el tamaño relativo de las empresas concentradoras, o que
se fusionan, y las particularidades del mercado respectivo. Dicho
índice, por ejemplo, no debería aumentar con las fusiones de empresas
relativamente pequeñas, pero sí con las fusiones de empresas
relativamente grandes.

\hypertarget{indice-de-linda}{%
\subsection{INDICE DE LINDA}\label{indice-de-linda}}

Es un índice que mide la desigualdad entre las cuotas de mercado de las
empresas, pero agrupadas en dos grupos en función de su tamaño y se
calcula mediante la siguiente formula:

1 N−1 X ഥ m L = N(N−1) σm=1 ഥXN−m

Donde:

\begin{itemize}
\item
  X ഥ m es la cuota de participación promedio de las m primeras empresas
  de la industria
\item
  X ഥ N−m es la cuota de participación promedio de las N-m restantes
  empresas de la industria
\item
  N es el número de empresas en el mercado o industria.
\end{itemize}

Sí L \textless{} que 0.20, entonces el mercado es desconcentrado con
alta competencia

Si 0.20 \textless{} L ≤ 0.50, entonces el mercado es moderadamente
concentrado y también con una moderada competencia Sí 0.50 \textless{} L
\textless{} 1, entonces existe una alta concentración con baja
competencia

Sí L = 1, entonces el mercado es muy concentrado y con posiciones de
dominio de mercado, que incluso se puede decir monopolio puro.

\hypertarget{mediciuxf3n-de-barreras}{%
\section{MEDICIÓN DE BARRERAS}\label{mediciuxf3n-de-barreras}}

\hypertarget{definiciuxf3n-de-barreras}{%
\subsection{Definición de barreras}\label{definiciuxf3n-de-barreras}}

``Las barreras a la entrada son aquellas situaciones y condiciones que
impiden o desalientan la entrada de nuevas empresas a un mercado
industrial, a pesar de que las empresas participantes en ella están
obteniendo beneficios económicos positivos lucrativos''.

\hypertarget{tipos-de-barreras-a-la-entrada-del-mercado}{%
\subsection{TIPOS DE BARRERAS A LA ENTRADA DEL
MERCADO}\label{tipos-de-barreras-a-la-entrada-del-mercado}}

BARRERAS LEGALES BARRERAS

\begin{itemize}
\item
  CONTROL Y REGULACIÓN A TRAVÉS DE NORMAS GUBERNAMENTALES
\item
  SISTEMAS ECONOMICOS CON DEFENZA DE COMPETENCIA
\item
  PATENTES Y FANQUICIAS
\end{itemize}

NATURALES BARRERAS

\begin{itemize}
\item
  TECNOLOGÍAS Y ECONOMIAS A GRAN ESCALA
\item
  GRANDES MAGNITUDES DE CAPITAL EN (I +D)
\item
  FIJACIÓN DE PRECIOS QUE DESALIENTAN LA ENTRADA
\end{itemize}

ESTRATEGICAS

\begin{itemize}
\item
  ECONOMIAS DE ESCALA
\item
  DIFERENCIACIÓN DEL PRODUCTO
\item
  INVERSIONES DE CAPITAL
\item
  DESVENTAJA DE COSTOS INDEPENDIENTEMENTE DE LA ESCALA
\item
  ACCESO A LOS CANALES DE DISTRIBUCIÓN
\item
  POLÍTICA GUBERNAMENTAL
\end{itemize}

OTRAS BARRERAS

\begin{itemize}
\tightlist
\item
  LICENCIAS ESPECÍFICAS DE FUNCIONAMIENTO Y PROTECCIÓN DE AUTORÍAS
\end{itemize}

\hypertarget{eficiencia-y-generaciuxf3n-de-excedentes}{%
\section{EFICIENCIA Y GENERACIÓN DE
EXCEDENTES}\label{eficiencia-y-generaciuxf3n-de-excedentes}}

En economía, se dice que una situación es eficiente si no resulta
posible mejorar el bienestar de alguna persona sin empeorar el de alguna
otra.

Esta definición se le atribuye al Italiano Wilfredo Pareto- 1909, por lo
que comúnmente se le conoce como la ``eficiencia en el sentido de
Pareto'' u ``óptimo de Pareto''. Aún cuando esta definición es bastante
general, se puede relacionar su aplicabilidad a una situación en la cual
la suma de los beneficios de los consumidores y de las empresas se hace
máxima. A esto se le conoce como ``enfoque de equilibrio parcial''.

A fin de cuantificar --al menos teóricamente-- la eficiencia de un
mercado, resulta necesario identificar los beneficios de quienes
participan en él. Para ello se apela a dos conceptos básicos: el valor
que tienen para los consumidores los bienes o servicios producidos y
vendidos, y el costo que tiene para las empresas producir y vender
dichos bienes o servicios.

Es decir, calculando el excedente total de la economía, que viene a ser
igual a la suma del excedente total de consumidores y productores, tal
como:

P máx P oEXTt = EXCt + EXPt = \{׬ P0 D(PQ )dPQ - Po Qo \} + ׬ Pm S(PQ
)dPQ Qo Qo

EXTt = BSBt − CSBt =׬ 0 D Q dQ - ׬ 0 S Q dQ

De esta relación, lo más importante es el beneficio máximo de los
productores.

\hypertarget{beneficio-maximo-de-las-empresas}{%
\subsection{BENEFICIO MAXIMO DE LAS
EMPRESAS}\label{beneficio-maximo-de-las-empresas}}

a: C = C(Q) + CF

Si se define el ingreso total igual a: IT = PQ

Y, ambas ecuaciones son continuas y diferenciables y las estructuras del
mercado son de competencia perfecta e imperfecta, los resultados varían
entre ellas. En efecto, simulemos la maximización de beneficios para
competencia perfecta e imperfecta.

Ejemplo:

\begin{enumerate}
\def\labelenumi{\arabic{enumi}.}
\item
  Supongamos que una industria de bienes homogéneos posee la siguiente
  función de demanda Q = 100 -- P. En el mercado existen N empresas con
  costos marginales idénticas e iguales a CMg = 2. a). Calcule el
  precio, la cantidad producida y el máximo beneficio, si compiten
  simultáneamente, b). Hallar el BSB, para el nivel de producción de
  equilibrio, c). Hallar el CSB, para el nivel de producción de
  equilibrio, d). Calcular la ganancia total de la economía para el
  nivel de producción de equilibrio. Graficar con los resultados
  obtenidos.
\item
  Suponga que cada empresa tiene unos costos totales C = 10 + 2qi; y que
  ambas empresas estiman que su demanda conjunta es igual a P = 320 --
  2(q1 + q2 ), a).¿Cuales son las funciones de beneficios de las
  empresas y sus funciones de reacción si tienen la conducta del modelo
  de Curnot?,
\end{enumerate}

\hypertarget{medidas-de-resultados-o-desempeuxf1o}{%
\section{MEDIDAS DE RESULTADOS O
DESEMPEÑO}\label{medidas-de-resultados-o-desempeuxf1o}}

\hypertarget{maximizaciuxf3n-de-beneficios-de-una-empresa-e-industria-perfectamente-competitiva}{%
\subsection{Maximización de beneficios de una empresa e industria
perfectamente
competitiva}\label{maximizaciuxf3n-de-beneficios-de-una-empresa-e-industria-perfectamente-competitiva}}

a). Maximización de beneficios de una empresa perfectamente competitiva

a.1). Corto lazo:

IT = IT(q) = P ഥ q

C = C(q) -- CF

B =IT -- C

B = P ഥ q − C(q) -- CF

∂IT ∂C

El Bmáx si BMg = 0 ↔ − = 0

∂q ∂q

IMg - CMg = 0

P = CMg

IMe = CMg

D( P ഥ ) = CMg

Donde: {[} ഥ P = Img = IMe = D(P ഥ ) =m de IT.{]} = CMg

a.2). Largo Plazo

IT = IT(q) = P ഥ q

CLP = C(q)LP

B =IT -- CLP

B = P ഥ q − C(q)LP

∂IT ∂CLP

El Bmáx si BMgLP = 0 ↔ − = 0 ∂q ∂q

IMg - CMgLP = 0 ഥ

P = CMgLP

IMe = CMgLP

D(P ഥ ) = CMgLP

Donde: {[} ഥ P = Img = IMe = D(P ഥ ) =m de IT.{]} = CMgLP condición debe
cumplirse que: CMeCP = CMeIP

b). MAXIMIZACIÓN DE BENEFICIOS DE UNA INDUSTRIA PERFECTAMENTE
COMPETITIVO

b.1). En el corto plazo

IT = IT(Q) = P ഥ Q

C = C(Q) -- CFI

B =IT -- C

B = P ഥ Q − C(Q) -- CFI

∂IT(Q) ∂C(Q) El Bmáx si BMgI = 0 ↔ − = 0

∂Q ∂(Q)

IMgI - CMgI = 0

P = CMgI

P ഥ = σn i=1 CMgi

IMe = σn i=1 CMgi

D( P ഥ ) = σn i=1 CMgi

Donde: {[} ഥ P = Img = IMe = D( P ഥ ) =m de IT.{]} = σn i=1 CM gi = CMgI

b.2). En el largo plazo

En el largo plazo las empresas de una industria esta obteniendo
beneficio nulos o normales. Es decir, la diferencia entre sus ingresos y
costos son iguales a cero (0) y por tanto están obteniendo beneficios
normales y por tanto están cubriendo sus costos operativos y
financieros. Sin embargo aquellas empresas que no cubren sus costos
pueden cerrar sus empresas o liquidar y, en cambio otras optimistamente,
esperarán a que los precios aumente en el futuro inmediato. Las
condiciones de equilibrio son los siguientes:

IMg = CMgLP = CMgCP = CMeLP = CMeLP.

O sea, las cinco medidas coinciden en un punto, lo que le permite
obtener beneficios nulos. Las empresas que sobreviven a esta situación
esperan una reacción inmediata de la demanda, lo que posibilitaría a que
la industria vuelva ser rentable, permitiendo el retorno de las antiguas
y el ingreso de nuevas empresas. Sí el precio de los factores
productivos son constantes, por que la demanda de los mismos es
imperceptible, entonces la entrada de empresas será hasta que
desaparezcan los beneficios lucrativos. Siendo la oferta de largo plazo
una línea horizontal.

b.2). En el largo plazo

Sin embargo, si los precios de los factores de producción varían, y su
uso es relativamente alta, entonces los costos variables aumentan y
consecuentemente, se desplazan hacia arriba y hacia la izquierda, en
tanto que las empresas reducen sus ofertas hasta que nuevamente, logren
obtener beneficios nulos. En este caso la curva de oferta tiene una
pendiente positiva y es el resultado de unir los puntos de equilibrio.

\hypertarget{maximizaciuxf3n-de-beneficios-de-una-empresa-e-industria-monopolica}{%
\subsection{MAXIMIZACIÓN DE BENEFICIOS DE UNA EMPRESA E INDUSTRIA
MONOPOLICA}\label{maximizaciuxf3n-de-beneficios-de-una-empresa-e-industria-monopolica}}

En el corto plazo

En el corto, tanto la empresa como la industria monopólica (Monopolio
puro) maximiza beneficios y lo hace cuando el Beneficio marginal (BMg)
es igual a cero. Es decir, cuando:

IT = IT(Q) = PQ = AQ - aQ2

C = C(Q) -- CF

B =IT -- C

B = PQ − C(Q) -- CF

∂IT ∂C

El Bmáx si BMg = 0 ↔ − = 0 ∂Q ∂Q

IMg - CMg = 0

1 P(1 - ) = CMgɳ PX 1

Donde: {[}IMg = P(1 - ) {]} = CMg

b). En el largo plazo

Dado que el monopolio es el que fija precios en el largo, no hay riesgos
de la entrada de nuevas empresas por las restricciones naturales,
legales, y tecnológicas por un lado; y por otra, no aparezcan nuevos
productos sustitutos, el monopolista puede continuar disfrutando de los
beneficios lucrativos.

Sin embargo, ante la aparición de un nuevo producto, el monopolista
terminaría sobreviviendo en el mercado obteniendo beneficios nulos,
hasta incluso terminaría por cerrar el negocio.

\hypertarget{el-monopolista-y-la-discriminaciuxf3n-de-precios}{%
\section{EL MONOPOLISTA Y LA DISCRIMINACIÓN DE
PRECIOS}\label{el-monopolista-y-la-discriminaciuxf3n-de-precios}}

El monopolista con la finalidad de aumentar o mejorar sus beneficios
extraordinarios ue llevar a cabo la practica de la discriminación de
precios; que consiste en vender el mismo producto a diferentes precios.
La discriminación de precios solo es posible sí i). Si la empresa tiene
poder monopólico; ii) sí Los compradores o consumidores no tienen la
posibilidad de practicar el arbitraje o la ley de un solo precio y iii)
los compradores no tienen información perfecta sobre precios y niveles
de producción.

Los monopolistas hacen practica de tres tipos de discriminación de
precios:

\begin{itemize}
\item
  La discriminación Perfecta de precios o de primer grado
\item
  La discriminación de precios de segundo grado y,
\item
  La discriminación de precios de tercer grado
\end{itemize}

\hypertarget{la-discriminaciuxf3n-perfecta-de-precios-o-de-primer-grado}{%
\subsection{LA DISCRIMINACIÓN PERFECTA DE PRECIOS O DE PRIMER
GRADO}\label{la-discriminaciuxf3n-perfecta-de-precios-o-de-primer-grado}}

Sólo es posible si cada comprador o cada grupo de compradores haya sido
identificado por separado por sus potencialidades, situación que le
permite al monopolista cobrarle a cada comprador o grupo de compradores
el precio máximo que esté dispuesto a pagar por el bien, (precio
diferente a cada comprador o grupo) hasta agotar sus respectivos
excedentes. Esta situación le permite al monopolista bajar el precio
moviéndose a lo largo de la curva de demanda, hasta que la última unidad
vendida sea igual al costo marginal de haber producido dicha unidad. O
sea, que el precio (P) sea igual al Costo Marginal (CMg).

Ejemplo: Supongamos que la demanda de un monopolista es igual a P = 105
-- (3/8)Q y el 1 1costo del monopolista es de C = Q3 − Q2100Q + 16. Si
el monopolista puede 32 4 discriminar perfectamente entre sus clientes,
cual será el rango de precios, la cantidad vendida y el máximo beneficio
para al menos 8 potenciales consumidores.

\hypertarget{discriminaciuxf3n-de-precios-de-segundo-grado}{%
\subsection{DISCRIMINACIÓN DE PRECIOS DE SEGUNDO
GRADO}\label{discriminaciuxf3n-de-precios-de-segundo-grado}}

Solo es posible si el monopolista cobra distintos precios a los
compradores del bien y lo hace por bloques o lotes de bienes. O sea por
cada bloque o lote negociado le cobra precios diferentes hasta agotar
las posibilidades de compra del consumidor, situación que le permite
bajar el precio por lotes hasta que el precio del ultimo lote vendido
sea igual al costo marginal de haber producido.

Ejemplo: supongamos que el costo de producción del monopolista es igual
a C = 0.05Q2 + 10,000, con el que enfrenta una función de demanda igual
a: P = 100 -- 0.05Q, supongamos que el monopolista discrimina precios en
segundo grado vendiendo lotes de bienes de las siguientes cantidades:
lote 01 entre 0 a 100, lote 02 = 200, lote 03 = 350, lote 04 = 500 y,
lote 05 = 667 unidades respectivamente. Se pide calcular el máximo
beneficio de la empresa.

\hypertarget{discriminaciuxf3n-de-precios-de-tercer-grado}{%
\subsection{DISCRIMINACIÓN DE PRECIOS DE TERCER
GRADO}\label{discriminaciuxf3n-de-precios-de-tercer-grado}}

Un monopolista discrimina precios en tercer grado, cuando vende el mismo
producto a precios distintos y en diferentes mercados, para lo cual debe
cumplirse además de las condiciones descritas las siguientes
condiciones: iv) que la empresa tenga un poder monopólico absoluto v)
que el monopolista tenga la capacidad de segmentar su mercado en
escenarios de espacios diferentes.

El monopolista maximiza beneficios seleccionando aquel nivel de
producción total para el cual la σn i=1 CMg = CMg de proveer el bien en
todos los mercados. O, cuando el IMg de cada mercado es igual al mismo
costo marginal de proveer el bien.

CMg1 = IMg2 = \ldots{} = IMgN = CMg

1 1 1 P1 1 − = P2 1 − = . . . = Pn 1 − = CMg. ɳpx1 ɳpx2 ɳpxn

Supongamos que un monopolista enfrenta a las siguiente funciones de
demanda de dos mercados: P 1 = 100 - 2Q 1 y P 2 = 180 - 3Q 2 , cuando su
costo total es igual a C = 400 + 3Q 2 . Calcular los precios y la
cantidad vendida en cada mercado y el máximo beneficio de la empresa.

\hypertarget{contro-y-reglamentaciuxf3n-del-monopolio}{%
\subsection{CONTRO Y REGLAMENTACIÓN DEL
MONOPOLIO}\label{contro-y-reglamentaciuxf3n-del-monopolio}}

Controlar y reglamentar el comportamiento del monopolista a través de:

Mediante la aplicación de impuestos

\begin{itemize}
\item
  Impuesto de suma fija aplicado directamente a la empresa T = To
\item
  Impuesto específico por unidad producida T =TQ
\item
  Impuesto Ad -- Valoren o porcentual sobre los beneficios T = tB
\item
  Impuesto porcentual sobre las ventas T = tIT
\end{itemize}

Mediante la fijación de precios máximo Lo hace a través de dos
criterios:

\begin{itemize}
\item
  Fijando el precio igual al Costo marginal: P = CMg
\item
  Fijando el precio igual al costo medio total: P = CMeT
\end{itemize}

\hypertarget{maximizacion-de-beneficios-de-una-empresa-e-industria-monopolistica}{%
\subsection{MAXIMIZACION DE BENEFICIOS DE UNA EMPRESA E INDUSTRIA
MONOPOLISTICA}\label{maximizacion-de-beneficios-de-una-empresa-e-industria-monopolistica}}

a). MAXIMIZACION DE BENEFICIOS DE LA EMPRESA E INDUSTRIA MONOPOLISTICA
EN EL CORTO PLAZO

a.1) Beneficios máximos de la empresa en el corto plazo

Cada empresa al enfrentar una demanda particular en la demanda total del
mercado, maximiza beneficios seleccionando un nivel de producción para
el cual el beneficio marginal es igual cero. Es decir:

IT = IT(q) = Pq

C = C(q) -- CF

B =IT -- C

B = Pq − C(q) -- CF

∂IT ∂C El Bmáx si BMg = 0 ↔ − = 0

∂q ∂q Img - CMg = 0

1 P(1 - ) = CMg

a.2) Beneficos máximos de la industria en el corto plazo: La industria
maximiza beneficios vendiendo cantidades a lo largo de la curva de
demanda total

b) EQUILIBRIO DE LA EMPRESA E INDUSTRIA EN EL LARGO PLAZO:

Las empresas y la industria monopolística en el largo plazo obtienen
beneficios nulos o normales, por que reajustan sus plantas de producción
al nuevo escenario del mercado donde las funciones de demanda para cada
monopolista se desplaza hacia abajo y hacia el origen, permitiéndoles
capturar un precio más bajo y para una cantidad mayor de ventas, en
consecuencia esto ocurre cuando:

IMgi = CMg i = Pi = D(P)i = CMgCPi = CMgLPi = CMeCPi = CMeLPi

\hypertarget{medidas-de-resultados-o-desempeuxf1o-1}{%
\section{MEDIDAS DE RESULTADOS O
DESEMPEÑO}\label{medidas-de-resultados-o-desempeuxf1o-1}}

\hypertarget{beneficios-economicos-o-rentabilidad-sobre-la-inversiuxf3n}{%
\subsection{BENEFICIOS ECONOMICOS O RENTABILIDAD SOBRE LA
INVERSIÓN}\label{beneficios-economicos-o-rentabilidad-sobre-la-inversiuxf3n}}

La rentabilidad mide la eficiencia con la cual una empresa utiliza sus
recursos físicos, naturales y humanos.

¿Qué significa esto? Decir que una empresa es eficiente es decir que no
desperdicia recursos, implica que la empresa optimiza la productividad
de sus recursos de manera eficiente con el objetivo de maximizar
beneficios o simplemente obtener beneficios.

En términos monetarios estos recursos son, por un lado, el capital (que
aportan los accionistas) y, por otro, la deuda (que aportan los
acreedores). Si una empresa utiliza recursos financieros muy elevados
pero obtiene unos beneficios pequeños, pensaremos que ha
``desperdiciado'' recursos financieros: ha utilizado muchos recursos y
ha obtenido poco beneficio con ellos. Por el contrario, si una empresa
ha utilizado pocos recursos pero ha obtenido unos beneficios
relativamente altos, podemos decir que ha ``aprovechado bien'' sus
recursos. Por ejemplo, puede que sea una empresa muy pequeña que, pese a
sus pocos recursos, está muy bien gestionada y obtiene beneficios
elevados. En realidad, hay varias medidas posibles de rentabilidad, pero
todas tienen la siguiente forma:

\hypertarget{el-valor-actual-neto}{%
\subsection{EL VALOR ACTUAL NETO}\label{el-valor-actual-neto}}

Es una medida de rentabilidad de la inversión que nos permite calcular
el valor presente de un determinado número de flujos de ingresos y
costos futuros originados por una inversión. La metodología consiste en
descontar al momento actual (es decir, actualizar mediante una tasa)
todos los flujos de caja (cash-flow) futuros o en determinar la
equivalencia en el tiempo 0 de los flujos de efectivo futuros que genera
un proyecto y comparar esta equivalencia con el desembolso inicial.
Dicha tasa de actualización (k) o de descuento (d) es el resultado del
producto entre el coste medio ponderado de capital (CMPC) y la tasa de
inflación del periodo. Cuando dicha equivalencia es mayor que el
desembolso inicial, entonces, es recomendable que el proyecto sea
aceptado. Se calcula mediante la siguiente formula:

N B t VAN = - I + σ t=1 t

1+kVAN = BNA -- I

La aproximación de Schneider usa el teorema del binomio para obtener una
formula de primer orden

\hypertarget{tasa-interna-de-retorno-tir}{%
\subsection{TASA INTERNA DE RETORNO
(TIR)}\label{tasa-interna-de-retorno-tir}}

La tasa interna de retorno o tasa interna de rentabilidad (TIR) de una
inversión es el promedio geométrico de los rendimientos futuros
esperados de dicha inversión, y nos indica la posibilidad de la
existencia de una oportunidad de inversión o reinversión. Otros autores
en términos simples lo definen la TIR, como aquella tasa de descuento
con la que el valor actual neto (VAN) o valor presente neto (VPN) es
igual a cero.

La TIR puede utilizarse como indicador de la rentabilidad de un
proyecto: a mayor TIR, mayor rentabilidad. En efecto, la TIR es un
criterio para decidir si se acepta o rechaza un proyecto de inversión.

Para ello, la TIR se compara con una tasa mínima o tasa de corte,
llamado coste de oportunidad de la inversión.

Otras definiciones:

Es la tasa que iguala la suma del valor actual de los ingresos con la
suma del valor actual de los costos previstos: σN i=1 VPI i = σ n i=1
VPC i

Es la tasa de interés para la cual los ingresos totales actualizados es
igual a los costos totales actualizados: ITac = Ctac.

Es la tasa de interés máxima a la que se pueden endeudar para no perder
dinero con la inversión.

\hypertarget{el-uxedndice-de-lerner-en-la-mediciuxf3n-del-desempeuxf1o}{%
\subsection{EL ÍNDICE DE LERNER EN LA MEDICIÓN DEL
DESEMPEÑO}\label{el-uxedndice-de-lerner-en-la-mediciuxf3n-del-desempeuxf1o}}

El Índice de Lerner, es un indicador que se utiliza para medir que tanto
poder tiene la empresa para fijar sus precios

P −CMg IMe −CMg IL =

IL =

P IMe 1 1 CMg = IMg = P(1 - )

CMg = IMg = P(1 - )

ɳpd ɳpd P −P(1 − ɳpd) 1 IMe −IMe(1 − ɳpd 1 ) IL =

IL =

P IMe P{[}1 −(1 − 1 ){]} ɳpdIL = P 1 IL = ɳpxd

IMe{[}1 −(1 − 1 ) ɳpdIL = IMe 1 IL = ɳpxd LA Q DE TOBIN EN

\hypertarget{la-mediciuxf3n-del-desempeuxf1o}{%
\subsection{LA MEDICIÓN DEL
DESEMPEÑO}\label{la-mediciuxf3n-del-desempeuxf1o}}

\hypertarget{indice-de-retorno-sobre-los-activos-roa}{%
\subsection{INDICE DE RETORNO SOBRE LOS ACTIVOS
(ROA)}\label{indice-de-retorno-sobre-los-activos-roa}}

Es un índice que mide la rentabilidad de una empresa con respecto a los
activos que posee una empresa y nos indica contablemente, que tan
eficiente es una empresa en el uso de sus activos para generar
utilidades. Se calcula mediante la siguiente formula:

ROA = UTILIDADES ACTIVOS 100 6.000

Ejemplo: ROA = 100 = 20. 30.000 Estos significa que la empresa esta
obteniendo una rentabilidad de 20\% con respecto a sus activos. Es decir
por cada unidad monetaria en estado activo, se obtiene una rentabilidad
de 20\%. En este sentido mientras mayor sea el ROA más rentable es
considerada la empresa pues genera más utilidades con menos recursos.

Los activos de una empresa es igual:

\hypertarget{uxedndice-de-retorno-sobre-el-patrimonio-roe}{%
\subsection{ÍNDICE DE RETORNO SOBRE EL PATRIMONIO
(ROE)}\label{uxedndice-de-retorno-sobre-el-patrimonio-roe}}

Es un índice que mide la rentabilidad con respecto al patrimonio que
posee la empresa. Y, nos indica contablemente, que tan eficiente es una
empresa en el uso de su patrimonio para generar utilidades. SE calcula
mediante la siguiente utilidad formula: ROE = 100 patrimonio Ejemplo:

\hypertarget{indice-de-rentbilidad-sobre-las-ventas-rov}{%
\subsection{INDICE DE RENTBILIDAD SOBRE LAS VENTAS
(ROV)}\label{indice-de-rentbilidad-sobre-las-ventas-rov}}

Es un índice que mide la rentabilidad con respecto a las ventas totales
de la empresa. Y, nos indica contablemente, que tan eficiente es una
empresa en sus ventas para generar sus utilidades, por periodo de tiempo
determinado. Se expresa en términos porcentuales.

\hypertarget{publicaciones-similares}{%
\section{Publicaciones Similares}\label{publicaciones-similares}}

Aquí te recomendamos algunas publicaciones similares que podrían ser de
tu interés:

\begin{itemize}
\item
  \href{../2023-06-12-introducion-organizacion-industrial/index.qmd}{1.
  Introducción a organización industrial}
\item
  \href{../2023-06-13-empresa-como-organizacion/index.qmd}{2. La Empresa
  como Organización. Promoviendo Valores Cooperativos, Humanos y
  Sociales}
\item
  \href{../2023-06-13-sistemas-economicos/index.qmd}{3. Introducción a
  los Sistemas Económicos. Cómo se distribuyen los recursos y se
  producen}
\item
  \href{../2023-06-15-mercado-relevante-oi-cap-2/index.qmd}{4. El
  Mercado Relevante Industrial de Bienes y el Mercado Geográfico}
\item
  \href{../2023-06-16-concentracion-poder-oi-cap3/index.qmd}{5. Medidas
  de concentracion}
\item
  \href{../2023-06-17-estructura-mercado-oi-cap4/index.qmd}{6.
  Estructura de mercado}
\end{itemize}

Esperamos que encuentres estas publicaciones igualmente interesantes y
útiles. ¡Disfruta de la lectura!


\printbibliography


\end{document}
