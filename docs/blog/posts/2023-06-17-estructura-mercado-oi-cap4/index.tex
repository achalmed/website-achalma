% Options for packages loaded elsewhere
\PassOptionsToPackage{unicode}{hyperref}
\PassOptionsToPackage{hyphens}{url}
\PassOptionsToPackage{dvipsnames,svgnames,x11names}{xcolor}
%
\documentclass[
  letterpaper,
  DIV=11,
  numbers=noendperiod]{scrartcl}

\usepackage{amsmath,amssymb}
\usepackage{iftex}
\ifPDFTeX
  \usepackage[T1]{fontenc}
  \usepackage[utf8]{inputenc}
  \usepackage{textcomp} % provide euro and other symbols
\else % if luatex or xetex
  \usepackage{unicode-math}
  \defaultfontfeatures{Scale=MatchLowercase}
  \defaultfontfeatures[\rmfamily]{Ligatures=TeX,Scale=1}
\fi
\usepackage{lmodern}
\ifPDFTeX\else  
    % xetex/luatex font selection
\fi
% Use upquote if available, for straight quotes in verbatim environments
\IfFileExists{upquote.sty}{\usepackage{upquote}}{}
\IfFileExists{microtype.sty}{% use microtype if available
  \usepackage[]{microtype}
  \UseMicrotypeSet[protrusion]{basicmath} % disable protrusion for tt fonts
}{}
\makeatletter
\@ifundefined{KOMAClassName}{% if non-KOMA class
  \IfFileExists{parskip.sty}{%
    \usepackage{parskip}
  }{% else
    \setlength{\parindent}{0pt}
    \setlength{\parskip}{6pt plus 2pt minus 1pt}}
}{% if KOMA class
  \KOMAoptions{parskip=half}}
\makeatother
\usepackage{xcolor}
\setlength{\emergencystretch}{3em} % prevent overfull lines
\setcounter{secnumdepth}{-\maxdimen} % remove section numbering
% Make \paragraph and \subparagraph free-standing
\ifx\paragraph\undefined\else
  \let\oldparagraph\paragraph
  \renewcommand{\paragraph}[1]{\oldparagraph{#1}\mbox{}}
\fi
\ifx\subparagraph\undefined\else
  \let\oldsubparagraph\subparagraph
  \renewcommand{\subparagraph}[1]{\oldsubparagraph{#1}\mbox{}}
\fi


\providecommand{\tightlist}{%
  \setlength{\itemsep}{0pt}\setlength{\parskip}{0pt}}\usepackage{longtable,booktabs,array}
\usepackage{calc} % for calculating minipage widths
% Correct order of tables after \paragraph or \subparagraph
\usepackage{etoolbox}
\makeatletter
\patchcmd\longtable{\par}{\if@noskipsec\mbox{}\fi\par}{}{}
\makeatother
% Allow footnotes in longtable head/foot
\IfFileExists{footnotehyper.sty}{\usepackage{footnotehyper}}{\usepackage{footnote}}
\makesavenoteenv{longtable}
\usepackage{graphicx}
\makeatletter
\def\maxwidth{\ifdim\Gin@nat@width>\linewidth\linewidth\else\Gin@nat@width\fi}
\def\maxheight{\ifdim\Gin@nat@height>\textheight\textheight\else\Gin@nat@height\fi}
\makeatother
% Scale images if necessary, so that they will not overflow the page
% margins by default, and it is still possible to overwrite the defaults
% using explicit options in \includegraphics[width, height, ...]{}
\setkeys{Gin}{width=\maxwidth,height=\maxheight,keepaspectratio}
% Set default figure placement to htbp
\makeatletter
\def\fps@figure{htbp}
\makeatother

\KOMAoption{captions}{tableheading,figureheading}
\makeatletter
\makeatother
\makeatletter
\makeatother
\makeatletter
\@ifpackageloaded{caption}{}{\usepackage{caption}}
\AtBeginDocument{%
\ifdefined\contentsname
  \renewcommand*\contentsname{Tabla de contenidos}
\else
  \newcommand\contentsname{Tabla de contenidos}
\fi
\ifdefined\listfigurename
  \renewcommand*\listfigurename{Listado de Figuras}
\else
  \newcommand\listfigurename{Listado de Figuras}
\fi
\ifdefined\listtablename
  \renewcommand*\listtablename{Listado de Tablas}
\else
  \newcommand\listtablename{Listado de Tablas}
\fi
\ifdefined\figurename
  \renewcommand*\figurename{Figura}
\else
  \newcommand\figurename{Figura}
\fi
\ifdefined\tablename
  \renewcommand*\tablename{Tabla}
\else
  \newcommand\tablename{Tabla}
\fi
}
\@ifpackageloaded{float}{}{\usepackage{float}}
\floatstyle{ruled}
\@ifundefined{c@chapter}{\newfloat{codelisting}{h}{lop}}{\newfloat{codelisting}{h}{lop}[chapter]}
\floatname{codelisting}{Listado}
\newcommand*\listoflistings{\listof{codelisting}{Listado de Listados}}
\makeatother
\makeatletter
\@ifpackageloaded{caption}{}{\usepackage{caption}}
\@ifpackageloaded{subcaption}{}{\usepackage{subcaption}}
\makeatother
\makeatletter
\@ifpackageloaded{tcolorbox}{}{\usepackage[skins,breakable]{tcolorbox}}
\makeatother
\makeatletter
\@ifundefined{shadecolor}{\definecolor{shadecolor}{rgb}{.97, .97, .97}}
\makeatother
\makeatletter
\makeatother
\makeatletter
\makeatother
\ifLuaTeX
\usepackage[bidi=basic]{babel}
\else
\usepackage[bidi=default]{babel}
\fi
\babelprovide[main,import]{spanish}
% get rid of language-specific shorthands (see #6817):
\let\LanguageShortHands\languageshorthands
\def\languageshorthands#1{}
\ifLuaTeX
  \usepackage{selnolig}  % disable illegal ligatures
\fi
\usepackage[]{biblatex}
\addbibresource{../../../../references.bib}
\IfFileExists{bookmark.sty}{\usepackage{bookmark}}{\usepackage{hyperref}}
\IfFileExists{xurl.sty}{\usepackage{xurl}}{} % add URL line breaks if available
\urlstyle{same} % disable monospaced font for URLs
\hypersetup{
  pdftitle={Medidas de concentracion},
  pdfauthor={Edison Achalma},
  pdflang={es},
  colorlinks=true,
  linkcolor={blue},
  filecolor={Maroon},
  citecolor={Blue},
  urlcolor={Blue},
  pdfcreator={LaTeX via pandoc}}

\title{Medidas de concentracion}
\usepackage{etoolbox}
\makeatletter
\providecommand{\subtitle}[1]{% add subtitle to \maketitle
  \apptocmd{\@title}{\par {\large #1 \par}}{}{}
}
\makeatother
\subtitle{Explorando los pilares fundamentales para comprender el
funcionamiento y éxito de la industria moderna}
\author{Edison Achalma}
\date{2023-06-17}

\begin{document}
\maketitle
\ifdefined\Shaded\renewenvironment{Shaded}{\begin{tcolorbox}[interior hidden, enhanced, sharp corners, breakable, borderline west={3pt}{0pt}{shadecolor}, boxrule=0pt, frame hidden]}{\end{tcolorbox}}\fi

LAS ESTRUCTURAS DE MERCADO

4.1. COMPETENCIA PERFECTA Y MONOPOLIO

4.2. COMPETENCIA MONOPOLISTICA Y COMPETENCIA PERFECTA

4.3. EL OLIGOPOLIO Y MODELOS DE OLIGOPOLIO

4.3.1. DUOPOLIO Y MONOPLIO DE COURNOT

4.3.2. DUOPOLIO DE STAKELBERG.

4.3.3. DUOPOLIO DE BERTRAAND

4.3.4. LA COMPETENCIA MONOPLISTICA DE CHAMBERLIN.

4.1. COMPETENCIA PERFECTA Y MONOPOLIO

4.1.1. definición de competencia perfecta y características

\begin{enumerate}
\def\labelenumi{\Alph{enumi})}
\tightlist
\item
  Definición
\item
  Característica
\end{enumerate}

\begin{enumerate}
\def\labelenumi{\arabic{enumi}.}
\tightlist
\item
  Atomicidad • Homogeneidad
\item
  Perfecta información
\item
  Perfecta movilidad
\item
  Los actores económicos son precio aceptantes.
\item
  La competencia es transparente, bajo la ley de un solo precio y sin
  publicidad y costos de transporte cero 5.1.1. Monopolio puro:
  Definición y características

  \begin{enumerate}
  \def\labelenumii{\Alph{enumii})}
  \tightlist
  \item
    Definición
  \item
    Características
  \end{enumerate}
\item
  único productor y vendedor en la industria y el mercado
\item
  Producto homogéneo
\item
  Información imperfecta
\item
  Restricciones: legales, tecnológicas y naturales
\item
  Fija e impone precios en el mercado
\item
  Los monopolios crecen mediante colusiones y fusiones. 6.1.1.
  Competencia monopolística y Oligopolio 6.1.1. competencia
  monopolística

  \begin{enumerate}
  \def\labelenumii{\Alph{enumii})}
  \tightlist
  \item
    Definición
  \item
    Características
  \end{enumerate}
\item
  Muchos productores y gran número de compradores
\item
  Productos diferenciados en el mercado
\item
  Libre entrada y salida de empresas
\item
  Información imperfecta
\item
  Fijan precios en el mercado
\item
  Experimentan crecimientos mediante fusiones
\item
  Competencia a través de la publicidad y marketing 7.1.1. Oligopolio y
  modelos

  \begin{enumerate}
  \def\labelenumii{\Alph{enumii})}
  \tightlist
  \item
    Definición
  \item
    Características
  \end{enumerate}
\item
  Reducido N° de empresas
\item
  Productos pueden ser homogéneos, diferencias y sustitutos
\item
  Restricciones a la entrada de nuevas empresas: Legales, tecnológicas y
  4 Las empresas compiten mediante precios y cantidades bien grandes.
\item
  Competencia imperfecta
\item
  Fijan Pecios y
\item
  Competencia a través de la publicidad y marketing. MODELOS DE
  OLIGOPOLIO: 7.5.1. El Duopolio de Cournot. 7.5.2. La Competencia
  Monopolística en el modelo de Cournot. 7.5.3. El Duopolio de
  Stackelberg 7.5.4. El Duopolio de Chamberlin 7.5.5. El Duopolio de
  Bertrand. 7.5.6. El Duopolio de Egdeworth 7.5.7. El Duopolio de Paul
  SWEZZY
\end{enumerate}

\begin{itemize}
\item
  \href{../2023-06-12-introducion-organizacion-industrial/index.qmd}{1.
  Introducción a organización industrial}
\item
  \href{../2023-06-13-empresa-como-organizacion/index.qmd}{2. La Empresa
  como Organización. Promoviendo Valores Cooperativos, Humanos y
  Sociales}
\item
  \href{../2023-06-13-sistemas-economicos/index.qmd}{3. Introducción a
  los Sistemas Económicos. Cómo se distribuyen los recursos y se
  producen}
\item
  \href{../2023-06-15-mercado-relevante-oi-cap-2/index.qmd}{4. El
  Mercado Relevante Industrial de Bienes y el Mercado Geográfico}
\item
  \href{../2023-06-16-concentracion-poder-oi-cap3/index.qmd}{5. Medidas
  de concentracion}
\item
  \href{../2023-06-17-estructura-mercado-oi-cap4/index.qmd}{6.
  Estructura de mercado}
\end{itemize}


\printbibliography


\end{document}
