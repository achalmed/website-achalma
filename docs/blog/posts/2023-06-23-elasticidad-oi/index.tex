% Options for packages loaded elsewhere
\PassOptionsToPackage{unicode}{hyperref}
\PassOptionsToPackage{hyphens}{url}
\PassOptionsToPackage{dvipsnames,svgnames,x11names}{xcolor}
%
\documentclass[
  a4paper,
]{article}

\usepackage{amsmath,amssymb}
\usepackage{iftex}
\ifPDFTeX
  \usepackage[T1]{fontenc}
  \usepackage[utf8]{inputenc}
  \usepackage{textcomp} % provide euro and other symbols
\else % if luatex or xetex
  \usepackage{unicode-math}
  \defaultfontfeatures{Scale=MatchLowercase}
  \defaultfontfeatures[\rmfamily]{Ligatures=TeX,Scale=1}
\fi
\usepackage{lmodern}
\ifPDFTeX\else  
    % xetex/luatex font selection
\fi
% Use upquote if available, for straight quotes in verbatim environments
\IfFileExists{upquote.sty}{\usepackage{upquote}}{}
\IfFileExists{microtype.sty}{% use microtype if available
  \usepackage[]{microtype}
  \UseMicrotypeSet[protrusion]{basicmath} % disable protrusion for tt fonts
}{}
\makeatletter
\@ifundefined{KOMAClassName}{% if non-KOMA class
  \IfFileExists{parskip.sty}{%
    \usepackage{parskip}
  }{% else
    \setlength{\parindent}{0pt}
    \setlength{\parskip}{6pt plus 2pt minus 1pt}}
}{% if KOMA class
  \KOMAoptions{parskip=half}}
\makeatother
\usepackage{xcolor}
\usepackage[top=2.54cm,right=2.54cm,bottom=2.54cm,left=2.54cm]{geometry}
\setlength{\emergencystretch}{3em} % prevent overfull lines
\setcounter{secnumdepth}{-\maxdimen} % remove section numbering
% Make \paragraph and \subparagraph free-standing
\ifx\paragraph\undefined\else
  \let\oldparagraph\paragraph
  \renewcommand{\paragraph}[1]{\oldparagraph{#1}\mbox{}}
\fi
\ifx\subparagraph\undefined\else
  \let\oldsubparagraph\subparagraph
  \renewcommand{\subparagraph}[1]{\oldsubparagraph{#1}\mbox{}}
\fi

\usepackage{color}
\usepackage{fancyvrb}
\newcommand{\VerbBar}{|}
\newcommand{\VERB}{\Verb[commandchars=\\\{\}]}
\DefineVerbatimEnvironment{Highlighting}{Verbatim}{commandchars=\\\{\}}
% Add ',fontsize=\small' for more characters per line
\newenvironment{Shaded}{}{}
\newcommand{\AlertTok}[1]{\textcolor[rgb]{1.00,0.33,0.33}{\textbf{#1}}}
\newcommand{\AnnotationTok}[1]{\textcolor[rgb]{0.42,0.45,0.49}{#1}}
\newcommand{\AttributeTok}[1]{\textcolor[rgb]{0.84,0.23,0.29}{#1}}
\newcommand{\BaseNTok}[1]{\textcolor[rgb]{0.00,0.36,0.77}{#1}}
\newcommand{\BuiltInTok}[1]{\textcolor[rgb]{0.84,0.23,0.29}{#1}}
\newcommand{\CharTok}[1]{\textcolor[rgb]{0.01,0.18,0.38}{#1}}
\newcommand{\CommentTok}[1]{\textcolor[rgb]{0.42,0.45,0.49}{#1}}
\newcommand{\CommentVarTok}[1]{\textcolor[rgb]{0.42,0.45,0.49}{#1}}
\newcommand{\ConstantTok}[1]{\textcolor[rgb]{0.00,0.36,0.77}{#1}}
\newcommand{\ControlFlowTok}[1]{\textcolor[rgb]{0.84,0.23,0.29}{#1}}
\newcommand{\DataTypeTok}[1]{\textcolor[rgb]{0.84,0.23,0.29}{#1}}
\newcommand{\DecValTok}[1]{\textcolor[rgb]{0.00,0.36,0.77}{#1}}
\newcommand{\DocumentationTok}[1]{\textcolor[rgb]{0.42,0.45,0.49}{#1}}
\newcommand{\ErrorTok}[1]{\textcolor[rgb]{1.00,0.33,0.33}{\underline{#1}}}
\newcommand{\ExtensionTok}[1]{\textcolor[rgb]{0.84,0.23,0.29}{\textbf{#1}}}
\newcommand{\FloatTok}[1]{\textcolor[rgb]{0.00,0.36,0.77}{#1}}
\newcommand{\FunctionTok}[1]{\textcolor[rgb]{0.44,0.26,0.76}{#1}}
\newcommand{\ImportTok}[1]{\textcolor[rgb]{0.01,0.18,0.38}{#1}}
\newcommand{\InformationTok}[1]{\textcolor[rgb]{0.42,0.45,0.49}{#1}}
\newcommand{\KeywordTok}[1]{\textcolor[rgb]{0.84,0.23,0.29}{#1}}
\newcommand{\NormalTok}[1]{\textcolor[rgb]{0.14,0.16,0.18}{#1}}
\newcommand{\OperatorTok}[1]{\textcolor[rgb]{0.14,0.16,0.18}{#1}}
\newcommand{\OtherTok}[1]{\textcolor[rgb]{0.44,0.26,0.76}{#1}}
\newcommand{\PreprocessorTok}[1]{\textcolor[rgb]{0.84,0.23,0.29}{#1}}
\newcommand{\RegionMarkerTok}[1]{\textcolor[rgb]{0.42,0.45,0.49}{#1}}
\newcommand{\SpecialCharTok}[1]{\textcolor[rgb]{0.00,0.36,0.77}{#1}}
\newcommand{\SpecialStringTok}[1]{\textcolor[rgb]{0.01,0.18,0.38}{#1}}
\newcommand{\StringTok}[1]{\textcolor[rgb]{0.01,0.18,0.38}{#1}}
\newcommand{\VariableTok}[1]{\textcolor[rgb]{0.89,0.38,0.04}{#1}}
\newcommand{\VerbatimStringTok}[1]{\textcolor[rgb]{0.01,0.18,0.38}{#1}}
\newcommand{\WarningTok}[1]{\textcolor[rgb]{1.00,0.33,0.33}{#1}}

\providecommand{\tightlist}{%
  \setlength{\itemsep}{0pt}\setlength{\parskip}{0pt}}\usepackage{longtable,booktabs,array}
\usepackage{calc} % for calculating minipage widths
% Correct order of tables after \paragraph or \subparagraph
\usepackage{etoolbox}
\makeatletter
\patchcmd\longtable{\par}{\if@noskipsec\mbox{}\fi\par}{}{}
\makeatother
% Allow footnotes in longtable head/foot
\IfFileExists{footnotehyper.sty}{\usepackage{footnotehyper}}{\usepackage{footnote}}
\makesavenoteenv{longtable}
\usepackage{graphicx}
\makeatletter
\def\maxwidth{\ifdim\Gin@nat@width>\linewidth\linewidth\else\Gin@nat@width\fi}
\def\maxheight{\ifdim\Gin@nat@height>\textheight\textheight\else\Gin@nat@height\fi}
\makeatother
% Scale images if necessary, so that they will not overflow the page
% margins by default, and it is still possible to overwrite the defaults
% using explicit options in \includegraphics[width, height, ...]{}
\setkeys{Gin}{width=\maxwidth,height=\maxheight,keepaspectratio}
% Set default figure placement to htbp
\makeatletter
\def\fps@figure{htbp}
\makeatother

\makeatletter
\makeatother
\makeatletter
\makeatother
\makeatletter
\@ifpackageloaded{caption}{}{\usepackage{caption}}
\AtBeginDocument{%
\ifdefined\contentsname
  \renewcommand*\contentsname{Tabla de contenidos}
\else
  \newcommand\contentsname{Tabla de contenidos}
\fi
\ifdefined\listfigurename
  \renewcommand*\listfigurename{Listado de Figuras}
\else
  \newcommand\listfigurename{Listado de Figuras}
\fi
\ifdefined\listtablename
  \renewcommand*\listtablename{Listado de Tablas}
\else
  \newcommand\listtablename{Listado de Tablas}
\fi
\ifdefined\figurename
  \renewcommand*\figurename{Figura}
\else
  \newcommand\figurename{Figura}
\fi
\ifdefined\tablename
  \renewcommand*\tablename{Tabla}
\else
  \newcommand\tablename{Tabla}
\fi
}
\@ifpackageloaded{float}{}{\usepackage{float}}
\floatstyle{ruled}
\@ifundefined{c@chapter}{\newfloat{codelisting}{h}{lop}}{\newfloat{codelisting}{h}{lop}[chapter]}
\floatname{codelisting}{Listado}
\newcommand*\listoflistings{\listof{codelisting}{Listado de Listados}}
\makeatother
\makeatletter
\@ifpackageloaded{caption}{}{\usepackage{caption}}
\@ifpackageloaded{subcaption}{}{\usepackage{subcaption}}
\makeatother
\makeatletter
\@ifpackageloaded{tcolorbox}{}{\usepackage[skins,breakable]{tcolorbox}}
\makeatother
\makeatletter
\@ifundefined{shadecolor}{\definecolor{shadecolor}{rgb}{.97, .97, .97}}
\makeatother
\makeatletter
\makeatother
\makeatletter
\makeatother
\ifLuaTeX
\usepackage[bidi=basic]{babel}
\else
\usepackage[bidi=default]{babel}
\fi
\babelprovide[main,import]{spanish}
% get rid of language-specific shorthands (see #6817):
\let\LanguageShortHands\languageshorthands
\def\languageshorthands#1{}
\ifLuaTeX
  \usepackage{selnolig}  % disable illegal ligatures
\fi
\usepackage[]{biblatex}
\addbibresource{../../../../references.bib}
\IfFileExists{bookmark.sty}{\usepackage{bookmark}}{\usepackage{hyperref}}
\IfFileExists{xurl.sty}{\usepackage{xurl}}{} % add URL line breaks if available
\urlstyle{same} % disable monospaced font for URLs
\hypersetup{
  pdftitle={Elasticidad},
  pdfauthor={Edison Achalma},
  pdflang={es},
  colorlinks=true,
  linkcolor={blue},
  filecolor={Maroon},
  citecolor={Blue},
  urlcolor={Blue},
  pdfcreator={LaTeX via pandoc}}

\title{Elasticidad}
\usepackage{etoolbox}
\makeatletter
\providecommand{\subtitle}[1]{% add subtitle to \maketitle
  \apptocmd{\@title}{\par {\large #1 \par}}{}{}
}
\makeatother
\subtitle{Expl}
\author{Edison Achalma}
\date{2023-06-23}

\begin{document}
\maketitle
\ifdefined\Shaded\renewenvironment{Shaded}{\begin{tcolorbox}[breakable, frame hidden, interior hidden, boxrule=0pt, borderline west={3pt}{0pt}{shadecolor}, enhanced, sharp corners]}{\end{tcolorbox}}\fi

Elasticidad precio de la demanda

\begin{Shaded}
\begin{Highlighting}[]
\ImportTok{import}\NormalTok{ matplotlib.pyplot }\ImportTok{as}\NormalTok{ plt}
\ImportTok{from}\NormalTok{ matplotlib.collections }\ImportTok{import}\NormalTok{ EventCollection}
\ImportTok{import}\NormalTok{ numpy }\ImportTok{as}\NormalTok{ np}

\CommentTok{\# Fijar el estado aleatorio para reproducibilidad}
\NormalTok{np.random.seed(}\DecValTok{19680801}\NormalTok{)}

\CommentTok{\# Crear datos aleatorios}
\NormalTok{xdata }\OperatorTok{=}\NormalTok{ np.random.random([}\DecValTok{2}\NormalTok{, }\DecValTok{10}\NormalTok{])}

\CommentTok{\# Dividir los datos en dos partes}
\NormalTok{xdata1 }\OperatorTok{=}\NormalTok{ xdata[}\DecValTok{0}\NormalTok{, :]}
\NormalTok{xdata2 }\OperatorTok{=}\NormalTok{ xdata[}\DecValTok{1}\NormalTok{, :]}

\CommentTok{\# Ordenar los datos para obtener curvas limpias}
\NormalTok{xdata1.sort()}
\NormalTok{xdata2.sort()}

\CommentTok{\# Crear algunos puntos de datos y}
\NormalTok{ydata1 }\OperatorTok{=}\NormalTok{ xdata1 }\OperatorTok{**} \DecValTok{2}
\NormalTok{ydata2 }\OperatorTok{=} \DecValTok{1} \OperatorTok{{-}}\NormalTok{ xdata2 }\OperatorTok{**} \DecValTok{3}

\CommentTok{\# Graficar los datos}
\NormalTok{fig }\OperatorTok{=}\NormalTok{ plt.figure()}
\NormalTok{ax }\OperatorTok{=}\NormalTok{ fig.add\_subplot(}\DecValTok{1}\NormalTok{, }\DecValTok{1}\NormalTok{, }\DecValTok{1}\NormalTok{)}
\NormalTok{ax.plot(xdata1, ydata1, color}\OperatorTok{=}\StringTok{\textquotesingle{}tab:blue\textquotesingle{}}\NormalTok{)}
\NormalTok{ax.plot(xdata2, ydata2, color}\OperatorTok{=}\StringTok{\textquotesingle{}tab:orange\textquotesingle{}}\NormalTok{)}

\CommentTok{\# Crear los eventos que marcan los puntos de datos en el eje x}
\NormalTok{xevents1 }\OperatorTok{=}\NormalTok{ EventCollection(xdata1, color}\OperatorTok{=}\StringTok{\textquotesingle{}tab:blue\textquotesingle{}}\NormalTok{, linelength}\OperatorTok{=}\FloatTok{0.05}\NormalTok{)}
\NormalTok{xevents2 }\OperatorTok{=}\NormalTok{ EventCollection(xdata2, color}\OperatorTok{=}\StringTok{\textquotesingle{}tab:orange\textquotesingle{}}\NormalTok{, linelength}\OperatorTok{=}\FloatTok{0.05}\NormalTok{)}

\CommentTok{\# Crear los eventos que marcan los puntos de datos en el eje y}
\NormalTok{yevents1 }\OperatorTok{=}\NormalTok{ EventCollection(ydata1, color}\OperatorTok{=}\StringTok{\textquotesingle{}tab:blue\textquotesingle{}}\NormalTok{, linelength}\OperatorTok{=}\FloatTok{0.05}\NormalTok{, orientation}\OperatorTok{=}\StringTok{\textquotesingle{}vertical\textquotesingle{}}\NormalTok{)}
\NormalTok{yevents2 }\OperatorTok{=}\NormalTok{ EventCollection(ydata2, color}\OperatorTok{=}\StringTok{\textquotesingle{}tab:orange\textquotesingle{}}\NormalTok{, linelength}\OperatorTok{=}\FloatTok{0.05}\NormalTok{, orientation}\OperatorTok{=}\StringTok{\textquotesingle{}vertical\textquotesingle{}}\NormalTok{)}

\CommentTok{\# Agregar los eventos al eje}
\NormalTok{ax.add\_collection(xevents1)}
\NormalTok{ax.add\_collection(xevents2)}
\NormalTok{ax.add\_collection(yevents1)}
\NormalTok{ax.add\_collection(yevents2)}

\CommentTok{\# Establecer los límites}
\NormalTok{ax.set\_xlim([}\DecValTok{0}\NormalTok{, }\DecValTok{1}\NormalTok{])}
\NormalTok{ax.set\_ylim([}\DecValTok{0}\NormalTok{, }\DecValTok{1}\NormalTok{])}

\NormalTok{ax.set\_title(}\StringTok{\textquotesingle{}Gráfico de línea con puntos de datos\textquotesingle{}}\NormalTok{)}

\CommentTok{\# Mostrar la gráfica}
\NormalTok{plt.show()}
\end{Highlighting}
\end{Shaded}

\begin{enumerate}
\def\labelenumi{\arabic{enumi}.}
\tightlist
\item
  \[
  \eta _{XP_y} = \frac{\frac {X_f^d - X_i^d}{X_i}}{\frac {P_f^d - P_i^d}{P_i^d}}
  \]
\end{enumerate}

\[
= \frac{\frac{\Delta X^d}{X_i^d}}{\frac{\Delta P_x^d}{P_i^d}}
\]

\begin{equation}\protect\hypertarget{eq-1}{}{
\frac{\Delta X^d P_i^d}{\Delta P_x^d X_i^d}
}\label{eq-1}\end{equation}

\begin{enumerate}
\def\labelenumi{\arabic{enumi}.}
\setcounter{enumi}{1}
\tightlist
\item
\end{enumerate}

\begin{equation}\protect\hypertarget{eq-2}{}{
\eta _{PX^d} = \frac{\partial X^d P_x}{\partial P_x X^d}
}\label{eq-2}\end{equation}

\begin{enumerate}
\def\labelenumi{\arabic{enumi}.}
\setcounter{enumi}{2}
\tightlist
\item
\end{enumerate}

\begin{equation}\protect\hypertarget{eq-3}{}{
\eta _{PX^d} = \frac{\partial \ln(\mathrm X)}{\partial  \ln(\mathrm P_x)}
}\label{eq-3}\end{equation}

\begin{enumerate}
\def\labelenumi{\arabic{enumi}.}
\setcounter{enumi}{3}
\item
  \begin{equation}\protect\hypertarget{eq-4}{}{
  \eta _{PX^d} = \frac{\Delta \% X^d}{\Delta \% P_x}
  }\label{eq-4}\end{equation}
\item
\end{enumerate}

\begin{equation}\protect\hypertarget{eq-5}{}{
\eta_{PX^d} = m_{ip} \frac{P_i^d}{X^d}
}\label{eq-5}\end{equation}

Ejemplo

\[
X^d = \frac{P_y P_z I^{0.2} N}{2 P_x}
\]

Aplicando la fórmula Ecuación~\ref{eq-2}

\[ 
\eta _{PX^d} = \frac{\partial \mathrm{X}}{\partial \mathrm{P_x}} \frac{\mathrm{P_x}}{\mathrm{X^d}} 
\]

\[
= - \frac{P_y P_z I^{0.2} N}{2 (P_x)^2} \frac{\mathrm{P_x}}{\mathrm{X^d} }
\]

reemplazamos \(x^d\) con su valor \[
= - \frac{P_y P_z I^{0.2} N}{2 (P_x)^2} \frac{\mathrm{P_x}}{\frac{P_y P_z I^{0.2} N}{2 P_x} }
\]

ordenando y resolviendo

\[
= - \frac{2 P_y P_z (P_x)^2 I^{0.2} N}{2 P_y P_z (P_x)^2 I^{0.2} N} = -1
\]

interpretación

Si el \(P_x\) aumenta en 1\% entonces \(X^d\) disminuye en 1\%.


\printbibliography


\end{document}
