% Options for packages loaded elsewhere
\PassOptionsToPackage{unicode}{hyperref}
\PassOptionsToPackage{hyphens}{url}
\PassOptionsToPackage{dvipsnames,svgnames,x11names}{xcolor}
%
\documentclass[
  a4paper,
]{article}

\usepackage{amsmath,amssymb}
\usepackage{iftex}
\ifPDFTeX
  \usepackage[T1]{fontenc}
  \usepackage[utf8]{inputenc}
  \usepackage{textcomp} % provide euro and other symbols
\else % if luatex or xetex
  \usepackage{unicode-math}
  \defaultfontfeatures{Scale=MatchLowercase}
  \defaultfontfeatures[\rmfamily]{Ligatures=TeX,Scale=1}
\fi
\usepackage{lmodern}
\ifPDFTeX\else  
    % xetex/luatex font selection
\fi
% Use upquote if available, for straight quotes in verbatim environments
\IfFileExists{upquote.sty}{\usepackage{upquote}}{}
\IfFileExists{microtype.sty}{% use microtype if available
  \usepackage[]{microtype}
  \UseMicrotypeSet[protrusion]{basicmath} % disable protrusion for tt fonts
}{}
\makeatletter
\@ifundefined{KOMAClassName}{% if non-KOMA class
  \IfFileExists{parskip.sty}{%
    \usepackage{parskip}
  }{% else
    \setlength{\parindent}{0pt}
    \setlength{\parskip}{6pt plus 2pt minus 1pt}}
}{% if KOMA class
  \KOMAoptions{parskip=half}}
\makeatother
\usepackage{xcolor}
\usepackage[top=2.54cm,right=2.54cm,bottom=2.54cm,left=2.54cm]{geometry}
\setlength{\emergencystretch}{3em} % prevent overfull lines
\setcounter{secnumdepth}{-\maxdimen} % remove section numbering
% Make \paragraph and \subparagraph free-standing
\ifx\paragraph\undefined\else
  \let\oldparagraph\paragraph
  \renewcommand{\paragraph}[1]{\oldparagraph{#1}\mbox{}}
\fi
\ifx\subparagraph\undefined\else
  \let\oldsubparagraph\subparagraph
  \renewcommand{\subparagraph}[1]{\oldsubparagraph{#1}\mbox{}}
\fi


\providecommand{\tightlist}{%
  \setlength{\itemsep}{0pt}\setlength{\parskip}{0pt}}\usepackage{longtable,booktabs,array}
\usepackage{calc} % for calculating minipage widths
% Correct order of tables after \paragraph or \subparagraph
\usepackage{etoolbox}
\makeatletter
\patchcmd\longtable{\par}{\if@noskipsec\mbox{}\fi\par}{}{}
\makeatother
% Allow footnotes in longtable head/foot
\IfFileExists{footnotehyper.sty}{\usepackage{footnotehyper}}{\usepackage{footnote}}
\makesavenoteenv{longtable}
\usepackage{graphicx}
\makeatletter
\def\maxwidth{\ifdim\Gin@nat@width>\linewidth\linewidth\else\Gin@nat@width\fi}
\def\maxheight{\ifdim\Gin@nat@height>\textheight\textheight\else\Gin@nat@height\fi}
\makeatother
% Scale images if necessary, so that they will not overflow the page
% margins by default, and it is still possible to overwrite the defaults
% using explicit options in \includegraphics[width, height, ...]{}
\setkeys{Gin}{width=\maxwidth,height=\maxheight,keepaspectratio}
% Set default figure placement to htbp
\makeatletter
\def\fps@figure{htbp}
\makeatother

% Preámbulo
\usepackage{comment} % Permite comentar secciones del código
\usepackage{marvosym} % Agrega símbolos adicionales
\usepackage{graphicx} % Permite insertar imágenes
\usepackage{mathptmx} % Fuente de texto matemática
\usepackage{amssymb} % Símbolos adicionales de matemáticas
\usepackage{lipsum} % Crea texto aleatorio
\usepackage{amsthm} % Teoremas y entornos de demostración
\usepackage{float} % Control de posiciones de figuras y tablas
\usepackage{rotating} % Rotación de elementos
\usepackage{multirow} % Celdas combinadas en tablas
\usepackage{tabularx} % Tablas con ancho de columna ajustable
\usepackage{mdframed} % Marcos alrededor de elementos flotantes

% Series de tiempo
\usepackage{booktabs}


% Configuración adicional

\makeatletter
\makeatother
\makeatletter
\makeatother
\makeatletter
\@ifpackageloaded{caption}{}{\usepackage{caption}}
\AtBeginDocument{%
\ifdefined\contentsname
  \renewcommand*\contentsname{Tabla de contenidos}
\else
  \newcommand\contentsname{Tabla de contenidos}
\fi
\ifdefined\listfigurename
  \renewcommand*\listfigurename{Listado de Figuras}
\else
  \newcommand\listfigurename{Listado de Figuras}
\fi
\ifdefined\listtablename
  \renewcommand*\listtablename{Listado de Tablas}
\else
  \newcommand\listtablename{Listado de Tablas}
\fi
\ifdefined\figurename
  \renewcommand*\figurename{Figura}
\else
  \newcommand\figurename{Figura}
\fi
\ifdefined\tablename
  \renewcommand*\tablename{Tabla}
\else
  \newcommand\tablename{Tabla}
\fi
}
\@ifpackageloaded{float}{}{\usepackage{float}}
\floatstyle{ruled}
\@ifundefined{c@chapter}{\newfloat{codelisting}{h}{lop}}{\newfloat{codelisting}{h}{lop}[chapter]}
\floatname{codelisting}{Listado}
\newcommand*\listoflistings{\listof{codelisting}{Listado de Listados}}
\makeatother
\makeatletter
\@ifpackageloaded{caption}{}{\usepackage{caption}}
\@ifpackageloaded{subcaption}{}{\usepackage{subcaption}}
\makeatother
\makeatletter
\@ifpackageloaded{tcolorbox}{}{\usepackage[skins,breakable]{tcolorbox}}
\makeatother
\makeatletter
\@ifundefined{shadecolor}{\definecolor{shadecolor}{rgb}{.97, .97, .97}}
\makeatother
\makeatletter
\makeatother
\makeatletter
\makeatother
\ifLuaTeX
\usepackage[bidi=basic]{babel}
\else
\usepackage[bidi=default]{babel}
\fi
\babelprovide[main,import]{spanish}
% get rid of language-specific shorthands (see #6817):
\let\LanguageShortHands\languageshorthands
\def\languageshorthands#1{}
\ifLuaTeX
  \usepackage{selnolig}  % disable illegal ligatures
\fi
\usepackage[]{biblatex}
\addbibresource{../../../../references.bib}
\IfFileExists{bookmark.sty}{\usepackage{bookmark}}{\usepackage{hyperref}}
\IfFileExists{xurl.sty}{\usepackage{xurl}}{} % add URL line breaks if available
\urlstyle{same} % disable monospaced font for URLs
\hypersetup{
  pdftitle={Domina las habilidades de edición de texto en Vim, una guía completa para maximizar tu productividad},
  pdfauthor={Edison Achalma},
  pdflang={es},
  colorlinks=true,
  linkcolor={blue},
  filecolor={Maroon},
  citecolor={Blue},
  urlcolor={Blue},
  pdfcreator={LaTeX via pandoc}}

\title{Domina las habilidades de edición de texto en Vim, una guía
completa para maximizar tu productividad}
\usepackage{etoolbox}
\makeatletter
\providecommand{\subtitle}[1]{% add subtitle to \maketitle
  \apptocmd{\@title}{\par {\large #1 \par}}{}{}
}
\makeatother
\subtitle{Descubre los atajos de teclado, comandos y técnicas avanzadas
de Vim para buscar, reemplazar y transformar texto de manera eficiente}
\author{Edison Achalma}
\date{2023-07-01}

\begin{document}
\maketitle
\ifdefined\Shaded\renewenvironment{Shaded}{\begin{tcolorbox}[enhanced, boxrule=0pt, breakable, frame hidden, interior hidden, borderline west={3pt}{0pt}{shadecolor}, sharp corners]}{\end{tcolorbox}}\fi

\hypertarget{introducciuxf3n-a-vim}{%
\section{Introducción a Vim}\label{introducciuxf3n-a-vim}}

¿Estás buscando mejorar tu fluidez y productividad al editar texto en
Vim? ¿Quieres dominar los atajos de teclado y comandos que te permitirán
aprovechar al máximo este potente editor de texto? ¡Has llegado al lugar
indicado!

En el mundo de la edición de texto, Vim destaca como una herramienta
poderosa y altamente personalizable. Sin embargo, para muchos usuarios
nuevos, su enfoque basado en modos y su amplio conjunto de comandos
pueden resultar abrumadores al principio. Pero no te preocupes, estamos
aquí para ayudarte a superar esa barrera inicial y llevar tus
habilidades de Vim al siguiente nivel.

En este blog, exploraremos a fondo los atajos de teclado y comandos
esenciales para usar Vim de manera eficiente. Te proporcionaremos una
guía completa que abarcará desde los conceptos básicos hasta técnicas
avanzadas, permitiéndote aprovechar al máximo este editor de texto
icónico.

Descubrirás cómo navegar rápidamente por tus archivos, editar y
manipular texto con fluidez, realizar búsquedas y reemplazos de manera
eficiente, trabajar con múltiples archivos y ventanas, personalizar Vim
según tus preferencias y mucho más. Además, te proporcionaremos consejos
y trucos prácticos para optimizar tu flujo de trabajo y ahorrar tiempo
en tus tareas diarias de edición de texto.

No importa si eres un principiante curioso o un usuario experimentado en
busca de nuevas técnicas, mi objetivo es ayudarte a desbloquear todo el
potencial de Vim y convertirte en un maestro de la edición eficiente.

¡Prepárate para desafiar tus límites, perfeccionar tus habilidades y
descubrir un nuevo nivel de productividad con Vim! Sigue leyendo y
comienza tu viaje hacia la maestría de Vim con nuestros atajos de
teclado y comandos indispensables.

¡La eficiencia está a solo unos atajos de distancia!

\hypertarget{quuxe9-es-vim}{%
\subsection{¿Qué es Vim?}\label{quuxe9-es-vim}}

Antes de sumergirnos en los fantásticos atajos de teclado y comandos de
Vim, es importante comprender qué es exactamente Vim y por qué es tan
popular entre los usuarios de edición de texto.

Vim, que significa ``Vi Improved'' (Vi mejorado), es un editor de texto
altamente configurable y poderoso. Se basa en el antiguo editor de línea
de comandos llamado Vi, que ha sido una herramienta estándar en los
sistemas Unix durante décadas.

\hypertarget{ventajas-de-utilizar-vim-como-editor-de-texto}{%
\subsection{Ventajas de utilizar Vim como editor de
texto}\label{ventajas-de-utilizar-vim-como-editor-de-texto}}

La principal ventaja de utilizar Vim es su enfoque en la eficiencia y la
productividad. Con los atajos de teclado y comandos adecuados, puedes
editar texto de forma más rápida y eficiente que nunca. No más tediosos
movimientos del ratón o desplazamientos interminables. Con Vim, puedes
mantener tus manos en el teclado y volar a través del texto como un
verdadero experto.

Además de su enfoque en la eficiencia, Vim ofrece una serie de
características que lo hacen destacar entre otros editores de texto.
Algunas de las ventajas más notables incluyen:

\begin{enumerate}
\def\labelenumi{\arabic{enumi}.}
\item
  \textbf{Modos de funcionamiento}: Vim tiene modos distintos, como el
  modo normal, el modo de inserción y el modo de comando. Cada modo
  tiene un propósito específico, lo que te permite realizar diversas
  acciones con facilidad y fluidez.
\item
  \textbf{Personalización y extensibilidad}: Vim es altamente
  personalizable. Puedes ajustar su apariencia, configurar atajos de
  teclado personalizados y aprovechar su amplia gama de complementos y
  extensiones para adaptarlo a tus preferencias y necesidades.
\item
  \textbf{Potentes capacidades de búsqueda y reemplazo}: Con Vim, puedes
  buscar y reemplazar texto de manera rápida y precisa. Sus comandos de
  búsqueda te permiten encontrar palabras o patrones específicos en todo
  el archivo o incluso en múltiples archivos a la vez.
\item
  \textbf{Edición en múltiples archivos}: Vim facilita el trabajo con
  múltiples archivos al mismo tiempo. Puedes abrir y editar varios
  archivos en diferentes ventanas o pestañas, lo que te permite alternar
  rápidamente entre ellos y mantener tu flujo de trabajo organizado.
\end{enumerate}

\hypertarget{configuraciuxf3n-inicial-de-vim}{%
\subsection{Configuración inicial de
Vim}\label{configuraciuxf3n-inicial-de-vim}}

Antes de sumergirnos en los atajos y comandos de Vim, es importante
realizar una configuración inicial para adaptarlo a tus preferencias.
Aquí hay algunos pasos clave para empezar:

\begin{enumerate}
\def\labelenumi{\arabic{enumi}.}
\item
  \textbf{Instalación de Vim}: Si aún no tienes Vim instalado en tu
  sistema, debes descargarlo e instalarlo. Puedes encontrar la última
  versión en el sitio web oficial de Vim o utilizar un gestor de
  paquetes si estás en un sistema operativo compatible.
\item
  \textbf{Creación del archivo .vimrc}: El archivo \texttt{.vimrc} es
  donde puedes personalizar la configuración de Vim. Puedes establecer
  atajos de teclado personalizados, activar o desactivar características
  específicas, y mucho más. Es tu espacio para hacer que Vim se sienta
  como en casa.
\item
  \textbf{Exploración de los ajustes básicos}: En tu archivo
  \texttt{.vimrc}, puedes establecer algunas configuraciones iniciales
  básicas, como el número de líneas visibles, el formato de tabulación,
  el resaltado de sintaxis y los atajos de teclado predefinidos. Estos
  ajustes pueden mejorar tu experiencia de edición desde el principio.
\end{enumerate}

¡Ahora estás listo para comenzar tu viaje con Vim! En los siguientes
apartados, te sumergirás en los atajos de teclado y comandos esenciales
que te ayudarán a aprovechar al máximo este poderoso editor de texto.

\hypertarget{modos-de-vim}{%
\section{Modos de Vim}\label{modos-de-vim}}

Cuando se trata de Vim, uno de los conceptos clave que necesitas
entender son sus modos de funcionamiento. Estos modos determinan cómo
interactúas con el texto y son fundamentales para utilizar Vim de manera
eficiente.

\hypertarget{modo-normal}{%
\subsection{Modo normal}\label{modo-normal}}

El modo normal es el modo principal de Vim. Aquí es donde puedes navegar
por el archivo, realizar ediciones rápidas y ejecutar comandos. En este
modo, las teclas que presiones se interpretan como comandos, no como
caracteres para ingresar. Es como convertir tu teclado en un control
remoto para manipular el texto.

\hypertarget{modo-de-inserciuxf3n}{%
\subsection{Modo de inserción}\label{modo-de-inserciuxf3n}}

El modo de inserción es donde puedes escribir y editar texto como en
cualquier otro editor de texto convencional. Cuando estás en este modo,
las teclas que presiones se insertan directamente en el archivo. Es el
modo en el que puedes escribir tus palabras y expresarte libremente.

\hypertarget{modo-de-comando}{%
\subsection{Modo de comando}\label{modo-de-comando}}

El modo de comando es donde puedes ejecutar comandos más avanzados en
Vim. Puedes utilizar estos comandos para buscar y reemplazar texto,
realizar cambios en el archivo y personalizar la configuración de Vim.
Aquí es donde puedes aprovechar el verdadero poder de Vim y llevar tus
habilidades de edición al siguiente nivel.

Para intercambiar entre cada uno de los modos usa los siguientes atajos
de teclado:

\begin{longtable}[]{@{}
  >{\raggedright\arraybackslash}p{(\columnwidth - 4\tabcolsep) * \real{0.0846}}
  >{\raggedright\arraybackslash}p{(\columnwidth - 4\tabcolsep) * \real{0.1045}}
  >{\raggedright\arraybackslash}p{(\columnwidth - 4\tabcolsep) * \real{0.8109}}@{}}
\toprule\noalign{}
\begin{minipage}[b]{\linewidth}\raggedright
Modo
\end{minipage} & \begin{minipage}[b]{\linewidth}\raggedright
Shortcuts or commands
\end{minipage} & \begin{minipage}[b]{\linewidth}\raggedright
Función
\end{minipage} \\
\midrule\noalign{}
\endhead
\bottomrule\noalign{}
\endlastfoot
Modo normal & \texttt{Esc} o \texttt{Ctrl+c} & Este atajo te lleva al
modo normal de Vim. Aquí puedes navegar por el archivo, realizar
ediciones rápidas y ejecutar comandos. \\
Modo de inserción & \texttt{i} & Presionar \texttt{i} te lleva al modo
de inserción. Aquí puedes escribir y editar texto como en cualquier otro
editor de texto. \\
Modo de comando & \texttt{:} & Al presionar \texttt{:} accederás al modo
de comando. Aquí puedes ejecutar comandos avanzados en Vim. Desde buscar
y reemplazar texto hasta personalizar la configuración. \\
\end{longtable}

\hypertarget{guardar-documentos-en-vim}{%
\section{Guardar documentos en Vim}\label{guardar-documentos-en-vim}}

Cuando trabajas en Vim, es importante saber cómo guardar tus documentos
para asegurarte de que tus cambios se guarden correctamente.
Afortunadamente, Vim ofrece atajos de teclado rápidos y sencillos para
realizar esta tarea. A continuación, se muestra una tabla con los atajos
de teclado y su función correspondiente:

\begin{longtable}[]{@{}
  >{\raggedright\arraybackslash}p{(\columnwidth - 2\tabcolsep) * \real{0.1273}}
  >{\raggedright\arraybackslash}p{(\columnwidth - 2\tabcolsep) * \real{0.8727}}@{}}
\toprule\noalign{}
\begin{minipage}[b]{\linewidth}\raggedright
Shortcuts or commands
\end{minipage} & \begin{minipage}[b]{\linewidth}\raggedright
Función
\end{minipage} \\
\midrule\noalign{}
\endhead
\bottomrule\noalign{}
\endlastfoot
\texttt{:w} & Presionar \texttt{:w} te permite guardar el archivo
actual. Esto guarda los cambios que hayas realizado sin cerrar Vim. \\
\texttt{:w\ nombre\_archivo} & Al utilizar \texttt{:w} seguido de un
nombre de archivo, puedes guardar el archivo actual con un nuevo nombre.
Por ejemplo \texttt{:w\ index.md} \\
\texttt{:wq} o \texttt{:x} & Si deseas guardar el archivo y salir de Vim
al mismo tiempo, puedes utilizar \texttt{:wq} o \texttt{:x}. Esto guarda
los cambios y cierra Vim en un solo paso. \\
\end{longtable}

\hypertarget{atajos-de-navegaciuxf3n-y-movimiento}{%
\section{Atajos de navegación y
movimiento}\label{atajos-de-navegaciuxf3n-y-movimiento}}

En Vim, existen diversos atajos de teclado que te permiten una mejor
navegaciòn y movimiento. A continuación, se muestran las diferentes
categorías de atajos junto con sus respectivas funciones:

\hypertarget{movimiento-buxe1sico-con-las-teclas-h-j-k-l}{%
\subsection{Movimiento básico con las teclas h, j, k,
l}\label{movimiento-buxe1sico-con-las-teclas-h-j-k-l}}

\begin{longtable}[]{@{}
  >{\raggedright\arraybackslash}p{(\columnwidth - 2\tabcolsep) * \real{0.2625}}
  >{\raggedright\arraybackslash}p{(\columnwidth - 2\tabcolsep) * \real{0.7375}}@{}}
\toprule\noalign{}
\begin{minipage}[b]{\linewidth}\raggedright
Shortcuts or commands
\end{minipage} & \begin{minipage}[b]{\linewidth}\raggedright
Función
\end{minipage} \\
\midrule\noalign{}
\endhead
\bottomrule\noalign{}
\endlastfoot
\texttt{h} & La tecla \texttt{h} te permite mover el cursor hacia la
izquierda. \\
\texttt{j} & La tecla \texttt{j} te permite mover el cursor hacia
abajo. \\
\texttt{k} & La tecla \texttt{k} te permite mover el cursor hacia
arriba. \\
\texttt{l} & La tecla \texttt{l} te permite mover el cursor hacia la
derecha. \\
\end{longtable}

\hypertarget{saltos-ruxe1pidos-en-el-archivo-con-gg-y-g}{%
\subsection{\texorpdfstring{Saltos rápidos en el archivo con \texttt{gg}
y
\texttt{G}}{Saltos rápidos en el archivo con gg y G}}\label{saltos-ruxe1pidos-en-el-archivo-con-gg-y-g}}

\begin{longtable}[]{@{}
  >{\raggedright\arraybackslash}p{(\columnwidth - 2\tabcolsep) * \real{0.3043}}
  >{\raggedright\arraybackslash}p{(\columnwidth - 2\tabcolsep) * \real{0.6957}}@{}}
\toprule\noalign{}
\begin{minipage}[b]{\linewidth}\raggedright
Shortcuts or commands
\end{minipage} & \begin{minipage}[b]{\linewidth}\raggedright
Función
\end{minipage} \\
\midrule\noalign{}
\endhead
\bottomrule\noalign{}
\endlastfoot
\texttt{gg} & Presionar \texttt{gg} te lleva al comienzo del archivo. \\
\texttt{G} & Presionar \texttt{G} te lleva al final del archivo. \\
\end{longtable}

\hypertarget{navegaciuxf3n-por-palabras-con-w-b-e}{%
\subsection{\texorpdfstring{Navegación por palabras con \texttt{w},
\texttt{b},
\texttt{e}}{Navegación por palabras con w, b, e}}\label{navegaciuxf3n-por-palabras-con-w-b-e}}

\begin{longtable}[]{@{}
  >{\raggedright\arraybackslash}p{(\columnwidth - 2\tabcolsep) * \real{0.2692}}
  >{\raggedright\arraybackslash}p{(\columnwidth - 2\tabcolsep) * \real{0.7308}}@{}}
\toprule\noalign{}
\begin{minipage}[b]{\linewidth}\raggedright
Shortcuts or commands
\end{minipage} & \begin{minipage}[b]{\linewidth}\raggedright
Función
\end{minipage} \\
\midrule\noalign{}
\endhead
\bottomrule\noalign{}
\endlastfoot
\texttt{w} & Presionar \texttt{w} te lleva al inicio de la siguiente
palabra. \\
\texttt{b} & Presionar \texttt{b} te lleva al inicio de la palabra
anterior. \\
\texttt{e} & Presionar \texttt{e} te lleva al final de la palabra
actual. \\
\end{longtable}

\hypertarget{desplazamiento-por-puxe1ginas-y-ventanas}{%
\subsection{Desplazamiento por páginas y
ventanas}\label{desplazamiento-por-puxe1ginas-y-ventanas}}

\begin{longtable}[]{@{}
  >{\raggedright\arraybackslash}p{(\columnwidth - 2\tabcolsep) * \real{0.1288}}
  >{\raggedright\arraybackslash}p{(\columnwidth - 2\tabcolsep) * \real{0.8712}}@{}}
\toprule\noalign{}
\begin{minipage}[b]{\linewidth}\raggedright
Shortcuts or commands
\end{minipage} & \begin{minipage}[b]{\linewidth}\raggedright
Función
\end{minipage} \\
\midrule\noalign{}
\endhead
\bottomrule\noalign{}
\endlastfoot
\texttt{Ctrl\ +\ f} & Presionar \texttt{Ctrl\ +\ f} te permite
desplazarte hacia adelante por una página. \\
\texttt{Ctrl\ +\ b} & Presionar \texttt{Ctrl\ +\ b} te permite
desplazarte hacia atrás por una página. \\
\texttt{Ctrl\ +\ u} & Presionar \texttt{Ctrl\ +\ u} te permite
desplazarte hacia arriba por media página. \\
\texttt{Ctrl\ +\ d} & Presionar \texttt{Ctrl\ +\ d} te permite
desplazarte hacia abajo por media página. \\
\texttt{Ctrl\ +\ w\ +\ w} & Presionar \texttt{Ctrl\ +\ w\ +\ w} te
permite cambiar de ventana en Vim. Si tienes varias ventanas abiertas,
este atajo te lleva a la siguiente ventana. \\
\end{longtable}

\hypertarget{otros-movimientos-del-cursor}{%
\subsection{Otros movimientos del
cursor}\label{otros-movimientos-del-cursor}}

\begin{longtable}[]{@{}
  >{\raggedright\arraybackslash}p{(\columnwidth - 2\tabcolsep) * \real{0.2100}}
  >{\raggedright\arraybackslash}p{(\columnwidth - 2\tabcolsep) * \real{0.7900}}@{}}
\toprule\noalign{}
\begin{minipage}[b]{\linewidth}\raggedright
Shortcuts or commands
\end{minipage} & \begin{minipage}[b]{\linewidth}\raggedright
Función
\end{minipage} \\
\midrule\noalign{}
\endhead
\bottomrule\noalign{}
\endlastfoot
\texttt{W} & Mover el cursor al inicio de la siguiente palabra. \\
\texttt{B} & Mover el cursor al inicio de la palabra anterior. \\
\texttt{E} & Desplazar el cursor al final de la palabra actual. \\
\texttt{0} & Posicionar el cursor al inicio de la línea actual. \\
\texttt{\$} o \texttt{Fin} & Posicionar el cursor al final de la línea
actual. \\
\texttt{\_} & Mover el cursor al primer carácter que no sea un espacio
en la línea actual. \\
\texttt{+} & Mover el cursor al primer carácter que no sea un espacio en
la línea siguiente. \\
\texttt{-} & Mover el cursor al primer carácter que no sea un espacio en
la línea anterior. \\
\texttt{shift\ +\ a} & Posicionar el cursor al final de la línea actual
y cambiar al modo insertar. \\
\texttt{shift\ +\ 5} & Mover el cursor del inicio al final de un
paréntesis y viceversa. \\
\end{longtable}

\hypertarget{navegaciuxf3n-por-el-documento}{%
\subsection{Navegación por el
documento}\label{navegaciuxf3n-por-el-documento}}

\begin{longtable}[]{@{}
  >{\raggedright\arraybackslash}p{(\columnwidth - 2\tabcolsep) * \real{0.2360}}
  >{\raggedright\arraybackslash}p{(\columnwidth - 2\tabcolsep) * \real{0.7640}}@{}}
\toprule\noalign{}
\begin{minipage}[b]{\linewidth}\raggedright
Shortcuts or commands
\end{minipage} & \begin{minipage}[b]{\linewidth}\raggedright
Función
\end{minipage} \\
\midrule\noalign{}
\endhead
\bottomrule\noalign{}
\endlastfoot
\texttt{5\ +\ Enter} & Dirigirse 5 líneas más abajo respecto a la línea
actual. \\
\texttt{10gg} & Posicionar el cursor en la línea 10 del documento. \\
\texttt{16\ +\ Mayús\ +\ g} & Dirigirse a la línea 16 del fichero de
texto. \\
\texttt{\}} & Hacer saltar el cursor al párrafo siguiente. \\
\texttt{\{} & Hacer saltar el cursor al párrafo anterior. \\
\texttt{gi} & Posicionar el cursor en la última palabra editada del
buffer actual. \\
\texttt{H} & Mover el cursor a la parte superior de la pantalla. \\
\texttt{M} & Mover el cursor a la parte media de la pantalla. \\
\texttt{L} & Mover el cursor a la parte inferior de la pantalla. \\
\texttt{Ctrl+e} & Mover una pantalla hacia abajo sin mover el cursor de
posición. \\
\texttt{Ctrl+y} & Mover una pantalla hacia arriba sin mover el cursor de
posición. \\
\end{longtable}

\hypertarget{atajos-de-ediciuxf3n-y-manipulaciuxf3n-de-texto}{%
\section{Atajos de edición y manipulación de
texto}\label{atajos-de-ediciuxf3n-y-manipulaciuxf3n-de-texto}}

En Vim, existen diversos atajos de teclado que te permiten realizar
acciones de edición y manipulación de texto de forma eficiente. A
continuación, se muestran las diferentes categorías de atajos junto con
sus respectivas funciones explicadas:

\hypertarget{inserciuxf3n-y-ediciuxf3n-de-texto}{%
\subsection{Inserción y edición de
texto}\label{inserciuxf3n-y-ediciuxf3n-de-texto}}

\begin{longtable}[]{@{}
  >{\raggedright\arraybackslash}p{(\columnwidth - 2\tabcolsep) * \real{0.1765}}
  >{\raggedright\arraybackslash}p{(\columnwidth - 2\tabcolsep) * \real{0.8235}}@{}}
\toprule\noalign{}
\begin{minipage}[b]{\linewidth}\raggedright
Shortcuts or commands
\end{minipage} & \begin{minipage}[b]{\linewidth}\raggedright
Función
\end{minipage} \\
\midrule\noalign{}
\endhead
\bottomrule\noalign{}
\endlastfoot
\texttt{i} & Presionar \texttt{i} te permite insertar texto en el lugar
donde se encuentra el cursor. \\
\texttt{a} & Presionar \texttt{a} te permite agregar texto después del
lugar donde se encuentra el cursor. \\
\texttt{A} & Presionar \texttt{A} te lleva al final de la línea actual
para comenzar a insertar texto. \\
\texttt{o} & Presionar \texttt{o} te permite abrir una nueva línea
debajo de la línea actual para comenzar a escribir. \\
\texttt{Ctrl+\ flecha\ derecha} & Avanzar una palabra hacia la derecha
en modo inserción. \\
\texttt{Ctrl+\ flecha\ izquierda} & Avanzar una palabra hacia la
izquierda en modo inserción. \\
\texttt{I} & Insertar texto al principio de una línea. \\
\texttt{A} & Insertar texto al final de una línea. \\
\texttt{O} & Crear una línea en blanco justo encima de la línea actual y
pasar al modo inserción. \\
\texttt{ea} & Insertar texto después de una palabra. \\
\texttt{Ctrl\ +\ p} & Autocompletar la palabra que tenemos escrita a
medias con otra palabra anterior al cursor. \\
\texttt{Ctrl\ +\ w\ +\ cursores} & Cambiar entre las distintas ventanas
abiertas en Vim. \\
\texttt{Ctrl+g} & Mostrar el número de líneas de un fichero. \\
\end{longtable}

\hypertarget{inserciuxf2n-de-texto-repetitivo-o-secuencial}{%
\subsection{Inserciòn de texto repetitivo o
secuencial}\label{inserciuxf2n-de-texto-repetitivo-o-secuencial}}

\begin{longtable}[]{@{}
  >{\raggedright\arraybackslash}p{(\columnwidth - 2\tabcolsep) * \real{0.5429}}
  >{\raggedright\arraybackslash}p{(\columnwidth - 2\tabcolsep) * \real{0.4571}}@{}}
\toprule\noalign{}
\begin{minipage}[b]{\linewidth}\raggedright
Shortcuts or commands
\end{minipage} & \begin{minipage}[b]{\linewidth}\raggedright
Función
\end{minipage} \\
\midrule\noalign{}
\endhead
\bottomrule\noalign{}
\endlastfoot
\texttt{40i-} + \texttt{Esc} + \texttt{Enter} & Inserta 40 guiones. \\
\texttt{10i} + \texttt{hola\ mundo} + \texttt{Esc} + \texttt{Enter} &
Escribe ``hola mundo'' 10 veces. \\
\texttt{i} + \texttt{achalma} + \texttt{Esc} + \texttt{20.} & Escribe
``achalma'' 20 veces. \\
\texttt{:put=range(1,10)} & Escribe los números del 1 al 10. \\
\end{longtable}

\hypertarget{eliminaciuxf3n-de-caracteres-palabras-y-luxedneas}{%
\subsection{Eliminación de caracteres, palabras y
líneas}\label{eliminaciuxf3n-de-caracteres-palabras-y-luxedneas}}

\begin{longtable}[]{@{}
  >{\raggedright\arraybackslash}p{(\columnwidth - 2\tabcolsep) * \real{0.2143}}
  >{\raggedright\arraybackslash}p{(\columnwidth - 2\tabcolsep) * \real{0.7857}}@{}}
\toprule\noalign{}
\begin{minipage}[b]{\linewidth}\raggedright
Shortcuts or commands
\end{minipage} & \begin{minipage}[b]{\linewidth}\raggedright
Función
\end{minipage} \\
\midrule\noalign{}
\endhead
\bottomrule\noalign{}
\endlastfoot
\texttt{x} & Presionar \texttt{x} te permite eliminar el carácter bajo
el cursor. \\
\texttt{dw} & Presionar \texttt{dw} te permite eliminar la palabra desde
el cursor hasta el final. \\
\texttt{dd} & Presionar \texttt{dd} te permite eliminar toda la línea
actual. \\
\end{longtable}

\hypertarget{copiar-y-pegar-texto}{%
\subsection{Copiar y pegar texto}\label{copiar-y-pegar-texto}}

\begin{longtable}[]{@{}
  >{\raggedright\arraybackslash}p{(\columnwidth - 2\tabcolsep) * \real{0.1927}}
  >{\raggedright\arraybackslash}p{(\columnwidth - 2\tabcolsep) * \real{0.8073}}@{}}
\toprule\noalign{}
\begin{minipage}[b]{\linewidth}\raggedright
Shortcuts or commands
\end{minipage} & \begin{minipage}[b]{\linewidth}\raggedright
Función
\end{minipage} \\
\midrule\noalign{}
\endhead
\bottomrule\noalign{}
\endlastfoot
\texttt{yy} & Presionar \texttt{yy} te permite copiar la línea
actual. \\
\texttt{2yy} & Copia 2 líneas a partir de donde está el cursor. \\
\texttt{y\$} & Copia desde la posición actual del cursor hasta el final
de la línea. \\
\texttt{yw} & Copia la palabra a partir de la posición actual del cursor
hasta el final de la palabra. \\
\texttt{yiw} & Copia la palabra actual. \\
\texttt{:10,20y} & Copia desde la línea 10 hasta la línea 20. \\
\texttt{120y} & Copia la línea 120. \\
\texttt{dd} & Corta la línea completa. \\
\texttt{2dd} & Corta 2 líneas a partir de donde está el cursor. \\
\texttt{d\$} o \texttt{D} & Corta desde la posición actual del cursor
hasta el final de la línea. \\
\texttt{dw} & Corta la palabra desde la posición actual del cursor hasta
el final de la palabra. \\
\texttt{diw} & Corta la palabra actual. \\
\texttt{x} & Corta un solo carácter. \\
\texttt{p} & Pega una línea debajo de la posición actual del cursor. \\
\texttt{:129put} & Pega el contenido del portapapeles en la línea 129
del documento. \\
\texttt{shift\ +\ p} & Pega una línea arriba de la posición actual del
cursor. \\
\end{longtable}

``¡No te preocupes, también puedes copiar y pegar texto en VIM
utilizando el modo visual! Sigue estos pasos sencillos:

\begin{enumerate}
\def\labelenumi{\arabic{enumi}.}
\tightlist
\item
  Posiciona el cursor en el punto desde donde deseas comenzar a copiar.
\item
  Presiona la letra `v' para entrar en el modo visual.
\item
  Utiliza los cursores o las teclas `j', `k', `h' y `l' para seleccionar
  el texto que deseas copiar, cortar o eliminar.
\item
  Si quieres copiar el texto seleccionado, simplemente presiona la tecla
  `y'. Si prefieres cortarlo, presiona `d'. Y si deseas eliminarlo por
  completo, usa la tecla `x'.
\item
  ¡Listo! Ahora puedes pegar el texto donde desees. Para ello, sal del
  modo visual y vuelve al modo normal, y luego presiona la tecla `p'.
\end{enumerate}

\hypertarget{mover-una-luxednea-de-posiciuxf3n}{%
\subsection{Mover una línea de
posición}\label{mover-una-luxednea-de-posiciuxf3n}}

\begin{longtable}[]{@{}
  >{\raggedright\arraybackslash}p{(\columnwidth - 2\tabcolsep) * \real{0.1221}}
  >{\raggedright\arraybackslash}p{(\columnwidth - 2\tabcolsep) * \real{0.8779}}@{}}
\toprule\noalign{}
\begin{minipage}[b]{\linewidth}\raggedright
Shortcuts or commands
\end{minipage} & \begin{minipage}[b]{\linewidth}\raggedright
Función
\end{minipage} \\
\midrule\noalign{}
\endhead
\bottomrule\noalign{}
\endlastfoot
\texttt{:.-1m.+1} & Mueve la línea justo encima de la línea actual a la
línea inferior justo después de la línea actual. \\
\texttt{:19m17} & Mueve la línea 19 a la posición de la línea 17. \\
\texttt{:.-4,.-2m.+8} & Mueve las líneas desde 4 líneas antes de la
actual hasta 2 líneas antes de la actual, a una posición 8 líneas
después de la posición actual del cursor. \\
\end{longtable}

\hypertarget{selecciuxf3n-de-texto-y-acciones-en-modo-visual}{%
\section{Selección de texto y acciones en modo
visual}\label{selecciuxf3n-de-texto-y-acciones-en-modo-visual}}

\hypertarget{modo-visual}{%
\subsection{Modo Visual}\label{modo-visual}}

\begin{longtable}[]{@{}
  >{\raggedright\arraybackslash}p{(\columnwidth - 2\tabcolsep) * \real{0.2917}}
  >{\raggedright\arraybackslash}p{(\columnwidth - 2\tabcolsep) * \real{0.7083}}@{}}
\toprule\noalign{}
\begin{minipage}[b]{\linewidth}\raggedright
Shortcuts or commands
\end{minipage} & \begin{minipage}[b]{\linewidth}\raggedright
Función
\end{minipage} \\
\midrule\noalign{}
\endhead
\bottomrule\noalign{}
\endlastfoot
\texttt{v} & Accede al modo visual. \\
\texttt{V} & Inicia el modo visual seleccionando toda una línea. \\
\texttt{Ctrl\ +\ v} & Inicia el modo Bloque visual. \\
\end{longtable}

\hypertarget{selecciuxf3n-de-palabras}{%
\subsection{Selección de Palabras}\label{selecciuxf3n-de-palabras}}

\begin{longtable}[]{@{}
  >{\raggedright\arraybackslash}p{(\columnwidth - 2\tabcolsep) * \real{0.1909}}
  >{\raggedright\arraybackslash}p{(\columnwidth - 2\tabcolsep) * \real{0.8091}}@{}}
\toprule\noalign{}
\begin{minipage}[b]{\linewidth}\raggedright
Shortcuts or commands
\end{minipage} & \begin{minipage}[b]{\linewidth}\raggedright
Función
\end{minipage} \\
\midrule\noalign{}
\endhead
\bottomrule\noalign{}
\endlastfoot
\texttt{iw} & Selecciona la palabra donde se encuentra el cursor. \\
\texttt{aw} & Selecciona la palabra donde se encuentra el cursor,
incluyendo el espacio que la precede. \\
\end{longtable}

\hypertarget{selecciuxf3n-de-bloques}{%
\subsection{Selección de Bloques}\label{selecciuxf3n-de-bloques}}

\begin{longtable}[]{@{}
  >{\raggedright\arraybackslash}p{(\columnwidth - 2\tabcolsep) * \real{0.1750}}
  >{\raggedright\arraybackslash}p{(\columnwidth - 2\tabcolsep) * \real{0.8250}}@{}}
\toprule\noalign{}
\begin{minipage}[b]{\linewidth}\raggedright
Shortcuts or commands
\end{minipage} & \begin{minipage}[b]{\linewidth}\raggedright
Función
\end{minipage} \\
\midrule\noalign{}
\endhead
\bottomrule\noalign{}
\endlastfoot
\texttt{ab} & Selecciona un bloque de texto delimitado por paréntesis
(). La selección incluye los paréntesis. \\
\texttt{it} & Selecciona un bloque de texto delimitado por paréntesis
(). La selección no incluye los paréntesis. \\
\texttt{aB} & Selecciona un bloque de texto delimitado por corchetes
\{\}. La selección incluye los corchetes. \\
\texttt{iB} & Selecciona un bloque de texto delimitado por corchetes
\{\}. La selección no incluye los corchetes. \\
\texttt{at} & Selecciona un bloque de texto delimitado por etiquetas
\textless\textgreater{} y \textless/\textgreater{} incluyendo las
etiquetas. \\
\texttt{it} & Selecciona un bloque de texto delimitado por etiquetas
\textless\textgreater{} y \textless/\textgreater{} sin incluir las
etiquetas. \\
\end{longtable}

\hypertarget{otras-acciones}{%
\subsection{Otras Acciones}\label{otras-acciones}}

\begin{longtable}[]{@{}
  >{\raggedright\arraybackslash}p{(\columnwidth - 2\tabcolsep) * \real{0.1892}}
  >{\raggedright\arraybackslash}p{(\columnwidth - 2\tabcolsep) * \real{0.8108}}@{}}
\toprule\noalign{}
\begin{minipage}[b]{\linewidth}\raggedright
Shortcuts or commands
\end{minipage} & \begin{minipage}[b]{\linewidth}\raggedright
Función
\end{minipage} \\
\midrule\noalign{}
\endhead
\bottomrule\noalign{}
\endlastfoot
\texttt{o} & Mueve el cursor a la parte inicial de un bloque delimitado
por (), \{\}, \textless\textgreater..\textless/\textgreater{} \\
\texttt{O} & Mueve el cursor a la parte final de un bloque delimitado
por (), \{\}, \textless\textgreater..\textless/\textgreater{} \\
\texttt{j} & Selecciona una frase entera y se mueve a la siguiente
línea. \\
\texttt{is} & Selecciona una frase hasta el primer punto. \\
\texttt{ip} & Selecciona un párrafo completo. \\
\texttt{b} & Selecciona desde el cursor hasta el inicio de la
palabra. \\
\texttt{e} & Selecciona desde el cursor hasta el final de la palabra. \\
\texttt{\$} & Selecciona desde el cursor hasta el final de la línea. \\
\texttt{\^{}} & Selecciona desde el cursor hasta el primer carácter
imprimible de la línea. \\
\texttt{awd\$p} & Selecciona una palabra y la mueve al final de un
párrafo. \\
\texttt{u} & Transforma todo el texto seleccionado en minúsculas. \\
\texttt{U} & Transforma todo el texto seleccionado en mayúsculas. \\
\texttt{\textgreater{}} & Mueve la línea en la que se encuentra el
cursor a la derecha. (Aplica sangría al texto.) \\
\texttt{\textless{}} & Mueve la línea en la que se encuentra el cursor a
la izquierda. (Aplica sangría al texto.) \\
\texttt{Esc} & Sale del modo visual. \\
\end{longtable}

\hypertarget{crear-marcas-en-un-fichero}{%
\section{Crear marcas en un fichero}\label{crear-marcas-en-un-fichero}}

\hypertarget{crear-marcas}{%
\subsection{Crear Marcas}\label{crear-marcas}}

\begin{longtable}[]{@{}
  >{\raggedright\arraybackslash}p{(\columnwidth - 2\tabcolsep) * \real{0.2143}}
  >{\raggedright\arraybackslash}p{(\columnwidth - 2\tabcolsep) * \real{0.7857}}@{}}
\toprule\noalign{}
\begin{minipage}[b]{\linewidth}\raggedright
Shortcuts or commands
\end{minipage} & \begin{minipage}[b]{\linewidth}\raggedright
Función
\end{minipage} \\
\midrule\noalign{}
\endhead
\bottomrule\noalign{}
\endlastfoot
\texttt{m\ +\ {[}a-z{]}} & Crea una marca local con el nombre
especificado. Puedes usar letras de a a z. \\
\end{longtable}

\hypertarget{trabajar-con-marcas}{%
\subsection{Trabajar con Marcas}\label{trabajar-con-marcas}}

\begin{longtable}[]{@{}
  >{\raggedright\arraybackslash}p{(\columnwidth - 2\tabcolsep) * \real{0.2100}}
  >{\raggedright\arraybackslash}p{(\columnwidth - 2\tabcolsep) * \real{0.7900}}@{}}
\toprule\noalign{}
\begin{minipage}[b]{\linewidth}\raggedright
Shortcuts or commands
\end{minipage} & \begin{minipage}[b]{\linewidth}\raggedright
Función
\end{minipage} \\
\midrule\noalign{}
\endhead
\bottomrule\noalign{}
\endlastfoot
\texttt{{[}a-z{]}} & Posiciona el cursor en la marca especificada.
Utiliza la letra correspondiente. \\
\texttt{\textquotesingle{}} & Regresa al inicio de la línea en la que se
creó la marca. \\
`\texttt{a} & Mueve el cursor al inicio de la línea en la que se creó la
marca. \\
` & Salta a la marca anterior. Si estás en la marca b, te trasladarás a
la marca a. \\
\end{longtable}

\hypertarget{eliminar-marcas}{%
\subsection{Eliminar Marcas}\label{eliminar-marcas}}

\begin{longtable}[]{@{}
  >{\raggedright\arraybackslash}p{(\columnwidth - 2\tabcolsep) * \real{0.2258}}
  >{\raggedright\arraybackslash}p{(\columnwidth - 2\tabcolsep) * \real{0.7742}}@{}}
\toprule\noalign{}
\begin{minipage}[b]{\linewidth}\raggedright
Shortcuts or commands
\end{minipage} & \begin{minipage}[b]{\linewidth}\raggedright
Función
\end{minipage} \\
\midrule\noalign{}
\endhead
\bottomrule\noalign{}
\endlastfoot
\texttt{:delmarks\ a} & Elimina la marca especificada (en este ejemplo,
la marca a). \\
\texttt{:delmarks!} & Elimina todas las marcas locales o minúsculas del
fichero. \\
\texttt{:delmarks\ a-d} & Elimina todas las marcas desde la marca a
hasta la marca d. \\
\texttt{:delmarks\ aBc} & Elimina las marcas especificadas (en este
ejemplo, las marcas a, B y c). \\
\end{longtable}

Nota: Si asignamos una marca con letras minúsculas, esta será una marca
local y solo estará disponible mientras el fichero esté abierto. Si
deseamos que la marca sea permanente, debemos asignarla con letras
mayúsculas, creando así marcas globales.

\hypertarget{aplicaciuxf3n-de-sangruxedas-a-un-texto}{%
\section{Aplicación de sangrías a un
texto}\label{aplicaciuxf3n-de-sangruxedas-a-un-texto}}

\hypertarget{aplicar-sangruxedas}{%
\subsection{Aplicar Sangrías}\label{aplicar-sangruxedas}}

\begin{longtable}[]{@{}
  >{\raggedright\arraybackslash}p{(\columnwidth - 2\tabcolsep) * \real{0.2500}}
  >{\raggedright\arraybackslash}p{(\columnwidth - 2\tabcolsep) * \real{0.7500}}@{}}
\toprule\noalign{}
\begin{minipage}[b]{\linewidth}\raggedright
Shortcuts or commands
\end{minipage} & \begin{minipage}[b]{\linewidth}\raggedright
Función
\end{minipage} \\
\midrule\noalign{}
\endhead
\bottomrule\noalign{}
\endlastfoot
\texttt{\textgreater{}\textgreater{}} & Sangra el texto hacia la
derecha. \\
\texttt{3\ \textgreater{}\textgreater{}} & Realiza un sangrado en 3
líneas a partir de la posición actual. \\
\end{longtable}

\hypertarget{deshacer-sangruxeda}{%
\subsection{Deshacer Sangría}\label{deshacer-sangruxeda}}

\begin{longtable}[]{@{}
  >{\raggedright\arraybackslash}p{(\columnwidth - 2\tabcolsep) * \real{0.2692}}
  >{\raggedright\arraybackslash}p{(\columnwidth - 2\tabcolsep) * \real{0.7308}}@{}}
\toprule\noalign{}
\begin{minipage}[b]{\linewidth}\raggedright
Shortcuts or commands
\end{minipage} & \begin{minipage}[b]{\linewidth}\raggedright
Función
\end{minipage} \\
\midrule\noalign{}
\endhead
\bottomrule\noalign{}
\endlastfoot
\texttt{\textless{}\textless{}} & Deshace la sangría, moviendo el texto
hacia la izquierda. \\
\end{longtable}

\hypertarget{pegar-y-ajustar-a-la-sangruxeda-actual}{%
\subsection{Pegar y Ajustar a la Sangría
Actual}\label{pegar-y-ajustar-a-la-sangruxeda-actual}}

\begin{longtable}[]{@{}
  >{\raggedright\arraybackslash}p{(\columnwidth - 2\tabcolsep) * \real{0.2386}}
  >{\raggedright\arraybackslash}p{(\columnwidth - 2\tabcolsep) * \real{0.7614}}@{}}
\toprule\noalign{}
\begin{minipage}[b]{\linewidth}\raggedright
Shortcuts or commands
\end{minipage} & \begin{minipage}[b]{\linewidth}\raggedright
Función
\end{minipage} \\
\midrule\noalign{}
\endhead
\bottomrule\noalign{}
\endlastfoot
\texttt{{]}p} & Pega el contenido del portapapeles y lo ajusta a la
sangría actual. \\
\end{longtable}

\hypertarget{aplicar-sangruxeda-en-modo-visual}{%
\subsection{Aplicar Sangría en Modo
Visual}\label{aplicar-sangruxeda-en-modo-visual}}

\begin{longtable}[]{@{}
  >{\raggedright\arraybackslash}p{(\columnwidth - 2\tabcolsep) * \real{0.2442}}
  >{\raggedright\arraybackslash}p{(\columnwidth - 2\tabcolsep) * \real{0.7558}}@{}}
\toprule\noalign{}
\begin{minipage}[b]{\linewidth}\raggedright
Shortcuts or commands
\end{minipage} & \begin{minipage}[b]{\linewidth}\raggedright
Función
\end{minipage} \\
\midrule\noalign{}
\endhead
\bottomrule\noalign{}
\endlastfoot
\texttt{v\ +\ j\ +\ \textgreater{}} & Selecciona una frase en modo
visual y la sangra hacia la derecha. \\
\end{longtable}

\hypertarget{aplicar-sangruxeda-en-modo-insertar}{%
\subsection{Aplicar Sangría en Modo
Insertar}\label{aplicar-sangruxeda-en-modo-insertar}}

\begin{longtable}[]{@{}
  >{\raggedright\arraybackslash}p{(\columnwidth - 4\tabcolsep) * \real{0.2165}}
  >{\raggedright\arraybackslash}p{(\columnwidth - 4\tabcolsep) * \real{0.7526}}
  >{\raggedright\arraybackslash}p{(\columnwidth - 4\tabcolsep) * \real{0.0309}}@{}}
\toprule\noalign{}
\begin{minipage}[b]{\linewidth}\raggedright
Shortcuts or commands
\end{minipage} & \begin{minipage}[b]{\linewidth}\raggedright
Función
\end{minipage} & \begin{minipage}[b]{\linewidth}\raggedright
\end{minipage} \\
\midrule\noalign{}
\endhead
\bottomrule\noalign{}
\endlastfoot
\texttt{Ctrl+t} & Aplica una sangría moviendo el texto hacia la derecha
en modo insertar. & X \\
\texttt{Ctrl+d} & Aplica una sangría moviendo el texto hacia la
izquierda en modo insertar. & \\
\end{longtable}

\hypertarget{deshacer-y-rehacer-cambios}{%
\section{Deshacer y rehacer cambios}\label{deshacer-y-rehacer-cambios}}

\hypertarget{deshacer-cambios}{%
\subsection{Deshacer Cambios}\label{deshacer-cambios}}

\begin{longtable}[]{@{}
  >{\raggedright\arraybackslash}p{(\columnwidth - 2\tabcolsep) * \real{0.4306}}
  >{\raggedright\arraybackslash}p{(\columnwidth - 2\tabcolsep) * \real{0.5694}}@{}}
\toprule\noalign{}
\begin{minipage}[b]{\linewidth}\raggedright
Shortcuts or commands o comando
\end{minipage} & \begin{minipage}[b]{\linewidth}\raggedright
Función
\end{minipage} \\
\midrule\noalign{}
\endhead
\bottomrule\noalign{}
\endlastfoot
\texttt{u} & Deshace el último cambio realizado. \\
\texttt{2u} & Deshace los 2 últimos cambios realizados. \\
\end{longtable}

\hypertarget{restaurar-cambio}{%
\subsection{Restaurar Cambio}\label{restaurar-cambio}}

\begin{longtable}[]{@{}
  >{\raggedright\arraybackslash}p{(\columnwidth - 2\tabcolsep) * \real{0.2562}}
  >{\raggedright\arraybackslash}p{(\columnwidth - 2\tabcolsep) * \real{0.7438}}@{}}
\toprule\noalign{}
\begin{minipage}[b]{\linewidth}\raggedright
Shortcuts or commands o comando
\end{minipage} & \begin{minipage}[b]{\linewidth}\raggedright
Función
\end{minipage} \\
\midrule\noalign{}
\endhead
\bottomrule\noalign{}
\endlastfoot
\texttt{U} & Restaura la última línea modificada. No se puede rehacer el
cambio después de restaurarlo. \\
\end{longtable}

\hypertarget{rehacer-cambios}{%
\subsection{Rehacer Cambios}\label{rehacer-cambios}}

\begin{longtable}[]{@{}
  >{\raggedright\arraybackslash}p{(\columnwidth - 2\tabcolsep) * \real{0.4429}}
  >{\raggedright\arraybackslash}p{(\columnwidth - 2\tabcolsep) * \real{0.5571}}@{}}
\toprule\noalign{}
\begin{minipage}[b]{\linewidth}\raggedright
Shortcuts or commands o comando
\end{minipage} & \begin{minipage}[b]{\linewidth}\raggedright
Función
\end{minipage} \\
\midrule\noalign{}
\endhead
\bottomrule\noalign{}
\endlastfoot
\texttt{Ctrl\ +\ r\ or\ :redo} & Rehace el último cambio realizado. \\
\texttt{3\ +\ Ctrl\ +\ r} & Rehace los últimos 3 cambios deshechos. \\
\end{longtable}

\hypertarget{deshacer-y-rehacer-por-tiempo}{%
\subsection{Deshacer y Rehacer por
Tiempo}\label{deshacer-y-rehacer-por-tiempo}}

\begin{longtable}[]{@{}
  >{\raggedright\arraybackslash}p{(\columnwidth - 2\tabcolsep) * \real{0.3605}}
  >{\raggedright\arraybackslash}p{(\columnwidth - 2\tabcolsep) * \real{0.6395}}@{}}
\toprule\noalign{}
\begin{minipage}[b]{\linewidth}\raggedright
Shortcuts or commands o comando
\end{minipage} & \begin{minipage}[b]{\linewidth}\raggedright
Función
\end{minipage} \\
\midrule\noalign{}
\endhead
\bottomrule\noalign{}
\endlastfoot
\texttt{:earlier\ 1h} & Deshace todos los cambios realizados en la
última hora. \\
\texttt{:later\ 30m} & Rehace los cambios realizados en la última media
hora. \\
\end{longtable}

\hypertarget{eliminar-carateres-y-lineas}{%
\section{Eliminar carateres y
lineas}\label{eliminar-carateres-y-lineas}}

\hypertarget{eliminar-caracteres}{%
\subsection{Eliminar Caracteres}\label{eliminar-caracteres}}

\begin{longtable}[]{@{}
  >{\raggedright\arraybackslash}p{(\columnwidth - 2\tabcolsep) * \real{0.3647}}
  >{\raggedright\arraybackslash}p{(\columnwidth - 2\tabcolsep) * \real{0.6353}}@{}}
\toprule\noalign{}
\begin{minipage}[b]{\linewidth}\raggedright
Shortcuts or commands o comando
\end{minipage} & \begin{minipage}[b]{\linewidth}\raggedright
Función
\end{minipage} \\
\midrule\noalign{}
\endhead
\bottomrule\noalign{}
\endlastfoot
\texttt{x} & Elimina un carácter en modo normal. \\
\texttt{3x} & Elimina 3 caracteres a partir de donde está el cursor. \\
\texttt{X} & Borra un carácter a la izquierda del cursor. \\
\texttt{r} & Reemplaza un carácter. \\
\texttt{s} & Elimina un carácter y cambia al modo insertar. \\
\end{longtable}

\hypertarget{eliminar-palabras}{%
\subsection{Eliminar Palabras}\label{eliminar-palabras}}

\begin{longtable}[]{@{}
  >{\raggedright\arraybackslash}p{(\columnwidth - 2\tabcolsep) * \real{0.2952}}
  >{\raggedright\arraybackslash}p{(\columnwidth - 2\tabcolsep) * \real{0.7048}}@{}}
\toprule\noalign{}
\begin{minipage}[b]{\linewidth}\raggedright
Shortcuts or commands o comando
\end{minipage} & \begin{minipage}[b]{\linewidth}\raggedright
Función
\end{minipage} \\
\midrule\noalign{}
\endhead
\bottomrule\noalign{}
\endlastfoot
\texttt{dw} & Borra la palabra desde el cursor hasta el final. \\
\texttt{cw} & Borra la palabra desde el cursor hasta el final y cambia
al modo insertar. \\
\texttt{diw} & Borra la palabra completa bajo el cursor. \\
\texttt{ciw} & Borra la palabra completa bajo el cursor y cambia al modo
insertar. \\
\end{longtable}

\hypertarget{eliminar-luxedneas}{%
\subsection{Eliminar Líneas}\label{eliminar-luxedneas}}

\begin{longtable}[]{@{}
  >{\raggedright\arraybackslash}p{(\columnwidth - 2\tabcolsep) * \real{0.2793}}
  >{\raggedright\arraybackslash}p{(\columnwidth - 2\tabcolsep) * \real{0.7207}}@{}}
\toprule\noalign{}
\begin{minipage}[b]{\linewidth}\raggedright
Shortcuts or commands o comando
\end{minipage} & \begin{minipage}[b]{\linewidth}\raggedright
Función
\end{minipage} \\
\midrule\noalign{}
\endhead
\bottomrule\noalign{}
\endlastfoot
\texttt{dd} & Borra/corta una línea completa. \\
\texttt{shift\ +\ d} & Borra desde el cursor hasta el final de la
línea. \\
\texttt{d0} & Borra desde el cursor hasta el inicio de la línea. \\
\texttt{cc\ o\ S} & Borra toda la línea y cambia al modo insertar. \\
\texttt{shift\ +\ c} & Borra desde el cursor hasta el final de la línea
y cambia al modo insertar. \\
\texttt{c0} & Borra desde el cursor hasta el inicio de la línea y cambia
al modo insertar. \\
\texttt{2dd} & Borra/corta 2 líneas a partir de donde está el cursor. \\
\texttt{2cc} & Borra/corta 2 líneas a partir de donde está el cursor y
cambia al modo insertar. \\
\end{longtable}

\hypertarget{eliminar-por-patruxf3n}{%
\subsection{Eliminar por Patrón}\label{eliminar-por-patruxf3n}}

\begin{longtable}[]{@{}
  >{\raggedright\arraybackslash}p{(\columnwidth - 2\tabcolsep) * \real{0.2719}}
  >{\raggedright\arraybackslash}p{(\columnwidth - 2\tabcolsep) * \real{0.7281}}@{}}
\toprule\noalign{}
\begin{minipage}[b]{\linewidth}\raggedright
Shortcuts or commands o comando
\end{minipage} & \begin{minipage}[b]{\linewidth}\raggedright
Función
\end{minipage} \\
\midrule\noalign{}
\endhead
\bottomrule\noalign{}
\endlastfoot
\texttt{:\%d} & Borra todas las líneas del archivo. \\
\texttt{dgg} & Borra desde el cursor hasta el inicio del archivo. \\
\texttt{dG} & Borra desde el cursor hasta el final del archivo. \\
\texttt{:{[}2{]},{[}4{]}} & Borra las líneas de la 2 a la 4. \\
\texttt{:1,.-1d} & Borra todas las líneas antes de la línea actual. \\
\texttt{:.+1,\$d} & Borra todas las líneas después de la línea
actual. \\
\texttt{:g/\textless{}patrón\textgreater{}/d} & Borra todas las líneas
que contienen un patrón específico. \\
\texttt{:g!/\textless{}patrón\textgreater{}/d} & Borra todas las líneas
que no contienen un patrón específico. \\
\texttt{:g/\^{}A/d} & Borra todas las líneas que comienzan con la letra
A. \\
\texttt{:g/\^{}\$/d} & Borra todas las líneas vacías en el documento. \\
\texttt{:.-2,.+8d} & Borra las líneas desde 2 líneas encima hasta 8
líneas debajo de la posición actual. \\
\end{longtable}

¡Hay otra forma de eliminar texto en Vim utilizando el modo visual!

Si quieres utilizar marcas para eliminar texto, simplemente sigue estos
pasos:

\begin{enumerate}
\def\labelenumi{\arabic{enumi}.}
\tightlist
\item
  Ve a la primera línea que deseas borrar y crea una marca, por ejemplo,
  presiona \texttt{ma}.
\item
  Luego, coloca el cursor en la última línea que deseas eliminar.
\item
  Finalmente, presiona \texttt{d\textquotesingle{}a} y todas las líneas
  desde la posición actual del cursor hasta la línea marcada con la
  marca \texttt{a} se eliminarán.
\end{enumerate}

Por otro lado, si prefieres eliminar texto en el modo visual, puedes
consultar el apartado de ``Atajos de teclado para copiar y pegar texto
en Vim''. ¡Es una forma rápida y eficiente de seleccionar y eliminar el
texto que desees!

\hypertarget{buxfasqueda-y-reemplazo}{%
\section{Búsqueda y reemplazo}\label{buxfasqueda-y-reemplazo}}

En Vim, la capacidad de buscar y reemplazar texto de manera eficiente es
fundamental. A continuación, se presentan las diferentes categorías de
acciones de búsqueda y reemplazo, junto con sus respectivos atajos de
teclado y comandos:

\hypertarget{buxfasqueda-de-texto-o-frase}{%
\subsection{Búsqueda de texto o
frase}\label{buxfasqueda-de-texto-o-frase}}

\begin{longtable}[]{@{}
  >{\raggedright\arraybackslash}p{(\columnwidth - 2\tabcolsep) * \real{0.1361}}
  >{\raggedright\arraybackslash}p{(\columnwidth - 2\tabcolsep) * \real{0.8639}}@{}}
\toprule\noalign{}
\begin{minipage}[b]{\linewidth}\raggedright
Shortcuts or commands
\end{minipage} & \begin{minipage}[b]{\linewidth}\raggedright
Función
\end{minipage} \\
\midrule\noalign{}
\endhead
\bottomrule\noalign{}
\endlastfoot
\texttt{/palabra\_a\_buscar} & Presionar \texttt{/} seguido del texto
que deseas buscar te lleva a la primera aparición de ese texto en el
archivo. \\
\texttt{n} & Presionar \texttt{n} te lleva a la siguiente aparición de
la búsqueda realizada previamente. \\
\texttt{N} & Presionar \texttt{N} te lleva a la aparición anterior de la
búsqueda realizada previamente. \\
\texttt{?palabra\_a\_busca} & Busca todas las palabras que contienen un
texto determinado y coloca el cursor en la primera ocurrencia anterior a
la ubicación actual del cursor. \\
\texttt{n} & Después de realizar una búsqueda, ``n'' lleva al siguiente
resultado de búsqueda. \\
\texttt{Shift+n} & Similar a ``n'', pero se desplaza hacia atrás en los
resultados de búsqueda. \\
\texttt{ggn} & Va a la primera aparición de la palabra buscada. \\
\texttt{Gn} & Va a la última aparición de la palabra buscada. \\
\texttt{3/palabra\_a\_busca} & Buscará todas las palabras que contengan
``palabra\_a\_busca'' y colocará el cursor en la tercera ocurrencia. \\
\end{longtable}

\hypertarget{navegaciuxf3n-entre-coincidencias-de-buxfasqueda}{%
\subsection{Navegación entre coincidencias de
búsqueda}\label{navegaciuxf3n-entre-coincidencias-de-buxfasqueda}}

\begin{longtable}[]{@{}
  >{\raggedright\arraybackslash}p{(\columnwidth - 2\tabcolsep) * \real{0.1810}}
  >{\raggedright\arraybackslash}p{(\columnwidth - 2\tabcolsep) * \real{0.8190}}@{}}
\toprule\noalign{}
\begin{minipage}[b]{\linewidth}\raggedright
Shortcuts or commands
\end{minipage} & \begin{minipage}[b]{\linewidth}\raggedright
Función
\end{minipage} \\
\midrule\noalign{}
\endhead
\bottomrule\noalign{}
\endlastfoot
\texttt{*} & Presionar \texttt{*} te lleva a la siguiente aparición de
la palabra en la que se encuentra el cursor. \\
\texttt{\#} & Presionar \texttt{\#} te lleva a la aparición anterior de
la palabra en la que se encuentra el cursor. \\
\end{longtable}

\hypertarget{reemplazo-de-texto}{%
\subsection{Reemplazo de texto}\label{reemplazo-de-texto}}

\begin{longtable}[]{@{}
  >{\raggedright\arraybackslash}p{(\columnwidth - 2\tabcolsep) * \real{0.1603}}
  >{\raggedright\arraybackslash}p{(\columnwidth - 2\tabcolsep) * \real{0.8397}}@{}}
\toprule\noalign{}
\begin{minipage}[b]{\linewidth}\raggedright
Shortcuts or commands
\end{minipage} & \begin{minipage}[b]{\linewidth}\raggedright
Función
\end{minipage} \\
\midrule\noalign{}
\endhead
\bottomrule\noalign{}
\endlastfoot
\texttt{:\%s/buscar/reemplazar/g} & Presionar
\texttt{:\%s/buscar/reemplazar/g} te permite reemplazar todas las
apariciones de ``buscar'' por ``reemplazar'' en todo el archivo. \\
\texttt{:s/buscar/reemplazar/g} & Presionar
\texttt{:s/buscar/reemplazar/g} te permite reemplazar solo la primera
aparición de ``buscar'' por ``reemplazar'' en la línea actual. \\
\end{longtable}

\hypertarget{reemplazo-de-caracteres}{%
\subsection{Reemplazo de caracteres}\label{reemplazo-de-caracteres}}

\begin{longtable}[]{@{}
  >{\raggedright\arraybackslash}p{(\columnwidth - 2\tabcolsep) * \real{0.1902}}
  >{\raggedright\arraybackslash}p{(\columnwidth - 2\tabcolsep) * \real{0.8098}}@{}}
\toprule\noalign{}
\begin{minipage}[b]{\linewidth}\raggedright
Shortcuts or commands o fórmula
\end{minipage} & \begin{minipage}[b]{\linewidth}\raggedright
Función
\end{minipage} \\
\midrule\noalign{}
\endhead
\bottomrule\noalign{}
\endlastfoot
\texttt{r} & Permite reemplazar un carácter cuando estás en modo
normal. \\
\texttt{R} & Activa el modo de reemplazo para cambiar los caracteres
deseados. No saldrás del modo de reemplazo hasta que presiones la tecla
ESC. \\
\texttt{ró} & Reemplaza la letra señalada por el cursor por ``ó''. \\
\texttt{10ró} & Reemplaza las 10 letras a partir de la posición actual
del cursor por ``ó''. \\
\end{longtable}

\hypertarget{buxfasqueda-y-reemplazo-de-palabras}{%
\subsection{Búsqueda y reemplazo de
palabras}\label{buxfasqueda-y-reemplazo-de-palabras}}

\begin{longtable}[]{@{}
  >{\raggedright\arraybackslash}p{(\columnwidth - 2\tabcolsep) * \real{0.1739}}
  >{\raggedright\arraybackslash}p{(\columnwidth - 2\tabcolsep) * \real{0.8261}}@{}}
\toprule\noalign{}
\begin{minipage}[b]{\linewidth}\raggedright
Shortcuts or commands o fórmula
\end{minipage} & \begin{minipage}[b]{\linewidth}\raggedright
Función
\end{minipage} \\
\midrule\noalign{}
\endhead
\bottomrule\noalign{}
\endlastfoot
\texttt{:s/achalma/Achalma} & Reemplaza todas las palabras que contienen
``achalma'' por ``Achalma'' solo en la línea actual. \\
\texttt{:s/achalma/Achalma/g} & Reemplaza todas las palabras que
contienen ``achalma'' por ``Achalma'' en la línea del cursor sin pedir
confirmación. \\
\texttt{:\%s/achalma/Achalma/g} & Reemplaza todas las palabras que
contienen ``achalma'' por ``Achalma'' en todo el documento sin pedir
confirmación. \\
\texttt{:\%s/achalma/Achalma/gc} & Reemplaza todas las ocurrencias de
``achalma'' por ``Achalma'' en el archivo pidiendo confirmación. \\
\texttt{:\%s!Achalma!ubuntu/scripts!gi} & Reemplaza todas las palabras
que contienen ``Achalma'' por ``ubuntu/scripts'' utilizando el
delimitador ``!'' y sin distinguir entre mayúsculas y minúsculas. \\
\texttt{:\%s/\textbackslash{}\textless{}geek\textbackslash{}\textgreater{}/Linux/gc}
& Reemplaza la palabra exacta ``geek'' por ``Linux'' en todo el
documento pidiendo confirmación. \\
\end{longtable}

\hypertarget{buxfasqueda-y-reemplazo-en-rangos-especuxedficos}{%
\subsection{Búsqueda y reemplazo en rangos
específicos}\label{buxfasqueda-y-reemplazo-en-rangos-especuxedficos}}

\begin{longtable}[]{@{}
  >{\raggedright\arraybackslash}p{(\columnwidth - 2\tabcolsep) * \real{0.1925}}
  >{\raggedright\arraybackslash}p{(\columnwidth - 2\tabcolsep) * \real{0.8075}}@{}}
\toprule\noalign{}
\begin{minipage}[b]{\linewidth}\raggedright
Shortcuts or commands o fórmula
\end{minipage} & \begin{minipage}[b]{\linewidth}\raggedright
Función
\end{minipage} \\
\midrule\noalign{}
\endhead
\bottomrule\noalign{}
\endlastfoot
\texttt{:5,12s/foo/bar/g} & Cambia ``foo'' por ``bar'' entre las líneas
5 y 12 (ambas incluidas). \\
\texttt{:\textquotesingle{}a,\textquotesingle{}bs/foo/bar/g} & Cambia
``foo'' por ``bar'' entre las marcas ``a'' y ``b''. \\
\texttt{:22s/Linux/Achalma/I} & Reemplaza todas las palabras que
contienen ``Linux'' por ``Achalma'' en la línea 22, distinguiendo entre
mayúsculas y minúsculas. \\
\texttt{:.s/Achalma/Linux/I} & Reemplaza todas las palabras que
contienen ``Achalma'' por ``Linux'' en la línea actual, distinguiendo
entre mayúsculas y minúsculas. \\
\end{longtable}

\hypertarget{transformaciuxf3n-de-texto}{%
\subsection{Transformación de texto}\label{transformaciuxf3n-de-texto}}

\begin{longtable}[]{@{}ll@{}}
\toprule\noalign{}
Shortcuts or commands & Función \\
\midrule\noalign{}
\endhead
\bottomrule\noalign{}
\endlastfoot
\texttt{g+u+u} & Transforma una frase completa a minúsculas. \\
\texttt{g+U+U} & Transforma una frase completa a mayúsculas. \\
\end{longtable}

\hypertarget{especificando-rangos}{%
\section{Especificando rangos}\label{especificando-rangos}}

\hypertarget{suxedmbolos-para-especificar-rangos}{%
\subsection{Símbolos para especificar
rangos}\label{suxedmbolos-para-especificar-rangos}}

\begin{longtable}[]{@{}
  >{\raggedright\arraybackslash}p{(\columnwidth - 2\tabcolsep) * \real{0.1061}}
  >{\raggedright\arraybackslash}p{(\columnwidth - 2\tabcolsep) * \real{0.8939}}@{}}
\toprule\noalign{}
\begin{minipage}[b]{\linewidth}\raggedright
Símbolo
\end{minipage} & \begin{minipage}[b]{\linewidth}\raggedright
Significado
\end{minipage} \\
\midrule\noalign{}
\endhead
\bottomrule\noalign{}
\endlastfoot
\texttt{\%} & Aplica la acción a todo el documento \\
\texttt{\$} & Aplica la acción a la última línea del documento \\
\texttt{.} & Representa la línea actual en la que se encuentra el
cursor \\
\end{longtable}

\hypertarget{especificaciuxf3n-de-rangos-mediante-marcas-y-nuxfameros-de-luxednea}{%
\subsection{Especificación de rangos mediante marcas y números de
línea}\label{especificaciuxf3n-de-rangos-mediante-marcas-y-nuxfameros-de-luxednea}}

\begin{longtable}[]{@{}
  >{\raggedright\arraybackslash}p{(\columnwidth - 2\tabcolsep) * \real{0.1507}}
  >{\raggedright\arraybackslash}p{(\columnwidth - 2\tabcolsep) * \real{0.8493}}@{}}
\toprule\noalign{}
\begin{minipage}[b]{\linewidth}\raggedright
Rango
\end{minipage} & \begin{minipage}[b]{\linewidth}\raggedright
Significado
\end{minipage} \\
\midrule\noalign{}
\endhead
\bottomrule\noalign{}
\endlastfoot
\texttt{\textquotesingle{}a,\textquotesingle{}b} & Bloque de líneas
entre las marcas `a' y `b' \\
\texttt{:17,20d} & Borra las líneas 17, 18, 19 y 20 \\
\texttt{:.-2,.+8y} & Copia el contenido desde 2 líneas encima hasta 8
líneas debajo \\
\end{longtable}

\hypertarget{juntar-2-luxedneas}{%
\section{Juntar 2 líneas}\label{juntar-2-luxedneas}}

\begin{longtable}[]{@{}
  >{\raggedright\arraybackslash}p{(\columnwidth - 2\tabcolsep) * \real{0.2952}}
  >{\raggedright\arraybackslash}p{(\columnwidth - 2\tabcolsep) * \real{0.7048}}@{}}
\toprule\noalign{}
\begin{minipage}[b]{\linewidth}\raggedright
Shortcuts or commands o fórmula
\end{minipage} & \begin{minipage}[b]{\linewidth}\raggedright
Función
\end{minipage} \\
\midrule\noalign{}
\endhead
\bottomrule\noalign{}
\endlastfoot
\texttt{Shift\ +\ j} & Une 2 líneas \\
\texttt{gJ} & Une la línea debajo del cursor con la actual sin dejar
espacio entre ellas \\
\texttt{:join} & Fórmula a utilizar si se desean unir 2 líneas \\
\end{longtable}

\hypertarget{atajos-de-teclado-para-realizar-pliegues-manuales}{%
\section{Atajos de teclado para realizar pliegues
manuales}\label{atajos-de-teclado-para-realizar-pliegues-manuales}}

\begin{longtable}[]{@{}
  >{\raggedright\arraybackslash}p{(\columnwidth - 2\tabcolsep) * \real{0.1789}}
  >{\raggedright\arraybackslash}p{(\columnwidth - 2\tabcolsep) * \real{0.8211}}@{}}
\toprule\noalign{}
\begin{minipage}[b]{\linewidth}\raggedright
Atajos de teclado
\end{minipage} & \begin{minipage}[b]{\linewidth}\raggedright
¿Qué hacen?
\end{minipage} \\
\midrule\noalign{}
\endhead
\bottomrule\noalign{}
\endlastfoot
\texttt{zf\ \textless{}movimiento\textgreater{}} & Con este atajo puedes
definir un pliegue en Vim. \\
\texttt{zf4j} & Puedes plegar las 4 líneas que se encuentran debajo del
cursor. \\
\texttt{zf2\}} & Este atajo pliega los 2 párrafos que están justo debajo
del cursor. \\
\texttt{zf2\{} & Con este atajo puedes plegar los 2 párrafos que están
justo encima del cursor. \\
\texttt{zo} & Abre el pliegue en el que está posicionado el cursor. \\
\texttt{zR} & Despliega todos los pliegues en el documento. \\
\texttt{zd} & Elimina el pliegue que se encuentra encima del cursor. \\
\texttt{zE} & Elimina todos los pliegues en el documento. \\
\texttt{zc} & Cierra el pliegue que se encuentra encima del cursor. \\
\texttt{zM} & Cierra todos los pliegues existentes en el documento. \\
\texttt{za} & Abre o cierra el pliegue en el que se encuentra el
cursor. \\
\texttt{zj} & Desplaza el cursor al siguiente pliegue. \\
\texttt{zk} & Desplaza el cursor al pliegue anterior. \\
\end{longtable}

\textbf{Nota:} Para realizar pliegues de forma manual, debes agregar la
siguiente línea al archivo de configuración
\texttt{\textasciitilde{}/.vimrc}: \texttt{set\ foldmethod=manual}.

\hypertarget{registros-o-portapapeles}{%
\section{Registros o portapapeles}\label{registros-o-portapapeles}}

\hypertarget{mostrar-y-gestionar-los-registros}{%
\subsection{Mostrar y gestionar los
registros}\label{mostrar-y-gestionar-los-registros}}

\begin{longtable}[]{@{}
  >{\raggedright\arraybackslash}p{(\columnwidth - 2\tabcolsep) * \real{0.1208}}
  >{\raggedright\arraybackslash}p{(\columnwidth - 2\tabcolsep) * \real{0.8792}}@{}}
\toprule\noalign{}
\begin{minipage}[b]{\linewidth}\raggedright
Shortcuts or commands
\end{minipage} & \begin{minipage}[b]{\linewidth}\raggedright
Función
\end{minipage} \\
\midrule\noalign{}
\endhead
\bottomrule\noalign{}
\endlastfoot
\texttt{:registers} o \texttt{:reg} & Muestra el contenido almacenado en
cada uno de los 48 registros o portapapeles de Vim. \\
\texttt{"1yy} & Copia la línea actual al registro numérico 1. \\
\texttt{"1p} & Pega el contenido del registro numérico 1. \\
\texttt{"ayy} & Copia la línea actual al registro nominal a. Si utilizas
``a'' en minúscula, se borrará completamente el contenido anterior del
portapapeles a. \\
\texttt{"Ayy} & Añade la línea actual al registro nominal a. Si utilizas
``A'' en mayúscula, el contenido previamente guardado en a no se
borrará, sino que se añadirá. \\
\texttt{"ap} & Pega el contenido del registro nominal a. \\
\texttt{:registers\ a} o \texttt{:reg\ a} & Muestra el contenido
guardado en el registro a. \\
\texttt{:reg\ 4c} & Muestra el contenido almacenado en el registro
numérico 4 y el registro nominal c. \\
\texttt{"+yy} & Copia la línea actual al portapapeles del sistema
operativo. \\
\texttt{"+p} & Pega el contenido almacenado en el portapapeles del
sistema operativo en el documento que estás editando en Vim. \\
\texttt{"*p} & Pega el texto seleccionado en tu navegador u otro
software de tu sistema operativo. No es necesario que el texto esté
copiado en el portapapeles del sistema. \\
\texttt{Ctrl+r+a} & Pega el contenido del registro a mientras estás en
modo de inserción de texto. \\
\texttt{Ctrl+r\ "} & Pega el contenido del registro predeterminado del
portapapeles de Vim en modo de inserción. \\
\texttt{Ctrl+r\ +} & Pega el contenido almacenado en el portapapeles del
sistema operativo en modo de inserción. \\
\texttt{Ctrl+r\ *} & Pega el texto seleccionado en cualquier aplicación
de tu sistema operativo en el modo de inserción de texto. No es
necesario que el texto esté copiado en el portapapeles del sistema. \\
\end{longtable}

\textbf{Nota:} Hay 26 registros nominales, desde ``a'' hasta ``z''.

\hypertarget{registros-numuxe9ricos}{%
\subsection{Registros numéricos}\label{registros-numuxe9ricos}}

\begin{longtable}[]{@{}
  >{\raggedright\arraybackslash}p{(\columnwidth - 2\tabcolsep) * \real{0.3182}}
  >{\raggedright\arraybackslash}p{(\columnwidth - 2\tabcolsep) * \real{0.6818}}@{}}
\toprule\noalign{}
\begin{minipage}[b]{\linewidth}\raggedright
Shortcuts or commands
\end{minipage} & \begin{minipage}[b]{\linewidth}\raggedright
¿Qué hacen?
\end{minipage} \\
\midrule\noalign{}
\endhead
\bottomrule\noalign{}
\endlastfoot
\texttt{"0yy} & Copia la línea actual al registro numérico 0. \\
\texttt{"0p} & Pega el contenido del registro numérico 0. \\
\texttt{"2yy} & Copia la línea actual al registro numérico 2. \\
\texttt{"2p} & Pega el contenido del registro numérico 2. \\
\ldots{} & \ldots{} \\
\texttt{"9yy} & Copia la línea actual al registro numérico 9. \\
\texttt{"9p} & Pega el contenido del registro numérico 9. \\
\end{longtable}

\hypertarget{registros-de-lectura}{%
\subsection{Registros de lectura}\label{registros-de-lectura}}

\begin{longtable}[]{@{}ll@{}}
\toprule\noalign{}
Registro & Contenido almacenado \\
\midrule\noalign{}
\endhead
\bottomrule\noalign{}
\endlastfoot
\texttt{":} & Guarda el último comando ejecutado. \\
\texttt{"\%"} & Contiene la ruta del archivo de Vim que estás
editando. \\
\texttt{"."} & Contiene el último texto insertado. \\
\end{longtable}

Estos registros de lectura te permiten acceder a información relevante,
como el último comando ejecutado, la ruta del archivo en edición y el
último texto insertado. Puedes utilizar estos registros en tus flujos de
trabajo de Vim para mejorar tu productividad.

\hypertarget{registros-especiales}{%
\subsection{Registros especiales}\label{registros-especiales}}

\begin{longtable}[]{@{}
  >{\raggedright\arraybackslash}p{(\columnwidth - 2\tabcolsep) * \real{0.0777}}
  >{\raggedright\arraybackslash}p{(\columnwidth - 2\tabcolsep) * \real{0.9223}}@{}}
\toprule\noalign{}
\begin{minipage}[b]{\linewidth}\raggedright
Registro
\end{minipage} & \begin{minipage}[b]{\linewidth}\raggedright
Contenido almacenado
\end{minipage} \\
\midrule\noalign{}
\endhead
\bottomrule\noalign{}
\endlastfoot
\texttt{"\_"} & Es el registro del ``agujero negro''. Cuando eliminamos
una palabra usando el atajo ``\_diw'', \\
& la palabra borrada se almacenará en este registro y el registro por
defecto \texttt{""} quedará vacío. \\
\texttt{":"} & Almacena el último comando ejecutado. \\
\texttt{"-"} & Contiene el contenido que hemos borrado o modificado y no
es mayor a una línea de longitud. \\
\texttt{"\textbackslash{}} & Guarda el contenido de la última
búsqueda. \\
\texttt{"=} & Permite realizar operaciones matemáticas en Vim. Por
ejemplo, puedes hacer una suma como \texttt{=2+2} \\
& en modo normal y luego pegar el resultado con la tecla \texttt{p}. \\
\end{longtable}

Estos registros especiales en Vim ofrecen funcionalidades adicionales,
como el registro del ``agujero negro'' para evitar que palabras borradas
se almacenen en el registro por defecto, el almacenamiento del último
comando ejecutado y el contenido borrado o modificado. También puedes
realizar operaciones matemáticas y acceder al contenido de la última
búsqueda.

Recuerda que todos los registros de Vim se almacenan permanentemente en
el archivo \texttt{\textasciitilde{}/.viminfo}, lo que significa que no
se borran al cerrar Vim. Esto te permite mantener tus registros
guardados y acceder a ellos en sesiones posteriores.

\hypertarget{cuxf3mo-usar-la-terminal-de-linux-dentro-de-vim}{%
\section{¿Cómo usar la terminal de Linux dentro de
Vim?}\label{cuxf3mo-usar-la-terminal-de-linux-dentro-de-vim}}

\begin{longtable}[]{@{}
  >{\raggedright\arraybackslash}p{(\columnwidth - 2\tabcolsep) * \real{0.1694}}
  >{\raggedright\arraybackslash}p{(\columnwidth - 2\tabcolsep) * \real{0.8306}}@{}}
\toprule\noalign{}
\begin{minipage}[b]{\linewidth}\raggedright
Shortcuts or commands
\end{minipage} & \begin{minipage}[b]{\linewidth}\raggedright
Función
\end{minipage} \\
\midrule\noalign{}
\endhead
\bottomrule\noalign{}
\endlastfoot
\texttt{:term} & Divide la pantalla en dos. Una parte muestra Vim y la
otra muestra la terminal. \\
& Puedes alternar entre las ventanas usando el atajo \texttt{Ctrl+w}. \\
\texttt{:!ls} & Ejecuta el comando \texttt{ls} dentro de Vim. Puedes
usar otros comandos, como \texttt{sudo\ apt\ update}, en su lugar. \\
\texttt{:r\ !\textless{}comando\textgreater{}} & Inserta la salida del
comando ejecutado dentro del archivo de Vim que estás editando. \\
\end{longtable}

\hypertarget{encadenar-comandos-en-vim}{%
\section{Encadenar comandos en Vim**}\label{encadenar-comandos-en-vim}}

\begin{longtable}[]{@{}
  >{\raggedright\arraybackslash}p{(\columnwidth - 2\tabcolsep) * \real{0.2692}}
  >{\raggedright\arraybackslash}p{(\columnwidth - 2\tabcolsep) * \real{0.7308}}@{}}
\toprule\noalign{}
\begin{minipage}[b]{\linewidth}\raggedright
Shortcuts or commands
\end{minipage} & \begin{minipage}[b]{\linewidth}\raggedright
Función
\end{minipage} \\
\midrule\noalign{}
\endhead
\bottomrule\noalign{}
\endlastfoot
\texttt{:w\ \textbackslash{}\textbar{}\ q} & Guarda los cambios en el
documento y lo cierra. \\
\texttt{:17,20y\ \textbackslash{}\textbar{}\ 122put} & Copia las líneas
del 17 al 20 y las pega en la línea 122. \\
\end{longtable}

Usando el símbolo de tubería \texttt{\textbar{}}, puedes encadenar la
ejecución de comandos en Vim para realizar acciones de forma simultánea.
Estos ejemplos muestran cómo guardar los cambios en un documento y
cerrarlo en una sola línea de comando, así como copiar un rango de
líneas y pegarlo en una ubicación específica del documento.

Esta técnica te permite agilizar tu flujo de trabajo al evitar tener que
introducir comandos por separado. Puedes aprovechar la potencia de Vim
combinando múltiples acciones en una sola línea de comando utilizando la
tubería \texttt{\textbar{}}.

\hypertarget{trabajo-con-muxfaltiples-archivos-y-ventanas}{%
\section{Trabajo con múltiples archivos y
ventanas}\label{trabajo-con-muxfaltiples-archivos-y-ventanas}}

En Vim, tienes la capacidad de trabajar con múltiples archivos y
ventanas para organizar tu flujo de trabajo de manera eficiente. A
continuación, se presentan las diferentes acciones relacionadas con la
apertura, cierre y navegación entre archivos y ventanas, junto con sus
respectivos Combinación de teclas o comandos:

\hypertarget{apertura-de-archivos}{%
\subsection{Apertura de archivos}\label{apertura-de-archivos}}

\begin{longtable}[]{@{}
  >{\raggedright\arraybackslash}p{(\columnwidth - 2\tabcolsep) * \real{0.1950}}
  >{\raggedright\arraybackslash}p{(\columnwidth - 2\tabcolsep) * \real{0.8050}}@{}}
\toprule\noalign{}
\begin{minipage}[b]{\linewidth}\raggedright
Shortcuts or commands
\end{minipage} & \begin{minipage}[b]{\linewidth}\raggedright
Función
\end{minipage} \\
\midrule\noalign{}
\endhead
\bottomrule\noalign{}
\endlastfoot
\texttt{:e\ nombre\_archivo} & Presionar \texttt{:e} seguido del nombre
del archivo te permite abrir ese archivo en una nueva pestaña. \\
\texttt{:badd\ +\ nombre\_fichero/buffer} & Añadir un buffer sin que se
muestre en pantalla. \\
\texttt{:ls} & Lista y enumera todos los buffers/archivos que tenemos
abiertos. \\
\texttt{:bn} & Ir al siguiente archivo/buffer que tenemos abierto
(Buffer\_next). \\
\texttt{:bp} & Ir al archivo/buffer anterior que tenemos abierto
(Buffer\_previous). \\
\texttt{:bl} & Mostrar en pantalla el último buffer abierto. \\
\texttt{:bf} & Mostrar en pantalla el primer buffer abierto. \\
\texttt{:b\_número\_buffer} & Cambiar al buffer especificado por su
número. Por ejemplo, \texttt{:b2} para cambiar al buffer número 2. \\
\texttt{:bd} & Cerrar únicamente el archivo/buffer actual que estamos
editando. \\
\texttt{:bd2} & Cerrar el buffer número 2. \\
\texttt{:bd\ nombre\_archivo} & Cerrar el buffer del archivo específico.
Por ejemplo, \texttt{:bd\ nombre\_archivo.py} cerrará el buffer del
archivo ``nombre\_archivo.py''. \\
\texttt{:ba} & Mostrar todos los buffers en pantalla. \\
\end{longtable}

\hypertarget{cierre-de-archivos}{%
\subsection{Cierre de archivos}\label{cierre-de-archivos}}

Llega el momento de decir adiós a Vim y cerrar el editor de texto. Pero,
¿cómo puedes hacerlo de manera rápida y sencilla? Vim también tiene
atajos de teclado específicos para salir del programa. Echa un vistazo a
la siguiente tabla con los atajos de teclado y sus funciones
correspondientes:

\begin{longtable}[]{@{}
  >{\raggedright\arraybackslash}p{(\columnwidth - 2\tabcolsep) * \real{0.1071}}
  >{\raggedright\arraybackslash}p{(\columnwidth - 2\tabcolsep) * \real{0.8929}}@{}}
\toprule\noalign{}
\begin{minipage}[b]{\linewidth}\raggedright
Shortcuts or commands
\end{minipage} & \begin{minipage}[b]{\linewidth}\raggedright
Función
\end{minipage} \\
\midrule\noalign{}
\endhead
\bottomrule\noalign{}
\endlastfoot
\texttt{:q} & Presionar \texttt{:q} te permite salir de Vim si no hay
cambios sin guardar en el archivo actual. \\
\texttt{:q!} & Si realizaste cambios en el archivo y deseas salir de Vim
sin guardarlos, puedes utilizar \texttt{:q!}. Esto te permite cerrar Vim
sin guardar los cambios. \\
\texttt{:wq} o \texttt{:x} & Si deseas guardar los cambios y salir de
Vim al mismo tiempo, puedes utilizar \texttt{:wq} o \texttt{:x}. Esto
guarda los cambios y cierra Vim en un solo paso. \\
\texttt{:wqa!} & El comando \texttt{:wqa!} te permite guardar todos los
archivos abiertos en Vim y salir, incluso si hay cambios sin guardar. \\
\texttt{:qa!} & Si deseas cerrar Vim sin guardar los cambios en ninguno
de los archivos abiertos, puedes utilizar \texttt{:qa!}. Esto cierra Vim
de inmediato y descarta todos los cambios realizados. \\
\end{longtable}

\hypertarget{trabajar-con-pestauxf1as}{%
\subsection{Trabajar con pestañas}\label{trabajar-con-pestauxf1as}}

Las pestañas en Vim son una forma práctica de organizar y alternar entre
múltiples archivos abiertos. Con las siguientes combinaciones de teclas,
podrás trabajar con pestañas de manera eficiente:

\begin{longtable}[]{@{}
  >{\raggedright\arraybackslash}p{(\columnwidth - 2\tabcolsep) * \real{0.2199}}
  >{\raggedright\arraybackslash}p{(\columnwidth - 2\tabcolsep) * \real{0.7801}}@{}}
\toprule\noalign{}
\begin{minipage}[b]{\linewidth}\raggedright
Shortcuts or commands
\end{minipage} & \begin{minipage}[b]{\linewidth}\raggedright
Función
\end{minipage} \\
\midrule\noalign{}
\endhead
\bottomrule\noalign{}
\endlastfoot
\texttt{:tabnew\ nombre\_archivo} & Abrir el archivo en una nueva
pestaña. \\
\texttt{:tabc} & Cerrar la pestaña actual. \\
\texttt{:tabo} & Cerrar todas las pestañas, excepto la actual. \\
\texttt{gt} o \texttt{:tabnext} & Cambiar a la siguiente pestaña. \\
\texttt{gT} o \texttt{:tabprev} & Cambiar a la pestaña anterior. \\
\texttt{Número\ gt} & Cambiar a la pestaña con el número
especificado. \\
\texttt{:tabs} & Listar y enumerar todas las pestañas abiertas. \\
\texttt{:tabmove\ N} & Mover la pestaña actual a la posición N
(1-indexed). \\
\texttt{:tabedit\ nombre\_fichero} & Abrir un nuevo archivo en una
pestaña nueva. Por ejemplo, \texttt{:tabedit\ archivo1.txt} abrirá
archivo1.txt en una nueva pestaña. \\
\texttt{:tabfind\ nombre\_fichero} & Buscar y abrir un archivo en una
pestaña nueva. Por ejemplo, \texttt{:tabfind\ archivo2.txt} buscará y
abrirá archivo2.txt en una nueva pestaña. \\
\texttt{:tab\ ball} & Abrir todos los buffers en pestañas separadas. \\
\texttt{:tab\ split} & Dividir la ventana actual en dos pestañas. \\
\texttt{:tabfirst} & Cambiar a la primera pestaña. \\
\texttt{:tablast} & Cambiar a la última pestaña. \\
\texttt{:tabm\ 3} & Mover la pestaña actual a una posición específica.
Por ejemplo, \texttt{:tabm\ 3} moverá la pestaña actual a la posición
3. \\
\texttt{:tabm\ 0} & Mover la pestaña actual a la última posición. \\
\texttt{:tabm} & Mover la pestaña actual a la siguiente posición. \\
\texttt{:tabclose!} o \texttt{:q!} & Cerrar la pestaña actual sin
guardar cambios. \\
\texttt{:tabclose} o \texttt{:q} & Cerrar la pestaña actual con
confirmación. \\
\texttt{:tabonly} & Cerrar todas las pestañas excepto la actual. \\
\texttt{vim\ -p\ nombre\_archivo\_1\ nombre\_archivo\_2} & Abrir
múltiples archivos en pestañas separadas. Por ejemplo,
\texttt{vim\ -p\ archivo1.sh\ archivo2.sh} abrirá archivo1.sh y
archivo2.sh en pestañas separadas. \\
\end{longtable}

\hypertarget{divisiuxf3n-de-la-ventana-en-paneles-y-cambio-entre-ventanas-y-paneles}{%
\subsection{División de la ventana en paneles y cambio entre ventanas y
paneles}\label{divisiuxf3n-de-la-ventana-en-paneles-y-cambio-entre-ventanas-y-paneles}}

\begin{longtable}[]{@{}
  >{\raggedright\arraybackslash}p{(\columnwidth - 2\tabcolsep) * \real{0.1683}}
  >{\raggedright\arraybackslash}p{(\columnwidth - 2\tabcolsep) * \real{0.8317}}@{}}
\toprule\noalign{}
\begin{minipage}[b]{\linewidth}\raggedright
Shortcuts or commands
\end{minipage} & \begin{minipage}[b]{\linewidth}\raggedright
Función
\end{minipage} \\
\midrule\noalign{}
\endhead
\bottomrule\noalign{}
\endlastfoot
\texttt{:sp} & Presionar \texttt{:sp} te permite dividir la ventana
horizontalmente, creando un nuevo panel en la parte superior. \\
\texttt{:vsp} & Presionar \texttt{:vsp} te permite dividir la ventana
verticalmente, creando un nuevo panel a la derecha. \\
\texttt{Ctrl+wv} & Presionar \texttt{Ctrl+w} seguido de \texttt{v}
dividirá la ventana actual en dos verticales. \\
\texttt{Ctrl+ws} & Presionar \texttt{Ctrl+w} seguido de \texttt{s}
dividirá la ventana actual en dos horizontales. \\
\texttt{Ctrl+w+(h/j/k/l)\ o\ flecha} & Cambiar a la ventana vecina en
una dirección específica. \texttt{Ctrl+w} seguido de una de las teclas
\texttt{h}, \texttt{j}, \texttt{k} o \texttt{l} cambiará a la ventana
vecina en la dirección indicada. \\
\texttt{Ctrl+ww} & Presionar \texttt{Ctrl+w} dos veces cambiará a la
siguiente ventana. \\
\texttt{Ctrl+w} & Presionar \texttt{Ctrl+w} cambiará a la ventana
anteriormente seleccionada. \\
\texttt{Ctrl+wq} & Presionar \texttt{Ctrl+w} seguido de \texttt{q}
cerrará la ventana actual. \\
\texttt{Ctrl+wx} & Presionar \texttt{Ctrl+w} seguido de \texttt{x}
intercambiará la posición de las ventanas. \\
\texttt{Ctrl+wr} & Presionar \texttt{Ctrl+w} seguido de \texttt{r}
rotará las ventanas hacia la derecha. \\
\texttt{Ctrl+wR} & Presionar \texttt{Ctrl+w} seguido de \texttt{R}
rotará las ventanas hacia la izquierda. \\
\texttt{:split\ nombre\_archivo} & Dividir la ventana actual
horizontalmente con un archivo. Por ejemplo,
\texttt{:split\ archivo2.txt} dividirá la ventana actual horizontalmente
y abrirá \texttt{archivo2.txt}. \\
\texttt{:resize\ 40} & Cambiar el tamaño de la ventana actual a 40
líneas. \\
\texttt{:vertical\ resize\ 80} & Cambiar el ancho de la ventana actual a
80 columnas. \\
\texttt{:vertical\ resize\ -5} & Reducir el ancho de la ventana actual
en 5 columnas. \\
\texttt{:vertical\ resize\ +5} & Aumentar el ancho de la ventana actual
en 5 columnas. \\
\texttt{Ctrl+w\ \textgreater{}} y \texttt{Ctrl+w\ \textless{}} &
Presionar \texttt{Ctrl+w} seguido de \texttt{\textgreater{}} o
\texttt{\textless{}} ajustará el tamaño de la ventana hacia la derecha o
izquierda, respectivamente. \\
\texttt{Ctrl+w\ +} y \texttt{Ctrl+w\ -} & Presionar \texttt{Ctrl+w}
seguido de \texttt{+} o \texttt{-} aumentará o reducirá el tamaño de la
ventana verticalmente, respectivamente. \\
\texttt{Ctrl+w\ =} & Presionar \texttt{Ctrl+w} seguido de \texttt{=}
igualará el tamaño de todas las ventanas. \\
\texttt{vim\ -o\ archivo1\ archivo2\ archivo3} & Abrir múltiples
archivos en ventanas separadas.
\texttt{vim\ -o\ archivo1.txt\ archivo2.txt\ archivo3.txt} abrirá
\texttt{archivo1.txt}, \texttt{archivo2.txt} y \texttt{archivo3.txt} en
ventanas separadas. \\
\end{longtable}

\hypertarget{personalizaciuxf3n-y-configuraciuxf3n-de-vim}{%
\section{Personalización y configuración de
Vim}\label{personalizaciuxf3n-y-configuraciuxf3n-de-vim}}

Una de las ventajas más destacadas de Vim es su alta capacidad de
personalización y configuración. Puedes adaptar el editor según tus
preferencias y necesidades. A continuación, te presento las diferentes
opciones para personalizar y configurar Vim, junto con sus respectivas
características:

\hypertarget{personalizaciuxf3n-del-archivo-.vimrc}{%
\subsection{Personalización del archivo
.vimrc}\label{personalizaciuxf3n-del-archivo-.vimrc}}

El archivo \texttt{.vimrc} es donde puedes definir tus preferencias de
Vim. Puedes personalizar aspectos como el color del fondo, el tamaño de
la fuente, las opciones de resaltado de sintaxis y mucho más. Edita este
archivo para que Vim se ajuste a tu estilo y preferencias personales.
¡Haz de Vim tu editor único!

\hypertarget{uso-de-plugins-y-extensiones}{%
\subsection{Uso de plugins y
extensiones}\label{uso-de-plugins-y-extensiones}}

Vim cuenta con una amplia gama de plugins y extensiones que te permiten
ampliar sus funcionalidades. Puedes agregar complementos para la
administración de proyectos, resaltado de sintaxis avanzado,
autocompletado y mucho más. Explora la comunidad de plugins de Vim y
encuentra las herramientas que se adapten mejor a tus necesidades.
¡Potencia tus capacidades editoriales!

\hypertarget{configuraciuxf3n-de-atajos-de-teclado-personalizados}{%
\subsection{Configuración de atajos de teclado
personalizados}\label{configuraciuxf3n-de-atajos-de-teclado-personalizados}}

Vim te permite personalizar los atajos de teclado según tus
preferencias. Puedes asignar tus propias combinaciones de teclas para
realizar acciones específicas, como guardar un archivo, buscar y
reemplazar, o incluso para ejecutar comandos personalizados. Configura
tus atajos de teclado para trabajar de manera más eficiente y adaptada a
tu flujo de trabajo. ¡Crea tu propio mapa de atajos!

\hypertarget{consejos-y-trucos-avanzados}{%
\section{Consejos y trucos
avanzados}\label{consejos-y-trucos-avanzados}}

En esta sección, exploraremos algunos consejos y trucos avanzados para
aprovechar al máximo Vim. Estos recursos te permitirán mejorar tu flujo
de trabajo y aumentar tu productividad. ¡Descubre cómo darle un impulso
a tu experiencia con Vim!

\hypertarget{uso-de-marcadores}{%
\subsection{Uso de marcadores}\label{uso-de-marcadores}}

Los marcadores son una herramienta útil para marcar y acceder
rápidamente a ubicaciones específicas en un archivo. Puedes colocar
marcadores en líneas específicas y luego saltar entre ellos con
facilidad. Usa \texttt{ma} para establecer un marcador en la posición
actual y \texttt{m{[}a-z{]}} para asignar un marcador a una letra
específica. Luego, utiliza \texttt{\textquotesingle{}a} para saltar al
marcador. ¡No pierdas nunca la pista de tus puntos clave!

\hypertarget{autocompletado-y-omnicomplete}{%
\subsection{Autocompletado y
omnicomplete}\label{autocompletado-y-omnicomplete}}

Vim ofrece características avanzadas de autocompletado que te ayudarán a
escribir código más rápido y eficientemente. Utiliza
\texttt{\textless{}Ctrl\ +\ n\textgreater{}} y
\texttt{\textless{}Ctrl\ +\ p\textgreater{}} para navegar por las
opciones de autocompletado mientras escribes. Además, con la función de
omnicomplete, puedes obtener sugerencias contextuales y completar
palabras automáticamente presionando
\texttt{\textless{}Ctrl\ +\ x\textgreater{}} seguido de
\texttt{\textless{}Ctrl\ +\ o\textgreater{}}. ¡Escribe código de forma
más rápida y precisa!

\hypertarget{trabajo-con-macros}{%
\subsection{Trabajo con macros}\label{trabajo-con-macros}}

Las macros te permiten grabar y reproducir una serie de comandos en Vim.
Esto es útil cuando tienes que realizar acciones repetitivas en varios
lugares del archivo. Graba una macro usando \texttt{q{[}a-z{]}} seguido
de los comandos que deseas grabar y luego reproduce la macro usando
\texttt{@{[}a-z{]}}. ¡Automatiza tareas tediosas y ahorra tiempo!

\hypertarget{uso-de-scripts-y-automatizaciuxf3n}{%
\subsection{Uso de scripts y
automatización}\label{uso-de-scripts-y-automatizaciuxf3n}}

Vim ofrece la posibilidad de utilizar scripts para personalizar y
automatizar tareas. Puedes crear tus propios scripts en lenguajes como
VimScript o Python y ejecutarlos dentro de Vim. Esto te permite realizar
acciones complejas y personalizadas de manera más eficiente. Explora la
documentación de Vim para aprender cómo escribir y ejecutar scripts.
¡Haz que Vim se adapte aún más a tus necesidades!

\hypertarget{recursos-adicionales}{%
\section{Recursos adicionales}\label{recursos-adicionales}}

En esta sección, te proporcionaremos algunos recursos adicionales para
que puedas profundizar en tus conocimientos sobre Vim y aprovechar al
máximo este poderoso editor de texto.

\hypertarget{referencias-y-documentaciuxf3n-oficial-de-vim}{%
\subsection{Referencias y documentación oficial de
Vim}\label{referencias-y-documentaciuxf3n-oficial-de-vim}}

La documentación oficial de Vim es una fuente invaluable de información.
Puedes acceder a ella a través del comando \texttt{:help} dentro de Vim.
Este recurso abarca todos los aspectos del editor y proporciona
explicaciones detalladas de los comandos, configuraciones y
características. Consulta la documentación oficial para tener una
referencia confiable y completa sobre Vim.

\hypertarget{comunidades-y-foros-de-usuarios-de-vim}{%
\subsection{Comunidades y foros de usuarios de
Vim}\label{comunidades-y-foros-de-usuarios-de-vim}}

Existen diversas comunidades en línea donde puedes encontrar apoyo y
compartir conocimientos con otros usuarios de Vim. Reddit cuenta con un
subreddit dedicado a Vim (/r/vim), donde puedes hacer preguntas,
compartir consejos y estar al tanto de las últimas novedades
relacionadas con Vim. Además, hay foros como VimGolf y Vim Tips Wiki que
ofrecen desafíos y consejos prácticos para mejorar tus habilidades en
Vim. Únete a estas comunidades para aprender de otros entusiastas de
Vim.

\hypertarget{libros-y-tutoriales-recomendados-sobre-vim}{%
\subsection{Libros y tutoriales recomendados sobre
Vim}\label{libros-y-tutoriales-recomendados-sobre-vim}}

Si prefieres un enfoque más estructurado y detallado, hay varios libros
y tutoriales disponibles para aprender Vim. Algunos títulos populares
incluyen ``Aprendiendo Vim en 21 días'' de Tony Steidler-Dennison, ``Vim
Book'' de Steve Oualline y ``Practical Vim'' de Drew Neil. Estos
recursos te guiarán paso a paso a través de los conceptos y técnicas
fundamentales de Vim. Además, hay numerosos tutoriales en línea, videos
y cursos que puedes encontrar en plataformas educativas como Udemy y
YouTube.

Explora estos recursos adicionales para continuar mejorando tus
habilidades en Vim y descubrir nuevas técnicas y trucos. Recuerda que la
práctica regular y la experimentación son clave para dominar Vim.


\printbibliography


\end{document}
