% Options for packages loaded elsewhere
\PassOptionsToPackage{unicode}{hyperref}
\PassOptionsToPackage{hyphens}{url}
\PassOptionsToPackage{dvipsnames,svgnames,x11names}{xcolor}
%
\documentclass[
  a4paper,
]{article}

\usepackage{amsmath,amssymb}
\usepackage{iftex}
\ifPDFTeX
  \usepackage[T1]{fontenc}
  \usepackage[utf8]{inputenc}
  \usepackage{textcomp} % provide euro and other symbols
\else % if luatex or xetex
  \usepackage{unicode-math}
  \defaultfontfeatures{Scale=MatchLowercase}
  \defaultfontfeatures[\rmfamily]{Ligatures=TeX,Scale=1}
\fi
\usepackage{lmodern}
\ifPDFTeX\else  
    % xetex/luatex font selection
\fi
% Use upquote if available, for straight quotes in verbatim environments
\IfFileExists{upquote.sty}{\usepackage{upquote}}{}
\IfFileExists{microtype.sty}{% use microtype if available
  \usepackage[]{microtype}
  \UseMicrotypeSet[protrusion]{basicmath} % disable protrusion for tt fonts
}{}
\makeatletter
\@ifundefined{KOMAClassName}{% if non-KOMA class
  \IfFileExists{parskip.sty}{%
    \usepackage{parskip}
  }{% else
    \setlength{\parindent}{0pt}
    \setlength{\parskip}{6pt plus 2pt minus 1pt}}
}{% if KOMA class
  \KOMAoptions{parskip=half}}
\makeatother
\usepackage{xcolor}
\usepackage[top=2.54cm,right=2.54cm,bottom=2.54cm,left=2.54cm]{geometry}
\setlength{\emergencystretch}{3em} % prevent overfull lines
\setcounter{secnumdepth}{-\maxdimen} % remove section numbering
% Make \paragraph and \subparagraph free-standing
\ifx\paragraph\undefined\else
  \let\oldparagraph\paragraph
  \renewcommand{\paragraph}[1]{\oldparagraph{#1}\mbox{}}
\fi
\ifx\subparagraph\undefined\else
  \let\oldsubparagraph\subparagraph
  \renewcommand{\subparagraph}[1]{\oldsubparagraph{#1}\mbox{}}
\fi


\providecommand{\tightlist}{%
  \setlength{\itemsep}{0pt}\setlength{\parskip}{0pt}}\usepackage{longtable,booktabs,array}
\usepackage{calc} % for calculating minipage widths
% Correct order of tables after \paragraph or \subparagraph
\usepackage{etoolbox}
\makeatletter
\patchcmd\longtable{\par}{\if@noskipsec\mbox{}\fi\par}{}{}
\makeatother
% Allow footnotes in longtable head/foot
\IfFileExists{footnotehyper.sty}{\usepackage{footnotehyper}}{\usepackage{footnote}}
\makesavenoteenv{longtable}
\usepackage{graphicx}
\makeatletter
\def\maxwidth{\ifdim\Gin@nat@width>\linewidth\linewidth\else\Gin@nat@width\fi}
\def\maxheight{\ifdim\Gin@nat@height>\textheight\textheight\else\Gin@nat@height\fi}
\makeatother
% Scale images if necessary, so that they will not overflow the page
% margins by default, and it is still possible to overwrite the defaults
% using explicit options in \includegraphics[width, height, ...]{}
\setkeys{Gin}{width=\maxwidth,height=\maxheight,keepaspectratio}
% Set default figure placement to htbp
\makeatletter
\def\fps@figure{htbp}
\makeatother

\makeatletter
\makeatother
\makeatletter
\makeatother
\makeatletter
\@ifpackageloaded{caption}{}{\usepackage{caption}}
\AtBeginDocument{%
\ifdefined\contentsname
  \renewcommand*\contentsname{Tabla de contenidos}
\else
  \newcommand\contentsname{Tabla de contenidos}
\fi
\ifdefined\listfigurename
  \renewcommand*\listfigurename{Listado de Figuras}
\else
  \newcommand\listfigurename{Listado de Figuras}
\fi
\ifdefined\listtablename
  \renewcommand*\listtablename{Listado de Tablas}
\else
  \newcommand\listtablename{Listado de Tablas}
\fi
\ifdefined\figurename
  \renewcommand*\figurename{Figura}
\else
  \newcommand\figurename{Figura}
\fi
\ifdefined\tablename
  \renewcommand*\tablename{Tabla}
\else
  \newcommand\tablename{Tabla}
\fi
}
\@ifpackageloaded{float}{}{\usepackage{float}}
\floatstyle{ruled}
\@ifundefined{c@chapter}{\newfloat{codelisting}{h}{lop}}{\newfloat{codelisting}{h}{lop}[chapter]}
\floatname{codelisting}{Listado}
\newcommand*\listoflistings{\listof{codelisting}{Listado de Listados}}
\makeatother
\makeatletter
\@ifpackageloaded{caption}{}{\usepackage{caption}}
\@ifpackageloaded{subcaption}{}{\usepackage{subcaption}}
\makeatother
\makeatletter
\@ifpackageloaded{tcolorbox}{}{\usepackage[skins,breakable]{tcolorbox}}
\makeatother
\makeatletter
\@ifundefined{shadecolor}{\definecolor{shadecolor}{rgb}{.97, .97, .97}}
\makeatother
\makeatletter
\makeatother
\makeatletter
\makeatother
\ifLuaTeX
\usepackage[bidi=basic]{babel}
\else
\usepackage[bidi=default]{babel}
\fi
\babelprovide[main,import]{spanish}
% get rid of language-specific shorthands (see #6817):
\let\LanguageShortHands\languageshorthands
\def\languageshorthands#1{}
\ifLuaTeX
  \usepackage{selnolig}  % disable illegal ligatures
\fi
\usepackage[]{biblatex}
\addbibresource{../../../../references.bib}
\IfFileExists{bookmark.sty}{\usepackage{bookmark}}{\usepackage{hyperref}}
\IfFileExists{xurl.sty}{\usepackage{xurl}}{} % add URL line breaks if available
\urlstyle{same} % disable monospaced font for URLs
\hypersetup{
  pdftitle={Notas de Clase Series de Tiempo},
  pdfauthor={Edison Achalma},
  pdflang={es},
  colorlinks=true,
  linkcolor={blue},
  filecolor={Maroon},
  citecolor={Blue},
  urlcolor={Blue},
  pdfcreator={LaTeX via pandoc}}

\title{Notas de Clase Series de Tiempo}
\usepackage{etoolbox}
\makeatletter
\providecommand{\subtitle}[1]{% add subtitle to \maketitle
  \apptocmd{\@title}{\par {\large #1 \par}}{}{}
}
\makeatother
\subtitle{Descubre cómo seleccionar hardware, descargar la imagen ISO y
preparar los medios de instalación. Exploraremos opciones para probar o
instalar Linux en tu equipo.}
\author{Edison Achalma}
\date{2023-08-27}

\begin{document}
\maketitle
\ifdefined\Shaded\renewenvironment{Shaded}{\begin{tcolorbox}[sharp corners, breakable, enhanced, frame hidden, interior hidden, borderline west={3pt}{0pt}{shadecolor}, boxrule=0pt]}{\end{tcolorbox}}\fi

\hypertarget{otros-modelos-de-series-de-tiempo-no-lineales}{%
\section{Otros Modelos de Series de Tiempo No
lineales}\label{otros-modelos-de-series-de-tiempo-no-lineales}}

\hypertarget{modelos-de-cambio-de-ruxe9gimen}{%
\subsection{Modelos de cambio de
régimen}\label{modelos-de-cambio-de-ruxe9gimen}}

En años recientes, los modelos de serie de tiempo han sido incorporados
en análisis de la existencia de diferentes estados que son generados por
procesos estocásticos subyacentes. En esta sección del curso revisaremos
algunos modelos de cambio de regimen. Restringimos nuestra revisión a
modelos que asuman que la dinámica de las series puede ser descrito por
modelos del tipo AR y dejamos fuera procesos del tipo MA.

En general distinguimos que existen dos tipos de modelos:

\begin{enumerate}
    \item Modelos caracterizados por una variable observable, por lo que tenemos certeza de los régimanes en cada uno de los momentos.
    
    \item Modelos en los que el régimen no puede ser observado por una variable, pero si conocemos el proceso estocástico subyacente. 
\end{enumerate}

\hypertarget{reguxedmenes-determinados-por-informaciuxf3n-observable}{%
\subsubsection{Regímenes determinados por información
observable}\label{reguxedmenes-determinados-por-informaciuxf3n-observable}}

En estos casos asumimos que el régimen ocurren en un momento \(t\) y
puede ser determinado por una variable observable. Este modelo es
conocido como el modelo autoregresivo con umbral (TAR, Threshold
Autoregressive model). En este caso también diremos que cuando el
régimen está determinado por la información de la misma serie será
llamdao Self-Exciting TAR (SETAR).

Veámos un ejemplo. Supongamos que existe un umbral, \(c\), para el
régimen que esta determinado por \(q_t = y_{t-1}\) y que el estado de la
naturaleza nos permite establecer dos estados o regímenes: \[
    y_t = 
    \begin{cases}
        \phi_{01} + \phi_{11} y_{t-1} + \varepsilon_t \text{ si } y_{t-1} \leq c \\
        \phi_{02} + \phi_{12} y_{t-1} + \varepsilon_t \text{ si } y_{t-1} > c 
    \end{cases}
\]

Donde asumiremos que \(\varepsilon_t\) es i.i.d y que cumple con:
\begin{equation*}
    \mathbb{E}[\varepsilon_t | \Omega_{t-1}] = 0
\end{equation*}

Donde \(\Omega_{t-1} = \{ y_{t-1}, y_{t-2}, \ldots \}\). Existe una
variante de este modelo que suaviza la transición entre regímenes
conocido como Smooth Transition AR (STAR) y puede ser especificado en su
modalidad de dos regémenes como: \[
    y_t = (\phi_{01} + \phi_{11} y_{t-1}) (1 - G(y_{t-1}; \gamma, c)) + (\phi_{02} + \phi_{12} y_{t-1}) G(y_{t-1}; \gamma, c) + \varepsilon_t
\]

Donde \(G(y_{t-1}; \gamma, c)\) es una función de distribución de
probabilidad que suviza la transición entre regímenes. La práctica común
es suponer que está tiene una forma logística: \[
    G(y_{t-1}; \gamma, c) = \frac{1}{1 + e^{-\gamma (y_{t-1} - c)}}
\]

Es posible hacer estensiones de lo anterior a modelos de orden superior
dando como resultado: \[
    y_t = 
    \begin{cases}
        \phi_{01} + \phi_{11} y_{t-1} + \phi_{21} y_{t-2} + \ldots + \phi_{p_1 1} y_{t-p_1} + \varepsilon_t \text{ si } y_{t-1} \leq c \\
        \phi_{02} + \phi_{12} y_{t-1} + \phi_{22} y_{t-2} + \ldots + \phi_{p_2 2} y_{t-p_2} + \varepsilon_t \text{ si } y_{t-1} > c 
    \end{cases}
\]

En el segundo caso: \begin{eqnarray*}
    y_t & = & (\phi_{01} + \phi_{11} y_{t-1} + \phi_{21} y_{t-2} + \ldots + \phi_{p_1 1} y_{t-p_1}) (1 - G(y_{t-1}; \gamma, c)) \\
    &  & + (\phi_{02} + \phi_{12} y_{t-1} + \phi_{22} y_{t-2} + \ldots + \phi_{p_2 2} y_{t-p_2}) G(y_{t-1}; \gamma, c) \\
    &  & + \varepsilon_t
\end{eqnarray*}

De igual forma que en el caso de los modelos ARIMA y VAR, el número de
rrezagos utilizados es determinado mediennte el uso de criterios de
información como el de Akaike: \[
    AIC(p_1, p_2) = n_1 ln(\hat{\sigma}^2_1) + n_2 ln(\hat{\sigma}^2_2) + 2(p_1 + 1) + 2(p_2 + 1)
\]

Los modelos SETAR y STAR generan procesos estacionarios siempre que
cumplan ciertas condiciones. En estas notas nos enfocaremos únicamente
en el modelo SETAR el cual genera en un proceso estacionario cuando:

\begin{enumerate}
    \item $\phi_{11} < 1$, $\phi_{12} < 1$, $\phi_{11} \cdot \phi_{12} < 1$
    
    \item $\phi_{11} = 1$, $\phi_{12} < 1$, $\phi_{01} > 0$
    
    \item $\phi_{11} < 1$, $\phi_{12} = 1$, $\phi_{02} < 0$
    
    \item $\phi_{11} = 1$, $\phi_{12} = 1$, $\phi_{02} < 0 < \phi_{01}$
    
    \item $\phi_{11} \cdot \phi_{12} = 1$, $\phi_{11} < 0$, $\phi_{02} + \phi_{12} \cdot \phi_{01} > 0$
\end{enumerate}

Finalmente, en ocasiones podemos estar interesados en modelos donde los
regímenes sean más de 2, es decir, digamos \(m\) umbrales bajo un modelo
SETAR o STAR. Por ejemplo, en el caso de un modelo SETAR podemos
verificar que \(m\) regímnes implican \(m + 1\) umbrales:
\(c_0, c_1, \ldots, c_m\). En cuyo caso: \begin{equation*}
    -\infty = c_0 < c_1 < \ldots < c_{m-1} < c_m = \infty
\end{equation*}

Así, tendríamos ecuaciones: \[
    y_t = \phi_{0j} + \phi_{ij} y_{t-1} + \varepsilon_t \text{ si } c_{j-1} < y_{t-1} < c_j
\]

Para \(j = 1, 2, \ldots, m\). De forma similar podemos recomponer el
modelo STAR.

\hypertarget{reguxedmenes-determinados-por-variables-no-observables}{%
\subsubsection{Regímenes determinados por variables no
observables}\label{reguxedmenes-determinados-por-variables-no-observables}}

Este tipo de modelosmasume que el regímen ocurre en el momento \(t\) y
que no puede ser observado, ya que este es determinado por un proceso no
observable, el cual denotamos como \(s_t\). En el caso de dos regímenes,
\(s_t\) puede ser asumido como que toma 2 valores: 1 y 2, por ejemplo.
Supongamos que el proceso subyacente tiene una forma del tipo AR(1) dado
por: \[
    y_t = 
    \begin{cases}
        \phi_{01} + \phi_{11} y_{t-1} + \varepsilon_t \text{ si } s_t = 1 \\
        \phi_{02} + \phi_{12} y_{t-1} + \varepsilon_t \text{ si } s_t = 2
    \end{cases}
    \label{eq_swching_obs}
\]

O en un formato más corto de notación: \[
    y_t = \phi_{0 s_t} + \phi_{1 s_t} y_{t-1} + \varepsilon_t
\]

Para complementar el modelo, las propiedades del proceso \(s_t\)
necesitan ser especificadas. El modelo más popular dentro de esta
familia es el propuesto por James Hamilton en 1989 el cual es conocido
como Markov Switching Model (MSM), en el cual el proceso \(s_t\) se
asume como un proceso de Markov de primer orden. Esto implica que el
regímen actula \(s_t\) sólo dependen del período \(s_{t-1}\).

Así, el modelo es completado mediante la definición de las
probabilidades de transición para moverse del un estado a otro:
\begin{eqnarray*}
    \mathbb{P}(s_t = 1 | s_{t-1} = 1) & = & p_{11} \\
    \mathbb{P}(s_t = 2 | s_{t-1} = 1) & = & p_{12} \\
    \mathbb{P}(s_t = 1 | s_{t-1} = 2) & = & p_{21} \\
    \mathbb{P}(s_t = 2 | s_{t-1} = 2) & = & p_{22} 
\end{eqnarray*}

Así, \(p_{ij}\) es igual a la probabilidad de que la cadena de Markov
pase del estado \(i\) en el momento \(t-1\) al estdo \(j\) en el tiempo
\(t\). En todo caso asumiremos que \(p_{ij} > 0\) y que:
\begin{eqnarray*}
    p_{11} + p_{12} = 1 \\
    p_{21} + p_{22} = 1 
\end{eqnarray*}

Otro tipo de probabilidades a analizar son las probabilidades
incondicionales de \(\mathbb{P}(s_t = i)\), \(i = 1, 2\). Usando la
teoría ergódica de la cadenas de Markov, estas probabilidades están
dadas por: \begin{eqnarray*}
    \mathbb{P}(s_t = 1) & = & \frac{1 - p_{22}}{2 - p_{11} - p_{22}} \\ 
    \mathbb{P}(s_t = 2) & = & \frac{1 - p_{11}}{2 - p_{11} - p_{22}}
\end{eqnarray*}

Un caso más general es el de múltiples regímenes en el cual \(s_t\)
puede tomar cualquier valor \(m > 2\), \(m \in \mathbb{N}\). Este modelo
se puede escribir como: \[
    y_t = \phi_{0j} + \phi_{1j} y_{t-1} + \varepsilon_t \text{ si } s_t = j \text{, para } j = 1, 2, \ldots, m
\]

Donde las probabilidades de transición estarán dadas por: \[
    p_{ij} = \mathbb{P}(s_t = j | s_{t-1} = i) \text{ para } i , j = 1, 2, \ldots, m
\]

Donde la ecuación anterior satisface que \(p_{ij} > 0\),
\(\forall i, j = = 1, 2, \ldots, m\) y que: \begin{equation*}
    \sum_{j=1}^m p_{ij} = 1 \text{ para } i = 1, 2, \ldots, m
\end{equation*}

Finalmente, plantearemos el procedimiento empiríco seguido para la
estimación de este tipo de modelos:

\begin{enumerate}
    \item Estimar un proceso AR(p)
    
    \item Probar una hipótesis nula de no linealidad de acurdo con alguno de los modelos SERAR, STAR o MSM
    
    \item Estimar lo parámetros
    
    \item Evluar los resultados del modelo
    
    \item Ajustar, en su caso, la estimación o modelo
    
    \item Pronósticar o realizar análisis impluso-respuesta
\end{enumerate}


\printbibliography


\end{document}
