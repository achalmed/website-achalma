% Preámbulo
\usepackage[utf8]{inputenc} % Codificación de entrada en UTF-8
\usepackage{comment} % Permite comentar secciones del código
\usepackage{marvosym} % Agrega símbolos adicionales
\usepackage{graphicx} % Permite insertar imágenes
\usepackage{float} % Permite controlar la ubicación de figuras y tablas
\usepackage[normalem]{ulem} % Permite subrayar texto
%\usepackage[spanish]{babel} % Configuración para escribir en español
%\selectlanguage{spanish} % Selecciona el idioma español
\useunder{\uline}{\ul}{} % Crea una nueva macro para subrayar
\newcommand{\myparagraph}[1]{\paragraph{#1}\mbox{}\\}


% Otros paquetes
%\usepackage[natbibapa]{apacite} % Formato de citación APA
\setcounter{secnumdepth}{3} % Numera las secciones
\usepackage{mathptmx} % Fuente de texto matemática
\usepackage{amssymb} % Símbolos adicionales de matemáticas
\usepackage{setspace} % Controla el espacio entre líneas
\usepackage{lipsum} % Crea texto aleatorio
\usepackage{multirow} % Permite crear tablas con varias filas
\usepackage{array} % Da formato a las tablas
\usepackage{subcaption} % Permite insertar subimágenes
%\usetikzlibrary{calc,positioning,shapes.geometric,shapes.symbols,shapes.misc} % Biblioteca de TikZ

% Cambiar titulo de bibliografía
%\addto\captionsspanish{\renewcommand{\bibname}{\centering REFERENCIAS BIBLIOGRÁFICAS}}

% Numeración de las secciones con números romanos mayúsculas
\renewcommand{\thesection}{\Roman{section}} %Las secciones estaran con numeración romana MAYUSCULA
\renewcommand{\thesubsection}{\arabic{section}.\arabic{subsection}}
\renewcommand{\theequation}{\arabic{section}.\arabic{equation}}
\renewcommand{\thetable}{\arabic{section}.\arabic{table}}
\renewcommand{\thefigure}{\arabic{section}.\arabic{figure}}


%---- DEDICATORIA -----------------------------------
\newenvironment{dedicatoria}
{\clearpage           % Nueva página
  \thispagestyle{empty}% No encabezados ni pie de página
  \vspace*{\stretch{8}}% some space at the top 
  \itshape             % Texto en cursiva (italica)
  \raggedleft          % flush to the right margin
}
{\par % end the paragraph
  \vspace{\stretch{0.6}} % space at bottom is three times that at the top
  \clearpage           % Termina la página
}

%----------------------------------------------


%---- PERMISOS -----------------------------------
\newenvironment{permisos}
{\clearpage           % Nueva página
  \thispagestyle{empty}% No encabezados ni pie de página
  \vspace*{\fill}
  \raggedright          % flush to the Left margin
}
{\par % end the paragraph
  \clearpage           % Termina la página
}

%----------------------------------------------


%---- AGRADECIMIENTOS -----------------------------------
\newenvironment{agradecimientos}
{\clearpage           % Nueva página
  \thispagestyle{empty}% No encabezados ni pie de página

}
{\par % end the paragraph
  \clearpage           % Termina la página
}

%-------------------------------------------------------


%---------------- ENCABEZADOS Y PIE DE PÁGINA --------------------
%\usepackage[left=3.81 cm,right=2.54 cm,top=2.54 cm,bottom=2.54 cm]{geometry}
%\usepackage{fancyhdr}
%\pagestyle{fancy}
%\fancyhf{}

%\renewcommand{\headrulewidth}{0.5 pt}
%\renewcommand{\footrulewidth}{0.5 pt}


%\lhead[]{\textit{\scriptsize UNIVERSIDAD NACIONAL SAN CRISTÓBAL DE HUAMANGA \\ \scriptsize{\@facultad}}}
%\rhead[]{\textit{\scriptsize \leftmark}}

%\lfoot[]{\textit{\scriptsize {\@title} \\ Bach. {\@author}}}
%\rfoot[]{\small \thepage}
%---------------------------------------

%MODIFICACIONES  A ENCABEZADO Y PIE DE PÁGINA

%\fancypagestyle{plain}{
%\fancyhf{}

%Líneas superiores e inferiores del encabezado y pie de página respectivamente.

%\renewcommand{\headrulewidth}{0.5 pt}
%\renewcommand{\footrulewidth}{0.5 pt}

% Contenido del encabezado y pie de página 

%\lhead[]{\textit{\scriptsize UNIVERSIDAD NACIONAL SAN CRISTÓBAL DE HUAMANGA \\ \scriptsize {\@facultad}}}
%\rhead[]{\textit{\scriptsize \leftmark}}


%\lfoot[]{\textit{\scriptsize {\@title} \\ Bach. {\@author}}}
%\rfoot[]{\small \thepage}
%}
%----------------------------------------------

%RMARDOW%

\usepackage{amsthm}
\usepackage{float}
\usepackage{rotating, graphicx}
\usepackage{multirow}
\usepackage{tabularx}

% new command for pretty oversets with \sim
\newcommand\simcal[1]{\stackrel{\sim}{\smash{\mathcal{#1}}\rule{0pt}{0.5ex}}}

\newcommand{\comma}{,\,}

\floatplacement{figure}{H}

\PassOptionsToPackage{table}{xcolor}

\usepackage{tcolorbox}

\definecolor{kcblue}{HTML}{D7DDEF}
\definecolor{kcdarkblue}{HTML}{2B4E70}

\makeatletter
\def\thm@space@setup{%
  \thm@preskip=8pt plus 2pt minus 4pt
  \thm@postskip=\thm@preskip
}
\makeatother

% \makeatletter % undo the wrong changes made by mathspec
% \let\RequirePackage\original@RequirePackage
% \let\usepackage\RequirePackage
% \makeatother

\newenvironment{rmdknit}
{\begin{center}
    \begin{tabular}{|p{0.9\textwidth}|}
      \hline \\
      }
      {
      \\\\\hline
    \end{tabular}
  \end{center}
}

\newenvironment{rmdnote}
{\begin{center}
    \begin{tabular}{|p{0.9\textwidth}|}
      \hline \\
      }
      {
      \\\\\hline
    \end{tabular}
  \end{center}
}

\newtcolorbox[auto counter, number within=section]{keyconcepts}[2][]{%
  colback=kcblue,colframe=kcdarkblue,fonttitle=\bfseries, title=Key Concept~#2, after title={\newline #1}, beforeafter skip=15pt}


