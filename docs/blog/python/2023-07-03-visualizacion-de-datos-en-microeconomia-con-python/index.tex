% Options for packages loaded elsewhere
\PassOptionsToPackage{unicode}{hyperref}
\PassOptionsToPackage{hyphens}{url}
\PassOptionsToPackage{dvipsnames,svgnames,x11names}{xcolor}
%
\documentclass[
  a4paper,
]{article}

\usepackage{amsmath,amssymb}
\usepackage{iftex}
\ifPDFTeX
  \usepackage[T1]{fontenc}
  \usepackage[utf8]{inputenc}
  \usepackage{textcomp} % provide euro and other symbols
\else % if luatex or xetex
  \usepackage{unicode-math}
  \defaultfontfeatures{Scale=MatchLowercase}
  \defaultfontfeatures[\rmfamily]{Ligatures=TeX,Scale=1}
\fi
\usepackage{lmodern}
\ifPDFTeX\else  
    % xetex/luatex font selection
\fi
% Use upquote if available, for straight quotes in verbatim environments
\IfFileExists{upquote.sty}{\usepackage{upquote}}{}
\IfFileExists{microtype.sty}{% use microtype if available
  \usepackage[]{microtype}
  \UseMicrotypeSet[protrusion]{basicmath} % disable protrusion for tt fonts
}{}
\makeatletter
\@ifundefined{KOMAClassName}{% if non-KOMA class
  \IfFileExists{parskip.sty}{%
    \usepackage{parskip}
  }{% else
    \setlength{\parindent}{0pt}
    \setlength{\parskip}{6pt plus 2pt minus 1pt}}
}{% if KOMA class
  \KOMAoptions{parskip=half}}
\makeatother
\usepackage{xcolor}
\usepackage[top=2.54cm,right=2.54cm,bottom=2.54cm,left=2.54cm]{geometry}
\setlength{\emergencystretch}{3em} % prevent overfull lines
\setcounter{secnumdepth}{-\maxdimen} % remove section numbering
% Make \paragraph and \subparagraph free-standing
\ifx\paragraph\undefined\else
  \let\oldparagraph\paragraph
  \renewcommand{\paragraph}[1]{\oldparagraph{#1}\mbox{}}
\fi
\ifx\subparagraph\undefined\else
  \let\oldsubparagraph\subparagraph
  \renewcommand{\subparagraph}[1]{\oldsubparagraph{#1}\mbox{}}
\fi


\providecommand{\tightlist}{%
  \setlength{\itemsep}{0pt}\setlength{\parskip}{0pt}}\usepackage{longtable,booktabs,array}
\usepackage{calc} % for calculating minipage widths
% Correct order of tables after \paragraph or \subparagraph
\usepackage{etoolbox}
\makeatletter
\patchcmd\longtable{\par}{\if@noskipsec\mbox{}\fi\par}{}{}
\makeatother
% Allow footnotes in longtable head/foot
\IfFileExists{footnotehyper.sty}{\usepackage{footnotehyper}}{\usepackage{footnote}}
\makesavenoteenv{longtable}
\usepackage{graphicx}
\makeatletter
\def\maxwidth{\ifdim\Gin@nat@width>\linewidth\linewidth\else\Gin@nat@width\fi}
\def\maxheight{\ifdim\Gin@nat@height>\textheight\textheight\else\Gin@nat@height\fi}
\makeatother
% Scale images if necessary, so that they will not overflow the page
% margins by default, and it is still possible to overwrite the defaults
% using explicit options in \includegraphics[width, height, ...]{}
\setkeys{Gin}{width=\maxwidth,height=\maxheight,keepaspectratio}
% Set default figure placement to htbp
\makeatletter
\def\fps@figure{htbp}
\makeatother

\makeatletter
\makeatother
\makeatletter
\makeatother
\makeatletter
\@ifpackageloaded{caption}{}{\usepackage{caption}}
\AtBeginDocument{%
\ifdefined\contentsname
  \renewcommand*\contentsname{Tabla de contenidos}
\else
  \newcommand\contentsname{Tabla de contenidos}
\fi
\ifdefined\listfigurename
  \renewcommand*\listfigurename{Listado de Figuras}
\else
  \newcommand\listfigurename{Listado de Figuras}
\fi
\ifdefined\listtablename
  \renewcommand*\listtablename{Listado de Tablas}
\else
  \newcommand\listtablename{Listado de Tablas}
\fi
\ifdefined\figurename
  \renewcommand*\figurename{Figura}
\else
  \newcommand\figurename{Figura}
\fi
\ifdefined\tablename
  \renewcommand*\tablename{Tabla}
\else
  \newcommand\tablename{Tabla}
\fi
}
\@ifpackageloaded{float}{}{\usepackage{float}}
\floatstyle{ruled}
\@ifundefined{c@chapter}{\newfloat{codelisting}{h}{lop}}{\newfloat{codelisting}{h}{lop}[chapter]}
\floatname{codelisting}{Listado}
\newcommand*\listoflistings{\listof{codelisting}{Listado de Listados}}
\makeatother
\makeatletter
\@ifpackageloaded{caption}{}{\usepackage{caption}}
\@ifpackageloaded{subcaption}{}{\usepackage{subcaption}}
\makeatother
\makeatletter
\@ifpackageloaded{tcolorbox}{}{\usepackage[skins,breakable]{tcolorbox}}
\makeatother
\makeatletter
\@ifundefined{shadecolor}{\definecolor{shadecolor}{rgb}{.97, .97, .97}}
\makeatother
\makeatletter
\makeatother
\makeatletter
\makeatother
\ifLuaTeX
\usepackage[bidi=basic]{babel}
\else
\usepackage[bidi=default]{babel}
\fi
\babelprovide[main,import]{spanish}
% get rid of language-specific shorthands (see #6817):
\let\LanguageShortHands\languageshorthands
\def\languageshorthands#1{}
\ifLuaTeX
  \usepackage{selnolig}  % disable illegal ligatures
\fi
\usepackage[]{biblatex}
\addbibresource{../../../../references.bib}
\IfFileExists{bookmark.sty}{\usepackage{bookmark}}{\usepackage{hyperref}}
\IfFileExists{xurl.sty}{\usepackage{xurl}}{} % add URL line breaks if available
\urlstyle{same} % disable monospaced font for URLs
\hypersetup{
  pdftitle={Visualización de datos en microeconomía con python},
  pdfauthor={Edison Achalma},
  pdflang={es},
  colorlinks=true,
  linkcolor={blue},
  filecolor={Maroon},
  citecolor={Blue},
  urlcolor={Blue},
  pdfcreator={LaTeX via pandoc}}

\title{Visualización de datos en microeconomía con python}
\usepackage{etoolbox}
\makeatletter
\providecommand{\subtitle}[1]{% add subtitle to \maketitle
  \apptocmd{\@title}{\par {\large #1 \par}}{}{}
}
\makeatother
\subtitle{Explora cómo la visualización de datos puede ayudarte a
comprender las interacciones económicas a nivel micro.}
\author{Edison Achalma}
\date{2023-07-03}

\begin{document}
\maketitle
\ifdefined\Shaded\renewenvironment{Shaded}{\begin{tcolorbox}[interior hidden, breakable, enhanced, borderline west={3pt}{0pt}{shadecolor}, frame hidden, boxrule=0pt, sharp corners]}{\end{tcolorbox}}\fi

\hypertarget{introducciuxf3n-a-la-visualizaciuxf3n-de-datos-en-microeconomuxeda}{%
\section{Introducción a la visualización de datos en
microeconomía}\label{introducciuxf3n-a-la-visualizaciuxf3n-de-datos-en-microeconomuxeda}}

\hypertarget{importancia-de-la-visualizaciuxf3n-de-datos-en-el-anuxe1lisis-microeconuxf3mico}{%
\subsection{Importancia de la visualización de datos en el análisis
microeconómico}\label{importancia-de-la-visualizaciuxf3n-de-datos-en-el-anuxe1lisis-microeconuxf3mico}}

\hypertarget{beneficios-de-utilizar-gruxe1ficos-en-la-interpretaciuxf3n-de-datos-en-microeconomuxeda}{%
\subsection{Beneficios de utilizar gráficos en la interpretación de
datos en
microeconomía}\label{beneficios-de-utilizar-gruxe1ficos-en-la-interpretaciuxf3n-de-datos-en-microeconomuxeda}}

\hypertarget{introducciuxf3n-a-las-bibliotecas-y-herramientas-utilizadas-en-la-visualizaciuxf3n-de-datos-en-microeconomuxeda}{%
\subsection{Introducción a las bibliotecas y herramientas utilizadas en
la visualización de datos en
microeconomía}\label{introducciuxf3n-a-las-bibliotecas-y-herramientas-utilizadas-en-la-visualizaciuxf3n-de-datos-en-microeconomuxeda}}

\hypertarget{gruxe1ficos-de-oferta-y-demanda}{%
\section{Gráficos de oferta y
demanda}\label{gruxe1ficos-de-oferta-y-demanda}}

\hypertarget{visualizaciuxf3n-de-la-relaciuxf3n-entre-la-oferta-y-demanda-de-un-bien-o-servicio}{%
\subsection{Visualización de la relación entre la oferta y demanda de un
bien o
servicio}\label{visualizaciuxf3n-de-la-relaciuxf3n-entre-la-oferta-y-demanda-de-un-bien-o-servicio}}

\hypertarget{utilizaciuxf3n-de-gruxe1ficos-de-oferta-y-demanda-para-analizar-equilibrios-de-mercado-elasticidad-etc.}{%
\subsection{Utilización de gráficos de oferta y demanda para analizar
equilibrios de mercado, elasticidad,
etc.}\label{utilizaciuxf3n-de-gruxe1ficos-de-oferta-y-demanda-para-analizar-equilibrios-de-mercado-elasticidad-etc.}}

\hypertarget{ejemplos-pruxe1cticos-de-visualizaciuxf3n-de-datos-de-oferta-y-demanda}{%
\subsection{Ejemplos prácticos de visualización de datos de oferta y
demanda}\label{ejemplos-pruxe1cticos-de-visualizaciuxf3n-de-datos-de-oferta-y-demanda}}

\hypertarget{gruxe1ficos-de-costos-y-producciuxf3n}{%
\section{Gráficos de costos y
producción}\label{gruxe1ficos-de-costos-y-producciuxf3n}}

\hypertarget{visualizaciuxf3n-de-los-costos-y-la-producciuxf3n-en-el-anuxe1lisis-microeconuxf3mico}{%
\subsection{Visualización de los costos y la producción en el análisis
microeconómico}\label{visualizaciuxf3n-de-los-costos-y-la-producciuxf3n-en-el-anuxe1lisis-microeconuxf3mico}}

\hypertarget{utilizaciuxf3n-de-gruxe1ficos-de-costos-y-producciuxf3n-para-entender-la-eficiencia-y-la-estructura-de-costos}{%
\subsection{Utilización de gráficos de costos y producción para entender
la eficiencia y la estructura de
costos}\label{utilizaciuxf3n-de-gruxe1ficos-de-costos-y-producciuxf3n-para-entender-la-eficiencia-y-la-estructura-de-costos}}

\hypertarget{ejemplos-pruxe1cticos-de-visualizaciuxf3n-de-datos-de-costos-y-producciuxf3n}{%
\subsection{Ejemplos prácticos de visualización de datos de costos y
producción}\label{ejemplos-pruxe1cticos-de-visualizaciuxf3n-de-datos-de-costos-y-producciuxf3n}}

\hypertarget{anuxe1lisis-de-preferencias-del-consumidor}{%
\section{Análisis de preferencias del
consumidor}\label{anuxe1lisis-de-preferencias-del-consumidor}}

\hypertarget{representaciuxf3n-gruxe1fica-de-las-preferencias-del-consumidor-utilizando-curvas-de-indiferencia}{%
\subsection{Representación gráfica de las preferencias del consumidor
utilizando curvas de
indiferencia}\label{representaciuxf3n-gruxe1fica-de-las-preferencias-del-consumidor-utilizando-curvas-de-indiferencia}}

\hypertarget{utilizaciuxf3n-de-gruxe1ficos-de-utilidad-y-curvas-de-indiferencia-para-analizar-decisiones-de-consumo}{%
\subsection{Utilización de gráficos de utilidad y curvas de indiferencia
para analizar decisiones de
consumo}\label{utilizaciuxf3n-de-gruxe1ficos-de-utilidad-y-curvas-de-indiferencia-para-analizar-decisiones-de-consumo}}

\hypertarget{ejemplos-pruxe1cticos-de-visualizaciuxf3n-de-datos-de-preferencias-del-consumidor}{%
\subsection{Ejemplos prácticos de visualización de datos de preferencias
del
consumidor}\label{ejemplos-pruxe1cticos-de-visualizaciuxf3n-de-datos-de-preferencias-del-consumidor}}

\hypertarget{visualizaciuxf3n-de-datos-en-anuxe1lisis-de-competencia}{%
\section{Visualización de datos en análisis de
competencia}\label{visualizaciuxf3n-de-datos-en-anuxe1lisis-de-competencia}}

\hypertarget{utilizaciuxf3n-de-gruxe1ficos-de-competencia-para-representar-estructuras-de-mercado-barreras-de-entrada-etc.}{%
\subsection{Utilización de gráficos de competencia para representar
estructuras de mercado, barreras de entrada,
etc.}\label{utilizaciuxf3n-de-gruxe1ficos-de-competencia-para-representar-estructuras-de-mercado-barreras-de-entrada-etc.}}

\hypertarget{anuxe1lisis-de-estrategias-de-precios-diferenciaciuxf3n-de-productos-y-poder-de-mercado-utilizando-gruxe1ficos}{%
\subsection{Análisis de estrategias de precios, diferenciación de
productos y poder de mercado utilizando
gráficos}\label{anuxe1lisis-de-estrategias-de-precios-diferenciaciuxf3n-de-productos-y-poder-de-mercado-utilizando-gruxe1ficos}}

\hypertarget{ejemplos-pruxe1cticos-de-visualizaciuxf3n-de-datos-en-anuxe1lisis-de-competencia}{%
\subsection{Ejemplos prácticos de visualización de datos en análisis de
competencia}\label{ejemplos-pruxe1cticos-de-visualizaciuxf3n-de-datos-en-anuxe1lisis-de-competencia}}

\hypertarget{visualizaciuxf3n-interactiva-de-datos-en-microeconomuxeda}{%
\section{Visualización interactiva de datos en
microeconomía}\label{visualizaciuxf3n-interactiva-de-datos-en-microeconomuxeda}}

\hypertarget{utilizaciuxf3n-de-bibliotecas-como-plotly-y-bokeh-para-crear-gruxe1ficos-interactivos-en-el-anuxe1lisis-microeconuxf3mico}{%
\subsection{Utilización de bibliotecas como Plotly y Bokeh para crear
gráficos interactivos en el análisis
microeconómico}\label{utilizaciuxf3n-de-bibliotecas-como-plotly-y-bokeh-para-crear-gruxe1ficos-interactivos-en-el-anuxe1lisis-microeconuxf3mico}}

\hypertarget{incorporaciuxf3n-de-herramientas-interactivas-como-zoom-selecciuxf3n-y-filtros-en-gruxe1ficos-de-microeconomuxeda}{%
\subsection{Incorporación de herramientas interactivas como zoom,
selección y filtros en gráficos de
microeconomía}\label{incorporaciuxf3n-de-herramientas-interactivas-como-zoom-selecciuxf3n-y-filtros-en-gruxe1ficos-de-microeconomuxeda}}

\hypertarget{ejemplos-pruxe1cticos-de-visualizaciuxf3n-interactiva-de-datos-en-microeconomuxeda}{%
\subsection{Ejemplos prácticos de visualización interactiva de datos en
microeconomía}\label{ejemplos-pruxe1cticos-de-visualizaciuxf3n-interactiva-de-datos-en-microeconomuxeda}}

\hypertarget{casos-de-estudio-y-ejemplos-pruxe1cticos}{%
\section{Casos de estudio y ejemplos
prácticos}\label{casos-de-estudio-y-ejemplos-pruxe1cticos}}

\hypertarget{aplicaciuxf3n-de-la-visualizaciuxf3n-de-datos-en-microeconomuxeda-en-diferentes-escenarios-como-anuxe1lisis-de-mercado-toma-de-decisiones-de-precios-etc.}{%
\subsection{Aplicación de la visualización de datos en microeconomía en
diferentes escenarios, como análisis de mercado, toma de decisiones de
precios,
etc.}\label{aplicaciuxf3n-de-la-visualizaciuxf3n-de-datos-en-microeconomuxeda-en-diferentes-escenarios-como-anuxe1lisis-de-mercado-toma-de-decisiones-de-precios-etc.}}

\hypertarget{ejemplos-de-visualizaciuxf3n-de-datos-en-microeconomuxeda-en-situaciones-reales}{%
\subsection{Ejemplos de visualización de datos en microeconomía en
situaciones
reales}\label{ejemplos-de-visualizaciuxf3n-de-datos-en-microeconomuxeda-en-situaciones-reales}}

\hypertarget{conclusiones-y-recursos-adicionales}{%
\section{Conclusiones y recursos
adicionales}\label{conclusiones-y-recursos-adicionales}}

\hypertarget{resumen-de-las-tuxe9cnicas-y-mejores-pruxe1cticas-en-la-visualizaciuxf3n-de-datos-en-microeconomuxeda}{%
\subsection{Resumen de las técnicas y mejores prácticas en la
visualización de datos en
microeconomía}\label{resumen-de-las-tuxe9cnicas-y-mejores-pruxe1cticas-en-la-visualizaciuxf3n-de-datos-en-microeconomuxeda}}

\hypertarget{recursos-adicionales-para-aprender-muxe1s-sobre-la-visualizaciuxf3n-de-datos-en-el-campo-de-la-microeconomuxeda}{%
\subsection{Recursos adicionales para aprender más sobre la
visualización de datos en el campo de la
microeconomía}\label{recursos-adicionales-para-aprender-muxe1s-sobre-la-visualizaciuxf3n-de-datos-en-el-campo-de-la-microeconomuxeda}}

\hypertarget{publicaciones-similares}{%
\section{Publicaciones Similares}\label{publicaciones-similares}}

Si te interesó este artículo, te recomendamos que explores otros blogs y
recursos relacionados que pueden ampliar tus conocimientos. Aquí te dejo
algunas sugerencias:


\printbibliography


\end{document}
