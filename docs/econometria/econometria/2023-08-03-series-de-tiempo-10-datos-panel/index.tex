% Options for packages loaded elsewhere
\PassOptionsToPackage{unicode}{hyperref}
\PassOptionsToPackage{hyphens}{url}
\PassOptionsToPackage{dvipsnames,svgnames,x11names}{xcolor}
%
\documentclass[
  a4paper,
]{article}

\usepackage{amsmath,amssymb}
\usepackage{iftex}
\ifPDFTeX
  \usepackage[T1]{fontenc}
  \usepackage[utf8]{inputenc}
  \usepackage{textcomp} % provide euro and other symbols
\else % if luatex or xetex
  \usepackage{unicode-math}
  \defaultfontfeatures{Scale=MatchLowercase}
  \defaultfontfeatures[\rmfamily]{Ligatures=TeX,Scale=1}
\fi
\usepackage{lmodern}
\ifPDFTeX\else  
    % xetex/luatex font selection
\fi
% Use upquote if available, for straight quotes in verbatim environments
\IfFileExists{upquote.sty}{\usepackage{upquote}}{}
\IfFileExists{microtype.sty}{% use microtype if available
  \usepackage[]{microtype}
  \UseMicrotypeSet[protrusion]{basicmath} % disable protrusion for tt fonts
}{}
\makeatletter
\@ifundefined{KOMAClassName}{% if non-KOMA class
  \IfFileExists{parskip.sty}{%
    \usepackage{parskip}
  }{% else
    \setlength{\parindent}{0pt}
    \setlength{\parskip}{6pt plus 2pt minus 1pt}}
}{% if KOMA class
  \KOMAoptions{parskip=half}}
\makeatother
\usepackage{xcolor}
\usepackage[top=2.54cm,right=2.54cm,bottom=2.54cm,left=2.54cm]{geometry}
\setlength{\emergencystretch}{3em} % prevent overfull lines
\setcounter{secnumdepth}{-\maxdimen} % remove section numbering
% Make \paragraph and \subparagraph free-standing
\ifx\paragraph\undefined\else
  \let\oldparagraph\paragraph
  \renewcommand{\paragraph}[1]{\oldparagraph{#1}\mbox{}}
\fi
\ifx\subparagraph\undefined\else
  \let\oldsubparagraph\subparagraph
  \renewcommand{\subparagraph}[1]{\oldsubparagraph{#1}\mbox{}}
\fi


\providecommand{\tightlist}{%
  \setlength{\itemsep}{0pt}\setlength{\parskip}{0pt}}\usepackage{longtable,booktabs,array}
\usepackage{calc} % for calculating minipage widths
% Correct order of tables after \paragraph or \subparagraph
\usepackage{etoolbox}
\makeatletter
\patchcmd\longtable{\par}{\if@noskipsec\mbox{}\fi\par}{}{}
\makeatother
% Allow footnotes in longtable head/foot
\IfFileExists{footnotehyper.sty}{\usepackage{footnotehyper}}{\usepackage{footnote}}
\makesavenoteenv{longtable}
\usepackage{graphicx}
\makeatletter
\def\maxwidth{\ifdim\Gin@nat@width>\linewidth\linewidth\else\Gin@nat@width\fi}
\def\maxheight{\ifdim\Gin@nat@height>\textheight\textheight\else\Gin@nat@height\fi}
\makeatother
% Scale images if necessary, so that they will not overflow the page
% margins by default, and it is still possible to overwrite the defaults
% using explicit options in \includegraphics[width, height, ...]{}
\setkeys{Gin}{width=\maxwidth,height=\maxheight,keepaspectratio}
% Set default figure placement to htbp
\makeatletter
\def\fps@figure{htbp}
\makeatother

\makeatletter
\makeatother
\makeatletter
\makeatother
\makeatletter
\@ifpackageloaded{caption}{}{\usepackage{caption}}
\AtBeginDocument{%
\ifdefined\contentsname
  \renewcommand*\contentsname{Tabla de contenidos}
\else
  \newcommand\contentsname{Tabla de contenidos}
\fi
\ifdefined\listfigurename
  \renewcommand*\listfigurename{Listado de Figuras}
\else
  \newcommand\listfigurename{Listado de Figuras}
\fi
\ifdefined\listtablename
  \renewcommand*\listtablename{Listado de Tablas}
\else
  \newcommand\listtablename{Listado de Tablas}
\fi
\ifdefined\figurename
  \renewcommand*\figurename{Figura}
\else
  \newcommand\figurename{Figura}
\fi
\ifdefined\tablename
  \renewcommand*\tablename{Tabla}
\else
  \newcommand\tablename{Tabla}
\fi
}
\@ifpackageloaded{float}{}{\usepackage{float}}
\floatstyle{ruled}
\@ifundefined{c@chapter}{\newfloat{codelisting}{h}{lop}}{\newfloat{codelisting}{h}{lop}[chapter]}
\floatname{codelisting}{Listado}
\newcommand*\listoflistings{\listof{codelisting}{Listado de Listados}}
\makeatother
\makeatletter
\@ifpackageloaded{caption}{}{\usepackage{caption}}
\@ifpackageloaded{subcaption}{}{\usepackage{subcaption}}
\makeatother
\makeatletter
\@ifpackageloaded{tcolorbox}{}{\usepackage[skins,breakable]{tcolorbox}}
\makeatother
\makeatletter
\@ifundefined{shadecolor}{\definecolor{shadecolor}{rgb}{.97, .97, .97}}
\makeatother
\makeatletter
\makeatother
\makeatletter
\makeatother
\ifLuaTeX
\usepackage[bidi=basic]{babel}
\else
\usepackage[bidi=default]{babel}
\fi
\babelprovide[main,import]{spanish}
% get rid of language-specific shorthands (see #6817):
\let\LanguageShortHands\languageshorthands
\def\languageshorthands#1{}
\ifLuaTeX
  \usepackage{selnolig}  % disable illegal ligatures
\fi
\usepackage[]{biblatex}
\addbibresource{../../../../references.bib}
\IfFileExists{bookmark.sty}{\usepackage{bookmark}}{\usepackage{hyperref}}
\IfFileExists{xurl.sty}{\usepackage{xurl}}{} % add URL line breaks if available
\urlstyle{same} % disable monospaced font for URLs
\hypersetup{
  pdftitle={Notas de Clase Series de Tiempo},
  pdfauthor={Edison Achalma},
  pdflang={es},
  colorlinks=true,
  linkcolor={blue},
  filecolor={Maroon},
  citecolor={Blue},
  urlcolor={Blue},
  pdfcreator={LaTeX via pandoc}}

\title{Notas de Clase Series de Tiempo}
\usepackage{etoolbox}
\makeatletter
\providecommand{\subtitle}[1]{% add subtitle to \maketitle
  \apptocmd{\@title}{\par {\large #1 \par}}{}{}
}
\makeatother
\subtitle{Descubre cómo seleccionar hardware, descargar la imagen ISO y
preparar los medios de instalación. Exploraremos opciones para probar o
instalar Linux en tu equipo.}
\author{Edison Achalma}
\date{2023-08-27}

\begin{document}
\maketitle
\ifdefined\Shaded\renewenvironment{Shaded}{\begin{tcolorbox}[sharp corners, interior hidden, enhanced, breakable, frame hidden, boxrule=0pt, borderline west={3pt}{0pt}{shadecolor}]}{\end{tcolorbox}}\fi

\hypertarget{modelos-de-datos-panel}{%
\section{Modelos de Datos Panel}\label{modelos-de-datos-panel}}

\hypertarget{motivaciuxf3n}{%
\subsection{Motivación}\label{motivaciuxf3n}}

Denotemos con \(Y_{it}\) el componente o individuo \(i\),
\(i = 1, 2, \ldots, N\) en el tiempo \(t\), \(t = 1, 2, \ldots, T\).
Típicamente las series de tiempo en forma de panel se caracterizan por
ser de dimensión de individuos corta y de dimensión temporar larga.

En general, escribiremos: \[
    Y_{it} = \alpha_i + \beta_i X_{it} + U_{it}
\]

Donde \(\alpha_i\) es un efecto fijo especificado como: \[
    \alpha_i = \beta_0 + \gamma_i Z_i
\]

\hypertarget{pruebas-de-rauxedces-unitarias-en-panel}{%
\subsection{Pruebas de Raíces Unitarias en
Panel}\label{pruebas-de-rauxedces-unitarias-en-panel}}

Las pruebas de raíces unitarias para panel suelen ser usadas
principalmente en casos macroeconómicos. De forma similar al caso de
series univariadas, asumiremos una forma de AR(1) para una serie en
panel: \[
    \Delta Y_{it} = \mu_i + \rho_i Y_{i t-1} + \sum_{i = 1}^{k_i} \varphi_{ij} \Delta Y_{i t-j} + \varepsilon_{it}
    \label{eq_AR_Panel}
\]

Donde \(i = 1, \ldots, N\), \(t = 1, \ldots, T\) y \(\varepsilon_{it}\)
es una v.a. iid que cumple con: \begin{eqnarray*}
    \mathbb{E}[\varepsilon_{it}] & = & 0 \\
    \mathbb{E}[\varepsilon_{it}^2] & = & \sigma^2_i < \infty \\
    \mathbb{E}[\varepsilon_{it}^4] & < & \infty 
\end{eqnarray*}

Al igual que en el caso univariado, en este tipo de pruebas buscamos
identificar cuando las series son I(1) y cuando I(0). Pasra tal efecto,
la prueba de raíz unitaria que utilizaremos está basada en una prueba
Dickey-Fuller Aumentada en la cual la hipótesis nula (\(H_0\)) es que
todas las series en el panel son no estacionarias, es decir, son I(1).
Es decir, \[
    H_0 : \rho_i = 0
\]

Por su parte, en el caso de la hipótesis nula tenndremos dos:

\begin{itemize}
    \item $H_1^A :$ Todas las series son I(0) -- caso homogéneo, o
    
    \item $H_1^B :$ Al menos una de las series es I(0) -- caso heterogéneo
\end{itemize}

\hypertarget{panel-var}{%
\subsection{Panel VAR}\label{panel-var}}

En esta sección extenderemos el caso del modelo VAR(p) a uno en forma
panel. En este caso asumimos que las viables exogenas son los \(p\)
rezagos de las \(k\) variables endogenas. Consideremos el siguiente caso
de un modelo panel VAR con efectos fijos --ciertamente es posible hacer
estimaciones con efectos aleatorios, no obstante requiere de supuestos
adicionales que no contemplamos en estas notas --, el cual es la forma
más común de estimación: \[
    \mathbf{Y}_{it} = \mu_i + \sum_{l = 1}^p \mathbf{A}_l \mathbf{Y}_{i t - l} + \mathbf{B} \mathbf{X}_{it} + \varepsilon_{it}
    \label{eq_PVAR}
\]

Donde \(\mathbf{Y}_{it}\) es un vector de variables endogenas
estacionarias para la unidad de corte transversal \(i\) en el tiempo
\(t\), \(\mathbf{X}_{it}\) es una matriz que contiene varaibles
exógenas, y \(\varepsilon_{it}\) es un término de error que cumple con:
\begin{eqnarray*}
    \mathbb{E}[\varepsilon_{it}] & = & 0 \\
    Var[\varepsilon_{it}] & = & \Sigma_\varepsilon
\end{eqnarray*}

Donde \(\Sigma_\varepsilon\) es una matriz definida positiva.

Para el porceso de estimación la ecuación (\ref{eq_PVAR}) se modifica en
su versión en diferencias para quedar como: \[
    \Delta \mathbf{Y}_{it} = \sum_{l = 1}^p \mathbf{A}_l \Delta \mathbf{Y}_{i t - l} + \mathbf{B} \Delta \mathbf{X}_{it} + \Delta \varepsilon_{it}
    \label{eq_Dinamic_PVAR}
\]

La ecuación (\ref{eq_Dinamic_PVAR}) se estima por un GMM.

\hypertarget{cointegraciuxf3n}{%
\subsection{Cointegración}\label{cointegraciuxf3n}}


\printbibliography


\end{document}
