% Options for packages loaded elsewhere
\PassOptionsToPackage{unicode}{hyperref}
\PassOptionsToPackage{hyphens}{url}
\PassOptionsToPackage{dvipsnames,svgnames,x11names}{xcolor}
%
\documentclass[
  a4paper,
]{article}

\usepackage{amsmath,amssymb}
\usepackage{iftex}
\ifPDFTeX
  \usepackage[T1]{fontenc}
  \usepackage[utf8]{inputenc}
  \usepackage{textcomp} % provide euro and other symbols
\else % if luatex or xetex
  \usepackage{unicode-math}
  \defaultfontfeatures{Scale=MatchLowercase}
  \defaultfontfeatures[\rmfamily]{Ligatures=TeX,Scale=1}
\fi
\usepackage{lmodern}
\ifPDFTeX\else  
    % xetex/luatex font selection
\fi
% Use upquote if available, for straight quotes in verbatim environments
\IfFileExists{upquote.sty}{\usepackage{upquote}}{}
\IfFileExists{microtype.sty}{% use microtype if available
  \usepackage[]{microtype}
  \UseMicrotypeSet[protrusion]{basicmath} % disable protrusion for tt fonts
}{}
\makeatletter
\@ifundefined{KOMAClassName}{% if non-KOMA class
  \IfFileExists{parskip.sty}{%
    \usepackage{parskip}
  }{% else
    \setlength{\parindent}{0pt}
    \setlength{\parskip}{6pt plus 2pt minus 1pt}}
}{% if KOMA class
  \KOMAoptions{parskip=half}}
\makeatother
\usepackage{xcolor}
\usepackage[top=2.54cm,right=2.54cm,bottom=2.54cm,left=2.54cm]{geometry}
\setlength{\emergencystretch}{3em} % prevent overfull lines
\setcounter{secnumdepth}{-\maxdimen} % remove section numbering
% Make \paragraph and \subparagraph free-standing
\ifx\paragraph\undefined\else
  \let\oldparagraph\paragraph
  \renewcommand{\paragraph}[1]{\oldparagraph{#1}\mbox{}}
\fi
\ifx\subparagraph\undefined\else
  \let\oldsubparagraph\subparagraph
  \renewcommand{\subparagraph}[1]{\oldsubparagraph{#1}\mbox{}}
\fi


\providecommand{\tightlist}{%
  \setlength{\itemsep}{0pt}\setlength{\parskip}{0pt}}\usepackage{longtable,booktabs,array}
\usepackage{calc} % for calculating minipage widths
% Correct order of tables after \paragraph or \subparagraph
\usepackage{etoolbox}
\makeatletter
\patchcmd\longtable{\par}{\if@noskipsec\mbox{}\fi\par}{}{}
\makeatother
% Allow footnotes in longtable head/foot
\IfFileExists{footnotehyper.sty}{\usepackage{footnotehyper}}{\usepackage{footnote}}
\makesavenoteenv{longtable}
\usepackage{graphicx}
\makeatletter
\def\maxwidth{\ifdim\Gin@nat@width>\linewidth\linewidth\else\Gin@nat@width\fi}
\def\maxheight{\ifdim\Gin@nat@height>\textheight\textheight\else\Gin@nat@height\fi}
\makeatother
% Scale images if necessary, so that they will not overflow the page
% margins by default, and it is still possible to overwrite the defaults
% using explicit options in \includegraphics[width, height, ...]{}
\setkeys{Gin}{width=\maxwidth,height=\maxheight,keepaspectratio}
% Set default figure placement to htbp
\makeatletter
\def\fps@figure{htbp}
\makeatother

% Preámbulo
\usepackage{comment} % Permite comentar secciones del código
\usepackage{marvosym} % Agrega símbolos adicionales
\usepackage{graphicx} % Permite insertar imágenes
\usepackage{mathptmx} % Fuente de texto matemática
\usepackage{amssymb} % Símbolos adicionales de matemáticas
\usepackage{lipsum} % Crea texto aleatorio
\usepackage{amsthm} % Teoremas y entornos de demostración
\usepackage{float} % Control de posiciones de figuras y tablas
\usepackage{rotating} % Rotación de elementos
\usepackage{multirow} % Celdas combinadas en tablas
\usepackage{tabularx} % Tablas con ancho de columna ajustable
\usepackage{mdframed} % Marcos alrededor de elementos flotantes

% Series de tiempo
\usepackage{booktabs}


% Configuración adicional

\makeatletter
\makeatother
\makeatletter
\makeatother
\makeatletter
\@ifpackageloaded{caption}{}{\usepackage{caption}}
\AtBeginDocument{%
\ifdefined\contentsname
  \renewcommand*\contentsname{Tabla de contenidos}
\else
  \newcommand\contentsname{Tabla de contenidos}
\fi
\ifdefined\listfigurename
  \renewcommand*\listfigurename{Listado de Figuras}
\else
  \newcommand\listfigurename{Listado de Figuras}
\fi
\ifdefined\listtablename
  \renewcommand*\listtablename{Listado de Tablas}
\else
  \newcommand\listtablename{Listado de Tablas}
\fi
\ifdefined\figurename
  \renewcommand*\figurename{Figura}
\else
  \newcommand\figurename{Figura}
\fi
\ifdefined\tablename
  \renewcommand*\tablename{Tabla}
\else
  \newcommand\tablename{Tabla}
\fi
}
\@ifpackageloaded{float}{}{\usepackage{float}}
\floatstyle{ruled}
\@ifundefined{c@chapter}{\newfloat{codelisting}{h}{lop}}{\newfloat{codelisting}{h}{lop}[chapter]}
\floatname{codelisting}{Listado}
\newcommand*\listoflistings{\listof{codelisting}{Listado de Listados}}
\makeatother
\makeatletter
\@ifpackageloaded{caption}{}{\usepackage{caption}}
\@ifpackageloaded{subcaption}{}{\usepackage{subcaption}}
\makeatother
\makeatletter
\@ifpackageloaded{tcolorbox}{}{\usepackage[skins,breakable]{tcolorbox}}
\makeatother
\makeatletter
\@ifundefined{shadecolor}{\definecolor{shadecolor}{rgb}{.97, .97, .97}}
\makeatother
\makeatletter
\makeatother
\makeatletter
\makeatother
\ifLuaTeX
\usepackage[bidi=basic]{babel}
\else
\usepackage[bidi=default]{babel}
\fi
\babelprovide[main,import]{spanish}
% get rid of language-specific shorthands (see #6817):
\let\LanguageShortHands\languageshorthands
\def\languageshorthands#1{}
\ifLuaTeX
  \usepackage{selnolig}  % disable illegal ligatures
\fi
\usepackage[]{biblatex}
\addbibresource{../../../../references.bib}
\IfFileExists{bookmark.sty}{\usepackage{bookmark}}{\usepackage{hyperref}}
\IfFileExists{xurl.sty}{\usepackage{xurl}}{} % add URL line breaks if available
\urlstyle{same} % disable monospaced font for URLs
\hypersetup{
  pdftitle={El capitalismo ¿Cáncer de nuestra era?},
  pdfauthor={Edison Achalma},
  pdflang={es},
  colorlinks=true,
  linkcolor={blue},
  filecolor={Maroon},
  citecolor={Blue},
  urlcolor={Blue},
  pdfcreator={LaTeX via pandoc}}

\title{El capitalismo ¿Cáncer de nuestra era?}
\usepackage{etoolbox}
\makeatletter
\providecommand{\subtitle}[1]{% add subtitle to \maketitle
  \apptocmd{\@title}{\par {\large #1 \par}}{}{}
}
\makeatother
\subtitle{Reflexiones sobre la pobreza y la desigualdad en un sistema
económico depredador e injusto.}
\author{Edison Achalma}
\date{2023-04-29}

\begin{document}
\maketitle
\ifdefined\Shaded\renewenvironment{Shaded}{\begin{tcolorbox}[breakable, boxrule=0pt, enhanced, borderline west={3pt}{0pt}{shadecolor}, interior hidden, sharp corners, frame hidden]}{\end{tcolorbox}}\fi

En 1987 el mundo tenía 1.5 billones de pobres (con menos de 1 dólar al
día) en 2015 son 1.9 billones. El índice internacional de pobreza se ha
modificado varias veces en nombre de buenas campañas de relaciones
públicas para el world bank.

El índice actual es 1.25 \$/día, ajustado a inflación es menor que en
ajustes anteriores. Mas allá de los defensores del capitalismo sean las
personas más ignorantes y ciegas de su privilegio de la modernidad,
nadie puede realmente vivir con 1.25 /día.

En USA el valor mínimo para sufrir malnutrición seria 4.05 \$/día. En
India el 60\% de las personas que viven con el dólar por día sufren
desnutrición. Porque esto es obsceno y barbárico y culpa de un modelo
económico abusivo, corrupto, depredador e injusto.

Porque el mundo produce 1.5x la cantidad de comida necesaria para
alimentar el mundo. Además, el 30\% de dicha comida se desperdicia
previa a su uso. Esto es solo una variable analizada fríamente sobre por
qué el capitalismo es el cáncer de nuestra era.

Y los que los defienden, los payasos más ignorantes de la historia.
Antes que me digas socialista, porque tu cerebro solo entiende el mundo
en blanco y negro. No tengo nostalgia del pasado ni quiero vivir en
Venezuela, Cuba, Rusia, China, Noruega o Finlandia.

Ese antagonismo asustador es para niños asustados con miedo al comunismo
como si fuera el monstruo debajo de tu cama. Usan esas historias de
países fracasados para hacerte creer que no existe alternativa (TINA:
there is no alternative).

El primer paso para crear una alternativa es admitir que la que tenemos
es un mierda que nos está destruyendo, física, mental y terrestremente.
Y que no tenemos ninguna antigua a la cual volver. Ojalá y más gente
abra los ojos antes que sea demasiado tarde.

Antes que hables del libre mercado, pregúntate desde tu lógica
libertaria por qué no sería legal la necrofilia o la expropiación de
órganos por empresas. (El capitalismo se puede poner muuuuucho peor).

\hypertarget{publicaciones-similares}{%
\section{Publicaciones Similares}\label{publicaciones-similares}}

Si te interesó este artículo, te recomendamos que explores otros blogs y
recursos relacionados que pueden ampliar tus conocimientos. Aquí te dejo
algunas sugerencias:

\begin{enumerate}
\def\labelenumi{\arabic{enumi}.}
\item
  \href{../2023-06-12-introducion-organizacion-industrial/index.qmd}{Introducción
  a organización industrial}
\item
  \href{../2023-06-13-empresa-como-organizacion/index.qmd}{La Empresa
  como Organización. Promoviendo Valores Cooperativos, Humanos y
  Sociales}
\item
  \href{../2023-06-13-sistemas-economicos/index.qmd}{Introducción a los
  Sistemas Económicos. Cómo se distribuyen los recursos y se producen}
\item
  \href{../2023-06-15-mercado-relevante-oi-cap-2/index.qmd}{El Mercado
  Relevante Industrial de Bienes y el Mercado Geográfico}
\item
  \href{../2023-06-16-concentracion-poder-oi-cap3/index.qmd}{Medidas de
  concentracion}
\item
  \href{../2023-06-17-estructura-mercado-oi-cap4/index.qmd}{Estructura
  de mercado}
\end{enumerate}

Esperamos que encuentres estas publicaciones igualmente interesantes y
útiles. ¡Disfruta de la lectura!


\printbibliography


\end{document}
