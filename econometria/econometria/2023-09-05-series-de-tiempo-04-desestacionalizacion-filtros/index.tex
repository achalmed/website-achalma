% Options for packages loaded elsewhere
\PassOptionsToPackage{unicode}{hyperref}
\PassOptionsToPackage{hyphens}{url}
\PassOptionsToPackage{dvipsnames,svgnames,x11names}{xcolor}
%
\documentclass[
  a4paper,
]{article}

\usepackage{amsmath,amssymb}
\usepackage{iftex}
\ifPDFTeX
  \usepackage[T1]{fontenc}
  \usepackage[utf8]{inputenc}
  \usepackage{textcomp} % provide euro and other symbols
\else % if luatex or xetex
  \usepackage{unicode-math}
  \defaultfontfeatures{Scale=MatchLowercase}
  \defaultfontfeatures[\rmfamily]{Ligatures=TeX,Scale=1}
\fi
\usepackage{lmodern}
\ifPDFTeX\else  
    % xetex/luatex font selection
\fi
% Use upquote if available, for straight quotes in verbatim environments
\IfFileExists{upquote.sty}{\usepackage{upquote}}{}
\IfFileExists{microtype.sty}{% use microtype if available
  \usepackage[]{microtype}
  \UseMicrotypeSet[protrusion]{basicmath} % disable protrusion for tt fonts
}{}
\makeatletter
\@ifundefined{KOMAClassName}{% if non-KOMA class
  \IfFileExists{parskip.sty}{%
    \usepackage{parskip}
  }{% else
    \setlength{\parindent}{0pt}
    \setlength{\parskip}{6pt plus 2pt minus 1pt}}
}{% if KOMA class
  \KOMAoptions{parskip=half}}
\makeatother
\usepackage{xcolor}
\usepackage[top=2.54cm,right=2.54cm,bottom=2.54cm,left=2.54cm]{geometry}
\setlength{\emergencystretch}{3em} % prevent overfull lines
\setcounter{secnumdepth}{-\maxdimen} % remove section numbering
% Make \paragraph and \subparagraph free-standing
\ifx\paragraph\undefined\else
  \let\oldparagraph\paragraph
  \renewcommand{\paragraph}[1]{\oldparagraph{#1}\mbox{}}
\fi
\ifx\subparagraph\undefined\else
  \let\oldsubparagraph\subparagraph
  \renewcommand{\subparagraph}[1]{\oldsubparagraph{#1}\mbox{}}
\fi


\providecommand{\tightlist}{%
  \setlength{\itemsep}{0pt}\setlength{\parskip}{0pt}}\usepackage{longtable,booktabs,array}
\usepackage{calc} % for calculating minipage widths
% Correct order of tables after \paragraph or \subparagraph
\usepackage{etoolbox}
\makeatletter
\patchcmd\longtable{\par}{\if@noskipsec\mbox{}\fi\par}{}{}
\makeatother
% Allow footnotes in longtable head/foot
\IfFileExists{footnotehyper.sty}{\usepackage{footnotehyper}}{\usepackage{footnote}}
\makesavenoteenv{longtable}
\usepackage{graphicx}
\makeatletter
\def\maxwidth{\ifdim\Gin@nat@width>\linewidth\linewidth\else\Gin@nat@width\fi}
\def\maxheight{\ifdim\Gin@nat@height>\textheight\textheight\else\Gin@nat@height\fi}
\makeatother
% Scale images if necessary, so that they will not overflow the page
% margins by default, and it is still possible to overwrite the defaults
% using explicit options in \includegraphics[width, height, ...]{}
\setkeys{Gin}{width=\maxwidth,height=\maxheight,keepaspectratio}
% Set default figure placement to htbp
\makeatletter
\def\fps@figure{htbp}
\makeatother

% Preámbulo
\usepackage{comment} % Permite comentar secciones del código
\usepackage{marvosym} % Agrega símbolos adicionales
\usepackage{graphicx} % Permite insertar imágenes
\usepackage{mathptmx} % Fuente de texto matemática
\usepackage{amssymb} % Símbolos adicionales de matemáticas
\usepackage{lipsum} % Crea texto aleatorio
\usepackage{amsthm} % Teoremas y entornos de demostración
\usepackage{float} % Control de posiciones de figuras y tablas
\usepackage{rotating} % Rotación de elementos
\usepackage{multirow} % Celdas combinadas en tablas
\usepackage{tabularx} % Tablas con ancho de columna ajustable
\usepackage{mdframed} % Marcos alrededor de elementos flotantes

% Series de tiempo
\usepackage{booktabs}


% Configuración adicional

\makeatletter
\makeatother
\makeatletter
\makeatother
\makeatletter
\@ifpackageloaded{caption}{}{\usepackage{caption}}
\AtBeginDocument{%
\ifdefined\contentsname
  \renewcommand*\contentsname{Tabla de contenidos}
\else
  \newcommand\contentsname{Tabla de contenidos}
\fi
\ifdefined\listfigurename
  \renewcommand*\listfigurename{Listado de Figuras}
\else
  \newcommand\listfigurename{Listado de Figuras}
\fi
\ifdefined\listtablename
  \renewcommand*\listtablename{Listado de Tablas}
\else
  \newcommand\listtablename{Listado de Tablas}
\fi
\ifdefined\figurename
  \renewcommand*\figurename{Figura}
\else
  \newcommand\figurename{Figura}
\fi
\ifdefined\tablename
  \renewcommand*\tablename{Tabla}
\else
  \newcommand\tablename{Tabla}
\fi
}
\@ifpackageloaded{float}{}{\usepackage{float}}
\floatstyle{ruled}
\@ifundefined{c@chapter}{\newfloat{codelisting}{h}{lop}}{\newfloat{codelisting}{h}{lop}[chapter]}
\floatname{codelisting}{Listado}
\newcommand*\listoflistings{\listof{codelisting}{Listado de Listados}}
\makeatother
\makeatletter
\@ifpackageloaded{caption}{}{\usepackage{caption}}
\@ifpackageloaded{subcaption}{}{\usepackage{subcaption}}
\makeatother
\makeatletter
\@ifpackageloaded{tcolorbox}{}{\usepackage[skins,breakable]{tcolorbox}}
\makeatother
\makeatletter
\@ifundefined{shadecolor}{\definecolor{shadecolor}{rgb}{.97, .97, .97}}
\makeatother
\makeatletter
\makeatother
\makeatletter
\makeatother
\ifLuaTeX
\usepackage[bidi=basic]{babel}
\else
\usepackage[bidi=default]{babel}
\fi
\babelprovide[main,import]{spanish}
% get rid of language-specific shorthands (see #6817):
\let\LanguageShortHands\languageshorthands
\def\languageshorthands#1{}
\ifLuaTeX
  \usepackage{selnolig}  % disable illegal ligatures
\fi
\usepackage[]{biblatex}
\addbibresource{../../../../references.bib}
\IfFileExists{bookmark.sty}{\usepackage{bookmark}}{\usepackage{hyperref}}
\IfFileExists{xurl.sty}{\usepackage{xurl}}{} % add URL line breaks if available
\urlstyle{same} % disable monospaced font for URLs
\hypersetup{
  pdftitle={Notas de Clase Series de Tiempo},
  pdfauthor={Edison Achalma},
  pdflang={es},
  colorlinks=true,
  linkcolor={blue},
  filecolor={Maroon},
  citecolor={Blue},
  urlcolor={Blue},
  pdfcreator={LaTeX via pandoc}}

\title{Notas de Clase Series de Tiempo}
\usepackage{etoolbox}
\makeatletter
\providecommand{\subtitle}[1]{% add subtitle to \maketitle
  \apptocmd{\@title}{\par {\large #1 \par}}{}{}
}
\makeatother
\subtitle{Descubre cómo seleccionar hardware, descargar la imagen ISO y
preparar los medios de instalación. Exploraremos opciones para probar o
instalar Linux en tu equipo.}
\author{Edison Achalma}
\date{2023-08-27}

\begin{document}
\maketitle
\ifdefined\Shaded\renewenvironment{Shaded}{\begin{tcolorbox}[interior hidden, borderline west={3pt}{0pt}{shadecolor}, enhanced, breakable, frame hidden, sharp corners, boxrule=0pt]}{\end{tcolorbox}}\fi

\hypertarget{desestacionalizaciuxf3n-y-filtrado-de-series}{%
\section{Desestacionalización y filtrado de
Series}\label{desestacionalizaciuxf3n-y-filtrado-de-series}}

\hypertarget{motivaciuxf3n}{%
\subsection{Motivación}\label{motivaciuxf3n}}

Existen múltiples enfoques para la desestacionalización de series.
Algunos modelos, por ejemplo, pueden estar basados en modelos ARIMA como
un conjunto de dummies. No obstante, en el caso partícular que
discutiremos en este curso, estará basado en un modelo ARIMA de la
serie. Este enfoque está basado en el modelo X11 de la oficina del censo
de Estados Unidos (Census Bureau) el cual es conocido como el modelo
X13-ARIMA-SEATS.\footnote{ La información y material respecto del modelo esta disponible en la dirección \url{https://www.census.gov/srd/www/x13as/}}
El modelo X13-ARIMA-SEATS es, como su nombre lo indica, la combinación
de un modelo ARIMA con componentes estacionales (por la traducción
literal de la palabra: seasonal).

Un modelo ARIMA estacional emplea la serie en diferencias y como
regresores los valores rezagados de las diferencia de la serie tantas
veces como procesos estacionales \(s\) existan en ésta, con el objeto de
remover los efectos aditivos de la estacionalidad. Sólo para entender
qué significa este mecanismo, recordemos que cuando se utiliza la
primera direncia de la serie respecto del periodo inmediato anterior se
remueve la tendencia. Por su parte, cuando se incluye la diferencia
respecto del mes \(s\) se está en el caso en que se modela la serie como
una media móvil en términos del rezago \(s\).

El modelo ARIMA estacional incluye como componentes autoregresivos y de
medias móviles a los valores rezagados de la serie en el periodo \(s\)
en diferencias. El ARIMA(p, d, q)(P, D, Q) estacional puede ser
expresado de la siguiente manera utilizando el operador rezago \(L\):

\begin{equation}\protect\hypertarget{eq-ARIMA_seas}{}{
\Theta_P(L^s) \theta_p(L) (1 - L^s)^D (1 - L)^d X_t = \Psi_Q(L^s) \psi_q(L) U_t
}\label{eq-ARIMA_seas}\end{equation}

Donde \(\Theta_P(.)\), \(\theta_p(.)\), \(\Psi_Q(.)\) y \(\psi_q(.)\)
son polinomios de \(L\) de orden \(P\), \(p\), \(Q\) y \(q\)
respectivamente. En general, la representación es de una serie no
estacionaria, aunque si \(D = d = 0\) y las raíces del polinomio
carácteristico (de los polinomios del lado izquierdo de la
(Ecuación~\ref{eq-ARIMA_seas}) todas son más grandes que 1 en valor
absoluto, el proceso modelado será estacionario.

Si bien es cierto que existen otras formas de modelar la
desestacionalización, como la modelación en diferencias con dummies para
identificar ciertos patrones regulares, en los algorimtos disponibles
como el X11 o X13-ARIMA-SEATS se emplea la formulación de la ecuación
(\ref{ARIMA_seas}). A continuación implementaremos la
desestacionalización de una serie.

Como ejemplo utilizaremos la serie del Índice Nacional de Precios al
Consumidor (INPC) en el periodo de enero de 2000 a julio de 2019.
Utilizando el Scrip Clase 11, diponible en el repositorio de GitHub,
podemos ver que la serie original del INPC y su ajuste estacional bajo
una metodología X13-ARIMA-SEATS son como se muestra en la Figura
(\ref{INPC_Adj}).

\begin{figure}
  \centering
    \includegraphics[width = 1.0 \textwidth]{INPC_Adj}
  \caption{Índice Nacional de Precios al Consumidor ($INPC_t$) y su serie desestacionalizada utilizando un proceso X13-ARIMA-SEATS.}
  \label{INPC_Adj}
\end{figure}

El mismo procesamiento puede ser seguido para todas las series que
busquemos analizar. En particular, en adelante, además del INPC que
incluimos en la lista, utilzaremos las siguientes series, así como su
versión desestacionalizada:

\begin{enumerate}
\def\labelenumi{\arabic{enumi}.}
\tightlist
\item
  Índice Nacional de Precios al Consumidor (base 2QJul2018 = 100),
  \(INPC_t\).
\item
  Tipo de Cambio FIX, \(TC_t\)
\item
  Tasa de rendimiento promedio mensual de los Cetes 28, en por ciento
  anual, \(CETE28_t\)
\item
  Indicador global de la actividad económica (base 2013 = 100),
  \(IGAE_t\)
\item
  Industrial Production Index o Índice de Producción Industrial de los
  Estados Unidos (base 2012 = 100), \(IPI_t\)
\end{enumerate}

\hypertarget{filtro-hodrick-prescott}{%
\subsection{Filtro Hodrick-Prescott}\label{filtro-hodrick-prescott}}

Como último tema de los procesos univariados y que no necesariamente
aplican a series estacionarias, a continuación desarrollaremos el
procedimiento conocido como filtro de Hodrick y Prescott (1997). El
trabajo de estos autores era determinar una técnica de regresión que
permitiera utilizar series agregadas o macroeconómicas para separarlas
en dos componentes: uno de ciclo de negocios y otro de tendencia. En su
trabajo orignal Hodrick y Prescott (1997) utilizaron datos trimestrales
de alguna series como el Gross Domestic Product (GNP, por sus siglas enn
Inglés), los agregados montearios M1, empleo, etc., de los Estados
Unidos que fueron observados posteriormente a la Segunda Guerra Mundial.

El marco conceptual de Hodrick y Prescott (1997) parte de suponer una
serie \(X_t\) que se puede descomponer en la suma de componente de
crecimiento tendencial, \(g_t\), y su componente de ciclio de negocios,
\(c_t\), de esta forma para \(t = 1, 2, \ldots, T\) tenemos que:

\begin{equation}\protect\hypertarget{eq-HP_Eq}{}{
X_t = g_t + c_t
}\label{eq-HP_Eq}\end{equation}

En la ecuación (\ref{HP_Eq}) se asume que el ajuste de la ruta seguida
por \(g_t\) es resultado de la suma de los cuadrados de su segunda
diferencia. En esa misma ecuación sumiremos que \(c_t\) son las
desviaciones de \(g_t\), las cuales en el largo plazo tienen una media
igual a cero (0). Por esta razón, se suele decir que el filtro de
Hodrick y Prescott represeta una una descomposición de la serie en su
componente de crecimiento natural y de sus desviaciones transitorias que
en promedio son cero, en el largo plazo.

Estas consideraciones que hemos mencionado señalan que el procesimiento
de Hodrick y Prescott (1997) implican resolver el siguiente problema
minimización para determinar cada uno de los componentes en que \(X_t\)
se puede descomponer:

\[
\min_{\{ g_t \}^T_{t = -1} } \left[ \sum^T_{t = 1} c^2_t + \lambda \sum^T_{t = 1} [ \Delta g_t - \Delta g_{t-1}]^2 \right]
\]

Donde \(\Delta g_t = g_t - g_{t-1}\) y
\(\Delta g_{t-1} = g_{t-1} - g_{t-2}\); \(c_t = X_t - g_t\), y el
parámetro \(\lambda\) es un número positivo que penaliza la variabilidad
en el crecimiento de las series. De acuerdo con el trabajo de Hodrick y
Prescott (1997) la constante \(\lambda\) debe tomar valores especificos
de acuerdo con la periodicidad de la series. Así, \(\lambda\) será:

\begin{enumerate}
\def\labelenumi{\arabic{enumi}.}
\tightlist
\item
  100 si la serie es de datos anuales
\item
  1,600 si la serie es de datos trimestrales
\item
  14,400 si la serie es de datos mensuales
\end{enumerate}

En resumen, podemos decir que el filtro de Hodrick y Prescott (1997) es
un algoritmo que mimiza las distancias o variaciones de la trayectoria
de largo plazo. De esta forma, determina una trayectoria estable de
largo plazo, por lo que las desviaciones respecto de esta trayectoria
serán componentes de ciclos de negocio o cambios transitorios (tanto
positivos como negativos).

A contiuación, ilustraremos el filtro de Hodrick y Prescott (1997) para
dos series desestacionalizadas: \(INPC_t\) y \(TC_t\). Las Figura
(\ref{INPC_HP}) y Figura (\ref{TC_HP}) muestran los resultados de la
implementación del filtro.

\begin{figure}
  \centering
    \includegraphics[width = 0.9 \textwidth]{INPC_HP}
  \caption{Descomposición del Índice Nacional de Precios al Consumidor ($INPC_t$) en su tendencia o trayectoria de largo plazo y su ciclo de corto plazo utilizando un filtro Hodrick y Prescott (1997).}
  \label{INPC_HP}
\end{figure}

\begin{figure}
  \centering
    \includegraphics[width = 0.9 \textwidth]{TC_HP}
  \caption{Descomposición del Tipo de Cambio FIX ($TC_t$) en su tendencia o trayectoria de largo plazo y su ciclo de corto plazo utilizando un filtro Hodrick y Prescott (1997).}
  \label{TC_HP}
\end{figure}


\printbibliography


\end{document}
