% Options for packages loaded elsewhere
\PassOptionsToPackage{unicode}{hyperref}
\PassOptionsToPackage{hyphens}{url}
\PassOptionsToPackage{dvipsnames,svgnames,x11names}{xcolor}
%
\documentclass[
  a4paper,
]{article}

\usepackage{amsmath,amssymb}
\usepackage{iftex}
\ifPDFTeX
  \usepackage[T1]{fontenc}
  \usepackage[utf8]{inputenc}
  \usepackage{textcomp} % provide euro and other symbols
\else % if luatex or xetex
  \usepackage{unicode-math}
  \defaultfontfeatures{Scale=MatchLowercase}
  \defaultfontfeatures[\rmfamily]{Ligatures=TeX,Scale=1}
\fi
\usepackage{lmodern}
\ifPDFTeX\else  
    % xetex/luatex font selection
\fi
% Use upquote if available, for straight quotes in verbatim environments
\IfFileExists{upquote.sty}{\usepackage{upquote}}{}
\IfFileExists{microtype.sty}{% use microtype if available
  \usepackage[]{microtype}
  \UseMicrotypeSet[protrusion]{basicmath} % disable protrusion for tt fonts
}{}
\makeatletter
\@ifundefined{KOMAClassName}{% if non-KOMA class
  \IfFileExists{parskip.sty}{%
    \usepackage{parskip}
  }{% else
    \setlength{\parindent}{0pt}
    \setlength{\parskip}{6pt plus 2pt minus 1pt}}
}{% if KOMA class
  \KOMAoptions{parskip=half}}
\makeatother
\usepackage{xcolor}
\usepackage[top=2.54cm,right=2.54cm,bottom=2.54cm,left=2.54cm]{geometry}
\setlength{\emergencystretch}{3em} % prevent overfull lines
\setcounter{secnumdepth}{-\maxdimen} % remove section numbering
% Make \paragraph and \subparagraph free-standing
\makeatletter
\ifx\paragraph\undefined\else
  \let\oldparagraph\paragraph
  \renewcommand{\paragraph}{
    \@ifstar
      \xxxParagraphStar
      \xxxParagraphNoStar
  }
  \newcommand{\xxxParagraphStar}[1]{\oldparagraph*{#1}\mbox{}}
  \newcommand{\xxxParagraphNoStar}[1]{\oldparagraph{#1}\mbox{}}
\fi
\ifx\subparagraph\undefined\else
  \let\oldsubparagraph\subparagraph
  \renewcommand{\subparagraph}{
    \@ifstar
      \xxxSubParagraphStar
      \xxxSubParagraphNoStar
  }
  \newcommand{\xxxSubParagraphStar}[1]{\oldsubparagraph*{#1}\mbox{}}
  \newcommand{\xxxSubParagraphNoStar}[1]{\oldsubparagraph{#1}\mbox{}}
\fi
\makeatother


\providecommand{\tightlist}{%
  \setlength{\itemsep}{0pt}\setlength{\parskip}{0pt}}\usepackage{longtable,booktabs,array}
\usepackage{calc} % for calculating minipage widths
% Correct order of tables after \paragraph or \subparagraph
\usepackage{etoolbox}
\makeatletter
\patchcmd\longtable{\par}{\if@noskipsec\mbox{}\fi\par}{}{}
\makeatother
% Allow footnotes in longtable head/foot
\IfFileExists{footnotehyper.sty}{\usepackage{footnotehyper}}{\usepackage{footnote}}
\makesavenoteenv{longtable}
\usepackage{graphicx}
\makeatletter
\def\maxwidth{\ifdim\Gin@nat@width>\linewidth\linewidth\else\Gin@nat@width\fi}
\def\maxheight{\ifdim\Gin@nat@height>\textheight\textheight\else\Gin@nat@height\fi}
\makeatother
% Scale images if necessary, so that they will not overflow the page
% margins by default, and it is still possible to overwrite the defaults
% using explicit options in \includegraphics[width, height, ...]{}
\setkeys{Gin}{width=\maxwidth,height=\maxheight,keepaspectratio}
% Set default figure placement to htbp
\makeatletter
\def\fps@figure{htbp}
\makeatother

% Preámbulo
\usepackage{comment} % Permite comentar secciones del código
\usepackage{marvosym} % Agrega símbolos adicionales
\usepackage{graphicx} % Permite insertar imágenes
\usepackage{mathptmx} % Fuente de texto matemática
\usepackage{amssymb} % Símbolos adicionales de matemáticas
\usepackage{lipsum} % Crea texto aleatorio
\usepackage{amsthm} % Teoremas y entornos de demostración
\usepackage{float} % Control de posiciones de figuras y tablas
\usepackage{rotating} % Rotación de elementos
\usepackage{multirow} % Celdas combinadas en tablas
\usepackage{tabularx} % Tablas con ancho de columna ajustable
\usepackage{mdframed} % Marcos alrededor de elementos flotantes

% Series de tiempo
\usepackage{booktabs}


% Configuración adicional

\makeatletter
\@ifpackageloaded{caption}{}{\usepackage{caption}}
\AtBeginDocument{%
\ifdefined\contentsname
  \renewcommand*\contentsname{Tabla de contenidos}
\else
  \newcommand\contentsname{Tabla de contenidos}
\fi
\ifdefined\listfigurename
  \renewcommand*\listfigurename{Listado de Figuras}
\else
  \newcommand\listfigurename{Listado de Figuras}
\fi
\ifdefined\listtablename
  \renewcommand*\listtablename{Listado de Tablas}
\else
  \newcommand\listtablename{Listado de Tablas}
\fi
\ifdefined\figurename
  \renewcommand*\figurename{Figura}
\else
  \newcommand\figurename{Figura}
\fi
\ifdefined\tablename
  \renewcommand*\tablename{Tabla}
\else
  \newcommand\tablename{Tabla}
\fi
}
\@ifpackageloaded{float}{}{\usepackage{float}}
\floatstyle{ruled}
\@ifundefined{c@chapter}{\newfloat{codelisting}{h}{lop}}{\newfloat{codelisting}{h}{lop}[chapter]}
\floatname{codelisting}{Listado}
\newcommand*\listoflistings{\listof{codelisting}{Listado de Listados}}
\makeatother
\makeatletter
\makeatother
\makeatletter
\@ifpackageloaded{caption}{}{\usepackage{caption}}
\@ifpackageloaded{subcaption}{}{\usepackage{subcaption}}
\makeatother
\ifLuaTeX
\usepackage[bidi=basic]{babel}
\else
\usepackage[bidi=default]{babel}
\fi
\babelprovide[main,import]{spanish}
% get rid of language-specific shorthands (see #6817):
\let\LanguageShortHands\languageshorthands
\def\languageshorthands#1{}
\ifLuaTeX
  \usepackage{selnolig}  % disable illegal ligatures
\fi
\usepackage[]{biblatex}
\addbibresource{../../../references.bib}
\usepackage{bookmark}

\IfFileExists{xurl.sty}{\usepackage{xurl}}{} % add URL line breaks if available
\urlstyle{same} % disable monospaced font for URLs
\hypersetup{
  pdftitle={Notas de Clase Series de Tiempo},
  pdfauthor={Edison Achalma},
  pdflang={es},
  colorlinks=true,
  linkcolor={blue},
  filecolor={Maroon},
  citecolor={Blue},
  urlcolor={Blue},
  pdfcreator={LaTeX via pandoc}}

\title{Notas de Clase Series de Tiempo}
\usepackage{etoolbox}
\makeatletter
\providecommand{\subtitle}[1]{% add subtitle to \maketitle
  \apptocmd{\@title}{\par {\large #1 \par}}{}{}
}
\makeatother
\subtitle{Descubre cómo seleccionar hardware, descargar la imagen ISO y
preparar los medios de instalación. Exploraremos opciones para probar o
instalar Linux en tu equipo.}
\author{Edison Achalma}
\date{2023-08-27}

\begin{document}
\maketitle

\section{\texorpdfstring{Modelos multivariados de volatilidad:
\(M - ARCH\) y
\(M - GARCH\)}{Modelos multivariados de volatilidad: M - ARCH y M - GARCH}}\label{modelos-multivariados-de-volatilidad-m---arch-y-m---garch}

\chapter{Cointegración}

Hasta ahora en el curso hemos usado el supuesto de que las series son
estacionarias para el conjunto de técnicas \(ARMA(p,q)\) y \(VAR(p)\).
No obstante, dado que relajamos el supuesto de estacionariedad
(incluyendo la estacionariedad en varianza) y que establecimos una serie
de pruebas para determinar cuando una serie es estadísticamente
estacionaria, ahora podemos plantear una técnica llamada Cointegración.
Para esta técnica consideraremos sólo series que son \(I(1)\) y
reconoceremos que se originó con los trabajos de Engle y Granger (1987),
Stock (1987) y Johansen (1988).

\subsection{Definición y propiedades del proceso de
cointegración}\label{definiciuxf3n-y-propiedades-del-proceso-de-cointegraciuxf3n}

Cointegración puede ser caracterizada o definida en palabras sencillas
como que dos o más variables tienen una relación común estable en el
largo plazo. Es decir, estas no suelen tomar caminos o trayectorias
diferentes, excepto por periodos de tiempo transitorios y eventuales. A
continuación, utilizaremos la definición de Engle y Grnager (1984) de
cointegración.

Sea \(\mathbf{Y}\) un vector de k-series de tiempo, decimos que los
elementos en \(\mathbf{Y}\) están cointegrados en un orden (d, c), es
decir, \(\mathbf{Y} \sim CI(d, c)\), si todos los elementos de
\(\mathbf{Y}\) son series integradas de orden d, I(d), y si existe al
menos una combinación lineal no trivial \(\mathbf{Z}\) de esas variables
que es de orden I(d - c), donde \(d \geq c > 0\), si y sólo si: \[
    \boldsymbol{\beta}_i' \mathbf{Y}_t = \mathbf{Y}_{it} \sim I(d-c)
\]

Donde \(i = 1, 2, \ldots, r\) y \(r < k\).

A los diferentes vectores \(\boldsymbol{\beta}_i\) se les denomina como
vectores de cointegración. El rango de la matriz de vectores de
cointegración \(r\) es el número de vectores de cointegración
linealmente independientes. En general diremos que los vectores de la
matriz de cointegración \(\boldsymbol{\beta}\) tendrá la forma de: \[
    \boldsymbol{\beta}' \mathbf{Y}_t = \mathbf{Z}_t
\]

Antes de continuar hagamos algunas observaciones. Si todas las variables
de \(\mathbf{Y}\) son I(1) y \(0 \leq r < k\), diremos que las series no
cointegran si \(r = 0\). Si esto pasa, entonces, como demostraremos más
adelante, la mejor opción será estimar un modelo VAR(p) en diferencias.
Adicionalmente, asumiremos que \(c = d = 1\), por lo que la relación de
cointegración, en su caso, generará combinaciones lineales
\(\mathbf{Z}\) estacionarias.

\subsection{Cointegración para modelos de más de una ecuación o para
modelos basados en Vectores
Autoregresivos}\label{cointegraciuxf3n-para-modelos-de-muxe1s-de-una-ecuaciuxf3n-o-para-modelos-basados-en-vectores-autoregresivos}

Sean \(Y_1, Y_2, \ldots, Y_k\) son series que forman \(\mathbf{Y}\) y
que todas son I(1), entonces los siguientes casos son posibles:

\begin{enumerate}
    \item Si $r = 1$ entonces se trata de un caso de cointegración de Granger.
    \item Si $r \geq 1$ entonces se trata de un caso de cointegración múltiple de Johansen.
\end{enumerate}

Por lo anterior, en este curso analizaremos el caso de Cointegración de
Johansen. Ahora plantearemos la forma de estimar el proceso de
cointgración. El primer paso para ello es determinar un modelo VAR(p)
con las k-series no estacionarias (series en niveles). Alegimos el valor
de \(p\) mediante el uso de los criterios de información. De esta forma
tendremos una especificación similar a: \[
    \mathbf{Y}_t = \sum_{j=1}^p \mathbf{A}_j \mathbf{Y}_{t-j} + \mathbf{D}_t + \mathbf{U}_t
    \label{VAR_CI}
\]

Donde \(\mathbf{U}_t\) es un término de error k-dimensional puramente
aleatorio; \(\mathbf{D}_t\) contiene los compeoentes deterministicos de
constante y tendencia, y \(\mathbf{A}_i\), \(i = 1, 2, \ldots, p\), son
matrices de \(k \times k\) coeficientes. Notemos que el VAR(p)
involucrado en este caso, a diferencia del VAR anteriormente estudiado,
puede incluir un término de tendencia. Esto en razón de que hemos
relajado el concepto de estacionariedad.

Si reescribimos la ecuación (\ref{VAR_CI}) en su forma de Vector
Corrector de Errores (VEC, por sus siglas en inglés) tenemos:
\begin{eqnarray}
    \mathbf{Y}_t - \mathbf{Y}_{t-1} & = & \Delta \mathbf{Y}_t \nonumber \\
    & = & \sum_{j=1}^p \mathbf{A}_j \mathbf{Y}_{t-j} + \mathbf{D}_t - \mathbf{Y}_{t-1} + \mathbf{U}_t \nonumber \\
    & = & (\mathbf{A}_1 - \mathbf{I}) \mathbf{Y}_{t-1} + \mathbf{A}_2 \mathbf{Y}_{t-2} + \ldots + \mathbf{A}_p \mathbf{Y}_{t-p} + \mathbf{D}_t + \mathbf{U}_t \nonumber \\
    & = & \left( \sum_{j=1}^{p} \mathbf{A}_j - \mathbf{I} \right) \mathbf{Y}_{t-1} + \sum_{j=1}^{p-1} \mathbf{A}^*_j \Delta \mathbf{Y}_{t-j} + \mathbf{D}_t \mathbf{U}_t \nonumber \\
    & = & - \left( \mathbf{I} - \sum_{j=1}^{p} \mathbf{A}_j \right) \mathbf{Y}_{t-1} + \sum_{j=1}^{p-1} \mathbf{A}^*_j \Delta \mathbf{Y}_{t-j} + \mathbf{D}_t \mathbf{U}_t \nonumber \\
    \Delta \mathbf{Y}_t & = & - \Pi \mathbf{Y}_{t-1} + \sum_{j=1}^{p-1} \mathbf{A}^*_j \Delta \mathbf{Y}_{t-j} + \mathbf{D}_t + \mathbf{U}_t
    \label{VAR_VEC}
\end{eqnarray}

Donde \(\mathbf{A}_j^* = - \sum_{i=j+1}^p \mathbf{A}_i\),
\(i = 1, 2, \ldots, p-1\), y la matriz \(\Pi\) representa todas las
relaciones de largo plazo entre las variables, por lo que la matriz es
de rango completo \(k \times k\). Por lo tanto, tenemos que dicha matriz
en la ecuación (\ref{VAR_VEC}) se puede factorizar como: \[
    \Pi_{(k \times k)} = \Gamma_{(k \times r)} \boldsymbol{\beta}_{(r \times k)}'
    \label{Pi_Matrix}
\]

Donde \(\boldsymbol{\beta}_{(r \times k)}' \mathbf{Y}_{t-1}\) son \(r\)
combinaciones linealmente independientes que son estacionarias.

Dada la ecuación (\ref{VAR_VEC}) podemos establecer la aproximación de
Johansen (1988) que se realiza mediante una estimación por Máxima
Verosimilitud de la ecuación: \[
    \Delta \mathbf{Y}_t + \Gamma \boldsymbol{\beta}' \mathbf{Y}_{t-1} = \sum_{j=1}^{p-1} \mathbf{A}^*_j \Delta \mathbf{Y}_{t-j} + \mathbf{D}_t + \mathbf{U}_t
\]

Donde una vez estimado el sistema: \[
    \boldsymbol{\beta} = [v_1, v_2, \ldots, v_r]
\]

Cada \(v_i\), \(i = 1, 2, \ldots, r\), es un vector propio que asociado
a los \(r\) valores propios positivos, mismos que estás asociados con la
prueba de hipótesis de cointegración. Dicha hipótesis está basada en dos
estadísticas con las que se determina el rango \(r\) de \(\Pi\):

\begin{enumerate}
    \item Prueba de Traza:
    
    $H_0 :$ Existen al menos $r$ valores propios positivos o Existen al menos $r$ relaciones de largo plazo estacionarias.
    
    \item Prueba del valor propio máximo o $\lambda_{max}$:
    
    $H_0 :$ Existen $r$ valores propios positivos o Existen $r$ relaciones de largo plazo estacionarias.
\end{enumerate}

\subsection{Ejemplo de cointegración}\label{ejemplo-de-cointegraciuxf3n}

Para ejemplificar el procedimiento de cointegración utilizaremos las
series de INPC, Tipo de Cambio, rendimiento de los Cetes a 28 días, IGAE
e Índice de Producción Industrial de Estados Unidos. Quizá el marco
teórico de la relación entre las variables no sea del todo correcta,
pero dejando de lado ese problema estimaremos si las 5 series
cointegran.

En el Scrip Clase 17 de la capeta de GoogleDrive se encuentra el
desarrollo de este ejemplo. Por principio, probaremos que todas las
series son I(1), lo cual es cierto (ver Scrip para mayores detalles). En
la Figura (\ref{G_Cointegracion}) se muestran las series en niveles y en
diferencias, con lo cual ilustramos como es viable que las series sean
I(1).

\begin{figure}
  \centering
    \includegraphics[width = 1.0\textwidth]{G_Cointegracion}
  \caption{Series en niveles (logatirmos) y en diferencias logarítmicas para la prueba de Cointegración.}
  \label{G_Cointegracion}
\end{figure}

Posteriormente, determinamos cuál es el orden adecuado de un VAR(p) en
niveles. En el Cuadro (\ref{Select_VAR_VEC}) mostramos los resultados de
los criterios de información para determinar el número de rezagos
óptimos, el cual resultó en \(p = 3\) para los criterios AIC y FPE,
\(p = 2\) para el criterio HQ y \(p = 1\) para el citerio SC. Por lo
tanto decidiremos utilizar un VAR(3) con tendencia y constante.

\begin{table}
\centering
\begin{tabular}{| c | c | c | c | c |}
\hline
    Rezagos & AIC & HQ & SC & FPE \\
\hline
    1 & -4.606707e+01 & -4.585260e+01 & -4.553568e+01 & 9.848467e-21 \\
    2 & -4.643287e+01 & -4.606521e+01 & -4.552191e+01 & 6.834064e-21 \\
    3 & -4.647783e+01 & -4.595697e+01 & -4.518730e+01 & 6.539757e-21 \\
    4 & -4.645834e+01 & -4.578428e+01 & -4.478824e+01 & 6.679778e-21 \\
    \vdots & \vdots & \vdots & \vdots & \vdots \\
\hline
    \multicolumn{5}{ l }{Nota: Se reporta el valor de los criterios de información.} \\
\end{tabular}
\caption{Criterios de información para diferentes especificaciones de modelos VAR(p) con término constante y tendencia de la series $LINPC_t$, $LTC_t$, $LCETE28_t$, $LIGAE_t$ y $LIPI_t$.}

\end{table}

El mismo número de rezagos los utilizaremos para probar la Cointegración
ya sea por una estadística de la Traza o por una de el máximo valor
propio. En el Scrip llamado Clase 18 que se encuentra en la carpeta de
GoogleDrive se ubican todos los resultados citados acontinuación.
Derivado de la exploración de los resultados sólo matraremos uno de los
caso en que las series cointegran y sólo para el caso de la prueba de la
traza. En el Cuadro (\ref{Traza_Test}) reportamos los resultados del
Test de Cointegración para un modelo con 3 rezagos.

\begin{table}
\centering
\begin{tabular}{| c | c | c | c | c |}
\hline
    r $\leq$ & Estadística & 10\% & 5\% & 1\% \\
\hline
    4 & 3.38 & 10.49 & 12.25 & 16.26 \\
    3 & 17.09 & 22.76 & 25.32 & 30.45 \\
    2 & 31.66 & 39.06 & 42.44 & 48.45 \\
    1 & 56.88 & 59.14 & 62.99 & 70.05 \\
    0 & 89.69 & 83.20 & 87.31 & 96.58 \\
\hline
\end{tabular}
\caption{Criterios de información para diferentes especificaciones de modelos VAR(p) con tendencia de la series $LINPC_t$, $LTC_t$, $LCETE28_t$, $LIGAE_t$ y $LIPI_t$.}

\end{table}

Los resutados del Cuadro (\ref{Traza_Test}) indican que acptamos la
hipótesis nula para el caso de \(r \leq 1\) al \(5\%\), por lo que
podemos concluir que existe evidencia estadística para probar que
existen al menos 1 vector de cointegración. Por lo que dicho vector es:
\[
    \boldsymbol{\beta} = \left[ 
    \begin{matrix}
    1.00000000 \\
    0.151162436 \\
    -0.042650912 \\
    0.163804862 \\
    0.229295743 \\
    -0.004350646 \\
    \end{matrix} \right]
\]

Donde el vector esta normalizado para la serie \(LINPC_t\), por lo que
concluímos que la relación de largo plazo que encontramos cointegra
estará dada por: \begin{eqnarray*}
    LINPC_t & = & -0.151162436 LTC_t + 0.042650912 LCETE28_t \\
    &  & - 0.163804862 LIGAE_t - 0.229295743 LIPI_t \\
    &  & + 0.004350646 t
\end{eqnarray*}

Considerando lo anterior, podemos determinar \(\hat{U}_t\) para esta
ecuación de cointegración. En la Figura (\ref{G_U_Cointegration})
mostramos los residuales estimados. Derivado de la impección visual
parecería que estos no son estacionarios, condición que debería ser
cierta. De esta forma, una prueba deseable es aplicar todas la pruebas
de raíces unitarias a esta serie para mostrar que es I(0). En el Scrip
llamado Clase 18 en la carpeta de GoogleDrive se muestran algunas
pruebas sobre esta serie y se encuentra que es posible que no sea
estacionaria.

\begin{figure}
  \centering
    \includegraphics[width = 1.0\textwidth]{G_U_Cointegration}
  \caption{Residuales estimados de la ecuación de cointegración.}
  \label{G_U_Cointegration}
\end{figure}


\printbibliography


\end{document}
