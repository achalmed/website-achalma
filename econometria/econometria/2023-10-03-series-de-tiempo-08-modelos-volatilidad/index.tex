% Options for packages loaded elsewhere
\PassOptionsToPackage{unicode}{hyperref}
\PassOptionsToPackage{hyphens}{url}
\PassOptionsToPackage{dvipsnames,svgnames,x11names}{xcolor}
%
\documentclass[
  a4paper,
]{article}

\usepackage{amsmath,amssymb}
\usepackage{iftex}
\ifPDFTeX
  \usepackage[T1]{fontenc}
  \usepackage[utf8]{inputenc}
  \usepackage{textcomp} % provide euro and other symbols
\else % if luatex or xetex
  \usepackage{unicode-math}
  \defaultfontfeatures{Scale=MatchLowercase}
  \defaultfontfeatures[\rmfamily]{Ligatures=TeX,Scale=1}
\fi
\usepackage{lmodern}
\ifPDFTeX\else  
    % xetex/luatex font selection
\fi
% Use upquote if available, for straight quotes in verbatim environments
\IfFileExists{upquote.sty}{\usepackage{upquote}}{}
\IfFileExists{microtype.sty}{% use microtype if available
  \usepackage[]{microtype}
  \UseMicrotypeSet[protrusion]{basicmath} % disable protrusion for tt fonts
}{}
\makeatletter
\@ifundefined{KOMAClassName}{% if non-KOMA class
  \IfFileExists{parskip.sty}{%
    \usepackage{parskip}
  }{% else
    \setlength{\parindent}{0pt}
    \setlength{\parskip}{6pt plus 2pt minus 1pt}}
}{% if KOMA class
  \KOMAoptions{parskip=half}}
\makeatother
\usepackage{xcolor}
\usepackage[top=2.54cm,right=2.54cm,bottom=2.54cm,left=2.54cm]{geometry}
\setlength{\emergencystretch}{3em} % prevent overfull lines
\setcounter{secnumdepth}{-\maxdimen} % remove section numbering
% Make \paragraph and \subparagraph free-standing
\makeatletter
\ifx\paragraph\undefined\else
  \let\oldparagraph\paragraph
  \renewcommand{\paragraph}{
    \@ifstar
      \xxxParagraphStar
      \xxxParagraphNoStar
  }
  \newcommand{\xxxParagraphStar}[1]{\oldparagraph*{#1}\mbox{}}
  \newcommand{\xxxParagraphNoStar}[1]{\oldparagraph{#1}\mbox{}}
\fi
\ifx\subparagraph\undefined\else
  \let\oldsubparagraph\subparagraph
  \renewcommand{\subparagraph}{
    \@ifstar
      \xxxSubParagraphStar
      \xxxSubParagraphNoStar
  }
  \newcommand{\xxxSubParagraphStar}[1]{\oldsubparagraph*{#1}\mbox{}}
  \newcommand{\xxxSubParagraphNoStar}[1]{\oldsubparagraph{#1}\mbox{}}
\fi
\makeatother


\providecommand{\tightlist}{%
  \setlength{\itemsep}{0pt}\setlength{\parskip}{0pt}}\usepackage{longtable,booktabs,array}
\usepackage{calc} % for calculating minipage widths
% Correct order of tables after \paragraph or \subparagraph
\usepackage{etoolbox}
\makeatletter
\patchcmd\longtable{\par}{\if@noskipsec\mbox{}\fi\par}{}{}
\makeatother
% Allow footnotes in longtable head/foot
\IfFileExists{footnotehyper.sty}{\usepackage{footnotehyper}}{\usepackage{footnote}}
\makesavenoteenv{longtable}
\usepackage{graphicx}
\makeatletter
\def\maxwidth{\ifdim\Gin@nat@width>\linewidth\linewidth\else\Gin@nat@width\fi}
\def\maxheight{\ifdim\Gin@nat@height>\textheight\textheight\else\Gin@nat@height\fi}
\makeatother
% Scale images if necessary, so that they will not overflow the page
% margins by default, and it is still possible to overwrite the defaults
% using explicit options in \includegraphics[width, height, ...]{}
\setkeys{Gin}{width=\maxwidth,height=\maxheight,keepaspectratio}
% Set default figure placement to htbp
\makeatletter
\def\fps@figure{htbp}
\makeatother

% Preámbulo
\usepackage{comment} % Permite comentar secciones del código
\usepackage{marvosym} % Agrega símbolos adicionales
\usepackage{graphicx} % Permite insertar imágenes
\usepackage{mathptmx} % Fuente de texto matemática
\usepackage{amssymb} % Símbolos adicionales de matemáticas
\usepackage{lipsum} % Crea texto aleatorio
\usepackage{amsthm} % Teoremas y entornos de demostración
\usepackage{float} % Control de posiciones de figuras y tablas
\usepackage{rotating} % Rotación de elementos
\usepackage{multirow} % Celdas combinadas en tablas
\usepackage{tabularx} % Tablas con ancho de columna ajustable
\usepackage{mdframed} % Marcos alrededor de elementos flotantes

% Series de tiempo
\usepackage{booktabs}


% Configuración adicional

\makeatletter
\@ifpackageloaded{caption}{}{\usepackage{caption}}
\AtBeginDocument{%
\ifdefined\contentsname
  \renewcommand*\contentsname{Tabla de contenidos}
\else
  \newcommand\contentsname{Tabla de contenidos}
\fi
\ifdefined\listfigurename
  \renewcommand*\listfigurename{Listado de Figuras}
\else
  \newcommand\listfigurename{Listado de Figuras}
\fi
\ifdefined\listtablename
  \renewcommand*\listtablename{Listado de Tablas}
\else
  \newcommand\listtablename{Listado de Tablas}
\fi
\ifdefined\figurename
  \renewcommand*\figurename{Figura}
\else
  \newcommand\figurename{Figura}
\fi
\ifdefined\tablename
  \renewcommand*\tablename{Tabla}
\else
  \newcommand\tablename{Tabla}
\fi
}
\@ifpackageloaded{float}{}{\usepackage{float}}
\floatstyle{ruled}
\@ifundefined{c@chapter}{\newfloat{codelisting}{h}{lop}}{\newfloat{codelisting}{h}{lop}[chapter]}
\floatname{codelisting}{Listado}
\newcommand*\listoflistings{\listof{codelisting}{Listado de Listados}}
\makeatother
\makeatletter
\makeatother
\makeatletter
\@ifpackageloaded{caption}{}{\usepackage{caption}}
\@ifpackageloaded{subcaption}{}{\usepackage{subcaption}}
\makeatother
\ifLuaTeX
\usepackage[bidi=basic]{babel}
\else
\usepackage[bidi=default]{babel}
\fi
\babelprovide[main,import]{spanish}
% get rid of language-specific shorthands (see #6817):
\let\LanguageShortHands\languageshorthands
\def\languageshorthands#1{}
\ifLuaTeX
  \usepackage{selnolig}  % disable illegal ligatures
\fi
\usepackage[]{biblatex}
\addbibresource{../../../references.bib}
\usepackage{bookmark}

\IfFileExists{xurl.sty}{\usepackage{xurl}}{} % add URL line breaks if available
\urlstyle{same} % disable monospaced font for URLs
\hypersetup{
  pdftitle={Notas de Clase Series de Tiempo},
  pdfauthor={Edison Achalma},
  pdflang={es},
  colorlinks=true,
  linkcolor={blue},
  filecolor={Maroon},
  citecolor={Blue},
  urlcolor={Blue},
  pdfcreator={LaTeX via pandoc}}

\title{Notas de Clase Series de Tiempo}
\usepackage{etoolbox}
\makeatletter
\providecommand{\subtitle}[1]{% add subtitle to \maketitle
  \apptocmd{\@title}{\par {\large #1 \par}}{}{}
}
\makeatother
\subtitle{Descubre cómo seleccionar hardware, descargar la imagen ISO y
preparar los medios de instalación. Exploraremos opciones para probar o
instalar Linux en tu equipo.}
\author{Edison Achalma}
\date{2023-08-27}

\begin{document}
\maketitle

\section{Modelos Univariados y Multivariados de
Volatilidad}\label{modelos-univariados-y-multivariados-de-volatilidad}

\subsection{Modelos ARCH y GARCH
Univariados}\label{modelos-arch-y-garch-univariados}

Estos modelos de Heterocedásticidad Condicional Autoregresiva (ARCH, por
sus siglas en inglés) y modelos Heterocedásticidad Condicional
Autoregresiva Generalizados (GARCH, por sus siglas en inglés) tienen la
característica de modelar situaciones como las que ilustra la Figutra
\ref{G_ARCH_Return}. Es decir: 1) existen zonas donde la variación de
los datos es mayor y zonas donde la variación es más estable--a estas
situaciones se les conoce como de variabilidad por clúster--, y 2) los
datos corresponden a innformación de alta frecuencia.

\begin{figure}
  \centering
    \includegraphics[width = 1.0 \textwidth]{G_ARCH_Return}
  \caption{ Rendimientos (diferenccias logarítmicas) de tres acciones seleccionadas: Apple, Pfizer, Tesla, enero 2011 a noviembre de 2020}
  \label{G_ARCH_Return}
\end{figure}

Para plantear el modelo supongamos --por simplicidad-- que hemos
construido y estimado un modelo AR(1). Es decir, asumamos que el proceso
subyacente para la media condicional está dada por: \[
    X_t = a_0 + a_1 X_{t-1} + U_t
\]

Donde \(\abs{a_1} < 1\) para garantizar la convergencia del proceso en
el largo plazo, en el cual: \begin{eqnarray*}
    \mathbb{E}[X_t] & = & \frac{a_0 }{1 - a_1} = \mu \\
    Var[X_t] & = & \frac{\sigma^2}{1 - a_1^2}
\end{eqnarray*}

Ahora, supongamos que este tipo de modelos pueden ser extendidos y
generalizados a un modelo ARMA(p, q), que incluya otras variables
exogénas. Denotemos a como \(\mathbf{Z}_t\) al conjunto que incluye los
componentes AR, MA y variables exogénas que pueden explicar a \(X_t\) de
forma que el proceso estará dado por: \[
    X_t = \mathbf{Z}_t \boldsymbol{\beta} + U_t
\]

Donde \(U_t\) es un proceso estacionario que representa el error
asociado a un proceso ARMA(p, q) y donde siguen diendo válidos los
supuestos: \begin{eqnarray*}
    \mathbb{E}[U_t] & = & 0 \\
    Var[U_t^2] & = & \sigma^2
\end{eqnarray*}

No obstante, en este caso podemos suponer que existe autocorrelación en
el término de error que puede ser capturada por un porceso similar a uno
de medias móviles (MA) dado por: \[
    U_t^2 = \gamma_0 + \gamma_1 U_{t-1}^2 + \gamma_2 U_{t-2}^2 + \ldots + \gamma_q U_{t-q}^2 + \nu_t
\]

Donde \(\nu_t\) es un ruido blanco y
\(U_{t-i} = X_{t-i} - \mathbf{Z}_{t-i} \boldsymbol{\beta}\), \$i = 1, 2
,\ldots \$. Si bien los procesos son estacionarios por los supuestos
antes enunciados, la varianza condicional estará dada por:
\begin{eqnarray*}
    \sigma^2_{t | t-1} & = & Var[ U_t | \Omega_{t-1} ] \\
    & = & \mathbb{E}[ U^2_t | \Omega_{t-1} ]
\end{eqnarray*}

Donde \(\Omega_{t-1} = \{U_{t-1}, U_{t-2}, \ldots \}\) es el conjunto de
toda la información pasada de \(U_t\) y observada hasta el momento
\(t-1\), por lo que: \begin{equation*}
    U_t | \Omega_{t-1} \sim \mathbb{D}(0, \sigma^2_{t | t-1})
\end{equation*}

Así, de forma similar a un proceso MA(q) podemos decir que la varianza
condicional tendrá efectos ARCH de orden \(q\) (ARCH(q)) cuando:

\[
  \sigma^2_{t|t-1} = \gamma_0 + \gamma_1 U_{t-1}^2 + \gamma_2 U_{t-2}^2 + \ldots + \gamma_q U_{t-q}^2
    \label{ARCH_Effect}
\]

Donde \(\mathbb{E}[\nu_t] = 0\) y \(\gamma_0\) y \(\gamma_i \geq 0\),
para \(i = 1, 2, \ldots, q-1\) y \(\gamma_q > 0\). Estas condiciones son
necesarias para garantizar que la varianza sea positiva. En general, la
varianza condicional se expresa de la forma \(\sigma^2_{t | t-1}\), no
obstante, para facilitar la notación, nos referiremos en cada caso a
esta simplemente como \(\sigma^2_{t}\).

Podemos generalizar está situación si asumimos a la varianza condicional
como dependiente de lo valores de la varianza rezagados, es decir, como
si fuera un proceso AR de orden \(p\) para la varianza y juntandolo con
la ecuación (\ref{ARCH_Effect}). Bollerslev (1986) y Taylor (1986)
generalizaron el problema de heterocedásticidad condicional. El modelo
se conoce como GARCH(p, q), el cual se especifica como: \[
    \sigma^2_t = \gamma_0 + \gamma_1 U_{t-1}^2 + \gamma_2 U_{t-2}^2 + \ldots + \gamma_q U_{t-q}^2 + \beta_1 \sigma^2_{t-1} + \beta_2 \sigma^2_{t-2} + \ldots + \beta_p \sigma^2_{t-p}
    \label{GARCH_Effect}
\]

Donde las condiciones de no negatividad son que \(\gamma_0 > 0\),
\(\gamma_i \geq 0\), \(i = 1, 2, \ldots, q-1\), \(\beta_j \geq 0\),
\(j = 1, 2, \ldots, p-1\), \(\gamma_q > 0\) y \(\beta_p > 0\). Además,
otra condición de convergencia es que: \begin{equation*}
    \gamma_1 + \ldots + \gamma_q + \beta_1 + \ldots + \beta_p < 1
\end{equation*}

Usando el operador rezago \(L\) en la ecuación (\ref{GARCH_Effect})
podemos obtener: \[
    \sigma^2_t = \gamma_0 + \alpha(L) U_t^2 + \beta(L) \sigma^2_t
    \label{GARCH_Effect_L}
\]

De donde podemos establecer: \[
    \sigma^2_t = \frac{\gamma_0}{1 - \beta(L)} + \frac{\alpha(L)}{1 - \beta(L)} U_t^2 
\]

Por lo que la ecuación (\ref{GARCH_Effect}) del GARCH(p, q) representa
un ARCH(\(\infty\)): \[
    \sigma^2_t = \frac{a_0}{1 - b_1 - b_2 - \ldots - b_p} + \sum_{i = 1}^\infty U_{t-i}^2 
\]

\subsection{Modelos ARCH y GARCH
Multivariados}\label{modelos-arch-y-garch-multivariados}

De forma similar a los modelo univariados, los modelos multivariados de
heterocedásticidad condicional asumen una estructura de la media
condicional. En este caso, descrita por un VAR(p) cuyo proceso
estocástico \(\mathbf{X}\) es estacionario de dimensión \(k\). De esta
forma la expresión reducida del modelo o el proceso VAR(p) estará dado
por: \[
    \mathbf{X}_t = \mathbf{\delta} + A_1 \mathbf{X}_{t-1} + A_2 \mathbf{X}_{t-2} + \ldots + A_p \mathbf{X}_{t-p} + \mathbf{U}_{t}
\]

Donde cada uno de las \(A_i\), \(i = 1, 2, \ldots, p\), son matrices
cuadradas de dimensión \(k\) y \(\mathbf{U}_t\) representa un vector de
dimensión \(k \times 1\) con los residuales en el momento del tiempo
\(t\) que son un proceso pueramente aleatorio. También se incorpora un
vector de términos constantes denominado como \(\mathbf{\delta}\), el
cual es de dimensión \(k \times 1\) --es este caso también es posible
incorporar procesos determinas adicionales--.

Así, suponemos que el término de error tendra estructura de vector:
\begin{equation*}
    \mathbf{U}_t = 
    \begin{bmatrix}
    U_{1t} \\ U_{2t} \\ \vdots \\ U_{Kt}
    \end{bmatrix}
\end{equation*}

De forma que diremos que: \begin{equation*}
    \mathbf{U}_t | \Omega_{t-1} \sim (0, \Sigma_{t | t-1})
\end{equation*}

Dicho lo anterior, entonces, el mode ARCH(q) multivariado será descrito
por: \[
    Vech(\Sigma_{t | t-1}) = \boldsymbol{\gamma}_0 + \Gamma_1 Vech(\mathbf{U}_{t-1} \mathbf{U}_{t-1}') + \ldots + \Gamma_q Vech(\mathbf{U}_{t-q} \mathbf{U}_{t-q}')
    \label{M_ARCH}
\]

Donde \(Vech\) es un operador que apila en un vector la parte superior
de la matriz a la cual se le aplique, \(\boldsymbol{\gamma}_0\) es un
vector de cosntantes, \(\Gamma_i\), \(i = 1, 2, \ldots\) son matrices de
coeficientes asociados a la estimación.

Para ilustrar la ecuación (\ref{M_ARCH}), tomemos un ejemplo de
\(K = 2\), de esta forma tenemos que un M-ARCH(1) será:
\begin{equation*}
    \Sigma_{t | t-1} = 
    \begin{bmatrix}
    \sigma^2_{1, t | t-1} & \sigma_{12, t | t-1} \\ \sigma_{21, t | t-1} & \sigma^2_{2, t | t-1}
    \end{bmatrix} = 
    \begin{bmatrix}
    \sigma_{11, t} & \sigma_{12, t} \\ \sigma_{21, t} & \sigma_{22, t}
    \end{bmatrix} =
    \Sigma_{t}
\end{equation*}

Donde hemos simplificado la notación de la varianzas y la condición de
que están en función de \(t-1\). Así, \begin{equation*}
    Vech(\Sigma_{t}) = 
    Vech \begin{bmatrix}
    \sigma_{11, t} & \sigma_{12, t} \\ \sigma_{21, t} & \sigma_{22, t}
    \end{bmatrix} =
    \begin{bmatrix}
    \sigma_{11, t} \\ \sigma_{12, t} \\ \sigma_{22, t}
    \end{bmatrix}
\end{equation*}

De esta forma, podemos establecer el modelo M-ARCH(1) con \(K = 2\) será
de la forma: \begin{equation*}
    \begin{bmatrix}
    \sigma_{11, t} \\ \sigma_{12, t} \\ \sigma_{22, t}
    \end{bmatrix} =
    \begin{bmatrix}
    \gamma_{10} \\ \gamma_{20} \\ \gamma_{30}
    \end{bmatrix} +
    \begin{bmatrix}
    \gamma_{11} & \gamma_{12} & \gamma_{13} \\ \gamma_{21} & \gamma_{22} & \gamma_{23} \\ \gamma_{31} & \gamma_{32} & \gamma_{33}
    \end{bmatrix} 
    \begin{bmatrix}
    U^2_{1, t-1} \\ U_{1, t-1} U_{2, t-1} \\ U^2_{2, t-1}
    \end{bmatrix}
\end{equation*}

Como notarán este tipo de procedimientos implica la estimación de muchos
paramétros. En este circunstacia, se suelen estimar modelos restringidos
para reducir el número de coeficientes estimados. por ejemplo, podríamos
querer estimar un caso como: \begin{equation*}
    \begin{bmatrix}
    \sigma_{11, t} \\ \sigma_{12, t} \\ \sigma_{22, t}
    \end{bmatrix} =
    \begin{bmatrix}
    \gamma_{10} \\ \gamma_{20} \\ \gamma_{30}
    \end{bmatrix} +
    \begin{bmatrix}
    \gamma_{11} & 0 & 0 \\ 0 & \gamma_{22} & 0 \\ 0 & 0 & \gamma_{33}
    \end{bmatrix} 
    \begin{bmatrix}
    U^2_{1, t-1} \\ U_{1, t-1} U_{2, t-1} \\ U^2_{2, t-1}
    \end{bmatrix}
\end{equation*}

Finalmente y de forma analóga al caso univariado, podemos plantear un
modelo M-GARCH(p, q) como: \[
    Vech(\Sigma_{t | t-1}) = \boldsymbol{\gamma}_0 + \sum_{j = 1}^q \Gamma_j Vech(\mathbf{U}_{t-j} \mathbf{U}_{t-j}') + \sum_{m = 1}^p \mathbf{G}_m Vech(\Sigma_{t-m | t-m-1})
    \label{M_GARCH}
\]

Donde cada una de las \(\mathbf{G}_m\) es una matriz de coeficientes.
Para ilustrar este caso retomemos el ejemplo anterior pera ahora para un
modelo M-GARCH(1, 1) con \(K = 2\) de formaque tendríamos:
\begin{equation*}
    \begin{bmatrix}
    \sigma_{11, t} \\ \sigma_{12, t} \\ \sigma_{22, t}
    \end{bmatrix} =
    \begin{bmatrix}
    \gamma_{10} \\ \gamma_{20} \\ \gamma_{30}
    \end{bmatrix} +
    \begin{bmatrix}
    \gamma_{11} & \gamma_{12} & \gamma_{13} \\ \gamma_{21} & \gamma_{22} & \gamma_{23} \\ \gamma_{31} & \gamma_{32} & \gamma_{33}
    \end{bmatrix} 
    \begin{bmatrix}
    U^2_{1, t-1} \\ U_{1, t-1} U_{2, t-1} \\ U^2_{2, t-1}
    \end{bmatrix} +
    \begin{bmatrix}
    g_{11} & g_{12} & g_{13} \\ g_{21} & g_{22} & g_{23} \\ g_{31} & g_{32} & g_{33}
    \end{bmatrix} 
    \begin{bmatrix}
    \sigma_{11, t-1} \\ \sigma_{12, t-1} \\ \sigma_{22, t-1}
    \end{bmatrix}
\end{equation*}

\subsection{Pruebas para detectar efectos
ARCH}\label{pruebas-para-detectar-efectos-arch}

La prueba que mostraremos es conocida como una ARCH-LM, la cual está
basada en una regresión de los residuales estimados de un modelo VAR(p)
o cualquier otra estimación que deseemos probar, con el objeto de
determinar si existen efectos ARCH --esta prueba se puede simplificar
para el caso univariado--.

Partamos de platear: \[
    Vech(\hat{\mathbf{U}}_t \hat{\mathbf{U}}_t') = \mathbf{B}_0 + \mathbf{B}_1 Vech(\hat{\mathbf{U}}_{t-1} \hat{\mathbf{U}}_{t-1}') + \ldots + \mathbf{B}_q Vech(\hat{\mathbf{U}}_{t-q} \hat{\mathbf{U}}_{t-q}') + \varepsilon_t
    \label{ARCH-LM}
\]

Dada la estimación en la ecuación (\ref{ARCH-LM}), plantemos la
estructura de hipótesis dada por: \begin{eqnarray*}
    H_0 & : & \mathbf{B}_1 = \mathbf{B}_2 = \ldots = \mathbf{B}_q = 0 \\
    H_a & : & No H_0    
\end{eqnarray*}

La estadística de prueba será determinada por: \[
    LM_{M-ARCH} = \frac{1}{2} T K (K + 1) - Traza \left( \hat{\Sigma}_{ARCH} \hat{\Sigma}^{-1}_{0} \right) \sim \chi^2_{[q K^2 (K + 1)^2 / 4]}
\]

Donde la matriz \(\hat{\Sigma}_{ARCH}\) es calcula de acuerdo con la
ecuación (\ref{ARCH-LM}) y la matriz \(\hat{\Sigma}_{0}\) sin considerar
una estructura dada para los errores.


\printbibliography


\end{document}
