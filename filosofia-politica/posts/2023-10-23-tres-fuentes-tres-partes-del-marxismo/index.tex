\documentclass[
  jou,
  floatsintext,
  longtable,
  a4paper,
  nolmodern,
  notxfonts,
  notimes,
  colorlinks=true,linkcolor=blue,citecolor=blue,urlcolor=blue]{apa7}

\usepackage{amsmath}
\usepackage{amssymb}



\usepackage[bidi=default]{babel}
\babelprovide[main,import]{spanish}
\StartBabelCommands{spanish}{captions} [unicode, fontenc=TU EU1 EU2, charset=utf8] \SetString{\keywordname}{Palabras
Claves}
\EndBabelCommands


% get rid of language-specific shorthands (see #6817):
\let\LanguageShortHands\languageshorthands
\def\languageshorthands#1{}

\RequirePackage{longtable}
\RequirePackage{threeparttablex}

\makeatletter
\renewcommand{\paragraph}{\@startsection{paragraph}{4}{\parindent}%
	{0\baselineskip \@plus 0.2ex \@minus 0.2ex}%
	{-.5em}%
	{\normalfont\normalsize\bfseries\typesectitle}}

\renewcommand{\subparagraph}[1]{\@startsection{subparagraph}{5}{0.5em}%
	{0\baselineskip \@plus 0.2ex \@minus 0.2ex}%
	{-\z@\relax}%
	{\normalfont\normalsize\bfseries\itshape\hspace{\parindent}{#1}\textit{\addperi}}{\relax}}
\makeatother




\usepackage{longtable, booktabs, multirow, multicol, colortbl, hhline, caption, array, float, xpatch}
\usepackage{subcaption}
\renewcommand\thesubfigure{\Alph{subfigure}}
\setcounter{topnumber}{2}
\setcounter{bottomnumber}{2}
\setcounter{totalnumber}{4}
\renewcommand{\topfraction}{0.85}
\renewcommand{\bottomfraction}{0.85}
\renewcommand{\textfraction}{0.15}
\renewcommand{\floatpagefraction}{0.7}

\usepackage{tcolorbox}
\tcbuselibrary{listings,theorems, breakable, skins}
\usepackage{fontawesome5}

\definecolor{quarto-callout-color}{HTML}{909090}
\definecolor{quarto-callout-note-color}{HTML}{0758E5}
\definecolor{quarto-callout-important-color}{HTML}{CC1914}
\definecolor{quarto-callout-warning-color}{HTML}{EB9113}
\definecolor{quarto-callout-tip-color}{HTML}{00A047}
\definecolor{quarto-callout-caution-color}{HTML}{FC5300}
\definecolor{quarto-callout-color-frame}{HTML}{ACACAC}
\definecolor{quarto-callout-note-color-frame}{HTML}{4582EC}
\definecolor{quarto-callout-important-color-frame}{HTML}{D9534F}
\definecolor{quarto-callout-warning-color-frame}{HTML}{F0AD4E}
\definecolor{quarto-callout-tip-color-frame}{HTML}{02B875}
\definecolor{quarto-callout-caution-color-frame}{HTML}{FD7E14}

%\newlength\Oldarrayrulewidth
%\newlength\Oldtabcolsep


\usepackage{hyperref}




\providecommand{\tightlist}{%
  \setlength{\itemsep}{0pt}\setlength{\parskip}{0pt}}
\usepackage{longtable,booktabs,array}
\usepackage{calc} % for calculating minipage widths
% Correct order of tables after \paragraph or \subparagraph
\usepackage{etoolbox}
\makeatletter
\patchcmd\longtable{\par}{\if@noskipsec\mbox{}\fi\par}{}{}
\makeatother
% Allow footnotes in longtable head/foot
\IfFileExists{footnotehyper.sty}{\usepackage{footnotehyper}}{\usepackage{footnote}}
\makesavenoteenv{longtable}

\usepackage{graphicx}
\makeatletter
\newsavebox\pandoc@box
\newcommand*\pandocbounded[1]{% scales image to fit in text height/width
  \sbox\pandoc@box{#1}%
  \Gscale@div\@tempa{\textheight}{\dimexpr\ht\pandoc@box+\dp\pandoc@box\relax}%
  \Gscale@div\@tempb{\linewidth}{\wd\pandoc@box}%
  \ifdim\@tempb\p@<\@tempa\p@\let\@tempa\@tempb\fi% select the smaller of both
  \ifdim\@tempa\p@<\p@\scalebox{\@tempa}{\usebox\pandoc@box}%
  \else\usebox{\pandoc@box}%
  \fi%
}
% Set default figure placement to htbp
\def\fps@figure{htbp}
\makeatother







\usepackage{newtx}

\defaultfontfeatures{Scale=MatchLowercase}
\defaultfontfeatures[\rmfamily]{Ligatures=TeX,Scale=1}





\title{Tres fuentes y tres partes del marxismo: según Lenin}


\shorttitle{Editar}


\usepackage{etoolbox}



\ccoppy{\textcopyright~2023}



\author{Edison Achalma}



\affiliation{
{Escuela Profesional de Economía, Universidad Nacional de San Cristóbal
de Huamanga}}




\leftheader{Achalma}

\date{2023-10-23}


\abstract{Primer parrafo de abstrac }

\keywords{keyword1, keyword2}

\authornote{\par{\addORCIDlink{Edison Achalma}{0000-0001-6996-3364}} 
\par{ }
\par{   El autor no tiene conflictos de interés que revelar.    Los
roles de autor se clasificaron utilizando la taxonomía de roles de
colaborador (CRediT; https://credit.niso.org/) de la siguiente
manera:  Edison Achalma:   conceptualización, redacción}
\par{La correspondencia relativa a este artículo debe dirigirse a Edison
Achalma, Email: \href{mailto:elmer.achalma.09@unsch.edu.pe}{elmer.achalma.09@unsch.edu.pe}}
}

\usepackage{pbalance} 
\usepackage{float}
\makeatletter
\let\oldtpt\ThreePartTable
\let\endoldtpt\endThreePartTable
\def\ThreePartTable{\@ifnextchar[\ThreePartTable@i \ThreePartTable@ii}
\def\ThreePartTable@i[#1]{\begin{figure}[!htbp]
\onecolumn
\begin{minipage}{0.5\textwidth}
\oldtpt[#1]
}
\def\ThreePartTable@ii{\begin{figure}[!htbp]
\onecolumn
\begin{minipage}{0.5\textwidth}
\oldtpt
}
\def\endThreePartTable{
\endoldtpt
\end{minipage}
\twocolumn
\end{figure}}
\makeatother


\makeatletter
\let\endoldlt\endlongtable		
\def\endlongtable{
\hline
\endoldlt}
\makeatother

\newenvironment{twocolumntable}% environment name
{% begin code
\begin{table*}[!htbp]%
\onecolumn%
}%
{%
\twocolumn%
\end{table*}%
}% end code

\urlstyle{same}



\makeatletter
\@ifpackageloaded{caption}{}{\usepackage{caption}}
\AtBeginDocument{%
\ifdefined\contentsname
  \renewcommand*\contentsname{Tabla de contenidos}
\else
  \newcommand\contentsname{Tabla de contenidos}
\fi
\ifdefined\listfigurename
  \renewcommand*\listfigurename{Listado de Figuras}
\else
  \newcommand\listfigurename{Listado de Figuras}
\fi
\ifdefined\listtablename
  \renewcommand*\listtablename{Listado de Tablas}
\else
  \newcommand\listtablename{Listado de Tablas}
\fi
\ifdefined\figurename
  \renewcommand*\figurename{Figura}
\else
  \newcommand\figurename{Figura}
\fi
\ifdefined\tablename
  \renewcommand*\tablename{Tabla}
\else
  \newcommand\tablename{Tabla}
\fi
}
\@ifpackageloaded{float}{}{\usepackage{float}}
\floatstyle{ruled}
\@ifundefined{c@chapter}{\newfloat{codelisting}{h}{lop}}{\newfloat{codelisting}{h}{lop}[chapter]}
\floatname{codelisting}{Listado}
\newcommand*\listoflistings{\listof{codelisting}{Listado de Listados}}
\makeatother
\makeatletter
\makeatother
\makeatletter
\@ifpackageloaded{caption}{}{\usepackage{caption}}
\@ifpackageloaded{subcaption}{}{\usepackage{subcaption}}
\makeatother
\makeatletter
\@ifpackageloaded{fontawesome5}{}{\usepackage{fontawesome5}}
\makeatother

% From https://tex.stackexchange.com/a/645996/211326
%%% apa7 doesn't want to add appendix section titles in the toc
%%% let's make it do it
\makeatletter
\xpatchcmd{\appendix}
  {\par}
  {\addcontentsline{toc}{section}{\@currentlabelname}\par}
  {}{}
\makeatother

%% Disable longtable counter
%% https://tex.stackexchange.com/a/248395/211326

\usepackage{etoolbox}

\makeatletter
\patchcmd{\LT@caption}
  {\bgroup}
  {\bgroup\global\LTpatch@captiontrue}
  {}{}
\patchcmd{\longtable}
  {\par}
  {\par\global\LTpatch@captionfalse}
  {}{}
\apptocmd{\endlongtable}
  {\ifLTpatch@caption\else\addtocounter{table}{-1}\fi}
  {}{}
\newif\ifLTpatch@caption
\makeatother

\begin{document}

\maketitle

\hypertarget{toc}{}
\tableofcontents
\newpage
\section[Introduction]{Tres fuentes y tres partes del marxismo}

\setcounter{secnumdepth}{-\maxdimen} % remove section numbering

\setlength\LTleft{0pt}


\section{Explorando las Tres Fuentes y Tres Partes del Marxismo según
Lenin}\label{explorando-las-tres-fuentes-y-tres-partes-del-marxismo-seguxfan-lenin}

Me complace compartir con ustedes uno de los textos más maravillosos de
síntesis del marxismo. Este texto ha sido desarrollado por Lenin en un
prodigio de cinco hojas, resumiendo toda la teoría marxista tal como se
había desarrollado en su momento. El texto se llama ``Tres Fuentes y
Tres Partes Integrantes del Marxismo'' y fue escrito por Lenin en 1913,
cuando se encontraba en el exilio.

En aquel momento, la Rusia zarista estaba acelerando la transformación
del sistema semi-feudal al capitalismo en toda Rusia. El partido
bolchevique venía de enfrentar el duro golpe de la contrarrevolución del
5, y los tambores de la Primera Guerra Mundial, que comenzaría en 1914,
ya se hacían sentir. Además, en Rusia, había hambrunas como resultado de
este rápido desarrollo capitalista con inversiones extranjeras. En este
contexto, era crucial consolidar al partido bolchevique.

``Tres Fuentes y Tres Partes Integrantes del Marxismo'' es un artículo
escrito para una revista llamada ``obrazovaniye,'' que significa
``educación.'' Esta revista surgió como continuación de otra revista
llamada ``mysl''' que el zarismo había cerrado. ``mysl''' significa
``pensamiento'' en ruso. En este entorno, Lenin escribió ``Tres Fuentes
y Tres Partes Integrantes del Marxismo'' un texto excelente para
introducirse al marxismo y comprender los conceptos básicos. Para
profundizar aún más en el marxismo, se puede consultar otro texto
titulado ``Karl Marx: Un Breve Esbozo Histórico,'' que Lenin escribió
para un diccionario filosófico y que ofrece una visión más completa de
la doctrina. El texto es un poco más extenso, con alrededor de 25 a 30
páginas, y es una excelente manera de comenzar a entender el marxismo.

Lenin comienza Tres Fuentes y Tres Partes Integrantes del Marxismo con
una frase poderosa: ``La doctrina de Marx es omnipotente porque es
verdadera.'' Esta afirmación refleja la profunda comprensión que tenía
Lenin del marxismo, que no consideraba una doctrina sectaria, sino que
surgía de lo más avanzado del pensamiento de su época.

Lenin también enfatiza que el marxismo sigue siendo atacado, incluso por
aquellos que se proclaman marxistas pero que en realidad tienen una
visión socio-reaccionaria. En resumen, el texto de Lenin de 1913 es un
testimonio de la relevancia continua del marxismo y su capacidad para
proporcionar una comprensión sólida de la sociedad y la lucha de clases.

El marxismo es un proyecto de investigación que nunca ha terminado.
Lamentablemente, en el siglo XX, debido a persecuciones, torturas,
asesinatos y prohibiciones de libros, no pudimos profundizar tanto como
quisiéramos. Sin embargo, en el siglo XXI, es esencial seguir
investigando y desarrollando el marxismo para poder construir una ola
revolucionaria que el planeta tanto necesita. Además, la ciencia se
vuelve cada vez más dialéctica a medida que avanza.

Entonces, como les decía, es importante destacar que la doctrina de Marx
no surge de la nada, sino que es una continuación directa e inmediata de
las doctrinas de las más grandes figuras de la filosofía, la economía
política y el socialismo. Esto significa que se basa en el conocimiento
más avanzado de su época.

Refiriéndonos a las fuentes del marxismo, se puede decir que el marxismo
es el heredero legítimo de la filosofía alemana, la economía política
inglesa y el socialismo francés.

La filosofía del marxismo es el materialismo, en contraposición al
idealismo alemán. Marx tomó la dialéctica de Hegel y la aplicó al
estudio de la sociedad, sugiendo el materialismo histórico. Esta
aplicación llevó a que la historia dejara de ser simplemente una
narrativa histórica o anécdota y pasara a convertirse en una ciencia. El
materialismo histórico, como base de esta nueva ciencia, plantea la
serie de cambios económicos en la historia.

La economía Política. A lo largo de su vida, Marx se dedicó a
profundizar en el tema económico, que fue su especialidad. Explicó cómo
la base económica es la que sostiene toda la superestructura política,
estatal, filosófica, ideológica y jurídica de la sociedad. Los conceptos
clave en este contexto son el valor, la plusvalía y la lucha de clases.

El socialismo. Partiendo del materialosmo histórico, descubre que la
lucha de clases es el motor de la historia y el rol de la clase
trabajadora como sepulturera del capitalismo

Estas tres partes fundamentales del marxismo son la \textbf{filosofía,
la economía política y el socialismo}. Lenin, en este texto, profundiza
de manera magistral en estos conceptos.

Entonces vamos con la primera parte.

\subsection{La filosofía.}\label{la-filosofuxeda.}

Lenin nos dirá que la filosofía del marxismo es el materialismo y
explicará lo que significa ser materialista en filosofía.

Básicamente, ser materialista en filosofía implica preguntarse \emph{si
existe una realidad fuera de nosotros. Y si esa realidad es cognosible}.
Porque, si podemos decir que hay una realidad pero que no podemos
conocerla, entonces caemos del lado del idealismo. Esto significa que
estamos en el terreno de la filosofía especulativa en lugar de buscar un
conocimiento realista, científico y objetivo de la realidad, que es lo
que propone el marxismo.

La filosofía marxista no utiliza otro método que no sea el científico y,
como tal, en la filosofía materialista, se divide en campos que te
plantean si estás del lado de la especulación, el delirio o las palabras
bonitas, o del conocimiento objetivo para transformar la realidad. Esta
es la gran disyuntiva.

El materialismo, en términos filosóficos, significa que la materia
existe y que hay una realidad objetiva. El idealismo, por otro lado,
argumenta que o bien la realidad objetiva no existe (como en el extremo
del solipsismo, donde solo existe uno mismo y todo lo demás es
proyección) o que si existe, no puede ser conocida y es incognoscible.

Podemos decir que el materialismo, en el marxismo, es heredero del
materialismo del siglo XVIII, que fue profundamente revolucionario. El
materialismo del siglo XVIII, representado por los enciclopedistas y los
revolucionarios franceses, luchó contra el clericalismo medieval, los
prejuicios medievales y el feudalismo. Sin embargo, este materialismo
era mecanicista.

El materialismo marxista se enriqueció con los logros de la filosofía
clásica alemana, especialmente con el sistema de Hegel, el cual, creía
que la sociedad ya había alcanzado su máximo nivel de racionalidad. Marx
y Engels, por otro lado, creían que la sociedad seguiría
desarrollándose. Fue Feuerbach quien influyó en Marx y Engels y los
ayudó a destruir la jerigonza idealista de Hegel y a enfocarse en la
realidad material. Así, el materialismo marxista se basa en la idea de
examinar y entender la realidad material.

El principal logro de Hegel, sin embargo, es la dialéctica, es decir, la
doctrina del desarrollo en su forma más completa, profunda y libre de
unilateralidad. Es la doctrina acerca de lo relativo del conocimiento
humano, que nos brinda un reflejo de la materia en perpetuo desarrollo.
Es decir, lo que Marx y Engels harán cuando se encuentren en 1844 en
París, siendo muy jóvenes, es llegar a la misma conclusión: que era
necesario poner la dialéctica hegeliana, que hablaba de una idea
absoluta que impulsaba el mundo en un movimiento dialéctico. Ellos
dicen: ``No, hay que ponerla con los pies en la tierra''. La dialéctica
debe encontrarse en la realidad objetiva. La dialéctica es una forma de
ver el desarrollo, la evolución de una manera muy plástica, dinámica y
contradictoria. Es lo contrario de la visión metafísica que divide entre
blanco y negro, entre sí y no, que no ve los saltos cualitativos ni las
contradicciones. Todo eso lo hace el pensamiento metafísico. Lo que
hacen Marx y Engels es desarrollar el materialismo dialéctico, es decir,
ver la realidad en su desarrollo.

Posteriormente, Marx y Engels aplican el materialismo al estudio de la
sociedad humana, nace el materialismo histórico. Es decir, a partir de
una visión materialista, lo que deben ver es que la historia tiene una
base material, a saber, el desarrollo económico de las sociedades. Lo
que van a mostrar es cómo, a partir del desarrollo de las fuerzas
productivas, surge un sistema social superior. Por ejemplo, cómo del
feudalismo nace el capitalismo. Descubren las relaciones entre la base y
la superestructura, es decir, cómo la base económica influye en la
superestructura política, ideológica, jurídica y artística. Todo eso
depende de la base material o economómica de la sociedad, del mismo modo
que el conocimiento del hombre refleja la naturaleza, es decir, la
materia en desarrollo, que existe independientemente del hombre, su
conocimiento social, es decir, las diversas opiniones y doctrinas
filosóficas, religiosas, políticas, etc., reflejan el régimen económico
de la sociedad.

Según Marx, las instituciones políticas son la superestructura que se
alza sobre la base económica. Con este descubrimiento, Marx y Engels
transformaron la historia de un anecdotario, de revistas tipo
``paparazzi'' que se centraban en detalles como el matrimonio del rey o
la reina, a una ciencia donde se debía analizar la base económica para
entender los fenómenos políticos. Este enfoque ha sido duramente
atacado, tachándolo de mecanicista o eurocentrista, entre otras
críticas. Sin embargo, en realidad, el materialismo histórico sigue
demostrando su vigencia.

El gran descubrimiento del materialismo histórico es que la base
económica permite entender lo que está sucediendo en la superestructura
de la sociedad, no de un modo mecánico, sino de un modo dialéctico, muy
dinámico. Es el avance económico el que nos permite comprender por qué
existen determinadas visiones políticas o ideológicas.

Dentro de esta división del trabajo, Marx se dedicó al estudio de la
economía, mientras que Engels se centró más en la historia, temas
militares y filosofía.

\subsection{Economía Politica}\label{economuxeda-politica}

La segunda fuente del marxismo proviene de la economía inglesa, donde
Adam Smith y David Ricardo ya planteaban que el valor de una mercancía
estaba relacionado con el trabajo. Inglaterra, como el primer país en
desarrollar el capitalismo, fue el lugar donde se observaron claramente
las leyes del capitalismo. Marx desarrolló las ideas de Smith y Ricardo,
argumentando que el valor de las mercancías se basaba en el tiempo
socialmente necesario para producirlas. Es decir, todos los que
participan en la producción de una mercancía contribuyen con una parte
del tiempo de trabajo necesario para crearla, y el promedio entre la
maquinaria más rápida y la más lenta determina el tiempo de trabajo
socialmente necesario para producir esa mercancía.

\{\{\{\{\{\{\{\{\{\{\{\{\{\}\}\}\}\}\}\}\}\}\}\}\}\} min 14:48
https://www.youtube.com/watch?v=o\_7l6VVMz5o

Este párrafo es brillante, ya que resume de manera concisa y clara los
primeros capítulos de ``El Capital'', una obra que no se puede resumir
en unas pocas frases.

Podríamos decir que aquí se hace alusión al fetichismo de la mercancía.
Mientras los economistas burgueses veían relaciones entre objetos, Marx
descubrió relaciones entre personas. En el intercambio de mercancías, se
expresa el vínculo establecido a través del mercado entre los
productores aislados, cuyos valores de cambio parecen ser algo natural
en el capitalismo. En realidad, Marx descubrió que esto es producto de
una etapa sociohistórica de la humanidad en la cual el metabolismo
social une a los productores aislados.

Con el tiempo, se fue generando el valor. Inicialmente, los productores
aislados hacían intercambios esporádicos y desarrollaron el trueque.
Pero con el desarrollo del mercado, las mercancías se convirtieron en el
valor de cambio de todas las demás, y finalmente, el dinero cumplió esta
función de manera máxima. El dinero unió indisolublemente en un todo
único la vida económica completa de los productores aislados. El
capital, por otro lado, representa un desarrollo posterior de este
vínculo. La fuerza de trabajo del hombre se transforma en mercancía,
pero no cualquier tipo de dinero, el dinero se convierte en capital
cuando está destinado a explotar al ser humano, es decir, cuando el
esfuerzo humano y su información se transforman en mercancía. Esto es
algo que Marx plantea, aunque aún no tenía la categoría de
``información'' cuando describió el sistema en el que existe el trabajo
simple, realizado con una pala, o el trabajo complejo, que implica
trabajar con una computadora.

En su desarrollo, el capital acumula plusvalía, lo que significa que se
desarrolla a partir de extraer o robar al obrero. Este proceso es un
robo organizado y legalizado, el robo del capital a través de la
obtención de plusvalía.

Lenin, en este texto, destaca que la plusvalía es la piedra angular del
marxismo. Para explicar esto, compáralo con la piedra angular en
arquitectura, que es aquella que da coherencia a todo el edificio. A
partir de la plusvalía, se entiende la globalidad del sistema. Cuando el
trabajo humano se transforma en mercancía y el obrero asalariado vende
su fuerza de trabajo, el propietario de la tierra, la fábrica o las
herramientas emplea una parte de la jornada para cubrir los gastos del
sustento básico de su familia (el salario), mientras que la otra parte
del tiempo trabaja gratis, creando para el capitalista la fuente de
ganancia y riqueza de la clase capitalista. En el cuadro adjunto se
muestra la tasa de plusvalía.

Si usted es un burgués o tiene 20.000 de plusvalía y paga un salario o
capital variable de 20.000, la tasa de plusvalía será del 100 por
ciento. Esto es lo que define Marx de manera genial en ``El Capital''.
Entonces, esto significa que el trabajador trabajará cuatro horas para
su sustento y cuatro horas para que el patrón lo siga explotando. Esto
hace que el capital se concentre cada vez en menos manos, lo que tiende
a la concentración y centralización del capital en un número cada vez
menor de manos.

Uno de los impresionantes descubrimientos de Marx es la tendencia
histórica del capitalismo: cada vez el capital se concentra en menos
manos, tendiendo hacia el monopolio. En esta etapa, la humanidad ha
experimentado la brutalidad de que un 1\% posee el 50 por ciento de la
riqueza. Tendrás otros datos por ahí que son simplemente impactantes en
cuanto a lo que ha desarrollado el capitalismo. Esto ha sido posible
gracias al golpe ideológico que nos han dado desde los años 80 hasta
ahora. Han intentado hacer que reneguemos de nuestra teoría, impidiendo
que se desarrolle y llegue a las nuevas generaciones.

A pesar de todo esto, el marxismo sigue avanzando. La explotación
continúa, al menos en Occidente, y el capitalismo sigue su curso. Aquí
tenemos la tendencia histórica que desarrolla de manera brutal la
contradicción fundamental del capitalismo: la producción es cada vez más
social, pero la apropiación es cada vez más individual. Piensa en
cuántas manos de trabajadores han intervenido en la fabricación de un
simple objeto, como un plástico verde. Desde la extracción del petróleo
hasta la producción del objeto, la producción se vuelve cada vez más
social y globalizada, pero la apropiación de los frutos de ese trabajo
es cada vez más individualista. Esta es la contradicción fundamental del
capitalismo, una que no puede resolver sino a través de la guerra, la
destrucción de la humanidad y del planeta. Y es precisamente esta
contradicción la que solo puede resolverse a través del socialismo y la
revolución socialista.

Bien, ahora bien, al hacer a los obreros más dependientes aún del
capital, el régimen capitalista crea una gran fuerza de trabajo
asociada. El capitalismo ha triunfado en el mundo entero, pero esa
victoria no es más que el preludio del triunfo del capital sobre el
trabajo. Es decir, nos ponen a todos del mismo lado, y es a ese
metabolismo mundial, es el capitalismo mundial, esa globalización
mundial, cada vez va preparando la fuerza de lo que va a ser su futuro:
la clase trabajadora. Por supuesto, de la mano de una teoría, de una
doctrina revolucionaria para acabar con el poder del capital.

Y ahí es donde surge la tercera fuente, la tercera parte integrante del
marxismo, que es el socialismo. Es decir, la visión de que tiene que
haber una fuerza social que destruya estas contradicciones, tomando el
poder y construyendo el poder socialista, el poder de los trabajadores,
el poder de las grandes mayorías.

El marxismo aquí surge también de lo que se había elaborado en el
socialismo francés, lo que llamó a los ``patriarcas'' del socialismo en
su momento, que son Fourier, Saint-Simon y Robert Owen. Digo, en su
momento, cuando Marx y Engels eran jóvenes y empezaban a armar la Liga
de los Comunistas. Estos serán aquellos a quienes la militancia y el
militante a Marx le llama la atención, cómo los militantes se juntaban
para hablar de la humanidad. Esto lo puede verse en los manuscritos de
1844. Digo, abrevaban en estos personajes, en estas personalidades.
Fourier, Charles Fourier, había muerto en 1837. En 1844, se encuentran
Marx y Engels. Digo, Fourier planteaba los falansterios, es decir,
predicaba la generación de una nueva humanidad a partir de generar
grupos humanos que vivieran todos juntos en comunidad, de acuerdo a
determinadas características psicosociales.

Y Charles Fourier es uno de los precursores de la psicología social. Él
planteó la idea de los ``falansterios'' como una manera de generar algún
día una nueva humanidad. Aquí en Misiones, un ``familisterio'' a la
manera de Fourier. Saint-Simon era un cóctel de ideas de Saint-Simon, el
tipo que renunció a todas sus prerrogativas feudales y se dedicó a
predicar una forma de socialismo que buscaba unir a las clases
productivas en contra de las clases dominantes. Es decir, unir a los
industriales, decía él, con los trabajadores o con las clases
agropecuarias para acabar con todo el feudalismo.

Saint-Simon fue un avanzado para su época. Él había visto que en la
Revolución Francesa había una lucha de clases, lo cual era mucho para su
época. También planteó cuestiones en relación al feminismo y participó
en la lucha de independencia de EE. UU. Murió pobre, pero contento y
digno, ya que fue un militante durante toda su vida.

Robert Owen estuvo a punto de descubrir la teoría de la plusvalía. Él
era un señor muy rico que tenía un gran fondo de producción de lana. Se
dio cuenta de que, a pesar de que las maquinarias seguían avanzando y él
obtenía cada vez más dinero, los obreros vivían igual que hace 50 años.
Entonces, empezó a desarrollar una metodología cooperativa. En su
tiempo, era muy respetado en las cortes europeas, como una especie de
personaje muy divertido. Cuando comenzó la comunidad de New Lanark, pasó
precisamente a ser visto como un tipo peligroso y fue irradiado. Fue el
creador de los jardines de infantes y uno de los padres del
cooperativismo. Sus obreros empezaron a dejar el alcohol en la comunidad
de Miramar, donde desarrolló un montón de elementos de vida comunitaria.

Robert Owen también perdió su fortuna, pero fundó las bases del
pensamiento cooperativista. Estos son los patriarcas del socialismo.
¿Cuál era la dificultad de los patriarcas del socialismo? Bueno, ellos
no podían descubrir las leyes del desarrollo capitalista ni señalar qué
fuerza social estaba en condiciones de convertirse en creadora de una
nueva sociedad. Ni la fuerza social que haría esto posible. Es decir,
ellos creían que iban a cambiar el mundo para mejor y que los burgueses
se unirían con los trabajadores para crear ese mundo mejor. Por eso son
utópicos. Ojo, no hay nada peyorativo aquí, sino un gran respeto por el
marxismo. Pueden profundizar sobre este tema en un texto de Engels que
convirtió a muchos trabajadores en Europa y en el mundo al socialismo.
Por supuesto, me refiero al ``Socialismo: Utopía y Ciencia'' o del
``Socialismo Utópico al Científico'', un gran texto donde se rinde un
gran homenaje a los patriarcas del socialismo.

Entonces, Marx, desde una visión materialista, va a decir que va a haber
una fuerza material en las condiciones materiales de producción, y es
ahí donde tenemos que encontrar la fuerza que barrerá con toda la
miseria moral, espiritual y material en la que nos sumerge el
capitalismo. Sí, y ahí es donde plantea la lucha de clases como motor de
la historia, las revoluciones, las tempestuosas revoluciones. Va a decir
que acompañaron en toda Europa, y especialmente en Francia, la caída del
régimen feudal, del régimen de la servidumbre. Tenía que ver con mayor
evidencia que la base de todo desarrollo y su fuerza motriz era la lucha
de clases.

``Los hombres han sido siempre, política y víctimas necias del engaño
ajeno y propio, y lo seguirán siendo mientras no aprendan a descubrir
detrás de todas las frases, declaraciones y promesas morales,
religiosas, políticas y sociales, los intereses de una y otra clase''.
Esta frase es espectacular, por favor, grábenla, úsenla en Facebook,
hagan una remera con esa frase.

Y surge de este texto ``Tres Fuentes y Tres Partes''. Cada frase es una
pelota enorme de sentido coagulado de años de investigación. Por eso,
hay que leerlo una y otra vez, incluso aprenderlo de memoria. ``Tres
Fuentes y Tres Partes'', señores y señoras, es un texto impresionante,
lo más profundo que la cultura humana haya dado. El marxismo, por lo
tanto, es un arma teórica para la liberación de la clase trabajadora.
Solo el materialismo filosófico de Marx señaló al proletariado la salida
de la esclavitud espiritual en la que se han consumido hasta hoy todas
las clases oprimidas. Solo la teoría económica de Marx explicó la
situación real del proletariado en el régimen general del capital. Y el
proletariado, por supuesto, estamos hablando de todos los trabajadores,
de todos los pueblos del mundo, en contra de los por qués. Si a eso se
refiere con proletariado, se dice que el proletariado es el que no tiene
más que sus propios recursos, y por eso somos proletarios. La única
riqueza que tenemos. Y aquí, entonces, vamos a llegar a un cuadro
comprensivo de lo que plantea ``Tres Fuentes y Tres Partes'', poniendo
algunos agregados que lo hacen aún más contextual. Veamos, las tres
fuentes son la filosofía clásica alemana. Van a ver en el centro las
tres cajas que están en el centro: Hegel, Herbart. Y ahí surge el
materialismo dialéctico y su aplicación a la historia, el materialismo
dialéctico. En el centro está la economía política inglesa, David
Ricardo y Adam Smith. Segunda fuente y segunda parte, la economía
política. Y la tercera fuente es el socialismo utópico inglés y francés.
Perdón, Saint-Simon, Fourier, Owen. Joven que era inglés, pero bueno,
pero en una de las fuentes y de ahí viene el socialismo científico. Pero
aparte, hay que ver un contexto de luchas obreras, de la lucha en la
clase obrera después de que en 1789, la Revolución Francesa, la
revolución burguesa, viniera con el ideal de libertad, igualdad y
fraternidad. Prontamente, se vio que era una nueva forma de opresión la
que ofrecía la burguesía. Está la lucha de los obreros de Lyon en 1831,
la lucha donde se destruyen las máquinas de hilado y donde los obreros
toman la ciudad de Lyon por breves días. Y en Inglaterra, el movimiento
cartista, la Freedom Charter de 1838-1842, demostraba la fuerza del
pueblo movilizado haciendo una serie de reclamos democráticos generales.
Este era el movimiento cartista y esto también forma parte del fermento
donde nacen las ideas marxistas.

Por otra parte, los descubrimientos científicos que se habían desplegado
por la época de Marx y Engels. Primero, la teoría celular, que
demostraba que tanto los animales como los seres humanos teníamos en
nuestro cuerpo la misma unidad funcional, que es la célula. Después, el
primer principio de la termodinámica, que habla de que nada se pierde,
todo se transforma. Estos serían dos duros golpes al creacionismo
religioso que hablaban de la creación del mundo, y mostraba cómo, en
realidad, no había creación del mundo, sino que había una dinámica
propia que hacía evolucionar a las especies vivas. Por último, la
evolución de las especies, que llegó tardíamente en 1859 cuando Charles
Darwin publicó su obra ``El Origen de las Especies'', que estaba
desarrollando esta visión de que lo que buscan es la evolución en el
mundo real y no en el nuevo día absoluto, como decía Hegel. Fue un
subidón de energía tremendo tanto para Marx como para Engels, haber
llegado a esta teoría de Charles Darwin, la evolución de las especies.
Mostraba que así como las sociedades humanas evolucionaban, primero
habían evolucionado las especies. Digamos que todas las teorías avalaban
el pensamiento marxista. Ahora bien, hasta aquí llegamos a un punto.
Permítame cinco minutos para plantear la especie de que estamos
haciendo. Una vez es el marxismo, su proyecto de investigación, como ya
les dije, y ese proyecto de investigación está truncado desde el siglo
19 hasta ahora. Han habido cambios y avances que no se han incorporado
en el marxismo. Fíjense ustedes que aquí dice ``el primer principio de
la termodinámica''. Bueno, precisamente lo que nunca pudimos integrar es
toda la derivación que viene a partir del segundo principio de la
termodinámica, que dice que el universo está constantemente en
degradación, entrando en entropía. El segundo principio de la
termodinámica es ineluctable, todos vamos a morir. Pero en el planeta
Tierra, hay una isla donde se crea orden y organización, donde las cosas
van en contra del sentido del universo. ¿Por qué pasa esto? Bueno, todo
eso nos lo fuimos perdiendo, toda esa evolución científica y toda esa
evolución no entró en nuestras categorías científicas, pese a que Lenin
nos lo había planteado.

Entonces, si la entropía es una medida del desorden y la
desorganización, lo contrario es la categoría de información. Sin
embargo, esta categoría tampoco se incorporó en nuestro aparato
conceptual, lo que nos llevó a perder un montón de evolución que ocurrió
durante el capitalismo. Esto, finalmente, desarticuló nuestro
pensamiento en el último cuarto del siglo XX.

Y aquí, tengo que ir con ustedes a ver qué leen. Ya nos había advertido
esto. Fíjense en el proyecto de investigación que planteó Lenin en los
Cuadernos Filosóficos, que básicamente decía que debíamos estudiar la
historia del conocimiento en general. Esto incluye la historia de la
ciencia, el desarrollo mental de los niños, de los animales, del
lenguaje, la psicología, porque sabía que por este lado nos iban a
desarticular. Sin embargo, esto no se pudo hacer debido a los desafíos
que conllevaba.

Lenin estaba planteando que necesitábamos aprender cómo evoluciona el
procesamiento de información. Por qué evoluciona desde la primera célula
hasta la almeja, del almejero al rinoceronte, de ahí a la sociedad
paleolítica, la sociedad feudal, y finalmente a la sociedad actual. La
evolución se debe al procesamiento de información, y sobrevivir en este
planeta significa incrementar la capacidad de procesar información.

Cualquier sistema vivo que no procesa información es eliminado, ya sea
por la selección natural o la selección cultural. En este mundo, para
sobrevivir, debes aprender a procesar información mejor que tus rivales.
La clase trabajadora debe aprender a procesar información, y todo esto
representa un desarrollo que se perdió en el siglo XX.

En 1947, Norbert Wiener desarrolla la teoría cibernética que explora
cómo la información controla a animales, máquinas y seres humanos. Esta
teoría puede integrarse en nuestro sistema de categorías. Ahora,
permítanme plantearles lo que Che Guevara planteaba en 1965 cuando era
apenas un infante de meses.

Che decía que el escolasticismo había frenado el desarrollo de la
filosofía marxista y proponía que debíamos estudiar al hombre nuevo y la
mujer nueva, es decir, el ser humano nuevo. Pueden encontrar estas ideas
en su texto ``El Socialismo y el Hombre en Cuba''. Es en este punto que
quiero invitarlos a continuar, ya que las Tres Fuentes y Tres Partes
representan una etapa histórica del marxismo. No son un conjunto de
ideas escritas en piedra, sino una guía que nos dice: `Hasta aquí hemos
llegado, ahora les toca a ustedes'.

A través de la Escuela Latinoamericana de Formación, los invito a seguir
explorando. Imaginen que si la historia la escriben los seres humanos,
debemos tener una concepción del ser humano. El marxismo la tiene, pero
está poco difundida. En el lugar donde debería haberse construido una
visión marxista del ser humano, entró el psicoanálisis. Por lo tanto,
debemos llegar al corolario de la concepción marxista del ser humano,
que es la visión del ser humano nuevo y la mujer nueva, tal como planteó
Marx. Además, el Che Guevara también abordó este tema claramente.

Por otro lado, al incorporar los últimos avances de la ciencia, debemos
incluir la teoría del caos, como el efecto mariposa, donde un pequeño
cambio puede generar un gran cambio, como ocurre en el clima, un sistema
caótico. También, debemos considerar la teoría de la complejidad, que
abarca la cibernética, la teoría del caos y va más allá. Esta teoría
estudia sistemas dinámicos complejos y refuerza la idea de Marx y Engels
de que todo en la Tierra se desarrolla de lo simple a lo complejo. Hay
una relación profunda entre la evolución de las especies y la evolución
de las sociedades humanas.

Sí, con lo cual ya estamos describiendo y explicando por qué hay una
evolución. Esta evolución se basa en el incremento del desarrollo de la
capacidad de procesar información, algo que explica la teoría del caos
sobre el mismo sistema que Marx y Engels describieron cuando planteaban
esta unidad entre lo social y lo natural. Utilizando categorías como
sistemas dinámicos complejos adaptativos, los complejos logos hablan de
esto.

Lo que quiero enfatizar a través de la Escuela Latinoamericana de
Formación es que debemos llevar a cabo una gran campaña de formación y
construcción de cuadros de pensamiento revolucionario que sitúe este
pensamiento a la altura del siglo 21. Estamos en la era de la
información y nuestras categorías filosóficas deben ser capaces de
abordar todos estos conceptos.

Todo esto se relaciona con las Tres Fuentes y Tres Partes. El marxismo
no es simplemente una caja de herramientas, y no podemos permitir que se
fragmente en el eurocentrismo o el cientificismo. El marxismo tiene una
vocación que los últimos avances científicos demuestran. Debemos retomar
esta vocación, reconstruyendo lo que los clásicos nos ofrecieron,
uniendo el pensamiento científico revolucionario con los últimos avances
científicos y respondiendo a las variantes del idealismo, incluso
aquellas que se presentan como marxistas. Nuestra meta es actualizar el
pensamiento marxista al siglo 21.

\section{Publicaciones Similares}\label{publicaciones-similares}

Si te interesó este artículo, te recomendamos que explores otros blogs y
recursos relacionados que pueden ampliar tus conocimientos. Aquí te dejo
algunas sugerencias:

\begin{enumerate}
\def\labelenumi{\arabic{enumi}.}
\tightlist
\item
  \href{https://achalmaedison.netlify.app/filosofia-politica/posts/2018-04-23-aparicion-pensamiento-socialista/index.pdf}{\faIcon{file-pdf}}
  \href{https://achalmaedison.netlify.app/filosofia-politica/posts/2018-04-23-aparicion-pensamiento-socialista}{Aparicion
  Pensamiento Socialista}
\item
  \href{https://achalmaedison.netlify.app/filosofia-politica/posts/2023-03-03-el-capitalismo/index.pdf}{\faIcon{file-pdf}}
  \href{https://achalmaedison.netlify.app/filosofia-politica/posts/2023-03-03-el-capitalismo}{El
  Capitalismo}
\item
  \href{https://achalmaedison.netlify.app/filosofia-politica/posts/2023-04-29-primero-de-mayo/index.pdf}{\faIcon{file-pdf}}
  \href{https://achalmaedison.netlify.app/filosofia-politica/posts/2023-04-29-primero-de-mayo}{Primero
  De Mayo}
\item
  \href{https://achalmaedison.netlify.app/filosofia-politica/posts/2023-05-19-seminario-de-filosofia-marxista/index.pdf}{\faIcon{file-pdf}}
  \href{https://achalmaedison.netlify.app/filosofia-politica/posts/2023-05-19-seminario-de-filosofia-marxista}{Seminario
  De Filosofia Marxista}
\item
  \href{https://achalmaedison.netlify.app/filosofia-politica/posts/2023-06-09-entendiendo-a-mariategui/index.pdf}{\faIcon{file-pdf}}
  \href{https://achalmaedison.netlify.app/filosofia-politica/posts/2023-06-09-entendiendo-a-mariategui}{Entendiendo
  A Mariategui}
\item
  \href{https://achalmaedison.netlify.app/filosofia-politica/posts/2023-06-09-naturaleza-humana/index.pdf}{\faIcon{file-pdf}}
  \href{https://achalmaedison.netlify.app/filosofia-politica/posts/2023-06-09-naturaleza-humana}{Naturaleza
  Humana}
\item
  \href{https://achalmaedison.netlify.app/filosofia-politica/posts/2023-10-23-tres-fuentes-tres-partes-del-marxismo/index.pdf}{\faIcon{file-pdf}}
  \href{https://achalmaedison.netlify.app/filosofia-politica/posts/2023-10-23-tres-fuentes-tres-partes-del-marxismo}{Tres
  Fuentes Tres Partes Del Marxismo}
\end{enumerate}

Esperamos que encuentres estas publicaciones igualmente interesantes y
útiles. ¡Disfruta de la lectura!






\end{document}
