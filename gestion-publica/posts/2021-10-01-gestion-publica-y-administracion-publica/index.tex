\documentclass[
  jou,
  floatsintext,
  longtable,
  a4paper,
  nolmodern,
  notxfonts,
  notimes,
  colorlinks=true,linkcolor=blue,citecolor=blue,urlcolor=blue]{apa7}

\usepackage{amsmath}
\usepackage{amssymb}



\usepackage[bidi=default]{babel}
\babelprovide[main,import]{spanish}
\StartBabelCommands{spanish}{captions} [unicode, fontenc=TU EU1 EU2, charset=utf8] \SetString{\keywordname}{Palabras
Claves}
\EndBabelCommands


% get rid of language-specific shorthands (see #6817):
\let\LanguageShortHands\languageshorthands
\def\languageshorthands#1{}

\RequirePackage{longtable}
\RequirePackage{threeparttablex}

\makeatletter
\renewcommand{\paragraph}{\@startsection{paragraph}{4}{\parindent}%
	{0\baselineskip \@plus 0.2ex \@minus 0.2ex}%
	{-.5em}%
	{\normalfont\normalsize\bfseries\typesectitle}}

\renewcommand{\subparagraph}[1]{\@startsection{subparagraph}{5}{0.5em}%
	{0\baselineskip \@plus 0.2ex \@minus 0.2ex}%
	{-\z@\relax}%
	{\normalfont\normalsize\bfseries\itshape\hspace{\parindent}{#1}\textit{\addperi}}{\relax}}
\makeatother




\usepackage{longtable, booktabs, multirow, multicol, colortbl, hhline, caption, array, float, xpatch}
\usepackage{subcaption}
\renewcommand\thesubfigure{\Alph{subfigure}}
\setcounter{topnumber}{2}
\setcounter{bottomnumber}{2}
\setcounter{totalnumber}{4}
\renewcommand{\topfraction}{0.85}
\renewcommand{\bottomfraction}{0.85}
\renewcommand{\textfraction}{0.15}
\renewcommand{\floatpagefraction}{0.7}

\usepackage{tcolorbox}
\tcbuselibrary{listings,theorems, breakable, skins}
\usepackage{fontawesome5}

\definecolor{quarto-callout-color}{HTML}{909090}
\definecolor{quarto-callout-note-color}{HTML}{0758E5}
\definecolor{quarto-callout-important-color}{HTML}{CC1914}
\definecolor{quarto-callout-warning-color}{HTML}{EB9113}
\definecolor{quarto-callout-tip-color}{HTML}{00A047}
\definecolor{quarto-callout-caution-color}{HTML}{FC5300}
\definecolor{quarto-callout-color-frame}{HTML}{ACACAC}
\definecolor{quarto-callout-note-color-frame}{HTML}{4582EC}
\definecolor{quarto-callout-important-color-frame}{HTML}{D9534F}
\definecolor{quarto-callout-warning-color-frame}{HTML}{F0AD4E}
\definecolor{quarto-callout-tip-color-frame}{HTML}{02B875}
\definecolor{quarto-callout-caution-color-frame}{HTML}{FD7E14}

%\newlength\Oldarrayrulewidth
%\newlength\Oldtabcolsep


\usepackage{hyperref}




\providecommand{\tightlist}{%
  \setlength{\itemsep}{0pt}\setlength{\parskip}{0pt}}
\usepackage{longtable,booktabs,array}
\usepackage{calc} % for calculating minipage widths
% Correct order of tables after \paragraph or \subparagraph
\usepackage{etoolbox}
\makeatletter
\patchcmd\longtable{\par}{\if@noskipsec\mbox{}\fi\par}{}{}
\makeatother
% Allow footnotes in longtable head/foot
\IfFileExists{footnotehyper.sty}{\usepackage{footnotehyper}}{\usepackage{footnote}}
\makesavenoteenv{longtable}

\usepackage{graphicx}
\makeatletter
\newsavebox\pandoc@box
\newcommand*\pandocbounded[1]{% scales image to fit in text height/width
  \sbox\pandoc@box{#1}%
  \Gscale@div\@tempa{\textheight}{\dimexpr\ht\pandoc@box+\dp\pandoc@box\relax}%
  \Gscale@div\@tempb{\linewidth}{\wd\pandoc@box}%
  \ifdim\@tempb\p@<\@tempa\p@\let\@tempa\@tempb\fi% select the smaller of both
  \ifdim\@tempa\p@<\p@\scalebox{\@tempa}{\usebox\pandoc@box}%
  \else\usebox{\pandoc@box}%
  \fi%
}
% Set default figure placement to htbp
\def\fps@figure{htbp}
\makeatother







\usepackage{newtx}

\defaultfontfeatures{Scale=MatchLowercase}
\defaultfontfeatures[\rmfamily]{Ligatures=TeX,Scale=1}





\title{Diferencias entre Gestión Pública y Administración Pública:
Aplicaciones en el Contexto Gubernamental}


\shorttitle{Gestión vs.~Administración Pública}


\usepackage{etoolbox}



\ccoppy{\textcopyright~2021}



\author{Edison Achalma}



\affiliation{
{Escuela Profesional de Economía, Universidad Nacional de San Cristóbal
de Huamanga}}




\leftheader{Achalma}

\date{2021-10-01}


\abstract{This article examines the concepts of public management and
public administration, detailing their definitions, differences, and
applications within the governmental context. Public administration
involves the efficient handling of public resources by various
institutions across different government levels, focusing on
bureaucratic structures and resource safeguarding. On the other hand,
public management emphasizes the use of appropriate means to achieve
collective goals through policy implementation, resource allocation, and
program management. The paper discusses how these two areas, while
related, differ in their dynamism, objectives, and impact on public
life, providing concrete examples like the ``Pensión 65'' program and
the Jorge Chávez Airport expansion to illustrate their practical
application. }

\keywords{public administration, public management, resource
allocation, policy implementation, public sector}

\authornote{\par{\addORCIDlink{Edison Achalma}{0000-0001-6996-3364}} 
\par{ }
\par{   El autor no tiene conflictos de interés que revelar.    Los
roles de autor se clasificaron utilizando la taxonomía de roles de
colaborador (CRediT; https://credit.niso.org/) de la siguiente
manera:  Edison Achalma:   conceptualización, redacción}
\par{La correspondencia relativa a este artículo debe dirigirse a Edison
Achalma, Email: \href{mailto:elmer.achalma.09@unsch.edu.pe}{elmer.achalma.09@unsch.edu.pe}}
}

\usepackage{pbalance} 
\usepackage{float}
\makeatletter
\let\oldtpt\ThreePartTable
\let\endoldtpt\endThreePartTable
\def\ThreePartTable{\@ifnextchar[\ThreePartTable@i \ThreePartTable@ii}
\def\ThreePartTable@i[#1]{\begin{figure}[!htbp]
\onecolumn
\begin{minipage}{0.5\textwidth}
\oldtpt[#1]
}
\def\ThreePartTable@ii{\begin{figure}[!htbp]
\onecolumn
\begin{minipage}{0.5\textwidth}
\oldtpt
}
\def\endThreePartTable{
\endoldtpt
\end{minipage}
\twocolumn
\end{figure}}
\makeatother


\makeatletter
\let\endoldlt\endlongtable		
\def\endlongtable{
\hline
\endoldlt}
\makeatother

\newenvironment{twocolumntable}% environment name
{% begin code
\begin{table*}[!htbp]%
\onecolumn%
}%
{%
\twocolumn%
\end{table*}%
}% end code

\urlstyle{same}



\makeatletter
\@ifpackageloaded{caption}{}{\usepackage{caption}}
\AtBeginDocument{%
\ifdefined\contentsname
  \renewcommand*\contentsname{Tabla de contenidos}
\else
  \newcommand\contentsname{Tabla de contenidos}
\fi
\ifdefined\listfigurename
  \renewcommand*\listfigurename{Listado de Figuras}
\else
  \newcommand\listfigurename{Listado de Figuras}
\fi
\ifdefined\listtablename
  \renewcommand*\listtablename{Listado de Tablas}
\else
  \newcommand\listtablename{Listado de Tablas}
\fi
\ifdefined\figurename
  \renewcommand*\figurename{Figura}
\else
  \newcommand\figurename{Figura}
\fi
\ifdefined\tablename
  \renewcommand*\tablename{Tabla}
\else
  \newcommand\tablename{Tabla}
\fi
}
\@ifpackageloaded{float}{}{\usepackage{float}}
\floatstyle{ruled}
\@ifundefined{c@chapter}{\newfloat{codelisting}{h}{lop}}{\newfloat{codelisting}{h}{lop}[chapter]}
\floatname{codelisting}{Listado}
\newcommand*\listoflistings{\listof{codelisting}{Listado de Listados}}
\makeatother
\makeatletter
\makeatother
\makeatletter
\@ifpackageloaded{caption}{}{\usepackage{caption}}
\@ifpackageloaded{subcaption}{}{\usepackage{subcaption}}
\makeatother
\makeatletter
\@ifpackageloaded{fontawesome5}{}{\usepackage{fontawesome5}}
\makeatother

% From https://tex.stackexchange.com/a/645996/211326
%%% apa7 doesn't want to add appendix section titles in the toc
%%% let's make it do it
\makeatletter
\xpatchcmd{\appendix}
  {\par}
  {\addcontentsline{toc}{section}{\@currentlabelname}\par}
  {}{}
\makeatother

%% Disable longtable counter
%% https://tex.stackexchange.com/a/248395/211326

\usepackage{etoolbox}

\makeatletter
\patchcmd{\LT@caption}
  {\bgroup}
  {\bgroup\global\LTpatch@captiontrue}
  {}{}
\patchcmd{\longtable}
  {\par}
  {\par\global\LTpatch@captionfalse}
  {}{}
\apptocmd{\endlongtable}
  {\ifLTpatch@caption\else\addtocounter{table}{-1}\fi}
  {}{}
\newif\ifLTpatch@caption
\makeatother

\begin{document}

\maketitle

\hypertarget{toc}{}
\tableofcontents
\newpage
\section[Introduction]{Diferencias entre Gestión Pública y
Administración Pública}

\setcounter{secnumdepth}{-\maxdimen} % remove section numbering

\setlength\LTleft{0pt}


\section{Explorando Definiciones, Conceptos y Aplicación de Gestión
Pública y Administración
Pública}\label{explorando-definiciones-conceptos-y-aplicaciuxf3n-de-gestiuxf3n-puxfablica-y-administraciuxf3n-puxfablica}

\subsection{Administración Pública}\label{administraciuxf3n-puxfablica}

La administración pública comprende las actividades ejecutadas para el
manejo eficiente de los recursos públicos. Estas pueden ser llevadas a
cabo por instituciones u organismos tanto públicos como privados, que
operan en cualquiera de los tres niveles de gobierno: nacional, regional
o local.

La administración pública se relaciona estrechamente con dos conceptos
fundamentales: la burocracia y el resguardo de los recursos del Estado.
Además, se caracteriza por tener objetivos que no siempre están
claramente definidos. Es una estructura duradera, ya que está
constituida por las instituciones del Estado.

\subsection{Gestión Pública}\label{gestiuxf3n-puxfablica}

La gestión pública se centra en el empleo de medios adecuados para
alcanzar objetivos colectivos, es decir, en las herramientas de decisión
para la asignación y distribución de los recursos públicos. En resumen,
es el conjunto de acciones y decisiones que las entidades de la
administración pública toman para cumplir con sus fines, objetivos y
metas específicos.

Los objetivos de la gestión pública están bien definidos tanto a corto
como a largo plazo.

\subsection{Diferencia entre Gestión Pública y Administración
Pública}\label{diferencia-entre-gestiuxf3n-puxfablica-y-administraciuxf3n-puxfablica}

\textbf{Gestión Pública:}

\begin{itemize}
\tightlist
\item
  Es un conjunto de procesos y acciones destinados a alcanzar fines,
  objetivos y metas específicos.
\item
  Sus resultados se logran a través de la gestión de políticas, recursos
  y programas.
\item
  Es más práctica y orientada a cumplir objetivos colectivos.
\item
  Puede cambiar con los gobiernos, adoptando diferentes enfoques
  políticos e ideológicos.
\item
  Se centra en la eficiencia y eficacia de las acciones gubernamentales.
\end{itemize}

\textbf{Administración Pública:}

\begin{itemize}
\tightlist
\item
  Es un conjunto de organizaciones con una estructura piramidal y
  jerárquica, basada en principios de legalidad, eficiencia y eficacia.
\item
  Está sujeta a los gobiernos de turno para canalizar y resolver
  demandas sociales.
\item
  Es más estable y perdura en el tiempo.
\item
  Puede verse influenciada por el ``cuoteo político'', donde los
  intereses políticos pueden prevalecer sobre el bien común.
\end{itemize}

\subsection{¿Son Sinónimos la Gestión Pública y la Administración
Pública?}\label{son-sinuxf3nimos-la-gestiuxf3n-puxfablica-y-la-administraciuxf3n-puxfablica}

Desde una perspectiva lingüística, para que dos términos sean sinónimos,
deben compartir el mismo campo semántico, clase gramatical y, lo más
importante, poseer significados similares. La administración pública se
enfoca en las actividades para el manejo de recursos, mientras que la
gestión pública trata sobre el uso de esos recursos para alcanzar
objetivos colectivos. A pesar de compartir campos semánticos y
gramaticales, sus significados, aunque cercanos, no son idénticos. Por
lo tanto, aunque tienen significados relacionados, no son sinónimos
exactos en el contexto de economía, administración y sector público.

\subsection{Importancia de la Gestión y Administración Pública en la
Gestión
Gubernamental}\label{importancia-de-la-gestiuxf3n-y-administraciuxf3n-puxfablica-en-la-gestiuxf3n-gubernamental}

\textbf{Gestión Pública:}

Es crucial porque determina cómo se utilizan los recursos públicos para
cumplir con planes y políticas en diferentes niveles (nacional,
sectorial, regional, local). La gestión pública es un conjunto de
instrumentos que se aplican para desarrollar las fuerzas productivas,
impactando directamente en la vida social y en cómo se gobierna y
administra a la sociedad. Refleja la capacidad institucional de los
gobiernos.

\textbf{Administración Pública:}

Su importancia radica en su conexión directa con la vida diaria de los
ciudadanos, proporcionando servicios públicos esenciales para la
convivencia comunitaria como transporte, seguridad y medio ambiente.

\subsection{Campo de Aplicación de la Gestión Pública y la
Administración Pública - Casos y
Ejemplos}\label{campo-de-aplicaciuxf3n-de-la-gestiuxf3n-puxfablica-y-la-administraciuxf3n-puxfablica---casos-y-ejemplos}

\textbf{Gestión Pública:}

\begin{itemize}
\tightlist
\item
  Opera a través de autoridades políticas y servidores públicos que
  toman decisiones y ejecutan acciones en el marco de sus respectivas
  leyes de creación. Los decisores políticos adoptan políticas para
  solucionar problemas comunitarios, mientras que los servidores
  públicos implementan estas políticas utilizando tecnologías de gestión
  y herramientas normativas y gerenciales.
\end{itemize}

\textbf{Administración Pública:}

\begin{itemize}
\tightlist
\item
  Es un campo donde los líderes promueven el bienestar comunitario y el
  cambio positivo en el sector público. Incluye la gestión de proyectos
  y programas, la supervisión de empleados y la evaluación de servicios
  y políticas públicas.
\end{itemize}

\textbf{Ejemplos:}

\begin{itemize}
\tightlist
\item
  \textbf{Programa Pensión 65:} Con su plan ``Saberes Productivos'',
  ofrece protección a personas mayores en situación de vulnerabilidad.
\item
  \textbf{Proyecto de Ampliación del Aeropuerto Jorge Chávez:} A cargo
  de Lima Airport Partners (LAP), busca modernizar el aeropuerto para
  2024, promoviendo el turismo, comercio y comunicaciones.
\end{itemize}

\section{Publicaciones Similares}\label{publicaciones-similares}

Si te interesó este artículo, te recomendamos que explores otros blogs y
recursos relacionados que pueden ampliar tus conocimientos. Aquí te dejo
algunas sugerencias:

\begin{enumerate}
\def\labelenumi{\arabic{enumi}.}
\tightlist
\item
  \href{https://achalmaedison.netlify.app/gestion-publica-herramientas/posts/2024-03-31-por-editar/index.pdf}{\faIcon{file-pdf}}
  \href{https://achalmaedison.netlify.app/gestion-publica-herramientas/posts/2024-03-31-por-editar}{Por
  Editar}
\end{enumerate}

Esperamos que encuentres estas publicaciones igualmente interesantes y
útiles. ¡Disfruta de la lectura!






\end{document}
