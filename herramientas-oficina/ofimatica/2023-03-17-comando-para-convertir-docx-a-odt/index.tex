\documentclass[
  jou,
  floatsintext,
  longtable,
  a4paper,
  nolmodern,
  notxfonts,
  notimes,
  colorlinks=true,linkcolor=blue,citecolor=blue,urlcolor=blue]{apa7}

\usepackage{amsmath}
\usepackage{amssymb}



\usepackage[bidi=default]{babel}
\babelprovide[main,import]{spanish}
\StartBabelCommands{spanish}{captions} [unicode, fontenc=TU EU1 EU2, charset=utf8] \SetString{\keywordname}{Palabras
Claves}
\EndBabelCommands


% get rid of language-specific shorthands (see #6817):
\let\LanguageShortHands\languageshorthands
\def\languageshorthands#1{}

\RequirePackage{longtable}
\RequirePackage{threeparttablex}

\makeatletter
\renewcommand{\paragraph}{\@startsection{paragraph}{4}{\parindent}%
	{0\baselineskip \@plus 0.2ex \@minus 0.2ex}%
	{-.5em}%
	{\normalfont\normalsize\bfseries\typesectitle}}

\renewcommand{\subparagraph}[1]{\@startsection{subparagraph}{5}{0.5em}%
	{0\baselineskip \@plus 0.2ex \@minus 0.2ex}%
	{-\z@\relax}%
	{\normalfont\normalsize\bfseries\itshape\hspace{\parindent}{#1}\textit{\addperi}}{\relax}}
\makeatother




\usepackage{longtable, booktabs, multirow, multicol, colortbl, hhline, caption, array, float, xpatch}
\usepackage{subcaption}
\renewcommand\thesubfigure{\Alph{subfigure}}
\setcounter{topnumber}{2}
\setcounter{bottomnumber}{2}
\setcounter{totalnumber}{4}
\renewcommand{\topfraction}{0.85}
\renewcommand{\bottomfraction}{0.85}
\renewcommand{\textfraction}{0.15}
\renewcommand{\floatpagefraction}{0.7}

\usepackage{tcolorbox}
\tcbuselibrary{listings,theorems, breakable, skins}
\usepackage{fontawesome5}

\definecolor{quarto-callout-color}{HTML}{909090}
\definecolor{quarto-callout-note-color}{HTML}{0758E5}
\definecolor{quarto-callout-important-color}{HTML}{CC1914}
\definecolor{quarto-callout-warning-color}{HTML}{EB9113}
\definecolor{quarto-callout-tip-color}{HTML}{00A047}
\definecolor{quarto-callout-caution-color}{HTML}{FC5300}
\definecolor{quarto-callout-color-frame}{HTML}{ACACAC}
\definecolor{quarto-callout-note-color-frame}{HTML}{4582EC}
\definecolor{quarto-callout-important-color-frame}{HTML}{D9534F}
\definecolor{quarto-callout-warning-color-frame}{HTML}{F0AD4E}
\definecolor{quarto-callout-tip-color-frame}{HTML}{02B875}
\definecolor{quarto-callout-caution-color-frame}{HTML}{FD7E14}

%\newlength\Oldarrayrulewidth
%\newlength\Oldtabcolsep


\usepackage{hyperref}



\usepackage{color}
\usepackage{fancyvrb}
\newcommand{\VerbBar}{|}
\newcommand{\VERB}{\Verb[commandchars=\\\{\}]}
\DefineVerbatimEnvironment{Highlighting}{Verbatim}{commandchars=\\\{\}}
% Add ',fontsize=\small' for more characters per line
\usepackage{framed}
\definecolor{shadecolor}{RGB}{241,243,245}
\newenvironment{Shaded}{\begin{snugshade}}{\end{snugshade}}
\newcommand{\AlertTok}[1]{\textcolor[rgb]{0.68,0.00,0.00}{#1}}
\newcommand{\AnnotationTok}[1]{\textcolor[rgb]{0.37,0.37,0.37}{#1}}
\newcommand{\AttributeTok}[1]{\textcolor[rgb]{0.40,0.45,0.13}{#1}}
\newcommand{\BaseNTok}[1]{\textcolor[rgb]{0.68,0.00,0.00}{#1}}
\newcommand{\BuiltInTok}[1]{\textcolor[rgb]{0.00,0.23,0.31}{#1}}
\newcommand{\CharTok}[1]{\textcolor[rgb]{0.13,0.47,0.30}{#1}}
\newcommand{\CommentTok}[1]{\textcolor[rgb]{0.37,0.37,0.37}{#1}}
\newcommand{\CommentVarTok}[1]{\textcolor[rgb]{0.37,0.37,0.37}{\textit{#1}}}
\newcommand{\ConstantTok}[1]{\textcolor[rgb]{0.56,0.35,0.01}{#1}}
\newcommand{\ControlFlowTok}[1]{\textcolor[rgb]{0.00,0.23,0.31}{\textbf{#1}}}
\newcommand{\DataTypeTok}[1]{\textcolor[rgb]{0.68,0.00,0.00}{#1}}
\newcommand{\DecValTok}[1]{\textcolor[rgb]{0.68,0.00,0.00}{#1}}
\newcommand{\DocumentationTok}[1]{\textcolor[rgb]{0.37,0.37,0.37}{\textit{#1}}}
\newcommand{\ErrorTok}[1]{\textcolor[rgb]{0.68,0.00,0.00}{#1}}
\newcommand{\ExtensionTok}[1]{\textcolor[rgb]{0.00,0.23,0.31}{#1}}
\newcommand{\FloatTok}[1]{\textcolor[rgb]{0.68,0.00,0.00}{#1}}
\newcommand{\FunctionTok}[1]{\textcolor[rgb]{0.28,0.35,0.67}{#1}}
\newcommand{\ImportTok}[1]{\textcolor[rgb]{0.00,0.46,0.62}{#1}}
\newcommand{\InformationTok}[1]{\textcolor[rgb]{0.37,0.37,0.37}{#1}}
\newcommand{\KeywordTok}[1]{\textcolor[rgb]{0.00,0.23,0.31}{\textbf{#1}}}
\newcommand{\NormalTok}[1]{\textcolor[rgb]{0.00,0.23,0.31}{#1}}
\newcommand{\OperatorTok}[1]{\textcolor[rgb]{0.37,0.37,0.37}{#1}}
\newcommand{\OtherTok}[1]{\textcolor[rgb]{0.00,0.23,0.31}{#1}}
\newcommand{\PreprocessorTok}[1]{\textcolor[rgb]{0.68,0.00,0.00}{#1}}
\newcommand{\RegionMarkerTok}[1]{\textcolor[rgb]{0.00,0.23,0.31}{#1}}
\newcommand{\SpecialCharTok}[1]{\textcolor[rgb]{0.37,0.37,0.37}{#1}}
\newcommand{\SpecialStringTok}[1]{\textcolor[rgb]{0.13,0.47,0.30}{#1}}
\newcommand{\StringTok}[1]{\textcolor[rgb]{0.13,0.47,0.30}{#1}}
\newcommand{\VariableTok}[1]{\textcolor[rgb]{0.07,0.07,0.07}{#1}}
\newcommand{\VerbatimStringTok}[1]{\textcolor[rgb]{0.13,0.47,0.30}{#1}}
\newcommand{\WarningTok}[1]{\textcolor[rgb]{0.37,0.37,0.37}{\textit{#1}}}

\providecommand{\tightlist}{%
  \setlength{\itemsep}{0pt}\setlength{\parskip}{0pt}}
\usepackage{longtable,booktabs,array}
\usepackage{calc} % for calculating minipage widths
% Correct order of tables after \paragraph or \subparagraph
\usepackage{etoolbox}
\makeatletter
\patchcmd\longtable{\par}{\if@noskipsec\mbox{}\fi\par}{}{}
\makeatother
% Allow footnotes in longtable head/foot
\IfFileExists{footnotehyper.sty}{\usepackage{footnotehyper}}{\usepackage{footnote}}
\makesavenoteenv{longtable}

\usepackage{graphicx}
\makeatletter
\newsavebox\pandoc@box
\newcommand*\pandocbounded[1]{% scales image to fit in text height/width
  \sbox\pandoc@box{#1}%
  \Gscale@div\@tempa{\textheight}{\dimexpr\ht\pandoc@box+\dp\pandoc@box\relax}%
  \Gscale@div\@tempb{\linewidth}{\wd\pandoc@box}%
  \ifdim\@tempb\p@<\@tempa\p@\let\@tempa\@tempb\fi% select the smaller of both
  \ifdim\@tempa\p@<\p@\scalebox{\@tempa}{\usebox\pandoc@box}%
  \else\usebox{\pandoc@box}%
  \fi%
}
% Set default figure placement to htbp
\def\fps@figure{htbp}
\makeatother







\usepackage{newtx}

\defaultfontfeatures{Scale=MatchLowercase}
\defaultfontfeatures[\rmfamily]{Ligatures=TeX,Scale=1}





\title{Editar: Editar}


\shorttitle{Editar}


\usepackage{etoolbox}



\ccoppy{\textcopyright~2023}



\author{Edison Achalma}



\affiliation{
{Escuela Profesional de Economía, Universidad Nacional de San Cristóbal
de Huamanga}}




\leftheader{Achalma}

\date{2024-03-31}


\abstract{Este abstract será actualizado una vez que se complete el
contenido final del artículo. }

\keywords{keyword1, keyword2}

\authornote{\par{\addORCIDlink{Edison Achalma}{0000-0001-6996-3364}} 
\par{ }
\par{   El autor no tiene conflictos de interés que revelar.    Los
roles de autor se clasificaron utilizando la taxonomía de roles de
colaborador (CRediT; https://credit.niso.org/) de la siguiente
manera:  Edison Achalma:   conceptualización, redacción}
\par{La correspondencia relativa a este artículo debe dirigirse a Edison
Achalma, Email: \href{mailto:elmer.achalma.09@unsch.edu.pe}{elmer.achalma.09@unsch.edu.pe}}
}

\usepackage{pbalance} 
\usepackage{float}
\makeatletter
\let\oldtpt\ThreePartTable
\let\endoldtpt\endThreePartTable
\def\ThreePartTable{\@ifnextchar[\ThreePartTable@i \ThreePartTable@ii}
\def\ThreePartTable@i[#1]{\begin{figure}[!htbp]
\onecolumn
\begin{minipage}{0.5\textwidth}
\oldtpt[#1]
}
\def\ThreePartTable@ii{\begin{figure}[!htbp]
\onecolumn
\begin{minipage}{0.5\textwidth}
\oldtpt
}
\def\endThreePartTable{
\endoldtpt
\end{minipage}
\twocolumn
\end{figure}}
\makeatother


\makeatletter
\let\endoldlt\endlongtable		
\def\endlongtable{
\hline
\endoldlt}
\makeatother

\newenvironment{twocolumntable}% environment name
{% begin code
\begin{table*}[!htbp]%
\onecolumn%
}%
{%
\twocolumn%
\end{table*}%
}% end code

\urlstyle{same}



\makeatletter
\@ifpackageloaded{caption}{}{\usepackage{caption}}
\AtBeginDocument{%
\ifdefined\contentsname
  \renewcommand*\contentsname{Tabla de contenidos}
\else
  \newcommand\contentsname{Tabla de contenidos}
\fi
\ifdefined\listfigurename
  \renewcommand*\listfigurename{Listado de Figuras}
\else
  \newcommand\listfigurename{Listado de Figuras}
\fi
\ifdefined\listtablename
  \renewcommand*\listtablename{Listado de Tablas}
\else
  \newcommand\listtablename{Listado de Tablas}
\fi
\ifdefined\figurename
  \renewcommand*\figurename{Figura}
\else
  \newcommand\figurename{Figura}
\fi
\ifdefined\tablename
  \renewcommand*\tablename{Tabla}
\else
  \newcommand\tablename{Tabla}
\fi
}
\@ifpackageloaded{float}{}{\usepackage{float}}
\floatstyle{ruled}
\@ifundefined{c@chapter}{\newfloat{codelisting}{h}{lop}}{\newfloat{codelisting}{h}{lop}[chapter]}
\floatname{codelisting}{Listado}
\newcommand*\listoflistings{\listof{codelisting}{Listado de Listados}}
\makeatother
\makeatletter
\makeatother
\makeatletter
\@ifpackageloaded{caption}{}{\usepackage{caption}}
\@ifpackageloaded{subcaption}{}{\usepackage{subcaption}}
\makeatother
\makeatletter
\@ifpackageloaded{fontawesome5}{}{\usepackage{fontawesome5}}
\makeatother

% From https://tex.stackexchange.com/a/645996/211326
%%% apa7 doesn't want to add appendix section titles in the toc
%%% let's make it do it
\makeatletter
\xpatchcmd{\appendix}
  {\par}
  {\addcontentsline{toc}{section}{\@currentlabelname}\par}
  {}{}
\makeatother

%% Disable longtable counter
%% https://tex.stackexchange.com/a/248395/211326

\usepackage{etoolbox}

\makeatletter
\patchcmd{\LT@caption}
  {\bgroup}
  {\bgroup\global\LTpatch@captiontrue}
  {}{}
\patchcmd{\longtable}
  {\par}
  {\par\global\LTpatch@captionfalse}
  {}{}
\apptocmd{\endlongtable}
  {\ifLTpatch@caption\else\addtocounter{table}{-1}\fi}
  {}{}
\newif\ifLTpatch@caption
\makeatother

\begin{document}

\maketitle

\hypertarget{toc}{}
\tableofcontents
\newpage
\section[Introduction]{Editar}

\setcounter{secnumdepth}{-\maxdimen} % remove section numbering

\setlength\LTleft{0pt}


Para convertir varios archivos con extensión \texttt{.docx} a
\texttt{.odt} a través de una línea de comando de Linux, puede utilizar
la herramienta~\texttt{soffice}~que viene con LibreOffice. Aquí hay un
ejemplo de cómo hacerlo de \texttt{docx} a \texttt{odt}

\texttt{soffice\ -\/-headless\ -\/-convert-to\ odt\ *.docx}

Comados para recuperar docx dañados

\texttt{sudo\ apt-get\ install\ docx2txt}

de doc a odt

\texttt{soffice\ -\/-headless\ -\/-convert-to\ odt\ *.doc}

de doc a docx

\texttt{soffice\ -\/-headless\ -\/-convert-to\ docx\ *.doc}

\paragraph{Extensiones de ODF.}\label{extensiones-de-odf}

Entre las extensiones típicas de los archivos ODF están las siguientes:

\begin{itemize}
\tightlist
\item
  \textbf{.odt}~-- documento de texto
\item
  \textbf{.ods}~-- libro de hojas de cálculo
\item
  \textbf{.odp}~-- presentación de diapositivas
\item
  \textbf{.odg}~-- ilustración o gráfico
\end{itemize}

\section{me manera masiva:}\label{me-manera-masiva}

Para convertir archivos de Microsoft Office a formatos de LibreOffice
utilizando la línea de comandos, puedes utilizar la herramienta
\texttt{soffice} (LibreOffice). Aquí tienes el comando general para
convertir archivos \texttt{.docx} a \texttt{.odt}:

\begin{Shaded}
\begin{Highlighting}[]
\ExtensionTok{soffice} \AttributeTok{{-}{-}headless} \AttributeTok{{-}{-}convert{-}to}\NormalTok{ odt }\PreprocessorTok{*}\NormalTok{.docx}
\end{Highlighting}
\end{Shaded}

Si deseas convertir varios tipos de archivos en varias carpetas, puedes
usar un script de shell para recorrer las carpetas y convertir los
archivos. A continuación, se presenta un ejemplo de cómo podrías hacerlo
en un entorno Unix (como Linux o macOS):

\begin{enumerate}
\def\labelenumi{\arabic{enumi}.}
\tightlist
\item
  Abre una terminal.
\item
  Navega a la carpeta principal que contiene las subcarpetas con los
  archivos que deseas convertir.
\item
  Crea y ejecuta un script de shell con el siguiente contenido:
\end{enumerate}

\begin{Shaded}
\begin{Highlighting}[]
\CommentTok{\#!/bin/bash}

\CommentTok{\# Convert DOCX to ODT}
\FunctionTok{find}\NormalTok{ . }\AttributeTok{{-}type}\NormalTok{ f }\AttributeTok{{-}name} \StringTok{"*.docx"} \AttributeTok{{-}exec}\NormalTok{ soffice }\AttributeTok{{-}{-}headless} \AttributeTok{{-}{-}convert{-}to}\NormalTok{ odt \{\} }\DataTypeTok{\textbackslash{};}

\CommentTok{\# Convert XLSX to ODS}
\FunctionTok{find}\NormalTok{ . }\AttributeTok{{-}type}\NormalTok{ f }\AttributeTok{{-}name} \StringTok{"*.xlsx"} \AttributeTok{{-}exec}\NormalTok{ soffice }\AttributeTok{{-}{-}headless} \AttributeTok{{-}{-}convert{-}to}\NormalTok{ ods \{\} }\DataTypeTok{\textbackslash{};}

\CommentTok{\# Convert PPTX to ODP}
\FunctionTok{find}\NormalTok{ . }\AttributeTok{{-}type}\NormalTok{ f }\AttributeTok{{-}name} \StringTok{"*.pptx"} \AttributeTok{{-}exec}\NormalTok{ soffice }\AttributeTok{{-}{-}headless} \AttributeTok{{-}{-}convert{-}to}\NormalTok{ odp \{\} }\DataTypeTok{\textbackslash{};}
\end{Highlighting}
\end{Shaded}

Guarda el script con un nombre, por ejemplo \texttt{convert\_files.sh},
y dale permisos de ejecución:

\begin{Shaded}
\begin{Highlighting}[]
\FunctionTok{chmod}\NormalTok{ +x convert\_files.sh}
\end{Highlighting}
\end{Shaded}

Luego, ejecuta el script:

\begin{Shaded}
\begin{Highlighting}[]
\ExtensionTok{./convert\_files.sh}
\end{Highlighting}
\end{Shaded}

Este script hará lo siguiente: -
\texttt{find\ .\ -type\ f\ -name\ "*.docx"}: Encuentra todos los
archivos \texttt{.docx} en el directorio actual y sus subdirectorios. -
\texttt{-exec\ soffice\ -\/-headless\ -\/-convert-to\ odt\ \{\}}: Para
cada archivo \texttt{.docx} encontrado, ejecuta el comando
\texttt{soffice\ -\/-headless\ -\/-convert-to\ odt} para convertirlo a
\texttt{.odt}.

El mismo proceso se aplica a los archivos \texttt{.xlsx} y
\texttt{.pptx} para convertirlos a \texttt{.ods} y \texttt{.odp},
respectivamente.

\textbf{Nota:} Asegúrate de tener \texttt{soffice} (LibreOffice)
instalado y accesible desde la línea de comandos.

\section{con variants:}\label{con-variants}

Si deseas convertir diferentes tipos de archivos de Microsoft Office a
sus respectivos formatos de LibreOffice (por ejemplo, \texttt{.docx} y
\texttt{.doc} a \texttt{.odt}, \texttt{.xlsx} y \texttt{.xls} a
\texttt{.ods}, y \texttt{.pptx} y \texttt{.ppt} a \texttt{.odp}), puedes
modificar el script para incluir todos los tipos de archivos que quieres
convertir.

Aquí tienes un ejemplo de un script de shell que maneja múltiples tipos
de archivos:

\begin{Shaded}
\begin{Highlighting}[]
\CommentTok{\#!/bin/bash}

\CommentTok{\# Convert DOCX and DOC to ODT}
\FunctionTok{find}\NormalTok{ . }\AttributeTok{{-}type}\NormalTok{ f }\DataTypeTok{\textbackslash{}(} \AttributeTok{{-}name} \StringTok{"*.docx"} \AttributeTok{{-}o} \AttributeTok{{-}name} \StringTok{"*.doc"} \DataTypeTok{\textbackslash{})} \AttributeTok{{-}exec}\NormalTok{ soffice }\AttributeTok{{-}{-}headless} \AttributeTok{{-}{-}convert{-}to}\NormalTok{ odt \{\} }\DataTypeTok{\textbackslash{};}

\CommentTok{\# Convert XLSX and XLS to ODS}
\FunctionTok{find}\NormalTok{ . }\AttributeTok{{-}type}\NormalTok{ f }\DataTypeTok{\textbackslash{}(} \AttributeTok{{-}name} \StringTok{"*.xlsx"} \AttributeTok{{-}o} \AttributeTok{{-}name} \StringTok{"*.xls"} \DataTypeTok{\textbackslash{})} \AttributeTok{{-}exec}\NormalTok{ soffice }\AttributeTok{{-}{-}headless} \AttributeTok{{-}{-}convert{-}to}\NormalTok{ ods \{\} }\DataTypeTok{\textbackslash{};}

\CommentTok{\# Convert PPTX and PPT to ODP}
\FunctionTok{find}\NormalTok{ . }\AttributeTok{{-}type}\NormalTok{ f }\DataTypeTok{\textbackslash{}(} \AttributeTok{{-}name} \StringTok{"*.pptx"} \AttributeTok{{-}o} \AttributeTok{{-}name} \StringTok{"*.ppt"} \DataTypeTok{\textbackslash{})} \AttributeTok{{-}exec}\NormalTok{ soffice }\AttributeTok{{-}{-}headless} \AttributeTok{{-}{-}convert{-}to}\NormalTok{ odp \{\} }\DataTypeTok{\textbackslash{};}
\end{Highlighting}
\end{Shaded}

Guarda el script con un nombre, por ejemplo
\texttt{convert\_all\_files.sh}, y dale permisos de ejecución:

\begin{Shaded}
\begin{Highlighting}[]
\FunctionTok{chmod}\NormalTok{ +x convert\_all\_files.sh}
\end{Highlighting}
\end{Shaded}

Luego, ejecuta el script:

\begin{Shaded}
\begin{Highlighting}[]
\ExtensionTok{./convert\_all\_files.sh}
\end{Highlighting}
\end{Shaded}

Este script hará lo siguiente:

\begin{enumerate}
\def\labelenumi{\arabic{enumi}.}
\tightlist
\item
  \textbf{Convertir archivos de Word (\texttt{.docx} y \texttt{.doc}) a
  ODT:}

  \begin{itemize}
  \tightlist
  \item
    \texttt{find\ .\ -type\ f\ \textbackslash{}(\ -name\ "*.docx"\ -o\ -name\ "*.doc"\ \textbackslash{})}:
    Encuentra todos los archivos \texttt{.docx} y \texttt{.doc} en el
    directorio actual y sus subdirectorios.
  \item
    \texttt{-exec\ soffice\ -\/-headless\ -\/-convert-to\ odt\ \{\}}:
    Para cada archivo encontrado, ejecuta el comando
    \texttt{soffice\ -\/-headless\ -\/-convert-to\ odt} para convertirlo
    a \texttt{.odt}.
  \end{itemize}
\item
  \textbf{Convertir archivos de Excel (\texttt{.xlsx} y \texttt{.xls}) a
  ODS:}

  \begin{itemize}
  \tightlist
  \item
    \texttt{find\ .\ -type\ f\ \textbackslash{}(\ -name\ "*.xlsx"\ -o\ -name\ "*.xls"\ \textbackslash{})}:
    Encuentra todos los archivos \texttt{.xlsx} y \texttt{.xls} en el
    directorio actual y sus subdirectorios.
  \item
    \texttt{-exec\ soffice\ -\/-headless\ -\/-convert-to\ ods\ \{\}}:
    Para cada archivo encontrado, ejecuta el comando
    \texttt{soffice\ -\/-headless\ -\/-convert-to\ ods} para convertirlo
    a \texttt{.ods}.
  \end{itemize}
\item
  \textbf{Convertir archivos de PowerPoint (\texttt{.pptx} y
  \texttt{.ppt}) a ODP:}

  \begin{itemize}
  \tightlist
  \item
    \texttt{find\ .\ -type\ f\ \textbackslash{}(\ -name\ "*.pptx"\ -o\ -name\ "*.ppt"\ \textbackslash{})}:
    Encuentra todos los archivos \texttt{.pptx} y \texttt{.ppt} en el
    directorio actual y sus subdirectorios.
  \item
    \texttt{-exec\ soffice\ -\/-headless\ -\/-convert-to\ odp\ \{\}}:
    Para cada archivo encontrado, ejecuta el comando
    \texttt{soffice\ -\/-headless\ -\/-convert-to\ odp} para convertirlo
    a \texttt{.odp}.
  \end{itemize}
\end{enumerate}

Este script asegurará que todos los archivos de los tipos especificados
en el directorio actual y sus subdirectorios se conviertan al formato
adecuado de LibreOffice.

\section{Publicaciones Similares}\label{publicaciones-similares}

Si te interesó este artículo, te recomendamos que explores otros blogs y
recursos relacionados que pueden ampliar tus conocimientos. Aquí te dejo
algunas sugerencias:

\begin{enumerate}
\def\labelenumi{\arabic{enumi}.}
\tightlist
\item
  \href{https://achalmaedison.netlify.app/herramientas-oficina/ofimatica/2022-12-05-01-introduccion-al-lenguaje-y-editor-vba/index.pdf}{\faIcon{file-pdf}}
  \href{https://achalmaedison.netlify.app/herramientas-oficina/ofimatica/2022-12-05-01-introduccion-al-lenguaje-y-editor-vba}{01
  Introduccion Al Lenguaje Y Editor Vba}
\item
  \href{https://achalmaedison.netlify.app/herramientas-oficina/ofimatica/2022-12-12-02-grabar-y-modificar/index.pdf}{\faIcon{file-pdf}}
  \href{https://achalmaedison.netlify.app/herramientas-oficina/ofimatica/2022-12-12-02-grabar-y-modificar}{02
  Grabar Y Modificar}
\item
  \href{https://achalmaedison.netlify.app/herramientas-oficina/ofimatica/2022-12-19-03-procedimientos/index.pdf}{\faIcon{file-pdf}}
  \href{https://achalmaedison.netlify.app/herramientas-oficina/ofimatica/2022-12-19-03-procedimientos}{03
  Procedimientos}
\item
  \href{https://achalmaedison.netlify.app/herramientas-oficina/ofimatica/2022-12-26-04-funciones-en-vba/index.pdf}{\faIcon{file-pdf}}
  \href{https://achalmaedison.netlify.app/herramientas-oficina/ofimatica/2022-12-26-04-funciones-en-vba}{04
  Funciones En Vba}
\item
  \href{https://achalmaedison.netlify.app/herramientas-oficina/ofimatica/2023-01-02-05-funciones-condicionales-estructuras-condicionales/index.pdf}{\faIcon{file-pdf}}
  \href{https://achalmaedison.netlify.app/herramientas-oficina/ofimatica/2023-01-02-05-funciones-condicionales-estructuras-condicionales}{05
  Funciones Condicionales Estructuras Condicionales}
\item
  \href{https://achalmaedison.netlify.app/herramientas-oficina/ofimatica/2023-01-09-06-funciones-iterativas-estructuras-repetitivas-o-bucles/index.pdf}{\faIcon{file-pdf}}
  \href{https://achalmaedison.netlify.app/herramientas-oficina/ofimatica/2023-01-09-06-funciones-iterativas-estructuras-repetitivas-o-bucles}{06
  Funciones Iterativas Estructuras Repetitivas O Bucles}
\item
  \href{https://achalmaedison.netlify.app/herramientas-oficina/ofimatica/2023-01-16-07-formularios/index.pdf}{\faIcon{file-pdf}}
  \href{https://achalmaedison.netlify.app/herramientas-oficina/ofimatica/2023-01-16-07-formularios}{07
  Formularios}
\item
  \href{https://achalmaedison.netlify.app/herramientas-oficina/ofimatica/2023-01-23-08-eventos/index.pdf}{\faIcon{file-pdf}}
  \href{https://achalmaedison.netlify.app/herramientas-oficina/ofimatica/2023-01-23-08-eventos}{08
  Eventos}
\item
  \href{https://achalmaedison.netlify.app/herramientas-oficina/ofimatica/2023-03-17-comando-para-convertir-docx-a-odt/index.pdf}{\faIcon{file-pdf}}
  \href{https://achalmaedison.netlify.app/herramientas-oficina/ofimatica/2023-03-17-comando-para-convertir-docx-a-odt}{Comando
  Para Convertir Docx A Odt}
\item
  \href{https://achalmaedison.netlify.app/herramientas-oficina/ofimatica/2023-04-03-buscar-reemplazar-en-libreoffice/index.pdf}{\faIcon{file-pdf}}
  \href{https://achalmaedison.netlify.app/herramientas-oficina/ofimatica/2023-04-03-buscar-reemplazar-en-libreoffice}{Buscar
  Reemplazar En Libreoffice}
\item
  \href{https://achalmaedison.netlify.app/herramientas-oficina/ofimatica/2023-05-21-anclaje-envoltura-alineacion-y-organizacion-de-objetos-en-llibreoffice/index.pdf}{\faIcon{file-pdf}}
  \href{https://achalmaedison.netlify.app/herramientas-oficina/ofimatica/2023-05-21-anclaje-envoltura-alineacion-y-organizacion-de-objetos-en-llibreoffice}{Anclaje
  Envoltura Alineacion Y Organizacion De Objetos En Llibreoffice}
\item
  \href{https://achalmaedison.netlify.app/herramientas-oficina/ofimatica/2023-05-31-combinando-hojas-de-excel-con-vba/index.pdf}{\faIcon{file-pdf}}
  \href{https://achalmaedison.netlify.app/herramientas-oficina/ofimatica/2023-05-31-combinando-hojas-de-excel-con-vba}{Combinando
  Hojas De Excel Con Vba}
\item
  \href{https://achalmaedison.netlify.app/herramientas-oficina/ofimatica/2023-06-05-separando-hojas-de-excel-en-documentos-individuales/index.pdf}{\faIcon{file-pdf}}
  \href{https://achalmaedison.netlify.app/herramientas-oficina/ofimatica/2023-06-05-separando-hojas-de-excel-en-documentos-individuales}{Separando
  Hojas De Excel En Documentos Individuales}
\item
  \href{https://achalmaedison.netlify.app/herramientas-oficina/ofimatica/2024-03-31-por-editar/index.pdf}{\faIcon{file-pdf}}
  \href{https://achalmaedison.netlify.app/herramientas-oficina/ofimatica/2024-03-31-por-editar}{Por
  Editar}
\end{enumerate}

Esperamos que encuentres estas publicaciones igualmente interesantes y
útiles. ¡Disfruta de la lectura!






\end{document}
