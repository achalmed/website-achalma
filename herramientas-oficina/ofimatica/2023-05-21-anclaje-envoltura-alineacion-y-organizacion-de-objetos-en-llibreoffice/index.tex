\documentclass[
  jou,
  floatsintext,
  longtable,
  a4paper,
  nolmodern,
  notxfonts,
  notimes,
  colorlinks=true,linkcolor=blue,citecolor=blue,urlcolor=blue]{apa7}

\usepackage{amsmath}
\usepackage{amssymb}



\usepackage[bidi=default]{babel}
\babelprovide[main,import]{spanish}
\StartBabelCommands{spanish}{captions} [unicode, fontenc=TU EU1 EU2, charset=utf8] \SetString{\keywordname}{Palabras
Claves}
\EndBabelCommands


% get rid of language-specific shorthands (see #6817):
\let\LanguageShortHands\languageshorthands
\def\languageshorthands#1{}

\RequirePackage{longtable}
\RequirePackage{threeparttablex}

\makeatletter
\renewcommand{\paragraph}{\@startsection{paragraph}{4}{\parindent}%
	{0\baselineskip \@plus 0.2ex \@minus 0.2ex}%
	{-.5em}%
	{\normalfont\normalsize\bfseries\typesectitle}}

\renewcommand{\subparagraph}[1]{\@startsection{subparagraph}{5}{0.5em}%
	{0\baselineskip \@plus 0.2ex \@minus 0.2ex}%
	{-\z@\relax}%
	{\normalfont\normalsize\bfseries\itshape\hspace{\parindent}{#1}\textit{\addperi}}{\relax}}
\makeatother




\usepackage{longtable, booktabs, multirow, multicol, colortbl, hhline, caption, array, float, xpatch}
\setcounter{topnumber}{2}
\setcounter{bottomnumber}{2}
\setcounter{totalnumber}{4}
\renewcommand{\topfraction}{0.85}
\renewcommand{\bottomfraction}{0.85}
\renewcommand{\textfraction}{0.15}
\renewcommand{\floatpagefraction}{0.7}

\usepackage{tcolorbox}
\tcbuselibrary{listings,theorems, breakable, skins}
\usepackage{fontawesome5}

\definecolor{quarto-callout-color}{HTML}{909090}
\definecolor{quarto-callout-note-color}{HTML}{0758E5}
\definecolor{quarto-callout-important-color}{HTML}{CC1914}
\definecolor{quarto-callout-warning-color}{HTML}{EB9113}
\definecolor{quarto-callout-tip-color}{HTML}{00A047}
\definecolor{quarto-callout-caution-color}{HTML}{FC5300}
\definecolor{quarto-callout-color-frame}{HTML}{ACACAC}
\definecolor{quarto-callout-note-color-frame}{HTML}{4582EC}
\definecolor{quarto-callout-important-color-frame}{HTML}{D9534F}
\definecolor{quarto-callout-warning-color-frame}{HTML}{F0AD4E}
\definecolor{quarto-callout-tip-color-frame}{HTML}{02B875}
\definecolor{quarto-callout-caution-color-frame}{HTML}{FD7E14}

%\newlength\Oldarrayrulewidth
%\newlength\Oldtabcolsep


\usepackage{hyperref}




\providecommand{\tightlist}{%
  \setlength{\itemsep}{0pt}\setlength{\parskip}{0pt}}
\usepackage{longtable,booktabs,array}
\usepackage{calc} % for calculating minipage widths
% Correct order of tables after \paragraph or \subparagraph
\usepackage{etoolbox}
\makeatletter
\patchcmd\longtable{\par}{\if@noskipsec\mbox{}\fi\par}{}{}
\makeatother
% Allow footnotes in longtable head/foot
\IfFileExists{footnotehyper.sty}{\usepackage{footnotehyper}}{\usepackage{footnote}}
\makesavenoteenv{longtable}

\usepackage{graphicx}
\makeatletter
\newsavebox\pandoc@box
\newcommand*\pandocbounded[1]{% scales image to fit in text height/width
  \sbox\pandoc@box{#1}%
  \Gscale@div\@tempa{\textheight}{\dimexpr\ht\pandoc@box+\dp\pandoc@box\relax}%
  \Gscale@div\@tempb{\linewidth}{\wd\pandoc@box}%
  \ifdim\@tempb\p@<\@tempa\p@\let\@tempa\@tempb\fi% select the smaller of both
  \ifdim\@tempa\p@<\p@\scalebox{\@tempa}{\usebox\pandoc@box}%
  \else\usebox{\pandoc@box}%
  \fi%
}
% Set default figure placement to htbp
\def\fps@figure{htbp}
\makeatother







\usepackage{newtx}

\defaultfontfeatures{Scale=MatchLowercase}
\defaultfontfeatures[\rmfamily]{Ligatures=TeX,Scale=1}





\title{Editar: Editar}


\shorttitle{Editar}


\usepackage{etoolbox}



\ccoppy{\textcopyright~2025}



\author{Edison Achalma}



\affiliation{
{Escuela Profesional de Economía, Universidad Nacional de San Cristóbal
de Huamanga}}




\leftheader{Achalma}

\date{2024-03-31}


\abstract{Este abstract será actualizado una vez que se complete el
contenido final del artículo. }

\keywords{keyword1, keyword2}

\authornote{\par{\addORCIDlink{Edison Achalma}{0000-0001-6996-3364}} 
\par{ }
\par{   Los autores no tienen conflictos de intereses que
revelar.    Los roles de autor se clasificaron utilizando la taxonomía
de roles de colaborador (CRediT; https://credit.niso.org/) de la
siguiente manera:  Edison Achalma:   conceptualización, redacción}
\par{La correspondencia relativa a este artículo debe dirigirse a Edison
Achalma, Email: \href{mailto:elmer.achalma.09@unsch.edu.pe}{elmer.achalma.09@unsch.edu.pe}}
}

\usepackage{pbalance} 
\usepackage{float}
\makeatletter
\let\oldtpt\ThreePartTable
\let\endoldtpt\endThreePartTable
\def\ThreePartTable{\@ifnextchar[\ThreePartTable@i \ThreePartTable@ii}
\def\ThreePartTable@i[#1]{\begin{figure}[!htbp]
\onecolumn
\begin{minipage}{0.5\textwidth}
\oldtpt[#1]
}
\def\ThreePartTable@ii{\begin{figure}[!htbp]
\onecolumn
\begin{minipage}{0.5\textwidth}
\oldtpt
}
\def\endThreePartTable{
\endoldtpt
\end{minipage}
\twocolumn
\end{figure}}
\makeatother


\makeatletter
\let\endoldlt\endlongtable		
\def\endlongtable{
\hline
\endoldlt}
\makeatother

\newenvironment{twocolumntable}% environment name
{% begin code
\begin{table*}[!htbp]%
\onecolumn%
}%
{%
\twocolumn%
\end{table*}%
}% end code

\urlstyle{same}



\makeatletter
\@ifpackageloaded{caption}{}{\usepackage{caption}}
\AtBeginDocument{%
\ifdefined\contentsname
  \renewcommand*\contentsname{Tabla de contenidos}
\else
  \newcommand\contentsname{Tabla de contenidos}
\fi
\ifdefined\listfigurename
  \renewcommand*\listfigurename{Listado de Figuras}
\else
  \newcommand\listfigurename{Listado de Figuras}
\fi
\ifdefined\listtablename
  \renewcommand*\listtablename{Listado de Tablas}
\else
  \newcommand\listtablename{Listado de Tablas}
\fi
\ifdefined\figurename
  \renewcommand*\figurename{Figura}
\else
  \newcommand\figurename{Figura}
\fi
\ifdefined\tablename
  \renewcommand*\tablename{Tabla}
\else
  \newcommand\tablename{Tabla}
\fi
}
\@ifpackageloaded{float}{}{\usepackage{float}}
\floatstyle{ruled}
\@ifundefined{c@chapter}{\newfloat{codelisting}{h}{lop}}{\newfloat{codelisting}{h}{lop}[chapter]}
\floatname{codelisting}{Listado}
\newcommand*\listoflistings{\listof{codelisting}{Listado de Listados}}
\makeatother
\makeatletter
\makeatother
\makeatletter
\@ifpackageloaded{caption}{}{\usepackage{caption}}
\@ifpackageloaded{subcaption}{}{\usepackage{subcaption}}
\makeatother

% From https://tex.stackexchange.com/a/645996/211326
%%% apa7 doesn't want to add appendix section titles in the toc
%%% let's make it do it
\makeatletter
\xpatchcmd{\appendix}
  {\par}
  {\addcontentsline{toc}{section}{\@currentlabelname}\par}
  {}{}
\makeatother

%% Disable longtable counter
%% https://tex.stackexchange.com/a/248395/211326

\usepackage{etoolbox}

\makeatletter
\patchcmd{\LT@caption}
  {\bgroup}
  {\bgroup\global\LTpatch@captiontrue}
  {}{}
\patchcmd{\longtable}
  {\par}
  {\par\global\LTpatch@captionfalse}
  {}{}
\apptocmd{\endlongtable}
  {\ifLTpatch@caption\else\addtocounter{table}{-1}\fi}
  {}{}
\newif\ifLTpatch@caption
\makeatother

\begin{document}

\maketitle

\hypertarget{toc}{}
\tableofcontents
\newpage
\section[Introduction]{Editar}

\setcounter{secnumdepth}{-\maxdimen} % remove section numbering

\setlength\LTleft{0pt}


\section{Anchor}\label{anchor}

En LibreOffice, el ``anchor'' (ancla en español) de una figura se
refiere al punto de referencia que determina cómo y dónde se posicionará
la figura dentro del documento. Aquí te explico los tipos de anclaje
disponibles y cómo afectan la disposición de tu figura:

\begin{enumerate}
\def\labelenumi{\arabic{enumi}.}
\item
  \textbf{Ancla a Párrafo (To Paragraph)}: La figura está vinculada a un
  párrafo específico. Si mueves el párrafo, la figura se mueve con él.
  Sin embargo, la posición exacta de la figura dentro del párrafo puede
  ajustarse manualmente.
\item
  \textbf{Ancla a Carácter (To Character)}: La figura se ancla a un
  carácter específico dentro del texto. Esto significa que si el texto
  se mueve o cambia, la figura se ajustará según la posición de ese
  carácter.
\item
  \textbf{Ancla a Página (To Page)}: La figura se fija a una posición
  absoluta en la página. No importa cómo se modifique el texto
  alrededor, la figura permanecerá en el mismo lugar de la página. Esto
  es útil para encabezados, pies de página, o imágenes que deben estar
  en una posición fija.
\item
  \textbf{Ancla a Trama (To Frame)}: Si estás usando marcos (frames),
  puedes anclar la figura a un marco específico. La figura entonces se
  moverá y ajustará con el marco.
\item
  \textbf{Ancla a Carácter (como una Imagen en Línea)}: Aunque es
  similar a ``To Character'', aquí la imagen actúa más como un carácter
  del texto, fluyendo con el texto como si fuera una letra o símbolo.
\end{enumerate}

\subsection{``To Character'' y ``As
Character''}\label{to-character-y-as-character}

En LibreOffice, ``To Character'' y ``As Character'' son dos formas
distintas de anclar imágenes o figuras, y cada una tiene su propia
manera de interactuar con el texto:

\begin{itemize}
\item
  \textbf{To Character}:

  \begin{itemize}
  \item
    \textbf{Posición}: La imagen se ancla a un carácter específico en el
    texto, pero no actúa como un carácter. Esto significa que la imagen
    se posicionará cerca del carácter al que está anclada, pero no
    necesariamente justo en línea con el texto.
  \item
    \textbf{Comportamiento}: Si mueves el carácter al que está anclada
    la imagen, la imagen se moverá con él. Sin embargo, la imagen puede
    estar posicionada fuera de la línea de texto, como flotando cerca de
    donde está el carácter.
  \item
    \textbf{Uso}: Es útil cuando quieres que una imagen se mueva con una
    parte específica del texto pero no necesariamente dentro del flujo
    del texto.
  \end{itemize}
\item
  \textbf{As Character}:

  \begin{itemize}
  \item
    \textbf{Posición}: La imagen o figura se inserta en el texto como si
    fuera un carácter adicional. Esto significa que se comporta
    exactamente como cualquier otra letra o símbolo en el documento.
  \item
    \textbf{Comportamiento}: La imagen se alinea dentro de la línea de
    texto, respetando el espaciado y el formato de párrafo. Si el texto
    alrededor cambia, la imagen se moverá y fluirá con el texto, justo
    como lo haría una letra.
  \item
    \textbf{Uso}: Es ideal para imágenes pequeñas que deben tratarse
    como parte del texto, como íconos, emoji o imágenes en línea que no
    deben interrumpir el flujo del texto.
  \end{itemize}
\end{itemize}

\textbf{Diferencias Clave}:

\begin{itemize}
\item
  \textbf{Interacción con el Texto}: ``As Character'' inserta la imagen
  directamente en el flujo del texto, mientras que ``To Character''
  ancla la imagen al texto pero permite un posicionamiento más libre
  fuera del flujo directo del texto.
\item
  \textbf{Flexibilidad de Posición}: ``To Character'' ofrece más control
  sobre la posición exacta de la imagen en relación con el carácter
  ancla, mientras que ``As Character'' sigue las reglas de formato del
  texto.
\item
  \textbf{Ajuste al Texto}: Con ``As Character'', cualquier cambio en el
  texto alrededor de la imagen (como rellenar, espaciado, etc.) afectará
  a la imagen como si fuera un carácter más. Con ``To Character'', la
  imagen puede moverse con el texto pero no se ajusta exactamente como
  un carácter dentro del mismo.
\end{itemize}

\section{Wrap}\label{wrap}

En LibreOffice, ``wrap'' (envoltura en español) se refiere a cómo el
texto fluye alrededor de una imagen o figura. Aquí te explico las
opciones de envoltura disponibles y cómo afectan la disposición del
texto y la imagen en tu documento:

Tipos de Envoltura (Wrap):

\begin{enumerate}
\def\labelenumi{\arabic{enumi}.}
\item
  \textbf{None (Sin Envoltura)}:

  \begin{itemize}
  \tightlist
  \item
    El texto no fluye alrededor de la imagen. La imagen actúa como si
    fuera un bloque de texto, empujando el texto hacia abajo o hacia
    arriba dependiendo de su posición.
  \end{itemize}
\item
  \textbf{Before (Antes)}:

  \begin{itemize}
  \tightlist
  \item
    El texto solo fluye a la izquierda de la imagen, dejando la parte
    derecha libre.
  \end{itemize}
\item
  \textbf{After (Después)}:

  \begin{itemize}
  \tightlist
  \item
    El texto solo fluye a la derecha de la imagen, dejando la parte
    izquierda sin texto.
  \end{itemize}
\item
  \textbf{Parallel (Paralelo)}:

  \begin{itemize}
  \tightlist
  \item
    El texto fluye tanto a la izquierda como a la derecha de la imagen,
    creando columnas de texto a ambos lados.
  \end{itemize}
\item
  \textbf{Through (A través)}:

  \begin{itemize}
  \tightlist
  \item
    El texto fluye a través de la imagen como si no estuviera ahí. Esta
    opción es menos común y se usa cuando la imagen tiene áreas
    transparentes o cuando quieres que el texto pase por encima de la
    imagen.
  \end{itemize}
\item
  \textbf{Optimal (Óptimo)}:

  \begin{itemize}
  \tightlist
  \item
    LibreOffice decide automáticamente la mejor forma de envolver el
    texto alrededor de la imagen basándose en la forma de la imagen y el
    espacio disponible.
  \end{itemize}
\end{enumerate}

Ajustes Adicionales:

\begin{itemize}
\item
  \textbf{Contorno (Contour)}:

  \begin{itemize}
  \tightlist
  \item
    Permite que el texto siga el contorno de la imagen, en lugar de
    simplemente envolverla en una caja rectangular. Esto se puede
    ajustar manualmente para obtener un envoltorio más preciso.
  \end{itemize}
\item
  \textbf{Espaciado}:

  \begin{itemize}
  \tightlist
  \item
    Puedes ajustar el espacio entre el texto y la imagen en todas
    direcciones (arriba, abajo, izquierda, derecha) para controlar mejor
    cómo se ve la envoltura.
  \end{itemize}
\item
  Para ajustes más detallados, como el contorno o el espaciado, puedes
  acceder a las propiedades de la imagen a través del panel lateral de
  LibreOffice o mediante el diálogo de propiedades que aparece al hacer
  clic derecho en la imagen y seleccionar ``Propiedades''.
\end{itemize}

\section{Align Objects}\label{align-objects}

En LibreOffice, ``Align Objects'' (Alinear Objetos) es una función que
te permite organizar y posicionar varias figuras, imágenes, formas, o
cualquier objeto gráfico en relación entre sí o con respecto a la
página. Aquí te explico cómo funciona y las opciones disponibles:

Aliniación de Objetos:

\textbf{Opciones de Alineación}:

\begin{itemize}
\item
  \textbf{Horizontal}:

  \begin{itemize}
  \item
    \textbf{Izquierda (Left)}: Alinea los objetos por su borde
    izquierdo.
  \item
    \textbf{Centro (Center)}: Alinea los objetos por su centro
    horizontal.
  \item
    \textbf{Derecha (Right)}: Alinea los objetos por su borde derecho.
  \end{itemize}
\item
  \textbf{Vertical}:

  \begin{itemize}
  \item
    \textbf{Arriba (Top)}: Alinea los objetos por su borde superior.
  \item
    \textbf{Centro (Center)}: Alinea los objetos por su centro vertical.
  \item
    \textbf{Abajo (Bottom)}: Alinea los objetos por su borde inferior.
  \end{itemize}
\end{itemize}

\textbf{Distribución}:

Además de la alineación, también puedes distribuir objetos para que el
espacio entre ellos sea uniforme:

\begin{itemize}
\item
  \textbf{Horizontal}: Distribuye objetos uniformemente a lo largo del
  eje horizontal.
\item
  \textbf{Vertical}: Distribuye objetos uniformemente a lo largo del eje
  vertical.
\end{itemize}

Consideraciones:

\begin{itemize}
\item
  \textbf{Anclaje}: La alineación puede verse afectada por cómo están
  anclados los objetos (a párrafo, carácter, página, etc.). Asegúrate de
  que el anclaje no interfiera con la alineación deseada.
\item
  \textbf{Grupos}: Si los objetos están agrupados, la alineación se
  aplicará al grupo como si fuera un solo objeto.
\item
  \textbf{Ajuste Manual}: A veces, después de usar la alineación
  automática, podrías querer hacer ajustes manuales para el
  posicionamiento exacto de los objetos.
\end{itemize}

\section{Arrange}\label{arrange}

En LibreOffice, ``Arrange'' (Organizar en español) se refiere a las
funciones que te permiten controlar el orden de apilamiento de objetos
gráficos como imágenes, formas, líneas, etc., en una página. Aquí está
cómo funciona y las opciones disponibles:

Opciones de Organización:

\begin{enumerate}
\def\labelenumi{\arabic{enumi}.}
\item
  \textbf{Bring to Front (Traer al Frente)}:

  \begin{itemize}
  \tightlist
  \item
    Coloca el objeto seleccionado por encima de todos los demás objetos.
    Es útil cuando quieres que un objeto sea visible por completo sin
    estar cubierto por otros.
  \end{itemize}
\item
  \textbf{Bring Forward (Traer hacia Adelante)}:

  \begin{itemize}
  \tightlist
  \item
    Mueve el objeto seleccionado una posición hacia adelante en el orden
    de apilamiento. Si hay varios objetos, este comando moverá el objeto
    uno nivel por encima de su posición actual.
  \end{itemize}
\item
  \textbf{Send to Back (Enviar al Fondo)}:

  \begin{itemize}
  \tightlist
  \item
    Coloca el objeto seleccionado detrás de todos los demás objetos.
    Esto es útil cuando necesitas que un objeto esté por debajo de otros
    sin interferir visualmente.
  \end{itemize}
\item
  \textbf{Send Backward (Enviar hacia Atrás)}:

  \begin{itemize}
  \tightlist
  \item
    Mueve el objeto seleccionado una posición hacia atrás en el orden de
    apilamiento. Así como ``Bring Forward'' mueve hacia adelante una
    posición, esta opción hace lo contrario.
  \end{itemize}
\end{enumerate}

Consideraciones:

\begin{itemize}
\item
  \textbf{Visibilidad}: El orden de apilamiento afecta cómo los objetos
  se superponen y, por tanto, qué partes de los objetos son visibles. Un
  objeto en la parte superior puede ocultar porciones de objetos que
  están ``más abajo'' en el orden de apilamiento.
\item
  \textbf{Grupos}: Si los objetos están agrupados, cualquier operación
  de organización se aplicará a todo el grupo como una sola unidad.
\item
  \textbf{Interacción con el Texto}: La organización de objetos también
  puede cambiar cómo estos interactúan con el texto, especialmente si
  los objetos están anclados a diferentes elementos del documento.
\item
  \textbf{Capas}: Aunque LibreOffice no tiene un sistema de capas como
  en programas de diseño gráfico avanzado, las funciones de organizar
  sirven para manejar una pseudo-capas a través del orden de
  apilamiento.
\end{itemize}

\section{Publicaciones Similares}\label{publicaciones-similares}

Si te interesó este artículo, te recomendamos que explores otros blogs y
recursos relacionados que pueden ampliar tus conocimientos. Aquí te dejo
algunas sugerencias:

\begin{enumerate}
\def\labelenumi{\arabic{enumi}.}
\tightlist
\item
  \href{https://achalmaedison.netlify.app/herramientas-oficina/ofimatica/2022-12-05-01-introduccion-al-lenguaje-y-editor-vba}{01
  Introduccion Al Lenguaje Y Editor Vba} Lee sin conexión
  \href{https://achalmaedison.netlify.app/herramientas-oficina/ofimatica/2022-12-05-01-introduccion-al-lenguaje-y-editor-vba/index.pdf}{PDF}
\item
  \href{https://achalmaedison.netlify.app/herramientas-oficina/ofimatica/2022-12-12-02-grabar-y-modificar}{02
  Grabar Y Modificar} Lee sin conexión
  \href{https://achalmaedison.netlify.app/herramientas-oficina/ofimatica/2022-12-12-02-grabar-y-modificar/index.pdf}{PDF}
\item
  \href{https://achalmaedison.netlify.app/herramientas-oficina/ofimatica/2022-12-19-03-procedimientos}{03
  Procedimientos} Lee sin conexión
  \href{https://achalmaedison.netlify.app/herramientas-oficina/ofimatica/2022-12-19-03-procedimientos/index.pdf}{PDF}
\item
  \href{https://achalmaedison.netlify.app/herramientas-oficina/ofimatica/2022-12-26-04-funciones-en-vba}{04
  Funciones En Vba} Lee sin conexión
  \href{https://achalmaedison.netlify.app/herramientas-oficina/ofimatica/2022-12-26-04-funciones-en-vba/index.pdf}{PDF}
\item
  \href{https://achalmaedison.netlify.app/herramientas-oficina/ofimatica/2023-01-02-05-funciones-condicionales-estructuras-condicionales}{05
  Funciones Condicionales Estructuras Condicionales} Lee sin conexión
  \href{https://achalmaedison.netlify.app/herramientas-oficina/ofimatica/2023-01-02-05-funciones-condicionales-estructuras-condicionales/index.pdf}{PDF}
\item
  \href{https://achalmaedison.netlify.app/herramientas-oficina/ofimatica/2023-01-09-06-funciones-iterativas-estructuras-repetitivas-o-bucles}{06
  Funciones Iterativas Estructuras Repetitivas O Bucles} Lee sin
  conexión
  \href{https://achalmaedison.netlify.app/herramientas-oficina/ofimatica/2023-01-09-06-funciones-iterativas-estructuras-repetitivas-o-bucles/index.pdf}{PDF}
\item
  \href{https://achalmaedison.netlify.app/herramientas-oficina/ofimatica/2023-01-16-07-formularios}{07
  Formularios} Lee sin conexión
  \href{https://achalmaedison.netlify.app/herramientas-oficina/ofimatica/2023-01-16-07-formularios/index.pdf}{PDF}
\item
  \href{https://achalmaedison.netlify.app/herramientas-oficina/ofimatica/2023-01-23-08-eventos}{08
  Eventos} Lee sin conexión
  \href{https://achalmaedison.netlify.app/herramientas-oficina/ofimatica/2023-01-23-08-eventos/index.pdf}{PDF}
\item
  \href{https://achalmaedison.netlify.app/herramientas-oficina/ofimatica/2023-05-31-combinando-hojas-de-excel-con-vba}{Combinando
  Hojas De Excel Con Vba} Lee sin conexión
  \href{https://achalmaedison.netlify.app/herramientas-oficina/ofimatica/2023-05-31-combinando-hojas-de-excel-con-vba/index.pdf}{PDF}
\item
  \href{https://achalmaedison.netlify.app/herramientas-oficina/ofimatica/2024-03-31-por-editar}{Por
  Editar} Lee sin conexión
  \href{https://achalmaedison.netlify.app/herramientas-oficina/ofimatica/2024-03-31-por-editar/index.pdf}{PDF}
\end{enumerate}

Esperamos que encuentres estas publicaciones igualmente interesantes y
útiles. ¡Disfruta de la lectura!






\end{document}
