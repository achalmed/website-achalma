\documentclass[
  man,
  floatsintext,
  longtable,
  a4paper,
  nolmodern,
  notxfonts,
  notimes,
  colorlinks=true,linkcolor=blue,citecolor=blue,urlcolor=blue]{apa7}

\usepackage{amsmath}
\usepackage{amssymb}



\usepackage[bidi=default]{babel}
\babelprovide[main,import]{spanish}
\StartBabelCommands{spanish}{captions} [unicode, fontenc=TU EU1 EU2, charset=utf8] \SetString{\keywordname}{Palabras
Claves}
\EndBabelCommands


% get rid of language-specific shorthands (see #6817):
\let\LanguageShortHands\languageshorthands
\def\languageshorthands#1{}

\RequirePackage{longtable}
\RequirePackage{threeparttablex}

\makeatletter
\renewcommand{\paragraph}{\@startsection{paragraph}{4}{\parindent}%
	{0\baselineskip \@plus 0.2ex \@minus 0.2ex}%
	{-.5em}%
	{\normalfont\normalsize\bfseries\typesectitle}}

\renewcommand{\subparagraph}[1]{\@startsection{subparagraph}{5}{0.5em}%
	{0\baselineskip \@plus 0.2ex \@minus 0.2ex}%
	{-\z@\relax}%
	{\normalfont\normalsize\bfseries\itshape\hspace{\parindent}{#1}\textit{\addperi}}{\relax}}
\makeatother




\usepackage{longtable, booktabs, multirow, multicol, colortbl, hhline, caption, array, float, xpatch}
\usepackage{subcaption}
\renewcommand\thesubfigure{\Alph{subfigure}}
\setcounter{topnumber}{2}
\setcounter{bottomnumber}{2}
\setcounter{totalnumber}{4}
\renewcommand{\topfraction}{0.85}
\renewcommand{\bottomfraction}{0.85}
\renewcommand{\textfraction}{0.15}
\renewcommand{\floatpagefraction}{0.7}

\usepackage{tcolorbox}
\tcbuselibrary{listings,theorems, breakable, skins}
\usepackage{fontawesome5}

\definecolor{quarto-callout-color}{HTML}{909090}
\definecolor{quarto-callout-note-color}{HTML}{0758E5}
\definecolor{quarto-callout-important-color}{HTML}{CC1914}
\definecolor{quarto-callout-warning-color}{HTML}{EB9113}
\definecolor{quarto-callout-tip-color}{HTML}{00A047}
\definecolor{quarto-callout-caution-color}{HTML}{FC5300}
\definecolor{quarto-callout-color-frame}{HTML}{ACACAC}
\definecolor{quarto-callout-note-color-frame}{HTML}{4582EC}
\definecolor{quarto-callout-important-color-frame}{HTML}{D9534F}
\definecolor{quarto-callout-warning-color-frame}{HTML}{F0AD4E}
\definecolor{quarto-callout-tip-color-frame}{HTML}{02B875}
\definecolor{quarto-callout-caution-color-frame}{HTML}{FD7E14}

%\newlength\Oldarrayrulewidth
%\newlength\Oldtabcolsep


\usepackage{hyperref}




\providecommand{\tightlist}{%
  \setlength{\itemsep}{0pt}\setlength{\parskip}{0pt}}
\usepackage{longtable,booktabs,array}
\usepackage{calc} % for calculating minipage widths
% Correct order of tables after \paragraph or \subparagraph
\usepackage{etoolbox}
\makeatletter
\patchcmd\longtable{\par}{\if@noskipsec\mbox{}\fi\par}{}{}
\makeatother
% Allow footnotes in longtable head/foot
\IfFileExists{footnotehyper.sty}{\usepackage{footnotehyper}}{\usepackage{footnote}}
\makesavenoteenv{longtable}

\usepackage{graphicx}
\makeatletter
\newsavebox\pandoc@box
\newcommand*\pandocbounded[1]{% scales image to fit in text height/width
  \sbox\pandoc@box{#1}%
  \Gscale@div\@tempa{\textheight}{\dimexpr\ht\pandoc@box+\dp\pandoc@box\relax}%
  \Gscale@div\@tempb{\linewidth}{\wd\pandoc@box}%
  \ifdim\@tempb\p@<\@tempa\p@\let\@tempa\@tempb\fi% select the smaller of both
  \ifdim\@tempa\p@<\p@\scalebox{\@tempa}{\usebox\pandoc@box}%
  \else\usebox{\pandoc@box}%
  \fi%
}
% Set default figure placement to htbp
\def\fps@figure{htbp}
\makeatother







\usepackage{newtx}

\defaultfontfeatures{Scale=MatchLowercase}
\defaultfontfeatures[\rmfamily]{Ligatures=TeX,Scale=1}





\title{Ideas de investigación en Econometría Análisis de variables y
relaciones económicas: Explora diferentes enfoques para investigar la
economía y sus interacciones a través de la Econometría}


\shorttitle{Editar}


\usepackage{etoolbox}



\ccoppy{\textcopyright~2025}



\author{Edison Achalma}



\affiliation{
{Escuela Profesional de Economía, Universidad Nacional de San Cristóbal
de Huamanga}}




\leftheader{Achalma}

\date{2023-06-03}


\abstract{Este abstract será actualizado una vez que se complete el
contenido final del artículo. }

\keywords{keyword1, keyword2}

\authornote{\par{\addORCIDlink{Edison Achalma}{0000-0001-6996-3364}} 
\par{ }
\par{   El autor no tiene conflictos de interés que revelar.    Los
roles de autor se clasificaron utilizando la taxonomía de roles de
colaborador (CRediT; https://credit.niso.org/) de la siguiente
manera:  Edison Achalma:   conceptualización, redacción}
\par{La correspondencia relativa a este artículo debe dirigirse a Edison
Achalma, Email: \href{mailto:elmer.achalma.09@unsch.edu.pe}{elmer.achalma.09@unsch.edu.pe}}
}

\makeatletter
\let\endoldlt\endlongtable
\def\endlongtable{
\hline
\endoldlt
}
\makeatother

\urlstyle{same}



\makeatletter
\@ifpackageloaded{caption}{}{\usepackage{caption}}
\AtBeginDocument{%
\ifdefined\contentsname
  \renewcommand*\contentsname{Tabla de contenidos}
\else
  \newcommand\contentsname{Tabla de contenidos}
\fi
\ifdefined\listfigurename
  \renewcommand*\listfigurename{Listado de Figuras}
\else
  \newcommand\listfigurename{Listado de Figuras}
\fi
\ifdefined\listtablename
  \renewcommand*\listtablename{Listado de Tablas}
\else
  \newcommand\listtablename{Listado de Tablas}
\fi
\ifdefined\figurename
  \renewcommand*\figurename{Figura}
\else
  \newcommand\figurename{Figura}
\fi
\ifdefined\tablename
  \renewcommand*\tablename{Tabla}
\else
  \newcommand\tablename{Tabla}
\fi
}
\@ifpackageloaded{float}{}{\usepackage{float}}
\floatstyle{ruled}
\@ifundefined{c@chapter}{\newfloat{codelisting}{h}{lop}}{\newfloat{codelisting}{h}{lop}[chapter]}
\floatname{codelisting}{Listado}
\newcommand*\listoflistings{\listof{codelisting}{Listado de Listados}}
\makeatother
\makeatletter
\makeatother
\makeatletter
\@ifpackageloaded{caption}{}{\usepackage{caption}}
\@ifpackageloaded{subcaption}{}{\usepackage{subcaption}}
\makeatother
\makeatletter
\@ifpackageloaded{fontawesome5}{}{\usepackage{fontawesome5}}
\makeatother

% From https://tex.stackexchange.com/a/645996/211326
%%% apa7 doesn't want to add appendix section titles in the toc
%%% let's make it do it
\makeatletter
\xpatchcmd{\appendix}
  {\par}
  {\addcontentsline{toc}{section}{\@currentlabelname}\par}
  {}{}
\makeatother

%% Disable longtable counter
%% https://tex.stackexchange.com/a/248395/211326

\usepackage{etoolbox}

\makeatletter
\patchcmd{\LT@caption}
  {\bgroup}
  {\bgroup\global\LTpatch@captiontrue}
  {}{}
\patchcmd{\longtable}
  {\par}
  {\par\global\LTpatch@captionfalse}
  {}{}
\apptocmd{\endlongtable}
  {\ifLTpatch@caption\else\addtocounter{table}{-1}\fi}
  {}{}
\newif\ifLTpatch@caption
\makeatother

\begin{document}

\maketitle

\hypertarget{toc}{}
\tableofcontents
\newpage
\section[Introduction]{Ideas de investigación en Econometría Análisis de
variables y relaciones económicas}

\setcounter{secnumdepth}{-\maxdimen} % remove section numbering

\setlength\LTleft{0pt}


\section{Propuesta de investigación: Determinantes microeconómicos de la
pobreza en el departamento de Ayacucho durante el periodo
2000-2020.}\label{propuesta-de-investigaciuxf3n-determinantes-microeconuxf3micos-de-la-pobreza-en-el-departamento-de-ayacucho-durante-el-periodo-2000-2020.}

\textbf{Justificación:} La pobreza es un fenómeno social de gran impacto
que afecta a nivel personal, familiar y nacional. Está presente en
nuestra vida diaria, en las noticias de los medios de comunicación y en
el discurso político. Además, es una realidad que afecta de manera
significativa a aquellos que se encuentran en situación de pobreza.

Por esta razón, resulta fundamental llevar a cabo una investigación
enfocada en los determinantes microeconómicos de la pobreza en el
departamento de Ayacucho. Esto permitirá comprender las causas y
factores que contribuyen a esta problemática en un ámbito específico y a
lo largo de un periodo de tiempo extenso (2000-2020).

\begin{longtable}[]{@{}
  >{\raggedright\arraybackslash}p{(\linewidth - 2\tabcolsep) * \real{0.1528}}
  >{\raggedright\arraybackslash}p{(\linewidth - 2\tabcolsep) * \real{0.8472}}@{}}
\toprule\noalign{}
\begin{minipage}[b]{\linewidth}\raggedright
Elemento
\end{minipage} & \begin{minipage}[b]{\linewidth}\raggedright
Descripción
\end{minipage} \\
\midrule\noalign{}
\endhead
\bottomrule\noalign{}
\endlastfoot
\textbf{Problema} & \textbf{Problema general:} ¿Cuáles son los
determinantes microeconómicos de la probabilidad de ser pobre en el Perú
según la encuesta de ENAHO 2020? \textbf{Problemas secundarios:} 1.
¿Cuál es la incidencia de las variables demográficas sobre la
probabilidad de ser pobre en el departamento de Ayacucho? 2. ¿Cuál es la
incidencia de las variables de capital humano sobre la probabilidad de
que el hogar sea pobre o no en el departamento de Ayacucho? 3. ¿Cuál es
la incidencia de la actividad económica sobre la probabilidad de que el
hogar sea pobre o no en el departamento de Ayacucho? \\
\textbf{Objetivos} & \textbf{Objetivo general:} Plantear un modelo de
probabilidad que permite identificar los determinantes de la pobreza de
los hogares en el departamento de Ayacucho según la encuesta de ENAHO
2020 \textbf{Objetivos específicos:} 1. Determinar cuál es la incidencia
de las variables demográficas sobre la probabilidad de que el hogar sea
pobre o no en el departamento de Ayacucho. 2. Precisar cuál es la
incidencia de las variables del capital humano sobre la probabilidad de
que el hogar sea pobre en el departamento de Ayacucho. 3. Mostrar cuál
es la incidencia de las variables del hogar sobre la probabilidad de que
el hogar sea pobre o no en el departamento. \\
\textbf{Hipótesis} & La edad del jefe del hogar tiene incidencia en la
probabilidad de que el hogar sea pobre o no. A menor edad, mayor
probabilidad de ser pobre. La escolaridad del jefe de hogar tiene
incidencia en la probabilidad de que el hogar sea pobre o no. A mayor
escolaridad, menor probabilidad de ser pobre. La rama de actividad al
que se dedica el jefe del hogar tiene una fuerte incidencia en la
probabilidad de que el hogar sea pobre o no. El número de miembros del
hogar que trabajan tiene una fuerte incidencia en la prueba de que el
hogar sea pobre o no. A mayor número de miembros del hogar que trabajan,
menor probabilidad de ser pobre. \\
\textbf{Variables} & \textbf{Endógenas:} Probabilidad de ser pobre
\textbf{Indicador:} Variable dummy que denota presencia/ausencia de
pobreza \textbf{Exógenas:} Demográficas, capital humano, actividad
económica, variables de localización \textbf{Indicadores:} Edad, sexo,
estado civil, número de miembros del hogar, posición de negocio,
informalidad, empleo, localización (costa, sierra y selva) \\
\textbf{Metodología} & \textbf{Tipo de investigación:} Aplicada
\textbf{Nivel de investigación:} Descriptivo, analítico \textbf{Método:}
Inductivo \textbf{Diseño:} Investigación por objetivo \\
\end{longtable}

\section{Propuesta de investigación: Gasto social y nivel de pobreza en
el departamento de Ayacucho: periodo
2000-2019.}\label{propuesta-de-investigaciuxf3n-gasto-social-y-nivel-de-pobreza-en-el-departamento-de-ayacucho-periodo-2000-2019.}

Variable causa: Gasto social. Variable efecto: Nivel de pobreza.

\textbf{Justificación:}

En el contexto de países de Medio Oriente y África, se observan
conflictos sociales, guerras civiles y alta inflación, lo que resulta en
una extrema pobreza que afecta a una gran parte de su población.

A pesar de que Perú cuenta con mejores condiciones geográficas,
estabilidad política y económica, presenta un porcentaje similar o
incluso mayor de extrema pobreza y desigualdad en su población,
especialmente en las zonas rurales. Estas áreas aún enfrentan problemas
estructurales en cuanto a servicios básicos como agua potable, energía
eléctrica, educación, salud y alimentación. Esto se debe, en parte, a
una deficiente distribución del gasto social en departamentos como
Ayacucho.

Por lo tanto, es necesario identificar las variables que explican los
niveles de pobreza y comprender por qué existe una brecha significativa
entre las zonas rurales y urbanas. Para lograrlo, se llevará a cabo un
análisis estadístico descriptivo y se aplicará un modelo econométrico
utilizando la base de datos de la Encuesta Nacional de Hogares (ENAHO)
del INEI.

\begin{longtable}[]{@{}
  >{\raggedright\arraybackslash}p{(\linewidth - 10\tabcolsep) * \real{0.1250}}
  >{\raggedright\arraybackslash}p{(\linewidth - 10\tabcolsep) * \real{0.3056}}
  >{\raggedright\arraybackslash}p{(\linewidth - 10\tabcolsep) * \real{0.1250}}
  >{\raggedright\arraybackslash}p{(\linewidth - 10\tabcolsep) * \real{0.1250}}
  >{\raggedright\arraybackslash}p{(\linewidth - 10\tabcolsep) * \real{0.1250}}
  >{\raggedright\arraybackslash}p{(\linewidth - 10\tabcolsep) * \real{0.1944}}@{}}
\toprule\noalign{}
\begin{minipage}[b]{\linewidth}\raggedright
Variable
\end{minipage} & \begin{minipage}[b]{\linewidth}\raggedright
Definición conceptual
\end{minipage} & \begin{minipage}[b]{\linewidth}\raggedright
Definición operacional
\end{minipage} & \begin{minipage}[b]{\linewidth}\raggedright
Indicadores
\end{minipage} & \begin{minipage}[b]{\linewidth}\raggedright
Unidad de medida
\end{minipage} & \begin{minipage}[b]{\linewidth}\raggedright
Valor final
\end{minipage} \\
\midrule\noalign{}
\endhead
\bottomrule\noalign{}
\endlastfoot
X = Gasto social & Es la búsqueda de logros en materia de equidad
social, a través del desarrollo del capital físico y humano, promoviendo
el aseguramiento de necesidades básicas. & & - Gasto en educación- Gasto
en salud- Programas sociales & Números & R-Squared Es la búsqueda de
logros en materia de equidad social, a través del desarrollo del capital
físico y humano, promoviendo el aseguramiento de necesidades básicas. \\
Y = Nivel de pobreza & Según el INEI (2007), se define la pobreza como
aquel conjunto de personas que no alcanzan a tener un nivel de
satisfacción mínimo respecto a un conjunto de necesidades básicas
relacionadas con la salud, nutrición, educación, vivienda, etc. & & -
Índice de Desarrollo Humano- Índice de Gini & Números & - \\
\end{longtable}

\section{Propuesta de investigación: Factores asociados a la
desnutrición en niños menores de 5 años en el
Perú}\label{propuesta-de-investigaciuxf3n-factores-asociados-a-la-desnutriciuxf3n-en-niuxf1os-menores-de-5-auxf1os-en-el-peruxfa}

\textbf{Justificación:}

La desnutrición infantil representa uno de los mayores desafíos en
términos de Salud Pública en el Perú.

\begin{longtable}[]{@{}
  >{\raggedright\arraybackslash}p{(\linewidth - 2\tabcolsep) * \real{0.1944}}
  >{\raggedright\arraybackslash}p{(\linewidth - 2\tabcolsep) * \real{0.8056}}@{}}
\toprule\noalign{}
\begin{minipage}[b]{\linewidth}\raggedright
Título
\end{minipage} & \begin{minipage}[b]{\linewidth}\raggedright
Contenido
\end{minipage} \\
\midrule\noalign{}
\endhead
\bottomrule\noalign{}
\endlastfoot
Problema de investigación & ¿De qué manera la pobreza y el analfabetismo
influyen en la desnutrición de los niños menores de 5 años en los
departamentos de Perú de 2009 a 2019? \\
Problemas específicos & ¿De qué manera se relaciona la desnutrición en
menores de 5 años con la pobreza en los departamentos de Perú? ¿De qué
manera se relaciona la desnutrición en menores de 5 años con el
analfabetismo en los departamentos de Perú? \\
Objetivo general & Determinar la influencia de la pobreza y el
analfabetismo en la desnutrición de los niños menores de 5 años en los
departamentos de Perú de 2009 a 2019. \\
Objetivos específicos & Explicar la relación de la desnutrición en
menores de 5 años con la pobreza en los departamentos de Perú. Explicar
la relación de la desnutrición en menores de 5 años con el analfabetismo
en los departamentos de Perú. \\
Hipótesis general & La pobreza y el analfabetismo influyen
significativamente en la desnutrición de los niños menores de 5 años en
los departamentos de Perú de 2009 a 2019. \\
Hipótesis específicas & La desnutrición en menores de 5 años se
relaciona significativamente con la pobreza en los departamentos de
Perú. La desnutrición en menores de 5 años se relaciona
significativamente con el analfabetismo en los departamentos de Perú. \\
Variable endógena & Desnutrición en niños menores de 5 años \\
Indicador & Tasa de desnutrición en menores de 5 años \\
Variables exógenas & Pobreza Analfabetismo Inflación \\
Indicadores & Tasa de pobreza Tasa de analfabetismo IPC \\
Tipo de investigación & Aplicada \\
Diseño de investigación & Descriptivo - correlacional \\
\end{longtable}

\section{Propuesta de investigación: Influencia de la recaudación de
impuestos prediales en la recaudación tributaria de la Municipalidad
Distrital de Huamanga en el año
2021.}\label{propuesta-de-investigaciuxf3n-influencia-de-la-recaudaciuxf3n-de-impuestos-prediales-en-la-recaudaciuxf3n-tributaria-de-la-municipalidad-distrital-de-huamanga-en-el-auxf1o-2021.}

\textbf{Justificación:}

Esta investigación tiene como objetivo analizar el impacto de la
recaudación de impuestos prediales en la recaudación tributaria de la
Municipalidad Distrital de Huamanga. Se busca determinar si los
ciudadanos otorgan la debida importancia al cumplimiento de sus
obligaciones tributarias, así como promover una mayor conciencia sobre
la importancia de realizar los pagos correspondientes.

Además, se plantea la posibilidad de investigar la influencia del
impuesto a la renta en los tributos nacionales, con el fin de ampliar el
alcance de la investigación.

Para llevar a cabo este estudio, se sugiere utilizar métodos como la
realización de encuestas para obtener datos precisos sobre la
recaudación de impuestos prediales. En caso de que no se encuentren
datos disponibles para el distrito de Huamanga, se podría considerar
ampliar el análisis a nivel nacional.

Fuentes de referencia:

\begin{enumerate}
\def\labelenumi{\arabic{enumi}.}
\item
  Guía de Meta: Recaudación de impuestos y tasas 2019-2020. Disponible
  en:
  \href{https://www.mef.gob.pe/contenidos/presu_publ/migl/metas/GUIA_META_2_rcaudacion2019_2020.pdf}{enlace
  a la fuente}.
\item
  Banco Central de Reserva del Perú: Estadísticas económicas. Disponible
  en:
  \href{https://estadisticas.bcrp.gob.pe/estadisticas/series/mensuales/resultados/RD13778DM/html}{enlace
  a la fuente}.
\end{enumerate}

\begin{longtable}[]{@{}
  >{\raggedright\arraybackslash}p{(\linewidth - 2\tabcolsep) * \real{0.2222}}
  >{\raggedright\arraybackslash}p{(\linewidth - 2\tabcolsep) * \real{0.7778}}@{}}
\toprule\noalign{}
\begin{minipage}[b]{\linewidth}\raggedright
Título
\end{minipage} & \begin{minipage}[b]{\linewidth}\raggedright
Modo Oración
\end{minipage} \\
\midrule\noalign{}
\endhead
\bottomrule\noalign{}
\endlastfoot
Problema de investigación & ¿De qué manera el impuesto predial se
relaciona con la recaudación tributaria en la Municipalidad Distrital de
Huamanga-2021? \\
Problema específico & ¿De qué manera el impuesto predial se relaciona
con los impuestos municipales en la Municipalidad Distrital de
Huamanga-2021? \\
& ¿De qué manera la recaudación tributaria se relaciona con el auto
valúo en la Municipalidad Distrital de Huamanga-2021? \\
& ¿De qué manera el impuesto predial se relaciona con las tasas
municipales en la Municipalidad Distrital de Huamanga-2021? \\
Objetivo general & Analizar si el impuesto predial se relaciona con la
recaudación tributaria en la municipalidad distrital de Huamanga-2021 \\
Objetivos específicos & - Demostrar si el impuesto predial se relaciona
con los impuestos municipales en la Municipalidad Distrital de
Huamanga-2021 \\
& - Demostrar si la recaudación tributaria se relaciona con el auto
valúo en la Municipalidad Distrital de Huamanga-2021 \\
& - Demostrar si el impuesto predial se relaciona con las tasas
municipales en la Municipalidad Distrital de Huamanga-2021 \\
Hipótesis general & El impuesto predial se relaciona con la recaudación
tributaria en la municipalidad distrital de Huamanga-2021 \\
Hipótesis específicas & - El impuesto predial se relaciona con los
impuestos municipales en la Municipalidad Distrital de Huamanga-2021 \\
& - La recaudación tributaria se relaciona con el auto valuó en la
Municipalidad Distrital de Huamanga-2021 \\
& - El impuesto predial se relaciona con las tasas municipales en la
Municipalidad Distrital de Huamanga-2021 \\
Variables e Indicadores & \textbf{Variable Independiente:} Impuesto
predial \\
& - Indicadores: Auto valúo, Predios \\
& \textbf{Variable Dependiente:} Recaudación tributaria \\
& - Indicadores: Impuestos municipales, Tasas municipales (licencias de
funcionamiento) \\
Metodología & - Tipo de investigación: Descriptivo y correlacional \\
\end{longtable}

\section{Propuesta de investigación: Propuesta de programa de inserción
escolar con formación e incorporación al mundo laboral de manera
emprendedora en la región
Ayacucho.}\label{propuesta-de-investigaciuxf3n-propuesta-de-programa-de-inserciuxf3n-escolar-con-formaciuxf3n-e-incorporaciuxf3n-al-mundo-laboral-de-manera-emprendedora-en-la-regiuxf3n-ayacucho.}

\begin{longtable}[]{@{}
  >{\raggedright\arraybackslash}p{(\linewidth - 4\tabcolsep) * \real{0.2466}}
  >{\raggedright\arraybackslash}p{(\linewidth - 4\tabcolsep) * \real{0.2603}}
  >{\raggedright\arraybackslash}p{(\linewidth - 4\tabcolsep) * \real{0.4932}}@{}}
\toprule\noalign{}
\begin{minipage}[b]{\linewidth}\raggedright
PROBLEMAS
\end{minipage} & \begin{minipage}[b]{\linewidth}\raggedright
OBJETIVOS
\end{minipage} & \begin{minipage}[b]{\linewidth}\raggedright
HIPÓTESIS
\end{minipage} \\
\midrule\noalign{}
\endhead
\bottomrule\noalign{}
\endlastfoot
¿Cómo se desempeñan las Startup en el Perú y cuáles son sus perspectivas
de desarrollo? & Describir el desempeño de las Startup en el Perú y
estimar sus perspectivas de desarrollo. & Esta investigación es de nivel
descriptivo, por lo que no precisa una hipótesis precisa. Sin embargo,
pueden surgir hipótesis ``emergentes'' durante el desarrollo de la
investigación. \\
ESPECÍFICO 1 & & \\
¿Cómo se desempeñan las Startup en el Perú? & Describir el desempeño de
las Startup en el Perú. & \\
& & \\
ESPECÍFICO 2 & & \\
¿Cuáles son las perspectivas de desarrollo de las Startup en el Perú? &
Estimar las perspectivas de desarrollo de las Startup en el Perú. & \\
& & \\
\end{longtable}

\begin{longtable}[]{@{}
  >{\raggedright\arraybackslash}p{(\linewidth - 4\tabcolsep) * \real{0.1781}}
  >{\raggedright\arraybackslash}p{(\linewidth - 4\tabcolsep) * \real{0.1781}}
  >{\raggedright\arraybackslash}p{(\linewidth - 4\tabcolsep) * \real{0.6438}}@{}}
\toprule\noalign{}
\begin{minipage}[b]{\linewidth}\raggedright
VARIABLES
\end{minipage} & \begin{minipage}[b]{\linewidth}\raggedright
(CATEGORÍA DE ANÁLISIS)
\end{minipage} & \begin{minipage}[b]{\linewidth}\raggedright
METODOLOGÍA
\end{minipage} \\
\midrule\noalign{}
\endhead
\bottomrule\noalign{}
\endlastfoot
Desempeño de la Startup & Desempeño & Propósito: Investigación
BásicaNivel: DescriptivoMétodo: Analítico-sintéticoEnfoque:
CuantitativoDiseño: Diseño de investigación no experimental \\
Perspectiva de desarrollo & Perspectiva & \\
& & \\
\end{longtable}

Se tomaría datos de la ENAHO viendo el nivel de empleo

\section{Propuesta de investigación: Choques externos e internos sobre
la dinámica de los ingresos tributarios en el Perú, periodo 2000 -
2019}\label{propuesta-de-investigaciuxf3n-choques-externos-e-internos-sobre-la-dinuxe1mica-de-los-ingresos-tributarios-en-el-peruxfa-periodo-2000---2019}

\begin{longtable}[]{@{}
  >{\raggedright\arraybackslash}p{(\linewidth - 4\tabcolsep) * \real{0.2500}}
  >{\raggedright\arraybackslash}p{(\linewidth - 4\tabcolsep) * \real{0.3194}}
  >{\raggedright\arraybackslash}p{(\linewidth - 4\tabcolsep) * \real{0.4306}}@{}}
\toprule\noalign{}
\begin{minipage}[b]{\linewidth}\raggedright
PROBLEMA DE INVESTIGACIÓN
\end{minipage} & \begin{minipage}[b]{\linewidth}\raggedright
OBJETIVOS
\end{minipage} & \begin{minipage}[b]{\linewidth}\raggedright
HIPÓTESIS
\end{minipage} \\
\midrule\noalign{}
\endhead
\bottomrule\noalign{}
\endlastfoot
¿De qué manera los shocks externos e internos afectan los ingresos
tributarios en Perú (2000-2019)? & Evaluar la influencia del Producto
Bruto Interno (PBI) y los precios de exportación en los ingresos
tributarios de Perú (1990-2019). & Los indicadores del Producto Bruto
Interno (PBI) e Índice de Precios de Exportación (IPX) han influido
positivamente en la dinámica de los ingresos tributarios de Perú
(2000-2019). \\
& Cuantificar la influencia de los precios de exportación en los
ingresos tributarios de Perú (2000-2019). & Existiría influencia del IPX
en la captación de ingresos tributarios de Perú en el periodo 2000 --
2019. \\
& Cuantificar la influencia del crecimiento del PBI en los ingresos
tributarios de Perú (2000-2019). & Existiría influencia del PBI en la
captación de los ingresos tributarios del Perú en el periodo 2000--
2019. \\
& Determinar el nivel de influencia y bidireccionalidad entre los
ingresos tributarios y el crecimiento económico de Perú (2000-2019).
& \\
& & \\
VARIABLES E INDICADORES & METODOLOGÍA & \\
Variable endógena: Ingresos tributarios del gobierno central & Tipo de
investigación: No experimental longitudinal & \\
Variables exógenas: Índice de precios de exportaciónProducto Bruto
Interno real & & \\
\end{longtable}

\section{Publicaciones Similares}\label{publicaciones-similares}

Si te interesó este artículo, te recomendamos que explores otros blogs y
recursos relacionados que pueden ampliar tus conocimientos. Aquí te dejo
algunas sugerencias:

\begin{enumerate}
\def\labelenumi{\arabic{enumi}.}
\tightlist
\item
  \href{https://achalmaedison.netlify.app/investigacion-metodologia/posts/2023-06-03-ideas-de-investigacion-para-economia/index.pdf}{\faIcon{file-pdf}}
  \href{https://achalmaedison.netlify.app/investigacion-metodologia/posts/2023-06-03-ideas-de-investigacion-para-economia}{Ideas
  De Investigacion Para Economia}
\item
  \href{https://achalmaedison.netlify.app/investigacion-metodologia/posts/2023-06-03-pautas-de-presentacion-del-informe-de-investigacion/index.pdf}{\faIcon{file-pdf}}
  \href{https://achalmaedison.netlify.app/investigacion-metodologia/posts/2023-06-03-pautas-de-presentacion-del-informe-de-investigacion}{Pautas
  De Presentacion Del Informe De Investigacion}
\item
  \href{https://achalmaedison.netlify.app/investigacion-metodologia/posts/2025-01-12-recursos-de-bibliografia-y-documentacion/index.pdf}{\faIcon{file-pdf}}
  \href{https://achalmaedison.netlify.app/investigacion-metodologia/posts/2025-01-12-recursos-de-bibliografia-y-documentacion}{Recursos
  De Bibliografia Y Documentacion}
\item
  \href{https://achalmaedison.netlify.app/investigacion-metodologia/posts/2025-02-09-recursos-para-traducción-y-correccion/index.pdf}{\faIcon{file-pdf}}
  \href{https://achalmaedison.netlify.app/investigacion-metodologia/posts/2025-02-09-recursos-para-traducción-y-correccion}{Recursos
  Para Traducción Y Correccion}
\end{enumerate}

Esperamos que encuentres estas publicaciones igualmente interesantes y
útiles. ¡Disfruta de la lectura!






\end{document}
