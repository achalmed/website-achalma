\documentclass[
  jou,
  floatsintext,
  longtable,
  a4paper,
  nolmodern,
  notxfonts,
  notimes,
  colorlinks=true,linkcolor=blue,citecolor=blue,urlcolor=blue]{apa7}

\usepackage{amsmath}
\usepackage{amssymb}



\usepackage[bidi=default]{babel}
\babelprovide[main,import]{spanish}
\StartBabelCommands{spanish}{captions} [unicode, fontenc=TU EU1 EU2, charset=utf8] \SetString{\keywordname}{Palabras
Claves}
\EndBabelCommands


% get rid of language-specific shorthands (see #6817):
\let\LanguageShortHands\languageshorthands
\def\languageshorthands#1{}

\RequirePackage{longtable}
\RequirePackage{threeparttablex}

\makeatletter
\renewcommand{\paragraph}{\@startsection{paragraph}{4}{\parindent}%
	{0\baselineskip \@plus 0.2ex \@minus 0.2ex}%
	{-.5em}%
	{\normalfont\normalsize\bfseries\typesectitle}}

\renewcommand{\subparagraph}[1]{\@startsection{subparagraph}{5}{0.5em}%
	{0\baselineskip \@plus 0.2ex \@minus 0.2ex}%
	{-\z@\relax}%
	{\normalfont\normalsize\bfseries\itshape\hspace{\parindent}{#1}\textit{\addperi}}{\relax}}
\makeatother




\usepackage{longtable, booktabs, multirow, multicol, colortbl, hhline, caption, array, float, xpatch}
\usepackage{subcaption}
\renewcommand\thesubfigure{\Alph{subfigure}}
\setcounter{topnumber}{2}
\setcounter{bottomnumber}{2}
\setcounter{totalnumber}{4}
\renewcommand{\topfraction}{0.85}
\renewcommand{\bottomfraction}{0.85}
\renewcommand{\textfraction}{0.15}
\renewcommand{\floatpagefraction}{0.7}

\usepackage{tcolorbox}
\tcbuselibrary{listings,theorems, breakable, skins}
\usepackage{fontawesome5}

\definecolor{quarto-callout-color}{HTML}{909090}
\definecolor{quarto-callout-note-color}{HTML}{0758E5}
\definecolor{quarto-callout-important-color}{HTML}{CC1914}
\definecolor{quarto-callout-warning-color}{HTML}{EB9113}
\definecolor{quarto-callout-tip-color}{HTML}{00A047}
\definecolor{quarto-callout-caution-color}{HTML}{FC5300}
\definecolor{quarto-callout-color-frame}{HTML}{ACACAC}
\definecolor{quarto-callout-note-color-frame}{HTML}{4582EC}
\definecolor{quarto-callout-important-color-frame}{HTML}{D9534F}
\definecolor{quarto-callout-warning-color-frame}{HTML}{F0AD4E}
\definecolor{quarto-callout-tip-color-frame}{HTML}{02B875}
\definecolor{quarto-callout-caution-color-frame}{HTML}{FD7E14}

%\newlength\Oldarrayrulewidth
%\newlength\Oldtabcolsep


\usepackage{hyperref}




\providecommand{\tightlist}{%
  \setlength{\itemsep}{0pt}\setlength{\parskip}{0pt}}
\usepackage{longtable,booktabs,array}
\usepackage{calc} % for calculating minipage widths
% Correct order of tables after \paragraph or \subparagraph
\usepackage{etoolbox}
\makeatletter
\patchcmd\longtable{\par}{\if@noskipsec\mbox{}\fi\par}{}{}
\makeatother
% Allow footnotes in longtable head/foot
\IfFileExists{footnotehyper.sty}{\usepackage{footnotehyper}}{\usepackage{footnote}}
\makesavenoteenv{longtable}

\usepackage{graphicx}
\makeatletter
\newsavebox\pandoc@box
\newcommand*\pandocbounded[1]{% scales image to fit in text height/width
  \sbox\pandoc@box{#1}%
  \Gscale@div\@tempa{\textheight}{\dimexpr\ht\pandoc@box+\dp\pandoc@box\relax}%
  \Gscale@div\@tempb{\linewidth}{\wd\pandoc@box}%
  \ifdim\@tempb\p@<\@tempa\p@\let\@tempa\@tempb\fi% select the smaller of both
  \ifdim\@tempa\p@<\p@\scalebox{\@tempa}{\usebox\pandoc@box}%
  \else\usebox{\pandoc@box}%
  \fi%
}
% Set default figure placement to htbp
\def\fps@figure{htbp}
\makeatother







\usepackage{newtx}

\defaultfontfeatures{Scale=MatchLowercase}
\defaultfontfeatures[\rmfamily]{Ligatures=TeX,Scale=1}





\title{Herramientas Digitales para la Traducción y Corrección: Recursos
para Investigadores y Estudiantes}


\shorttitle{Recursos de Investigación}


\usepackage{etoolbox}



\ccoppy{\textcopyright~2025}



\author{Edison Achalma}



\affiliation{
{Escuela Profesional de Economía, Universidad Nacional de San Cristóbal
de Huamanga}}




\leftheader{Achalma}

\date{2025-01-12}


\abstract{This article provides an overview of digital tools that go
beyond literal translation, aiming to enhance accuracy in academic and
research contexts. It lists various websites offering non-literal
translations, grammar correction tools, and resources for finding
synonyms and deeper word meanings. These resources are particularly
valuable for researchers and students dealing with multilingual texts,
ensuring precision in translation and writing. The article does not
evaluate these tools but presents them as options for improving
translation and writing quality. }

\keywords{Translation Tools, Grammar Correction, Academic
Writing, Non-Literal Translation, Multilingual Resources}

\authornote{\par{\addORCIDlink{Edison Achalma}{0000-0001-6996-3364}} 
\par{ }
\par{   El autor no tiene conflictos de interés que revelar.    Los
roles de autor se clasificaron utilizando la taxonomía de roles de
colaborador (CRediT; https://credit.niso.org/) de la siguiente
manera:  Edison Achalma:   conceptualización, redacción}
\par{La correspondencia relativa a este artículo debe dirigirse a Edison
Achalma, Email: \href{mailto:elmer.achalma.09@unsch.edu.pe}{elmer.achalma.09@unsch.edu.pe}}
}

\usepackage{pbalance} 
\usepackage{float}
\makeatletter
\let\oldtpt\ThreePartTable
\let\endoldtpt\endThreePartTable
\def\ThreePartTable{\@ifnextchar[\ThreePartTable@i \ThreePartTable@ii}
\def\ThreePartTable@i[#1]{\begin{figure}[!htbp]
\onecolumn
\begin{minipage}{0.5\textwidth}
\oldtpt[#1]
}
\def\ThreePartTable@ii{\begin{figure}[!htbp]
\onecolumn
\begin{minipage}{0.5\textwidth}
\oldtpt
}
\def\endThreePartTable{
\endoldtpt
\end{minipage}
\twocolumn
\end{figure}}
\makeatother


\makeatletter
\let\endoldlt\endlongtable		
\def\endlongtable{
\hline
\endoldlt}
\makeatother

\newenvironment{twocolumntable}% environment name
{% begin code
\begin{table*}[!htbp]%
\onecolumn%
}%
{%
\twocolumn%
\end{table*}%
}% end code

\urlstyle{same}



\makeatletter
\@ifpackageloaded{caption}{}{\usepackage{caption}}
\AtBeginDocument{%
\ifdefined\contentsname
  \renewcommand*\contentsname{Tabla de contenidos}
\else
  \newcommand\contentsname{Tabla de contenidos}
\fi
\ifdefined\listfigurename
  \renewcommand*\listfigurename{Listado de Figuras}
\else
  \newcommand\listfigurename{Listado de Figuras}
\fi
\ifdefined\listtablename
  \renewcommand*\listtablename{Listado de Tablas}
\else
  \newcommand\listtablename{Listado de Tablas}
\fi
\ifdefined\figurename
  \renewcommand*\figurename{Figura}
\else
  \newcommand\figurename{Figura}
\fi
\ifdefined\tablename
  \renewcommand*\tablename{Tabla}
\else
  \newcommand\tablename{Tabla}
\fi
}
\@ifpackageloaded{float}{}{\usepackage{float}}
\floatstyle{ruled}
\@ifundefined{c@chapter}{\newfloat{codelisting}{h}{lop}}{\newfloat{codelisting}{h}{lop}[chapter]}
\floatname{codelisting}{Listado}
\newcommand*\listoflistings{\listof{codelisting}{Listado de Listados}}
\makeatother
\makeatletter
\makeatother
\makeatletter
\@ifpackageloaded{caption}{}{\usepackage{caption}}
\@ifpackageloaded{subcaption}{}{\usepackage{subcaption}}
\makeatother
\makeatletter
\@ifpackageloaded{fontawesome5}{}{\usepackage{fontawesome5}}
\makeatother

% From https://tex.stackexchange.com/a/645996/211326
%%% apa7 doesn't want to add appendix section titles in the toc
%%% let's make it do it
\makeatletter
\xpatchcmd{\appendix}
  {\par}
  {\addcontentsline{toc}{section}{\@currentlabelname}\par}
  {}{}
\makeatother

%% Disable longtable counter
%% https://tex.stackexchange.com/a/248395/211326

\usepackage{etoolbox}

\makeatletter
\patchcmd{\LT@caption}
  {\bgroup}
  {\bgroup\global\LTpatch@captiontrue}
  {}{}
\patchcmd{\longtable}
  {\par}
  {\par\global\LTpatch@captionfalse}
  {}{}
\apptocmd{\endlongtable}
  {\ifLTpatch@caption\else\addtocounter{table}{-1}\fi}
  {}{}
\newif\ifLTpatch@caption
\makeatother

\begin{document}

\maketitle

\hypertarget{toc}{}
\tableofcontents
\newpage
\section[Introduction]{Herramientas Digitales para la Traducción y
Corrección}

\setcounter{secnumdepth}{-\maxdimen} % remove section numbering

\setlength\LTleft{0pt}


\section{A todos los investigadores y
estudiantes}\label{a-todos-los-investigadores-y-estudiantes}

En el ámbito académico y de investigación, la precisión de la traducción
es crucial. Google Translate, aunque útil, a menudo traduce frases
largas de manera literal, es decir, palabra por palabra, lo que puede
llevar a errores de interpretación. Por ello, eh recopilado una lista de
recursos más eficaces para ayudarte a traducir con mayor exactitud,
corregir gramática y encontrar sinónimos.

\subsection{Sitios web de traducción no
literal}\label{sitios-web-de-traducciuxf3n-no-literal}

Estos sitios van más allá de la traducción literal para proporcionar
significados más adecuados en el contexto:

\begin{itemize}
\tightlist
\item
  \href{http://mobile.reverso.net/en}{Reverso}
\item
  \href{https://www.wordreference.com/}{WordReference}
\item
  \href{http://www.worldlingo.com/}{WorldLingo}
\item
  \href{https://www.babelfish.com/}{Babelfish}
\item
  \href{http://translation2.paralink.com/}{Paralink Translation}
\item
  \href{https://www.freetranslation.com/}{FreeTranslation}
\end{itemize}

\subsection{Sitios web para corrección y
gramática}\label{sitios-web-para-correcciuxf3n-y-gramuxe1tica}

Para asegurar que tu escrito en inglés sea gramaticalmente correcto:

\begin{itemize}
\tightlist
\item
  \href{https://www.grammarly.com/m}{Grammarly}
\item
  \href{https://www.grammarcheck.net/}{GrammarCheck}
\item
  \href{http://mobile.reverso.net/en}{Reverso Spelling}
\item
  \href{https://www.onlinecorrection.com/}{OnlineCorrection}
\item
  \href{https://spellcheckplus.com/}{SpellCheckPlus}
\end{itemize}

\subsection{Sitios web para significados y
sinónimos}\label{sitios-web-para-significados-y-sinuxf3nimos}

Explora el significado más profundo de las palabras y encuentra
alternativas expresivas:

\begin{itemize}
\tightlist
\item
  \href{https://www.thesaurus.com/}{Thesaurus}
\item
  \href{http://www.englishdaily626.com/}{English Daily 626}
\end{itemize}

\subsection{Mejores sitios para traducción a nivel
mundial}\label{mejores-sitios-para-traducciuxf3n-a-nivel-mundial}

Listado de herramientas reconocidas por su precisión en la traducción
internacional:

\begin{itemize}
\tightlist
\item
  \href{https://www.deepl.com/es/home}{DeepL} Conocido por sus
  traducciones de alta calidad y contextually precisas.
\item
  \href{http://www.worldlingo.com/}{WorldLingo} Ideal para traducir
  documentos largos y complejos.
\item
  \href{https://www.freetranslation.com/}{FreeTranslation}
\item
  \href{https://www.babelfish.com/}{Babelfish}
\item
  \href{https://www.wordreference.com/}{WordReference}
\item
  \href{https://www.translation2.paralink.com/}{Paralink Translation}
  Especialmente útil para la traducción de archivos de investigación.
\item
  \href{https://www.onlinecorrection.com/}{OnlineCorrection}
\item
  \href{https://www.grammarcheck.net/}{GrammarCheck}
\end{itemize}

\subsection{Sitios para corregir la
escritura}\label{sitios-para-corregir-la-escritura}

Mejora tu escritura con estas herramientas de corrección:

\begin{itemize}
\tightlist
\item
  \href{http://www.afterthedeadline.com/}{After the Deadline}
\item
  \href{http://www.reverso.net/spell-chec/english-spelling-grammar/}{Reverso
  Spell Check}
\item
  \href{https://prowritingaid.com/}{ProWritingAid}
\end{itemize}

\subsection{Diccionario multilingüe:}\label{diccionario-multilinguxfce}

Para aquellos momentos en que necesitas consultar términos en varios
idiomas:

\begin{itemize}
\tightlist
\item
  \href{https://www.linguee.es/}{Linguee}
\end{itemize}

Esperamos que estos recursos te ayuden a mejorar tus traducciones y
escritos, facilitando tu labor académica y de investigación.

\section{Publicaciones Similares}\label{publicaciones-similares}

Si te interesó este artículo, te recomendamos que explores otros blogs y
recursos relacionados que pueden ampliar tus conocimientos. Aquí te dejo
algunas sugerencias:

\begin{enumerate}
\def\labelenumi{\arabic{enumi}.}
\tightlist
\item
  \href{https://achalmaedison.netlify.app/investigacion-metodologia/posts/2023-06-03-ideas-de-investigacion-para-economia/index.pdf}{\faIcon{file-pdf}}
  \href{https://achalmaedison.netlify.app/investigacion-metodologia/posts/2023-06-03-ideas-de-investigacion-para-economia}{Ideas
  De Investigacion Para Economia}
\item
  \href{https://achalmaedison.netlify.app/investigacion-metodologia/posts/2023-06-03-pautas-de-presentacion-del-informe-de-investigacion/index.pdf}{\faIcon{file-pdf}}
  \href{https://achalmaedison.netlify.app/investigacion-metodologia/posts/2023-06-03-pautas-de-presentacion-del-informe-de-investigacion}{Pautas
  De Presentacion Del Informe De Investigacion}
\item
  \href{https://achalmaedison.netlify.app/investigacion-metodologia/posts/2025-01-12-recursos-de-bibliografia-y-documentacion/index.pdf}{\faIcon{file-pdf}}
  \href{https://achalmaedison.netlify.app/investigacion-metodologia/posts/2025-01-12-recursos-de-bibliografia-y-documentacion}{Recursos
  De Bibliografia Y Documentacion}
\item
  \href{https://achalmaedison.netlify.app/investigacion-metodologia/posts/2025-02-09-recursos-para-traducción-y-correccion/index.pdf}{\faIcon{file-pdf}}
  \href{https://achalmaedison.netlify.app/investigacion-metodologia/posts/2025-02-09-recursos-para-traducción-y-correccion}{Recursos
  Para Traducción Y Correccion}
\end{enumerate}

Esperamos que encuentres estas publicaciones igualmente interesantes y
útiles. ¡Disfruta de la lectura!






\end{document}
