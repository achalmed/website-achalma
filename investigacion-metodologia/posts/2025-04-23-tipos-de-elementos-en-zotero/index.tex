\documentclass[
  jou,
  floatsintext,
  longtable,
  a4paper,
  nolmodern,
  notxfonts,
  notimes,
  colorlinks=true,linkcolor=blue,citecolor=blue,urlcolor=blue]{apa7}

\usepackage{amsmath}
\usepackage{amssymb}



\usepackage[bidi=default]{babel}
\babelprovide[main,import]{spanish}
\StartBabelCommands{spanish}{captions} [unicode, fontenc=TU EU1 EU2, charset=utf8] \SetString{\keywordname}{Palabras
Claves}
\EndBabelCommands


% get rid of language-specific shorthands (see #6817):
\let\LanguageShortHands\languageshorthands
\def\languageshorthands#1{}

\RequirePackage{longtable}
\RequirePackage{threeparttablex}

\makeatletter
\renewcommand{\paragraph}{\@startsection{paragraph}{4}{\parindent}%
	{0\baselineskip \@plus 0.2ex \@minus 0.2ex}%
	{-.5em}%
	{\normalfont\normalsize\bfseries\typesectitle}}

\renewcommand{\subparagraph}[1]{\@startsection{subparagraph}{5}{0.5em}%
	{0\baselineskip \@plus 0.2ex \@minus 0.2ex}%
	{-\z@\relax}%
	{\normalfont\normalsize\bfseries\itshape\hspace{\parindent}{#1}\textit{\addperi}}{\relax}}
\makeatother




\usepackage{longtable, booktabs, multirow, multicol, colortbl, hhline, caption, array, float, xpatch}
\usepackage{subcaption}
\renewcommand\thesubfigure{\Alph{subfigure}}
\setcounter{topnumber}{2}
\setcounter{bottomnumber}{2}
\setcounter{totalnumber}{4}
\renewcommand{\topfraction}{0.85}
\renewcommand{\bottomfraction}{0.85}
\renewcommand{\textfraction}{0.15}
\renewcommand{\floatpagefraction}{0.7}

\usepackage{tcolorbox}
\tcbuselibrary{listings,theorems, breakable, skins}
\usepackage{fontawesome5}

\definecolor{quarto-callout-color}{HTML}{909090}
\definecolor{quarto-callout-note-color}{HTML}{0758E5}
\definecolor{quarto-callout-important-color}{HTML}{CC1914}
\definecolor{quarto-callout-warning-color}{HTML}{EB9113}
\definecolor{quarto-callout-tip-color}{HTML}{00A047}
\definecolor{quarto-callout-caution-color}{HTML}{FC5300}
\definecolor{quarto-callout-color-frame}{HTML}{ACACAC}
\definecolor{quarto-callout-note-color-frame}{HTML}{4582EC}
\definecolor{quarto-callout-important-color-frame}{HTML}{D9534F}
\definecolor{quarto-callout-warning-color-frame}{HTML}{F0AD4E}
\definecolor{quarto-callout-tip-color-frame}{HTML}{02B875}
\definecolor{quarto-callout-caution-color-frame}{HTML}{FD7E14}

%\newlength\Oldarrayrulewidth
%\newlength\Oldtabcolsep


\usepackage{hyperref}




\providecommand{\tightlist}{%
  \setlength{\itemsep}{0pt}\setlength{\parskip}{0pt}}
\usepackage{longtable,booktabs,array}
\usepackage{calc} % for calculating minipage widths
% Correct order of tables after \paragraph or \subparagraph
\usepackage{etoolbox}
\makeatletter
\patchcmd\longtable{\par}{\if@noskipsec\mbox{}\fi\par}{}{}
\makeatother
% Allow footnotes in longtable head/foot
\IfFileExists{footnotehyper.sty}{\usepackage{footnotehyper}}{\usepackage{footnote}}
\makesavenoteenv{longtable}

\usepackage{graphicx}
\makeatletter
\newsavebox\pandoc@box
\newcommand*\pandocbounded[1]{% scales image to fit in text height/width
  \sbox\pandoc@box{#1}%
  \Gscale@div\@tempa{\textheight}{\dimexpr\ht\pandoc@box+\dp\pandoc@box\relax}%
  \Gscale@div\@tempb{\linewidth}{\wd\pandoc@box}%
  \ifdim\@tempb\p@<\@tempa\p@\let\@tempa\@tempb\fi% select the smaller of both
  \ifdim\@tempa\p@<\p@\scalebox{\@tempa}{\usebox\pandoc@box}%
  \else\usebox{\pandoc@box}%
  \fi%
}
% Set default figure placement to htbp
\def\fps@figure{htbp}
\makeatother







\usepackage{newtx}

\defaultfontfeatures{Scale=MatchLowercase}
\defaultfontfeatures[\rmfamily]{Ligatures=TeX,Scale=1}





\title{Tipos de Elementos en Zotero: Guía para Citas}


\shorttitle{Guía de Tipos de Elementos en Zotero}


\usepackage{etoolbox}



\ccoppy{\textcopyright~2025}



\author{Edison Achalma}



\affiliation{
{Escuela Profesional de Economía, Universidad Nacional de San Cristóbal
de Huamanga}}




\leftheader{Achalma}

\date{2025-04-23}


\abstract{This article provides a comprehensive guide to the item types
supported by Zotero, a widely used reference management tool. It details
the purpose, fields, and appropriate use of each item type, such as
Artwork, Book, Journal Article, and more, based on official Zotero
documentation. The guide emphasizes the importance of selecting the
correct item type to ensure accurate citations and compliance with APA
standards. It also covers the use of the ``Extra'' field for unsupported
types and additional metadata, offering practical tips for academic
researchers and students. The article aligns with SEO best practices to
enhance discoverability while maintaining academic rigor for scholarly
publications. }

\keywords{Zotero item types, reference management, APA
citation, academic writing, bibliographic metadata}

\authornote{\par{\addORCIDlink{Edison Achalma}{0000-0001-6996-3364}} 
\par{ }
\par{   El autor no tiene conflictos de interés que revelar.    Los
roles de autor se clasificaron utilizando la taxonomía de roles de
colaborador (CRediT; https://credit.niso.org/) de la siguiente
manera:  Edison Achalma:   conceptualización, redacción}
\par{La correspondencia relativa a este artículo debe dirigirse a Edison
Achalma, Email: \href{mailto:elmer.achalma.09@unsch.edu.pe}{elmer.achalma.09@unsch.edu.pe}}
}

\usepackage{pbalance} 
\usepackage{float}
\makeatletter
\let\oldtpt\ThreePartTable
\let\endoldtpt\endThreePartTable
\def\ThreePartTable{\@ifnextchar[\ThreePartTable@i \ThreePartTable@ii}
\def\ThreePartTable@i[#1]{\begin{figure}[!htbp]
\onecolumn
\begin{minipage}{0.5\textwidth}
\oldtpt[#1]
}
\def\ThreePartTable@ii{\begin{figure}[!htbp]
\onecolumn
\begin{minipage}{0.5\textwidth}
\oldtpt
}
\def\endThreePartTable{
\endoldtpt
\end{minipage}
\twocolumn
\end{figure}}
\makeatother


\makeatletter
\let\endoldlt\endlongtable		
\def\endlongtable{
\hline
\endoldlt}
\makeatother

\newenvironment{twocolumntable}% environment name
{% begin code
\begin{table*}[!htbp]%
\onecolumn%
}%
{%
\twocolumn%
\end{table*}%
}% end code

\urlstyle{same}



\makeatletter
\@ifpackageloaded{caption}{}{\usepackage{caption}}
\AtBeginDocument{%
\ifdefined\contentsname
  \renewcommand*\contentsname{Tabla de contenidos}
\else
  \newcommand\contentsname{Tabla de contenidos}
\fi
\ifdefined\listfigurename
  \renewcommand*\listfigurename{Listado de Figuras}
\else
  \newcommand\listfigurename{Listado de Figuras}
\fi
\ifdefined\listtablename
  \renewcommand*\listtablename{Listado de Tablas}
\else
  \newcommand\listtablename{Listado de Tablas}
\fi
\ifdefined\figurename
  \renewcommand*\figurename{Figura}
\else
  \newcommand\figurename{Figura}
\fi
\ifdefined\tablename
  \renewcommand*\tablename{Tabla}
\else
  \newcommand\tablename{Tabla}
\fi
}
\@ifpackageloaded{float}{}{\usepackage{float}}
\floatstyle{ruled}
\@ifundefined{c@chapter}{\newfloat{codelisting}{h}{lop}}{\newfloat{codelisting}{h}{lop}[chapter]}
\floatname{codelisting}{Listado}
\newcommand*\listoflistings{\listof{codelisting}{Listado de Listados}}
\makeatother
\makeatletter
\makeatother
\makeatletter
\@ifpackageloaded{caption}{}{\usepackage{caption}}
\@ifpackageloaded{subcaption}{}{\usepackage{subcaption}}
\makeatother
\makeatletter
\@ifpackageloaded{fontawesome5}{}{\usepackage{fontawesome5}}
\makeatother

% From https://tex.stackexchange.com/a/645996/211326
%%% apa7 doesn't want to add appendix section titles in the toc
%%% let's make it do it
\makeatletter
\xpatchcmd{\appendix}
  {\par}
  {\addcontentsline{toc}{section}{\@currentlabelname}\par}
  {}{}
\makeatother

%% Disable longtable counter
%% https://tex.stackexchange.com/a/248395/211326

\usepackage{etoolbox}

\makeatletter
\patchcmd{\LT@caption}
  {\bgroup}
  {\bgroup\global\LTpatch@captiontrue}
  {}{}
\patchcmd{\longtable}
  {\par}
  {\par\global\LTpatch@captionfalse}
  {}{}
\apptocmd{\endlongtable}
  {\ifLTpatch@caption\else\addtocounter{table}{-1}\fi}
  {}{}
\newif\ifLTpatch@caption
\makeatother

\begin{document}

\maketitle

\hypertarget{toc}{}
\tableofcontents
\newpage
\section[Introduction]{Tipos de Elementos en Zotero}

\setcounter{secnumdepth}{-\maxdimen} % remove section numbering

\setlength\LTleft{0pt}


Esta guía proporciona una descripción detallada de los tipos de
elementos soportados por Zotero, basada en la documentación oficial de
Zotero.

\subsection{1. Artwork (Obra de arte)}\label{artwork-obra-de-arte}

\begin{itemize}
\tightlist
\item
  \textbf{Cuándo utilizar}:

  \begin{itemize}
  \tightlist
  \item
    Para registrar obras de arte como pinturas al óleo, esculturas,
    fotografías o cualquier tipo de imagen visual, incluidas figuras
    científicas (por ejemplo, gráficos o diagramas).
  \item
    Usar este tipo en lugar de ``Document'' para elementos visuales con
    un enfoque artístico o científico que requieran citar al creador
    como un artista.
  \end{itemize}
\item
  \textbf{Campos y su propósito}:

  \begin{itemize}
  \tightlist
  \item
    \textbf{Item Type}: Seleccionar ``Artwork'' para categorizar como
    obra de arte.
  \item
    \textbf{Artist}: Nombre del creador de la obra (persona o entidad).
    Ingresar en formato de dos campos (Apellido, Nombre) para personas o
    un campo para instituciones.
  \item
    \textbf{Title}: Título principal de la obra, en mayúsculas según el
    idioma (sentence case).
  \item
    \textbf{Abstract}: Breve descripción de la obra, si es relevante.
  \item
    \textbf{Date}: Año de creación o publicación de la obra.
  \item
    \textbf{Medium}: Tipo de obra o material usado (por ejemplo,
    ``Pintura al óleo'', ``Escultura de madera'', ``Gráfico de
    dispersión'').
  \item
    \textbf{Artwork Size}: Dimensiones físicas de la obra (por ejemplo,
    ``50x70 cm'').
  \item
    \textbf{Place}: Lugar donde se creó o exhibe la obra, si aplica.
  \item
    \textbf{Publisher}: Institución o galería que publica o exhibe la
    obra.
  \item
    \textbf{Language}: Código ISO del idioma (por ejemplo, ``es-ES''
    para español).
  \item
    \textbf{Short Title}: Versión abreviada del título para citas
    posteriores en estilos de notas.
  \item
    \textbf{URL}: Dirección web donde se puede acceder a una
    representación digital de la obra.
  \item
    \textbf{Accessed}: Fecha en que se accedió a la obra en línea.
  \item
    \textbf{Archive}: Nombre del archivo o museo donde se encuentra la
    obra.
  \item
    \textbf{Loc. in Archive}: Ubicación específica en el archivo (por
    ejemplo, número de sala o caja).
  \item
    \textbf{Library Catalog}: Catálogo o base de datos de donde se
    importó la información.
  \item
    \textbf{Call Number}: Número de clasificación en una biblioteca o
    archivo.
  \item
    \textbf{Rights}: Información sobre derechos de autor o licencias.
  \item
    \textbf{Extra}: Información adicional, como notas o campos no
    soportados (por ejemplo, ``Original Date: 1886'').
  \item
    \textbf{Date Added/Modified}: Campos automáticos con la fecha de
    adición o modificación.
  \end{itemize}
\item
  \textbf{Información oficial de Zotero}:

  \begin{itemize}
  \tightlist
  \item
    Este tipo es ideal para citar obras de arte o imágenes visuales,
    diferenciándolas de documentos genéricos. La documentación
    recomienda usar el campo ``Medium'' para describir el material o
    tipo de obra, y ``Artwork Size'' para dimensiones físicas. Si la
    obra no encaja perfectamente, se puede adaptar usando el campo
    ``Extra'' para detalles específicos.
  \end{itemize}
\end{itemize}

\subsection{2. Audio Recording (Grabación de
audio)}\label{audio-recording-grabaciuxf3n-de-audio}

\begin{itemize}
\tightlist
\item
  \textbf{Cuándo utilizar}:

  \begin{itemize}
  \tightlist
  \item
    Para música, grabaciones de voz, efectos de sonido, grabaciones de
    archivo o figuras científicas basadas en audio.
  \item
    Usar en lugar de ``Podcast'' si no es un programa distribuido en
    línea por suscripción.
  \end{itemize}
\item
  \textbf{Campos y su propósito}:

  \begin{itemize}
  \tightlist
  \item
    \textbf{Item Type}: Seleccionar ``Audio Recording''.
  \item
    \textbf{Performer}: Nombre del intérprete principal (persona o
    grupo). Ingresar en formato de dos campos para personas.
  \item
    \textbf{Composer}: Nombre del compositor de la música, si aplica.
  \item
    \textbf{Words By}: Autor de las letras o texto hablado (por ejemplo,
    letrista o guionista).
  \item
    \textbf{Title}: Título de la grabación (por ejemplo, nombre de la
    canción o pieza).
  \item
    \textbf{Abstract}: Descripción breve del contenido.
  \item
    \textbf{Date}: Fecha de publicación o grabación.
  \item
    \textbf{Format}: Formato de la grabación (por ejemplo, ``CD'',
    ``MP3'').
  \item
    \textbf{Running Time}: Duración de la grabación (por ejemplo, ``3:45
    mins'').
  \item
    \textbf{Label}: Compañía discográfica que distribuye la grabación.
  \item
    \textbf{Series}: Nombre de una serie de grabaciones, si aplica.
  \item
    \textbf{Series Title}: Título de una serie específica dentro de una
    publicación.
  \item
    \textbf{Volume}: Número de volumen, si es parte de una colección.
  \item
    \textbf{Place}: Lugar de publicación o grabación.
  \item
    \textbf{Publisher}: Editorial o productora de la grabación.
  \item
    \textbf{ISBN}: Número ISBN, si la grabación está publicada como
    parte de un conjunto.
  \item
    \textbf{Language}: Código ISO del idioma.
  \item
    \textbf{Short Title}: Título abreviado para citas posteriores.
  \item
    \textbf{URL}: Dirección web de la grabación.
  \item
    \textbf{Accessed}: Fecha de acceso en línea.
  \item
    \textbf{Archive/Loc. in Archive}: Archivo y ubicación específica
    donde se encuentra.
  \item
    \textbf{Library Catalog/Call Number}: Catálogo y número de
    clasificación.
  \item
    \textbf{Rights/Extra/Date Added/Modified}: Igual que en ``Artwork''.
  \end{itemize}
\item
  \textbf{Información oficial de Zotero}:

  \begin{itemize}
  \tightlist
  \item
    Este tipo cubre cualquier grabación de audio, desde música hasta
    grabaciones históricas. El campo ``Format'' es clave para
    especificar el medio físico o digital. Los creadores como
    ``Performer'' o ``Composer'' son esenciales para citas precisas.
    Para roles adicionales (por ejemplo, productores), usar el campo
    ``Extra'' con el formato ``Director: Nombre \textbar\textbar{}
    Apellido''.
  \end{itemize}
\end{itemize}

\subsection{3. Bill (Proyecto de ley)}\label{bill-proyecto-de-ley}

\begin{itemize}
\tightlist
\item
  \textbf{Cuándo utilizar}:

  \begin{itemize}
  \tightlist
  \item
    Para propuestas de legislación presentadas ante un cuerpo
    legislativo.
  \item
    Usar en lugar de ``Statute'' si la legislación aún no ha sido
    aprobada.
  \end{itemize}
\item
  \textbf{Campos y su propósito}:

  \begin{itemize}
  \tightlist
  \item
    \textbf{Item Type}: Seleccionar ``Bill''.
  \item
    \textbf{Sponsor}: Autor principal del proyecto (persona o entidad).
  \item
    \textbf{Cosponsor}: Coautores o patrocinadores adicionales.
  \item
    \textbf{Title}: Título oficial del proyecto de ley.
  \item
    \textbf{Abstract}: Resumen del contenido del proyecto.
  \item
    \textbf{Bill Number}: Número asignado al proyecto de ley.
  \item
    \textbf{Code}: Nombre del código donde se publica el proyecto.
  \item
    \textbf{Code Volume}: Volumen del código.
  \item
    \textbf{Code Number}: Número específico dentro del código.
  \item
    \textbf{Section}: Sección específica del proyecto citada.
  \item
    \textbf{Date}: Fecha de presentación del proyecto.
  \item
    \textbf{Legislative Body}: Cuerpo legislativo que debate el proyecto
    (por ejemplo, ``Senado'').
  \item
    \textbf{Session}: Sesión legislativa en la que se presentó.
  \item
    \textbf{History}: Información sobre el historial procesal del
    proyecto.
  \item
    \textbf{Language/Short Title/URL/Accessed/Rights/Extra/Date
    Added/Modified}: Igual que en ``Artwork''.
  \end{itemize}
\item
  \textbf{Información oficial de Zotero}:

  \begin{itemize}
  \tightlist
  \item
    Este tipo es específico para legislación propuesta. El campo ``Bill
    Number'' es crucial para identificar el proyecto, y ``History''
    permite rastrear su progreso. La documentación sugiere usar
    ``Extra'' para campos legales adicionales no soportados, como
    ``Event Place''.
  \end{itemize}
\end{itemize}

\subsection{4. Blog Post (Entrada de
blog)}\label{blog-post-entrada-de-blog}

\begin{itemize}
\tightlist
\item
  \textbf{Cuándo utilizar}:

  \begin{itemize}
  \tightlist
  \item
    Para artículos o entradas publicadas en un blog personal.
  \item
    Usar ``Magazine Article'' o ``Newspaper Article'' para publicaciones
    en línea de medios más grandes (por ejemplo, blogs de NYT).
  \end{itemize}
\item
  \textbf{Campos y su propósito}:

  \begin{itemize}
  \tightlist
  \item
    \textbf{Item Type}: Seleccionar ``Blog Post''.
  \item
    \textbf{Author}: Autor de la entrada, en formato de dos campos.
  \item
    \textbf{Commenter}: Persona que comenta la entrada (no usado en
    citas).
  \item
    \textbf{Title}: Título de la entrada del blog.
  \item
    \textbf{Abstract}: Resumen del contenido.
  \item
    \textbf{Blog Title}: Nombre del blog que contiene la entrada.
  \item
    \textbf{Website Type}: Género del sitio web (por ejemplo, ``Blog
    personal''), raramente usado.
  \item
    \textbf{Date}: Fecha de publicación de la entrada.
  \item
    \textbf{Language/Short Title/URL/Accessed/Rights/Extra/Date
    Added/Modified}: Igual que en ``Artwork''.
  \end{itemize}
\item
  \textbf{Información oficial de Zotero}:

  \begin{itemize}
  \tightlist
  \item
    Este tipo es ideal para blogs personales, diferenciándolos de
    artículos de medios establecidos. El campo ``Blog Title'' es
    esencial para contextualizar la entrada. La documentación recomienda
    usar ``Webpage'' si no se ajusta a otros tipos más específicos.
  \end{itemize}
\end{itemize}

\subsection{5. Book (Libro)}\label{book-libro}

\begin{itemize}
\tightlist
\item
  \textbf{Cuándo utilizar}:

  \begin{itemize}
  \tightlist
  \item
    Para libros publicados, incluidas monografías, antologías o textos
    académicos.
  \item
    Usar ``Report'' para documentos gubernamentales, informes técnicos o
    manuales.
  \item
    Puede adaptarse para ítems inusuales si otros tipos no encajan.
  \end{itemize}
\item
  \textbf{Campos y su propósito}:

  \begin{itemize}
  \tightlist
  \item
    \textbf{Item Type}: Seleccionar ``Book''.
  \item
    \textbf{Author}: Autor principal del libro.
  \item
    \textbf{Editor}: Editor del libro, si aplica.
  \item
    \textbf{Translator}: Traductor, si el libro es una traducción.
  \item
    \textbf{Contributor}: Colaboradores adicionales (no usados en
    citas).
  \item
    \textbf{Title}: Título completo del libro.
  \item
    \textbf{Abstract}: Resumen del contenido.
  \item
    \textbf{Series}: Nombre de la serie editorial, si aplica.
  \item
    \textbf{Series Number}: Número del libro en la serie.
  \item
    \textbf{Volume}: Volumen del libro, si es parte de una colección.
  \item
    \textbf{\# of Volumes}: Número total de volúmenes en la obra.
  \item
    \textbf{Edition}: Número de edición (por ejemplo, ``2'' para segunda
    edición).
  \item
    \textbf{Place}: Lugar de publicación.
  \item
    \textbf{Publisher}: Editorial que publicó el libro.
  \item
    \textbf{Date}: Año de publicación.
  \item
    \textbf{\# of Pages}: Número total de páginas.
  \item
    \textbf{ISBN}: Número ISBN del libro.
  \item
    \textbf{Language/Short Title/URL/Accessed/Archive/Loc. in
    Archive/Library Catalog/Call Number/Rights/Extra/Date
    Added/Modified}: Igual que en ``Artwork''.
  \end{itemize}
\item
  \textbf{Información oficial de Zotero}:

  \begin{itemize}
  \tightlist
  \item
    Este tipo es versátil y puede adaptarse a muchos ítems inusuales. La
    documentación recomienda usar ``Type'' o ``Extra'' para especificar
    subtipos (por ejemplo, ``Novela'' o ``Biografía''). El campo
    ``Edition'' debe contener solo números, y ``Series'' es clave para
    colecciones editoriales.
  \end{itemize}
\end{itemize}

\subsection{6. Book Section (Sección de
libro)}\label{book-section-secciuxf3n-de-libro}

\begin{itemize}
\tightlist
\item
  \textbf{Cuándo utilizar}:

  \begin{itemize}
  \tightlist
  \item
    Para capítulos, prólogos, introducciones, apéndices, comentarios u
    otras secciones de un libro.
  \item
    Usar en lugar de ``Book'' cuando se cita una parte específica de un
    libro.
  \end{itemize}
\item
  \textbf{Campos y su propósito}:

  \begin{itemize}
  \tightlist
  \item
    \textbf{Item Type}: Seleccionar ``Book Section''.
  \item
    \textbf{Author}: Autor de la sección.
  \item
    \textbf{Editor/Translator/Contributor}: Igual que en ``Book''.
  \item
    \textbf{Title}: Título de la sección o capítulo.
  \item
    \textbf{Book Title}: Título del libro que contiene la sección.
  \item
    \textbf{Abstract/Series/Series Number/Volume/\# of
    Volumes/Edition/Place/Publisher/Date/\# of Pages/ISBN/Language/Short
    Title/URL/Accessed/Archive/Loc. in Archive/Library Catalog/Call
    Number/Rights/Extra/Date Added/Modified}: Igual que en ``Book''.
  \item
    \textbf{Pages}: Rango de páginas de la sección dentro del libro.
  \end{itemize}
\item
  \textbf{Información oficial de Zotero}:

  \begin{itemize}
  \tightlist
  \item
    Este tipo es específico para citar partes de libros. El campo ``Book
    Title'' es obligatorio para contextualizar la sección, y ``Pages''
    es crucial para ubicar la cita. La documentación sugiere usar
    ``Extra'' para campos como ``Chapter Number''.
  \end{itemize}
\end{itemize}

\subsection{7. Case (Caso legal)}\label{case-caso-legal}

\begin{itemize}
\tightlist
\item
  \textbf{Cuándo utilizar}:

  \begin{itemize}
  \tightlist
  \item
    Para casos legales, ya sean publicados o inéditos.
  \item
    Usar en lugar de ``Document'' para citas legales formales.
  \end{itemize}
\item
  \textbf{Campos y su propósito}:

  \begin{itemize}
  \tightlist
  \item
    \textbf{Item Type}: Seleccionar ``Case''.
  \item
    \textbf{Author}: Abogado o entidad que presenta el caso (rara vez
    usado).
  \item
    \textbf{Counsel}: Abogado que argumenta el caso.
  \item
    \textbf{Case Name}: Título oficial del caso (por ejemplo, ``Roe v.
    Wade'').
  \item
    \textbf{Abstract}: Resumen del caso.
  \item
    \textbf{Court}: Tribunal donde se argumentó el caso.
  \item
    \textbf{Date Decided}: Fecha de la decisión del caso.
  \item
    \textbf{Docket Number}: Número de expediente asignado.
  \item
    \textbf{Reporter}: Publicación donde se reporta el caso.
  \item
    \textbf{Reporter Volume}: Volumen del reporter.
  \item
    \textbf{First Page}: Primera página del caso en el reporter.
  \item
    \textbf{History}: Recursos relacionados con el historial del caso.
  \item
    \textbf{Language/Short Title/URL/Accessed/Rights/Extra/Date
    Added/Modified}: Igual que en ``Artwork''.
  \end{itemize}
\item
  \textbf{Información oficial de Zotero}:

  \begin{itemize}
  \tightlist
  \item
    Este tipo es específico para citas legales. Los campos ``Case Name''
    y ``Court'' son esenciales, y ``History'' permite rastrear el
    progreso del caso. La documentación recomienda consultar la sección
    de ``Legal Citations'' para soporte adicional.
  \end{itemize}
\end{itemize}

\subsection{8. Conference Paper (Artículo en
conferencia)}\label{conference-paper-artuxedculo-en-conferencia}

\begin{itemize}
\tightlist
\item
  \textbf{Cuándo utilizar}:

  \begin{itemize}
  \tightlist
  \item
    Para trabajos presentados en conferencias y publicados en actas
    formales (por ejemplo, como libro o revista).
  \item
    Usar ``Presentation'' para trabajos no publicados en actas.
  \end{itemize}
\item
  \textbf{Campos y su propósito}:

  \begin{itemize}
  \tightlist
  \item
    \textbf{Item Type}: Seleccionar ``Conference Paper''.
  \item
    \textbf{Author}: Autor del artículo.
  \item
    \textbf{Title}: Título del artículo.
  \item
    \textbf{Abstract}: Resumen del contenido.
  \item
    \textbf{Proceedings Title}: Título de las actas donde se publicó.
  \item
    \textbf{Conference Name}: Nombre de la conferencia.
  \item
    \textbf{Place}: Lugar de publicación de las actas.
  \item
    \textbf{Publisher}: Editorial de las actas.
  \item
    \textbf{Date}: Año de publicación.
  \item
    \textbf{Volume}: Volumen de las actas.
  \item
    \textbf{Pages}: Rango de páginas del artículo.
  \item
    \textbf{Series}: Serie de la publicación, si aplica.
  \item
    \textbf{DOI}: Identificador DOI del artículo.
  \item
    \textbf{ISBN}: ISBN de las actas.
  \item
    \textbf{Language/Short Title/URL/Accessed/Archive/Loc. in
    Archive/Library Catalog/Call Number/Rights/Extra/Date
    Added/Modified}: Igual que en ``Artwork''.
  \end{itemize}
\item
  \textbf{Información oficial de Zotero}:

  \begin{itemize}
  \tightlist
  \item
    Este tipo es para artículos publicados en actas, no para
    presentaciones orales. Los campos ``Proceedings Title'' y
    ``Conference Name'' son clave para contextualizar. Si la conferencia
    y la publicación tienen lugares diferentes, usar ``Extra'' para
    ``Event Place''.
  \end{itemize}
\end{itemize}

\subsection{9. Dictionary Entry (Entrada de
diccionario)}\label{dictionary-entry-entrada-de-diccionario}

\begin{itemize}
\tightlist
\item
  \textbf{Cuándo utilizar}:

  \begin{itemize}
  \tightlist
  \item
    Para entradas publicadas en un diccionario.
  \item
    Usar en lugar de ``Encyclopedia Article'' si el recurso es
    específicamente un diccionario.
  \end{itemize}
\item
  \textbf{Campos y su propósito}:

  \begin{itemize}
  \tightlist
  \item
    \textbf{Item Type}: Seleccionar ``Dictionary Entry''.
  \item
    \textbf{Author}: Autor de la entrada, si se especifica.
  \item
    \textbf{Title}: Título de la entrada (por ejemplo, la palabra
    definida).
  \item
    \textbf{Dictionary Title}: Título del diccionario.
  \item
    \textbf{Abstract/Series/Series Number/Volume/\# of
    Volumes/Edition/Place/Publisher/Date/Pages/ISBN/Language/Short
    Title/URL/Accessed/Archive/Loc. in Archive/Library Catalog/Call
    Number/Rights/Extra/Date Added/Modified}: Igual que en ``Book''.
  \end{itemize}
\item
  \textbf{Información oficial de Zotero}:

  \begin{itemize}
  \tightlist
  \item
    Este tipo es específico para diccionarios. El campo ``Dictionary
    Title'' es esencial, y ``Pages'' puede usarse para entradas largas.
    La documentación sugiere usar ``Book Section'' si la entrada es más
    extensa o compleja.
  \end{itemize}
\end{itemize}

\subsection{10. Document (Documento)}\label{document-documento}

\begin{itemize}
\tightlist
\item
  \textbf{Cuándo utilizar}:

  \begin{itemize}
  \tightlist
  \item
    Para documentos genéricos no cubiertos por otros tipos.
  \item
    Evitar su uso si es posible, ya que tiene pocos campos y soporte
    limitado en estilos de cita.
  \end{itemize}
\item
  \textbf{Campos y su propósito}:

  \begin{itemize}
  \tightlist
  \item
    \textbf{Item Type}: Seleccionar ``Document''.
  \item
    \textbf{Author}: Autor del documento.
  \item
    \textbf{Title}: Título del documento.
  \item
    \textbf{Abstract}: Resumen del contenido.
  \item
    \textbf{Publisher}: Entidad que publica el documento.
  \item
    \textbf{Date}: Fecha de publicación.
  \item
    \textbf{Language/Short Title/URL/Accessed/Archive/Loc. in
    Archive/Library Catalog/Call Number/Rights/Extra/Date
    Added/Modified}: Igual que en ``Artwork''.
  \end{itemize}
\item
  \textbf{Información oficial de Zotero}:

  \begin{itemize}
  \tightlist
  \item
    Este tipo es un último recurso debido a su falta de campos
    específicos. La documentación recomienda usar ``Book'', ``Report'' o
    ``Manuscript'' para mejores resultados en citas.
  \end{itemize}
\end{itemize}

\subsection{11. Email (Correo
electrónico)}\label{email-correo-electruxf3nico}

\begin{itemize}
\tightlist
\item
  \textbf{Cuándo utilizar}:

  \begin{itemize}
  \tightlist
  \item
    Para mensajes enviados por correo electrónico u otras formas de
    comunicación personal.
  \item
    Usar en lugar de ``Letter'' si la comunicación es digital.
  \end{itemize}
\item
  \textbf{Campos y su propósito}:

  \begin{itemize}
  \tightlist
  \item
    \textbf{Item Type}: Seleccionar ``Email''.
  \item
    \textbf{Author}: Remitente del correo.
  \item
    \textbf{Recipient}: Destinatario del correo.
  \item
    \textbf{Subject}: Asunto del correo electrónico.
  \item
    \textbf{Abstract}: Resumen del contenido.
  \item
    \textbf{Date}: Fecha de envío.
  \item
    \textbf{Language/Short Title/URL/Accessed/Rights/Extra/Date
    Added/Modified}: Igual que en ``Artwork''.
  \end{itemize}
\item
  \textbf{Información oficial de Zotero}:

  \begin{itemize}
  \tightlist
  \item
    Este tipo es adecuado para comunicaciones personales digitales. El
    campo ``Subject'' actúa como título, y ``Recipient'' es importante
    para el contexto. La documentación sugiere usar ``Letter'' para
    comunicaciones similares no electrónicas.
  \end{itemize}
\end{itemize}

\subsection{12. Encyclopedia Article (Artículo de
enciclopedia)}\label{encyclopedia-article-artuxedculo-de-enciclopedia}

\begin{itemize}
\tightlist
\item
  \textbf{Cuándo utilizar}:

  \begin{itemize}
  \tightlist
  \item
    Para artículos o capítulos publicados en una enciclopedia.
  \item
    Usar en lugar de ``Dictionary Entry'' si el recurso es una
    enciclopedia general.
  \end{itemize}
\item
  \textbf{Campos y su propósito}:

  \begin{itemize}
  \tightlist
  \item
    \textbf{Item Type}: Seleccionar ``Encyclopedia Article''.
  \item
    \textbf{Author}: Autor del artículo.
  \item
    \textbf{Title}: Título del artículo.
  \item
    \textbf{Encyclopedia Title}: Título de la enciclopedia.
  \item
    \textbf{Abstract/Series/Series Number/Volume/\# of
    Volumes/Edition/Place/Publisher/Date/Pages/ISBN/Language/Short
    Title/URL/Accessed/Archive/Loc. in Archive/Library Catalog/Call
    Number/Rights/Extra/Date Added/Modified}: Igual que en ``Book''.
  \end{itemize}
\item
  \textbf{Información oficial de Zotero}:

  \begin{itemize}
  \tightlist
  \item
    Este tipo es para enciclopedias, con ``Encyclopedia Title'' como
    campo clave. Similar a ``Book Section'', pero optimizado para
    enciclopedias. Usar ``Extra'' para detalles adicionales.
  \end{itemize}
\end{itemize}

\subsection{13. Film (Película)}\label{film-peluxedcula}

\begin{itemize}
\tightlist
\item
  \textbf{Cuándo utilizar}:

  \begin{itemize}
  \tightlist
  \item
    Para películas artísticas, incluyendo ficción, no ficción y
    documentales.
  \item
    Usar ``Video Recording'' para videos no artísticos (por ejemplo,
    YouTube).
  \end{itemize}
\item
  \textbf{Campos y su propósito}:

  \begin{itemize}
  \tightlist
  \item
    \textbf{Item Type}: Seleccionar ``Film''.
  \item
    \textbf{Director}: Director de la película (autor principal).
  \item
    \textbf{Producer/Scriptwriter}: Productores o guionistas, si aplica.
  \item
    \textbf{Cast Member}: Actores principales (no usado en citas).
  \item
    \textbf{Title}: Título de la película.
  \item
    \textbf{Abstract}: Resumen de la trama o contenido.
  \item
    \textbf{Date}: Año de estreno.
  \item
    \textbf{Format}: Formato de la película (por ejemplo, ``DVD'',
    ``Streaming'').
  \item
    \textbf{Running Time}: Duración de la película.
  \item
    \textbf{Distributor}: Compañía que distribuye la película.
  \item
    \textbf{Genre}: Género de la película (por ejemplo, ``Drama'').
  \item
    \textbf{Place/Publisher/ISBN/Language/Short
    Title/URL/Accessed/Archive/Loc. in Archive/Library Catalog/Call
    Number/Rights/Extra/Date Added/Modified}: Igual que en ``Artwork''.
  \end{itemize}
\item
  \textbf{Información oficial de Zotero}:

  \begin{itemize}
  \tightlist
  \item
    Este tipo es para películas con enfoque artístico. El campo
    ``Director'' es clave, y roles como ``Producer'' deben especificarse
    en ``Extra''. La documentación recomienda usar etiquetas en
    ``Extra'' para roles adicionales.
  \end{itemize}
\end{itemize}

\subsection{14. Forum Post (Publicación en
foro)}\label{forum-post-publicaciuxf3n-en-foro}

\begin{itemize}
\tightlist
\item
  \textbf{Cuándo utilizar}:

  \begin{itemize}
  \tightlist
  \item
    Para publicaciones en foros en línea, incluidas redes sociales como
    tweets o comentarios de Facebook.
  \item
    Usar en lugar de ``Blog Post'' si la publicación es en un foro o red
    social.
  \end{itemize}
\item
  \textbf{Campos y su propósito}:

  \begin{itemize}
  \tightlist
  \item
    \textbf{Item Type}: Seleccionar ``Forum Post''.
  \item
    \textbf{Author}: Autor de la publicación.
  \item
    \textbf{Commenter}: Persona que comenta (no usado en citas).
  \item
    \textbf{Title}: Título de la publicación o mensaje.
  \item
    \textbf{Abstract}: Resumen del contenido.
  \item
    \textbf{Forum/Listserv Title}: Nombre del foro o plataforma (por
    ejemplo, ``Twitter'').
  \item
    \textbf{Post Type}: Tipo de publicación (por ejemplo, ``Tweet'').
  \item
    \textbf{Date}: Fecha de publicación.
  \item
    \textbf{Language/Short Title/URL/Accessed/Rights/Extra/Date
    Added/Modified}: Igual que en ``Artwork''.
  \end{itemize}
\item
  \textbf{Información oficial de Zotero}:

  \begin{itemize}
  \tightlist
  \item
    Este tipo cubre publicaciones en foros y redes sociales. El campo
    ``Post Type'' permite especificar el formato (por ejemplo,
    ``Tweet''). La documentación sugiere usar ``Webpage'' para
    publicaciones en línea más genéricas.
  \end{itemize}
\end{itemize}

\subsection{15. Hearing (Audiencia)}\label{hearing-audiencia}

\begin{itemize}
\tightlist
\item
  \textbf{Cuándo utilizar}:

  \begin{itemize}
  \tightlist
  \item
    Para informes o registros de audiencias formales de un cuerpo
    legislativo.
  \item
    Usar en lugar de ``Document'' para citas legales específicas.
  \end{itemize}
\item
  \textbf{Campos y su propósito}:

  \begin{itemize}
  \tightlist
  \item
    \textbf{Item Type}: Seleccionar ``Hearing''.
  \item
    \textbf{Contributor}: Persona o entidad asociada (autor principal).
  \item
    \textbf{Title}: Título de la audiencia.
  \item
    \textbf{Abstract}: Resumen del contenido.
  \item
    \textbf{Committee}: Comité que realiza la audiencia.
  \item
    \textbf{Place}: Lugar de la audiencia.
  \item
    \textbf{Publisher}: Entidad que publica el registro.
  \item
    \textbf{Date}: Fecha de la audiencia.
  \item
    \textbf{\# of Volumes}: Número de volúmenes del registro.
  \item
    \textbf{Document Number}: Número de identificación del registro.
  \item
    \textbf{Pages}: Rango de páginas citado.
  \item
    \textbf{Legislative Body}: Cuerpo legislativo (por ejemplo,
    ``Congreso'').
  \item
    \textbf{Session}: Sesión legislativa.
  \item
    \textbf{History}: Historial procesal de la audiencia.
  \item
    \textbf{Language/Short Title/URL/Accessed/Rights/Extra/Date
    Added/Modified}: Igual que en ``Artwork''.
  \end{itemize}
\item
  \textbf{Información oficial de Zotero}:

  \begin{itemize}
  \tightlist
  \item
    Este tipo es para audiencias legislativas formales. Los campos
    ``Committee'' y ``Document Number'' son esenciales. La documentación
    recomienda usar ``Legal Citations'' para más detalles.
  \end{itemize}
\end{itemize}

\subsection{16. Instant Message (Mensaje
instantáneo)}\label{instant-message-mensaje-instantuxe1neo}

\begin{itemize}
\tightlist
\item
  \textbf{Cuándo utilizar}:

  \begin{itemize}
  \tightlist
  \item
    Para mensajes enviados por servicios de mensajería instantánea o
    chat.
  \item
    Usar en lugar de ``Email'' si la comunicación es en tiempo real.
  \end{itemize}
\item
  \textbf{Campos y su propósito}:

  \begin{itemize}
  \tightlist
  \item
    \textbf{Item Type}: Seleccionar ``Instant Message''.
  \item
    \textbf{Author}: Remitente del mensaje.
  \item
    \textbf{Recipient}: Destinatario del mensaje.
  \item
    \textbf{Title}: Título o asunto del mensaje.
  \item
    \textbf{Abstract}: Resumen del contenido.
  \item
    \textbf{Date}: Fecha de envío.
  \item
    \textbf{Language/Short Title/URL/Accessed/Rights/Extra/Date
    Added/Modified}: Igual que en ``Artwork''.
  \end{itemize}
\item
  \textbf{Información oficial de Zotero}:

  \begin{itemize}
  \tightlist
  \item
    Este tipo es para comunicaciones personales en tiempo real. Similar
    a ``Email'', pero enfocado en chats. La documentación sugiere usar
    ``Letter'' para comunicaciones similares no digitales.
  \end{itemize}
\end{itemize}

\subsection{17. Interview (Entrevista)}\label{interview-entrevista}

\begin{itemize}
\tightlist
\item
  \textbf{Cuándo utilizar}:

  \begin{itemize}
  \tightlist
  \item
    Para grabaciones, transcripciones o registros de entrevistas.
  \item
    Usar en lugar de ``Document'' para entrevistas formales o
    publicadas.
  \end{itemize}
\item
  \textbf{Campos y su propósito}:

  \begin{itemize}
  \tightlist
  \item
    \textbf{Item Type}: Seleccionar ``Interview''.
  \item
    \textbf{Interview With}: Persona entrevistada (autor principal).
  \item
    \textbf{Interviewer}: Persona que realiza la entrevista.
  \item
    \textbf{Title}: Título de la entrevista, si tiene uno.
  \item
    \textbf{Abstract}: Resumen del contenido.
  \item
    \textbf{Date}: Fecha de la entrevista.
  \item
    \textbf{Medium}: Formato de la entrevista (por ejemplo, ``Grabación
    de audio'', ``Transcripción'').
  \item
    \textbf{Language/Short Title/URL/Accessed/Archive/Loc. in
    Archive/Library Catalog/Call Number/Rights/Extra/Date
    Added/Modified}: Igual que en ``Artwork''.
  \end{itemize}
\item
  \textbf{Información oficial de Zotero}:

  \begin{itemize}
  \tightlist
  \item
    Este tipo es para entrevistas formales o publicadas. El campo
    ``Interview With'' es clave, y ``Medium'' especifica el formato. La
    documentación recomienda usar ``Extra'' para detalles como ``Event
    Date''.
  \end{itemize}
\end{itemize}

\subsection{18. Journal Article (Artículo de revista
académica)}\label{journal-article-artuxedculo-de-revista-acaduxe9mica}

\begin{itemize}
\tightlist
\item
  \textbf{Cuándo utilizar}:

  \begin{itemize}
  \tightlist
  \item
    Para artículos publicados en revistas académicas, ya sea en formato
    impreso u online.
  \item
    Usar en lugar de ``Magazine Article'' para publicaciones académicas
    revisadas por pares.
  \end{itemize}
\item
  \textbf{Campos y su propósito}:

  \begin{itemize}
  \tightlist
  \item
    \textbf{Item Type}: Seleccionar ``Journal Article''.
  \item
    \textbf{Author}: Autor del artículo.
  \item
    \textbf{Reviewed Author}: Autor de la obra reseñada, si es una
    reseña.
  \item
    \textbf{Title}: Título del artículo.
  \item
    \textbf{Publication}: Nombre de la revista.
  \item
    \textbf{Abstract}: Resumen del artículo.
  \item
    \textbf{Volume}: Volumen de la revista.
  \item
    \textbf{Issue}: Número de la revista.
  \item
    \textbf{Pages}: Rango de páginas del artículo.
  \item
    \textbf{Date}: Fecha de publicación.
  \item
    \textbf{Series/Series Title}: Serie o sección especial de la
    revista.
  \item
    \textbf{Journal Abbr}: Abreviatura de la revista (según estándares
    como MEDLINE).
  \item
    \textbf{DOI}: Identificador DOI del artículo.
  \item
    \textbf{ISSN}: Número ISSN de la revista.
  \item
    \textbf{Language/Short Title/URL/Accessed/Archive/Loc. in
    Archive/Library Catalog/Call Number/Rights/Extra/Date
    Added/Modified}: Igual que en ``Artwork''.
  \end{itemize}
\item
  \textbf{Información oficial de Zotero}:

  \begin{itemize}
  \tightlist
  \item
    Este tipo es para artículos académicos revisados por pares. Los
    campos ``Publication'', ``Volume'' e ``Issue'' son esenciales. La
    documentación recomienda usar ``Extra'' para campos como ``PMID'' o
    ``Status'' (por ejemplo, ``in press'').
  \end{itemize}
\end{itemize}

\subsection{19. Letter (Carta)}\label{letter-carta}

\begin{itemize}
\tightlist
\item
  \textbf{Cuándo utilizar}:

  \begin{itemize}
  \tightlist
  \item
    Para cartas enviadas entre personas u organizaciones, especialmente
    comunicaciones históricas o personales.
  \item
    Usar en lugar de ``Email'' para comunicaciones no digitales.
  \end{itemize}
\item
  \textbf{Campos y su propósito}:

  \begin{itemize}
  \tightlist
  \item
    \textbf{Item Type}: Seleccionar ``Letter''.
  \item
    \textbf{Author}: Remitente de la carta.
  \item
    \textbf{Recipient}: Destinatario de la carta.
  \item
    \textbf{Title}: Título o descripción de la carta.
  \item
    \textbf{Abstract}: Resumen del contenido.
  \item
    \textbf{Type}: Tipo de carta (por ejemplo, ``Correspondencia
    privada'').
  \item
    \textbf{Date}: Fecha de envío.
  \item
    \textbf{Language/Short Title/URL/Accessed/Archive/Loc. in
    Archive/Library Catalog/Call Number/Rights/Extra/Date
    Added/Modified}: Igual que en ``Artwork''.
  \end{itemize}
\item
  \textbf{Información oficial de Zotero}:

  \begin{itemize}
  \tightlist
  \item
    Este tipo es para comunicaciones personales no electrónicas. El
    campo ``Type'' permite especificar el propósito de la carta. La
    documentación sugiere usar ``Manuscript'' para cartas históricas con
    valor archivístico.
  \end{itemize}
\end{itemize}

\subsection{20. Magazine Article (Artículo de
revista)}\label{magazine-article-artuxedculo-de-revista}

\begin{itemize}
\tightlist
\item
  \textbf{Cuándo utilizar}:

  \begin{itemize}
  \tightlist
  \item
    Para artículos publicados en revistas populares, comerciales o no
    académicas.
  \item
    Usar en lugar de ``Journal Article'' si no es una publicación
    revisada por pares.
  \end{itemize}
\item
  \textbf{Campos y su propósito}:

  \begin{itemize}
  \tightlist
  \item
    \textbf{Item Type}: Seleccionar ``Magazine Article''.
  \item
    \textbf{Author}: Autor del artículo.
  \item
    \textbf{Title}: Título del artículo.
  \item
    \textbf{Publication}: Nombre de la revista.
  \item
    \textbf{Abstract}: Resumen del contenido.
  \item
    \textbf{Volume/Issue}: Volumen y número de la revista, si aplica.
  \item
    \textbf{Date}: Fecha de publicación.
  \item
    \textbf{Pages}: Rango de páginas.
  \item
    \textbf{ISSN}: Número ISSN de la revista.
  \item
    \textbf{Language/Short Title/URL/Accessed/Archive/Loc. in
    Archive/Library Catalog/Call Number/Rights/Extra/Date
    Added/Modified}: Igual que en ``Artwork''.
  \end{itemize}
\item
  \textbf{Información oficial de Zotero}:

  \begin{itemize}
  \tightlist
  \item
    Este tipo es para revistas no académicas. Los campos ``Publication''
    y ``Date'' son clave. La documentación recomienda usar ``Journal
    Article'' para publicaciones académicas y ``Newspaper Article'' para
    diarios.
  \end{itemize}
\end{itemize}

\subsection{21. Manuscript (Manuscrito)}\label{manuscript-manuscrito}

\begin{itemize}
\tightlist
\item
  \textbf{Cuándo utilizar}:

  \begin{itemize}
  \tightlist
  \item
    Para manuscritos inéditos, ya sean históricos o modernos (por
    ejemplo, trabajos no publicados, borradores o artículos en proceso).
  \item
    Usar en lugar de ``Document'' para documentos con valor archivístico
    o académico.
  \end{itemize}
\item
  \textbf{Campos y su propósito}:

  \begin{itemize}
  \tightlist
  \item
    \textbf{Item Type}: Seleccionar ``Manuscript''.
  \item
    \textbf{Author}: Autor del manuscrito.
  \item
    \textbf{Title}: Título del manuscrito.
  \item
    \textbf{Abstract}: Resumen del contenido.
  \item
    \textbf{Type}: Tipo o estado del manuscrito (por ejemplo,
    ``Manuscrito inédito'', ``Trabajo en proceso'').
  \item
    \textbf{Date}: Fecha de creación o redacción.
  \item
    \textbf{Place}: Lugar donde se creó el manuscrito.
  \item
    \textbf{\# of Pages}: Número de páginas.
  \item
    \textbf{Language/Short Title/URL/Accessed/Archive/Loc. in
    Archive/Library Catalog/Call Number/Rights/Extra/Date
    Added/Modified}: Igual que en ``Artwork''.
  \end{itemize}
\item
  \textbf{Información oficial de Zotero}:

  \begin{itemize}
  \tightlist
  \item
    Este tipo es versátil para documentos inéditos. El campo ``Type'' es
    crucial para especificar el estado. La documentación recomienda usar
    ``Extra'' para detalles archivísticos adicionales.
  \end{itemize}
\end{itemize}

\subsection{22. Map (Mapa)}\label{map-mapa}

\begin{itemize}
\tightlist
\item
  \textbf{Cuándo utilizar}:

  \begin{itemize}
  \tightlist
  \item
    Para mapas o modelos geográficos.
  \item
    Usar en lugar de ``Artwork'' si el enfoque es cartográfico.
  \end{itemize}
\item
  \textbf{Campos y su propósito}:

  \begin{itemize}
  \tightlist
  \item
    \textbf{Item Type}: Seleccionar ``Map''.
  \item
    \textbf{Cartographer}: Creador del mapa (autor principal).
  \item
    \textbf{Title}: Título del mapa.
  \item
    \textbf{Abstract}: Descripción del contenido.
  \item
    \textbf{Type}: Tipo o género del mapa (por ejemplo, ``Mapa
    topográfico'').
  \item
    \textbf{Scale}: Escala del mapa (por ejemplo, ``1:50,000'').
  \item
    \textbf{Series/Series Title}: Serie cartográfica, si aplica.
  \item
    \textbf{Edition}: Edición del mapa.
  \item
    \textbf{Place/Publisher/Date/ISBN/Language/Short
    Title/URL/Accessed/Archive/Loc. in Archive/Library Catalog/Call
    Number/Rights/Extra/Date Added/Modified}: Igual que en ``Artwork''.
  \end{itemize}
\item
  \textbf{Información oficial de Zotero}:

  \begin{itemize}
  \tightlist
  \item
    Este tipo es para recursos cartográficos. El campo ``Scale'' es
    único y esencial. La documentación sugiere usar ``Type'' para
    géneros específicos y ``Extra'' para detalles adicionales.
  \end{itemize}
\end{itemize}

\subsection{23. Newspaper Article (Artículo de
periódico)}\label{newspaper-article-artuxedculo-de-periuxf3dico}

\begin{itemize}
\tightlist
\item
  \textbf{Cuándo utilizar}:

  \begin{itemize}
  \tightlist
  \item
    Para artículos publicados en periódicos, ya sea en formato impreso u
    online.
  \item
    Usar en lugar de ``Magazine Article'' si el medio es un diario.
  \end{itemize}
\item
  \textbf{Campos y su propósito}:

  \begin{itemize}
  \tightlist
  \item
    \textbf{Item Type}: Seleccionar ``Newspaper Article''.
  \item
    \textbf{Author}: Autor del artículo.
  \item
    \textbf{Title}: Título del artículo.
  \item
    \textbf{Publication}: Nombre del periódico.
  \item
    \textbf{Abstract}: Resumen del contenido.
  \item
    \textbf{Place}: Lugar de publicación del periódico.
  \item
    \textbf{Edition}: Edición del periódico (por ejemplo, ``Edición
    matutina'').
  \item
    \textbf{Date}: Fecha de publicación.
  \item
    \textbf{Section}: Sección del periódico (por ejemplo,
    ``Internacional'').
  \item
    \textbf{Pages}: Rango de páginas.
  \item
    \textbf{ISSN}: Número ISSN del periódico.
  \item
    \textbf{Language/Short Title/URL/Accessed/Archive/Loc. in
    Archive/Library Catalog/Call Number/Rights/Extra/Date
    Added/Modified}: Igual que en ``Artwork''.
  \end{itemize}
\item
  \textbf{Información oficial de Zotero}:

  \begin{itemize}
  \tightlist
  \item
    Este tipo es para artículos de prensa. Los campos ``Publication'' y
    ``Section'' son clave para contextualizar. La documentación
    recomienda usar ``Magazine Article'' para revistas no académicas.
  \end{itemize}
\end{itemize}

\subsection{24. Patent (Patente)}\label{patent-patente}

\begin{itemize}
\tightlist
\item
  \textbf{Cuándo utilizar}:

  \begin{itemize}
  \tightlist
  \item
    Para patentes otorgadas por una invención.
  \item
    Usar en lugar de ``Document'' para citas legales específicas.
  \end{itemize}
\item
  \textbf{Campos y su propósito}:

  \begin{itemize}
  \tightlist
  \item
    \textbf{Item Type}: Seleccionar ``Patent''.
  \item
    \textbf{Inventor}: Creador de la invención (autor principal).
  \item
    \textbf{Attorney/Agent}: Abogado o agente que representa al
    inventor.
  \item
    \textbf{Title}: Título de la patente.
  \item
    \textbf{Abstract}: Resumen de la invención.
  \item
    \textbf{Country}: País que emite la patente.
  \item
    \textbf{Assignee}: Entidad a la que se asignan los derechos.
  \item
    \textbf{Issuing Authority}: Oficina que emite la patente.
  \item
    \textbf{Patent Number}: Número de la patente.
  \item
    \textbf{Filing Date}: Fecha de presentación de la solicitud.
  \item
    \textbf{Issue Date}: Fecha de emisión de la patente.
  \item
    \textbf{Application Number}: Número de la solicitud.
  \item
    \textbf{Priority Numbers}: Números de prioridad internacional.
  \item
    \textbf{References}: Recursos relacionados con la patente.
  \item
    \textbf{Legal Status}: Estado legal de la patente.
  \item
    \textbf{Language/Short Title/URL/Accessed/Rights/Extra/Date
    Added/Modified}: Igual que en ``Artwork''.
  \end{itemize}
\item
  \textbf{Información oficial de Zotero}:

  \begin{itemize}
  \tightlist
  \item
    Este tipo es para patentes formales. Los campos ``Patent Number'' e
    ``Issuing Authority'' son esenciales. La documentación recomienda
    usar ``Extra'' para campos legales adicionales.
  \end{itemize}
\end{itemize}

\subsection{25. Podcast (Podcast)}\label{podcast-podcast}

\begin{itemize}
\tightlist
\item
  \textbf{Cuándo utilizar}:

  \begin{itemize}
  \tightlist
  \item
    Para episodios de programas de audio o video distribuidos en línea,
    generalmente por suscripción.
  \item
    Usar en lugar de ``Audio Recording'' si el contenido es un podcast.
  \end{itemize}
\item
  \textbf{Campos y su propósito}:

  \begin{itemize}
  \tightlist
  \item
    \textbf{Item Type}: Seleccionar ``Podcast''.
  \item
    \textbf{Podcaster}: Anfitrión del podcast (autor principal).
  \item
    \textbf{Guest}: Invitado en el episodio, si aplica.
  \item
    \textbf{Title}: Título del episodio.
  \item
    \textbf{Abstract}: Resumen del contenido.
  \item
    \textbf{Program Title}: Nombre del programa de podcast.
  \item
    \textbf{Episode Number}: Número del episodio.
  \item
    \textbf{Date}: Fecha de publicación.
  \item
    \textbf{Running Time}: Duración del episodio.
  \item
    \textbf{Format}: Formato del podcast (por ejemplo, ``MP3'').
  \item
    \textbf{Language/Short Title/URL/Accessed/Rights/Extra/Date
    Added/Modified}: Igual que en ``Artwork''.
  \end{itemize}
\item
  \textbf{Información oficial de Zotero}:

  \begin{itemize}
  \tightlist
  \item
    Este tipo es para podcasts distribuidos en línea. El campo ``Program
    Title'' contextualiza el episodio, y ``Episode Number'' es útil para
    identificación. La documentación sugiere usar ``Audio Recording''
    para grabaciones no distribuidas como podcasts.
  \end{itemize}
\end{itemize}

\subsection{26. Presentation
(Presentación)}\label{presentation-presentaciuxf3n}

\begin{itemize}
\tightlist
\item
  \textbf{Cuándo utilizar}:

  \begin{itemize}
  \tightlist
  \item
    Para presentaciones en conferencias, simposios o reuniones, no
    publicadas en actas.
  \item
    Usar ``Conference Paper'' si la presentación se publicó en actas.
  \end{itemize}
\item
  \textbf{Campos y su propósito}:

  \begin{itemize}
  \tightlist
  \item
    \textbf{Item Type}: Seleccionar ``Presentation''.
  \item
    \textbf{Presenter}: Persona que realiza la presentación (autor
    principal).
  \item
    \textbf{Title}: Título de la presentación.
  \item
    \textbf{Abstract}: Resumen del contenido.
  \item
    \textbf{Meeting Name}: Nombre del evento o conferencia.
  \item
    \textbf{Place}: Lugar donde se realizó la presentación.
  \item
    \textbf{Date}: Fecha de la presentación.
  \item
    \textbf{Type}: Formato de la presentación (por ejemplo, ``Póster'',
    ``Conferencia'').
  \item
    \textbf{Language/Short Title/URL/Accessed/Rights/Extra/Date
    Added/Modified}: Igual que en ``Artwork''.
  \end{itemize}
\item
  \textbf{Información oficial de Zotero}:

  \begin{itemize}
  \tightlist
  \item
    Este tipo es para presentaciones no publicadas. El campo ``Type''
    permite especificar el formato, y ``Meeting Name'' es clave para el
    contexto. La documentación sugiere usar ``Extra'' para ``Event
    Date'' si es diferente de la fecha de publicación.
  \end{itemize}
\end{itemize}

\subsection{27. Radio Broadcast (Emisión de
radio)}\label{radio-broadcast-emisiuxf3n-de-radio}

\begin{itemize}
\tightlist
\item
  \textbf{Cuándo utilizar}:

  \begin{itemize}
  \tightlist
  \item
    Para programas de radio, como noticias, series de entretenimiento o
    grabaciones archivadas.
  \item
    Usar en lugar de ``Podcast'' si no es un programa distribuido por
    suscripción.
  \end{itemize}
\item
  \textbf{Campos y su propósito}:

  \begin{itemize}
  \tightlist
  \item
    \textbf{Item Type}: Seleccionar ``Radio Broadcast''.
  \item
    \textbf{Director}: Director del programa (autor principal).
  \item
    \textbf{Producer/Scriptwriter/Cast Member/Guest}: Roles adicionales,
    si aplica.
  \item
    \textbf{Title}: Título del episodio o programa.
  \item
    \textbf{Abstract}: Resumen del contenido.
  \item
    \textbf{Program Title}: Nombre del programa de radio.
  \item
    \textbf{Episode Number}: Número del episodio, si aplica.
  \item
    \textbf{Date}: Fecha de emisión.
  \item
    \textbf{Running Time}: Duración del programa.
  \item
    \textbf{Network}: Cadena de radio que emite el programa.
  \item
    \textbf{Format}: Formato de la emisión (por ejemplo, ``Streaming'').
  \item
    \textbf{Language/Short Title/URL/Accessed/Archive/Loc. in
    Archive/Library Catalog/Call Number/Rights/Extra/Date
    Added/Modified}: Igual que en ``Artwork''.
  \end{itemize}
\item
  \textbf{Información oficial de Zotero}:

  \begin{itemize}
  \tightlist
  \item
    Este tipo es para emisiones de radio, incluidas grabaciones
    archivadas. El campo ``Network'' es importante para contextualizar.
    La documentación recomienda usar ``Extra'' para roles como
    ``Producer''.
  \end{itemize}
\end{itemize}

\subsection{28. Report (Informe)}\label{report-informe}

\begin{itemize}
\tightlist
\item
  \textbf{Cuándo utilizar}:

  \begin{itemize}
  \tightlist
  \item
    Para informes publicados por organizaciones, gobiernos o
    instituciones, incluidos preprints y documentos de trabajo.
  \item
    Usar en lugar de ``Book'' para informes técnicos o gubernamentales.
  \end{itemize}
\item
  \textbf{Campos y su propósito}:

  \begin{itemize}
  \tightlist
  \item
    \textbf{Item Type}: Seleccionar ``Report''.
  \item
    \textbf{Author}: Autor del informe.
  \item
    \textbf{Title}: Título del informe.
  \item
    \textbf{Abstract}: Resumen del contenido.
  \item
    \textbf{Report Number}: Número asignado al informe.
  \item
    \textbf{Report Type}: Tipo de informe (por ejemplo, ``Informe
    técnico'').
  \item
    \textbf{Institution}: Organización que publica el informe.
  \item
    \textbf{Place}: Lugar de publicación.
  \item
    \textbf{Date}: Fecha de publicación.
  \item
    \textbf{Pages}: Número de páginas.
  \item
    \textbf{Series/Series Number}: Serie del informe, si aplica.
  \item
    \textbf{DOI}: Identificador DOI, si aplica.
  \item
    \textbf{Language/Short Title/URL/Accessed/Archive/Loc. in
    Archive/Library Catalog/Call Number/Rights/Extra/Date
    Added/Modified}: Igual que en ``Artwork''.
  \end{itemize}
\item
  \textbf{Información oficial de Zotero}:

  \begin{itemize}
  \tightlist
  \item
    Este tipo es versátil para informes institucionales. Los campos
    ``Report Number'' e ``Institution'' son clave. La documentación
    sugiere usar ``Extra'' para campos como ``Status'' (por ejemplo,
    ``preprint'').
  \end{itemize}
\end{itemize}

\subsection{29. Software (Software)}\label{software-software}

\begin{itemize}
\tightlist
\item
  \textbf{Cuándo utilizar}:

  \begin{itemize}
  \tightlist
  \item
    Para programas de computadora, aplicaciones o software.
  \item
    Usar en lugar de ``Document'' para citas de software específicas.
  \end{itemize}
\item
  \textbf{Campos y su propósito}:

  \begin{itemize}
  \tightlist
  \item
    \textbf{Item Type}: Seleccionar ``Software''.
  \item
    \textbf{Programmer}: Creador del software (autor principal).
  \item
    \textbf{Title}: Nombre del software.
  \item
    \textbf{Abstract}: Descripción del software.
  \item
    \textbf{Version}: Versión del software (por ejemplo, ``2.3.1'').
  \item
    \textbf{System}: Sistema operativo o plataforma (por ejemplo,
    ``Windows'').
  \item
    \textbf{Company}: Empresa que publica el software.
  \item
    \textbf{Date}: Fecha de publicación.
  \item
    \textbf{Language}: Lenguaje de programación, si aplica.
  \item
    \textbf{Short Title/URL/Accessed/Rights/Extra/Date Added/Modified}:
    Igual que en ``Artwork''.
  \end{itemize}
\item
  \textbf{Información oficial de Zotero}:

  \begin{itemize}
  \tightlist
  \item
    Este tipo es para software y aplicaciones. El campo ``Version'' es
    crucial para identificación. La documentación recomienda usar
    ``Extra'' para detalles adicionales.
  \end{itemize}
\end{itemize}

\subsection{30. Statute (Ley)}\label{statute-ley}

\begin{itemize}
\tightlist
\item
  \textbf{Cuándo utilizar}:

  \begin{itemize}
  \tightlist
  \item
    Para leyes o legislación aprobada.
  \item
    Usar en lugar de ``Bill'' si la legislación ya está promulgada.
  \end{itemize}
\item
  \textbf{Campos y su propósito}:

  \begin{itemize}
  \tightlist
  \item
    \textbf{Item Type}: Seleccionar ``Statute''.
  \item
    \textbf{Sponsor}: Autor principal de la ley.
  \item
    \textbf{Title}: Título oficial de la ley.
  \item
    \textbf{Abstract}: Resumen del contenido.
  \item
    \textbf{Name of Act}: Nombre completo de la ley.
  \item
    \textbf{Code}: Código donde se publica la ley.
  \item
    \textbf{Code Volume}: Volumen del código.
  \item
    \textbf{Code Number}: Número de la ley en el código.
  \item
    \textbf{Public Law Number}: Número de la ley pública.
  \item
    \textbf{Date Enacted}: Fecha de promulgación.
  \item
    \textbf{Section}: Sección citada.
  \item
    \textbf{Pages}: Páginas en el código.
  \item
    \textbf{Legislative Body}: Cuerpo que aprobó la ley.
  \item
    \textbf{Session}: Sesión legislativa.
  \item
    \textbf{History}: Historial de la ley.
  \item
    \textbf{Language/Short Title/URL/Accessed/Rights/Extra/Date
    Added/Modified}: Igual que en ``Artwork''.
  \end{itemize}
\item
  \textbf{Información oficial de Zotero}:

  \begin{itemize}
  \tightlist
  \item
    Este tipo es para legislación aprobada. Los campos ``Name of Act'' y
    ``Public Law Number'' son esenciales. La documentación recomienda
    usar ``Legal Citations'' para más detalles.
  \end{itemize}
\end{itemize}

\subsection{31. Thesis (Tesis)}\label{thesis-tesis}

\begin{itemize}
\tightlist
\item
  \textbf{Cuándo utilizar}:

  \begin{itemize}
  \tightlist
  \item
    Para tesis de grado, ya sean publicadas o inéditas.
  \item
    Usar en lugar de ``Manuscript'' para trabajos académicos formales.
  \end{itemize}
\item
  \textbf{Campos y su propósito}:

  \begin{itemize}
  \tightlist
  \item
    \textbf{Item Type}: Seleccionar ``Thesis''.
  \item
    \textbf{Author}: Autor de la tesis.
  \item
    \textbf{Title}: Título de la tesis.
  \item
    \textbf{Abstract}: Resumen del contenido.
  \item
    \textbf{Type}: Tipo de tesis (por ejemplo, ``Tesis doctoral'',
    ``Tesis de maestría'').
  \item
    \textbf{University}: Universidad que otorga el grado.
  \item
    \textbf{Place}: Lugar de la universidad.
  \item
    \textbf{Date}: Año de presentación.
  \item
    \textbf{\# of Pages}: Número de páginas.
  \item
    \textbf{Language/Short Title/URL/Accessed/Archive/Loc. in
    Archive/Library Catalog/Call Number/Rights/Extra/Date
    Added/Modified}: Igual que en ``Artwork''.
  \end{itemize}
\item
  \textbf{Información oficial de Zotero}:

  \begin{itemize}
  \tightlist
  \item
    Este tipo es para tesis académicas. El campo ``Type'' debe incluir
    ``tesis'' o ``disertación''. La documentación recomienda usar
    ``Extra'' para campos como ``Submitted Date''.
  \end{itemize}
\end{itemize}

\subsection{32. TV Broadcast (Emisión de
televisión)}\label{tv-broadcast-emisiuxf3n-de-televisiuxf3n}

\begin{itemize}
\tightlist
\item
  \textbf{Cuándo utilizar}:

  \begin{itemize}
  \tightlist
  \item
    Para episodios de series de televisión o programas emitidos.
  \item
    Usar en lugar de ``Video Recording'' para emisiones formales.
  \end{itemize}
\item
  \textbf{Campos y su propósito}:

  \begin{itemize}
  \tightlist
  \item
    \textbf{Item Type}: Seleccionar ``TV Broadcast''.
  \item
    \textbf{Director}: Director del episodio (autor principal).
  \item
    \textbf{Producer/Scriptwriter/Cast Member/Guest}: Roles adicionales.
  \item
    \textbf{Title}: Título del episodio.
  \item
    \textbf{Abstract}: Resumen del contenido.
  \item
    \textbf{Program Title}: Nombre del programa.
  \item
    \textbf{Episode Number}: Número del episodio.
  \item
    \textbf{Date}: Fecha de emisión.
  \item
    \textbf{Running Time}: Duración del episodio.
  \item
    \textbf{Network}: Cadena de televisión.
  \item
    \textbf{Format}: Formato de la emisión (por ejemplo, ``Streaming'').
  \item
    \textbf{Language/Short Title/URL/Accessed/Archive/Loc. in
    Archive/Library Catalog/Call Number/Rights/Extra/Date
    Added/Modified}: Igual que en ``Artwork''.
  \end{itemize}
\item
  \textbf{Información oficial de Zotero}:

  \begin{itemize}
  \tightlist
  \item
    Este tipo es para emisiones de televisión. El campo ``Program
    Title'' es clave para contextualizar. La documentación recomienda
    usar ``Extra'' para roles como ``Producer''.
  \end{itemize}
\end{itemize}

\subsection{33. Video Recording (Grabación de
video)}\label{video-recording-grabaciuxf3n-de-video}

\begin{itemize}
\tightlist
\item
  \textbf{Cuándo utilizar}:

  \begin{itemize}
  \tightlist
  \item
    Para videos que no encajan en ``Film'', ``TV Broadcast'' o
    ``Podcast'' (por ejemplo, videos de YouTube o figuras científicas).
  \item
    Usar en lugar de ``Document'' para contenido audiovisual.
  \end{itemize}
\item
  \textbf{Campos y su propósito}:

  \begin{itemize}
  \tightlist
  \item
    \textbf{Item Type}: Seleccionar ``Video Recording''.
  \item
    \textbf{Director}: Director del video (autor principal).
  \item
    \textbf{Producer/Scriptwriter}: Roles adicionales.
  \item
    \textbf{Title}: Título del video.
  \item
    \textbf{Abstract}: Resumen del contenido.
  \item
    \textbf{Date}: Fecha de publicación.
  \item
    \textbf{Format}: Formato del video (por ejemplo, ``MP4'').
  \item
    \textbf{Running Time}: Duración del video.
  \item
    \textbf{Studio}: Estudio que produce el video.
  \item
    \textbf{Language/Short Title/URL/Accessed/Archive/Loc. in
    Archive/Library Catalog/Call Number/Rights/Extra/Date
    Added/Modified}: Igual que en ``Artwork''.
  \end{itemize}
\item
  \textbf{Información oficial de Zotero}:

  \begin{itemize}
  \tightlist
  \item
    Este tipo es para videos genéricos. El campo ``Format'' es
    importante para especificar el medio. La documentación recomienda
    usar ``Film'' para películas artísticas.
  \end{itemize}
\end{itemize}

\subsection{34. Webpage (Página web)}\label{webpage-puxe1gina-web}

\begin{itemize}
\tightlist
\item
  \textbf{Cuándo utilizar}:

  \begin{itemize}
  \tightlist
  \item
    Para páginas web genéricas que no encajan en tipos más específicos
    como ``Blog Post'' o ``Report''.
  \item
    Usar como última opción si no hay un tipo más adecuado.
  \end{itemize}
\item
  \textbf{Campos y su propósito}:

  \begin{itemize}
  \tightlist
  \item
    \textbf{Item Type}: Seleccionar ``Webpage''.
  \item
    \textbf{Author}: Autor de la página.
  \item
    \textbf{Title}: Título de la página.
  \item
    \textbf{Abstract}: Resumen del contenido.
  \item
    \textbf{Website Title}: Nombre del sitio web.
  \item
    \textbf{Website Type}: Género del sitio (por ejemplo, ``Sitio
    corporativo''), raramente usado.
  \item
    \textbf{Date}: Fecha de publicación o actualización.
  \item
    \textbf{Language/Short Title/URL/Accessed/Rights/Extra/Date
    Added/Modified}: Igual que en ``Artwork''.
  \end{itemize}
\item
  \textbf{Información oficial de Zotero}:

  \begin{itemize}
  \tightlist
  \item
    Este tipo es genérico para contenido en línea. El campo ``Website
    Title'' es clave para contextualizar. La documentación recomienda
    usar tipos más específicos siempre que sea posible.
  \end{itemize}
\end{itemize}

\subsection{35. Attachment (Archivo
adjunto)}\label{attachment-archivo-adjunto}

\begin{itemize}
\tightlist
\item
  \textbf{Cuándo utilizar}:

  \begin{itemize}
  \tightlist
  \item
    Para archivos independientes (por ejemplo, PDF, JPEG, DOCX) no
    asociados a un ítem principal.
  \item
    Evitar su uso, ya que tiene funcionalidad limitada (no se pueden
    citar ni buscar adecuadamente).
  \end{itemize}
\item
  \textbf{Campos y su propósito}:

  \begin{itemize}
  \tightlist
  \item
    \textbf{Item Type}: Seleccionar ``Attachment''.
  \item
    \textbf{Title}: Nombre del archivo.
  \item
    \textbf{Date Added/Modified}: Campos automáticos.
  \end{itemize}
\item
  \textbf{Información oficial de Zotero}:

  \begin{itemize}
  \tightlist
  \item
    Este tipo es para archivos sueltos, pero la documentación
    desaconseja su uso. Se recomienda asociar archivos a ítems completos
    (por ejemplo, ``Journal Article'' con un PDF adjunto).
  \end{itemize}
\end{itemize}

\subsection{36. Note (Nota)}\label{note-nota}

\begin{itemize}
\tightlist
\item
  \textbf{Cuándo utilizar}:

  \begin{itemize}
  \tightlist
  \item
    Para notas independientes usadas para organización o anotaciones.
  \item
    No es ideal para citas, ya que usa los primeros 120 caracteres como
    título.
  \end{itemize}
\item
  \textbf{Campos y su propósito}:

  \begin{itemize}
  \tightlist
  \item
    \textbf{Item Type}: Seleccionar ``Note''.
  \item
    \textbf{Title}: Primeros 120 caracteres de la nota (automático).
  \item
    \textbf{Date Added/Modified}: Campos automáticos.
  \end{itemize}
\item
  \textbf{Información oficial de Zotero}:

  \begin{itemize}
  \tightlist
  \item
    Este tipo es para notas internas. La documentación advierte que
    citar notas no es confiable y sugiere usar ítems completos para
    comentarios en bibliografías.
  \end{itemize}
\end{itemize}

\section{Tipos de Elementos No Soportados
Formalmente}\label{tipos-de-elementos-no-soportados-formalmente}

Zotero no soporta formalmente los siguientes tipos, pero pueden citarse
usando el campo ``Extra'' con el formato \texttt{Type:\ CSL\ Type}.
Ejemplo: \texttt{Type:\ dataset}.

\begin{itemize}
\tightlist
\item
  \textbf{Dataset (Conjunto de datos)}:

  \begin{itemize}
  \tightlist
  \item
    \textbf{Cuándo utilizar}: Para datos crudos o conjuntos de datos.
  \item
    \textbf{Campos sugeridos}: Usar ``Report'' o ``Document'' y agregar
    \texttt{Type:\ dataset} en ``Extra''. Incluir ``Identifier'',
    ``Version'', ``Repository''.
  \item
    \textbf{Información oficial}: La documentación recomienda usar
    ``Extra'' para ``DOI'' o ``Format''.
  \end{itemize}
\item
  \textbf{Figure (Figura)}:

  \begin{itemize}
  \tightlist
  \item
    \textbf{Cuándo utilizar}: Para figuras en trabajos académicos.
  \item
    \textbf{Campos sugeridos}: Usar ``Artwork'' y agregar
    \texttt{Type:\ figure} en ``Extra''. Incluir ``Medium''.
  \item
    \textbf{Información oficial}: Similar a ``Artwork'', pero enfocado
    en figuras científicas.
  \end{itemize}
\item
  \textbf{Musical Score (Partitura musical)}:

  \begin{itemize}
  \tightlist
  \item
    \textbf{Cuándo utilizar}: Para partituras escritas.
  \item
    \textbf{Campos sugeridos}: Usar ``Book'' o ``Manuscript'' y agregar
    \texttt{Type:\ musical\_score} en ``Extra''.
  \item
    \textbf{Información oficial}: La documentación sugiere usar
    ``Extra'' para detalles específicos.
  \end{itemize}
\item
  \textbf{Pamphlet (Folleto)}:

  \begin{itemize}
  \tightlist
  \item
    \textbf{Cuándo utilizar}: Para trabajos publicados informalmente.
  \item
    \textbf{Campos sugeridos}: Usar ``Report'' y agregar
    \texttt{Type:\ pamphlet} en ``Extra''.
  \item
    \textbf{Información oficial}: Menos técnico que ``Report''.
  \end{itemize}
\item
  \textbf{Book Review (Reseña de libro)}:

  \begin{itemize}
  \tightlist
  \item
    \textbf{Cuándo utilizar}: Para reseñas de libros publicadas.
  \item
    \textbf{Campos sugeridos}: Usar ``Journal Article'', ``Magazine
    Article'' o ``Newspaper Article'' con ``Reviewed Author'' y agregar
    \texttt{Type:\ review-book} en ``Extra''.
  \item
    \textbf{Información oficial}: La documentación recomienda
    especificar ``Reviewed Title'' en ``Extra''.
  \end{itemize}
\item
  \textbf{Treaty (Tratado)}:

  \begin{itemize}
  \tightlist
  \item
    \textbf{Cuándo utilizar}: Para tratados legales entre naciones.
  \item
    \textbf{Campos sugeridos}: Usar ``Document'' y agregar
    \texttt{Type:\ treaty} en ``Extra''.
  \item
    \textbf{Información oficial}: La documentación sugiere consultar
    ``Legal Citations''.
  \end{itemize}
\end{itemize}

\section{Campos Adicionales en
``Extra''}\label{campos-adicionales-en-extra}

Para campos no soportados, usar el formato
\texttt{CSL\ Variable:\ Value} en el campo ``Extra''. Ejemplos:

\begin{itemize}
\tightlist
\item
  \texttt{PMID:\ 123456}
\item
  \texttt{Status:\ in\ press}
\item
  \texttt{Director:\ Kubrick\ \textbar{}\textbar{}\ Stanley}
\end{itemize}

Campos comunes:

\begin{itemize}
\tightlist
\item
  \textbf{PMID/PMCID}: Identificadores de PubMed.
\item
  \textbf{Status}: Estado de publicación (por ejemplo, ``forthcoming'').
\item
  \textbf{Original Date}: Fecha original de publicación (formato ISO).
\item
  \textbf{Director/Editorial Director/Illustrator}: Roles de creadores
  adicionales.
\end{itemize}

\section{Consejos Generales}\label{consejos-generales}

\begin{itemize}
\tightlist
\item
  \textbf{Selección del tipo de elemento}: Elegir el tipo más específico
  posible para mejorar la precisión de las citas.
\item
  \textbf{Uso de ``Extra''}: Aprovechar este campo para información no
  cubierta por los campos estándar.
\item
  \textbf{Citas legales}: Consultar la sección ``Legal Citations'' de
  Zotero para soporte adicional.
\item
  \textbf{Archivos}: Siempre asociar archivos adjuntos a ítems completos
  para mejor funcionalidad.
\item
  \textbf{Idioma}: Usar códigos ISO (por ejemplo, ``es-ES'') para
  consistencia en el formato de títulos.
\end{itemize}

\section{Publicaciones Similares}\label{publicaciones-similares}

Si te interesó este artículo, te recomendamos que explores otros blogs y
recursos relacionados que pueden ampliar tus conocimientos. Aquí te dejo
algunas sugerencias:

\begin{enumerate}
\def\labelenumi{\arabic{enumi}.}
\tightlist
\item
  \href{https://achalmaedison.netlify.app/investigacion-metodologia/posts/2023-06-03-ideas-de-investigacion-para-economia/index.pdf}{\faIcon{file-pdf}}
  \href{https://achalmaedison.netlify.app/investigacion-metodologia/posts/2023-06-03-ideas-de-investigacion-para-economia}{Ideas
  De Investigacion Para Economia}
\item
  \href{https://achalmaedison.netlify.app/investigacion-metodologia/posts/2023-06-03-pautas-de-presentacion-del-informe-de-investigacion/index.pdf}{\faIcon{file-pdf}}
  \href{https://achalmaedison.netlify.app/investigacion-metodologia/posts/2023-06-03-pautas-de-presentacion-del-informe-de-investigacion}{Pautas
  De Presentacion Del Informe De Investigacion}
\item
  \href{https://achalmaedison.netlify.app/investigacion-metodologia/posts/2025-01-12-recursos-de-bibliografia-y-documentacion/index.pdf}{\faIcon{file-pdf}}
  \href{https://achalmaedison.netlify.app/investigacion-metodologia/posts/2025-01-12-recursos-de-bibliografia-y-documentacion}{Recursos
  De Bibliografia Y Documentacion}
\item
  \href{https://achalmaedison.netlify.app/investigacion-metodologia/posts/2025-02-09-recursos-para-traducción-y-correccion/index.pdf}{\faIcon{file-pdf}}
  \href{https://achalmaedison.netlify.app/investigacion-metodologia/posts/2025-02-09-recursos-para-traducción-y-correccion}{Recursos
  Para Traducción Y Correccion}
\end{enumerate}

Esperamos que encuentres estas publicaciones igualmente interesantes y
útiles. ¡Disfruta de la lectura!






\end{document}
