\documentclass[
  stu,
  floatsintext,
  longtable,
  a4paper,
  nolmodern,
  notxfonts,
  notimes,
  colorlinks=true,linkcolor=blue,citecolor=blue,urlcolor=blue]{apa7}

\usepackage{amsmath}
\usepackage{amssymb}



\usepackage[bidi=default]{babel}
\babelprovide[main,import]{spanish}
\StartBabelCommands{spanish}{captions} [unicode, fontenc=TU EU1 EU2, charset=utf8] \SetString{\keywordname}{Palabras
Claves}
\EndBabelCommands


% get rid of language-specific shorthands (see #6817):
\let\LanguageShortHands\languageshorthands
\def\languageshorthands#1{}

\RequirePackage{longtable}
\RequirePackage{threeparttablex}

\makeatletter
\renewcommand{\paragraph}{\@startsection{paragraph}{4}{\parindent}%
	{0\baselineskip \@plus 0.2ex \@minus 0.2ex}%
	{-.5em}%
	{\normalfont\normalsize\bfseries\typesectitle}}

\renewcommand{\subparagraph}[1]{\@startsection{subparagraph}{5}{0.5em}%
	{0\baselineskip \@plus 0.2ex \@minus 0.2ex}%
	{-\z@\relax}%
	{\normalfont\normalsize\bfseries\itshape\hspace{\parindent}{#1}\textit{\addperi}}{\relax}}
\makeatother




\usepackage{longtable, booktabs, multirow, multicol, colortbl, hhline, caption, array, float, xpatch}
\usepackage{subcaption}
\renewcommand\thesubfigure{\Alph{subfigure}}
\setcounter{topnumber}{2}
\setcounter{bottomnumber}{2}
\setcounter{totalnumber}{4}
\renewcommand{\topfraction}{0.85}
\renewcommand{\bottomfraction}{0.85}
\renewcommand{\textfraction}{0.15}
\renewcommand{\floatpagefraction}{0.7}

\usepackage{tcolorbox}
\tcbuselibrary{listings,theorems, breakable, skins}
\usepackage{fontawesome5}

\definecolor{quarto-callout-color}{HTML}{909090}
\definecolor{quarto-callout-note-color}{HTML}{0758E5}
\definecolor{quarto-callout-important-color}{HTML}{CC1914}
\definecolor{quarto-callout-warning-color}{HTML}{EB9113}
\definecolor{quarto-callout-tip-color}{HTML}{00A047}
\definecolor{quarto-callout-caution-color}{HTML}{FC5300}
\definecolor{quarto-callout-color-frame}{HTML}{ACACAC}
\definecolor{quarto-callout-note-color-frame}{HTML}{4582EC}
\definecolor{quarto-callout-important-color-frame}{HTML}{D9534F}
\definecolor{quarto-callout-warning-color-frame}{HTML}{F0AD4E}
\definecolor{quarto-callout-tip-color-frame}{HTML}{02B875}
\definecolor{quarto-callout-caution-color-frame}{HTML}{FD7E14}

%\newlength\Oldarrayrulewidth
%\newlength\Oldtabcolsep


\usepackage{hyperref}




\providecommand{\tightlist}{%
  \setlength{\itemsep}{0pt}\setlength{\parskip}{0pt}}
\usepackage{longtable,booktabs,array}
\usepackage{calc} % for calculating minipage widths
% Correct order of tables after \paragraph or \subparagraph
\usepackage{etoolbox}
\makeatletter
\patchcmd\longtable{\par}{\if@noskipsec\mbox{}\fi\par}{}{}
\makeatother
% Allow footnotes in longtable head/foot
\IfFileExists{footnotehyper.sty}{\usepackage{footnotehyper}}{\usepackage{footnote}}
\makesavenoteenv{longtable}

\usepackage{graphicx}
\makeatletter
\newsavebox\pandoc@box
\newcommand*\pandocbounded[1]{% scales image to fit in text height/width
  \sbox\pandoc@box{#1}%
  \Gscale@div\@tempa{\textheight}{\dimexpr\ht\pandoc@box+\dp\pandoc@box\relax}%
  \Gscale@div\@tempb{\linewidth}{\wd\pandoc@box}%
  \ifdim\@tempb\p@<\@tempa\p@\let\@tempa\@tempb\fi% select the smaller of both
  \ifdim\@tempa\p@<\p@\scalebox{\@tempa}{\usebox\pandoc@box}%
  \else\usebox{\pandoc@box}%
  \fi%
}
% Set default figure placement to htbp
\def\fps@figure{htbp}
\makeatother







\usepackage{newtx}

\defaultfontfeatures{Scale=MatchLowercase}
\defaultfontfeatures[\rmfamily]{Ligatures=TeX,Scale=1}





\title{Teoría y política monetaria: Editar}


\shorttitle{Editar}


\usepackage{etoolbox}


\course{Teoría y política monetaria}
\professor{Dr.~Zenón Quispe Misaico}
\duedate{05/11/2025}

\ccoppy{\textcopyright~2025}



\author{Edison Achalma}



\affiliation{
{Escuela Profesional de Economía, Universidad Nacional de San Cristóbal
de Huamanga}}




\leftheader{Achalma}

\date{2025-05-11}


\abstract{Descubre cómo crear tu propio sitio web estático con Blogdown,
una herramienta poderosa que combina R Markdown y Hugo. Aprende a usar
comandos sencillos para personalizar, construir y alojar tu sitio web de
manera fácil y rápida. ¡Comienza tu proyecto web hoy mismo! }

\keywords{keyword1, keyword2}

\authornote{\par{\addORCIDlink{Edison Achalma}{0000-0001-6996-3364}} 
\par{ }
\par{   El autor no tiene conflictos de interés que revelar.    Los
roles de autor se clasificaron utilizando la taxonomía de roles de
colaborador (CRediT; https://credit.niso.org/) de la siguiente
manera:  Edison Achalma:   conceptualización, redacción}
\par{La correspondencia relativa a este artículo debe dirigirse a Edison
Achalma, Email: \href{mailto:elmer.achalma.09@unsch.edu.pe}{elmer.achalma.09@unsch.edu.pe}}
}

\makeatletter
\let\endoldlt\endlongtable
\def\endlongtable{
\hline
\endoldlt
}
\makeatother

\urlstyle{same}



\makeatletter
\@ifpackageloaded{caption}{}{\usepackage{caption}}
\AtBeginDocument{%
\ifdefined\contentsname
  \renewcommand*\contentsname{Tabla de contenidos}
\else
  \newcommand\contentsname{Tabla de contenidos}
\fi
\ifdefined\listfigurename
  \renewcommand*\listfigurename{Listado de Figuras}
\else
  \newcommand\listfigurename{Listado de Figuras}
\fi
\ifdefined\listtablename
  \renewcommand*\listtablename{Listado de Tablas}
\else
  \newcommand\listtablename{Listado de Tablas}
\fi
\ifdefined\figurename
  \renewcommand*\figurename{Figura}
\else
  \newcommand\figurename{Figura}
\fi
\ifdefined\tablename
  \renewcommand*\tablename{Tabla}
\else
  \newcommand\tablename{Tabla}
\fi
}
\@ifpackageloaded{float}{}{\usepackage{float}}
\floatstyle{ruled}
\@ifundefined{c@chapter}{\newfloat{codelisting}{h}{lop}}{\newfloat{codelisting}{h}{lop}[chapter]}
\floatname{codelisting}{Listado}
\newcommand*\listoflistings{\listof{codelisting}{Listado de Listados}}
\makeatother
\makeatletter
\makeatother
\makeatletter
\@ifpackageloaded{caption}{}{\usepackage{caption}}
\@ifpackageloaded{subcaption}{}{\usepackage{subcaption}}
\makeatother
\makeatletter
\@ifpackageloaded{fontawesome5}{}{\usepackage{fontawesome5}}
\makeatother

% From https://tex.stackexchange.com/a/645996/211326
%%% apa7 doesn't want to add appendix section titles in the toc
%%% let's make it do it
\makeatletter
\xpatchcmd{\appendix}
  {\par}
  {\addcontentsline{toc}{section}{\@currentlabelname}\par}
  {}{}
\makeatother

%% Disable longtable counter
%% https://tex.stackexchange.com/a/248395/211326

\usepackage{etoolbox}

\makeatletter
\patchcmd{\LT@caption}
  {\bgroup}
  {\bgroup\global\LTpatch@captiontrue}
  {}{}
\patchcmd{\longtable}
  {\par}
  {\par\global\LTpatch@captionfalse}
  {}{}
\apptocmd{\endlongtable}
  {\ifLTpatch@caption\else\addtocounter{table}{-1}\fi}
  {}{}
\newif\ifLTpatch@caption
\makeatother

\begin{document}

\maketitle

\hypertarget{toc}{}
\tableofcontents
\newpage
\section[Introduction]{Teoría y política monetaria}

\setcounter{secnumdepth}{-\maxdimen} % remove section numbering

\setlength\LTleft{0pt}


Dinero, eje central.

Bitcoin, activo altamente especulativo.

\begin{itemize}
\tightlist
\item
  Transaccional
\item
  Inter temporal
\item
  óptimo
\item
  Dinámico, Ajuste de acuerdo a las necesidades
\item
  Estocástico (aleatorio)
\end{itemize}

Token Vale

\begin{itemize}
\tightlist
\item
  Foward looking, decision hacia delante
\item
  Equilibrio general
\item
  Bienestar
\end{itemize}

Tenemos tres partes el bienestar menos el sacrificio del tiempo en la
actividad - los usos i represntado por funcion de oscio (tiempo
disponible), el resultado debe ser positivo

\[
L = Max E (\sum_{t = 1}^{\infty}  \beta_{t}  (u(C_{t}, \frac{M_{t}}{P_{t}},\epsilon_{t})) -  \int\limits_0^1 v(h_{t}(i),\epsilon_{t})di+) + \lambda (\int_{0}^{1} h_{t}(i) w_{t} (i)\,di + \int_{0}^{1} \pi_{t} (i) + \,di + M_{t-1} + \beta_{t-1}(1+R_{t-1}) )
\]

\[
- \int_{0}^{1} P_{t} (i) C_{t}\,di + M_{t} + \beta_{t} + T_{t}
\]

\(\beta^{t}\) : Factor de actualización. Tiene un rendimiento
\(\frac{1}{1+R_{t}}\) debe ser iguak a renta de capital humano.

\(u\) : Bienestar. Continu en el tiempo y espacio

\(C_{t}\) Canasta de bienes y servicios. estatico (discreto) en el
tiempo y espacio

\(\frac{M_{t}}{P_{t}}\) : Valor real del dinero (líquido y divisible)

\(\epsilon_{t}\) : Evento estocástico que afecta al bienestar

\(h_{t}(i)\) : Horas dedicadas a la actividad i

se agrega sumatoria por que tiene agreacion discreta

\(w_{t}(i)\) : Salario que se paga por hacer i actividades

\(\pi_{t} (i)\) : Renta que se paga por participar en actividad i

\(M_{t}\) : Dinero

\(M_{t-1}\) : Dinero que tengo pero que decidí antes

\(\beta_{t}\) : ahorro

\(\beta_{t-1}(1+R_{t-1})\) : ahorro capitalizado

\(T_{t}\): impuesto

lo que mehos hecho es caracterizar el recurso disponible

Holaaaaaaa

\section{Publicaciones Similares}\label{publicaciones-similares}

Si te interesó este artículo, te recomendamos que explores otros blogs y
recursos relacionados que pueden ampliar tus conocimientos. Aquí te dejo
algunas sugerencias:

\begin{enumerate}
\def\labelenumi{\arabic{enumi}.}
\tightlist
\item
  \href{https://achalmaedison.netlify.app/macroeconomia/posts/2021-07-19-01-conceptos-basicos/index.pdf}{\faIcon{file-pdf}}
  \href{https://achalmaedison.netlify.app/macroeconomia/posts/2021-07-19-01-conceptos-basicos}{01
  Conceptos Basicos}
\item
  \href{https://achalmaedison.netlify.app/macroeconomia/posts/2021-07-26-02-teoria-de-consumo/index.pdf}{\faIcon{file-pdf}}
  \href{https://achalmaedison.netlify.app/macroeconomia/posts/2021-07-26-02-teoria-de-consumo}{02
  Teoria De Consumo}
\item
  \href{https://achalmaedison.netlify.app/macroeconomia/posts/2021-08-02-03-teoria-de-la-inversion/index.pdf}{\faIcon{file-pdf}}
  \href{https://achalmaedison.netlify.app/macroeconomia/posts/2021-08-02-03-teoria-de-la-inversion}{03
  Teoria De La Inversion}
\item
  \href{https://achalmaedison.netlify.app/macroeconomia/posts/2021-08-09-04-tipo-de-cambio/index.pdf}{\faIcon{file-pdf}}
  \href{https://achalmaedison.netlify.app/macroeconomia/posts/2021-08-09-04-tipo-de-cambio}{04
  Tipo De Cambio}
\item
  \href{https://achalmaedison.netlify.app/macroeconomia/posts/2021-12-20-05-modelo-de-mundell-fleming/index.pdf}{\faIcon{file-pdf}}
  \href{https://achalmaedison.netlify.app/macroeconomia/posts/2021-12-20-05-modelo-de-mundell-fleming}{05
  Modelo De Mundell Fleming}
\item
  \href{https://achalmaedison.netlify.app/macroeconomia/posts/2021-12-27-06-sector-externo/index.pdf}{\faIcon{file-pdf}}
  \href{https://achalmaedison.netlify.app/macroeconomia/posts/2021-12-27-06-sector-externo}{06
  Sector Externo}
\item
  \href{https://achalmaedison.netlify.app/macroeconomia/posts/2022-01-03-07-fluctuaciones-de-corto-plazo/index.pdf}{\faIcon{file-pdf}}
  \href{https://achalmaedison.netlify.app/macroeconomia/posts/2022-01-03-07-fluctuaciones-de-corto-plazo}{07
  Fluctuaciones De Corto Plazo}
\item
  \href{https://achalmaedison.netlify.app/macroeconomia/posts/2022-01-10-08-ciclos-economicos-reales-rbc/index.pdf}{\faIcon{file-pdf}}
  \href{https://achalmaedison.netlify.app/macroeconomia/posts/2022-01-10-08-ciclos-economicos-reales-rbc}{08
  Ciclos Economicos Reales Rbc}
\item
  \href{https://achalmaedison.netlify.app/macroeconomia/posts/2022-01-17-09-crecimiento-economico/index.pdf}{\faIcon{file-pdf}}
  \href{https://achalmaedison.netlify.app/macroeconomia/posts/2022-01-17-09-crecimiento-economico}{09
  Crecimiento Economico}
\item
  \href{https://achalmaedison.netlify.app/macroeconomia/posts/2022-01-24-10-economia-monetaria/index.pdf}{\faIcon{file-pdf}}
  \href{https://achalmaedison.netlify.app/macroeconomia/posts/2022-01-24-10-economia-monetaria}{10
  Economia Monetaria}
\item
  \href{https://achalmaedison.netlify.app/macroeconomia/posts/2022-01-31-11-modelos-de-empleo/index.pdf}{\faIcon{file-pdf}}
  \href{https://achalmaedison.netlify.app/macroeconomia/posts/2022-01-31-11-modelos-de-empleo}{11
  Modelos De Empleo}
\item
  \href{https://achalmaedison.netlify.app/macroeconomia/posts/2024-03-31-por-editar/index.pdf}{\faIcon{file-pdf}}
  \href{https://achalmaedison.netlify.app/macroeconomia/posts/2024-03-31-por-editar}{Por
  Editar}
\end{enumerate}

Esperamos que encuentres estas publicaciones igualmente interesantes y
útiles. ¡Disfruta de la lectura!






\end{document}
