\documentclass[
  jou,
  floatsintext,
  longtable,
  a4paper,
  nolmodern,
  notxfonts,
  notimes,
  colorlinks=true,linkcolor=blue,citecolor=blue,urlcolor=blue]{apa7}

\usepackage{amsmath}
\usepackage{amssymb}



\usepackage[bidi=default]{babel}
\babelprovide[main,import]{spanish}
\StartBabelCommands{spanish}{captions} [unicode, fontenc=TU EU1 EU2, charset=utf8] \SetString{\keywordname}{Palabras
Claves}
\EndBabelCommands


% get rid of language-specific shorthands (see #6817):
\let\LanguageShortHands\languageshorthands
\def\languageshorthands#1{}

\RequirePackage{longtable}
\RequirePackage{threeparttablex}

\makeatletter
\renewcommand{\paragraph}{\@startsection{paragraph}{4}{\parindent}%
	{0\baselineskip \@plus 0.2ex \@minus 0.2ex}%
	{-.5em}%
	{\normalfont\normalsize\bfseries\typesectitle}}

\renewcommand{\subparagraph}[1]{\@startsection{subparagraph}{5}{0.5em}%
	{0\baselineskip \@plus 0.2ex \@minus 0.2ex}%
	{-\z@\relax}%
	{\normalfont\normalsize\bfseries\itshape\hspace{\parindent}{#1}\textit{\addperi}}{\relax}}
\makeatother




\usepackage{longtable, booktabs, multirow, multicol, colortbl, hhline, caption, array, float, xpatch}
\setcounter{topnumber}{2}
\setcounter{bottomnumber}{2}
\setcounter{totalnumber}{4}
\renewcommand{\topfraction}{0.85}
\renewcommand{\bottomfraction}{0.85}
\renewcommand{\textfraction}{0.15}
\renewcommand{\floatpagefraction}{0.7}

\usepackage{tcolorbox}
\tcbuselibrary{listings,theorems, breakable, skins}
\usepackage{fontawesome5}

\definecolor{quarto-callout-color}{HTML}{909090}
\definecolor{quarto-callout-note-color}{HTML}{0758E5}
\definecolor{quarto-callout-important-color}{HTML}{CC1914}
\definecolor{quarto-callout-warning-color}{HTML}{EB9113}
\definecolor{quarto-callout-tip-color}{HTML}{00A047}
\definecolor{quarto-callout-caution-color}{HTML}{FC5300}
\definecolor{quarto-callout-color-frame}{HTML}{ACACAC}
\definecolor{quarto-callout-note-color-frame}{HTML}{4582EC}
\definecolor{quarto-callout-important-color-frame}{HTML}{D9534F}
\definecolor{quarto-callout-warning-color-frame}{HTML}{F0AD4E}
\definecolor{quarto-callout-tip-color-frame}{HTML}{02B875}
\definecolor{quarto-callout-caution-color-frame}{HTML}{FD7E14}

%\newlength\Oldarrayrulewidth
%\newlength\Oldtabcolsep


\usepackage{hyperref}




\providecommand{\tightlist}{%
  \setlength{\itemsep}{0pt}\setlength{\parskip}{0pt}}
\usepackage{longtable,booktabs,array}
\usepackage{calc} % for calculating minipage widths
% Correct order of tables after \paragraph or \subparagraph
\usepackage{etoolbox}
\makeatletter
\patchcmd\longtable{\par}{\if@noskipsec\mbox{}\fi\par}{}{}
\makeatother
% Allow footnotes in longtable head/foot
\IfFileExists{footnotehyper.sty}{\usepackage{footnotehyper}}{\usepackage{footnote}}
\makesavenoteenv{longtable}

\usepackage{graphicx}
\makeatletter
\newsavebox\pandoc@box
\newcommand*\pandocbounded[1]{% scales image to fit in text height/width
  \sbox\pandoc@box{#1}%
  \Gscale@div\@tempa{\textheight}{\dimexpr\ht\pandoc@box+\dp\pandoc@box\relax}%
  \Gscale@div\@tempb{\linewidth}{\wd\pandoc@box}%
  \ifdim\@tempb\p@<\@tempa\p@\let\@tempa\@tempb\fi% select the smaller of both
  \ifdim\@tempa\p@<\p@\scalebox{\@tempa}{\usebox\pandoc@box}%
  \else\usebox{\pandoc@box}%
  \fi%
}
% Set default figure placement to htbp
\def\fps@figure{htbp}
\makeatother







\usepackage{newtx}

\defaultfontfeatures{Scale=MatchLowercase}
\defaultfontfeatures[\rmfamily]{Ligatures=TeX,Scale=1}





\title{Medidas de concentracion: Explorando los pilares fundamentales
para comprender el funcionamiento y éxito de la industria moderna}


\shorttitle{Editar}


\usepackage{etoolbox}



\ccoppy{\textcopyright~2025}



\author{Edison Achalma}



\affiliation{
{Escuela Profesional de Economía, Universidad Nacional de San Cristóbal
de Huamanga}}




\leftheader{Achalma}

\date{2023-06-17}


\abstract{Primer parrafo de abstrac }

\keywords{keyword1, keyword2}

\authornote{\par{\addORCIDlink{Edison Achalma}{0000-0001-6996-3364}} 
\par{ }
\par{   Los autores no tienen conflictos de intereses que
revelar.    Los roles de autor se clasificaron utilizando la taxonomía
de roles de colaborador (CRediT; https://credit.niso.org/) de la
siguiente manera:  Edison Achalma:   conceptualización, redacción}
\par{La correspondencia relativa a este artículo debe dirigirse a Edison
Achalma, Email: \href{mailto:elmer.achalma.09@unsch.edu.pe}{elmer.achalma.09@unsch.edu.pe}}
}

\usepackage{pbalance} 
\usepackage{float}
\makeatletter
\let\oldtpt\ThreePartTable
\let\endoldtpt\endThreePartTable
\def\ThreePartTable{\@ifnextchar[\ThreePartTable@i \ThreePartTable@ii}
\def\ThreePartTable@i[#1]{\begin{figure}[!htbp]
\onecolumn
\begin{minipage}{0.5\textwidth}
\oldtpt[#1]
}
\def\ThreePartTable@ii{\begin{figure}[!htbp]
\onecolumn
\begin{minipage}{0.5\textwidth}
\oldtpt
}
\def\endThreePartTable{
\endoldtpt
\end{minipage}
\twocolumn
\end{figure}}
\makeatother


\makeatletter
\let\endoldlt\endlongtable		
\def\endlongtable{
\hline
\endoldlt}
\makeatother

\newenvironment{twocolumntable}% environment name
{% begin code
\begin{table*}[!htbp]%
\onecolumn%
}%
{%
\twocolumn%
\end{table*}%
}% end code

\urlstyle{same}



\makeatletter
\@ifpackageloaded{caption}{}{\usepackage{caption}}
\AtBeginDocument{%
\ifdefined\contentsname
  \renewcommand*\contentsname{Tabla de contenidos}
\else
  \newcommand\contentsname{Tabla de contenidos}
\fi
\ifdefined\listfigurename
  \renewcommand*\listfigurename{Listado de Figuras}
\else
  \newcommand\listfigurename{Listado de Figuras}
\fi
\ifdefined\listtablename
  \renewcommand*\listtablename{Listado de Tablas}
\else
  \newcommand\listtablename{Listado de Tablas}
\fi
\ifdefined\figurename
  \renewcommand*\figurename{Figura}
\else
  \newcommand\figurename{Figura}
\fi
\ifdefined\tablename
  \renewcommand*\tablename{Tabla}
\else
  \newcommand\tablename{Tabla}
\fi
}
\@ifpackageloaded{float}{}{\usepackage{float}}
\floatstyle{ruled}
\@ifundefined{c@chapter}{\newfloat{codelisting}{h}{lop}}{\newfloat{codelisting}{h}{lop}[chapter]}
\floatname{codelisting}{Listado}
\newcommand*\listoflistings{\listof{codelisting}{Listado de Listados}}
\makeatother
\makeatletter
\makeatother
\makeatletter
\@ifpackageloaded{caption}{}{\usepackage{caption}}
\@ifpackageloaded{subcaption}{}{\usepackage{subcaption}}
\makeatother

% From https://tex.stackexchange.com/a/645996/211326
%%% apa7 doesn't want to add appendix section titles in the toc
%%% let's make it do it
\makeatletter
\xpatchcmd{\appendix}
  {\par}
  {\addcontentsline{toc}{section}{\@currentlabelname}\par}
  {}{}
\makeatother

%% Disable longtable counter
%% https://tex.stackexchange.com/a/248395/211326

\usepackage{etoolbox}

\makeatletter
\patchcmd{\LT@caption}
  {\bgroup}
  {\bgroup\global\LTpatch@captiontrue}
  {}{}
\patchcmd{\longtable}
  {\par}
  {\par\global\LTpatch@captionfalse}
  {}{}
\apptocmd{\endlongtable}
  {\ifLTpatch@caption\else\addtocounter{table}{-1}\fi}
  {}{}
\newif\ifLTpatch@caption
\makeatother

\begin{document}

\maketitle

\hypertarget{toc}{}
\tableofcontents
\newpage
\section[Introduction]{Medidas de concentracion}

\setcounter{secnumdepth}{-\maxdimen} % remove section numbering

\setlength\LTleft{0pt}


LAS ESTRUCTURAS DE MERCADO

4.1. COMPETENCIA PERFECTA Y MONOPOLIO

4.2. COMPETENCIA MONOPOLISTICA Y COMPETENCIA PERFECTA

4.3. EL OLIGOPOLIO Y MODELOS DE OLIGOPOLIO

4.3.1. DUOPOLIO Y MONOPLIO DE COURNOT

4.3.2. DUOPOLIO DE STAKELBERG.

4.3.3. DUOPOLIO DE BERTRAAND

4.3.4. LA COMPETENCIA MONOPLISTICA DE CHAMBERLIN.

4.1. COMPETENCIA PERFECTA Y MONOPOLIO

4.1.1. definición de competencia perfecta y características

\begin{enumerate}
\def\labelenumi{\Alph{enumi})}
\tightlist
\item
  Definición
\item
  Característica
\end{enumerate}

\begin{enumerate}
\def\labelenumi{\arabic{enumi}.}
\tightlist
\item
  Atomicidad • Homogeneidad
\item
  Perfecta información
\item
  Perfecta movilidad
\item
  Los actores económicos son precio aceptantes.
\item
  La competencia es transparente, bajo la ley de un solo precio y sin
  publicidad y costos de transporte cero 5.1.1. Monopolio puro:
  Definición y características

  \begin{enumerate}
  \def\labelenumii{\Alph{enumii})}
  \tightlist
  \item
    Definición
  \item
    Características
  \end{enumerate}
\item
  único productor y vendedor en la industria y el mercado
\item
  Producto homogéneo
\item
  Información imperfecta
\item
  Restricciones: legales, tecnológicas y naturales
\item
  Fija e impone precios en el mercado
\item
  Los monopolios crecen mediante colusiones y fusiones. 6.1.1.
  Competencia monopolística y Oligopolio 6.1.1. competencia
  monopolística

  \begin{enumerate}
  \def\labelenumii{\Alph{enumii})}
  \tightlist
  \item
    Definición
  \item
    Características
  \end{enumerate}
\item
  Muchos productores y gran número de compradores
\item
  Productos diferenciados en el mercado
\item
  Libre entrada y salida de empresas
\item
  Información imperfecta
\item
  Fijan precios en el mercado
\item
  Experimentan crecimientos mediante fusiones
\item
  Competencia a través de la publicidad y marketing 7.1.1. Oligopolio y
  modelos

  \begin{enumerate}
  \def\labelenumii{\Alph{enumii})}
  \tightlist
  \item
    Definición
  \item
    Características
  \end{enumerate}
\item
  Reducido N° de empresas
\item
  Productos pueden ser homogéneos, diferencias y sustitutos
\item
  Restricciones a la entrada de nuevas empresas: Legales, tecnológicas y
  4 Las empresas compiten mediante precios y cantidades bien grandes.
\item
  Competencia imperfecta
\item
  Fijan Pecios y
\item
  Competencia a través de la publicidad y marketing. MODELOS DE
  OLIGOPOLIO: 7.5.1. El Duopolio de Cournot. 7.5.2. La Competencia
  Monopolística en el modelo de Cournot. 7.5.3. El Duopolio de
  Stackelberg 7.5.4. El Duopolio de Chamberlin 7.5.5. El Duopolio de
  Bertrand. 7.5.6. El Duopolio de Egdeworth 7.5.7. El Duopolio de Paul
  SWEZZY
\end{enumerate}

\section{Publicaciones Similares}\label{publicaciones-similares}

Si te interesó este artículo, te recomendamos que explores otros blogs y
recursos relacionados que pueden ampliar tus conocimientos. Aquí te dejo
algunas sugerencias:

\begin{enumerate}
\def\labelenumi{\arabic{enumi}.}
\tightlist
\item
  \href{https://achalmaedison.netlify.app/microeconomia/organizacion-industrial/2023-06-12-introducion-a-organizacion-industrial}{Introducion
  A Organizacion Industrial} Lee sin conexión
  \href{https://achalmaedison.netlify.app/microeconomia/organizacion-industrial/2023-06-12-introducion-a-organizacion-industrial/index.pdf}{PDF}
\item
  \href{https://achalmaedison.netlify.app/microeconomia/organizacion-industrial/2023-06-13-empresa-como-organizacion}{Empresa
  Como Organizacion} Lee sin conexión
  \href{https://achalmaedison.netlify.app/microeconomia/organizacion-industrial/2023-06-13-empresa-como-organizacion/index.pdf}{PDF}
\item
  \href{https://achalmaedison.netlify.app/microeconomia/organizacion-industrial/2023-06-13-sistemas-economicos}{Sistemas
  Economicos} Lee sin conexión
  \href{https://achalmaedison.netlify.app/microeconomia/organizacion-industrial/2023-06-13-sistemas-economicos/index.pdf}{PDF}
\item
  \href{https://achalmaedison.netlify.app/microeconomia/organizacion-industrial/2023-06-15-mercado-relevante}{Mercado
  Relevante} Lee sin conexión
  \href{https://achalmaedison.netlify.app/microeconomia/organizacion-industrial/2023-06-15-mercado-relevante/index.pdf}{PDF}
\item
  \href{https://achalmaedison.netlify.app/microeconomia/organizacion-industrial/2023-06-16-medidas-concentracion-desempeño}{Medidas
  Concentracion Desempeño} Lee sin conexión
  \href{https://achalmaedison.netlify.app/microeconomia/organizacion-industrial/2023-06-16-medidas-concentracion-desempeño/index.pdf}{PDF}
\item
  \href{https://achalmaedison.netlify.app/microeconomia/organizacion-industrial/2023-06-17-estructura-del-mercado}{Estructura
  Del Mercado} Lee sin conexión
  \href{https://achalmaedison.netlify.app/microeconomia/organizacion-industrial/2023-06-17-estructura-del-mercado/index.pdf}{PDF}
\item
  \href{https://achalmaedison.netlify.app/microeconomia/organizacion-industrial/2023-06-23-elasticidad}{Elasticidad}
  Lee sin conexión
  \href{https://achalmaedison.netlify.app/microeconomia/organizacion-industrial/2023-06-23-elasticidad/index.pdf}{PDF}
\end{enumerate}

Esperamos que encuentres estas publicaciones igualmente interesantes y
útiles. ¡Disfruta de la lectura!






\end{document}
