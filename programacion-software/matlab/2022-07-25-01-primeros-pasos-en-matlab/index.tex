\documentclass[
  jou,
  floatsintext,
  longtable,
  a4paper,
  nolmodern,
  notxfonts,
  notimes,
  colorlinks=true,linkcolor=blue,citecolor=blue,urlcolor=blue]{apa7}

\usepackage{amsmath}
\usepackage{amssymb}



\usepackage[bidi=default]{babel}
\babelprovide[main,import]{spanish}
\StartBabelCommands{spanish}{captions} [unicode, fontenc=TU EU1 EU2, charset=utf8] \SetString{\keywordname}{Palabras
Claves}
\EndBabelCommands


% get rid of language-specific shorthands (see #6817):
\let\LanguageShortHands\languageshorthands
\def\languageshorthands#1{}

\RequirePackage{longtable}
\RequirePackage{threeparttablex}

\makeatletter
\renewcommand{\paragraph}{\@startsection{paragraph}{4}{\parindent}%
	{0\baselineskip \@plus 0.2ex \@minus 0.2ex}%
	{-.5em}%
	{\normalfont\normalsize\bfseries\typesectitle}}

\renewcommand{\subparagraph}[1]{\@startsection{subparagraph}{5}{0.5em}%
	{0\baselineskip \@plus 0.2ex \@minus 0.2ex}%
	{-\z@\relax}%
	{\normalfont\normalsize\bfseries\itshape\hspace{\parindent}{#1}\textit{\addperi}}{\relax}}
\makeatother




\usepackage{longtable, booktabs, multirow, multicol, colortbl, hhline, caption, array, float, xpatch}
\setcounter{topnumber}{2}
\setcounter{bottomnumber}{2}
\setcounter{totalnumber}{4}
\renewcommand{\topfraction}{0.85}
\renewcommand{\bottomfraction}{0.85}
\renewcommand{\textfraction}{0.15}
\renewcommand{\floatpagefraction}{0.7}

\usepackage{tcolorbox}
\tcbuselibrary{listings,theorems, breakable, skins}
\usepackage{fontawesome5}

\definecolor{quarto-callout-color}{HTML}{909090}
\definecolor{quarto-callout-note-color}{HTML}{0758E5}
\definecolor{quarto-callout-important-color}{HTML}{CC1914}
\definecolor{quarto-callout-warning-color}{HTML}{EB9113}
\definecolor{quarto-callout-tip-color}{HTML}{00A047}
\definecolor{quarto-callout-caution-color}{HTML}{FC5300}
\definecolor{quarto-callout-color-frame}{HTML}{ACACAC}
\definecolor{quarto-callout-note-color-frame}{HTML}{4582EC}
\definecolor{quarto-callout-important-color-frame}{HTML}{D9534F}
\definecolor{quarto-callout-warning-color-frame}{HTML}{F0AD4E}
\definecolor{quarto-callout-tip-color-frame}{HTML}{02B875}
\definecolor{quarto-callout-caution-color-frame}{HTML}{FD7E14}

%\newlength\Oldarrayrulewidth
%\newlength\Oldtabcolsep


\usepackage{hyperref}




\providecommand{\tightlist}{%
  \setlength{\itemsep}{0pt}\setlength{\parskip}{0pt}}
\usepackage{longtable,booktabs,array}
\usepackage{calc} % for calculating minipage widths
% Correct order of tables after \paragraph or \subparagraph
\usepackage{etoolbox}
\makeatletter
\patchcmd\longtable{\par}{\if@noskipsec\mbox{}\fi\par}{}{}
\makeatother
% Allow footnotes in longtable head/foot
\IfFileExists{footnotehyper.sty}{\usepackage{footnotehyper}}{\usepackage{footnote}}
\makesavenoteenv{longtable}

\usepackage{graphicx}
\makeatletter
\newsavebox\pandoc@box
\newcommand*\pandocbounded[1]{% scales image to fit in text height/width
  \sbox\pandoc@box{#1}%
  \Gscale@div\@tempa{\textheight}{\dimexpr\ht\pandoc@box+\dp\pandoc@box\relax}%
  \Gscale@div\@tempb{\linewidth}{\wd\pandoc@box}%
  \ifdim\@tempb\p@<\@tempa\p@\let\@tempa\@tempb\fi% select the smaller of both
  \ifdim\@tempa\p@<\p@\scalebox{\@tempa}{\usebox\pandoc@box}%
  \else\usebox{\pandoc@box}%
  \fi%
}
% Set default figure placement to htbp
\def\fps@figure{htbp}
\makeatother







\usepackage{newtx}

\defaultfontfeatures{Scale=MatchLowercase}
\defaultfontfeatures[\rmfamily]{Ligatures=TeX,Scale=1}





\title{Introducción a MATLAB: Interfaz y Primeros Comandos}


\shorttitle{Intro MATLAB}


\usepackage{etoolbox}



\ccoppy{\textcopyright~2022}



\author{Edison Achalma}



\affiliation{
{Escuela Profesional de Economía, Universidad Nacional de San Cristóbal
de Huamanga}}




\leftheader{Achalma}

\date{2022-07-25}


\abstract{Este abstract será actualizado una vez que se complete el
contenido final del artículo. }

\keywords{keyword1, keyword2}

\authornote{\par{\addORCIDlink{Edison Achalma}{0000-0001-6996-3364}} 
\par{ }
\par{   El autor no tiene conflictos de interés que revelar.    Los
roles de autor se clasificaron utilizando la taxonomía de roles de
colaborador (CRediT; https://credit.niso.org/) de la siguiente
manera:  Edison Achalma:   conceptualización, redacción}
\par{La correspondencia relativa a este artículo debe dirigirse a Edison
Achalma, Email: \href{mailto:elmer.achalma.09@unsch.edu.pe}{elmer.achalma.09@unsch.edu.pe}}
}

\usepackage{pbalance} 
\usepackage{float}
\makeatletter
\let\oldtpt\ThreePartTable
\let\endoldtpt\endThreePartTable
\def\ThreePartTable{\@ifnextchar[\ThreePartTable@i \ThreePartTable@ii}
\def\ThreePartTable@i[#1]{\begin{figure}[!htbp]
\onecolumn
\begin{minipage}{0.5\textwidth}
\oldtpt[#1]
}
\def\ThreePartTable@ii{\begin{figure}[!htbp]
\onecolumn
\begin{minipage}{0.5\textwidth}
\oldtpt
}
\def\endThreePartTable{
\endoldtpt
\end{minipage}
\twocolumn
\end{figure}}
\makeatother


\makeatletter
\let\endoldlt\endlongtable		
\def\endlongtable{
\hline
\endoldlt}
\makeatother

\newenvironment{twocolumntable}% environment name
{% begin code
\begin{table*}[!htbp]%
\onecolumn%
}%
{%
\twocolumn%
\end{table*}%
}% end code

\urlstyle{same}



\makeatletter
\@ifpackageloaded{caption}{}{\usepackage{caption}}
\AtBeginDocument{%
\ifdefined\contentsname
  \renewcommand*\contentsname{Tabla de contenidos}
\else
  \newcommand\contentsname{Tabla de contenidos}
\fi
\ifdefined\listfigurename
  \renewcommand*\listfigurename{Listado de Figuras}
\else
  \newcommand\listfigurename{Listado de Figuras}
\fi
\ifdefined\listtablename
  \renewcommand*\listtablename{Listado de Tablas}
\else
  \newcommand\listtablename{Listado de Tablas}
\fi
\ifdefined\figurename
  \renewcommand*\figurename{Figura}
\else
  \newcommand\figurename{Figura}
\fi
\ifdefined\tablename
  \renewcommand*\tablename{Tabla}
\else
  \newcommand\tablename{Tabla}
\fi
}
\@ifpackageloaded{float}{}{\usepackage{float}}
\floatstyle{ruled}
\@ifundefined{c@chapter}{\newfloat{codelisting}{h}{lop}}{\newfloat{codelisting}{h}{lop}[chapter]}
\floatname{codelisting}{Listado}
\newcommand*\listoflistings{\listof{codelisting}{Listado de Listados}}
\makeatother
\makeatletter
\makeatother
\makeatletter
\@ifpackageloaded{caption}{}{\usepackage{caption}}
\@ifpackageloaded{subcaption}{}{\usepackage{subcaption}}
\makeatother
\makeatletter
\@ifpackageloaded{fontawesome5}{}{\usepackage{fontawesome5}}
\makeatother

% From https://tex.stackexchange.com/a/645996/211326
%%% apa7 doesn't want to add appendix section titles in the toc
%%% let's make it do it
\makeatletter
\xpatchcmd{\appendix}
  {\par}
  {\addcontentsline{toc}{section}{\@currentlabelname}\par}
  {}{}
\makeatother

%% Disable longtable counter
%% https://tex.stackexchange.com/a/248395/211326

\usepackage{etoolbox}

\makeatletter
\patchcmd{\LT@caption}
  {\bgroup}
  {\bgroup\global\LTpatch@captiontrue}
  {}{}
\patchcmd{\longtable}
  {\par}
  {\par\global\LTpatch@captionfalse}
  {}{}
\apptocmd{\endlongtable}
  {\ifLTpatch@caption\else\addtocounter{table}{-1}\fi}
  {}{}
\newif\ifLTpatch@caption
\makeatother

\begin{document}

\maketitle

\hypertarget{toc}{}
\tableofcontents
\newpage
\section[Introduction]{Introducción a MATLAB}

\setcounter{secnumdepth}{-\maxdimen} % remove section numbering

\setlength\LTleft{0pt}


Este artículo está actualmente en proceso de edición, y todas las
secciones serán ampliadas y refinadas en futuras revisiones.

\section{Publicaciones Similares}\label{publicaciones-similares}

Si te interesó este artículo, te recomendamos que explores otros blogs y
recursos relacionados que pueden ampliar tus conocimientos. Aquí te dejo
algunas sugerencias:

\begin{enumerate}
\def\labelenumi{\arabic{enumi}.}
\tightlist
\item
  \href{https://achalmaedison.netlify.app/programacion-software/matlab/2022-07-25-01-primeros-pasos-en-matlab/index.pdf}{\faIcon{file-pdf}}
  \href{https://achalmaedison.netlify.app/programacion-software/matlab/2022-07-25-01-primeros-pasos-en-matlab}{01
  Primeros Pasos En Matlab}
\item
  \href{https://achalmaedison.netlify.app/programacion-software/matlab/2022-08-01-02-elementos-basicos-en-matlab-i/index.pdf}{\faIcon{file-pdf}}
  \href{https://achalmaedison.netlify.app/programacion-software/matlab/2022-08-01-02-elementos-basicos-en-matlab-i}{02
  Elementos Basicos En Matlab I}
\item
  \href{https://achalmaedison.netlify.app/programacion-software/matlab/2022-08-08-03-elementos-basicos-en-matlab-ii/index.pdf}{\faIcon{file-pdf}}
  \href{https://achalmaedison.netlify.app/programacion-software/matlab/2022-08-08-03-elementos-basicos-en-matlab-ii}{03
  Elementos Basicos En Matlab Ii}
\item
  \href{https://achalmaedison.netlify.app/programacion-software/matlab/2022-08-15-04-programacion-en-matlab/index.pdf}{\faIcon{file-pdf}}
  \href{https://achalmaedison.netlify.app/programacion-software/matlab/2022-08-15-04-programacion-en-matlab}{04
  Programacion En Matlab}
\item
  \href{https://achalmaedison.netlify.app/programacion-software/matlab/2022-08-22-05-visualizacion-de-datos-en-matlab/index.pdf}{\faIcon{file-pdf}}
  \href{https://achalmaedison.netlify.app/programacion-software/matlab/2022-08-22-05-visualizacion-de-datos-en-matlab}{05
  Visualizacion De Datos En Matlab}
\item
  \href{https://achalmaedison.netlify.app/programacion-software/matlab/2022-08-29-06-sentencias-de-control/index.pdf}{\faIcon{file-pdf}}
  \href{https://achalmaedison.netlify.app/programacion-software/matlab/2022-08-29-06-sentencias-de-control}{06
  Sentencias De Control}
\item
  \href{https://achalmaedison.netlify.app/programacion-software/matlab/2022-09-05-07-metodos-numericos/index.pdf}{\faIcon{file-pdf}}
  \href{https://achalmaedison.netlify.app/programacion-software/matlab/2022-09-05-07-metodos-numericos}{07
  Metodos Numericos}
\item
  \href{https://achalmaedison.netlify.app/programacion-software/matlab/2022-09-12-08-tipos-de-datos-heterogeneos/index.pdf}{\faIcon{file-pdf}}
  \href{https://achalmaedison.netlify.app/programacion-software/matlab/2022-09-12-08-tipos-de-datos-heterogeneos}{08
  Tipos De Datos Heterogeneos}
\item
  \href{https://achalmaedison.netlify.app/programacion-software/matlab/2022-09-19-09-funciones-avanzadas/index.pdf}{\faIcon{file-pdf}}
  \href{https://achalmaedison.netlify.app/programacion-software/matlab/2022-09-19-09-funciones-avanzadas}{09
  Funciones Avanzadas}
\item
  \href{https://achalmaedison.netlify.app/programacion-software/matlab/2022-09-26-10-toolbox-optimization/index.pdf}{\faIcon{file-pdf}}
  \href{https://achalmaedison.netlify.app/programacion-software/matlab/2022-09-26-10-toolbox-optimization}{10
  Toolbox Optimization}
\item
  \href{https://achalmaedison.netlify.app/programacion-software/matlab/2022-10-03-11-aplicaciones/index.pdf}{\faIcon{file-pdf}}
  \href{https://achalmaedison.netlify.app/programacion-software/matlab/2022-10-03-11-aplicaciones}{11
  Aplicaciones}
\end{enumerate}

Esperamos que encuentres estas publicaciones igualmente interesantes y
útiles. ¡Disfruta de la lectura!






\end{document}
