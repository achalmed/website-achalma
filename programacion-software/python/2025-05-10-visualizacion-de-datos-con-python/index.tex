\documentclass[
  jou,
  floatsintext,
  longtable,
  a4paper,
  nolmodern,
  notxfonts,
  notimes,
  colorlinks=true,linkcolor=blue,citecolor=blue,urlcolor=blue]{apa7}

\usepackage{amsmath}
\usepackage{amssymb}



\usepackage[bidi=default]{babel}
\babelprovide[main,import]{spanish}
\StartBabelCommands{spanish}{captions} [unicode, fontenc=TU EU1 EU2, charset=utf8] \SetString{\keywordname}{Palabras
Claves}
\EndBabelCommands


% get rid of language-specific shorthands (see #6817):
\let\LanguageShortHands\languageshorthands
\def\languageshorthands#1{}

\RequirePackage{longtable}
\RequirePackage{threeparttablex}

\makeatletter
\renewcommand{\paragraph}{\@startsection{paragraph}{4}{\parindent}%
	{0\baselineskip \@plus 0.2ex \@minus 0.2ex}%
	{-.5em}%
	{\normalfont\normalsize\bfseries\typesectitle}}

\renewcommand{\subparagraph}[1]{\@startsection{subparagraph}{5}{0.5em}%
	{0\baselineskip \@plus 0.2ex \@minus 0.2ex}%
	{-\z@\relax}%
	{\normalfont\normalsize\bfseries\itshape\hspace{\parindent}{#1}\textit{\addperi}}{\relax}}
\makeatother




\usepackage{longtable, booktabs, multirow, multicol, colortbl, hhline, caption, array, float, xpatch}
\usepackage{subcaption}
\renewcommand\thesubfigure{\Alph{subfigure}}
\setcounter{topnumber}{2}
\setcounter{bottomnumber}{2}
\setcounter{totalnumber}{4}
\renewcommand{\topfraction}{0.85}
\renewcommand{\bottomfraction}{0.85}
\renewcommand{\textfraction}{0.15}
\renewcommand{\floatpagefraction}{0.7}

\usepackage{tcolorbox}
\tcbuselibrary{listings,theorems, breakable, skins}
\usepackage{fontawesome5}

\definecolor{quarto-callout-color}{HTML}{909090}
\definecolor{quarto-callout-note-color}{HTML}{0758E5}
\definecolor{quarto-callout-important-color}{HTML}{CC1914}
\definecolor{quarto-callout-warning-color}{HTML}{EB9113}
\definecolor{quarto-callout-tip-color}{HTML}{00A047}
\definecolor{quarto-callout-caution-color}{HTML}{FC5300}
\definecolor{quarto-callout-color-frame}{HTML}{ACACAC}
\definecolor{quarto-callout-note-color-frame}{HTML}{4582EC}
\definecolor{quarto-callout-important-color-frame}{HTML}{D9534F}
\definecolor{quarto-callout-warning-color-frame}{HTML}{F0AD4E}
\definecolor{quarto-callout-tip-color-frame}{HTML}{02B875}
\definecolor{quarto-callout-caution-color-frame}{HTML}{FD7E14}

%\newlength\Oldarrayrulewidth
%\newlength\Oldtabcolsep


\usepackage{hyperref}



\usepackage{color}
\usepackage{fancyvrb}
\newcommand{\VerbBar}{|}
\newcommand{\VERB}{\Verb[commandchars=\\\{\}]}
\DefineVerbatimEnvironment{Highlighting}{Verbatim}{commandchars=\\\{\}}
% Add ',fontsize=\small' for more characters per line
\usepackage{framed}
\definecolor{shadecolor}{RGB}{241,243,245}
\newenvironment{Shaded}{\begin{snugshade}}{\end{snugshade}}
\newcommand{\AlertTok}[1]{\textcolor[rgb]{0.68,0.00,0.00}{#1}}
\newcommand{\AnnotationTok}[1]{\textcolor[rgb]{0.37,0.37,0.37}{#1}}
\newcommand{\AttributeTok}[1]{\textcolor[rgb]{0.40,0.45,0.13}{#1}}
\newcommand{\BaseNTok}[1]{\textcolor[rgb]{0.68,0.00,0.00}{#1}}
\newcommand{\BuiltInTok}[1]{\textcolor[rgb]{0.00,0.23,0.31}{#1}}
\newcommand{\CharTok}[1]{\textcolor[rgb]{0.13,0.47,0.30}{#1}}
\newcommand{\CommentTok}[1]{\textcolor[rgb]{0.37,0.37,0.37}{#1}}
\newcommand{\CommentVarTok}[1]{\textcolor[rgb]{0.37,0.37,0.37}{\textit{#1}}}
\newcommand{\ConstantTok}[1]{\textcolor[rgb]{0.56,0.35,0.01}{#1}}
\newcommand{\ControlFlowTok}[1]{\textcolor[rgb]{0.00,0.23,0.31}{\textbf{#1}}}
\newcommand{\DataTypeTok}[1]{\textcolor[rgb]{0.68,0.00,0.00}{#1}}
\newcommand{\DecValTok}[1]{\textcolor[rgb]{0.68,0.00,0.00}{#1}}
\newcommand{\DocumentationTok}[1]{\textcolor[rgb]{0.37,0.37,0.37}{\textit{#1}}}
\newcommand{\ErrorTok}[1]{\textcolor[rgb]{0.68,0.00,0.00}{#1}}
\newcommand{\ExtensionTok}[1]{\textcolor[rgb]{0.00,0.23,0.31}{#1}}
\newcommand{\FloatTok}[1]{\textcolor[rgb]{0.68,0.00,0.00}{#1}}
\newcommand{\FunctionTok}[1]{\textcolor[rgb]{0.28,0.35,0.67}{#1}}
\newcommand{\ImportTok}[1]{\textcolor[rgb]{0.00,0.46,0.62}{#1}}
\newcommand{\InformationTok}[1]{\textcolor[rgb]{0.37,0.37,0.37}{#1}}
\newcommand{\KeywordTok}[1]{\textcolor[rgb]{0.00,0.23,0.31}{\textbf{#1}}}
\newcommand{\NormalTok}[1]{\textcolor[rgb]{0.00,0.23,0.31}{#1}}
\newcommand{\OperatorTok}[1]{\textcolor[rgb]{0.37,0.37,0.37}{#1}}
\newcommand{\OtherTok}[1]{\textcolor[rgb]{0.00,0.23,0.31}{#1}}
\newcommand{\PreprocessorTok}[1]{\textcolor[rgb]{0.68,0.00,0.00}{#1}}
\newcommand{\RegionMarkerTok}[1]{\textcolor[rgb]{0.00,0.23,0.31}{#1}}
\newcommand{\SpecialCharTok}[1]{\textcolor[rgb]{0.37,0.37,0.37}{#1}}
\newcommand{\SpecialStringTok}[1]{\textcolor[rgb]{0.13,0.47,0.30}{#1}}
\newcommand{\StringTok}[1]{\textcolor[rgb]{0.13,0.47,0.30}{#1}}
\newcommand{\VariableTok}[1]{\textcolor[rgb]{0.07,0.07,0.07}{#1}}
\newcommand{\VerbatimStringTok}[1]{\textcolor[rgb]{0.13,0.47,0.30}{#1}}
\newcommand{\WarningTok}[1]{\textcolor[rgb]{0.37,0.37,0.37}{\textit{#1}}}

\providecommand{\tightlist}{%
  \setlength{\itemsep}{0pt}\setlength{\parskip}{0pt}}
\usepackage{longtable,booktabs,array}
\usepackage{calc} % for calculating minipage widths
% Correct order of tables after \paragraph or \subparagraph
\usepackage{etoolbox}
\makeatletter
\patchcmd\longtable{\par}{\if@noskipsec\mbox{}\fi\par}{}{}
\makeatother
% Allow footnotes in longtable head/foot
\IfFileExists{footnotehyper.sty}{\usepackage{footnotehyper}}{\usepackage{footnote}}
\makesavenoteenv{longtable}

\usepackage{graphicx}
\makeatletter
\newsavebox\pandoc@box
\newcommand*\pandocbounded[1]{% scales image to fit in text height/width
  \sbox\pandoc@box{#1}%
  \Gscale@div\@tempa{\textheight}{\dimexpr\ht\pandoc@box+\dp\pandoc@box\relax}%
  \Gscale@div\@tempb{\linewidth}{\wd\pandoc@box}%
  \ifdim\@tempb\p@<\@tempa\p@\let\@tempa\@tempb\fi% select the smaller of both
  \ifdim\@tempa\p@<\p@\scalebox{\@tempa}{\usebox\pandoc@box}%
  \else\usebox{\pandoc@box}%
  \fi%
}
% Set default figure placement to htbp
\def\fps@figure{htbp}
\makeatother







\usepackage{newtx}

\defaultfontfeatures{Scale=MatchLowercase}
\defaultfontfeatures[\rmfamily]{Ligatures=TeX,Scale=1}





\title{Visualización de Datos con Python: Técnicas y Ejemplos de
Gráficos para Análisis de Datos}


\shorttitle{Visualización con Python}


\usepackage{etoolbox}



\ccoppy{\textcopyright~2025}



\author{Elmer Achalma}



\affiliation{
{Economía, Universidad Nacional de San Cristóbal de Huamanga}}




\leftheader{Achalma}

\date{2025-05-10}


\abstract{Este documento presenta una colección de ejemplos prácticos de
visualización de datos utilizando Python y la biblioteca Matplotlib. Se
incluyen gráficos de líneas, barras, histogramas, circulares, de caja y
combinados, cada uno acompañado de código comentado y optimizado. El
objetivo es proporcionar una guía educativa para estudiantes y
profesionales interesados en representar datos de manera clara y
efectiva, con énfasis en buenas prácticas de diseño y presentación. }

\keywords{Visualización de datos, Python, Matplotlib, Gráficos
estadísticos, Análisis de datos}

\authornote{\par{\addORCIDlink{Elmer Achalma}{0000-0001-6996-3364}} 

\par{   Los autores no tienen conflictos de intereses que
revelar.  Agradezco a mis maestros por su orientación en el aprendizaje
de la programación y visualización de datos, a mis padres por su apoyo
constante durante mi formación, y a Dios por la salud y fortaleza para
perseguir mis metas. Me comprometo a seguir mejorando mis habilidades en
la universidad y en todos los ámbitos, siempre con respeto.  Los roles
de autor se clasificaron utilizando la taxonomía de roles de colaborador
(CRediT; https://credit.niso.org/) de la siguiente manera:  Elmer
Achalma:   Investigación, Programación, Redacción}
\par{La correspondencia relativa a este artículo debe dirigirse a Elmer
Achalma, Email: \href{mailto:elmer.achalma.09@unsch.edu.pe}{elmer.achalma.09@unsch.edu.pe}}
}

\usepackage{pbalance} 
\usepackage{float}
\makeatletter
\let\oldtpt\ThreePartTable
\let\endoldtpt\endThreePartTable
\def\ThreePartTable{\@ifnextchar[\ThreePartTable@i \ThreePartTable@ii}
\def\ThreePartTable@i[#1]{\begin{figure}[!htbp]
\onecolumn
\begin{minipage}{0.5\textwidth}
\oldtpt[#1]
}
\def\ThreePartTable@ii{\begin{figure}[!htbp]
\onecolumn
\begin{minipage}{0.5\textwidth}
\oldtpt
}
\def\endThreePartTable{
\endoldtpt
\end{minipage}
\twocolumn
\end{figure}}
\makeatother


\makeatletter
\let\endoldlt\endlongtable		
\def\endlongtable{
\hline
\endoldlt}
\makeatother

\newenvironment{twocolumntable}% environment name
{% begin code
\begin{table*}[!htbp]%
\onecolumn%
}%
{%
\twocolumn%
\end{table*}%
}% end code

\urlstyle{same}



\makeatletter
\@ifpackageloaded{caption}{}{\usepackage{caption}}
\AtBeginDocument{%
\ifdefined\contentsname
  \renewcommand*\contentsname{Tabla de contenidos}
\else
  \newcommand\contentsname{Tabla de contenidos}
\fi
\ifdefined\listfigurename
  \renewcommand*\listfigurename{Listado de Figuras}
\else
  \newcommand\listfigurename{Listado de Figuras}
\fi
\ifdefined\listtablename
  \renewcommand*\listtablename{Listado de Tablas}
\else
  \newcommand\listtablename{Listado de Tablas}
\fi
\ifdefined\figurename
  \renewcommand*\figurename{Figura}
\else
  \newcommand\figurename{Figura}
\fi
\ifdefined\tablename
  \renewcommand*\tablename{Tabla}
\else
  \newcommand\tablename{Tabla}
\fi
}
\@ifpackageloaded{float}{}{\usepackage{float}}
\floatstyle{ruled}
\@ifundefined{c@chapter}{\newfloat{codelisting}{h}{lop}}{\newfloat{codelisting}{h}{lop}[chapter]}
\floatname{codelisting}{Listado}
\newcommand*\listoflistings{\listof{codelisting}{Listado de Listados}}
\makeatother
\makeatletter
\makeatother
\makeatletter
\@ifpackageloaded{caption}{}{\usepackage{caption}}
\@ifpackageloaded{subcaption}{}{\usepackage{subcaption}}
\makeatother

% From https://tex.stackexchange.com/a/645996/211326
%%% apa7 doesn't want to add appendix section titles in the toc
%%% let's make it do it
\makeatletter
\xpatchcmd{\appendix}
  {\par}
  {\addcontentsline{toc}{section}{\@currentlabelname}\par}
  {}{}
\makeatother

%% Disable longtable counter
%% https://tex.stackexchange.com/a/248395/211326

\usepackage{etoolbox}

\makeatletter
\patchcmd{\LT@caption}
  {\bgroup}
  {\bgroup\global\LTpatch@captiontrue}
  {}{}
\patchcmd{\longtable}
  {\par}
  {\par\global\LTpatch@captionfalse}
  {}{}
\apptocmd{\endlongtable}
  {\ifLTpatch@caption\else\addtocounter{table}{-1}\fi}
  {}{}
\newif\ifLTpatch@caption
\makeatother

\begin{document}

\maketitle

\hypertarget{toc}{}
\tableofcontents
\newpage
\section[Introduction]{Visualización de Datos con Python}

\setcounter{secnumdepth}{-\maxdimen} % remove section numbering

\setlength\LTleft{0pt}


\section{Grafico de lineas}\label{grafico-de-lineas}

\begin{Shaded}
\begin{Highlighting}[]
\ImportTok{import}\NormalTok{ numpy }\ImportTok{as}\NormalTok{ np}
\ImportTok{import}\NormalTok{ matplotlib.pyplot }\ImportTok{as}\NormalTok{ plt}

\CommentTok{\# Configurar el tamaño de la figura para mejor visualización}
\NormalTok{plt.figure(figsize}\OperatorTok{=}\NormalTok{(}\DecValTok{10}\NormalTok{, }\DecValTok{8}\NormalTok{))}

\CommentTok{\# Datos de consumo de carne bovina (kg por habitante)}
\NormalTok{consumo\_bovino }\OperatorTok{=}\NormalTok{ [}\FloatTok{22.1}\NormalTok{, }\FloatTok{22.1}\NormalTok{, }\FloatTok{23.1}\NormalTok{, }\FloatTok{23.9}\NormalTok{, }\FloatTok{24.6}\NormalTok{, }\FloatTok{21.7}\NormalTok{, }\FloatTok{23.5}\NormalTok{, }\FloatTok{22.0}\NormalTok{, }\FloatTok{22.5}\NormalTok{, }\FloatTok{23.6}\NormalTok{, }\FloatTok{21.7}\NormalTok{]}
\NormalTok{anios\_bovino }\OperatorTok{=}\NormalTok{ [}\DecValTok{2001}\NormalTok{, }\DecValTok{2002}\NormalTok{, }\DecValTok{2003}\NormalTok{, }\DecValTok{2004}\NormalTok{, }\DecValTok{2005}\NormalTok{, }\DecValTok{2006}\NormalTok{, }\DecValTok{2007}\NormalTok{, }\DecValTok{2008}\NormalTok{, }\DecValTok{2009}\NormalTok{, }\DecValTok{2010}\NormalTok{, }\DecValTok{2011}\NormalTok{]}

\CommentTok{\# Datos de consumo de carne porcina (kg por habitante)}
\NormalTok{consumo\_porcino }\OperatorTok{=}\NormalTok{ [}\FloatTok{17.9}\NormalTok{, }\FloatTok{19.4}\NormalTok{, }\FloatTok{19.1}\NormalTok{, }\FloatTok{18.3}\NormalTok{, }\FloatTok{19.3}\NormalTok{, }\FloatTok{22.5}\NormalTok{, }\FloatTok{23.5}\NormalTok{, }\FloatTok{25.0}\NormalTok{, }\FloatTok{24.0}\NormalTok{, }\FloatTok{24.4}\NormalTok{, }\FloatTok{25.6}\NormalTok{]}
\NormalTok{anios\_porcino }\OperatorTok{=}\NormalTok{ [}\DecValTok{2001}\NormalTok{, }\DecValTok{2002}\NormalTok{, }\DecValTok{2003}\NormalTok{, }\DecValTok{2004}\NormalTok{, }\DecValTok{2005}\NormalTok{, }\DecValTok{2006}\NormalTok{, }\DecValTok{2007}\NormalTok{, }\DecValTok{2008}\NormalTok{, }\DecValTok{2009}\NormalTok{, }\DecValTok{2010}\NormalTok{, }\DecValTok{2011}\NormalTok{]}

\CommentTok{\# Graficar consumo de carne bovina con marcadores circulares y línea discontinua}
\NormalTok{plt.plot(anios\_bovino, consumo\_bovino, marker}\OperatorTok{=}\StringTok{\textquotesingle{}o\textquotesingle{}}\NormalTok{, linestyle}\OperatorTok{=}\StringTok{\textquotesingle{}{-}{-}\textquotesingle{}}\NormalTok{, color}\OperatorTok{=}\StringTok{\textquotesingle{}red\textquotesingle{}}\NormalTok{, label}\OperatorTok{=}\StringTok{\textquotesingle{}Carne Bovina\textquotesingle{}}\NormalTok{)}

\CommentTok{\# Graficar consumo de carne porcina con marcadores de diamante y línea discontinua}
\NormalTok{plt.plot(anios\_porcino, consumo\_porcino, marker}\OperatorTok{=}\StringTok{\textquotesingle{}d\textquotesingle{}}\NormalTok{, linestyle}\OperatorTok{=}\StringTok{\textquotesingle{}{-}{-}\textquotesingle{}}\NormalTok{, color}\OperatorTok{=}\StringTok{\textquotesingle{}blue\textquotesingle{}}\NormalTok{, label}\OperatorTok{=}\StringTok{\textquotesingle{}Carne Porcina\textquotesingle{}}\NormalTok{)}

\CommentTok{\# Etiquetas de los ejes y título con formato adecuado}
\NormalTok{plt.xlabel(}\StringTok{\textquotesingle{}Año\textquotesingle{}}\NormalTok{)}
\NormalTok{plt.ylabel(}\StringTok{\textquotesingle{}Consumo (kg por habitante)\textquotesingle{}}\NormalTok{)}
\NormalTok{plt.title(}\StringTok{\textquotesingle{}Consumo Anual de Carne en Chile (2001{-}2011)\textquotesingle{}}\NormalTok{)}

\CommentTok{\# Añadir leyenda en la esquina inferior derecha}
\NormalTok{plt.legend(loc}\OperatorTok{=}\StringTok{\textquotesingle{}lower right\textquotesingle{}}\NormalTok{)}

\CommentTok{\# Configurar marcas en el eje x para mostrar cada año}
\NormalTok{plt.xticks(anios\_bovino)}

\CommentTok{\# Añadir una cuadrícula para mejorar la legibilidad}
\NormalTok{plt.grid(}\VariableTok{True}\NormalTok{, linestyle}\OperatorTok{=}\StringTok{\textquotesingle{}{-}{-}\textquotesingle{}}\NormalTok{, alpha}\OperatorTok{=}\FloatTok{0.7}\NormalTok{)}

\CommentTok{\# Ajustar el diseño para evitar recortes de etiquetas}
\NormalTok{plt.tight\_layout()}

\CommentTok{\# Guardar la figura en un archivo (opcional, se puede descomentar para usar)}
\CommentTok{\# plt.savefig(\textquotesingle{}consumo\_carne\_chile.png\textquotesingle{})}

\CommentTok{\# Mostrar el gráfico}
\NormalTok{plt.show()}
\end{Highlighting}
\end{Shaded}

\begin{Shaded}
\begin{Highlighting}[]
\ImportTok{import}\NormalTok{ matplotlib.pyplot }\ImportTok{as}\NormalTok{ plt}
\ImportTok{import}\NormalTok{ numpy }\ImportTok{as}\NormalTok{ np}

\NormalTok{x }\OperatorTok{=}\NormalTok{ np.linspace(}\DecValTok{0}\NormalTok{, }\DecValTok{2} \OperatorTok{*}\NormalTok{ np.pi, }\DecValTok{200}\NormalTok{)}
\NormalTok{y }\OperatorTok{=}\NormalTok{ np.sin(x)}

\NormalTok{fig, ax }\OperatorTok{=}\NormalTok{ plt.subplots()}
\NormalTok{ax.plot(x, y)}
\NormalTok{plt.show()}
\end{Highlighting}
\end{Shaded}

hol

\begin{figure}

\caption{\label{fig-polar}}

\centering{

\begin{Shaded}
\begin{Highlighting}[]
\ImportTok{import}\NormalTok{ numpy }\ImportTok{as}\NormalTok{ np}
\ImportTok{import}\NormalTok{ matplotlib.pyplot }\ImportTok{as}\NormalTok{ plt}

\NormalTok{r }\OperatorTok{=}\NormalTok{ np.arange(}\DecValTok{0}\NormalTok{, }\DecValTok{2}\NormalTok{, }\FloatTok{0.01}\NormalTok{)}
\NormalTok{theta }\OperatorTok{=} \DecValTok{2} \OperatorTok{*}\NormalTok{ np.pi }\OperatorTok{*}\NormalTok{ r}
\NormalTok{fig, ax }\OperatorTok{=}\NormalTok{ plt.subplots(}
\NormalTok{  subplot\_kw }\OperatorTok{=}\NormalTok{ \{}\StringTok{\textquotesingle{}projection\textquotesingle{}}\NormalTok{: }\StringTok{\textquotesingle{}polar\textquotesingle{}}\NormalTok{\} }
\NormalTok{)}
\NormalTok{ax.plot(theta, r)}
\NormalTok{ax.set\_rticks([}\FloatTok{0.5}\NormalTok{, }\DecValTok{1}\NormalTok{, }\FloatTok{1.5}\NormalTok{, }\DecValTok{2}\NormalTok{])}
\NormalTok{ax.grid(}\VariableTok{True}\NormalTok{)}
\NormalTok{plt.show()}
\end{Highlighting}
\end{Shaded}

}

\end{figure}%

\section{Gráfico de barras}\label{gruxe1fico-de-barras}

\subsection{horizontal}\label{horizontal}

\begin{Shaded}
\begin{Highlighting}[]
\ImportTok{import}\NormalTok{ matplotlib.pyplot }\ImportTok{as}\NormalTok{ plt}

\CommentTok{\# Configurar el tamaño de la figura para una mejor visualización}
\NormalTok{plt.figure(figsize}\OperatorTok{=}\NormalTok{(}\DecValTok{10}\NormalTok{, }\DecValTok{8}\NormalTok{))}

\CommentTok{\# Tipos de legumbres y sus respectivos consumos en kg por habitante en 2001}
\NormalTok{tipos\_legumbres }\OperatorTok{=}\NormalTok{ [}\StringTok{"Poroto"}\NormalTok{, }\StringTok{"Lenteja"}\NormalTok{, }\StringTok{"Garbanzo"}\NormalTok{, }\StringTok{"Arveja"}\NormalTok{]}
\NormalTok{consumo\_legumbres }\OperatorTok{=}\NormalTok{ [}\FloatTok{2.1}\NormalTok{, }\FloatTok{1.0}\NormalTok{, }\FloatTok{0.3}\NormalTok{, }\FloatTok{0.5}\NormalTok{]}

\CommentTok{\# Crear un gráfico de barras con color personalizado y borde}
\NormalTok{plt.bar(tipos\_legumbres, consumo\_legumbres, color}\OperatorTok{=}\StringTok{\textquotesingle{}green\textquotesingle{}}\NormalTok{, edgecolor}\OperatorTok{=}\StringTok{\textquotesingle{}black\textquotesingle{}}\NormalTok{, alpha}\OperatorTok{=}\FloatTok{0.7}\NormalTok{)}

\CommentTok{\# Etiquetas de los ejes y título con formato adecuado}
\NormalTok{plt.xlabel(}\StringTok{\textquotesingle{}Tipos de Legumbres\textquotesingle{}}\NormalTok{)}
\NormalTok{plt.ylabel(}\StringTok{\textquotesingle{}Consumo (kg por habitante)\textquotesingle{}}\NormalTok{)}
\NormalTok{plt.title(}\StringTok{\textquotesingle{}Consumo de Legumbres en Chile (2001)\textquotesingle{}}\NormalTok{)}

\CommentTok{\# Añadir una cuadrícula en el eje y para facilitar la lectura}
\NormalTok{plt.grid(}\VariableTok{True}\NormalTok{, axis}\OperatorTok{=}\StringTok{\textquotesingle{}y\textquotesingle{}}\NormalTok{, linestyle}\OperatorTok{=}\StringTok{\textquotesingle{}{-}{-}\textquotesingle{}}\NormalTok{, alpha}\OperatorTok{=}\FloatTok{0.7}\NormalTok{)}

\CommentTok{\# Ajustar el diseño para evitar recortes de etiquetas}
\NormalTok{plt.tight\_layout()}

\CommentTok{\# Guardar la figura en un archivo (opcional, descomentar para usar)}
\CommentTok{\# plt.savefig(\textquotesingle{}consumo\_legumbres\_2001.png\textquotesingle{})}

\CommentTok{\# Mostrar el gráfico}
\NormalTok{plt.show()}
\end{Highlighting}
\end{Shaded}

\subsection{vertical}\label{vertical}

\begin{Shaded}
\begin{Highlighting}[]
\NormalTok{plt.figure(figsize}\OperatorTok{=}\NormalTok{[}\DecValTok{10}\NormalTok{,}\DecValTok{8}\NormalTok{])}
\CommentTok{\#Consumo de legumbres en el 2001}
\NormalTok{legumbres}\OperatorTok{=}\NormalTok{[}\StringTok{"Poroto"}\NormalTok{,}\StringTok{"Lenteja"}\NormalTok{,}\StringTok{"Garbanzo"}\NormalTok{,}\StringTok{"Arveja"}\NormalTok{]}
\NormalTok{consumo}\OperatorTok{=}\NormalTok{[}\FloatTok{2.1}\NormalTok{, }\FloatTok{1.0}\NormalTok{, }\FloatTok{0.3}\NormalTok{, }\FloatTok{0.5}\NormalTok{]}

\NormalTok{plt.barh(legumbres,consumo)}

\NormalTok{plt.ylabel(}\StringTok{"Tipos de legumbre"}\NormalTok{)}
\NormalTok{plt.xlabel(}\StringTok{"Consumo (kg/hab)"}\NormalTok{)}
\NormalTok{plt.title(}\StringTok{"Consumo de Legumbres en el 2001"}\NormalTok{)}
\NormalTok{plt.show()}
\end{Highlighting}
\end{Shaded}

\section{Histograma}\label{histograma}

\begin{Shaded}
\begin{Highlighting}[]
\ImportTok{import}\NormalTok{ matplotlib.pyplot }\ImportTok{as}\NormalTok{ plt}

\CommentTok{\# Configurar el tamaño de la figura para una mejor visualización}
\NormalTok{plt.figure(figsize}\OperatorTok{=}\NormalTok{(}\DecValTok{10}\NormalTok{, }\DecValTok{8}\NormalTok{))}

\CommentTok{\# Datos de niveles de glucosa (mg/dl)}
\NormalTok{niveles\_glucosa }\OperatorTok{=}\NormalTok{ [}\DecValTok{52}\NormalTok{, }\DecValTok{54}\NormalTok{, }\DecValTok{55}\NormalTok{, }\DecValTok{57}\NormalTok{, }\DecValTok{56}\NormalTok{, }\DecValTok{57}\NormalTok{, }\DecValTok{54}\NormalTok{, }\DecValTok{59}\NormalTok{, }\DecValTok{60}\NormalTok{, }\DecValTok{57}\NormalTok{, }\DecValTok{52}\NormalTok{, }\DecValTok{62}\NormalTok{, }\DecValTok{64}\NormalTok{, }\DecValTok{68}\NormalTok{, }\DecValTok{64}\NormalTok{, }\DecValTok{72}\NormalTok{, }\DecValTok{77}\NormalTok{, }\DecValTok{80}\NormalTok{, }
                   \DecValTok{76}\NormalTok{, }\DecValTok{79}\NormalTok{, }\DecValTok{81}\NormalTok{, }\DecValTok{85}\NormalTok{, }\DecValTok{88}\NormalTok{, }\DecValTok{84}\NormalTok{, }\DecValTok{89}\NormalTok{, }\DecValTok{92}\NormalTok{, }\DecValTok{85}\NormalTok{, }\DecValTok{92}\NormalTok{, }\DecValTok{94}\NormalTok{, }\DecValTok{93}\NormalTok{, }\DecValTok{92}\NormalTok{, }\DecValTok{99}\NormalTok{, }\DecValTok{100}\NormalTok{, }\DecValTok{105}\NormalTok{, }\DecValTok{106}\NormalTok{, }\DecValTok{107}\NormalTok{, }\DecValTok{109}\NormalTok{]}

\CommentTok{\# Bordes de los intervalos para el histograma (bins)}
\NormalTok{intervalos }\OperatorTok{=}\NormalTok{ [}\DecValTok{50}\NormalTok{, }\DecValTok{60}\NormalTok{, }\DecValTok{70}\NormalTok{, }\DecValTok{80}\NormalTok{, }\DecValTok{90}\NormalTok{, }\DecValTok{100}\NormalTok{, }\DecValTok{110}\NormalTok{, }\DecValTok{120}\NormalTok{, }\DecValTok{130}\NormalTok{]}

\CommentTok{\# Crear el histograma con color personalizado y bordes}
\NormalTok{plt.hist(niveles\_glucosa, bins}\OperatorTok{=}\NormalTok{intervalos, color}\OperatorTok{=}\StringTok{\textquotesingle{}skyblue\textquotesingle{}}\NormalTok{, edgecolor}\OperatorTok{=}\StringTok{\textquotesingle{}black\textquotesingle{}}\NormalTok{, alpha}\OperatorTok{=}\FloatTok{0.7}\NormalTok{)}

\CommentTok{\# Etiquetas de los ejes y título con formato adecuado}
\NormalTok{plt.xlabel(}\StringTok{\textquotesingle{}Nivel de Glucosa (mg/dl)\textquotesingle{}}\NormalTok{)}
\NormalTok{plt.ylabel(}\StringTok{\textquotesingle{}Número de Pacientes\textquotesingle{}}\NormalTok{)}
\NormalTok{plt.title(}\StringTok{\textquotesingle{}Distribución de Niveles de Glucosa en Pacientes\textquotesingle{}}\NormalTok{)}

\CommentTok{\# Añadir una cuadrícula en el eje y para facilitar la lectura}
\NormalTok{plt.grid(}\VariableTok{True}\NormalTok{, axis}\OperatorTok{=}\StringTok{\textquotesingle{}y\textquotesingle{}}\NormalTok{, linestyle}\OperatorTok{=}\StringTok{\textquotesingle{}{-}{-}\textquotesingle{}}\NormalTok{, alpha}\OperatorTok{=}\FloatTok{0.7}\NormalTok{)}

\CommentTok{\# Ajustar el diseño para evitar recortes de etiquetas}
\NormalTok{plt.tight\_layout()}

\CommentTok{\# Guardar la figura en un archivo (opcional, descomentar para usar)}
\CommentTok{\# plt.savefig(\textquotesingle{}distribucion\_glucosa.png\textquotesingle{})}

\CommentTok{\# Mostrar el gráfico}
\NormalTok{plt.show()}
\end{Highlighting}
\end{Shaded}

\section{Grafico circular}\label{grafico-circular}

\begin{Shaded}
\begin{Highlighting}[]
\ImportTok{import}\NormalTok{ matplotlib.pyplot }\ImportTok{as}\NormalTok{ plt}

\CommentTok{\# Configurar el tamaño de la figura para una mejor visualización}
\NormalTok{plt.figure(figsize}\OperatorTok{=}\NormalTok{(}\DecValTok{10}\NormalTok{, }\DecValTok{8}\NormalTok{))}

\CommentTok{\# Datos de marcas de autos y sus ventas (en alguna unidad, ej. miles de unidades)}
\NormalTok{marcas\_autos }\OperatorTok{=}\NormalTok{ [}\StringTok{"Kia"}\NormalTok{, }\StringTok{"Toyota"}\NormalTok{, }\StringTok{"Nissan"}\NormalTok{, }\StringTok{"Suzuki"}\NormalTok{, }\StringTok{"Audi"}\NormalTok{]}
\NormalTok{ventas }\OperatorTok{=}\NormalTok{ [}\FloatTok{10.5}\NormalTok{, }\FloatTok{15.3}\NormalTok{, }\FloatTok{14.2}\NormalTok{, }\FloatTok{16.1}\NormalTok{, }\FloatTok{9.8}\NormalTok{]}
\CommentTok{\# Resaltar la primera marca (Kia) ligeramente}
\NormalTok{resaltar }\OperatorTok{=}\NormalTok{ [}\FloatTok{0.1}\NormalTok{, }\DecValTok{0}\NormalTok{, }\DecValTok{0}\NormalTok{, }\DecValTok{0}\NormalTok{, }\DecValTok{0}\NormalTok{]}

\CommentTok{\# Aplicar un estilo visual predefinido (ggplot)}
\NormalTok{plt.style.use(}\StringTok{"ggplot"}\NormalTok{)}

\CommentTok{\# Crear el gráfico de pastel}
\NormalTok{plt.pie(x}\OperatorTok{=}\NormalTok{ventas, explode}\OperatorTok{=}\NormalTok{resaltar, labels}\OperatorTok{=}\NormalTok{marcas\_autos, autopct}\OperatorTok{=}\StringTok{"}\SpecialCharTok{\%.2f\%\%}\StringTok{"}\NormalTok{, shadow}\OperatorTok{=}\VariableTok{True}\NormalTok{, startangle}\OperatorTok{=}\DecValTok{20}\NormalTok{)}

\CommentTok{\# Asegurar que el gráfico sea circular}
\NormalTok{plt.axis}\OperatorTok{=}\NormalTok{(}\StringTok{"equal"}\NormalTok{)}

\CommentTok{\# Añadir un título descriptivo}
\NormalTok{plt.title(}\StringTok{"Distribución de Ventas de Autos en EE.UU."}\NormalTok{)}

\CommentTok{\# Añadir una leyenda en la esquina superior izquierda}
\NormalTok{plt.legend(marcas\_autos, loc}\OperatorTok{=}\StringTok{"upper left"}\NormalTok{)}

\CommentTok{\# Ajustar el diseño para evitar recortes}
\NormalTok{plt.tight\_layout()}

\CommentTok{\# Guardar la figura en un archivo (opcional, descomentar para usar)}
\CommentTok{\# plt.savefig(\textquotesingle{}ventas\_autos\_eeuu.png\textquotesingle{})}

\CommentTok{\# Mostrar el gráfico}
\NormalTok{plt.show()}
\end{Highlighting}
\end{Shaded}

\section{Grafico de Donut}\label{grafico-de-donut}

\begin{Shaded}
\begin{Highlighting}[]
\ImportTok{import}\NormalTok{ matplotlib.pyplot }\ImportTok{as}\NormalTok{ plt}

\CommentTok{\# Configurar el tamaño de la figura para una mejor visualización}
\NormalTok{plt.figure(figsize}\OperatorTok{=}\NormalTok{(}\DecValTok{10}\NormalTok{, }\DecValTok{8}\NormalTok{))}

\CommentTok{\# Datos de marcas de autos y sus ventas (en alguna unidad, ej. miles de unidades)}
\NormalTok{marcas\_autos }\OperatorTok{=}\NormalTok{ [}\StringTok{"Kia"}\NormalTok{, }\StringTok{"Toyota"}\NormalTok{, }\StringTok{"Nissan"}\NormalTok{, }\StringTok{"Suzuki"}\NormalTok{, }\StringTok{"Audi"}\NormalTok{]}
\NormalTok{ventas }\OperatorTok{=}\NormalTok{ [}\FloatTok{10.5}\NormalTok{, }\FloatTok{15.3}\NormalTok{, }\FloatTok{14.2}\NormalTok{, }\FloatTok{16.1}\NormalTok{, }\FloatTok{9.8}\NormalTok{]}
\CommentTok{\# Resaltar la primera marca (Kia) ligeramente}
\NormalTok{resaltar }\OperatorTok{=}\NormalTok{ [}\FloatTok{0.1}\NormalTok{, }\DecValTok{0}\NormalTok{, }\DecValTok{0}\NormalTok{, }\DecValTok{0}\NormalTok{, }\DecValTok{0}\NormalTok{]}

\CommentTok{\# Aplicar un estilo visual predefinido (ggplot)}
\NormalTok{plt.style.use(}\StringTok{"ggplot"}\NormalTok{)}

\CommentTok{\# Crear el gráfico de pastel (donut chart)}
\NormalTok{plt.pie(ventas, explode}\OperatorTok{=}\NormalTok{resaltar, labels}\OperatorTok{=}\NormalTok{marcas\_autos, autopct}\OperatorTok{=}\StringTok{"}\SpecialCharTok{\%.2f\%\%}\StringTok{"}\NormalTok{, shadow}\OperatorTok{=}\VariableTok{True}\NormalTok{, startangle}\OperatorTok{=}\DecValTok{20}\NormalTok{)}

\CommentTok{\# Asegurar que el gráfico sea circular}
\NormalTok{plt.axis}\OperatorTok{=}\NormalTok{(}\StringTok{"equal"}\NormalTok{)}

\CommentTok{\# Añadir un título descriptivo}
\NormalTok{plt.title(}\StringTok{"Distribución de Ventas de Autos en EE.UU."}\NormalTok{)}

\CommentTok{\# Añadir una leyenda en la esquina superior izquierda}
\NormalTok{plt.legend(marcas\_autos, loc}\OperatorTok{=}\StringTok{"upper left"}\NormalTok{)}

\CommentTok{\# Añadir un círculo central para crear el efecto de "donut chart"}
\NormalTok{circulo\_central }\OperatorTok{=}\NormalTok{ plt.Circle(xy}\OperatorTok{=}\NormalTok{(}\DecValTok{0}\NormalTok{, }\DecValTok{0}\NormalTok{), radius}\OperatorTok{=}\FloatTok{0.75}\NormalTok{, facecolor}\OperatorTok{=}\StringTok{"white"}\NormalTok{)}
\NormalTok{plt.gca().add\_artist(circulo\_central)}

\CommentTok{\# Ajustar el diseño para evitar recortes}
\NormalTok{plt.tight\_layout()}

\CommentTok{\# Guardar la figura en un archivo (opcional, descomentar para usar)}
\CommentTok{\# plt.savefig(\textquotesingle{}ventas\_autos\_eeuu\_donut.png\textquotesingle{})}

\CommentTok{\# Mostrar el gráfico}
\NormalTok{plt.show()}
\end{Highlighting}
\end{Shaded}

\section{Grafico de cajas}\label{grafico-de-cajas}

\begin{Shaded}
\begin{Highlighting}[]
\ImportTok{import}\NormalTok{ matplotlib.pyplot }\ImportTok{as}\NormalTok{ plt}

\CommentTok{\# Configurar el tamaño de la figura para una mejor visualización}
\NormalTok{plt.figure(figsize}\OperatorTok{=}\NormalTok{(}\DecValTok{10}\NormalTok{, }\DecValTok{8}\NormalTok{))}

\CommentTok{\# Datos de las edades de los alumnos}
\NormalTok{edades\_alumnos }\OperatorTok{=}\NormalTok{ [}\DecValTok{12}\NormalTok{, }\DecValTok{13}\NormalTok{, }\DecValTok{12}\NormalTok{, }\DecValTok{17}\NormalTok{, }\DecValTok{16}\NormalTok{, }\DecValTok{15}\NormalTok{, }\DecValTok{14}\NormalTok{, }\DecValTok{15}\NormalTok{, }\DecValTok{15}\NormalTok{, }\DecValTok{16}\NormalTok{, }\DecValTok{14}\NormalTok{, }\DecValTok{12}\NormalTok{, }\DecValTok{15}\NormalTok{, }\DecValTok{16}\NormalTok{, }\DecValTok{14}\NormalTok{, }\DecValTok{12}\NormalTok{, }\DecValTok{16}\NormalTok{, }\DecValTok{17}\NormalTok{]}

\CommentTok{\# Crear el diagrama de caja con un estilo personalizado}
\NormalTok{plt.boxplot(edades\_alumnos, vert}\OperatorTok{=}\VariableTok{True}\NormalTok{, patch\_artist}\OperatorTok{=}\VariableTok{True}\NormalTok{, }
\NormalTok{            boxprops}\OperatorTok{=}\BuiltInTok{dict}\NormalTok{(facecolor}\OperatorTok{=}\StringTok{\textquotesingle{}lightblue\textquotesingle{}}\NormalTok{, edgecolor}\OperatorTok{=}\StringTok{\textquotesingle{}black\textquotesingle{}}\NormalTok{),}
\NormalTok{            medianprops}\OperatorTok{=}\BuiltInTok{dict}\NormalTok{(color}\OperatorTok{=}\StringTok{\textquotesingle{}red\textquotesingle{}}\NormalTok{), whiskerprops}\OperatorTok{=}\BuiltInTok{dict}\NormalTok{(color}\OperatorTok{=}\StringTok{\textquotesingle{}black\textquotesingle{}}\NormalTok{),}
\NormalTok{            capprops}\OperatorTok{=}\BuiltInTok{dict}\NormalTok{(color}\OperatorTok{=}\StringTok{\textquotesingle{}black\textquotesingle{}}\NormalTok{), flierprops}\OperatorTok{=}\BuiltInTok{dict}\NormalTok{(marker}\OperatorTok{=}\StringTok{\textquotesingle{}o\textquotesingle{}}\NormalTok{, color}\OperatorTok{=}\StringTok{\textquotesingle{}orange\textquotesingle{}}\NormalTok{, markersize}\OperatorTok{=}\DecValTok{8}\NormalTok{))}

\CommentTok{\# Etiquetas de los ejes y título con formato adecuado}
\NormalTok{plt.ylabel(}\StringTok{\textquotesingle{}Edad (años)\textquotesingle{}}\NormalTok{)}
\NormalTok{plt.title(}\StringTok{\textquotesingle{}Distribución de Edades de Alumnos de Secundaria\textquotesingle{}}\NormalTok{)}

\CommentTok{\# Añadir una cuadrícula en el eje y para facilitar la lectura}
\NormalTok{plt.grid(}\VariableTok{True}\NormalTok{, axis}\OperatorTok{=}\StringTok{\textquotesingle{}y\textquotesingle{}}\NormalTok{, linestyle}\OperatorTok{=}\StringTok{\textquotesingle{}{-}{-}\textquotesingle{}}\NormalTok{, alpha}\OperatorTok{=}\FloatTok{0.7}\NormalTok{)}

\CommentTok{\# Configurar las etiquetas del eje x (opcional, ya que solo hay un grupo)}
\NormalTok{plt.xticks([}\DecValTok{1}\NormalTok{], [}\StringTok{\textquotesingle{}Alumnos\textquotesingle{}}\NormalTok{])}

\CommentTok{\# Ajustar el diseño para evitar recortes}
\NormalTok{plt.tight\_layout()}

\CommentTok{\# Guardar la figura en un archivo (opcional, descomentar para usar)}
\CommentTok{\# plt.savefig(\textquotesingle{}edades\_alumnos\_secundaria.png\textquotesingle{})}

\CommentTok{\# Mostrar el gráfico}
\NormalTok{plt.show()}
\end{Highlighting}
\end{Shaded}

\section{Grafico de barras
combinadas}\label{grafico-de-barras-combinadas}

\begin{Shaded}
\begin{Highlighting}[]
\ImportTok{import}\NormalTok{ matplotlib.pyplot }\ImportTok{as}\NormalTok{ plt}

\CommentTok{\# Configurar el tamaño de la figura para una mejor visualización}
\NormalTok{plt.figure(figsize}\OperatorTok{=}\NormalTok{(}\DecValTok{10}\NormalTok{, }\DecValTok{8}\NormalTok{))}

\CommentTok{\# Datos de ventas de autos por día (en unidades)}
\NormalTok{ventas\_toyota }\OperatorTok{=}\NormalTok{ [}\DecValTok{10}\NormalTok{, }\DecValTok{15}\NormalTok{, }\DecValTok{19}\NormalTok{, }\DecValTok{14}\NormalTok{, }\DecValTok{9}\NormalTok{]}
\NormalTok{ventas\_audi }\OperatorTok{=}\NormalTok{ [}\DecValTok{15}\NormalTok{, }\DecValTok{25}\NormalTok{, }\DecValTok{27}\NormalTok{, }\DecValTok{24}\NormalTok{, }\DecValTok{28}\NormalTok{]}
\NormalTok{dias }\OperatorTok{=}\NormalTok{ [}\DecValTok{0}\NormalTok{, }\DecValTok{1}\NormalTok{, }\DecValTok{2}\NormalTok{, }\DecValTok{3}\NormalTok{, }\DecValTok{4}\NormalTok{]  }\CommentTok{\# Posiciones para los días}
\NormalTok{ancho\_barras }\OperatorTok{=} \FloatTok{0.5}  \CommentTok{\# Grosor de las barras}

\CommentTok{\# Crear el gráfico de barras apiladas}
\NormalTok{barras\_toyota }\OperatorTok{=}\NormalTok{ plt.bar(dias, ventas\_toyota, ancho\_barras, label}\OperatorTok{=}\StringTok{\textquotesingle{}Toyota\textquotesingle{}}\NormalTok{, color}\OperatorTok{=}\StringTok{\textquotesingle{}skyblue\textquotesingle{}}\NormalTok{, edgecolor}\OperatorTok{=}\StringTok{\textquotesingle{}black\textquotesingle{}}\NormalTok{)}
\NormalTok{barras\_audi }\OperatorTok{=}\NormalTok{ plt.bar(dias, ventas\_audi, ancho\_barras, bottom}\OperatorTok{=}\NormalTok{ventas\_toyota, label}\OperatorTok{=}\StringTok{\textquotesingle{}Audi\textquotesingle{}}\NormalTok{, color}\OperatorTok{=}\StringTok{\textquotesingle{}salmon\textquotesingle{}}\NormalTok{, edgecolor}\OperatorTok{=}\StringTok{\textquotesingle{}black\textquotesingle{}}\NormalTok{)}

\CommentTok{\# Configurar las etiquetas del eje x con los días}
\NormalTok{plt.xticks(dias, [}\StringTok{\textquotesingle{}Día 1\textquotesingle{}}\NormalTok{, }\StringTok{\textquotesingle{}Día 2\textquotesingle{}}\NormalTok{, }\StringTok{\textquotesingle{}Día 3\textquotesingle{}}\NormalTok{, }\StringTok{\textquotesingle{}Día 4\textquotesingle{}}\NormalTok{, }\StringTok{\textquotesingle{}Día 5\textquotesingle{}}\NormalTok{])}

\CommentTok{\# Etiquetas de los ejes y título con formato adecuado}
\NormalTok{plt.xlabel(}\StringTok{\textquotesingle{}Días\textquotesingle{}}\NormalTok{)}
\NormalTok{plt.ylabel(}\StringTok{\textquotesingle{}Unidades Vendidas\textquotesingle{}}\NormalTok{)}
\NormalTok{plt.title(}\StringTok{\textquotesingle{}Ventas de Autos por Día (Toyota vs. Audi)\textquotesingle{}}\NormalTok{)}

\CommentTok{\# Configurar las marcas del eje y para mayor claridad}
\NormalTok{plt.yticks([}\DecValTok{0}\NormalTok{, }\DecValTok{5}\NormalTok{, }\DecValTok{10}\NormalTok{, }\DecValTok{15}\NormalTok{, }\DecValTok{20}\NormalTok{, }\DecValTok{25}\NormalTok{, }\DecValTok{30}\NormalTok{, }\DecValTok{35}\NormalTok{, }\DecValTok{40}\NormalTok{, }\DecValTok{45}\NormalTok{, }\DecValTok{50}\NormalTok{])}

\CommentTok{\# Añadir una cuadrícula en el eje y para facilitar la lectura}
\NormalTok{plt.grid(}\VariableTok{True}\NormalTok{, axis}\OperatorTok{=}\StringTok{\textquotesingle{}y\textquotesingle{}}\NormalTok{, linestyle}\OperatorTok{=}\StringTok{\textquotesingle{}{-}{-}\textquotesingle{}}\NormalTok{, alpha}\OperatorTok{=}\FloatTok{0.7}\NormalTok{)}

\CommentTok{\# Añadir una leyenda para identificar las marcas}
\NormalTok{plt.legend()}

\CommentTok{\# Ajustar el diseño para evitar recortes}
\NormalTok{plt.tight\_layout()}

\CommentTok{\# Guardar la figura en un archivo (opcional, descomentar para usar)}
\CommentTok{\# plt.savefig(\textquotesingle{}ventas\_autos\_toyota\_audi.png\textquotesingle{})}

\CommentTok{\# Mostrar el gráfico}
\NormalTok{plt.show()}
\end{Highlighting}
\end{Shaded}

\section{Graficos combinados}\label{graficos-combinados}

\begin{Shaded}
\begin{Highlighting}[]
\ImportTok{import}\NormalTok{ matplotlib.pyplot }\ImportTok{as}\NormalTok{ plt}

\CommentTok{\# Configurar el tamaño de la figura para una mejor visualización}
\NormalTok{plt.figure(figsize}\OperatorTok{=}\NormalTok{(}\DecValTok{10}\NormalTok{, }\DecValTok{8}\NormalTok{))}

\CommentTok{\# Datos de vacunación proyectada (en número de pacientes)}
\NormalTok{vacunacion\_proyectada }\OperatorTok{=}\NormalTok{ [}\DecValTok{250}\NormalTok{, }\DecValTok{120}\NormalTok{, }\DecValTok{270}\NormalTok{, }\DecValTok{560}\NormalTok{, }\DecValTok{450}\NormalTok{, }\DecValTok{280}\NormalTok{, }\DecValTok{550}\NormalTok{]}
\CommentTok{\# Datos de vacunación real (en número de pacientes)}
\NormalTok{vacunacion\_real }\OperatorTok{=}\NormalTok{ [}\DecValTok{150}\NormalTok{, }\DecValTok{300}\NormalTok{, }\DecValTok{120}\NormalTok{, }\DecValTok{550}\NormalTok{, }\DecValTok{500}\NormalTok{, }\DecValTok{240}\NormalTok{, }\DecValTok{600}\NormalTok{]}
\NormalTok{meses }\OperatorTok{=}\NormalTok{ [}\StringTok{"Enero"}\NormalTok{, }\StringTok{"Febrero"}\NormalTok{, }\StringTok{"Marzo"}\NormalTok{, }\StringTok{"Abril"}\NormalTok{, }\StringTok{"Mayo"}\NormalTok{, }\StringTok{"Junio"}\NormalTok{, }\StringTok{"Julio"}\NormalTok{]}

\CommentTok{\# Graficar la vacunación proyectada como una línea con marcadores}
\NormalTok{plt.plot(meses, vacunacion\_proyectada, marker}\OperatorTok{=}\StringTok{\textquotesingle{}d\textquotesingle{}}\NormalTok{, linestyle}\OperatorTok{=}\StringTok{\textquotesingle{}{-}{-}\textquotesingle{}}\NormalTok{, color}\OperatorTok{=}\StringTok{\textquotesingle{}red\textquotesingle{}}\NormalTok{, label}\OperatorTok{=}\StringTok{\textquotesingle{}Vacunación Proyectada\textquotesingle{}}\NormalTok{)}

\CommentTok{\# Graficar la vacunación real como barras}
\NormalTok{plt.bar(meses, vacunacion\_real, color}\OperatorTok{=}\StringTok{\textquotesingle{}skyblue\textquotesingle{}}\NormalTok{, edgecolor}\OperatorTok{=}\StringTok{\textquotesingle{}black\textquotesingle{}}\NormalTok{, alpha}\OperatorTok{=}\FloatTok{0.7}\NormalTok{, label}\OperatorTok{=}\StringTok{\textquotesingle{}Vacunación Real\textquotesingle{}}\NormalTok{)}

\CommentTok{\# Etiquetas de los ejes y título con formato adecuado}
\NormalTok{plt.xlabel(}\StringTok{\textquotesingle{}Meses\textquotesingle{}}\NormalTok{)}
\NormalTok{plt.ylabel(}\StringTok{\textquotesingle{}Número de Pacientes Vacunados\textquotesingle{}}\NormalTok{)}
\NormalTok{plt.title(}\StringTok{\textquotesingle{}Vacunación Real vs. Proyectada (Enero {-} Julio)\textquotesingle{}}\NormalTok{)}

\CommentTok{\# Añadir una leyenda}
\NormalTok{plt.legend()}

\CommentTok{\# Añadir una cuadrícula en el eje y para facilitar la lectura}
\NormalTok{plt.grid(}\VariableTok{True}\NormalTok{, axis}\OperatorTok{=}\StringTok{\textquotesingle{}y\textquotesingle{}}\NormalTok{, linestyle}\OperatorTok{=}\StringTok{\textquotesingle{}{-}{-}\textquotesingle{}}\NormalTok{, alpha}\OperatorTok{=}\FloatTok{0.7}\NormalTok{)}

\CommentTok{\# Ajustar el diseño para evitar recortes}
\NormalTok{plt.tight\_layout()}

\CommentTok{\# Guardar la figura en un archivo (opcional, descomentar para usar)}
\CommentTok{\# plt.savefig(\textquotesingle{}vacunacion\_real\_vs\_proyectada.png\textquotesingle{})}

\CommentTok{\# Mostrar el gráfico}
\NormalTok{plt.show()}
\end{Highlighting}
\end{Shaded}







\end{document}
