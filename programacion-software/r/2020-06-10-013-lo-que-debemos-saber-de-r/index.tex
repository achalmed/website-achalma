\documentclass[
  jou,
  floatsintext,
  longtable,
  a4paper,
  nolmodern,
  notxfonts,
  notimes,
  colorlinks=true,linkcolor=blue,citecolor=blue,urlcolor=blue]{apa7}

\usepackage{amsmath}
\usepackage{amssymb}



\usepackage[bidi=default]{babel}
\babelprovide[main,import]{spanish}
\StartBabelCommands{spanish}{captions} [unicode, fontenc=TU EU1 EU2, charset=utf8] \SetString{\keywordname}{Palabras
Claves}
\EndBabelCommands


% get rid of language-specific shorthands (see #6817):
\let\LanguageShortHands\languageshorthands
\def\languageshorthands#1{}

\RequirePackage{longtable}
\RequirePackage{threeparttablex}

\makeatletter
\renewcommand{\paragraph}{\@startsection{paragraph}{4}{\parindent}%
	{0\baselineskip \@plus 0.2ex \@minus 0.2ex}%
	{-.5em}%
	{\normalfont\normalsize\bfseries\typesectitle}}

\renewcommand{\subparagraph}[1]{\@startsection{subparagraph}{5}{0.5em}%
	{0\baselineskip \@plus 0.2ex \@minus 0.2ex}%
	{-\z@\relax}%
	{\normalfont\normalsize\bfseries\itshape\hspace{\parindent}{#1}\textit{\addperi}}{\relax}}
\makeatother




\usepackage{longtable, booktabs, multirow, multicol, colortbl, hhline, caption, array, float, xpatch}
\usepackage{subcaption}
\renewcommand\thesubfigure{\Alph{subfigure}}
\setcounter{topnumber}{2}
\setcounter{bottomnumber}{2}
\setcounter{totalnumber}{4}
\renewcommand{\topfraction}{0.85}
\renewcommand{\bottomfraction}{0.85}
\renewcommand{\textfraction}{0.15}
\renewcommand{\floatpagefraction}{0.7}

\usepackage{tcolorbox}
\tcbuselibrary{listings,theorems, breakable, skins}
\usepackage{fontawesome5}

\definecolor{quarto-callout-color}{HTML}{909090}
\definecolor{quarto-callout-note-color}{HTML}{0758E5}
\definecolor{quarto-callout-important-color}{HTML}{CC1914}
\definecolor{quarto-callout-warning-color}{HTML}{EB9113}
\definecolor{quarto-callout-tip-color}{HTML}{00A047}
\definecolor{quarto-callout-caution-color}{HTML}{FC5300}
\definecolor{quarto-callout-color-frame}{HTML}{ACACAC}
\definecolor{quarto-callout-note-color-frame}{HTML}{4582EC}
\definecolor{quarto-callout-important-color-frame}{HTML}{D9534F}
\definecolor{quarto-callout-warning-color-frame}{HTML}{F0AD4E}
\definecolor{quarto-callout-tip-color-frame}{HTML}{02B875}
\definecolor{quarto-callout-caution-color-frame}{HTML}{FD7E14}

%\newlength\Oldarrayrulewidth
%\newlength\Oldtabcolsep


\usepackage{hyperref}



\usepackage{color}
\usepackage{fancyvrb}
\newcommand{\VerbBar}{|}
\newcommand{\VERB}{\Verb[commandchars=\\\{\}]}
\DefineVerbatimEnvironment{Highlighting}{Verbatim}{commandchars=\\\{\}}
% Add ',fontsize=\small' for more characters per line
\usepackage{framed}
\definecolor{shadecolor}{RGB}{241,243,245}
\newenvironment{Shaded}{\begin{snugshade}}{\end{snugshade}}
\newcommand{\AlertTok}[1]{\textcolor[rgb]{0.68,0.00,0.00}{#1}}
\newcommand{\AnnotationTok}[1]{\textcolor[rgb]{0.37,0.37,0.37}{#1}}
\newcommand{\AttributeTok}[1]{\textcolor[rgb]{0.40,0.45,0.13}{#1}}
\newcommand{\BaseNTok}[1]{\textcolor[rgb]{0.68,0.00,0.00}{#1}}
\newcommand{\BuiltInTok}[1]{\textcolor[rgb]{0.00,0.23,0.31}{#1}}
\newcommand{\CharTok}[1]{\textcolor[rgb]{0.13,0.47,0.30}{#1}}
\newcommand{\CommentTok}[1]{\textcolor[rgb]{0.37,0.37,0.37}{#1}}
\newcommand{\CommentVarTok}[1]{\textcolor[rgb]{0.37,0.37,0.37}{\textit{#1}}}
\newcommand{\ConstantTok}[1]{\textcolor[rgb]{0.56,0.35,0.01}{#1}}
\newcommand{\ControlFlowTok}[1]{\textcolor[rgb]{0.00,0.23,0.31}{\textbf{#1}}}
\newcommand{\DataTypeTok}[1]{\textcolor[rgb]{0.68,0.00,0.00}{#1}}
\newcommand{\DecValTok}[1]{\textcolor[rgb]{0.68,0.00,0.00}{#1}}
\newcommand{\DocumentationTok}[1]{\textcolor[rgb]{0.37,0.37,0.37}{\textit{#1}}}
\newcommand{\ErrorTok}[1]{\textcolor[rgb]{0.68,0.00,0.00}{#1}}
\newcommand{\ExtensionTok}[1]{\textcolor[rgb]{0.00,0.23,0.31}{#1}}
\newcommand{\FloatTok}[1]{\textcolor[rgb]{0.68,0.00,0.00}{#1}}
\newcommand{\FunctionTok}[1]{\textcolor[rgb]{0.28,0.35,0.67}{#1}}
\newcommand{\ImportTok}[1]{\textcolor[rgb]{0.00,0.46,0.62}{#1}}
\newcommand{\InformationTok}[1]{\textcolor[rgb]{0.37,0.37,0.37}{#1}}
\newcommand{\KeywordTok}[1]{\textcolor[rgb]{0.00,0.23,0.31}{\textbf{#1}}}
\newcommand{\NormalTok}[1]{\textcolor[rgb]{0.00,0.23,0.31}{#1}}
\newcommand{\OperatorTok}[1]{\textcolor[rgb]{0.37,0.37,0.37}{#1}}
\newcommand{\OtherTok}[1]{\textcolor[rgb]{0.00,0.23,0.31}{#1}}
\newcommand{\PreprocessorTok}[1]{\textcolor[rgb]{0.68,0.00,0.00}{#1}}
\newcommand{\RegionMarkerTok}[1]{\textcolor[rgb]{0.00,0.23,0.31}{#1}}
\newcommand{\SpecialCharTok}[1]{\textcolor[rgb]{0.37,0.37,0.37}{#1}}
\newcommand{\SpecialStringTok}[1]{\textcolor[rgb]{0.13,0.47,0.30}{#1}}
\newcommand{\StringTok}[1]{\textcolor[rgb]{0.13,0.47,0.30}{#1}}
\newcommand{\VariableTok}[1]{\textcolor[rgb]{0.07,0.07,0.07}{#1}}
\newcommand{\VerbatimStringTok}[1]{\textcolor[rgb]{0.13,0.47,0.30}{#1}}
\newcommand{\WarningTok}[1]{\textcolor[rgb]{0.37,0.37,0.37}{\textit{#1}}}

\providecommand{\tightlist}{%
  \setlength{\itemsep}{0pt}\setlength{\parskip}{0pt}}
\usepackage{longtable,booktabs,array}
\usepackage{calc} % for calculating minipage widths
% Correct order of tables after \paragraph or \subparagraph
\usepackage{etoolbox}
\makeatletter
\patchcmd\longtable{\par}{\if@noskipsec\mbox{}\fi\par}{}{}
\makeatother
% Allow footnotes in longtable head/foot
\IfFileExists{footnotehyper.sty}{\usepackage{footnotehyper}}{\usepackage{footnote}}
\makesavenoteenv{longtable}

\usepackage{graphicx}
\makeatletter
\newsavebox\pandoc@box
\newcommand*\pandocbounded[1]{% scales image to fit in text height/width
  \sbox\pandoc@box{#1}%
  \Gscale@div\@tempa{\textheight}{\dimexpr\ht\pandoc@box+\dp\pandoc@box\relax}%
  \Gscale@div\@tempb{\linewidth}{\wd\pandoc@box}%
  \ifdim\@tempb\p@<\@tempa\p@\let\@tempa\@tempb\fi% select the smaller of both
  \ifdim\@tempa\p@<\p@\scalebox{\@tempa}{\usebox\pandoc@box}%
  \else\usebox{\pandoc@box}%
  \fi%
}
% Set default figure placement to htbp
\def\fps@figure{htbp}
\makeatother







\usepackage{newtx}

\defaultfontfeatures{Scale=MatchLowercase}
\defaultfontfeatures[\rmfamily]{Ligatures=TeX,Scale=1}





\title{Lo que debemos saber de R: Explorando las capacidades de R y su
uso en el entorno Linux}


\shorttitle{Editar}


\usepackage{etoolbox}



\ccoppy{\textcopyright~2020}



\author{Edison Achalma}



\affiliation{
{Escuela Profesional de Economía, Universidad Nacional de San Cristóbal
de Huamanga}}




\leftheader{Achalma}

\date{2020-06-10}


\abstract{Primer parrafo de abstrac }

\keywords{keyword1, keyword2}

\authornote{\par{\addORCIDlink{Edison Achalma}{0000-0001-6996-3364}} 
\par{ }
\par{   El autor no tiene conflictos de interés que revelar.    Los
roles de autor se clasificaron utilizando la taxonomía de roles de
colaborador (CRediT; https://credit.niso.org/) de la siguiente
manera:  Edison Achalma:   conceptualización, redacción}
\par{La correspondencia relativa a este artículo debe dirigirse a Edison
Achalma, Email: \href{mailto:elmer.achalma.09@unsch.edu.pe}{elmer.achalma.09@unsch.edu.pe}}
}

\usepackage{pbalance} 
\usepackage{float}
\makeatletter
\let\oldtpt\ThreePartTable
\let\endoldtpt\endThreePartTable
\def\ThreePartTable{\@ifnextchar[\ThreePartTable@i \ThreePartTable@ii}
\def\ThreePartTable@i[#1]{\begin{figure}[!htbp]
\onecolumn
\begin{minipage}{0.5\textwidth}
\oldtpt[#1]
}
\def\ThreePartTable@ii{\begin{figure}[!htbp]
\onecolumn
\begin{minipage}{0.5\textwidth}
\oldtpt
}
\def\endThreePartTable{
\endoldtpt
\end{minipage}
\twocolumn
\end{figure}}
\makeatother


\makeatletter
\let\endoldlt\endlongtable		
\def\endlongtable{
\hline
\endoldlt}
\makeatother

\newenvironment{twocolumntable}% environment name
{% begin code
\begin{table*}[!htbp]%
\onecolumn%
}%
{%
\twocolumn%
\end{table*}%
}% end code

\urlstyle{same}



\makeatletter
\@ifpackageloaded{tcolorbox}{}{\usepackage[skins,breakable]{tcolorbox}}
\@ifpackageloaded{fontawesome5}{}{\usepackage{fontawesome5}}
\definecolor{quarto-callout-color}{HTML}{909090}
\definecolor{quarto-callout-note-color}{HTML}{0758E5}
\definecolor{quarto-callout-important-color}{HTML}{CC1914}
\definecolor{quarto-callout-warning-color}{HTML}{EB9113}
\definecolor{quarto-callout-tip-color}{HTML}{00A047}
\definecolor{quarto-callout-caution-color}{HTML}{FC5300}
\definecolor{quarto-callout-color-frame}{HTML}{acacac}
\definecolor{quarto-callout-note-color-frame}{HTML}{4582ec}
\definecolor{quarto-callout-important-color-frame}{HTML}{d9534f}
\definecolor{quarto-callout-warning-color-frame}{HTML}{f0ad4e}
\definecolor{quarto-callout-tip-color-frame}{HTML}{02b875}
\definecolor{quarto-callout-caution-color-frame}{HTML}{fd7e14}
\makeatother
\makeatletter
\@ifpackageloaded{caption}{}{\usepackage{caption}}
\AtBeginDocument{%
\ifdefined\contentsname
  \renewcommand*\contentsname{Tabla de contenidos}
\else
  \newcommand\contentsname{Tabla de contenidos}
\fi
\ifdefined\listfigurename
  \renewcommand*\listfigurename{Listado de Figuras}
\else
  \newcommand\listfigurename{Listado de Figuras}
\fi
\ifdefined\listtablename
  \renewcommand*\listtablename{Listado de Tablas}
\else
  \newcommand\listtablename{Listado de Tablas}
\fi
\ifdefined\figurename
  \renewcommand*\figurename{Figura}
\else
  \newcommand\figurename{Figura}
\fi
\ifdefined\tablename
  \renewcommand*\tablename{Tabla}
\else
  \newcommand\tablename{Tabla}
\fi
}
\@ifpackageloaded{float}{}{\usepackage{float}}
\floatstyle{ruled}
\@ifundefined{c@chapter}{\newfloat{codelisting}{h}{lop}}{\newfloat{codelisting}{h}{lop}[chapter]}
\floatname{codelisting}{Listado}
\newcommand*\listoflistings{\listof{codelisting}{Listado de Listados}}
\makeatother
\makeatletter
\makeatother
\makeatletter
\@ifpackageloaded{caption}{}{\usepackage{caption}}
\@ifpackageloaded{subcaption}{}{\usepackage{subcaption}}
\makeatother
\makeatletter
\@ifpackageloaded{fontawesome5}{}{\usepackage{fontawesome5}}
\makeatother

% From https://tex.stackexchange.com/a/645996/211326
%%% apa7 doesn't want to add appendix section titles in the toc
%%% let's make it do it
\makeatletter
\xpatchcmd{\appendix}
  {\par}
  {\addcontentsline{toc}{section}{\@currentlabelname}\par}
  {}{}
\makeatother

%% Disable longtable counter
%% https://tex.stackexchange.com/a/248395/211326

\usepackage{etoolbox}

\makeatletter
\patchcmd{\LT@caption}
  {\bgroup}
  {\bgroup\global\LTpatch@captiontrue}
  {}{}
\patchcmd{\longtable}
  {\par}
  {\par\global\LTpatch@captionfalse}
  {}{}
\apptocmd{\endlongtable}
  {\ifLTpatch@caption\else\addtocounter{table}{-1}\fi}
  {}{}
\newif\ifLTpatch@caption
\makeatother

\begin{document}

\maketitle

\hypertarget{toc}{}
\tableofcontents
\newpage
\section[Introduction]{Lo que debemos saber de R}

\setcounter{secnumdepth}{-\maxdimen} % remove section numbering

\setlength\LTleft{0pt}


\section{Lo que debemos saber}\label{lo-que-debemos-saber}

\subsection{Tipos de datos}\label{tipos-de-datos}

En R, es fundamental comprender los diferentes tipos de datos
disponibles. A continuación, exploraremos los tres tipos básicos de
datos en R y cómo se utilizan en la programación.

\subsubsection{1. Tipos de datos
numéricos}\label{tipos-de-datos-numuxe9ricos}

Los datos numéricos en R se dividen en dos tipos principales:

\begin{enumerate}
\def\labelenumi{\alph{enumi}.}
\item
  Números reales, se conoce como \texttt{double}. Estos son los números
  más comunes y se utilizan para representar valores decimales. Por
  ejemplo, 3.14 y 2.71828 son números reales en R. La precisión de los
  números reales en R depende de la máquina en la que se ejecuta el
  programa.
\item
  Números enteros, se conoce como \texttt{integer}. Estos son números
  que no contienen decimales y se utilizan para representar valores
  enteros. Por ejemplo, 1, 2, -5 son ejemplos de números enteros en R.
  Los números enteros se utilizan cuando no se requiere precisión
  decimal.
\end{enumerate}

\subsubsection{2. Tipo de datos lógico}\label{tipo-de-datos-luxf3gico}

El tipo de dato lógico en R se conoce como \texttt{booleano}. Este tipo
de dato puede tener uno de dos valores: TRUE o FALSE. Los valores
booleanos se utilizan principalmente para realizar operaciones de
comparación y evaluación lógica en los programas. Por ejemplo, se puede
usar una expresión lógica para verificar si una condición es verdadera o
falsa.

\subsubsection{3. Tipo de datos
carácter}\label{tipo-de-datos-caruxe1cter}

El tipo de dato carácter en R se utiliza para almacenar letras
\texttt{text} y símbolos \texttt{strings}. Los datos de tipo carácter se
definen utilizando comillas simples ('\,`) o comillas dobles (``\,``).
Por ejemplo,''Hola'' y 'RStudio' son ejemplos de datos de tipo carácter
en R. Los datos de tipo carácter se utilizan con frecuencia para
almacenar texto legible por humanos, como nombres, descripciones o
mensajes.

\begin{quote}
Es importante comprender estos tipos de datos en R, ya que nos permiten
manipular y realizar operaciones en los datos de manera adecuada. Cada
tipo de dato tiene sus propias características y funciones asociadas que
nos permiten realizar tareas específicas en la programación.
\end{quote}

\subsection{Estructura de datos}\label{estructura-de-datos}

Las estructuras de datos nos permiten organizar y manipular la
información de manera eficiente. A continuación, exploraremos las
principales estructuras de datos disponibles en R y cómo se utilizan en
la programación.

\subsubsection{1. Escalar}\label{escalar}

Un escalar es un dato individual, como un número o una palabra, que no
está agrupado con otros elementos. En R, los escalares pueden ser de
diferentes tipos de datos, como numéricos, lógicos o caracteres. Estos
datos se utilizan cuando solo necesitamos almacenar una única
observación.

\subsubsection{2. Vector}\label{vector}

Un vector es una colección ordenada de elementos del mismo tipo de dato.
Puede contener números, valores lógicos o caracteres. En R, los vectores
son utilizados para almacenar conjuntos de datos relacionados. Por
ejemplo, podemos tener un vector de edades o un vector de nombres. Los
vectores son una de las estructuras de datos más utilizadas en R y nos
permiten realizar operaciones y cálculos de manera eficiente.

\textbf{Vectores}

Concatenación de elementos con \textbf{\texttt{c()}}: Se utiliza la
función \texttt{c()} para concatenar elementos y crear vectores en R.

\begin{Shaded}
\begin{Highlighting}[]
\FunctionTok{c}\NormalTok{(}\FloatTok{0.5}\NormalTok{, }\FloatTok{0.6}\NormalTok{, }\FloatTok{0.25}\NormalTok{) }\CommentTok{\# números decimales (double)}
\FunctionTok{c}\NormalTok{(}\DecValTok{9}\NormalTok{L, }\DecValTok{10}\NormalTok{L, }\DecValTok{11}\NormalTok{L, }\DecValTok{12}\NormalTok{L, }\DecValTok{13}\NormalTok{L) }\CommentTok{\# números enteros (integer)}
\FunctionTok{c}\NormalTok{(}\DecValTok{9}\SpecialCharTok{:}\DecValTok{13}\NormalTok{) }\CommentTok{\# secuencia de números enteros (integer sequence)}
\FunctionTok{c}\NormalTok{(}\ConstantTok{TRUE}\NormalTok{, }\ConstantTok{FALSE}\NormalTok{, }\ConstantTok{FALSE}\NormalTok{) }\CommentTok{\# valores lógicos (logical)}
\FunctionTok{c}\NormalTok{(}\DecValTok{1} \SpecialCharTok{+} \DecValTok{0}\NormalTok{i, }\DecValTok{2} \SpecialCharTok{+} \DecValTok{4}\NormalTok{i) }\CommentTok{\# números complejos (complex)}
\FunctionTok{c}\NormalTok{(}\StringTok{"a"}\NormalTok{, }\StringTok{"b"}\NormalTok{, }\StringTok{"c"}\NormalTok{) }\CommentTok{\# caracteres (character)}
\end{Highlighting}
\end{Shaded}

\textbf{Acciones con vectores}

\begin{enumerate}
\def\labelenumi{\arabic{enumi}.}
\item
  Asignar los vectores a nombres:

  Creamos un vector llamado ``dbl'' que contiene los números decimales
  0.5, 0.6 y 0.25.

\begin{Shaded}
\begin{Highlighting}[]
\NormalTok{dbl }\OtherTok{\textless{}{-}} \FunctionTok{c}\NormalTok{(}\FloatTok{0.5}\NormalTok{, }\FloatTok{0.6}\NormalTok{, }\FloatTok{0.25}\NormalTok{)}
\end{Highlighting}
\end{Shaded}

  Creamos un vector llamado ``chr'' que contiene los caracteres ``a'',
  ``b'' y ``c''.

\begin{Shaded}
\begin{Highlighting}[]
\NormalTok{chr }\OtherTok{\textless{}{-}} \FunctionTok{c}\NormalTok{(}\StringTok{"a"}\NormalTok{, }\StringTok{"b"}\NormalTok{, }\StringTok{"c"}\NormalTok{)}
\end{Highlighting}
\end{Shaded}
\item
  Imprimir los vectores ``dbl'' y ``chr'' en la consola:

  Visualizamos en la consola el contenido del vector ``dbl'', que son
  los números decimales 0.5, 0.6 y 0.25.

\begin{Shaded}
\begin{Highlighting}[]
\NormalTok{dbl}
\end{Highlighting}
\end{Shaded}

  Visualizamos en la consola el contenido del vector ``chr'', que son
  los caracteres ``a'', ``b'' y ``c''.

\begin{Shaded}
\begin{Highlighting}[]
\NormalTok{chr}
\end{Highlighting}
\end{Shaded}
\item
  Verificar el número de elementos en ``dbl'' y ``chr'':

  Calculamos y mostramos en la consola la longitud del vector ``dbl'',
  que es 3.

\begin{Shaded}
\begin{Highlighting}[]
\FunctionTok{length}\NormalTok{(dbl)}
\end{Highlighting}
\end{Shaded}

  Calculamos y mostramos en la consola la longitud del vector ``chr'',
  que es 3.

\begin{Shaded}
\begin{Highlighting}[]
\FunctionTok{length}\NormalTok{(chr)}
\end{Highlighting}
\end{Shaded}
\item
  Verificar el tipo de dato de ``dbl'' y ``chr'':

  Visualizamos en la consola el tipo de dato del vector ``dbl'', que es
  ``double'' (números decimales).

\begin{Shaded}
\begin{Highlighting}[]
\FunctionTok{typeof}\NormalTok{(dbl)}
\end{Highlighting}
\end{Shaded}

  Visualizamos en la consola el tipo de dato del vector ``chr'', que es
  ``character'' (caracteres).

\begin{Shaded}
\begin{Highlighting}[]
\FunctionTok{typeof}\NormalTok{(chr)}
\end{Highlighting}
\end{Shaded}
\item
  Combinar dos vectores:

  Se puede combinar el vector ``dbl'' consigo mismo utilizando la
  función ``c()'', creando un nuevo vector que contiene los elementos
  duplicados del vector original.

\begin{Shaded}
\begin{Highlighting}[]
\FunctionTok{c}\NormalTok{(dbl, dbl)}
\end{Highlighting}
\end{Shaded}

  Tambien se puede combina el vector ``dbl'' con el vector ``chr''
  utilizando la función ``c()'', creando un nuevo vector que contiene
  los elementos de ambos vectores concatenados.

\begin{Shaded}
\begin{Highlighting}[]
\FunctionTok{c}\NormalTok{(dbl, chr)}
\end{Highlighting}
\end{Shaded}
\end{enumerate}

\begin{tcolorbox}[enhanced jigsaw, colback=white, opacitybacktitle=0.6, leftrule=.75mm, coltitle=black, title=\textcolor{quarto-callout-note-color}{\faInfo}\hspace{0.5em}{Nota}, colbacktitle=quarto-callout-note-color!10!white, toptitle=1mm, bottomtitle=1mm, arc=.35mm, left=2mm, breakable, titlerule=0mm, bottomrule=.15mm, toprule=.15mm, opacityback=0, rightrule=.15mm, colframe=quarto-callout-note-color-frame]

El cambio automático del tipo de datos del vector resultante se denomina
coerción. La coerción garantiza que se mantiene el mismo tipo de datos
para cada elemento del vector.

\end{tcolorbox}

\textbf{Operaciones aritméticas con vectores}

\begin{enumerate}
\def\labelenumi{\arabic{enumi}.}
\item
  Definamos dos nuevos vectores numéricos llamados \texttt{a} y
  \texttt{b} con 4 elementos cada uno:

\begin{Shaded}
\begin{Highlighting}[]
\NormalTok{a }\OtherTok{\textless{}{-}} \FunctionTok{c}\NormalTok{(}\DecValTok{1}\NormalTok{, }\DecValTok{2}\NormalTok{, }\DecValTok{3}\NormalTok{, }\DecValTok{4}\NormalTok{)}
\NormalTok{b }\OtherTok{\textless{}{-}} \FunctionTok{c}\NormalTok{(}\DecValTok{10}\NormalTok{, }\DecValTok{20}\NormalTok{, }\DecValTok{30}\NormalTok{, }\DecValTok{40}\NormalTok{)}
\end{Highlighting}
\end{Shaded}
\item
  Realizamos una multiplicación escalar de \texttt{a} por 5, lo que
  significa que cada elemento en \texttt{a} se multiplica por 5:

\begin{Shaded}
\begin{Highlighting}[]
\NormalTok{a }\SpecialCharTok{*} \DecValTok{5}
\end{Highlighting}
\end{Shaded}
\item
  Realizamos una multiplicación de vectores entre \texttt{a} y
  \texttt{b}, lo que implica multiplicar cada elemento en \texttt{a} por
  el elemento correspondiente en \texttt{b}:

\begin{Shaded}
\begin{Highlighting}[]
\NormalTok{a }\SpecialCharTok{*}\NormalTok{ b}
\end{Highlighting}
\end{Shaded}
\item
  Creamos un nuevo vector numérico llamado \texttt{v} con longitud 5.

\begin{Shaded}
\begin{Highlighting}[]
\NormalTok{v }\OtherTok{\textless{}{-}} \FunctionTok{c}\NormalTok{(}\FloatTok{1.1}\NormalTok{, }\FloatTok{1.2}\NormalTok{, }\FloatTok{1.3}\NormalTok{, }\FloatTok{1.4}\NormalTok{, }\FloatTok{1.5}\NormalTok{)}
\NormalTok{a }\SpecialCharTok{*}\NormalTok{ v}
\end{Highlighting}
\end{Shaded}
\end{enumerate}

\begin{tcolorbox}[enhanced jigsaw, colback=white, opacitybacktitle=0.6, leftrule=.75mm, coltitle=black, title=\textcolor{quarto-callout-note-color}{\faInfo}\hspace{0.5em}{Nota}, colbacktitle=quarto-callout-note-color!10!white, toptitle=1mm, bottomtitle=1mm, arc=.35mm, left=2mm, breakable, titlerule=0mm, bottomrule=.15mm, toprule=.15mm, opacityback=0, rightrule=.15mm, colframe=quarto-callout-note-color-frame]

Las operaciones aritméticas de los vectores se realizan por elementos.
si dos vectores no tienen la misma longitud, el vector más corto se
reciclará para que coincida con el más largo (en este caso, se vuelve a
utilizar el primer elemento de a).

\end{tcolorbox}

\subsubsection{3. Matriz}\label{matriz}

Una matriz es una estructura bidimensional que contiene elementos
organizados en filas y columnas. Todos los elementos de una matriz deben
ser del mismo tipo de dato. Las matrices son útiles para almacenar datos
tabulares, como una tabla de datos con variables en filas y
observaciones en columnas. En R, podemos realizar operaciones
matriciales y manipular los datos de manera eficiente utilizando esta
estructura.

\textbf{Matrices}

\begin{enumerate}
\def\labelenumi{\arabic{enumi}.}
\item
  Combinamos los vectores \texttt{a} y \texttt{b}, definidas
  anteriormente, por columnas utilizando la función \texttt{cbind()}:

\begin{Shaded}
\begin{Highlighting}[]
\NormalTok{A }\OtherTok{\textless{}{-}} \FunctionTok{cbind}\NormalTok{(a, b)}
\NormalTok{A}
\end{Highlighting}
\end{Shaded}

  Esta opción combina los vectores \texttt{a} y \texttt{b} por columnas,
  creando una matriz \texttt{A} donde los elementos de \texttt{a} forman
  la primera columna y los elementos de \texttt{b} forman la segunda
  columna.
\item
  Combinamos los vectores \texttt{a} y \texttt{b} por filas utilizando
  la función \texttt{rbind()}:

\begin{Shaded}
\begin{Highlighting}[]
\NormalTok{B }\OtherTok{\textless{}{-}} \FunctionTok{rbind}\NormalTok{(a, b)}
\NormalTok{B}
\end{Highlighting}
\end{Shaded}

  En esta opción, los vectores \texttt{a} y \texttt{b} se combinan por
  filas para crear una matriz \texttt{B}. Los elementos de \texttt{a}
  forman la primera fila y los elementos de \texttt{b} forman la segunda
  fila.
\item
  Creamos una matriz a partir de los elementos de vector \texttt{a}
  utilizando la función \texttt{matrix()}:

\begin{Shaded}
\begin{Highlighting}[]
\NormalTok{A }\OtherTok{\textless{}{-}} \FunctionTok{matrix}\NormalTok{(a, }\AttributeTok{ncol =} \DecValTok{2}\NormalTok{, }\AttributeTok{nrow =} \DecValTok{2}\NormalTok{)}
\NormalTok{A}
\end{Highlighting}
\end{Shaded}

  Aquí se utiliza la función \texttt{matrix()} para crear una matriz
  \texttt{A} a partir de los elementos del vector \texttt{a}. Se
  especifica que la matriz tendrá 2 columnas y 2 filas. Los argumentos
  nrow y ncol indican el número de filas y el número de columnas de que
  consta la matriz resultante.
\item
  Para 4 elementos y ncol =2 la matriz sólo puede tener 2 filas. Por lo
  tanto no es necesario especificar ambos argumentos

\begin{Shaded}
\begin{Highlighting}[]
\NormalTok{A }\OtherTok{\textless{}{-}} \FunctionTok{matrix}\NormalTok{(a, }\AttributeTok{ncol =} \DecValTok{2}\NormalTok{)}
\NormalTok{A}
\end{Highlighting}
\end{Shaded}

  En esta variante, se crea una matriz \texttt{A} con 2 columnas y se
  ajusta automáticamente el número de filas según la longitud del vector
  \texttt{a}.
\item
  Por defecto la matriz se rellena columna a columna (R trata
  internamente un objeto matriz como vector columna). si la matriz debe
  rellenarse fila a fila se requiere el argumento
  \texttt{byrow\ =\ TRUE}

\begin{Shaded}
\begin{Highlighting}[]
\NormalTok{B }\OtherTok{\textless{}{-}} \FunctionTok{matrix}\NormalTok{(a, }\AttributeTok{ncol =} \DecValTok{2}\NormalTok{, }\AttributeTok{byrow =} \ConstantTok{TRUE}\NormalTok{)}
\NormalTok{B}
\end{Highlighting}
\end{Shaded}

  En esta opción, se crea una matriz \texttt{B} con 2 columnas y se
  especifica que los elementos del vector \texttt{a} se distribuirán por
  filas \texttt{byrow\ =\ TRUE}, es decir, los primeros elementos de
  \texttt{a} formarán la primera fila, los siguientes elementos formarán
  la segunda fila, y así sucesivamente.
\end{enumerate}

\textbf{Acciones con matrices}

\begin{enumerate}
\def\labelenumi{\arabic{enumi}.}
\item
  Verificamos el número de filas de la matriz \texttt{A} utilizando la
  función \texttt{nrow()}:

\begin{Shaded}
\begin{Highlighting}[]
\FunctionTok{nrow}\NormalTok{(A)}
\end{Highlighting}
\end{Shaded}

  Esta línea de código devuelve el número de filas de la matriz
  \texttt{A}.
\item
  Verificamos el número de columnas de la matriz \texttt{A} utilizando
  la función \texttt{ncol()}:

\begin{Shaded}
\begin{Highlighting}[]
\FunctionTok{ncol}\NormalTok{(A)}
\end{Highlighting}
\end{Shaded}

  Aquí se obtiene el número de columnas de la matriz \texttt{A}.
\item
  Verificamos la dimensión (número de filas y columnas) de la matriz
  \texttt{A} utilizando la función \texttt{dim()}:

\begin{Shaded}
\begin{Highlighting}[]
\FunctionTok{dim}\NormalTok{(A)}
\end{Highlighting}
\end{Shaded}

  Esta línea de código devuelve la dimensión de la matriz \texttt{A} en
  formato \texttt{{[}nrow,\ ncol{]}}.
\item
  Combinamos dos matrices \texttt{A} por columnas utilizando la función
  \texttt{cbind()} y almacenamos el resultado en \texttt{D.wide}:

\begin{Shaded}
\begin{Highlighting}[]
\NormalTok{D.wide }\OtherTok{\textless{}{-}} \FunctionTok{cbind}\NormalTok{(A, A)}
\NormalTok{D.wide}
\end{Highlighting}
\end{Shaded}

  En esta línea se crea una nueva matriz \texttt{D.wide} que combina las
  matrices \texttt{A} y \texttt{A} por columnas.
\item
  Combinamos dos matrices \texttt{A} por filas utilizando la función
  \texttt{rbind()} y almacenamos el resultado en \texttt{D.long}:

\begin{Shaded}
\begin{Highlighting}[]
\NormalTok{D.long }\OtherTok{\textless{}{-}} \FunctionTok{rbind}\NormalTok{(A, A)}
\NormalTok{D.long}
\end{Highlighting}
\end{Shaded}

  Aquí se crea una nueva matriz \texttt{D.long} que combina las matrices
  \texttt{A} y \texttt{A} por filas.
\item
  Combinamos las matrices \texttt{D.wide} y \texttt{D.long} por columnas
  utilizando la función \texttt{cbind()} y almacenamos el resultado en
  \texttt{D}:

\begin{Shaded}
\begin{Highlighting}[]
\CommentTok{\# D \textless{}{-} cbind(D.wide, D.long)}
\end{Highlighting}
\end{Shaded}

  En esta línea se crea una nueva matriz \texttt{D} que combina las
  matrices \texttt{D.wide} y \texttt{D.long} por columnas.
\end{enumerate}

\textbf{Operaciones aritméticas con matrices}

\begin{enumerate}
\def\labelenumi{\arabic{enumi}.}
\item
  Suma de la matriz \texttt{B} consigo misma utilizando el operador
  \texttt{+}:

\begin{Shaded}
\begin{Highlighting}[]
\NormalTok{B }\SpecialCharTok{+}\NormalTok{ B}
\end{Highlighting}
\end{Shaded}

  Esta línea de código realiza la suma de la matriz \texttt{B} con ella
  misma.
\item
  Multiplicación escalar de la matriz \texttt{B} por 2 utilizando el
  operador \texttt{*}:

\begin{Shaded}
\begin{Highlighting}[]
\NormalTok{B }\SpecialCharTok{*} \DecValTok{2}
\end{Highlighting}
\end{Shaded}

  Aquí se realiza la multiplicación de cada elemento de la matriz
  \texttt{B} por 2.
\item
  Multiplicación elemento a elemento de la matriz \texttt{B} consigo
  misma y almacenar el resultado en \texttt{a}:

\begin{Shaded}
\begin{Highlighting}[]
\NormalTok{a }\OtherTok{\textless{}{-}}\NormalTok{ B }\SpecialCharTok{*}\NormalTok{ B}
\NormalTok{a}
\end{Highlighting}
\end{Shaded}

  En esta línea se realiza la multiplicación elemento a elemento de la
  matriz \texttt{B} con ella misma, y el resultado se almacena en la
  matriz \texttt{a}.
\item
  Multiplicación de matrices utilizando el operador \texttt{\%*\%}:

\begin{Shaded}
\begin{Highlighting}[]
\NormalTok{C }\OtherTok{\textless{}{-}}\NormalTok{ B }\SpecialCharTok{\%*\%}\NormalTok{ B}
\NormalTok{C}
\end{Highlighting}
\end{Shaded}

  Aquí se realiza la multiplicación de matrices entre la matriz
  \texttt{B} y ella misma, y el resultado se almacena en la matriz
  \texttt{C}.
\end{enumerate}

\textbf{Otras operaciones con matrices:}

\begin{enumerate}
\def\labelenumi{\arabic{enumi}.}
\item
  Transposición de la matriz \texttt{D.wide} utilizando la función
  \texttt{t()}:

\begin{Shaded}
\begin{Highlighting}[]
\FunctionTok{t}\NormalTok{(D.wide)}
\end{Highlighting}
\end{Shaded}

  Esta línea de código transpone la matriz \texttt{D.wide}, es decir,
  intercambia las filas por columnas y viceversa.
\item
  Cálculo del determinante de la matriz \texttt{B} utilizando la función
  \texttt{det()}:

\begin{Shaded}
\begin{Highlighting}[]
\FunctionTok{det}\NormalTok{(B)}
\end{Highlighting}
\end{Shaded}

  Aquí se calcula el determinante de la matriz \texttt{B}.
\item
  Cálculo de la inversa de la matriz \texttt{B} utilizando la función
  \texttt{solve()} (solo si el determinante es diferente de 0):

\begin{Shaded}
\begin{Highlighting}[]
\FunctionTok{solve}\NormalTok{(B)}
\end{Highlighting}
\end{Shaded}

  En esta línea se calcula la inversa de la matriz \texttt{B}, siempre y
  cuando el determinante sea diferente de 0.
\item
  Cálculo de los valores propios (eigenvalues) de una matriz cuadrada y
  simétrica utilizando la función \texttt{eigen()}:

\begin{Shaded}
\begin{Highlighting}[]
\FunctionTok{eigen}\NormalTok{(B)}
\end{Highlighting}
\end{Shaded}

  Aquí se calculan los valores propios de la matriz \texttt{B}. Esta
  operación solo es aplicable a matrices cuadradas y simétricas.
\end{enumerate}

\subsubsection{4. Data frame}\label{data-frame}

Un data frame es una estructura similar a una matriz, pero más flexible.
Puede contener columnas con diferentes tipos de datos, lo que lo hace
ideal para almacenar conjuntos de datos heterogéneos. Los data frames
son muy utilizados en el análisis de datos, ya que nos permiten
manipular y explorar datos de manera eficiente. Podemos realizar
operaciones de filtrado, selección y transformación en los data frames
para obtener información significativa.

\textbf{Creación del data frame:}

\begin{enumerate}
\def\labelenumi{\arabic{enumi}.}
\item
  Creamos vectores con diferentes tipos de datos, como números decimales
  (\texttt{dbl}), números enteros (\texttt{int}), valores lógicos
  (\texttt{lgl}) y caracteres (\texttt{chr}):

\begin{Shaded}
\begin{Highlighting}[]
\NormalTok{dbl }\OtherTok{\textless{}{-}} \FunctionTok{c}\NormalTok{(}\FloatTok{0.5}\NormalTok{, }\FloatTok{0.6}\NormalTok{, }\FloatTok{0.25}\NormalTok{, }\FloatTok{1.2}\NormalTok{, }\FloatTok{0.333}\NormalTok{) }\CommentTok{\# números decimales (double)}
\NormalTok{int }\OtherTok{\textless{}{-}} \FunctionTok{c}\NormalTok{(}\DecValTok{9}\NormalTok{L, }\DecValTok{10}\NormalTok{L, }\DecValTok{11}\NormalTok{L, }\DecValTok{12}\NormalTok{L, }\DecValTok{13}\NormalTok{L) }\CommentTok{\# números enteros (integer)}
\NormalTok{lgl }\OtherTok{\textless{}{-}} \FunctionTok{c}\NormalTok{(}\ConstantTok{TRUE}\NormalTok{, }\ConstantTok{FALSE}\NormalTok{, }\ConstantTok{FALSE}\NormalTok{, }\ConstantTok{TRUE}\NormalTok{, }\ConstantTok{TRUE}\NormalTok{) }\CommentTok{\# valores lógicos (logical)}
\NormalTok{chr }\OtherTok{\textless{}{-}} \FunctionTok{c}\NormalTok{(}\StringTok{"a"}\NormalTok{, }\StringTok{"b"}\NormalTok{, }\StringTok{"c"}\NormalTok{, }\StringTok{"d"}\NormalTok{, }\StringTok{"e"}\NormalTok{) }\CommentTok{\# caracteres (character)}
\end{Highlighting}
\end{Shaded}

  Cada vector tiene elementos que representan valores de su respectivo
  tipo de dato.
\item
  Utilizamos la función \texttt{data.frame()} para combinar los vectores
  en un data frame llamado \texttt{df}:

\begin{Shaded}
\begin{Highlighting}[]
\NormalTok{df }\OtherTok{\textless{}{-}} \FunctionTok{data.frame}\NormalTok{(dbl, int, lgl, chr)}
\end{Highlighting}
\end{Shaded}

  El data frame \texttt{df} se crea utilizando los vectores
  \texttt{dbl}, \texttt{int}, \texttt{lgl} y \texttt{chr} como columnas.
\item
  Mostamos el contenido del data frame en la consola:

\begin{Shaded}
\begin{Highlighting}[]
\NormalTok{df}
\end{Highlighting}
\end{Shaded}

  Esto imprime el contenido del data frame \texttt{df}.
\end{enumerate}

\textbf{Acciones con data frames:}

\begin{enumerate}
\def\labelenumi{\arabic{enumi}.}
\item
  Verificamos el número de filas del data frame utilizando la función
  \texttt{nrow()}:

\begin{Shaded}
\begin{Highlighting}[]
\FunctionTok{nrow}\NormalTok{(df)}
\end{Highlighting}
\end{Shaded}

  Esta línea de código devuelve el número de filas en el data frame
  \texttt{df}.
\item
  Verificamos el número de columnas del data frame utilizando la función
  \texttt{ncol()}:

\begin{Shaded}
\begin{Highlighting}[]
\FunctionTok{ncol}\NormalTok{(df)}
\end{Highlighting}
\end{Shaded}

  Aquí se obtiene el número de columnas en el data frame \texttt{df}.
\item
  Verificamos la dimensión (número de filas y columnas) del data frame
  utilizando la función \texttt{dim()}:

\begin{Shaded}
\begin{Highlighting}[]
\FunctionTok{dim}\NormalTok{(df)}
\end{Highlighting}
\end{Shaded}

  Esta línea de código devuelve la dimensión del data frame \texttt{df}
  en formato \texttt{{[}nrow,\ ncol{]}}, es decir, el número de filas y
  columnas que tiene el data frame.
\end{enumerate}

\subsubsection{5. Lista}\label{lista}

Una lista es una estructura de datos genérica que puede contener
diferentes objetos, como vectores, matrices, data frames o incluso otras
listas. A diferencia de las otras estructuras, las listas no tienen
restricciones en cuanto a los tipos de datos o la longitud de los
componentes individuales. Las listas son muy flexibles y se utilizan
cuando necesitamos almacenar objetos de diferentes tipos o estructuras
complejas.

\textbf{Creación de la lista}

\begin{enumerate}
\def\labelenumi{\arabic{enumi}.}
\item
  Creamos una variable \texttt{a} que contiene un \textbf{escalar} de
  tipo entero (\texttt{1L}):

\begin{Shaded}
\begin{Highlighting}[]
\NormalTok{a }\OtherTok{\textless{}{-}} \DecValTok{1}\NormalTok{L}
\end{Highlighting}
\end{Shaded}
\item
  Creamos un \textbf{vector numérico} \texttt{dbl} con 5 elementos:

\begin{Shaded}
\begin{Highlighting}[]
\NormalTok{dbl }\OtherTok{\textless{}{-}} \FunctionTok{c}\NormalTok{(}\FloatTok{0.5}\NormalTok{, }\FloatTok{0.6}\NormalTok{, }\FloatTok{0.25}\NormalTok{, }\FloatTok{1.2}\NormalTok{, }\FloatTok{0.333}\NormalTok{)}
\end{Highlighting}
\end{Shaded}
\item
  Creamos un \textbf{vector de caracteres} \texttt{chr} con 3 elementos:

\begin{Shaded}
\begin{Highlighting}[]
\NormalTok{chr }\OtherTok{\textless{}{-}} \FunctionTok{c}\NormalTok{(}\StringTok{"a"}\NormalTok{, }\StringTok{"b"}\NormalTok{, }\StringTok{"c"}\NormalTok{)}
\end{Highlighting}
\end{Shaded}
\item
  Creamos un vector \texttt{v} con 4 elementos de tipo numérico:

\begin{Shaded}
\begin{Highlighting}[]
\NormalTok{v }\OtherTok{\textless{}{-}} \FunctionTok{c}\NormalTok{(}\FloatTok{1.1}\NormalTok{, }\FloatTok{1.2}\NormalTok{, }\FloatTok{1.3}\NormalTok{, }\FloatTok{1.4}\NormalTok{)}
\end{Highlighting}
\end{Shaded}
\item
  Creamos una matriz \texttt{mat} de tamaño 2x2 a partir del vector
  \texttt{v}:

\begin{Shaded}
\begin{Highlighting}[]
\NormalTok{mat }\OtherTok{\textless{}{-}} \FunctionTok{matrix}\NormalTok{(v, }\AttributeTok{ncol =} \DecValTok{2}\NormalTok{)}
\end{Highlighting}
\end{Shaded}

  La matriz \texttt{mat} tiene 2 columnas y los elementos del vector
  \texttt{v} se llenan por columnas.
\item
  Creamos una lista \texttt{l} que contiene los elementos \texttt{a},
  \texttt{dbl}, \texttt{chr} y \texttt{mat}:

\begin{Shaded}
\begin{Highlighting}[]
\NormalTok{l }\OtherTok{\textless{}{-}} \FunctionTok{list}\NormalTok{(a, dbl, chr, mat)}
\end{Highlighting}
\end{Shaded}

  La lista \texttt{l} contiene estos elementos en ese orden.
\item
  Finalmente, visualizamos el contenido de la lista en la consola:

\begin{Shaded}
\begin{Highlighting}[]
\NormalTok{l}
\end{Highlighting}
\end{Shaded}

  Esto imprime el contenido de la lista \texttt{l}.
\end{enumerate}

\begin{quote}
Es importante comprender estas estructuras de datos en R, ya que nos
permiten organizar y manipular la información de manera efectiva. Cada
estructura tiene sus propias características y funciones asociadas que
nos facilitan el trabajo con los datos en la programación.
\end{quote}

\section{Publicaciones Similares}\label{publicaciones-similares}

Si te interesó este artículo, te recomendamos que explores otros blogs y
recursos relacionados que pueden ampliar tus conocimientos. Aquí te dejo
algunas sugerencias:

\begin{enumerate}
\def\labelenumi{\arabic{enumi}.}
\tightlist
\item
  \href{https://achalmaedison.netlify.app/programacion-software/r/2020-06-10-011-instalacion-de-r/index.pdf}{\faIcon{file-pdf}}
  \href{https://achalmaedison.netlify.app/programacion-software/r/2020-06-10-011-instalacion-de-r}{011
  Instalacion De R}
\item
  \href{https://achalmaedison.netlify.app/programacion-software/r/2020-06-10-012-que-ofrece-r/index.pdf}{\faIcon{file-pdf}}
  \href{https://achalmaedison.netlify.app/programacion-software/r/2020-06-10-012-que-ofrece-r}{012
  Que Ofrece R}
\item
  \href{https://achalmaedison.netlify.app/programacion-software/r/2020-06-10-013-lo-que-debemos-saber-de-r/index.pdf}{\faIcon{file-pdf}}
  \href{https://achalmaedison.netlify.app/programacion-software/r/2020-06-10-013-lo-que-debemos-saber-de-r}{013
  Lo Que Debemos Saber De R}
\item
  \href{https://achalmaedison.netlify.app/programacion-software/r/2021-03-027-01-introduccion-al-programa/index.pdf}{\faIcon{file-pdf}}
  \href{https://achalmaedison.netlify.app/programacion-software/r/2021-03-027-01-introduccion-al-programa}{2021
  03 027 01 Introduccion Al Programa}
\item
  \href{https://achalmaedison.netlify.app/programacion-software/r/2021-04-05-02-manipulacion-de-datos/index.pdf}{\faIcon{file-pdf}}
  \href{https://achalmaedison.netlify.app/programacion-software/r/2021-04-05-02-manipulacion-de-datos}{02
  Manipulacion De Datos}
\item
  \href{https://achalmaedison.netlify.app/programacion-software/r/2021-04-12-03-visualizacion-de-datos/index.pdf}{\faIcon{file-pdf}}
  \href{https://achalmaedison.netlify.app/programacion-software/r/2021-04-12-03-visualizacion-de-datos}{03
  Visualizacion De Datos}
\item
  \href{https://achalmaedison.netlify.app/programacion-software/r/2022-11-07-04-modelo-de-machine-learning-i-analisis-exploratorio/index.pdf}{\faIcon{file-pdf}}
  \href{https://achalmaedison.netlify.app/programacion-software/r/2022-11-07-04-modelo-de-machine-learning-i-analisis-exploratorio}{04
  Modelo De Machine Learning I Analisis Exploratorio}
\item
  \href{https://achalmaedison.netlify.app/programacion-software/r/2022-11-14-05-modelo-de-machine-learning-ii-modelo-de-clasificacion/index.pdf}{\faIcon{file-pdf}}
  \href{https://achalmaedison.netlify.app/programacion-software/r/2022-11-14-05-modelo-de-machine-learning-ii-modelo-de-clasificacion}{05
  Modelo De Machine Learning Ii Modelo De Clasificacion}
\item
  \href{https://achalmaedison.netlify.app/programacion-software/r/2022-11-21-06-modelo-de-machine-learning-iii-modelo-de-regresion/index.pdf}{\faIcon{file-pdf}}
  \href{https://achalmaedison.netlify.app/programacion-software/r/2022-11-21-06-modelo-de-machine-learning-iii-modelo-de-regresion}{06
  Modelo De Machine Learning Iii Modelo De Regresion}
\item
  \href{https://achalmaedison.netlify.app/programacion-software/r/2022-11-28-07-modelo-de-machine-learning-iv-tex-mining/index.pdf}{\faIcon{file-pdf}}
  \href{https://achalmaedison.netlify.app/programacion-software/r/2022-11-28-07-modelo-de-machine-learning-iv-tex-mining}{07
  Modelo De Machine Learning Iv Tex Mining}
\end{enumerate}

Esperamos que encuentres estas publicaciones igualmente interesantes y
útiles. ¡Disfruta de la lectura!






\end{document}
