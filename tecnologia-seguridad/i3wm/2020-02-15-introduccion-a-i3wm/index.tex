\documentclass[
  jou,
  floatsintext,
  longtable,
  a4paper,
  nolmodern,
  notxfonts,
  notimes,
  colorlinks=true,linkcolor=blue,citecolor=blue,urlcolor=blue]{apa7}

\usepackage{amsmath}
\usepackage{amssymb}



\usepackage[bidi=default]{babel}
\babelprovide[main,import]{spanish}
\StartBabelCommands{spanish}{captions} [unicode, fontenc=TU EU1 EU2, charset=utf8] \SetString{\keywordname}{Palabras
Claves}
\EndBabelCommands


% get rid of language-specific shorthands (see #6817):
\let\LanguageShortHands\languageshorthands
\def\languageshorthands#1{}

\RequirePackage{longtable}
\RequirePackage{threeparttablex}

\makeatletter
\renewcommand{\paragraph}{\@startsection{paragraph}{4}{\parindent}%
	{0\baselineskip \@plus 0.2ex \@minus 0.2ex}%
	{-.5em}%
	{\normalfont\normalsize\bfseries\typesectitle}}

\renewcommand{\subparagraph}[1]{\@startsection{subparagraph}{5}{0.5em}%
	{0\baselineskip \@plus 0.2ex \@minus 0.2ex}%
	{-\z@\relax}%
	{\normalfont\normalsize\bfseries\itshape\hspace{\parindent}{#1}\textit{\addperi}}{\relax}}
\makeatother




\usepackage{longtable, booktabs, multirow, multicol, colortbl, hhline, caption, array, float, xpatch}
\usepackage{subcaption}
\renewcommand\thesubfigure{\Alph{subfigure}}
\setcounter{topnumber}{2}
\setcounter{bottomnumber}{2}
\setcounter{totalnumber}{4}
\renewcommand{\topfraction}{0.85}
\renewcommand{\bottomfraction}{0.85}
\renewcommand{\textfraction}{0.15}
\renewcommand{\floatpagefraction}{0.7}

\usepackage{tcolorbox}
\tcbuselibrary{listings,theorems, breakable, skins}
\usepackage{fontawesome5}

\definecolor{quarto-callout-color}{HTML}{909090}
\definecolor{quarto-callout-note-color}{HTML}{0758E5}
\definecolor{quarto-callout-important-color}{HTML}{CC1914}
\definecolor{quarto-callout-warning-color}{HTML}{EB9113}
\definecolor{quarto-callout-tip-color}{HTML}{00A047}
\definecolor{quarto-callout-caution-color}{HTML}{FC5300}
\definecolor{quarto-callout-color-frame}{HTML}{ACACAC}
\definecolor{quarto-callout-note-color-frame}{HTML}{4582EC}
\definecolor{quarto-callout-important-color-frame}{HTML}{D9534F}
\definecolor{quarto-callout-warning-color-frame}{HTML}{F0AD4E}
\definecolor{quarto-callout-tip-color-frame}{HTML}{02B875}
\definecolor{quarto-callout-caution-color-frame}{HTML}{FD7E14}

%\newlength\Oldarrayrulewidth
%\newlength\Oldtabcolsep


\usepackage{hyperref}




\providecommand{\tightlist}{%
  \setlength{\itemsep}{0pt}\setlength{\parskip}{0pt}}
\usepackage{longtable,booktabs,array}
\usepackage{calc} % for calculating minipage widths
% Correct order of tables after \paragraph or \subparagraph
\usepackage{etoolbox}
\makeatletter
\patchcmd\longtable{\par}{\if@noskipsec\mbox{}\fi\par}{}{}
\makeatother
% Allow footnotes in longtable head/foot
\IfFileExists{footnotehyper.sty}{\usepackage{footnotehyper}}{\usepackage{footnote}}
\makesavenoteenv{longtable}

\usepackage{graphicx}
\makeatletter
\newsavebox\pandoc@box
\newcommand*\pandocbounded[1]{% scales image to fit in text height/width
  \sbox\pandoc@box{#1}%
  \Gscale@div\@tempa{\textheight}{\dimexpr\ht\pandoc@box+\dp\pandoc@box\relax}%
  \Gscale@div\@tempb{\linewidth}{\wd\pandoc@box}%
  \ifdim\@tempb\p@<\@tempa\p@\let\@tempa\@tempb\fi% select the smaller of both
  \ifdim\@tempa\p@<\p@\scalebox{\@tempa}{\usebox\pandoc@box}%
  \else\usebox{\pandoc@box}%
  \fi%
}
% Set default figure placement to htbp
\def\fps@figure{htbp}
\makeatother







\usepackage{newtx}

\defaultfontfeatures{Scale=MatchLowercase}
\defaultfontfeatures[\rmfamily]{Ligatures=TeX,Scale=1}





\title{Editar: Editar}


\shorttitle{Editar}


\usepackage{etoolbox}



\ccoppy{\textcopyright~2020}



\author{Edison Achalma}



\affiliation{
{Escuela Profesional de Economía, Universidad Nacional de San Cristóbal
de Huamanga}}




\leftheader{Achalma}

\date{2020-02-15}


\abstract{Descubre cómo crear tu propio sitio web estático con Blogdown,
una herramienta poderosa que combina R Markdown y Hugo. Aprende a usar
comandos sencillos para personalizar, construir y alojar tu sitio web de
manera fácil y rápida. ¡Comienza tu proyecto web hoy mismo! }

\keywords{keyword1, keyword2}

\authornote{\par{\addORCIDlink{Edison Achalma}{0000-0001-6996-3364}} 
\par{ }
\par{   El autor no tiene conflictos de interés que revelar.    Los
roles de autor se clasificaron utilizando la taxonomía de roles de
colaborador (CRediT; https://credit.niso.org/) de la siguiente
manera:  Edison Achalma:   conceptualización, redacción}
\par{La correspondencia relativa a este artículo debe dirigirse a Edison
Achalma, Email: \href{mailto:elmer.achalma.09@unsch.edu.pe}{elmer.achalma.09@unsch.edu.pe}}
}

\usepackage{pbalance} 
\usepackage{float}
\makeatletter
\let\oldtpt\ThreePartTable
\let\endoldtpt\endThreePartTable
\def\ThreePartTable{\@ifnextchar[\ThreePartTable@i \ThreePartTable@ii}
\def\ThreePartTable@i[#1]{\begin{figure}[!htbp]
\onecolumn
\begin{minipage}{0.5\textwidth}
\oldtpt[#1]
}
\def\ThreePartTable@ii{\begin{figure}[!htbp]
\onecolumn
\begin{minipage}{0.5\textwidth}
\oldtpt
}
\def\endThreePartTable{
\endoldtpt
\end{minipage}
\twocolumn
\end{figure}}
\makeatother


\makeatletter
\let\endoldlt\endlongtable		
\def\endlongtable{
\hline
\endoldlt}
\makeatother

\newenvironment{twocolumntable}% environment name
{% begin code
\begin{table*}[!htbp]%
\onecolumn%
}%
{%
\twocolumn%
\end{table*}%
}% end code

\urlstyle{same}



\makeatletter
\@ifpackageloaded{caption}{}{\usepackage{caption}}
\AtBeginDocument{%
\ifdefined\contentsname
  \renewcommand*\contentsname{Tabla de contenidos}
\else
  \newcommand\contentsname{Tabla de contenidos}
\fi
\ifdefined\listfigurename
  \renewcommand*\listfigurename{Listado de Figuras}
\else
  \newcommand\listfigurename{Listado de Figuras}
\fi
\ifdefined\listtablename
  \renewcommand*\listtablename{Listado de Tablas}
\else
  \newcommand\listtablename{Listado de Tablas}
\fi
\ifdefined\figurename
  \renewcommand*\figurename{Figura}
\else
  \newcommand\figurename{Figura}
\fi
\ifdefined\tablename
  \renewcommand*\tablename{Tabla}
\else
  \newcommand\tablename{Tabla}
\fi
}
\@ifpackageloaded{float}{}{\usepackage{float}}
\floatstyle{ruled}
\@ifundefined{c@chapter}{\newfloat{codelisting}{h}{lop}}{\newfloat{codelisting}{h}{lop}[chapter]}
\floatname{codelisting}{Listado}
\newcommand*\listoflistings{\listof{codelisting}{Listado de Listados}}
\makeatother
\makeatletter
\makeatother
\makeatletter
\@ifpackageloaded{caption}{}{\usepackage{caption}}
\@ifpackageloaded{subcaption}{}{\usepackage{subcaption}}
\makeatother
\makeatletter
\@ifpackageloaded{fontawesome5}{}{\usepackage{fontawesome5}}
\makeatother

% From https://tex.stackexchange.com/a/645996/211326
%%% apa7 doesn't want to add appendix section titles in the toc
%%% let's make it do it
\makeatletter
\xpatchcmd{\appendix}
  {\par}
  {\addcontentsline{toc}{section}{\@currentlabelname}\par}
  {}{}
\makeatother

%% Disable longtable counter
%% https://tex.stackexchange.com/a/248395/211326

\usepackage{etoolbox}

\makeatletter
\patchcmd{\LT@caption}
  {\bgroup}
  {\bgroup\global\LTpatch@captiontrue}
  {}{}
\patchcmd{\longtable}
  {\par}
  {\par\global\LTpatch@captionfalse}
  {}{}
\apptocmd{\endlongtable}
  {\ifLTpatch@caption\else\addtocounter{table}{-1}\fi}
  {}{}
\newif\ifLTpatch@caption
\makeatother

\begin{document}

\maketitle

\hypertarget{toc}{}
\tableofcontents
\newpage
\section[Introduction]{Editar}

\setcounter{secnumdepth}{-\maxdimen} % remove section numbering

\setlength\LTleft{0pt}


\section{¿Qué es i3wm?}\label{quuxe9-es-i3wm}

\subsection{Definición de i3wm}\label{definiciuxf3n-de-i3wm}

\subsection{Características principales de
i3wm}\label{caracteruxedsticas-principales-de-i3wm}

\section{Ventajas de utilizar i3wm}\label{ventajas-de-utilizar-i3wm}

\subsection{Eficiencia y rendimiento}\label{eficiencia-y-rendimiento}

\subsection{Enfoque minimalista}\label{enfoque-minimalista}

\subsection{Personalización y
flexibilidad}\label{personalizaciuxf3n-y-flexibilidad}

\subsection{Uso eficiente del espacio en el
escritorio}\label{uso-eficiente-del-espacio-en-el-escritorio}

\section{Conceptos fundamentales de
i3wm}\label{conceptos-fundamentales-de-i3wm}

\subsection{Gestión de ventanas en mosaico (tiling window
management)}\label{gestiuxf3n-de-ventanas-en-mosaico-tiling-window-management}

\subsection{Espacios de trabajo (workspaces) 3.3. Modos y barras de
estado (modes and status
bars)}\label{espacios-de-trabajo-workspaces-3.3.-modos-y-barras-de-estado-modes-and-status-bars}

\subsection{Atajos de teclado y comandos
básicos}\label{atajos-de-teclado-y-comandos-buxe1sicos}

\section{Instalación de i3wm}\label{instalaciuxf3n-de-i3wm}

\subsection{Requisitos del sistema}\label{requisitos-del-sistema}

\subsection{Instalación en distribuciones populares (ejemplo: Ubuntu,
Arch
Linux)}\label{instalaciuxf3n-en-distribuciones-populares-ejemplo-ubuntu-arch-linux}

\section{Primeros pasos con i3wm}\label{primeros-pasos-con-i3wm}

\subsection{Iniciar sesión en i3wm}\label{iniciar-sesiuxf3n-en-i3wm}

\subsection{Orientación básica en el entorno de
i3wm}\label{orientaciuxf3n-buxe1sica-en-el-entorno-de-i3wm}

\subsection{Navegación entre ventanas y
workspaces}\label{navegaciuxf3n-entre-ventanas-y-workspaces}

\section{Recursos adicionales y
comunidad}\label{recursos-adicionales-y-comunidad}

\subsection{Documentación oficial y guías de
referencia}\label{documentaciuxf3n-oficial-y-guuxedas-de-referencia}

\subsection{Comunidades en línea y foros de
discusión}\label{comunidades-en-luxednea-y-foros-de-discusiuxf3n}

\subsection{Sitios web y blogs recomendados para aprender más sobre
i3wm}\label{sitios-web-y-blogs-recomendados-para-aprender-muxe1s-sobre-i3wm}

\section{Publicaciones Similares}\label{publicaciones-similares}

Si te interesó este artículo, te recomendamos que explores otros blogs y
recursos relacionados que pueden ampliar tus conocimientos. Aquí te dejo
algunas sugerencias:

\begin{enumerate}
\def\labelenumi{\arabic{enumi}.}
\tightlist
\item
  \href{https://achalmaedison.netlify.app/tecnologia-seguridad/i3wm/2020-02-15-introduccion-a-i3wm/index.pdf}{\faIcon{file-pdf}}
  \href{https://achalmaedison.netlify.app/tecnologia-seguridad/i3wm/2020-02-15-introduccion-a-i3wm}{Introduccion
  A I3wm}
\item
  \href{https://achalmaedison.netlify.app/tecnologia-seguridad/i3wm/2020-02-16-guia-de-instalacion-y-configuracion-de-i3wm/index.pdf}{\faIcon{file-pdf}}
  \href{https://achalmaedison.netlify.app/tecnologia-seguridad/i3wm/2020-02-16-guia-de-instalacion-y-configuracion-de-i3wm}{Guia
  De Instalacion Y Configuracion De I3wm}
\item
  \href{https://achalmaedison.netlify.app/tecnologia-seguridad/i3wm/2020-02-17-atajos-de-teclado-y-comandos-esenciales-de-i3wm/index.pdf}{\faIcon{file-pdf}}
  \href{https://achalmaedison.netlify.app/tecnologia-seguridad/i3wm/2020-02-17-atajos-de-teclado-y-comandos-esenciales-de-i3wm}{Atajos
  De Teclado Y Comandos Esenciales De I3wm}
\item
  \href{https://achalmaedison.netlify.app/tecnologia-seguridad/i3wm/2020-02-18-personalizando-de-i3wm/index.pdf}{\faIcon{file-pdf}}
  \href{https://achalmaedison.netlify.app/tecnologia-seguridad/i3wm/2020-02-18-personalizando-de-i3wm}{Personalizando
  De I3wm}
\item
  \href{https://achalmaedison.netlify.app/tecnologia-seguridad/i3wm/2020-02-18-trabajando-con-ventanas-en-i3wm/index.pdf}{\faIcon{file-pdf}}
  \href{https://achalmaedison.netlify.app/tecnologia-seguridad/i3wm/2020-02-18-trabajando-con-ventanas-en-i3wm}{Trabajando
  Con Ventanas En I3wm}
\item
  \href{https://achalmaedison.netlify.app/tecnologia-seguridad/i3wm/2020-02-19-aumenta-tu-productividad-con-i3wm/index.pdf}{\faIcon{file-pdf}}
  \href{https://achalmaedison.netlify.app/tecnologia-seguridad/i3wm/2020-02-19-aumenta-tu-productividad-con-i3wm}{Aumenta
  Tu Productividad Con I3wm}
\item
  \href{https://achalmaedison.netlify.app/tecnologia-seguridad/i3wm/2020-02-20-i3wm-vs-otros-gestores-de-ventanas/index.pdf}{\faIcon{file-pdf}}
  \href{https://achalmaedison.netlify.app/tecnologia-seguridad/i3wm/2020-02-20-i3wm-vs-otros-gestores-de-ventanas}{I3wm
  Vs Otros Gestores De Ventanas}
\item
  \href{https://achalmaedison.netlify.app/tecnologia-seguridad/i3wm/2020-02-21-integracion-de-i3wm-en-tu-entorno-de-trabajo/index.pdf}{\faIcon{file-pdf}}
  \href{https://achalmaedison.netlify.app/tecnologia-seguridad/i3wm/2020-02-21-integracion-de-i3wm-en-tu-entorno-de-trabajo}{Integracion
  De I3wm En Tu Entorno De Trabajo}
\item
  \href{https://achalmaedison.netlify.app/tecnologia-seguridad/i3wm/2020-02-22-casos-de-uso-avanzados-de-i3wm/index.pdf}{\faIcon{file-pdf}}
  \href{https://achalmaedison.netlify.app/tecnologia-seguridad/i3wm/2020-02-22-casos-de-uso-avanzados-de-i3wm}{Casos
  De Uso Avanzados De I3wm}
\item
  \href{https://achalmaedison.netlify.app/tecnologia-seguridad/i3wm/2020-02-23-solucion-de-problemas-comunes-en-i3wm/index.pdf}{\faIcon{file-pdf}}
  \href{https://achalmaedison.netlify.app/tecnologia-seguridad/i3wm/2020-02-23-solucion-de-problemas-comunes-en-i3wm}{Solucion
  De Problemas Comunes En I3wm}
\end{enumerate}

Esperamos que encuentres estas publicaciones igualmente interesantes y
útiles. ¡Disfruta de la lectura!






\end{document}
