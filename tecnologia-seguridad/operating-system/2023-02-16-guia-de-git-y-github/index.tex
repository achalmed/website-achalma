\documentclass[
  doc,
  floatsintext,
  longtable,
  a4paper,
  nolmodern,
  notxfonts,
  notimes,
  colorlinks=true,linkcolor=blue,citecolor=blue,urlcolor=blue]{apa7}

\usepackage{amsmath}
\usepackage{amssymb}

\geometry{inner=1in, outer=1in}
\fancyhfoffset[LE,RO]{0cm}


\usepackage[bidi=default]{babel}
\babelprovide[main,import]{spanish}
\StartBabelCommands{spanish}{captions} [unicode, fontenc=TU EU1 EU2, charset=utf8] \SetString{\keywordname}{Palabras
Claves}
\EndBabelCommands


% get rid of language-specific shorthands (see #6817):
\let\LanguageShortHands\languageshorthands
\def\languageshorthands#1{}

\RequirePackage{longtable}
\RequirePackage{threeparttablex}

\makeatletter
\renewcommand{\paragraph}{\@startsection{paragraph}{4}{\parindent}%
	{0\baselineskip \@plus 0.2ex \@minus 0.2ex}%
	{-.5em}%
	{\normalfont\normalsize\bfseries\typesectitle}}

\renewcommand{\subparagraph}[1]{\@startsection{subparagraph}{5}{0.5em}%
	{0\baselineskip \@plus 0.2ex \@minus 0.2ex}%
	{-\z@\relax}%
	{\normalfont\normalsize\bfseries\itshape\hspace{\parindent}{#1}\textit{\addperi}}{\relax}}
\makeatother




\usepackage{longtable, booktabs, multirow, multicol, colortbl, hhline, caption, array, float, xpatch}
\usepackage{subcaption}
\renewcommand\thesubfigure{\Alph{subfigure}}
\setcounter{topnumber}{2}
\setcounter{bottomnumber}{2}
\setcounter{totalnumber}{4}
\renewcommand{\topfraction}{0.85}
\renewcommand{\bottomfraction}{0.85}
\renewcommand{\textfraction}{0.15}
\renewcommand{\floatpagefraction}{0.7}

\usepackage{tcolorbox}
\tcbuselibrary{listings,theorems, breakable, skins}
\usepackage{fontawesome5}

\definecolor{quarto-callout-color}{HTML}{909090}
\definecolor{quarto-callout-note-color}{HTML}{0758E5}
\definecolor{quarto-callout-important-color}{HTML}{CC1914}
\definecolor{quarto-callout-warning-color}{HTML}{EB9113}
\definecolor{quarto-callout-tip-color}{HTML}{00A047}
\definecolor{quarto-callout-caution-color}{HTML}{FC5300}
\definecolor{quarto-callout-color-frame}{HTML}{ACACAC}
\definecolor{quarto-callout-note-color-frame}{HTML}{4582EC}
\definecolor{quarto-callout-important-color-frame}{HTML}{D9534F}
\definecolor{quarto-callout-warning-color-frame}{HTML}{F0AD4E}
\definecolor{quarto-callout-tip-color-frame}{HTML}{02B875}
\definecolor{quarto-callout-caution-color-frame}{HTML}{FD7E14}

%\newlength\Oldarrayrulewidth
%\newlength\Oldtabcolsep


\usepackage{hyperref}



\usepackage{color}
\usepackage{fancyvrb}
\newcommand{\VerbBar}{|}
\newcommand{\VERB}{\Verb[commandchars=\\\{\}]}
\DefineVerbatimEnvironment{Highlighting}{Verbatim}{commandchars=\\\{\}}
% Add ',fontsize=\small' for more characters per line
\usepackage{framed}
\definecolor{shadecolor}{RGB}{241,243,245}
\newenvironment{Shaded}{\begin{snugshade}}{\end{snugshade}}
\newcommand{\AlertTok}[1]{\textcolor[rgb]{0.68,0.00,0.00}{#1}}
\newcommand{\AnnotationTok}[1]{\textcolor[rgb]{0.37,0.37,0.37}{#1}}
\newcommand{\AttributeTok}[1]{\textcolor[rgb]{0.40,0.45,0.13}{#1}}
\newcommand{\BaseNTok}[1]{\textcolor[rgb]{0.68,0.00,0.00}{#1}}
\newcommand{\BuiltInTok}[1]{\textcolor[rgb]{0.00,0.23,0.31}{#1}}
\newcommand{\CharTok}[1]{\textcolor[rgb]{0.13,0.47,0.30}{#1}}
\newcommand{\CommentTok}[1]{\textcolor[rgb]{0.37,0.37,0.37}{#1}}
\newcommand{\CommentVarTok}[1]{\textcolor[rgb]{0.37,0.37,0.37}{\textit{#1}}}
\newcommand{\ConstantTok}[1]{\textcolor[rgb]{0.56,0.35,0.01}{#1}}
\newcommand{\ControlFlowTok}[1]{\textcolor[rgb]{0.00,0.23,0.31}{\textbf{#1}}}
\newcommand{\DataTypeTok}[1]{\textcolor[rgb]{0.68,0.00,0.00}{#1}}
\newcommand{\DecValTok}[1]{\textcolor[rgb]{0.68,0.00,0.00}{#1}}
\newcommand{\DocumentationTok}[1]{\textcolor[rgb]{0.37,0.37,0.37}{\textit{#1}}}
\newcommand{\ErrorTok}[1]{\textcolor[rgb]{0.68,0.00,0.00}{#1}}
\newcommand{\ExtensionTok}[1]{\textcolor[rgb]{0.00,0.23,0.31}{#1}}
\newcommand{\FloatTok}[1]{\textcolor[rgb]{0.68,0.00,0.00}{#1}}
\newcommand{\FunctionTok}[1]{\textcolor[rgb]{0.28,0.35,0.67}{#1}}
\newcommand{\ImportTok}[1]{\textcolor[rgb]{0.00,0.46,0.62}{#1}}
\newcommand{\InformationTok}[1]{\textcolor[rgb]{0.37,0.37,0.37}{#1}}
\newcommand{\KeywordTok}[1]{\textcolor[rgb]{0.00,0.23,0.31}{\textbf{#1}}}
\newcommand{\NormalTok}[1]{\textcolor[rgb]{0.00,0.23,0.31}{#1}}
\newcommand{\OperatorTok}[1]{\textcolor[rgb]{0.37,0.37,0.37}{#1}}
\newcommand{\OtherTok}[1]{\textcolor[rgb]{0.00,0.23,0.31}{#1}}
\newcommand{\PreprocessorTok}[1]{\textcolor[rgb]{0.68,0.00,0.00}{#1}}
\newcommand{\RegionMarkerTok}[1]{\textcolor[rgb]{0.00,0.23,0.31}{#1}}
\newcommand{\SpecialCharTok}[1]{\textcolor[rgb]{0.37,0.37,0.37}{#1}}
\newcommand{\SpecialStringTok}[1]{\textcolor[rgb]{0.13,0.47,0.30}{#1}}
\newcommand{\StringTok}[1]{\textcolor[rgb]{0.13,0.47,0.30}{#1}}
\newcommand{\VariableTok}[1]{\textcolor[rgb]{0.07,0.07,0.07}{#1}}
\newcommand{\VerbatimStringTok}[1]{\textcolor[rgb]{0.13,0.47,0.30}{#1}}
\newcommand{\WarningTok}[1]{\textcolor[rgb]{0.37,0.37,0.37}{\textit{#1}}}

\providecommand{\tightlist}{%
  \setlength{\itemsep}{0pt}\setlength{\parskip}{0pt}}
\usepackage{longtable,booktabs,array}
\usepackage{calc} % for calculating minipage widths
% Correct order of tables after \paragraph or \subparagraph
\usepackage{etoolbox}
\makeatletter
\patchcmd\longtable{\par}{\if@noskipsec\mbox{}\fi\par}{}{}
\makeatother
% Allow footnotes in longtable head/foot
\IfFileExists{footnotehyper.sty}{\usepackage{footnotehyper}}{\usepackage{footnote}}
\makesavenoteenv{longtable}

\usepackage{graphicx}
\makeatletter
\newsavebox\pandoc@box
\newcommand*\pandocbounded[1]{% scales image to fit in text height/width
  \sbox\pandoc@box{#1}%
  \Gscale@div\@tempa{\textheight}{\dimexpr\ht\pandoc@box+\dp\pandoc@box\relax}%
  \Gscale@div\@tempb{\linewidth}{\wd\pandoc@box}%
  \ifdim\@tempb\p@<\@tempa\p@\let\@tempa\@tempb\fi% select the smaller of both
  \ifdim\@tempa\p@<\p@\scalebox{\@tempa}{\usebox\pandoc@box}%
  \else\usebox{\pandoc@box}%
  \fi%
}
% Set default figure placement to htbp
\def\fps@figure{htbp}
\makeatother







\usepackage{newtx}

\defaultfontfeatures{Scale=MatchLowercase}
\defaultfontfeatures[\rmfamily]{Ligatures=TeX,Scale=1}





\title{Guía de Git Cómo trabajar en equipo en proyectos: Aprende a usar
Git para controlar versiones, colaborar con otros desarrolladores y
mantener tu código organizado.}


\shorttitle{Editar}


\usepackage{etoolbox}



\ccoppy{\textcopyright~2023}



\author{Edison Achalma}



\affiliation{
{Escuela Profesional de Economía, Universidad Nacional de San Cristóbal
de Huamanga}}




\leftheader{Achalma}

\date{2023-02-16}


\abstract{Primer parrafo de abstrac }

\keywords{keyword1, keyword2}

\authornote{\par{\addORCIDlink{Edison Achalma}{0000-0001-6996-3364}} 
\par{ }
\par{   El autor no tiene conflictos de interés que revelar.    Los
roles de autor se clasificaron utilizando la taxonomía de roles de
colaborador (CRediT; https://credit.niso.org/) de la siguiente
manera:  Edison Achalma:   conceptualización, redacción}
\par{La correspondencia relativa a este artículo debe dirigirse a Edison
Achalma, Email: \href{mailto:elmer.achalma.09@unsch.edu.pe}{elmer.achalma.09@unsch.edu.pe}}
}

\makeatletter
\let\endoldlt\endlongtable
\def\endlongtable{
\hline
\endoldlt
}
\makeatother

\urlstyle{same}



\makeatletter
\@ifpackageloaded{caption}{}{\usepackage{caption}}
\AtBeginDocument{%
\ifdefined\contentsname
  \renewcommand*\contentsname{Tabla de contenidos}
\else
  \newcommand\contentsname{Tabla de contenidos}
\fi
\ifdefined\listfigurename
  \renewcommand*\listfigurename{Listado de Figuras}
\else
  \newcommand\listfigurename{Listado de Figuras}
\fi
\ifdefined\listtablename
  \renewcommand*\listtablename{Listado de Tablas}
\else
  \newcommand\listtablename{Listado de Tablas}
\fi
\ifdefined\figurename
  \renewcommand*\figurename{Figura}
\else
  \newcommand\figurename{Figura}
\fi
\ifdefined\tablename
  \renewcommand*\tablename{Tabla}
\else
  \newcommand\tablename{Tabla}
\fi
}
\@ifpackageloaded{float}{}{\usepackage{float}}
\floatstyle{ruled}
\@ifundefined{c@chapter}{\newfloat{codelisting}{h}{lop}}{\newfloat{codelisting}{h}{lop}[chapter]}
\floatname{codelisting}{Listado}
\newcommand*\listoflistings{\listof{codelisting}{Listado de Listados}}
\makeatother
\makeatletter
\makeatother
\makeatletter
\@ifpackageloaded{caption}{}{\usepackage{caption}}
\@ifpackageloaded{subcaption}{}{\usepackage{subcaption}}
\makeatother
\makeatletter
\@ifpackageloaded{fontawesome5}{}{\usepackage{fontawesome5}}
\makeatother

% From https://tex.stackexchange.com/a/645996/211326
%%% apa7 doesn't want to add appendix section titles in the toc
%%% let's make it do it
\makeatletter
\xpatchcmd{\appendix}
  {\par}
  {\addcontentsline{toc}{section}{\@currentlabelname}\par}
  {}{}
\makeatother

%% Disable longtable counter
%% https://tex.stackexchange.com/a/248395/211326

\usepackage{etoolbox}

\makeatletter
\patchcmd{\LT@caption}
  {\bgroup}
  {\bgroup\global\LTpatch@captiontrue}
  {}{}
\patchcmd{\longtable}
  {\par}
  {\par\global\LTpatch@captionfalse}
  {}{}
\apptocmd{\endlongtable}
  {\ifLTpatch@caption\else\addtocounter{table}{-1}\fi}
  {}{}
\newif\ifLTpatch@caption
\makeatother

\begin{document}

\maketitle

\hypertarget{toc}{}
\tableofcontents
\newpage
\section[Introduction]{Guía de Git Cómo trabajar en equipo en proyectos}

\setcounter{secnumdepth}{-\maxdimen} % remove section numbering

\setlength\LTleft{0pt}


\section{Guía esencial de Git y
GitHub}\label{guuxeda-esencial-de-git-y-github}

Esta guía te introduce a los fundamentos de Git y GitHub, desde la
instalación hasta la gestión avanzada de proyectos. Ideal tanto para
principiantes como para quienes buscan perfeccionar sus habilidades en
control de versiones.

Git es un sistema de control de versiones (SCV) esencial para rastrear
cambios en el código, colaborar en equipo y experimentar con nuevas
ideas mediante ramas. Plataformas como GitHub potencian esta
colaboración al hospedar repositorios y facilitar el intercambio de
código.

\subsection{¿Cómo funciona Git?}\label{cuxf3mo-funciona-git}

Git organiza los proyectos en tres áreas principales: -
\textbf{Directorio de trabajo}: Donde editas tus archivos. -
\textbf{Área de preparación (staging)}: Donde preparas los cambios antes
de confirmarlos. - \textbf{Directorio Git}: Almacena las instantáneas
confirmadas de tu proyecto.

Disponible en Linux, Windows y macOS, Git tiene una curva de aprendizaje
inicial, pero su dominio abre un mundo de posibilidades para gestionar
proyectos eficientemente.

\subsection{Comandos básicos de Git}\label{comandos-buxe1sicos-de-git}

Aquí tienes los comandos fundamentales para empezar:

\begin{enumerate}
\def\labelenumi{\arabic{enumi}.}
\tightlist
\item
  \texttt{git\ init}: Inicia un nuevo repositorio en el directorio
  actual.
\item
  \texttt{git\ clone\ {[}url{]}}: Copia un repositorio existente a tu
  máquina.
\item
  \texttt{git\ add\ {[}file{]}}: Añade un archivo al área de
  preparación.
\item
  \texttt{git\ commit\ -m\ "mensaje"}: Guarda los cambios con un mensaje
  descriptivo.
\item
  \texttt{git\ status}: Muestra el estado actual del repositorio.
\item
  \texttt{git\ log}: Lista el historial de commits.
\item
  \texttt{git\ diff}: Compara cambios no confirmados con el último
  commit.
\item
  \texttt{git\ branch}: Muestra las ramas existentes.
\item
  \texttt{git\ checkout\ {[}branch{]}}: Cambia a una rama específica.
\item
  \texttt{git\ merge\ {[}branch{]}}: Fusiona una rama con la actual.
\item
  \texttt{git\ config\ -\/-global\ user.name\ "tu-nombre"}: Configura tu
  nombre de usuario.
\item
  \texttt{git\ config\ -\/-global\ user.email\ "tu-email@example.com"}:
  Configura tu correo.
\end{enumerate}

\subsection{Instalación de Git en
Ubuntu}\label{instalaciuxf3n-de-git-en-ubuntu}

\subsubsection{Método 1: Paquetes predeterminados (rápido y
estable)}\label{muxe9todo-1-paquetes-predeterminados-ruxe1pido-y-estable}

\begin{enumerate}
\def\labelenumi{\arabic{enumi}.}
\item
  Verifica si Git está instalado:

\begin{Shaded}
\begin{Highlighting}[]
\FunctionTok{git} \AttributeTok{{-}{-}version}
\end{Highlighting}
\end{Shaded}

  Ejemplo de salida: \texttt{git\ version\ 2.34.1}
\item
  Si no está instalado, actualiza e instala con APT:

\begin{Shaded}
\begin{Highlighting}[]
\FunctionTok{sudo}\NormalTok{ apt update}
\FunctionTok{sudo}\NormalTok{ apt install git}
\end{Highlighting}
\end{Shaded}
\item
  Configura tu identidad:

\begin{Shaded}
\begin{Highlighting}[]
\FunctionTok{git}\NormalTok{ config }\AttributeTok{{-}{-}global}\NormalTok{ user.name }\StringTok{"Tu Nombre"}
\FunctionTok{git}\NormalTok{ config }\AttributeTok{{-}{-}global}\NormalTok{ user.email }\StringTok{"tu.correo@example.com"}
\end{Highlighting}
\end{Shaded}
\item
  Verifica la configuración:

\begin{Shaded}
\begin{Highlighting}[]
\FunctionTok{git}\NormalTok{ config }\AttributeTok{{-}{-}list}
\end{Highlighting}
\end{Shaded}
\end{enumerate}

\subsubsection{Método 2: Desde la fuente (versión más
reciente)}\label{muxe9todo-2-desde-la-fuente-versiuxf3n-muxe1s-reciente}

\begin{enumerate}
\def\labelenumi{\arabic{enumi}.}
\item
  Instala las dependencias:

\begin{Shaded}
\begin{Highlighting}[]
\FunctionTok{sudo}\NormalTok{ apt update}
\FunctionTok{sudo}\NormalTok{ apt install libz{-}dev libssl{-}dev libcurl4{-}gnutls{-}dev libexpat1{-}dev gettext cmake gcc}
\end{Highlighting}
\end{Shaded}
\item
  Descarga y descomprime la versión deseada (ejemplo: 2.34.1):

\begin{Shaded}
\begin{Highlighting}[]
\FunctionTok{mkdir}\NormalTok{ tmp }\KeywordTok{\&\&} \BuiltInTok{cd}\NormalTok{ tmp}
\ExtensionTok{curl} \AttributeTok{{-}o}\NormalTok{ git.tar.gz https://mirrors.edge.kernel.org/pub/software/scm/git/git{-}2.34.1.tar.gz}
\FunctionTok{tar} \AttributeTok{{-}zxf}\NormalTok{ git.tar.gz}
\BuiltInTok{cd}\NormalTok{ git{-}}\PreprocessorTok{*}
\end{Highlighting}
\end{Shaded}
\item
  Compila e instala:

\begin{Shaded}
\begin{Highlighting}[]
\FunctionTok{make}\NormalTok{ prefix=/usr/local all}
\FunctionTok{sudo}\NormalTok{ make prefix=/usr/local install}
\BuiltInTok{exec}\NormalTok{ bash}
\end{Highlighting}
\end{Shaded}
\item
  Confirma la instalación:

\begin{Shaded}
\begin{Highlighting}[]
\FunctionTok{git} \AttributeTok{{-}{-}version}
\end{Highlighting}
\end{Shaded}
\end{enumerate}

\subsection{Configuración de claves SSH para
GitHub}\label{configuraciuxf3n-de-claves-ssh-para-github}

\subsubsection{Generar una clave SSH}\label{generar-una-clave-ssh}

\begin{enumerate}
\def\labelenumi{\arabic{enumi}.}
\item
  Verifica claves existentes:

\begin{Shaded}
\begin{Highlighting}[]
\FunctionTok{ls} \AttributeTok{{-}al}\NormalTok{ \textasciitilde{}/.ssh}
\end{Highlighting}
\end{Shaded}

  Si no hay claves, crea el directorio:
  \texttt{mkdir\ \textasciitilde{}/.ssh}.
\item
  Genera un par de claves:

\begin{Shaded}
\begin{Highlighting}[]
\FunctionTok{ssh{-}keygen} \AttributeTok{{-}t}\NormalTok{ rsa }\AttributeTok{{-}b}\NormalTok{ 4096 }\AttributeTok{{-}C} \StringTok{"tu.email@example.com"}
\end{Highlighting}
\end{Shaded}

  Acepta el nombre predeterminado y añade una contraseña (opcional).
\end{enumerate}

\subsubsection{Añadir la clave a
ssh-agent}\label{auxf1adir-la-clave-a-ssh-agent}

\begin{enumerate}
\def\labelenumi{\arabic{enumi}.}
\item
  Inicia el agente:

\begin{Shaded}
\begin{Highlighting}[]
\BuiltInTok{eval} \StringTok{"}\VariableTok{$(}\FunctionTok{ssh{-}agent} \AttributeTok{{-}s}\VariableTok{)}\StringTok{"}
\end{Highlighting}
\end{Shaded}
\item
  Añade la clave privada:

\begin{Shaded}
\begin{Highlighting}[]
\FunctionTok{ssh{-}add}\NormalTok{ \textasciitilde{}/.ssh/id\_rsa}
\end{Highlighting}
\end{Shaded}
\end{enumerate}

\subsubsection{Vincular la clave a
GitHub}\label{vincular-la-clave-a-github}

\begin{enumerate}
\def\labelenumi{\arabic{enumi}.}
\item
  Copia la clave pública:

  \begin{itemize}
  \tightlist
  \item
    Linux/Mac: \texttt{cat\ \textasciitilde{}/.ssh/id\_rsa.pub}
  \item
    Windows:
    \texttt{clip\ \textless{}\ \textasciitilde{}/.ssh/id\_rsa.pub}
  \end{itemize}
\item
  En GitHub, ve a \emph{Settings \textgreater{} SSH and GPG keys
  \textgreater{} New SSH key}, pega la clave y guárdala.
\item
  Prueba la conexión:

\begin{Shaded}
\begin{Highlighting}[]
\FunctionTok{ssh} \AttributeTok{{-}T}\NormalTok{ git@github.com}
\end{Highlighting}
\end{Shaded}

  Resultado esperado:
  \texttt{Hi\ tu\_usuario!\ You\textquotesingle{}ve\ successfully\ authenticated...}
\end{enumerate}

\subsection{Crear un repositorio
local}\label{crear-un-repositorio-local}

\begin{enumerate}
\def\labelenumi{\arabic{enumi}.}
\item
  Inicia un repositorio:

\begin{Shaded}
\begin{Highlighting}[]
\FunctionTok{git}\NormalTok{ init }\PreprocessorTok{[}\SpecialStringTok{nombre}\PreprocessorTok{{-}}\SpecialStringTok{del}\PreprocessorTok{{-}}\SpecialStringTok{proyecto}\PreprocessorTok{]}
\end{Highlighting}
\end{Shaded}
\item
  Añade archivos y haz un commit:

\begin{Shaded}
\begin{Highlighting}[]
\FunctionTok{git}\NormalTok{ add .}
\FunctionTok{git}\NormalTok{ commit }\AttributeTok{{-}m} \StringTok{"Primer commit"}
\end{Highlighting}
\end{Shaded}
\end{enumerate}

\subsection{Clonar un repositorio}\label{clonar-un-repositorio}

\subsubsection{Clonación básica}\label{clonaciuxf3n-buxe1sica}

Clona un repositorio remoto:

\begin{Shaded}
\begin{Highlighting}[]
\FunctionTok{git}\NormalTok{ clone https://github.com/usuario/repositorio.git}
\end{Highlighting}
\end{Shaded}

Clonación en carpeta específica

\begin{Shaded}
\begin{Highlighting}[]
\FunctionTok{git}\NormalTok{ clone https://github.com/usuario/repositorio.git /ruta/deseada}
\end{Highlighting}
\end{Shaded}

\subsubsection{Clonación superficial}\label{clonaciuxf3n-superficial}

Solo las últimas n confirmaciones:

\begin{Shaded}
\begin{Highlighting}[]
\FunctionTok{git}\NormalTok{ clone }\AttributeTok{{-}{-}depth}\OperatorTok{=}\NormalTok{1 https://github.com/usuario/repositorio.git}
\end{Highlighting}
\end{Shaded}

\subsubsection{Clonar una rama
específica}\label{clonar-una-rama-especuxedfica}

\begin{Shaded}
\begin{Highlighting}[]
\FunctionTok{git}\NormalTok{ clone }\AttributeTok{{-}{-}branch}\OperatorTok{=}\NormalTok{nombre{-}rama https://github.com/usuario/repositorio.git}
\end{Highlighting}
\end{Shaded}

\subsection{Subir un proyecto a
GitHub}\label{subir-un-proyecto-a-github}

\begin{enumerate}
\def\labelenumi{\arabic{enumi}.}
\item
  Crea un repositorio en GitHub (público o privado).
\item
  En tu proyecto local:

\begin{Shaded}
\begin{Highlighting}[]
\FunctionTok{git}\NormalTok{ init}
\FunctionTok{git}\NormalTok{ add .}
\FunctionTok{git}\NormalTok{ commit }\AttributeTok{{-}m} \StringTok{"Inicial"}
\FunctionTok{git}\NormalTok{ remote add origin git@github.com:usuario/repositorio.git}
\FunctionTok{git}\NormalTok{ push }\AttributeTok{{-}u}\NormalTok{ origin main}
\end{Highlighting}
\end{Shaded}
\item
  Si aparece el error remote origin already exists:

\begin{Shaded}
\begin{Highlighting}[]
\FunctionTok{git}\NormalTok{ remote rm origin}
\end{Highlighting}
\end{Shaded}

  Luego repite el paso 2.
\end{enumerate}

\subsection{Observar el repositorio}\label{observar-el-repositorio}

\begin{itemize}
\item
  \texttt{git\ status}: Muestra el estado actual.
\item
  \texttt{git\ diff}: Compara cambios no confirmados.
\item
  \texttt{git\ log}
\item
  \texttt{git\ log\ -\/-graph}
\item
  \texttt{git\ log\ -\/-graph\ -\/-pretty=oneline}
\item
  \texttt{git\ log\ -\/-oneline}: Historial compacto. Ejemplo:

\begin{Shaded}
\begin{Highlighting}[]
\ExtensionTok{7e320e8}\NormalTok{ update}
\end{Highlighting}
\end{Shaded}
\item
  git blame {[}archivo{]}: Autores y fechas de cambios.
\end{itemize}

\subsection{Trabajar con ramas}\label{trabajar-con-ramas}

\begin{enumerate}
\def\labelenumi{\arabic{enumi}.}
\item
  Crea y cambia a una rama:

\begin{Shaded}
\begin{Highlighting}[]
\FunctionTok{git}\NormalTok{ checkout }\AttributeTok{{-}b}\NormalTok{ nueva{-}rama}
\end{Highlighting}
\end{Shaded}
\item
  Fusiona ramas:

\begin{Shaded}
\begin{Highlighting}[]
\FunctionTok{git}\NormalTok{ checkout main}
\FunctionTok{git}\NormalTok{ merge nueva{-}rama}
\end{Highlighting}
\end{Shaded}
\item
  Etiqueta un commit:

\begin{Shaded}
\begin{Highlighting}[]
\FunctionTok{git}\NormalTok{ tag v1.0.0}
\FunctionTok{git}\NormalTok{ push origin main }\AttributeTok{{-}{-}tags}
\end{Highlighting}
\end{Shaded}
\end{enumerate}

\subsection{Sincronización}\label{sincronizaciuxf3n}

\begin{itemize}
\tightlist
\item
  \texttt{git\ fetch\ origin}: Descarga cambios remotos sin fusionarlos.
\item
  \texttt{git\ pull\ origin\ main}: Descarga y fusiona cambios.
\item
  \texttt{git\ push\ origin\ main}: Envía cambios locales al remoto.
\end{itemize}

\section{Conclusión}\label{conclusiuxf3n}

Dominar Git y GitHub es clave para gestionar proyectos de desarrollo.
Practica estos comandos y consulta \texttt{git\ -\/-help} para más
detalles.

\section{Publicaciones Similares}\label{publicaciones-similares}

Si te interesó este artículo, te recomendamos que explores otros blogs y
recursos relacionados que pueden ampliar tus conocimientos. Aquí te dejo
algunas sugerencias:

\begin{enumerate}
\def\labelenumi{\arabic{enumi}.}
\tightlist
\item
  \href{https://achalmaedison.netlify.app/tecnologia-seguridad/operating-system/2017-05-21-comandos-de-informacion-windows/index.pdf}{\faIcon{file-pdf}}
  \href{https://achalmaedison.netlify.app/tecnologia-seguridad/operating-system/2017-05-21-comandos-de-informacion-windows}{Comandos
  De Informacion Windows}
\item
  \href{https://achalmaedison.netlify.app/tecnologia-seguridad/operating-system/2019-06-19-adb/index.pdf}{\faIcon{file-pdf}}
  \href{https://achalmaedison.netlify.app/tecnologia-seguridad/operating-system/2019-06-19-adb}{Adb}
\item
  \href{https://achalmaedison.netlify.app/tecnologia-seguridad/operating-system/2021-08-17-limpieza-y-optimizacion-de-pc/index.pdf}{\faIcon{file-pdf}}
  \href{https://achalmaedison.netlify.app/tecnologia-seguridad/operating-system/2021-08-17-limpieza-y-optimizacion-de-pc}{Limpieza
  Y Optimizacion De Pc}
\item
  \href{https://achalmaedison.netlify.app/tecnologia-seguridad/operating-system/2021-10-21-usando-apk-en-windown-11/index.pdf}{\faIcon{file-pdf}}
  \href{https://achalmaedison.netlify.app/tecnologia-seguridad/operating-system/2021-10-21-usando-apk-en-windown-11}{Usando
  Apk En Windown 11}
\item
  \href{https://achalmaedison.netlify.app/tecnologia-seguridad/operating-system/2022-05-12-gestionar-versiones-de-jdk-en-kubuntu/index.pdf}{\faIcon{file-pdf}}
  \href{https://achalmaedison.netlify.app/tecnologia-seguridad/operating-system/2022-05-12-gestionar-versiones-de-jdk-en-kubuntu}{Gestionar
  Versiones De Jdk En Kubuntu}
\item
  \href{https://achalmaedison.netlify.app/tecnologia-seguridad/operating-system/2022-07-21-instalar-tor-browser/index.pdf}{\faIcon{file-pdf}}
  \href{https://achalmaedison.netlify.app/tecnologia-seguridad/operating-system/2022-07-21-instalar-tor-browser}{Instalar
  Tor Browser}
\item
  \href{https://achalmaedison.netlify.app/tecnologia-seguridad/operating-system/2022-08-14-crear-enlaces-duros-o-hard-link-en-linux/index.pdf}{\faIcon{file-pdf}}
  \href{https://achalmaedison.netlify.app/tecnologia-seguridad/operating-system/2022-08-14-crear-enlaces-duros-o-hard-link-en-linux}{Crear
  Enlaces Duros O Hard Link En Linux}
\item
  \href{https://achalmaedison.netlify.app/tecnologia-seguridad/operating-system/2022-09-27-comandos-vim/index.pdf}{\faIcon{file-pdf}}
  \href{https://achalmaedison.netlify.app/tecnologia-seguridad/operating-system/2022-09-27-comandos-vim}{Comandos
  Vim}
\item
  \href{https://achalmaedison.netlify.app/tecnologia-seguridad/operating-system/2023-02-16-guia-de-git-y-github/index.pdf}{\faIcon{file-pdf}}
  \href{https://achalmaedison.netlify.app/tecnologia-seguridad/operating-system/2023-02-16-guia-de-git-y-github}{Guia
  De Git Y Github}
\item
  \href{https://achalmaedison.netlify.app/tecnologia-seguridad/operating-system/2023-05-02-00-primeros-pasos-en-linux/index.pdf}{\faIcon{file-pdf}}
  \href{https://achalmaedison.netlify.app/tecnologia-seguridad/operating-system/2023-05-02-00-primeros-pasos-en-linux}{00
  Primeros Pasos En Linux}
\item
  \href{https://achalmaedison.netlify.app/tecnologia-seguridad/operating-system/2023-06-17-01-introduccion-linux/index.pdf}{\faIcon{file-pdf}}
  \href{https://achalmaedison.netlify.app/tecnologia-seguridad/operating-system/2023-06-17-01-introduccion-linux}{01
  Introduccion Linux}
\item
  \href{https://achalmaedison.netlify.app/tecnologia-seguridad/operating-system/2023-06-18-02-distribuciones-linux/index.pdf}{\faIcon{file-pdf}}
  \href{https://achalmaedison.netlify.app/tecnologia-seguridad/operating-system/2023-06-18-02-distribuciones-linux}{02
  Distribuciones Linux}
\item
  \href{https://achalmaedison.netlify.app/tecnologia-seguridad/operating-system/2023-06-19-03-instalacion-linux/index.pdf}{\faIcon{file-pdf}}
  \href{https://achalmaedison.netlify.app/tecnologia-seguridad/operating-system/2023-06-19-03-instalacion-linux}{03
  Instalacion Linux}
\item
  \href{https://achalmaedison.netlify.app/tecnologia-seguridad/operating-system/2023-06-20-04-administracion-particiones-volumenes/index.pdf}{\faIcon{file-pdf}}
  \href{https://achalmaedison.netlify.app/tecnologia-seguridad/operating-system/2023-06-20-04-administracion-particiones-volumenes}{04
  Administracion Particiones Volumenes}
\item
  \href{https://achalmaedison.netlify.app/tecnologia-seguridad/operating-system/2023-07-01-atajos-de-teclado-y-comandos-para-usar-vim/index.pdf}{\faIcon{file-pdf}}
  \href{https://achalmaedison.netlify.app/tecnologia-seguridad/operating-system/2023-07-01-atajos-de-teclado-y-comandos-para-usar-vim}{Atajos
  De Teclado Y Comandos Para Usar Vim}
\item
  \href{https://achalmaedison.netlify.app/tecnologia-seguridad/operating-system/2024-07-15-instalando-specitify/index.pdf}{\faIcon{file-pdf}}
  \href{https://achalmaedison.netlify.app/tecnologia-seguridad/operating-system/2024-07-15-instalando-specitify}{Instalando
  Specitify}
\end{enumerate}

Esperamos que encuentres estas publicaciones igualmente interesantes y
útiles. ¡Disfruta de la lectura!






\end{document}
