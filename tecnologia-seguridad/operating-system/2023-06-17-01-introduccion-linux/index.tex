\documentclass[
  jou,
  floatsintext,
  longtable,
  a4paper,
  nolmodern,
  notxfonts,
  notimes,
  colorlinks=true,linkcolor=blue,citecolor=blue,urlcolor=blue]{apa7}

\usepackage{amsmath}
\usepackage{amssymb}



\usepackage[bidi=default]{babel}
\babelprovide[main,import]{spanish}
\StartBabelCommands{spanish}{captions} [unicode, fontenc=TU EU1 EU2, charset=utf8] \SetString{\keywordname}{Palabras
Claves}
\EndBabelCommands


% get rid of language-specific shorthands (see #6817):
\let\LanguageShortHands\languageshorthands
\def\languageshorthands#1{}

\RequirePackage{longtable}
\RequirePackage{threeparttablex}

\makeatletter
\renewcommand{\paragraph}{\@startsection{paragraph}{4}{\parindent}%
	{0\baselineskip \@plus 0.2ex \@minus 0.2ex}%
	{-.5em}%
	{\normalfont\normalsize\bfseries\typesectitle}}

\renewcommand{\subparagraph}[1]{\@startsection{subparagraph}{5}{0.5em}%
	{0\baselineskip \@plus 0.2ex \@minus 0.2ex}%
	{-\z@\relax}%
	{\normalfont\normalsize\bfseries\itshape\hspace{\parindent}{#1}\textit{\addperi}}{\relax}}
\makeatother




\usepackage{longtable, booktabs, multirow, multicol, colortbl, hhline, caption, array, float, xpatch}
\usepackage{subcaption}
\renewcommand\thesubfigure{\Alph{subfigure}}
\setcounter{topnumber}{2}
\setcounter{bottomnumber}{2}
\setcounter{totalnumber}{4}
\renewcommand{\topfraction}{0.85}
\renewcommand{\bottomfraction}{0.85}
\renewcommand{\textfraction}{0.15}
\renewcommand{\floatpagefraction}{0.7}

\usepackage{tcolorbox}
\tcbuselibrary{listings,theorems, breakable, skins}
\usepackage{fontawesome5}

\definecolor{quarto-callout-color}{HTML}{909090}
\definecolor{quarto-callout-note-color}{HTML}{0758E5}
\definecolor{quarto-callout-important-color}{HTML}{CC1914}
\definecolor{quarto-callout-warning-color}{HTML}{EB9113}
\definecolor{quarto-callout-tip-color}{HTML}{00A047}
\definecolor{quarto-callout-caution-color}{HTML}{FC5300}
\definecolor{quarto-callout-color-frame}{HTML}{ACACAC}
\definecolor{quarto-callout-note-color-frame}{HTML}{4582EC}
\definecolor{quarto-callout-important-color-frame}{HTML}{D9534F}
\definecolor{quarto-callout-warning-color-frame}{HTML}{F0AD4E}
\definecolor{quarto-callout-tip-color-frame}{HTML}{02B875}
\definecolor{quarto-callout-caution-color-frame}{HTML}{FD7E14}

%\newlength\Oldarrayrulewidth
%\newlength\Oldtabcolsep


\usepackage{hyperref}




\providecommand{\tightlist}{%
  \setlength{\itemsep}{0pt}\setlength{\parskip}{0pt}}
\usepackage{longtable,booktabs,array}
\usepackage{calc} % for calculating minipage widths
% Correct order of tables after \paragraph or \subparagraph
\usepackage{etoolbox}
\makeatletter
\patchcmd\longtable{\par}{\if@noskipsec\mbox{}\fi\par}{}{}
\makeatother
% Allow footnotes in longtable head/foot
\IfFileExists{footnotehyper.sty}{\usepackage{footnotehyper}}{\usepackage{footnote}}
\makesavenoteenv{longtable}

\usepackage{graphicx}
\makeatletter
\newsavebox\pandoc@box
\newcommand*\pandocbounded[1]{% scales image to fit in text height/width
  \sbox\pandoc@box{#1}%
  \Gscale@div\@tempa{\textheight}{\dimexpr\ht\pandoc@box+\dp\pandoc@box\relax}%
  \Gscale@div\@tempb{\linewidth}{\wd\pandoc@box}%
  \ifdim\@tempb\p@<\@tempa\p@\let\@tempa\@tempb\fi% select the smaller of both
  \ifdim\@tempa\p@<\p@\scalebox{\@tempa}{\usebox\pandoc@box}%
  \else\usebox{\pandoc@box}%
  \fi%
}
% Set default figure placement to htbp
\def\fps@figure{htbp}
\makeatother







\usepackage{newtx}

\defaultfontfeatures{Scale=MatchLowercase}
\defaultfontfeatures[\rmfamily]{Ligatures=TeX,Scale=1}





\title{Introduccion a GNU/Linux: Descubre por qué GNU/Linux supera a
Windows y macOS en términos de libertad de uso, seguridad y amplia
variedad de opciones de personalización.}


\shorttitle{Editar}


\usepackage{etoolbox}



\ccoppy{\textcopyright~2023}



\author{Edison Achalma}



\affiliation{
{Escuela Profesional de Economía, Universidad Nacional de San Cristóbal
de Huamanga}}




\leftheader{Achalma}

\date{2023-06-17}


\abstract{Primer parrafo de abstrac }

\keywords{keyword1, keyword2}

\authornote{\par{\addORCIDlink{Edison Achalma}{0000-0001-6996-3364}} 
\par{ }
\par{   El autor no tiene conflictos de interés que revelar.    Los
roles de autor se clasificaron utilizando la taxonomía de roles de
colaborador (CRediT; https://credit.niso.org/) de la siguiente
manera:  Edison Achalma:   conceptualización, redacción}
\par{La correspondencia relativa a este artículo debe dirigirse a Edison
Achalma, Email: \href{mailto:elmer.achalma.09@unsch.edu.pe}{elmer.achalma.09@unsch.edu.pe}}
}

\usepackage{pbalance} 
\usepackage{float}
\makeatletter
\let\oldtpt\ThreePartTable
\let\endoldtpt\endThreePartTable
\def\ThreePartTable{\@ifnextchar[\ThreePartTable@i \ThreePartTable@ii}
\def\ThreePartTable@i[#1]{\begin{figure}[!htbp]
\onecolumn
\begin{minipage}{0.5\textwidth}
\oldtpt[#1]
}
\def\ThreePartTable@ii{\begin{figure}[!htbp]
\onecolumn
\begin{minipage}{0.5\textwidth}
\oldtpt
}
\def\endThreePartTable{
\endoldtpt
\end{minipage}
\twocolumn
\end{figure}}
\makeatother


\makeatletter
\let\endoldlt\endlongtable		
\def\endlongtable{
\hline
\endoldlt}
\makeatother

\newenvironment{twocolumntable}% environment name
{% begin code
\begin{table*}[!htbp]%
\onecolumn%
}%
{%
\twocolumn%
\end{table*}%
}% end code

\urlstyle{same}



\makeatletter
\@ifpackageloaded{caption}{}{\usepackage{caption}}
\AtBeginDocument{%
\ifdefined\contentsname
  \renewcommand*\contentsname{Tabla de contenidos}
\else
  \newcommand\contentsname{Tabla de contenidos}
\fi
\ifdefined\listfigurename
  \renewcommand*\listfigurename{Listado de Figuras}
\else
  \newcommand\listfigurename{Listado de Figuras}
\fi
\ifdefined\listtablename
  \renewcommand*\listtablename{Listado de Tablas}
\else
  \newcommand\listtablename{Listado de Tablas}
\fi
\ifdefined\figurename
  \renewcommand*\figurename{Figura}
\else
  \newcommand\figurename{Figura}
\fi
\ifdefined\tablename
  \renewcommand*\tablename{Tabla}
\else
  \newcommand\tablename{Tabla}
\fi
}
\@ifpackageloaded{float}{}{\usepackage{float}}
\floatstyle{ruled}
\@ifundefined{c@chapter}{\newfloat{codelisting}{h}{lop}}{\newfloat{codelisting}{h}{lop}[chapter]}
\floatname{codelisting}{Listado}
\newcommand*\listoflistings{\listof{codelisting}{Listado de Listados}}
\makeatother
\makeatletter
\makeatother
\makeatletter
\@ifpackageloaded{caption}{}{\usepackage{caption}}
\@ifpackageloaded{subcaption}{}{\usepackage{subcaption}}
\makeatother
\makeatletter
\@ifpackageloaded{fontawesome5}{}{\usepackage{fontawesome5}}
\makeatother

% From https://tex.stackexchange.com/a/645996/211326
%%% apa7 doesn't want to add appendix section titles in the toc
%%% let's make it do it
\makeatletter
\xpatchcmd{\appendix}
  {\par}
  {\addcontentsline{toc}{section}{\@currentlabelname}\par}
  {}{}
\makeatother

%% Disable longtable counter
%% https://tex.stackexchange.com/a/248395/211326

\usepackage{etoolbox}

\makeatletter
\patchcmd{\LT@caption}
  {\bgroup}
  {\bgroup\global\LTpatch@captiontrue}
  {}{}
\patchcmd{\longtable}
  {\par}
  {\par\global\LTpatch@captionfalse}
  {}{}
\apptocmd{\endlongtable}
  {\ifLTpatch@caption\else\addtocounter{table}{-1}\fi}
  {}{}
\newif\ifLTpatch@caption
\makeatother

\begin{document}

\maketitle

\hypertarget{toc}{}
\tableofcontents
\newpage
\section[Introduction]{Introduccion a GNU/Linux}

\setcounter{secnumdepth}{-\maxdimen} % remove section numbering

\setlength\LTleft{0pt}


¡Hola, estimado lector! Ya sea que acabes de adentrarte en el fascinante
mundo de GNU/Linux y su acogedora comunidad, o que lleves un tiempo
utilizando Linux y estés ansioso por aprender cada vez más, esta página
es perfecta para ti.

\section{¿Qué es GNU/Linux?
Introducción}\label{quuxe9-es-gnulinux-introducciuxf3n}

\subsection{GNU/Linux}\label{gnulinux}

Es un sistema operativo de software libre y de código abierto que surge
de la combinación del sistema GNU, desarrollado por la FSF, y el núcleo
o kernel Linux, creado por Linus Torvalds.

A diferencia de otros sistemas operativos cerrados, GNU/Linux no se
presenta como un producto único, sino como una base sobre la cual se han
construido y continúan desarrollándose numerosas propuestas y
distribuciones.

Existen una amplia variedad de distribuciones diseñadas para diferentes
usos y usuarios. Sin embargo, lo importante aquí es que, a pesar de las
diferencias entre ellas, todas comparten una base común:

\subsection{Kernel Linux}\label{kernel-linux}

El kernel, o núcleo, es el componente central de cualquier sistema
operativo. En términos sencillos, podríamos decir que se encarga de
establecer la comunicación entre los componentes de software del sistema
y los recursos de hardware de la máquina. Cada distribución GNU/Linux
elige una versión específica del kernel, que no siempre coincide con la
más reciente.

\subsection{GNU}\label{gnu}

En este contexto, nos referimos al conjunto de herramientas propias del
proyecto GNU, que conforman los componentes fundamentales del sistema y,
en general, son independientes del entorno de escritorio utilizado.
Algunas de las herramientas más conocidas son el intérprete de comandos
Bash, el compilador GCC y el entorno de escritorio GNOME, que también
forma parte del proyecto GNU.

Sin embargo, esta flexibilidad también conlleva una gran fragmentación
de proyectos en todos los niveles, incluyendo distribuciones, entornos
gráficos, herramientas, entre otros. A pesar de esto, GNU/Linux sigue
siendo una plataforma sólida y poderosa para aquellos que valoran la
libertad, la personalización y la comunidad que la rodea.

\section{El kernel Linux}\label{el-kernel-linux}

Linux es el kernel desarrollado por Linus Torvalds en 1991 y se ofrece
actualmente bajo la licencia GPL v2. Lo que comenzó como un proyecto
liderado por Linus con la colaboración voluntaria de otros
programadores, ha evolucionado en un proyecto de proporciones
gigantescas. En él participan empresas de renombre como Red Hat, Intel,
Samsung, Dell y Oracle, sin mencionar a Microsoft y Google, que son
miembros Platino de la Linux Foundation.

Linux en sí mismo no es un sistema operativo, pero representa la parte
más importante que lo compone, es decir, el kernel. El núcleo Linux no
solo es utilizado por el sistema GNU/Linux y todas las distribuciones
que lo conforman, sino que también es el kernel elegido por Google para
dar vida a Android, el sistema operativo más utilizado en smartphones y
tablets.

Linux está diseñado para ejecutarse en una amplia variedad de
arquitecturas, desde x86-64 (la más común en la mayoría de las
computadoras de 64 bits) hasta i386 (para computadoras Intel de 32
bits), ARM, PowerPC, MIPS, OpenRISC y muchas más. De hecho, el uso de
Linux se extiende mucho más allá de las computadoras y los teléfonos
inteligentes, y se encuentra presente en routers, refrigeradores,
lavadoras, automóviles, relojes, drones, robots y muchos otros
dispositivos.

\section{El sistema operativo GNU}\label{el-sistema-operativo-gnu}

El sistema operativo GNU es un proyecto que se inició en 1983 por
Richard Stallman y la Free Software Foundation (FSF). Su objetivo
principal es desarrollar un sistema operativo completo compuesto por
software libre, es decir, software que respeta la libertad de los
usuarios para ejecutar, copiar, distribuir, estudiar, modificar y
mejorar el software.

El nombre ``GNU'' es un acrónimo recursivo que significa ``GNU's Not
Unix'', lo cual refleja la intención del proyecto de crear un sistema
operativo compatible con Unix pero libre de las restricciones propias de
las versiones propietarias de Unix.

El sistema operativo GNU está compuesto por una amplia gama de
componentes de software, incluyendo el núcleo, compiladores,
bibliotecas, herramientas de desarrollo y aplicaciones. Estos
componentes son distribuidos bajo licencias que garantizan la libertad
de los usuarios y fomentan la colaboración y el intercambio de
conocimientos.

Uno de los aspectos fundamentales del sistema operativo GNU es su
enfoque filosófico y ético. Stallman y la FSF promueven los valores del
software libre y luchan por la libertad del usuario en el entorno
digital. Consideran que el software propietario y las restricciones
impuestas por las licencias restrictivas limitan la libertad de los
usuarios y generan desigualdad en el acceso al conocimiento y la
tecnología.

\section{¿Qué Ventajas Ofrece GNU/Linux Frente a Windows y
macOS?}\label{quuxe9-ventajas-ofrece-gnulinux-frente-a-windows-y-macos}

Cuando hablamos de sistemas operativos, existen diferentes opciones
entre las cuales elegir. En este caso, nos centraremos en las ventajas
que GNU/Linux ofrece en comparación con Windows y macOS, dos sistemas
operativos muy conocidos.

\begin{enumerate}
\def\labelenumi{\arabic{enumi}.}
\item
  \textbf{Libertad y control}: Una de las grandes ventajas de GNU/Linux
  es la libertad que brinda a los usuarios. Pueden utilizar, modificar y
  distribuir el software de acuerdo con sus necesidades y preferencias.
  Esto significa que tienes el control total sobre tu sistema operativo
  y puedes adaptarlo según tus propias necesidades.
\item
  \textbf{Seguridad y estabilidad:} GNU/Linux destaca por su seguridad y
  estabilidad. Debido a su naturaleza de código abierto, un gran número
  de desarrolladores pueden revisarlo y mejorarlo constantemente. Esto
  ayuda a identificar y corregir rápidamente cualquier problema o
  vulnerabilidad, lo que resulta en un sistema operativo más seguro y
  estable en comparación con Windows y macOS.
\item
  \textbf{Variedad de opciones:} GNU/Linux ofrece una amplia variedad de
  distribuciones o ``distros'', cada una con sus características y
  enfoques particulares. Esto te permite elegir la distribución que
  mejor se adapte a tus necesidades y preferencias. Además, puedes
  personalizar tu experiencia de usuario según tus propios gustos y
  requisitos.
\item
  \textbf{Comunidad y soporte:} En el mundo de GNU/Linux, existe una
  comunidad activa y colaborativa. Puedes encontrar foros, grupos de
  discusión y recursos en línea donde puedes obtener ayuda, compartir
  conocimientos y resolver problemas. Además, la comunidad proporciona
  soporte técnico y existe una abundante documentación disponible para
  aprender más sobre el sistema operativo.
\item
  \textbf{Costo:} Una de las grandes ventajas de GNU/Linux es su costo.
  Muchas distribuciones son de código abierto y se pueden obtener de
  forma gratuita. Esto supone un ahorro significativo en comparación con
  los sistemas operativos comerciales como Windows y macOS. Además, la
  mayoría del software disponible para GNU/Linux también es gratuito, lo
  que te permite ahorrar aún más en la adquisición de programas
  adicionales.
\end{enumerate}

\section{Publicaciones Similares}\label{publicaciones-similares}

Si te interesó este artículo, te recomendamos que explores otros blogs y
recursos relacionados que pueden ampliar tus conocimientos. Aquí te dejo
algunas sugerencias:

\begin{enumerate}
\def\labelenumi{\arabic{enumi}.}
\tightlist
\item
  \href{https://achalmaedison.netlify.app/tecnologia-seguridad/operating-system/2017-05-21-comandos-de-informacion-windows/index.pdf}{\faIcon{file-pdf}}
  \href{https://achalmaedison.netlify.app/tecnologia-seguridad/operating-system/2017-05-21-comandos-de-informacion-windows}{Comandos
  De Informacion Windows}
\item
  \href{https://achalmaedison.netlify.app/tecnologia-seguridad/operating-system/2019-06-19-adb/index.pdf}{\faIcon{file-pdf}}
  \href{https://achalmaedison.netlify.app/tecnologia-seguridad/operating-system/2019-06-19-adb}{Adb}
\item
  \href{https://achalmaedison.netlify.app/tecnologia-seguridad/operating-system/2021-08-17-limpieza-y-optimizacion-de-pc/index.pdf}{\faIcon{file-pdf}}
  \href{https://achalmaedison.netlify.app/tecnologia-seguridad/operating-system/2021-08-17-limpieza-y-optimizacion-de-pc}{Limpieza
  Y Optimizacion De Pc}
\item
  \href{https://achalmaedison.netlify.app/tecnologia-seguridad/operating-system/2021-10-21-usando-apk-en-windown-11/index.pdf}{\faIcon{file-pdf}}
  \href{https://achalmaedison.netlify.app/tecnologia-seguridad/operating-system/2021-10-21-usando-apk-en-windown-11}{Usando
  Apk En Windown 11}
\item
  \href{https://achalmaedison.netlify.app/tecnologia-seguridad/operating-system/2022-05-12-gestionar-versiones-de-jdk-en-kubuntu/index.pdf}{\faIcon{file-pdf}}
  \href{https://achalmaedison.netlify.app/tecnologia-seguridad/operating-system/2022-05-12-gestionar-versiones-de-jdk-en-kubuntu}{Gestionar
  Versiones De Jdk En Kubuntu}
\item
  \href{https://achalmaedison.netlify.app/tecnologia-seguridad/operating-system/2022-07-21-instalar-tor-browser/index.pdf}{\faIcon{file-pdf}}
  \href{https://achalmaedison.netlify.app/tecnologia-seguridad/operating-system/2022-07-21-instalar-tor-browser}{Instalar
  Tor Browser}
\item
  \href{https://achalmaedison.netlify.app/tecnologia-seguridad/operating-system/2022-08-14-crear-enlaces-duros-o-hard-link-en-linux/index.pdf}{\faIcon{file-pdf}}
  \href{https://achalmaedison.netlify.app/tecnologia-seguridad/operating-system/2022-08-14-crear-enlaces-duros-o-hard-link-en-linux}{Crear
  Enlaces Duros O Hard Link En Linux}
\item
  \href{https://achalmaedison.netlify.app/tecnologia-seguridad/operating-system/2022-09-27-comandos-vim/index.pdf}{\faIcon{file-pdf}}
  \href{https://achalmaedison.netlify.app/tecnologia-seguridad/operating-system/2022-09-27-comandos-vim}{Comandos
  Vim}
\item
  \href{https://achalmaedison.netlify.app/tecnologia-seguridad/operating-system/2023-02-16-guia-de-git-y-github/index.pdf}{\faIcon{file-pdf}}
  \href{https://achalmaedison.netlify.app/tecnologia-seguridad/operating-system/2023-02-16-guia-de-git-y-github}{Guia
  De Git Y Github}
\item
  \href{https://achalmaedison.netlify.app/tecnologia-seguridad/operating-system/2023-05-02-00-primeros-pasos-en-linux/index.pdf}{\faIcon{file-pdf}}
  \href{https://achalmaedison.netlify.app/tecnologia-seguridad/operating-system/2023-05-02-00-primeros-pasos-en-linux}{00
  Primeros Pasos En Linux}
\item
  \href{https://achalmaedison.netlify.app/tecnologia-seguridad/operating-system/2023-06-17-01-introduccion-linux/index.pdf}{\faIcon{file-pdf}}
  \href{https://achalmaedison.netlify.app/tecnologia-seguridad/operating-system/2023-06-17-01-introduccion-linux}{01
  Introduccion Linux}
\item
  \href{https://achalmaedison.netlify.app/tecnologia-seguridad/operating-system/2023-06-18-02-distribuciones-linux/index.pdf}{\faIcon{file-pdf}}
  \href{https://achalmaedison.netlify.app/tecnologia-seguridad/operating-system/2023-06-18-02-distribuciones-linux}{02
  Distribuciones Linux}
\item
  \href{https://achalmaedison.netlify.app/tecnologia-seguridad/operating-system/2023-06-19-03-instalacion-linux/index.pdf}{\faIcon{file-pdf}}
  \href{https://achalmaedison.netlify.app/tecnologia-seguridad/operating-system/2023-06-19-03-instalacion-linux}{03
  Instalacion Linux}
\item
  \href{https://achalmaedison.netlify.app/tecnologia-seguridad/operating-system/2023-06-20-04-administracion-particiones-volumenes/index.pdf}{\faIcon{file-pdf}}
  \href{https://achalmaedison.netlify.app/tecnologia-seguridad/operating-system/2023-06-20-04-administracion-particiones-volumenes}{04
  Administracion Particiones Volumenes}
\item
  \href{https://achalmaedison.netlify.app/tecnologia-seguridad/operating-system/2023-07-01-atajos-de-teclado-y-comandos-para-usar-vim/index.pdf}{\faIcon{file-pdf}}
  \href{https://achalmaedison.netlify.app/tecnologia-seguridad/operating-system/2023-07-01-atajos-de-teclado-y-comandos-para-usar-vim}{Atajos
  De Teclado Y Comandos Para Usar Vim}
\item
  \href{https://achalmaedison.netlify.app/tecnologia-seguridad/operating-system/2024-07-15-instalando-specitify/index.pdf}{\faIcon{file-pdf}}
  \href{https://achalmaedison.netlify.app/tecnologia-seguridad/operating-system/2024-07-15-instalando-specitify}{Instalando
  Specitify}
\end{enumerate}

Esperamos que encuentres estas publicaciones igualmente interesantes y
útiles. ¡Disfruta de la lectura!






\end{document}
