\documentclass[
  jou,
  floatsintext,
  longtable,
  a4paper,
  nolmodern,
  notxfonts,
  notimes,
  colorlinks=true,linkcolor=blue,citecolor=blue,urlcolor=blue]{apa7}

\usepackage{amsmath}
\usepackage{amssymb}



\usepackage[bidi=default]{babel}
\babelprovide[main,import]{spanish}
\StartBabelCommands{spanish}{captions} [unicode, fontenc=TU EU1 EU2, charset=utf8] \SetString{\keywordname}{Palabras
Claves}
\EndBabelCommands


% get rid of language-specific shorthands (see #6817):
\let\LanguageShortHands\languageshorthands
\def\languageshorthands#1{}

\RequirePackage{longtable}
\RequirePackage{threeparttablex}

\makeatletter
\renewcommand{\paragraph}{\@startsection{paragraph}{4}{\parindent}%
	{0\baselineskip \@plus 0.2ex \@minus 0.2ex}%
	{-.5em}%
	{\normalfont\normalsize\bfseries\typesectitle}}

\renewcommand{\subparagraph}[1]{\@startsection{subparagraph}{5}{0.5em}%
	{0\baselineskip \@plus 0.2ex \@minus 0.2ex}%
	{-\z@\relax}%
	{\normalfont\normalsize\bfseries\itshape\hspace{\parindent}{#1}\textit{\addperi}}{\relax}}
\makeatother




\usepackage{longtable, booktabs, multirow, multicol, colortbl, hhline, caption, array, float, xpatch}
\usepackage{subcaption}
\renewcommand\thesubfigure{\Alph{subfigure}}
\setcounter{topnumber}{2}
\setcounter{bottomnumber}{2}
\setcounter{totalnumber}{4}
\renewcommand{\topfraction}{0.85}
\renewcommand{\bottomfraction}{0.85}
\renewcommand{\textfraction}{0.15}
\renewcommand{\floatpagefraction}{0.7}

\usepackage{tcolorbox}
\tcbuselibrary{listings,theorems, breakable, skins}
\usepackage{fontawesome5}

\definecolor{quarto-callout-color}{HTML}{909090}
\definecolor{quarto-callout-note-color}{HTML}{0758E5}
\definecolor{quarto-callout-important-color}{HTML}{CC1914}
\definecolor{quarto-callout-warning-color}{HTML}{EB9113}
\definecolor{quarto-callout-tip-color}{HTML}{00A047}
\definecolor{quarto-callout-caution-color}{HTML}{FC5300}
\definecolor{quarto-callout-color-frame}{HTML}{ACACAC}
\definecolor{quarto-callout-note-color-frame}{HTML}{4582EC}
\definecolor{quarto-callout-important-color-frame}{HTML}{D9534F}
\definecolor{quarto-callout-warning-color-frame}{HTML}{F0AD4E}
\definecolor{quarto-callout-tip-color-frame}{HTML}{02B875}
\definecolor{quarto-callout-caution-color-frame}{HTML}{FD7E14}

%\newlength\Oldarrayrulewidth
%\newlength\Oldtabcolsep


\usepackage{hyperref}




\providecommand{\tightlist}{%
  \setlength{\itemsep}{0pt}\setlength{\parskip}{0pt}}
\usepackage{longtable,booktabs,array}
\usepackage{calc} % for calculating minipage widths
% Correct order of tables after \paragraph or \subparagraph
\usepackage{etoolbox}
\makeatletter
\patchcmd\longtable{\par}{\if@noskipsec\mbox{}\fi\par}{}{}
\makeatother
% Allow footnotes in longtable head/foot
\IfFileExists{footnotehyper.sty}{\usepackage{footnotehyper}}{\usepackage{footnote}}
\makesavenoteenv{longtable}

\usepackage{graphicx}
\makeatletter
\newsavebox\pandoc@box
\newcommand*\pandocbounded[1]{% scales image to fit in text height/width
  \sbox\pandoc@box{#1}%
  \Gscale@div\@tempa{\textheight}{\dimexpr\ht\pandoc@box+\dp\pandoc@box\relax}%
  \Gscale@div\@tempb{\linewidth}{\wd\pandoc@box}%
  \ifdim\@tempb\p@<\@tempa\p@\let\@tempa\@tempb\fi% select the smaller of both
  \ifdim\@tempa\p@<\p@\scalebox{\@tempa}{\usebox\pandoc@box}%
  \else\usebox{\pandoc@box}%
  \fi%
}
% Set default figure placement to htbp
\def\fps@figure{htbp}
\makeatother







\usepackage{newtx}

\defaultfontfeatures{Scale=MatchLowercase}
\defaultfontfeatures[\rmfamily]{Ligatures=TeX,Scale=1}





\title{Administracion de particiones y volumenes: Descubre cómo
organizar y proteger tus datos en GNU/Linux con Particiones, Volúmenes
LVM y el cifrado LUKS}


\shorttitle{Editar}


\usepackage{etoolbox}



\ccoppy{\textcopyright~2023}



\author{Edison Achalma}



\affiliation{
{Escuela Profesional de Economía, Universidad Nacional de San Cristóbal
de Huamanga}}




\leftheader{Achalma}

\date{2023-06-20}


\abstract{Primer parrafo de abstrac }

\keywords{keyword1, keyword2}

\authornote{\par{\addORCIDlink{Edison Achalma}{0000-0001-6996-3364}} 
\par{ }
\par{   El autor no tiene conflictos de interés que revelar.    Los
roles de autor se clasificaron utilizando la taxonomía de roles de
colaborador (CRediT; https://credit.niso.org/) de la siguiente
manera:  Edison Achalma:   conceptualización, redacción}
\par{La correspondencia relativa a este artículo debe dirigirse a Edison
Achalma, Email: \href{mailto:elmer.achalma.09@unsch.edu.pe}{elmer.achalma.09@unsch.edu.pe}}
}

\usepackage{pbalance} 
\usepackage{float}
\makeatletter
\let\oldtpt\ThreePartTable
\let\endoldtpt\endThreePartTable
\def\ThreePartTable{\@ifnextchar[\ThreePartTable@i \ThreePartTable@ii}
\def\ThreePartTable@i[#1]{\begin{figure}[!htbp]
\onecolumn
\begin{minipage}{0.5\textwidth}
\oldtpt[#1]
}
\def\ThreePartTable@ii{\begin{figure}[!htbp]
\onecolumn
\begin{minipage}{0.5\textwidth}
\oldtpt
}
\def\endThreePartTable{
\endoldtpt
\end{minipage}
\twocolumn
\end{figure}}
\makeatother


\makeatletter
\let\endoldlt\endlongtable		
\def\endlongtable{
\hline
\endoldlt}
\makeatother

\newenvironment{twocolumntable}% environment name
{% begin code
\begin{table*}[!htbp]%
\onecolumn%
}%
{%
\twocolumn%
\end{table*}%
}% end code

\urlstyle{same}



\makeatletter
\@ifpackageloaded{caption}{}{\usepackage{caption}}
\AtBeginDocument{%
\ifdefined\contentsname
  \renewcommand*\contentsname{Tabla de contenidos}
\else
  \newcommand\contentsname{Tabla de contenidos}
\fi
\ifdefined\listfigurename
  \renewcommand*\listfigurename{Listado de Figuras}
\else
  \newcommand\listfigurename{Listado de Figuras}
\fi
\ifdefined\listtablename
  \renewcommand*\listtablename{Listado de Tablas}
\else
  \newcommand\listtablename{Listado de Tablas}
\fi
\ifdefined\figurename
  \renewcommand*\figurename{Figura}
\else
  \newcommand\figurename{Figura}
\fi
\ifdefined\tablename
  \renewcommand*\tablename{Tabla}
\else
  \newcommand\tablename{Tabla}
\fi
}
\@ifpackageloaded{float}{}{\usepackage{float}}
\floatstyle{ruled}
\@ifundefined{c@chapter}{\newfloat{codelisting}{h}{lop}}{\newfloat{codelisting}{h}{lop}[chapter]}
\floatname{codelisting}{Listado}
\newcommand*\listoflistings{\listof{codelisting}{Listado de Listados}}
\makeatother
\makeatletter
\makeatother
\makeatletter
\@ifpackageloaded{caption}{}{\usepackage{caption}}
\@ifpackageloaded{subcaption}{}{\usepackage{subcaption}}
\makeatother
\makeatletter
\@ifpackageloaded{fontawesome5}{}{\usepackage{fontawesome5}}
\makeatother

% From https://tex.stackexchange.com/a/645996/211326
%%% apa7 doesn't want to add appendix section titles in the toc
%%% let's make it do it
\makeatletter
\xpatchcmd{\appendix}
  {\par}
  {\addcontentsline{toc}{section}{\@currentlabelname}\par}
  {}{}
\makeatother

%% Disable longtable counter
%% https://tex.stackexchange.com/a/248395/211326

\usepackage{etoolbox}

\makeatletter
\patchcmd{\LT@caption}
  {\bgroup}
  {\bgroup\global\LTpatch@captiontrue}
  {}{}
\patchcmd{\longtable}
  {\par}
  {\par\global\LTpatch@captionfalse}
  {}{}
\apptocmd{\endlongtable}
  {\ifLTpatch@caption\else\addtocounter{table}{-1}\fi}
  {}{}
\newif\ifLTpatch@caption
\makeatother

\begin{document}

\maketitle

\hypertarget{toc}{}
\tableofcontents
\newpage
\section[Introduction]{Administracion de particiones y volumenes}

\setcounter{secnumdepth}{-\maxdimen} % remove section numbering

\setlength\LTleft{0pt}


¡Hola, lector! Te doy la más cordial bienvenida a esta nueva entrega de
nuestra serie de introducción a Linux. En esta ocasión, vamos a
sumergirnos en un tema fascinante y fundamental: las Particiones y
Volúmenes en Linux.

Si ya has leído nuestros artículos anteriores sobre GNU/Linux,
distribuciones y entornos de escritorio, estás listo para adentrarte en
el apasionante mundo de la administración del sistema. Aquí descubrirás
cómo estructurar y gestionar tus particiones, así como los volúmenes que
albergan tus datos.

\section{Esquema de Particiones: MBR y
GPT}\label{esquema-de-particiones-mbr-y-gpt}

Dos de los esquemas más comunes son MBR (Master Boot Record) y GPT (GUID
Partition Table). Estos esquemas determinan cómo se organiza y gestiona
el espacio en tu disco duro.

\textbf{1. MBR (Master Boot Record):} Es el esquema más antiguo y
ampliamente utilizado. Permite dividir el disco en hasta 4 particiones
primarias, o 3 particiones primarias y una extendida que puede contener
múltiples particiones lógicas. Es compatible con la mayoría de los
sistemas operativos, pero tiene limitaciones, como la capacidad máxima
de 2 terabytes para cada partición.

\textbf{2. GPT (GUID Partition Table):} Es el esquema más moderno y
robusto. Puede manejar discos de mayor capacidad y admite hasta 128
particiones. Además, ofrece beneficios adicionales, como la redundancia
de datos y la recuperación ante fallas. GPT es compatible con sistemas
UEFI (Unified Extensible Firmware Interface) y es la elección
recomendada para discos de más de 2 terabytes.

\section{¿Cómo se genera el esquema de particiones en
GNU/Linux?}\label{cuxf3mo-se-genera-el-esquema-de-particiones-en-gnulinux}

Este proceso es crucial para organizar y aprovechar al máximo el espacio
de tu disco duro.

En GNU/Linux, puedes elegir el esquema de particiones cuando quemas la
imagen ISO en un medio USB. Normalmente, se recomienda utilizar GPT si
tu equipo es relativamente moderno. Puedes trabajar con MBR o GPT
indistintamente, pero la elección adecuada marcará la diferencia.

Una de las aplicaciones más populares para crear medios de instalación
es Rufus (disponible actualmente solo para Windows). Te permite
seleccionar MBR o GPT, pero de forma predeterminada, la opción MBR está
seleccionada. Algunas herramientas pueden no pedirte esta elección y
preparar los medios con MBR por compatibilidad, así que es importante
elegir la aplicación adecuada.

Independientemente del esquema utilizado, la mayoría de las
distribuciones crean una tabla de particiones por defecto durante el
asistente de instalación. Esta tabla suele incluir una partición
primaria para instalar el sistema operativo y una partición lógica
reservada para el área de intercambio.

En el caso de GNU/Linux, es común crear una partición primaria para el
sistema operativo y una partición lógica para el área de intercambio o
Swap. Sin embargo, si lo deseas, puedes personalizar la tabla de
particiones según tus necesidades, creando particiones adicionales o
redimensionando las existentes.

Si estás acostumbrado a Windows, probablemente estés familiarizado con
la nomenclatura basada en letras para distinguir entre diferentes
volúmenes (particiones y unidades físicas). Por ejemplo, la unidad C es
donde se encuentra instalado el sistema operativo, y las unidades D, E,
F, etc., representan otras particiones del mismo disco o de unidades
externas.

En GNU/Linux, no existe esa nomenclatura. El sistema operativo se
origina desde el directorio raíz, representado como ``/''. Todos los
demás directorios derivan del directorio raíz, independientemente de la
partición en la que se encuentren.

Aquí tienes un ejemplo de cómo podrías representar tres particiones en
una instalación típica, donde además de la partición del sistema
operativo y la de intercambio, decides utilizar una partición separada
para el directorio ``/home''.

\section{Administrador de Volúmenes
LVM}\label{administrador-de-voluxfamenes-lvm}

El Administrador de Volúmenes LVM (Logical Volume Manager) es una
herramienta que te permite crear volúmenes lógicos a partir de uno o
varios discos físicos. ¿Por qué es tan genial? Pues porque te brinda
flexibilidad y control sobre tus particiones, permitiéndote
redimensionarlas en caliente, es decir, sin necesidad de reiniciar el
sistema. ¡Es como la magia del almacenamiento en acción!

Imagínate que tienes una partición que se está quedando sin espacio,
pero en cambio, tienes espacio libre en otra partición. Con LVM, puedes
agrandar la partición necesitada y aprovechar ese espacio libre. ¡Adiós
a los dolores de cabeza por falta de espacio!

¿Pero cómo funciona? Muy sencillo. LVM utiliza tres elementos
principales: \textbf{volúmenes físicos (PV)}, \textbf{grupos de
volúmenes (VG)} y \textbf{volúmenes lógicos (LV)}. Los volúmenes físicos
son los discos duros o particiones que utilizaremos para crear nuestro
espacio de almacenamiento. Los grupos de volúmenes actúan como
contenedores que agrupan los volúmenes físicos, mientras que los
volúmenes lógicos son las unidades de almacenamiento que se utilizan
como particiones en tu sistema.

Con LVM, puedes crear, eliminar, redimensionar y administrar estos
volúmenes lógicos de manera dinámica, adaptándolos según tus
necesidades. Además, también puedes crear instantáneas de tus volúmenes
lógicos para hacer copias de seguridad o probar cambios sin arriesgar
tus datos. ¡Es como tener superpoderes de gestión de almacenamiento!

\section{Cifrado de Disco LUKS}\label{cifrado-de-disco-luks}

El cifrado de disco LUKS (Linux Unified Key Setup) te permite encriptar
tus particiones o discos completos en GNU/Linux. ¿Por qué es importante?
Porque asegura la confidencialidad de tu información, evitando que
terceros no autorizados accedan a tus datos en caso de pérdida o robo
del equipo. Es como un fuerte escudo para tus archivos más preciados.

LUKS utiliza algoritmos criptográficos de alta seguridad, como AES
(Advanced Encryption Standard), para proteger tus datos. Puedes elegir
una contraseña fuerte o incluso utilizar una clave en un dispositivo USB
para desbloquear tus discos en el arranque. ¡Tú tienes el control total
de tus claves!

Una vez que hayas configurado el cifrado LUKS en tu disco, cada vez que
inicies tu sistema, se te pedirá la contraseña o la clave del USB para
desbloquear el disco. Después de desbloquearlo, podrás utilizar tu
sistema de forma normal, pero con la tranquilidad de saber que tus datos
están protegidos.

Además, LUKS también te brinda la posibilidad de crear contenedores
encriptados, donde puedes almacenar archivos y carpetas sensibles. Estos
contenedores se comportan como archivos normales, pero están protegidos
por una capa adicional de seguridad. Puedes abrirlos y acceder a su
contenido solo con la contraseña correcta. ¡Es como tener una caja
fuerte virtual en tu sistema!

Es importante tener en cuenta que el cifrado de disco LUKS puede afectar
ligeramente el rendimiento del sistema, ya que cada vez que accedas a
tus datos, se realizará la desencriptación en tiempo real. Sin embargo,
esta pequeña pérdida de velocidad vale la pena para garantizar la
seguridad de tus archivos.

Ahora que conoces los beneficios del cifrado de disco LUKS, no dudes en
explorar esta poderosa herramienta y mantener tus datos a salvo de
miradas indiscretas.

\section{Publicaciones Similares}\label{publicaciones-similares}

Si te interesó este artículo, te recomendamos que explores otros blogs y
recursos relacionados que pueden ampliar tus conocimientos. Aquí te dejo
algunas sugerencias:

\begin{enumerate}
\def\labelenumi{\arabic{enumi}.}
\tightlist
\item
  \href{https://achalmaedison.netlify.app/tecnologia-seguridad/operating-system/2017-05-21-comandos-de-informacion-windows/index.pdf}{\faIcon{file-pdf}}
  \href{https://achalmaedison.netlify.app/tecnologia-seguridad/operating-system/2017-05-21-comandos-de-informacion-windows}{Comandos
  De Informacion Windows}
\item
  \href{https://achalmaedison.netlify.app/tecnologia-seguridad/operating-system/2019-06-19-adb/index.pdf}{\faIcon{file-pdf}}
  \href{https://achalmaedison.netlify.app/tecnologia-seguridad/operating-system/2019-06-19-adb}{Adb}
\item
  \href{https://achalmaedison.netlify.app/tecnologia-seguridad/operating-system/2021-08-17-limpieza-y-optimizacion-de-pc/index.pdf}{\faIcon{file-pdf}}
  \href{https://achalmaedison.netlify.app/tecnologia-seguridad/operating-system/2021-08-17-limpieza-y-optimizacion-de-pc}{Limpieza
  Y Optimizacion De Pc}
\item
  \href{https://achalmaedison.netlify.app/tecnologia-seguridad/operating-system/2021-10-21-usando-apk-en-windown-11/index.pdf}{\faIcon{file-pdf}}
  \href{https://achalmaedison.netlify.app/tecnologia-seguridad/operating-system/2021-10-21-usando-apk-en-windown-11}{Usando
  Apk En Windown 11}
\item
  \href{https://achalmaedison.netlify.app/tecnologia-seguridad/operating-system/2022-05-12-gestionar-versiones-de-jdk-en-kubuntu/index.pdf}{\faIcon{file-pdf}}
  \href{https://achalmaedison.netlify.app/tecnologia-seguridad/operating-system/2022-05-12-gestionar-versiones-de-jdk-en-kubuntu}{Gestionar
  Versiones De Jdk En Kubuntu}
\item
  \href{https://achalmaedison.netlify.app/tecnologia-seguridad/operating-system/2022-07-21-instalar-tor-browser/index.pdf}{\faIcon{file-pdf}}
  \href{https://achalmaedison.netlify.app/tecnologia-seguridad/operating-system/2022-07-21-instalar-tor-browser}{Instalar
  Tor Browser}
\item
  \href{https://achalmaedison.netlify.app/tecnologia-seguridad/operating-system/2022-08-14-crear-enlaces-duros-o-hard-link-en-linux/index.pdf}{\faIcon{file-pdf}}
  \href{https://achalmaedison.netlify.app/tecnologia-seguridad/operating-system/2022-08-14-crear-enlaces-duros-o-hard-link-en-linux}{Crear
  Enlaces Duros O Hard Link En Linux}
\item
  \href{https://achalmaedison.netlify.app/tecnologia-seguridad/operating-system/2022-09-27-comandos-vim/index.pdf}{\faIcon{file-pdf}}
  \href{https://achalmaedison.netlify.app/tecnologia-seguridad/operating-system/2022-09-27-comandos-vim}{Comandos
  Vim}
\item
  \href{https://achalmaedison.netlify.app/tecnologia-seguridad/operating-system/2023-02-16-guia-de-git-y-github/index.pdf}{\faIcon{file-pdf}}
  \href{https://achalmaedison.netlify.app/tecnologia-seguridad/operating-system/2023-02-16-guia-de-git-y-github}{Guia
  De Git Y Github}
\item
  \href{https://achalmaedison.netlify.app/tecnologia-seguridad/operating-system/2023-05-02-00-primeros-pasos-en-linux/index.pdf}{\faIcon{file-pdf}}
  \href{https://achalmaedison.netlify.app/tecnologia-seguridad/operating-system/2023-05-02-00-primeros-pasos-en-linux}{00
  Primeros Pasos En Linux}
\item
  \href{https://achalmaedison.netlify.app/tecnologia-seguridad/operating-system/2023-06-17-01-introduccion-linux/index.pdf}{\faIcon{file-pdf}}
  \href{https://achalmaedison.netlify.app/tecnologia-seguridad/operating-system/2023-06-17-01-introduccion-linux}{01
  Introduccion Linux}
\item
  \href{https://achalmaedison.netlify.app/tecnologia-seguridad/operating-system/2023-06-18-02-distribuciones-linux/index.pdf}{\faIcon{file-pdf}}
  \href{https://achalmaedison.netlify.app/tecnologia-seguridad/operating-system/2023-06-18-02-distribuciones-linux}{02
  Distribuciones Linux}
\item
  \href{https://achalmaedison.netlify.app/tecnologia-seguridad/operating-system/2023-06-19-03-instalacion-linux/index.pdf}{\faIcon{file-pdf}}
  \href{https://achalmaedison.netlify.app/tecnologia-seguridad/operating-system/2023-06-19-03-instalacion-linux}{03
  Instalacion Linux}
\item
  \href{https://achalmaedison.netlify.app/tecnologia-seguridad/operating-system/2023-06-20-04-administracion-particiones-volumenes/index.pdf}{\faIcon{file-pdf}}
  \href{https://achalmaedison.netlify.app/tecnologia-seguridad/operating-system/2023-06-20-04-administracion-particiones-volumenes}{04
  Administracion Particiones Volumenes}
\item
  \href{https://achalmaedison.netlify.app/tecnologia-seguridad/operating-system/2023-07-01-atajos-de-teclado-y-comandos-para-usar-vim/index.pdf}{\faIcon{file-pdf}}
  \href{https://achalmaedison.netlify.app/tecnologia-seguridad/operating-system/2023-07-01-atajos-de-teclado-y-comandos-para-usar-vim}{Atajos
  De Teclado Y Comandos Para Usar Vim}
\item
  \href{https://achalmaedison.netlify.app/tecnologia-seguridad/operating-system/2024-07-15-instalando-specitify/index.pdf}{\faIcon{file-pdf}}
  \href{https://achalmaedison.netlify.app/tecnologia-seguridad/operating-system/2024-07-15-instalando-specitify}{Instalando
  Specitify}
\end{enumerate}

Esperamos que encuentres estas publicaciones igualmente interesantes y
útiles. ¡Disfruta de la lectura!






\end{document}
